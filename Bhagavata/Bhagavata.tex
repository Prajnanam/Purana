         || ॐ नमो भगवते वासुदेवाय ||
जन्माद्यस्य यतोऽन्वयादितरतश्चार्थेष्वभिज्ञः स्वराट्‌
तेने ब्रह्म हृदा य आदिकवये मुह्यन्ति यत्सूरयः
तेजोवारिमृदां यथा विनिमयो यत्र त्रिसर्गोऽमृषा
धाम्ना स्वेन सदा निरस्तकुहकं सत्यं परं धीमहि
धर्मः प्रोज्झितकैतवोऽत्र परमो निर्मत्सराणां सतां
वेद्यं वास्तवमत्र वस्तु शिवदं तापत्रयोन्मूलनम्‌
श्रीमद्भागवते महामुनिकृते किं वा परैरीश्वरः
सद्यो हृद्यवरुध्यतेऽत्र कृतिभिः शुश्रूषुभिस्तत्क्षणात्‌
निगमकल्पतरोर्गलितं फलं
शुकमुखादमृतद्रवसंयुतम्‌
पिबत भागवतं रसमालयं
मुहुरहो रसिका भुवि भावुकाः
नैमिषेऽनिमिषक्षेत्रे ईशयः शौनकादयः
सत्रं स्वर्गाय लोकाय सहस्रसममासत
त एकदा तु मुनयः प्रातर्हुतहुताग्नयः
सत्कृतं सूतमासीनं पप्रच्छुरिदमादरात्‌
ऋषय ऊचुः
त्वया खलु पुराणानि सेतिहासानि चानघ
आख्यातान्यप्यधीतानि धर्मशास्त्राणि यान्युत
यानि वेदविदां श्रेष्ठो भगवान्बादरायणः
अन्ये च मुनयः सूत परावरविदो विदुः
वेत्थ त्वं सौम्य तत्सर्वं तत्त्वतस्तदनुग्रहात्‌
ब्रूयुः स्निग्धस्य शिष्यस्य गुरवो गुह्यमप्युत
तत्र तत्राञ्जसायुष्मन्भवता यद्विनिश्चितम्‌
पुंसामेकान्ततः श्रेयस्तन्नः शंसितुमर्हसि
प्रायेणाल्पायुषः सभ्य कलावस्मिन्युगे जनाः
मन्दाः सुमन्दमतयो मन्दभाग्या ह्युपद्रुताः
भूरीणि भूरिकर्माणि श्रोतव्यानि विभागशः
अतः साधोऽत्र यत्सारं समुद्धृत्य मनीषया
ब्रूहि भद्राय भूतानां येनात्मा सुप्रसीदति
सूत जानासि भद्रं ते भगवान्सात्वतां पतिः
देवक्यां वसुदेवस्य जातो यस्य चिकीर्षया
तन्नः शुष्रूषमाणानामर्हस्यङ्गानुवर्णितुम्‌
यस्यावतारो भूतानां क्षेमाय च भवाय च
आपन्नः संसृतिं घोरां यन्नाम विवशो गृणन्‌
ततः सद्यो विमुच्येत यद्बिभेति स्वयं भयम्‌
यत्पादसंश्रयाः सूत मुनयः प्रशमायनाः
सद्यः पुनन्त्युपस्पृष्टाः स्वर्धुन्यापोऽनुसेवया
को वा भगवतस्तस्य पुण्यश्लोकेड्यकर्मणः
शुद्धिकामो न शृणुयाद्यशः कलिमलापहम्‌
तस्य कर्माण्युदाराणि परिगीतानि सूरिभिः
ब्रूहि नः श्रद्दधानानां लीलया दधतः कलाः
अथाख्याहि हरेर्धीमन्नवतारकथाः शुभाः
ईला विदधतः स्वैरमीश्वरस्यात्ममायया
वयं तु न वितृप्याम उत्तमश्लोकविक्रमे
यच्छृण्वतां रसज्ञानां स्वादु स्वादु पदे पदे
कृतवान्किल कर्माणि सह रामेण केशवः
अतिमर्त्यानि भगवान्गूढः कपटमानुषः
कलिमागतमाज्ञाय क्षेत्रेऽस्मिन्वैष्णवे वयम्‌
आसीना दीर्घसत्रेण कथायां सक्षणा हरेः
त्वं नः सन्दर्शितो धात्रा दुस्तरं निस्तितीर्षताम्‌
कलिं सत्त्वहरं पुंसां कर्णधार इवार्णवम्‌
ब्रूहि योगेश्वरे कृष्णे ब्रह्मण्ये धर्मवर्मणि
स्वां काष्ठामधुनोपेते धर्मः कं शरणं गतः
व्यास उवाच
इति सम्प्रश्नसंहृष्टो विप्राणां रौमहर्शणिः
प्रतिपूज्य वचस्तेशां प्रवक्तुमुपचक्रमे
सूत उवाच
यं प्रव्रजन्तमनुपेतमपेतकृत्यं द्वैपायनो विरहकातर आजुहाव
पुत्रेति तन्मयतया तरवोऽभिनेदुस्तं सर्वभूतहृदयं मुनिमानतोऽस्मि
यः स्वानुभावमखिलश्रुतिसारमेकमध्यात्मदीपमतितितीर्षतां तमोऽन्धम्‌
संसारिणां करुणयाह पुराणगुह्यं तं व्याससूनुमुपयामि गुरुं मुनीनाम्‌
नारायणं नमस्कृत्य नरं चैव नरोत्तमम्‌
देवीं सरस्वतीं व्यासं ततो जयमुदीरयेत्‌
मुनयः साधु पृष्टोऽहं भवद्भिर्लोकमङ्गलम्‌
यत्कृतः कृष्णसम्प्रश्नो येनात्मा सुप्रसीदति
स वै पुंसां परो धर्मो यतो भक्तिरधोक्षजे
अहैतुक्यप्रतिहता ययात्मा सुप्रसीदति
वासुदेवे भगवति भक्तियोगः प्रयोजितः
जनयत्याशु वैराग्यं ज्ञानं च यदहैतुकम्‌
धर्मः स्वनुष्ठितः पुंसां विष्वक्सेनकथासु यः
नोत्पादयेद्यदि रतिं श्रम एव हि केवलम्‌
धर्मस्य ह्यापवर्ग्यस्य नार्थोऽर्थायोपकल्पते
नार्थस्य धर्मैकान्तस्य कामो लाभाय हि स्मृतः
कामस्य नेन्द्रियप्रीतिर्लाभो जीवेत यावता
जीवस्य तत्त्वजिज्ञासा नार्थो यश्चेह कर्मभिः
वदन्ति तत्तत्त्वविदस्तत्त्वं यज्ज्ञानमद्वयम्‌
ब्रह्मेति परमात्मेति भगवानिति शब्द्यते
तच्छ्रद्दधाना मुनयो ज्ञानवैराग्ययुक्तया
पश्यन्त्यात्मनि चात्मानं भक्त्या श्रुतगृहीतया
अतः पुम्भिर्द्विजश्रेष्ठा वर्णाश्रमविभागशः
स्वनुष्ठितस्य धर्मस्य संसिद्धिर्हरितोषणम्‌
तस्मादेकेन मनसा भगवान्सात्वतां पतिः
श्रोतव्यः कीर्तितव्यश्च ध्येयः पूज्यश्च नित्यदा
यदनुध्यासिना युक्ताः कर्मग्रन्थिनिबन्धनम्‌
छिन्दन्ति कोविदास्तस्य को न कुर्यात्कथारतिम्‌
शुश्रूषोः श्रद्दधानस्य वासुदेवकथारुचिः
स्यान्महत्सेवया विप्राः पुण्यतीर्थनिषेवणात्‌
शृण्वतां स्वकथाः कृष्णः पुण्यश्रवणकीर्तनः
हृद्यन्तःस्थो ह्यभद्राणि विधुनोति सुहृत्सताम्‌
नष्टप्रायेष्वभद्रेषु नित्यं भागवतसेवया
भगवत्युत्तमश्लोके भक्तिर्भवति नैष्ठिकी
तदा रजस्तमोभावाः कामलोभादयश्च ये
चेत एतैरनाविद्धं स्थितं सत्त्वे प्रसीदति
एवं प्रसन्नमनसो भगवद्भक्तियोगतः
भगवत्तत्त्वविज्ञानं मुक्तसङ्गस्य जायते
भिद्यते हृदयग्रन्थिश्छिद्यन्ते सर्वसंशयाः
क्षीयन्ते चास्य कर्माणि दृष्ट एवात्मनीश्वरे
अतो वै कवयो नित्यं भक्तिं परमया मुदा
वासुदेवे भगवति कुर्वन्त्यात्मप्रसादनीम्‌
सत्त्वं रजस्तम इति प्रकृतेर्गुणास्तैर्युक्तः परमपुरुष एक इहास्य धत्ते
स्थित्यादये हरिविरिञ्चिहरेति संज्ञाः श्रेयांसि तत्र खलु सत्त्वतनोर्नृणां स्युः
पार्थिवाद्दारुणो धूमस्तस्मादग्निस्त्रयीमयः
तमसस्तु रजस्तस्मात्सत्त्वं यद्ब्रह्मदर्शनम्‌
भेजिरे मुनयोऽथाग्रे भगवन्तमधोक्षजम्‌
सत्त्वं विशुद्धं क्षेमाय कल्पन्ते येऽनु तानिह
मुमुक्षवो घोररूपान्हित्वा भूतपतीनथ
नारायणकलाः शान्ता भजन्ति ह्यनसूयवः
रजस्तमःप्रकृतयः समशीला भजन्ति वै
पितृभूतप्रजेशादीन्श्रियैश्वर्यप्रजेप्सवः
वासुदेवपरा वेदा वासुदेवपरा मखाः
वासुदेवपरा योग वासुदेवपराः क्रियाः
वासुदेवपरं ज्ञानं वासुदेवपरं तपः
वासुदेवपरो धर्मो वासुदेवपरा गतिः
स एवेदं ससर्जाग्रे भगवानात्ममायया
सदसद्रूपया चासौ गुणमयागुणो विभुः
तया विलसितेष्वेषु गुणेषु गुणवानिव
अन्तःप्रविष्ट आभाति विज्ञानेन विजृम्भितः
यथा ह्यवहितो वह्निर्दारुष्वेकः स्वयोनिषु
नानेव भाति विश्वात्मा भूतेषु च तथा पुमान्‌
असौ गुणमयैर्भावैर्भूतसूक्ष्मेन्द्रियात्मभिः
स्वनिर्मितेषु निर्विष्टो भुङ्क्ते भूतेषु तद्गुणान्‌
भावयत्येष सत्त्वेन लोकान्वै लोकभावनः
लीलावतारानुरतो देवतिर्यङ्नरादिषु
सूत उवाच
जगृहे पौरुषं रूपं भगवान्महदादिभिः
सम्भूतं षोडशकलमादौ लोकसिसृक्षया
यस्याम्भसि शयानस्य योगनिद्रां वितन्वतः
नाभिह्रदाम्बुजादासीद्ब्रह्मा विश्वसृजां पतिः
यस्यावयवसंस्थानैः कल्पितो लोकविस्तरः
तद्वै भगवतो रूपं विशुद्धं सत्त्वमूर्जितम्‌
पश्यन्त्यदो रूपमदभ्रचक्षुषा सहस्रपादोरुभुजाननाद्भुतम्‌
सहस्रमूर्धश्रवणाक्षिनासिकं सहस्रमौल्यम्बरकुण्डलोल्लसत्‌
एतन्नानावताराणां निधानं बीजमव्ययम्‌
यस्यांशांशेन सृज्यन्ते देवतिर्यङ्नरादयः
स एव प्रथमं देवः कौमारं सर्गमाश्रितः
चचार दुश्चरं ब्रह्मा ब्रह्मचर्यमखण्डितम्‌
द्वितीयं तु भवायास्य रसातलगतां महीम्‌
उद्धरिष्यन्नुपादत्त यज्ञेशः सौकरं वपुः
तृतीयमृषिसर्गं वै देवर्षित्वमुपेत्य सः
तन्त्रं सात्वतमाचष्ट नैष्कर्म्यं कर्मणां यतः
तुर्ये धर्मकलासर्गे नरनारायणावृषी
भूत्वात्मोपशमोपेतमकरोद्दुश्चरं तपः
पञ्चमः कपिलो नाम सिद्धेशः कालविप्लुतम्‌
प्रोवाचासुरये साङ्ख्यं तत्त्वग्रामविनिर्णयम्‌
षष्ठमत्रेरपत्यत्वं वृतः प्राप्तोऽनसूयया
आन्वीक्षिकीमलर्काय प्रह्लादादिभ्य ऊचिवान्‌
ततः सप्तम आकूत्यां रुचेर्यज्ञोऽभ्यजायत
स यामाद्यैः सुरगणैरपात्स्वायम्भुवान्तरम्‌
अष्टमे मेरुदेव्यां तु नाभेर्जात उरुक्रमः
दर्शयन्वर्त्म धीराणां सर्वाश्रमनमस्कृतम्‌
ऋषिभिर्याचितो भेजे नवमं पार्थिवं वपुः
दुग्धेमामोषधीर्विप्रास्तेनायं स उशत्तमः
रूपं स जगृहे मात्स्यं चाक्षुषोदधिसम्प्लवे
नाव्यारोप्य महीमय्यामपाद्वैवस्वतं मनुम्‌
सुरासुराणामुदधिं मथ्नतां मन्दराचलम्‌
दध्रे कमठरूपेण पृष्ठ एकादशे विभुः
धान्वन्तरं द्वादशमं त्रयोदशममेव च
अपाययत्सुरानन्यान्मोहिन्या मोहयन्स्त्रिया
चतुर्दशं नारसिंहं बिभ्रद्दैत्येन्द्रमूर्जितम्‌
ददार करजैरूरावेरकां कटकृद्यथा
पञ्चदशं वामनकं कृत्वागादध्वरं बलेः
पदत्रयं याचमानः प्रत्यादित्सुस्त्रिपिष्टपम्‌
अवतारे षोडशमे पश्यन्ब्रह्मद्रुहो नृपान्‌
त्रिःसप्तकृत्वः कुपितो निःक्षत्रामकरोन्महीम्‌
ततः सप्तदशे जातः सत्यवत्यां पराशरात्‌
चक्रे वेदतरोः शाखा दृष्ट्वा पुंसोऽल्पमेधसः
नरदेवत्वमापन्नः सुरकार्यचिकीर्षया
समुद्रनिग्रहादीनि चक्रे वीर्याण्यतः परम्‌
एकोनविंशे विंशतिमे वृष्णिषु प्राप्य जन्मनी
रामकृष्णाविति भुवो भगवानहरद्भरम्‌
ततः कलौ सम्प्रवृत्ते सम्मोहाय सुरद्विषाम्‌
बुद्धो नाम्नाञ्जनसुतः कीकटेषु भविष्यति
अथासौ युगसन्ध्यायां दस्युप्रायेषु राजसु
जनिता विष्णुयशसो नाम्ना कल्किर्जगत्पतिः
अवतारा ह्यसङ्ख्येया हरेः सत्त्वनिधेर्द्विजाः
यथाविदासिनः कुल्याः सरसः स्युः सहस्रशः
ऋषयो मनवो देवा मनुपुत्रा महौजसः
कलाः सर्वे हरेरेव सप्रजापतयः स्मृताः
एते चांशकलाः पुंसः कृष्णस्तु भगवान्स्वयम्‌
इन्द्रारिव्याकुलं लोकं मृडयन्ति युगे युगे
जन्म गुह्यं भगवतो य एतत्प्रयतो नरः
सायं प्रातर्गृणन्भक्त्या दुःखग्रामाद्विमुच्यते
एतद्रूपं भगवतो ह्यरूपस्य चिदात्मनः
मायागुणैर्विरचितं महदादिभिरात्मनि
यथा नभसि मेघौघो रेणुर्वा पार्थिवोऽनिले
एवं द्रष्टरि दृश्यत्वमारोपितमबुद्धिभिः
अतः परं यदव्यक्तमव्यूढगुणबृंहितम्‌
अदृष्टाश्रुतवस्तुत्वात्स जीवो यत्पुनर्भवः
यत्रेमे सदसद्रूपे प्रतिषिद्धे स्वसंविदा
अविद्ययात्मनि कृते इति तद्ब्रह्मदर्शनम्‌
यद्येषोपरता देवी माया वैशारदी मतिः
सम्पन्न एवेति विदुर्महिम्नि स्वे महीयते
एवं च जन्मानि कर्माणि ह्यकर्तुरजनस्य च
वर्णयन्ति स्म कवयो वेदगुह्यानि हृत्पतेः
स वा इदं विश्वममोघलीलः सृजत्यवत्यत्ति न सज्जतेऽस्मिन्‌
भूतेषु चान्तर्हित आत्मतन्त्रः षाड्वर्गिकं जिघ्रति षड्गुणेशः
न चास्य कश्चिन्निपुणेन धातुरवैति जन्तुः कुमनीष ऊतीः
नामानि रूपाणि मनोवचोभिः सन्तन्वतो नटचर्यामिवाज्ञः
स वेद धातुः पदवीं परस्य दुरन्तवीर्यस्य रथाङ्गपाणेः
योऽमायया सन्ततयानुवृत्त्या भजेत तत्पादसरोजगन्धम्‌
अथेह धन्या भगवन्त इत्थं यद्वासुदेवेऽखिललोकनाथे
कुर्वन्ति सर्वात्मकमात्मभावं न यत्र भूयः परिवर्त उग्रः
इदं भागवतं नाम पुराणं ब्रह्मसम्मितम्‌
उत्तमश्लोकचरितं चकार भगवानृषिः
निःश्रेयसाय लोकस्य धन्यं स्वस्त्ययनं महत्‌
तदिदं ग्राहयामाससुतमात्मवतां वरम्‌
सर्ववेदेतिहासानां सारं सारं समुद्धृतम्‌
स तु संश्रावयामासमहाराजं परीक्षितम्‌
प्रायोपविष्टं गङ्गायां परीतं परमर्षिभिः
कृष्णे स्वधामोपगते धर्मज्ञानादिभिः सह
कलौ नष्टदृशामेष पुराणार्कोऽधुनोदितः
तत्र कीर्तयतो विप्रा विप्रर्षेर्भूरितेजसः
अहं चाध्यगमं तत्र निविष्टस्तदनुग्रहात्‌
सोऽहं वः श्रावयिष्यामि यथाधीतं यथामति
व्यास उवाच
इति ब्रुवाणं संस्तूय मुनीनां दीर्घसत्रिणाम्‌
वृद्धः कुलपतिः सूतं बह्वृचः शौनकोऽब्रवीत्‌
शौनक उवाच
सूत सूत महाभाग वद नो वदतां वर
कथां भागवतीं पुण्यां यदाह भगवाञ्छुकः
कस्मिन्युगे प्रवृत्तेयं स्थाने वा केन हेतुना
कुतः सञ्चोदितः कृष्णः कृतवान्संहितां मुनिः
तस्य पुत्रो महायोगी समदृङ्निर्विकल्पकः
एकान्तमतिरुन्निद्रो गूढो मूढ इवेयते
दृष्ट्वानुयान्तमृषिमात्मजमप्यनग्नं देव्यो ह्रिया परिदधुर्न सुतस्य चित्रम्‌
तद्वीक्ष्य पृच्छति मुनौ जगदुस्तवास्ति स्त्रीपुम्भिदा न तु सुतस्य विविक्तदृष्टेः
कथमालक्षितः पौरैः सम्प्राप्तः कुरुजाङ्गलान्‌
उन्मत्तमूकजडवद्विचरन्गजसाह्वये
कथं वा पाण्डवेयस्य राजर्षेर्मुनिना सह
संवादः समभूत्तात यत्रैषा सात्वती श्रुतिः
स गोदोहनमात्रं हि गृहेषु गृहमेधिनाम्‌
अवेक्षते महाभागस्तीर्थीकुर्वंस्तदाश्रमम्‌
अभिमन्युसुतं सूत प्राहुर्भागवतोत्तमम्‌
तस्य जन्म महाश्चर्यं कर्माणि च गृणीहि नः
स सम्राट्कस्य वा हेतोः पाण्डूनां मानवर्धनः
प्रायोपविष्टो गङ्गायामनादृत्याधिराट्श्रियम्‌
नमन्ति यत्पादनिकेतमात्मनः शिवाय हानीय धनानि शत्रवः
कथं स वीरः श्रियमङ्ग दुस्त्यजां युवैषतोत्स्रष्टुमहो सहासुभिः
शिवाय लोकस्य भवाय भूतये य उत्तमश्लोकपरायणा जनाः
जीवन्ति नात्मार्थमसौ पराश्रयं मुमोच निर्विद्य कुतः कलेवरम्‌
तत्सर्वं नः समाचक्ष्व पृष्टो यदिह किञ्चन
मन्ये त्वां विषये वाचां स्नातमन्यत्र छान्दसात्‌
सूत उवाच
द्वापरे समनुप्राप्ते तृतीये युगपर्यये
जातः पराशराद्योगी वासव्यां कलया हरेः
स कदाचित्सरस्वत्या उपस्पृश्य जलं शुचिः
विविक्त एक आसीन उदिते रविमण्डले
परावरज्ञः स ऋषिः कालेनाव्यक्तरंहसा
युगधर्मव्यतिकरं प्राप्तं भुवि युगे युगे
भौतिकानां च भावानां शक्तिह्रासं च तत्कृतम्‌
अश्रद्दधानान्निःसत्त्वान्दुर्मेधान्ह्रसितायुषः
दुर्भगांश्च जनान्वीक्ष्य मुनिर्दिव्येन चक्षुषा
सर्ववर्णाश्रमाणां यद्दध्यौ हितममोघदृक्‌
चातुर्होत्रं कर्म शुद्धं प्रजानां वीक्ष्य वैदिकम्‌
व्यदधाद्यज्ञसन्तत्यै वेदमेकं चतुर्विधम्‌
ऋग्यजुःसामाथर्वाख्या वेदाश्चत्वार उद्धृताः
इतिहासपुराणं च पञ्चमो वेद उच्यते
तत्रर्ग्वेदधरः पैलः सामगो जैमिनिः कविः
वैशम्पायन एवैको निष्णातो यजुषामुत
अथर्वाङ्गिरसामासीत्सुमन्तुर्दारुणो मुनिः
इतिहासपुराणानां पिता मे रोमहर्षणः
त एत ऋषयो वेदं स्वं स्वं व्यस्यन्ननेकधा
शिष्यैः प्रशिष्यैस्तच्छिष्यैर्वेदास्ते शाखिनोऽभवन्‌
त एव वेदा दुर्मेधैर्धार्यन्ते पुरुषैर्यथा
एवं चकार भगवान्व्यासः कृपणवत्सलः
स्त्रीशूद्रद्विजबन्धूनां त्रयी न श्रुतिगोचरा
कर्मश्रेयसि मूढानां श्रेय एवं भवेदिह
इति भारतमाख्यानं कृपया मुनिना कृतम्‌
एवं प्रवृत्तस्य सदा भूतानां श्रेयसि द्विजाः
सर्वात्मकेनापि यदा नातुष्यद्धृदयं ततः
नातिप्रसीदद्धृदयः सरस्वत्यास्तटे शुचौ
वितर्कयन्विविक्तस्थ इदं चोवाच धर्मवित्‌
धृतव्रतेन हि मया छन्दांसि गुरवोऽग्नयः
मानिता निर्व्यलीकेन गृहीतं चानुशासनम्‌
भारतव्यपदेशेन ह्याम्नायार्थश्च प्रदर्शितः
दृश्यते यत्र धर्मादि स्त्रीशूद्रादिभिरप्युत
तथापि बत मे दैह्यो ह्यात्मा चैवात्मना विभुः
असम्पन्न इवाभाति ब्रह्मवर्चस्य सत्तमः
किं वा भागवता धर्मा न प्रायेण निरूपिताः
प्रियाः परमहंसानां त एव ह्यच्युतप्रियाः
तस्यैवं खिलमात्मानं मन्यमानस्य खिद्यतः
कृष्णस्य नारदोऽभ्यागादाश्रमं प्रागुदाहृतम्‌
तमभिज्ञाय सहसा प्रत्युत्थायागतं मुनिः
पूजयामास विधिवन्नारदं सुरपूजितम्‌
सूत उवाच
अथ तं सुखमासीन उपासीनं बृहच्छ्रवाः
देवर्षिः प्राह विप्रर्षिं वीणापाणिः स्मयन्निव
नारद उवाच
पाराशर्य महाभाग भवतः कच्चिदात्मना
परितुष्यति शारीर आत्मा मानस एव वा
जिज्ञासितं सुसम्पन्नमपि ते महदद्भुतम्‌
कृतवान्भारतं यस्त्वं सर्वार्थपरिबृंहितम्‌
जिज्ञासितमधीतं च ब्रह्म यत्तत्सनातनम्‌
तथापि शोचस्यात्मानमकृतार्थ इव प्रभो
व्यास उवाच
अस्त्येव मे सर्वमिदं त्वयोक्तं तथापि नात्मा परितुष्यते मे
तन्मूलमव्यक्तमगाधबोधं पृच्छामहे त्वात्मभवात्मभूतम्‌
स वै भवान्वेद समस्तगुह्यमुपासितो यत्पुरुषः पुराणः
परावरेशो मनसैव विश्वं सृजत्यवत्यत्ति गुणैरसङ्गः
त्वं पर्यटन्नर्क इव त्रिलोकीमन्तश्चरो वायुरिवात्मसाक्षी
परावरे ब्रह्मणि धर्मतो व्रतैः स्नातस्य मे न्यूनमलं विचक्ष्व
श्रीनारद उवाच
भवतानुदितप्रायं यशो भगवतोऽमलम्‌
येनैवासौ न तुष्येत मन्ये तद्दर्शनं खिलम्‌
यथा धर्मादयश्चार्था मुनिवर्यानुकीर्तिताः
न तथा वासुदेवस्य महिमा ह्यनुवर्णितः
न यद्वचश्चित्रपदं हरेर्यशो जगत्पवित्रं प्रगृणीत कर्हिचित्‌
तद्वायसं तीर्थमुशन्ति मानसा न यत्र हंसा निरमन्त्युशिक्क्षयाः
तद्वाग्विसर्गो जनताघविप्लवो यस्मिन्प्रतिश्लोकमबद्धवत्यपि
नामान्यनन्तस्य यशोऽङ्कितानि यत्शृण्वन्ति गायन्ति गृणन्ति साधवः
नैष्कर्म्यमप्यच्युतभाववर्जितं न शोभते ज्ञानमलं निरञ्जनम्‌
कुतः पुनः शश्वदभद्रमीश्वरे न चार्पितं कर्म यदप्यकारणम्‌
अथो महाभाग भवानमोघदृक्षुचिश्रवाः सत्यरतो धृतव्रतः
उरुक्रमस्याखिलबन्धमुक्तये समाधिनानुस्मर तद्विचेष्टितम्‌
ततोऽन्यथा किञ्चन यद्विवक्षतः पृथग्दृशस्तत्कृतरूपनामभिः
न कर्हिचित्क्वापि च दुःस्थिता मतिर्लभेत वाताहतनौरिवास्पदम्‌
जुगुप्सितं धर्मकृतेऽनुशासतः स्वभावरक्तस्य महान्व्यतिक्रमः
यद्वाक्यतो धर्म इतीतरः स्थितो न मन्यते तस्य निवारणं जनः
विचक्षणोऽस्यार्हति वेदितुं विभोरनन्तपारस्य निवृत्तितः सुखम्‌
प्रवर्तमानस्य गुणैरनात्मनस्ततो भवान्दर्शय चेष्टितं विभोः
त्यक्त्वा स्वधर्मं चरणाम्बुजं हरेर्भजन्नपक्वोऽथ पतेत्ततो यदि
यत्र क्व वाभद्रमभूदमुष्य किं को वार्थ आप्तोऽभजतां स्वधर्मतः
तस्यैव हेतोः प्रयतेत कोविदो न लभ्यते यद्भ्रमतामुपर्यधः
तल्लभ्यते दुःखवदन्यतः सुखं कालेन सर्वत्र गभीररंहसा
न वै जनो जातु कथञ्चनाव्रजेन्मुकुन्दसेव्यन्यवदङ्ग संसृतिम्‌
स्मरन्मुकुन्दाङ्घ्र्युपगूहनं पुनर्विहातुमिच्छेन्न रसग्रहो जनः
इदं हि विश्वं भगवानिवेतरो यतो जगत्स्थाननिरोधसम्भवाः
तद्धि स्वयं वेद भवांस्तथापि ते प्रादेशमात्रं भवतः प्रदर्शितम्‌
त्वमात्मनात्मानमवेह्यमोघदृक्परस्य पुंसः परमात्मनः कलाम्‌
अजं प्रजातं जगतः शिवाय तन्महानुभावाभ्युदयोऽधिगण्यताम्‌
इदं हि पुंसस्तपसः श्रुतस्य वा स्विष्टस्य सूक्तस्य च बुद्धिदत्तयोः
अविच्युतोऽर्थः कविभिर्निरूपितो यदुत्तमश्लोकगुणानुवर्णनम्‌
अहं पुरातीतभवेऽभवं मुने दास्यास्तु कस्याश्चन वेदवादिनाम्‌
निरूपितो बालक एव योगिनां शुश्रूषणे प्रावृषि निर्विविक्षताम्‌
ते मय्यपेताखिलचापलेऽर्भके दान्तेऽधृतक्रीडनकेऽनुवर्तिनि
चक्रुः कृपां यद्यपि तुल्यदर्शनाः शुश्रूषमाणे मुनयोऽल्पभाषिणि
उच्छिष्टलेपाननुमोदितो द्विजैः सकृत्स्म भुञ्जे तदपास्तकिल्बिषः
एवं प्रवृत्तस्य विशुद्धचेतसस्तद्धर्म एवात्मरुचिः प्रजायते
तत्रान्वहं कृष्णकथाः प्रगायतामनुग्रहेणाशृणवं मनोहराः
ताः श्रद्धया मेऽनुपदं विशृण्वतः प्रियश्रवस्यङ्ग ममाभवद्रुचिः
तस्मिंस्तदा लब्धरुचेर्महामते प्रियश्रवस्यस्खलिता मतिर्मम
ययाहमेतत्सदसत्स्वमायया पश्ये मयि ब्रह्मणि कल्पितं परे
इत्थं शरत्प्रावृषिकावृतू हरेर्विशृण्वतो मेऽनुसवं यशोऽमलम्‌
सङ्कीर्त्यमानं मुनिभिर्महात्मभिर्भक्तिः प्रवृत्तात्मरजस्तमोपहा
तस्यैवं मेऽनुरक्तस्य प्रश्रितस्य हतैनसः
श्रद्दधानस्य बालस्य दान्तस्यानुचरस्य च
ज्ञानं गुह्यतमं यत्तत्साक्षाद्भगवतोदितम्‌
अन्ववोचन्गमिष्यन्तः कृपया दीनवत्सलाः
येनैवाहं भगवतो वासुदेवस्य वेधसः
मायानुभावमविदं येन गच्छन्ति तत्पदम्‌
एतत्संसूचितं ब्रह्मंस्तापत्रयचिकित्सितम्‌
यदीश्वरे भगवति कर्म ब्रह्मणि भावितम्‌
आमयो यश्च भूतानां जायते येन सुव्रत
तदेव ह्यामयं द्रव्यं न पुनाति चिकित्सितम्‌
एवं नृणां क्रियायोगाः सर्वे संसृतिहेतवः
त एवात्मविनाशाय कल्पन्ते कल्पिताः परे
यदत्र क्रियते कर्म भगवत्परितोषणम्‌
ज्ञानं यत्तदधीनं हि भक्तियोगसमन्वितम्‌
कुर्वाणा यत्र कर्माणि भगवच्छिक्षयासकृत्‌
गृणन्ति गुणनामानि कृष्णस्यानुस्मरन्ति च
ओं नमो भगवते तुभ्यं वासुदेवाय धीमहि
प्रद्युम्नायानिरुद्धाय नमः सङ्कर्षणाय च
इति मूर्त्यभिधानेन मन्त्रमूर्तिममूर्तिकम्‌
यजते यज्ञपुरुषं स सम्यग्दर्शनः पुमान्‌
इमं स्वनिगमं ब्रह्मन्नवेत्य मदनुष्ठितम्‌
अदान्मे ज्ञानमैश्वर्यं स्वस्मिन्भावं च केशवः
त्वमप्यदभ्रश्रुत विश्रुतं विभोः समाप्यते येन विदां बुभुत्सितम्‌
प्राख्याहि दुःखैर्मुहुरर्दितात्मनां सङ्क्लेशनिर्वाणमुशन्ति नान्यथा
सूत उवाच
एवं निशम्य भगवान्देवर्षेर्जन्म कर्म च
भूयः पप्रच्छ तं ब्रह्मन्व्यासः सत्यवतीसुतः
व्यास उवाच
भिक्षुभिर्विप्रवसिते विज्ञानादेष्टृभिस्तव
वर्तमानो वयस्याद्ये ततः किमकरोद्भवान्‌
स्वायम्भुव कया वृत्त्या वर्तितं ते परं वयः
कथं चेदमुदस्राक्षीः काले प्राप्ते कलेवरम्‌
प्राक्कल्पविषयामेतां स्मृतिं ते मुनिसत्तम
न ह्येष व्यवधात्काल एष सर्वनिराकृतिः
नारद उवाच
भिक्षुभिर्विप्रवसिते विज्ञानादेष्टृभिर्मम
वर्तमानो वयस्याद्ये तत एतदकारषम्‌
एकात्मजा मे जननी योषिन्मूढा च किङ्करी
मय्यात्मजेऽनन्यगतौ चक्रे स्नेहानुबन्धनम्‌
सास्वतन्त्रा न कल्पासीद्योगक्षेमं ममेच्छती
ईशस्य हि वशे लोको योषा दारुमयी यथा
अहं च तद्ब्रह्मकुले ऊषिवांस्तदुपेक्षया
दिग्देशकालाव्युत्पन्नो बालकः पञ्चहायनः
एकदा निर्गतां गेहाद्दुहन्तीं निशि गां पथि
सर्पोऽदशत्पदा स्पृष्टः कृपणां कालचोदितः
तदा तदहमीशस्य भक्तानां शमभीप्सतः
अनुग्रहं मन्यमानः प्रातिष्ठं दिशमुत्तराम्‌
स्फीताञ्जनपदांस्तत्र पुरग्रामव्रजाकरान्‌
खेटखर्वटवाटीश्च वनान्युपवनानि च
चित्रधातुविचित्राद्रीनिभभग्नभुजद्रुमान्‌
जलाशयाञ्छिवजलान्नलिनीः सुरसेविताः
चित्रस्वनैः पत्ररथैर्विभ्रमद्भ्रमरश्रियः
नलवेणुशरस्तन्ब कुशकीचकगह्वरम्‌
एक एवातियातोऽहमद्राक्षं विपिनं महत्‌
घोरं प्रतिभयाकारं व्यालोलूकशिवाजिरम्‌
परिश्रान्तेन्द्रियात्माहं तृट्परीतो बुभुक्षितः
स्नात्वा पीत्वा ह्रदे नद्या उपस्पृष्टो गतश्रमः
तस्मिन्निर्मनुजेऽरण्ये पिप्पलोपस्थ आश्रितः
आत्मनात्मानमात्मस्थं यथाश्रुतमचिन्तयम्‌
ध्यायतश्चरणाम्भोजं भावनिर्जितचेतसा
औत्कण्ठ्याश्रुकलाक्षस्य हृद्यासीन्मे शनैर्हरिः
प्रेमातिभरनिर्भिन्न पुलकाङ्गोऽतिनिर्वृतः
आनन्दसम्प्लवे लीनो नापश्यमुभयं मुने
रूपं भगवतो यत्तन्मनःकान्तं शुचापहम्‌
अपश्यन्सहसोत्तस्थे वैक्लव्याद्दुर्मना इव
दिदृक्षुस्तदहं भूयः प्रणिधाय मनो हृदि
वीक्षमाणोऽपि नापश्यमवितृप्त इवातुरः
एवं यतन्तं विजने मामाहागोचरो गिराम्‌
गम्भीरश्लक्ष्णया वाचा शुचः प्रशमयन्निव
हन्तास्मिञ्जन्मनि भवान्मा मां द्रष्टुमिहार्हति
अविपक्वकषायाणां दुर्दर्शोऽहं कुयोगिनाम्‌
सकृद्यद्दर्शितं रूपमेतत्कामाय तेऽनघ
मत्कामः शनकैः साधु सर्वान्मुञ्चति हृच्छयान्‌
सत्सेवयादीर्घयापि जाता मयि दृढा मतिः
हित्वावद्यमिमं लोकं गन्ता मज्जनतामसि
मतिर्मयि निबद्धेयं न विपद्येत कर्हिचित्‌
प्रजासर्गनिरोधेऽपि स्मृतिश्च मदनुग्रहात्‌
एतावदुक्त्वोपरराम तन्महद्भूतं नभोलिङ्गमलिङ्गमीश्वरम्‌
अहं च तस्मै महतां महीयसे शीर्ष्णावनामं विदधेऽनुकम्पितः
नामान्यनन्तस्य हतत्रपः पठन्गुह्यानि भद्राणि कृतानि च स्मरन्‌
गां पर्यटंस्तुष्टमना गतस्पृहः कालं प्रतीक्षन्विमदो विमत्सरः
एवं कृष्णमतेर्ब्रह्मन्नासक्तस्यामलात्मनः
कालः प्रादुरभूत्काले तडित्सौदामनी यथा
प्रयुज्यमाने मयि तां शुद्धां भागवतीं तनुम्‌
आरब्धकर्मनिर्वाणो न्यपतत्पाञ्चभौतिकः
कल्पान्त इदमादाय शयानेऽम्भस्युदन्वतः
शिशयिषोरनुप्राणं विविशेऽन्तरहं विभोः
सहस्रयुगपर्यन्ते उत्थायेदं सिसृक्षतः
मरीचिमिश्रा ऋषयः प्राणेभ्योऽहं च जज्ञिरे
अन्तर्बहिश्च लोकांस्त्रीन्पर्येम्यस्कन्दितव्रतः
अनुग्रहान्महाविष्णोरविघातगतिः क्वचित्‌
देवदत्तामिमां वीणां स्वरब्रह्मविभूषिताम्‌
मूर्च्छयित्वा हरिकथां गायमानश्चराम्यहम्‌
प्रगायतः स्ववीर्याणि तीर्थपादः प्रियश्रवाः
आहूत इव मे शीघ्रं दर्शनं याति चेतसि
एतद्ध्यातुरचित्तानां मात्रास्पर्शेच्छया मुहुः
भवसिन्धुप्लवो दृष्टो हरिचर्यानुवर्णनम्‌
यमादिभिर्योगपथैः कामलोभहतो मुहुः
मुकुन्दसेवया यद्वत्तथात्माद्धा न शाम्यति
सर्वं तदिदमाख्यातं यत्पृष्टोऽहं त्वयानघ
जन्मकर्मरहस्यं मे भवतश्चात्मतोषणम्‌
सूत उवाच
एवं सम्भाष्य भगवान्नारदो वासवीसुतम्‌
आमन्त्र्य वीणां रणयन्ययौ यादृच्छिको मुनिः
अहो देवर्षिर्धन्योऽयं यत्कीर्तिं शार्ङ्गधन्वनः
गायन्माद्यन्निदं तन्त्र्या रमयत्यातुरं जगत्‌
शौनक उवाच
निर्गते नारदे सूत भगवान्बादरायणः
श्रुतवांस्तदभिप्रेतं ततः किमकरोद्विभुः
सूत उवाच
ब्रह्मनद्यां सरस्वत्यामाश्रमः पश्चिमे तटे
शम्याप्रास इति प्रोक्त ऋषीणां सत्रवर्धनः
तस्मिन्स्व आश्रमे व्यासो बदरीषण्डमण्डिते
आसीनोऽप उपस्पृश्य प्रणिदध्यौ मनः स्वयम्‌
भक्तियोगेन मनसि सम्यक्प्रणिहितेऽमले
अपश्यत्पुरुषं पूर्णं मायां च तदपाश्रयम्‌
यया सम्मोहितो जीव आत्मानं त्रिगुणात्मकम्‌
परोऽपि मनुतेऽनर्थं तत्कृतं चाभिपद्यते
अनर्थोपशमं साक्षाद्भक्तियोगमधोक्षजे
लोकस्याजानतो विद्वांश्चक्रे सात्वतसंहिताम्‌
यस्यां वै श्रूयमाणायां कृष्णे परमपूरुषे
भक्तिरुत्पद्यते पुंसः शोकमोहभयापहा
स संहितां भागवतीं कृत्वानुक्रम्य चात्मजम्‌
शुकमध्यापयामास निवृत्तिनिरतं मुनिः
शौनक उवाच
स वै निवृत्तिनिरतः सर्वत्रोपेक्षको मुनिः
कस्य वा बृहतीमेतामात्मारामः समभ्यसत्‌
सूत उवाच
आत्मारामाश्च मुनयो निर्ग्रन्था अप्युरुक्रमे
कुर्वन्त्यहैतुकीं भक्तिमित्थम्भूतगुणो हरिः
हरेर्गुणाक्षिप्तमतिर्भगवान्बादरायणिः
अध्यगान्महदाख्यानं नित्यं विष्णुजनप्रियः
परीक्षितोऽथ राजर्षेर्जन्मकर्मविलापनम्‌
संस्थां च पाण्डुपुत्राणां वक्ष्ये कृष्णकथोदयम्‌
यदा मृधे कौरवसृञ्जयानां वीरेष्वथो वीरगतिं गतेषु
वृकोदराविद्धगदाभिमर्श भग्नोरुदण्डे धृतराष्ट्रपुत्रे
भर्तुः प्रियं द्रौणिरिति स्म पश्यन्कृष्णासुतानां स्वपतां शिरांसि
उपाहरद्विप्रियमेव तस्य जुगुप्सितं कर्म विगर्हयन्ति
माता शिशूनां निधनं सुतानां निशम्य घोरं परितप्यमाना
तदारुदद्वाष्पकलाकुलाक्षी तां सान्त्वयन्नाह किरीटमाली
तदा शुचस्ते प्रमृजामि भद्रे यद्ब्रह्मबन्धोः शिर आततायिनः
गाण्डीवमुक्तैर्विशिखैरुपाहरे त्वाक्रम्य यत्स्नास्यसि दग्धपुत्रा
इति प्रियां वल्गुविचित्रजल्पैः स सान्त्वयित्वाच्युतमित्रसूतः
अन्वाद्रवद्दंशित उग्रधन्वा कपिध्वजो गुरुपुत्रं रथेन
तमापतन्तं स विलक्ष्य दूरात्कुमारहोद्विग्नमना रथेन
पराद्रवत्प्राणपरीप्सुरुर्व्यां यावद्गमं रुद्रभयाद्यथा कः
यदाशरणमात्मानमैक्षत श्रान्तवाजिनम्‌
अस्त्रं ब्रह्मशिरो मेने आत्मत्राणं द्विजात्मजः
अथोपस्पृश्य सलिलं सन्दधे तत्समाहितः
अजानन्नपि संहारं प्राणकृच्छ्र उपस्थिते
ततः प्रादुष्कृतं तेजः प्रचण्डं सर्वतो दिशम्‌
प्राणापदमभिप्रेक्ष्य विष्णुं जिष्णुरुवाच ह
अर्जुन उवाच
कृष्ण कृष्ण महाबाहो भक्तानामभयङ्कर
त्वमेको दह्यमानानामपवर्गोऽसि संसृतेः
त्वमाद्यः पुरुषः साक्षादीश्वरः प्रकृतेः परः
मायां व्युदस्य चिच्छक्त्या कैवल्ये स्थित आत्मनि
स एव जीवलोकस्य मायामोहितचेतसः
विधत्से स्वेन वीर्येण श्रेयो धर्मादिलक्षणम्‌
तथायं चावतारस्ते भुवो भारजिहीर्षया
स्वानां चानन्यभावानामनुध्यानाय चासकृत्‌
किमिदं स्वित्कुतो वेति देवदेव न वेद्म्यहम्‌
सर्वतो मुखमायाति तेजः परमदारुणम्‌
श्रीभगवानुवाच
वेत्थेदं द्रोणपुत्रस्य ब्राह्ममस्त्रं प्रदर्शितम्‌
नैवासौ वेद संहारं प्राणबाध उपस्थिते
न ह्यस्यान्यतमं किञ्चिदस्त्रं प्रत्यवकर्शनम्‌
जह्यस्त्रतेज उन्नद्धमस्त्रज्ञो ह्यस्त्रतेजसा
सूत उवाच
श्रुत्वा भगवता प्रोक्तं फाल्गुनः परवीरहा
स्पृष्ट्वापस्तं परिक्रम्य ब्राह्मं ब्राह्मास्त्रं सन्दधे
संहत्यान्योन्यमुभयोस्तेजसी शरसंवृते
आवृत्य रोदसी खं च ववृधातेऽर्कवह्निवत्‌
दृष्ट्वास्त्रतेजस्तु तयोस्त्रील्लोकान्प्रदहन्महत्‌
दह्यमानाः प्रजाः सर्वाः सांवर्तकममंसत
प्रजोपद्रवमालक्ष्य लोकव्यतिकरं च तम्‌
मतं च वासुदेवस्य सञ्जहारार्जुनो द्वयम्‌
तत आसाद्य तरसा दारुणं गौतमीसुतम्‌
बबन्धामर्षताम्राक्षः पशुं रशनया यथा
शिबिराय निनीषन्तं रज्ज्वा बद्ध्वा रिपुं बलात्‌
प्राहार्जुनं प्रकुपितो भगवानम्बुजेक्षणः
मैनं पार्थार्हसि त्रातुं ब्रह्मबन्धुमिमं जहि
योऽसावनागसः सुप्तानवधीन्निशि बालकान्‌
मत्तं प्रमत्तमुन्मत्तं सुप्तं बालं स्त्रियं जडम्‌
प्रपन्नं विरथं भीतं न रिपुं हन्ति धर्मवित्‌
स्वप्राणान्यः परप्राणैः प्रपुष्णात्यघृणः खलः
तद्वधस्तस्य हि श्रेयो यद्दोषाद्यात्यधः पुमान्‌
प्रतिश्रुतं च भवता पाञ्चाल्यै शृण्वतो मम
आहरिष्ये शिरस्तस्य यस्ते मानिनि पुत्रहा
तदसौ वध्यतां पाप आतताय्यात्मबन्धुहा
भर्तुश्च विप्रियं वीर कृतवान्कुलपांसनः
सूत उवाच
एवं परीक्षता धर्मं पार्थः कृष्णेन चोदितः
नैच्छद्धन्तुं गुरुसुतं यद्यप्यात्महनं महान्‌
अथोपेत्य स्वशिबिरं गोविन्दप्रियसारथिः
न्यवेदयत्तं प्रियायै शोचन्त्या आत्मजान्हतान्‌
तथाहृतं पशुवत्पाशबद्धमवाङ्मुखं कर्मजुगुप्सितेन
निरीक्ष्य कृष्णापकृतं गुरोः सुतं वामस्वभावा कृपया ननाम च
उवाच चासहन्त्यस्य बन्धनानयनं सती
मुच्यतां मुच्यतामेष ब्राह्मणो नितरां गुरुः
सरहस्यो धनुर्वेदः सविसर्गोपसंयमः
अस्त्रग्रामश्च भवता शिक्षितो यदनुग्रहात्‌
स एष भगवान्द्रोणः प्रजारूपेण वर्तते
तस्यात्मनोऽर्धं पत्न्यास्ते नान्वगाद्वीरसूः कृपी
तद्धर्मज्ञ महाभाग भवद्भिर्गौरवं कुलम्‌
वृजिनं नार्हति प्राप्तुं पूज्यं वन्द्यमभीक्ष्णशः
मा रोदीदस्य जननी गौतमी पतिदेवता
यथाहं मृतवत्सार्ता रोदिम्यश्रुमुखी मुहुः
यैः कोपितं ब्रह्मकुलं राजन्यैरजितात्मभिः
तत्कुलं प्रदहत्याशु सानुबन्धं शुचार्पितम्‌
सूत उवाच
धर्म्यं न्याय्यं सकरुणं निर्व्यलीकं समं महत्‌
राजा धर्मसुतो राज्ञ्याःप्रत्यनन्दद्वचो द्विजाः
नकुलः सहदेवश्च युयुधानो धनञ्जयः
भगवान्देवकीपुत्रो ये चान्ये याश्च योषितः
तत्राहामर्षितो भीमस्तस्य श्रेयान्वधः स्मृतः
न भर्तुर्नात्मनश्चार्थे योऽहन्सुप्तान्शिशून्वृथा
निशम्य भीमगदितं द्रौपद्याश्च चतुर्भुजः
आलोक्य वदनं सख्युरिदमाह हसन्निव
श्रीभगवानुवाच
ब्रह्मबन्धुर्न हन्तव्य आततायी वधार्हणः
मयैवोभयमाम्नातं परिपाह्यनुशासनम्‌
कुरु प्रतिश्रुतं सत्यं यत्तत्सान्त्वयता प्रियाम्‌
प्रियं च भीमसेनस्य पाञ्चाल्या मह्यमेव च
सूत उवाच
अर्जुनः सहसाज्ञाय हरेर्हार्दमथासिना
मणिं जहार मूर्धन्यं द्विजस्य सहमूर्धजम्‌
विमुच्य रशनाबद्धं बालहत्याहतप्रभम्‌
तेजसा मणिना हीनं शिबिरान्निरयापयत्‌
वपनं द्रविणादानं स्थानान्निर्यापणं तथा
एष हि ब्रह्मबन्धूनां वधो नान्योऽस्ति दैहिकः
पुत्रशोकातुराः सर्वे पाण्डवाः सह कृष्णया
स्वानां मृतानां यत्कृत्यं चक्रुर्निर्हरणादिकम्‌
सूत उवाच
अथ ते सम्परेतानां स्वानामुदकमिच्छताम्‌
दातुं सकृष्णा गङ्गायां पुरस्कृत्य ययुः स्त्रियः
ते निनीयोदकं सर्वे विलप्य च भृशं पुनः
आप्लुता हरिपादाब्जरजःपूतसरिज्जले
तत्रासीनं कुरुपतिं धृतराष्ट्रं सहानुजम्‌
गान्धारीं पुत्रशोकार्तां पृथां कृष्णां च माधवः
सान्त्वयामास मुनिभिर्हतबन्धूञ्शुचार्पितान्‌
भूतेषु कालस्य गतिं दर्शयन्न प्रतिक्रियाम्‌
साधयित्वाजातशत्रोः स्वं राज्यं कितवैर्हृतम्‌
घातयित्वासतो राज्ञः कचस्पर्शक्षतायुषः
याजयित्वाश्वमेधैस्तं त्रिभिरुत्तमकल्पकैः
तद्यशः पावनं दिक्षु शतमन्योरिवातनोत्‌
आमन्त्र्य पाण्डुपुत्रांश्च शैनेयोद्धवसंयुतः
द्वैपायनादिभिर्विप्रैः पूजितैः प्रतिपूजितः
गन्तुं कृतमतिर्ब्रह्मन्द्वारकां रथमास्थितः
उपलेभेऽभिधावन्तीमुत्तरां भयविह्वलाम्‌
उत्तरोवाच
पाहि पाहि महायोगिन्देवदेव जगत्पते
नान्यं त्वदभयं पश्ये यत्र मृत्युः परस्परम्‌
अभिद्रवति मामीश शरस्तप्तायसो विभो
कामं दहतु मां नाथ मा मे गर्भो निपात्यताम्‌
सूत उवाच
उपधार्य वचस्तस्या भगवान्भक्तवत्सलः
अपाण्डवमिदं कर्तुं द्रौणेरस्त्रमबुध्यत
तर्ह्येवाथ मुनिश्रेष्ठ पाण्डवाः पञ्च सायकान्‌
आत्मनोऽभिमुखान्दीप्तानालक्ष्यास्त्राण्युपाददुः
व्यसनं वीक्ष्य तत्तेषामनन्यविषयात्मनाम्‌
सुदर्शनेन स्वास्त्रेण स्वानां रक्षां व्यधाद्विभुः
अन्तःस्थः सर्वभूतानामात्मा योगेश्वरो हरिः
स्वमाययावृणोद्गर्भं वैराट्याः कुरुतन्तवे
यद्यप्यस्त्रं ब्रह्मशिरस्त्वमोघं चाप्रतिक्रियम्‌
वैष्णवं तेज आसाद्य समशाम्यद्भृगूद्वह
मा मंस्था ह्येतदाश्चर्यं सर्वाश्चर्यमये ञ्च्युते
य इदं मायया देव्या सृजत्यवति हन्त्यजः
ब्रह्मतेजोविनिर्मुक्तैरात्मजैः सह कृष्णया
प्रयाणाभिमुखं कृष्णमिदमाह पृथा सती
कुन्त्युवाच
नमस्ये पुरुषं त्वाद्यमीश्वरं प्रकृतेः परम्‌
अलक्ष्यं सर्वभूतानामन्तर्बहिरवस्थितम्‌
मायाजवनिकाच्छन्नमज्ञाधोक्षजमव्ययम्‌
न लक्ष्यसे मूढदृशा नटो नाट्यधरो यथा
तथा परमहंसानां मुनीनाममलात्मनाम्‌
भक्तियोगविधानार्थं कथं पश्येम हि स्त्रियः
कृष्णाय वासुदेवाय देवकीनन्दनाय च
नन्दगोपकुमाराय गोविन्दाय नमो नमः
नमः पङ्कजनाभाय नमः पङ्कजमालिने
नमः पङ्कजनेत्राय नमस्ते पङ्कजाङ्घ्रये
यथा हृषीकेश खलेन देवकी कंसेन रुद्धातिचिरं शुचार्पिता
विमोचिताहं च सहात्मजा विभो त्वयैव नाथेन मुहुर्विपद्गणात्‌
विषान्महाग्नेः पुरुषाददर्शनादसत्सभाया वनवासकृच्छ्रतः
मृधे मृधेऽनेकमहारथास्त्रतो द्रौण्यस्त्रतश्चास्म हरेऽभिरक्षिताः
विपदः सन्तु ताः शश्वत्तत्र तत्र जगद्गुरो
भवतो दर्शनं यत्स्यादपुनर्भवदर्शनम्‌
जन्मैश्वर्यश्रुतश्रीभिरेधमानमदः पुमान्‌
नैवार्हत्यभिधातुं वै त्वामकिञ्चनगोचरम्‌
नमोऽकिञ्चनवित्ताय निवृत्तगुणवृत्तये
आत्मारामाय शान्ताय कैवल्यपतये नमः
मन्ये त्वां कालमीशानमनादिनिधनं विभुम्‌
समं चरन्तं सर्वत्र भूतानां यन्मिथः कलिः
न वेद कश्चिद्भगवंश्चिकीर्षितं तवेहमानस्य नृणां विडम्बनम्‌
न यस्य कश्चिद्दयितोऽस्ति कर्हिचिद्द्वेष्यश्च यस्मिन्विषमा मतिर्नृणाम्‌
जन्म कर्म च विश्वात्मन्नजस्याकर्तुरात्मनः
तिर्यङ्न्ऱ्‌^ईषिषु यादःसु तदत्यन्तविडम्बनम्‌
गोप्याददे त्वयि कृतागसि दाम तावद्या ते दशाश्रुकलिलाञ्जनसम्भ्रमाक्षम्‌
वक्त्रं निनीय भयभावनया स्थितस्य सा मां विमोहयति भीरपि यद्बिभेति
केचिदाहुरजं जातं पुण्यश्लोकस्य कीर्तये
यदोः प्रियस्यान्ववाये मलयस्येव चन्दनम्‌
अपरे वसुदेवस्य देवक्यां याचितोऽभ्यगात्‌
अजस्त्वमस्य क्षेमाय वधाय च सुरद्विषाम्‌
भारावतारणायान्ये भुवो नाव इवोदधौ
सीदन्त्या भूरिभारेण जातो ह्यात्मभुवार्थितः
भवेऽस्मिन्क्लिश्यमानानामविद्याकामकर्मभिः
श्रवणस्मरणार्हाणि करिष्यन्निति केचन
शृण्वन्ति गायन्ति गृणन्त्यभीक्ष्णशः स्मरन्ति नन्दन्ति तवेहितं जनाः
त एव पश्यन्त्यचिरेण तावकं भवप्रवाहोपरमं पदाम्बुजम्‌
अप्यद्य नस्त्वं स्वकृतेहित प्रभो जिहाससि स्वित्सुहृदोऽनुजीविनः
येषां न चान्यद्भवतः पदाम्बुजात्परायणं राजसु योजितांहसाम्‌
के वयं नामरूपाभ्यां यदुभिः सह पाण्डवाः
भवतोऽदर्शनं यर्हि हृषीकाणामिवेशितुः
नेयं शोभिष्यते तत्र यथेदानीं गदाधर
त्वत्पदैरङ्किता भाति स्वलक्षणविलक्षितैः
इमे जनपदाः स्वृद्धाः सुपक्वौषधिवीरुधः
वनाद्रिनद्युदन्वन्तो ह्येधन्ते तव वीक्षितैः
अथ विश्वेश विश्वात्मन्विश्वमूर्ते स्वकेषु मे
स्नेहपाशमिमं छिन्धि दृढं पाण्डुषु वृष्णिषु
त्वयि मेऽनन्यविषया मतिर्मधुपतेऽसकृत्‌
रतिमुद्वहतादद्धा गङ्गेवौघमुदन्वति
श्रीकृष्ण कृष्णसख वृष्ण्यृषभावनिध्रुग्राजन्यवंशदहनानपवर्गवीर्य
गोविन्द गोद्विजसुरार्तिहरावतार योगेश्वराखिलगुरो भगवन्नमस्ते
सूत उवाच
पृथयेत्थं कलपदैः परिणूताखिलोदयः
मन्दं जहास वैकुण्ठो मोहयन्निव मायया
तां बाढमित्युपामन्त्र्य प्रविश्य गजसाह्वयम्‌
स्त्रियश्च स्वपुरं यास्यन्प्रेम्णा राज्ञा निवारितः
व्यासाद्यैरीश्वरेहाज्ञैः कृष्णेनाद्भुतकर्मणा
प्रबोधितोऽपीतिहासैर्नाबुध्यत शुचार्पितः
आह राजा धर्मसुतश्चिन्तयन्सुहृदां वधम्‌
प्राकृतेनात्मना विप्राः स्नेहमोहवशं गतः
अहो मे पश्यताज्ञानं हृदि रूढं दुरात्मनः
पारक्यस्यैव देहस्य बह्व्यो मेऽक्षौहिणीर्हताः
बालद्विजसुहृन्मित्र पितृभ्रातृगुरुद्रुहः
न मे स्यान्निरयान्मोक्षो ह्यपि वर्षायुतायुतैः
नैनो राज्ञः प्रजाभर्तुर्धर्मयुद्धे वधो द्विषाम्‌
इति मे न तु बोधाय कल्पते शासनं वचः
स्त्रीणां मद्धतबन्धूनां द्रोहो योऽसाविहोत्थितः
कर्मभिर्गृहमेधीयैर्नाहं कल्पो व्यपोहितुम्‌
यथा पङ्केन पङ्काम्भः सुरया वा सुराकृतम्‌
भूतहत्यां तथैवैकां न यज्ञैर्मार्ष्टुमर्हति
सूत उवाच
इति भीतः प्रजाद्रोहात्सर्वधर्मविवित्सया
ततो विनशनं प्रागाद्यत्र देवव्रतोऽपतत्‌
तदा ते भ्रातरः सर्वे सदश्वैः स्वर्णभूषितैः
अन्वगच्छन्रथैर्विप्रा व्यासधौम्यादयस्तथा
भगवानपि विप्रर्षे रथेन सधनञ्जयः
स तैर्व्यरोचत नृपः कुवेर इव गुह्यकैः
दृष्ट्वा निपतितं भूमौ दिवश्च्युतमिवामरम्‌
प्रणेमुः पाण्डवा भीष्मं सानुगाः सह चक्रिणा
तत्र ब्रह्मर्षयः सर्वे देवर्षयश्च सत्तम
राजर्षयश्च तत्रासन्द्रष्टुं भरतपुङ्गवम्‌
पर्वतो नारदो धौम्यो भगवान्बादरायणः
बृहदश्वो भरद्वाजः सशिष्यो रेणुकासुतः
वसिष्ठ इन्द्रप्रमदस्त्रितो गृत्समदोऽसितः
कक्षीवान्गौतमोऽत्रिश्च कौशिकोऽथ सुदर्शनः
अन्ये च मुनयो ब्रह्मन्ब्रह्मरातादयोऽमलाः
शिष्यैरुपेता आजग्मुः कश्यपाङ्गिरसादयः
तान्समेतान्महाभागानुपलभ्य वसूत्तमः
पूजयामास धर्मज्ञो देशकालविभागवित्‌
कृष्णं च तत्प्रभावज्ञ आसीनं जगदीश्वरम्‌
हृदिस्थं पूजयामास माययोपात्तविग्रहम्‌
पाण्डुपुत्रानुपासीनान्प्रश्रयप्रेमसङ्गतान्‌
अभ्याचष्टानुरागाश्रैरन्धीभूतेन चक्षुषा
अहो कष्टमहोऽन्याय्यं यद्यूयं धर्मनन्दनाः
जीवितुं नार्हथ क्लिष्टं विप्रधर्माच्युताश्रयाः
संस्थितेऽतिरथे पाण्डौ पृथा बालप्रजा वधूः
युष्मत्कृते बहून्क्लेशान्प्राप्ता तोकवती मुहुः
सर्वं कालकृतं मन्ये भवतां च यदप्रियम्‌
सपालो यद्वशे लोको वायोरिव घनावलिः
यत्र धर्मसुतो राजा गदापाणिर्वृकोदरः
कृष्णोऽस्त्री गाण्डिवं चापं सुहृत्कृष्णस्ततो विपत्‌
न ह्यस्य कर्हिचिद्राजन्पुमान्वेद विधित्सितम्‌
यद्विजिज्ञासया युक्ता मुह्यन्ति कवयोऽपि हि
तस्मादिदं दैवतन्त्रं व्यवस्य भरतर्षभ
तस्यानुविहितोऽनाथा नाथ पाहि प्रजाः प्रभो
एष वै भगवान्साक्षादाद्यो नारायणः पुमान्‌
मोहयन्मायया लोकं गूढश्चरति वृष्णिषु
अस्यानुभावं भगवान्वेद गुह्यतमं शिवः
देवर्षिर्नारदः साक्षाद्भगवान्कपिलो नृप
यं मन्यसे मातुलेयं प्रियं मित्रं सुहृत्तमम्‌
अकरोः सचिवं दूतं सौहृदादथ सारथिम्‌
सर्वात्मनः समदृशो ह्यद्वयस्यानहङ्कृतेः
तत्कृतं मतिवैषम्यं निरवद्यस्य न क्वचित्‌
तथाप्येकान्तभक्तेषु पश्य भूपानुकम्पितम्‌
यन्मेऽसूंस्त्यजतः साक्षात्कृष्णो दर्शनमागतः
भक्त्यावेश्य मनो यस्मिन्वाचा यन्नाम कीर्तयन्‌
त्यजन्कलेवरं योगी मुच्यते कामकर्मभिः
स देवदेवो भगवान्प्रतीक्षतां कलेवरं यावदिदं हिनोम्यहम्‌
प्रसन्नहासारुणलोचनोल्लसन्मुखाम्बुजो ध्यानपथश्चतुर्भुजः
सूत उवाच
युधिष्ठिरस्तदाकर्ण्य शयानं शरपञ्जरे
अपृच्छद्विविधान्धर्मानृषीणां चानुशृण्वताम्‌
पुरुषस्वभावविहितान्यथावर्णं यथाश्रमम्‌
वैराग्यरागोपाधिभ्यामाम्नातोभयलक्षणान्‌
दानधर्मान्राजधर्मान्मोक्षधर्मान्विभागशः
स्त्रीधर्मान्भगवद्धर्मान्समासव्यासयोगतः
धर्मार्थकाममोक्षांश्च सहोपायान्यथा मुने
नानाख्यानेतिहासेषु वर्णयामास तत्त्ववित्‌
धर्मं प्रवदतस्तस्य स कालः प्रत्युपस्थितः
यो योगिनश्छन्दमृत्योर्वाञ्छितस्तूत्तरायणः
तदोपसंहृत्य गिरः सहस्रणीर्विमुक्तसङ्गं मन आदिपूरुषे
कृष्णे लसत्पीतपटे चतुर्भुजे पुरः स्थितेऽमीलितदृग्व्यधारयत्‌
विशुद्धया धारणया हताशुभस्तदीक्षयैवाशु गतायुधश्रमः
निवृत्तसर्वेन्द्रियवृत्तिविभ्रमस्तुष्टाव जन्यं विसृजञ्जनार्दनम्‌
श्रीभीष्म उवाच
इति मतिरुपकल्पिता वितृष्णा भगवति सात्वतपुङ्गवे विभूम्नि
स्वसुखमुपगते क्वचिद्विहर्तुं प्रकृतिमुपेयुषि यद्भवप्रवाहः
त्रिभुवनकमनं तमालवर्णं रविकरगौरवराम्बरं दधाने
वपुरलककुलावृताननाब्जं विजयसखे रतिरस्तु मेऽनवद्या
युधि तुरगरजोविधूम्रविष्वक्कचलुलितश्रमवार्यलङ्कृतास्ये
मम निशितशरैर्विभिद्यमान त्वचि विलसत्कवचेऽस्तु कृष्ण आत्मा
सपदि सखिवचो निशम्य मध्ये निजपरयोर्बलयो रथं निवेश्य
स्थितवति परसैनिकायुरक्ष्णा हृतवति पार्थसखे रतिर्ममास्तु
व्यवहितपृतनामुखं निरीक्ष्य स्वजनवधाद्विमुखस्य दोषबुद्ध्या
कुमतिमहरदात्मविद्यया यश्चरणरतिः परमस्य तस्य मेऽस्तु
स्वनिगममपहाय मत्प्रतिज्ञामृतमधिकर्तुमवप्लुतो रथस्थः
धृतरथचरणोऽभ्ययाच्चलद्गुर्हरिरिव हन्तुमिभं गतोत्तरीयः
शितविशिखहतो विशीर्णदंशः क्षतजपरिप्लुत आततायिनो मे
प्रसभमभिससार मद्वधार्थं स भवतु मे भगवान्गतिर्मुकुन्दः
विजयरथकुटुम्ब आत्ततोत्रे धृतहयरश्मिनि तच्छ्रियेक्षणीये
भगवति रतिरस्तु मे मुमूर्षोर्यमिह निरीक्ष्य हता गताः स्वरूपम्‌
ललितगतिविलासवल्गुहास प्रणयनिरीक्षणकल्पितोरुमानाः
कृतमनुकृतवत्य उन्मदान्धाः प्रकृतिमगन्किल यस्य गोपवध्वः
मुनिगणनृपवर्यसङ्कुलेऽन्तः सदसि युधिष्ठिरराजसूय एषाम्‌
अर्हणमुपपेद ईक्षणीयो मम दृशिगोचर एष आविरात्मा
तमिममहमजं शरीरभाजां हृदि हृदि धिष्ठितमात्मकल्पितानाम्‌
प्रतिदृशमिव नैकधार्कमेकं समधिगतोऽस्मि विधूतभेदमोहः
सूत उवाच
कृष्ण एवं भगवति मनोवाग्दृष्टिवृत्तिभिः
आत्मन्यात्मानमावेश्य सोऽन्तःश्वास उपारमत्‌
सम्पद्यमानमाज्ञाय भीष्मं ब्रह्मणि निष्कले
सर्वे बभूवुस्ते तूष्णीं वयांसीव दिनात्यये
तत्र दुन्दुभयो नेदुर्देवमानववादिताः
शशंसुः साधवो राज्ञां खात्पेतुः पुष्पवृष्टयः
तस्य निर्हरणादीनि सम्परेतस्य भार्गव
युधिष्ठिरः कारयित्वा मुहूर्तं दुःखितोऽभवत्‌
तुष्टुवुर्मुनयो हृष्टाः कृष्णं तद्गुह्यनामभिः
ततस्ते कृष्णहृदयाः स्वाश्रमान्प्रययुः पुनः
ततो युधिष्ठिरो गत्वा सहकृष्णो गजाह्वयम्‌
पितरं सान्त्वयामास गान्धारीं च तपस्विनीम्‌
पित्रा चानुमतो राजा वासुदेवानुमोदितः
चकार राज्यं धर्मेण पितृपैतामहं विभुः
शौनक उवाच
हत्वा स्वरिक्थस्पृध आततायिनो युधिष्ठिरो धर्मभृतां वरिष्ठः
सहानुजैः प्रत्यवरुद्धभोजनः कथं प्रवृत्तः किमकारषीत्ततः
सूत उवाच
वंशं कुरोर्वंशदवाग्निनिर्हृतं संरोहयित्वा भवभावनो हरिः
निवेशयित्वा निजराज्य ईश्वरो युधिष्ठिरं प्रीतमना बभूव ह
निशम्य भीष्मोक्तमथाच्युतोक्तं प्रवृत्तविज्ञानविधूतविभ्रमः
शशास गामिन्द्र इवाजिताश्रयः परिध्युपान्तामनुजानुवर्तितः
कामं ववर्ष पर्जन्यः सर्वकामदुघा मही
सिषिचुः स्म व्रजान्गावः पयसोधस्वतीर्मुदा
नद्यः समुद्रा गिरयः सवनस्पतिवीरुधः
फलन्त्योषधयः सर्वाः काममन्वृतु तस्य वै
नाधयो व्याधयः क्लेशा दैवभूतात्महेतवः
अजातशत्रावभवन्जन्तूनां राज्ञि कर्हिचित्‌
उषित्वा हास्तिनपुरे मासान्कतिपयान्हरिः
सुहृदां च विशोकाय स्वसुश्च प्रियकाम्यया
आमन्त्र्य चाभ्यनुज्ञातः परिष्वज्याभिवाद्य तम्‌
आरुरोह रथं कैश्चित्परिष्वक्तोऽभिवादितः
सुभद्रा द्रौपदी कुन्ती विराटतनया तथा
गान्धारी धृतराष्ट्रश्च युयुत्सुर्गौतमो यमौ
वृकोदरश्च धौम्यश्च स्त्रियो मत्स्यसुतादयः
न सेहिरे विमुह्यन्तो विरहं शार्ङ्गधन्वनः
सत्सङ्गान्मुक्तदुःसङ्गो हातुं नोत्सहते बुधः
कीर्त्यमानं यशो यस्य सकृदाकर्ण्य रोचनम्‌
तस्मिन्न्यस्तधियः पार्थाः सहेरन्विरहं कथम्‌
दर्शनस्पर्शसंलाप शयनासनभोजनैः
सर्वे तेऽनिमिषैरक्षैस्तमनु द्रुतचेतसः
वीक्षन्तः स्नेहसम्बद्धा विचेलुस्तत्र तत्र ह
न्यरुन्धन्नुद्गलद्बाष्पमौत्कण्ठ्याद्देवकीसुते
निर्यात्यगारान्नोऽभद्रमिति स्याद्बान्धवस्त्रियः
मृदङ्गशङ्खभेर्यश्च वीणापणवगोमुखाः
धुन्धुर्यानकघण्टाद्या नेदुर्दुन्दुभयस्तथा
प्रासादशिखरारूढाः कुरुनार्यो दिदृक्षया
ववृषुः कुसुमैः कृष्णं प्रेमव्रीडास्मितेक्षणाः
सितातपत्रं जग्राह मुक्तादामविभूषितम्‌
रत्नदण्डं गुडाकेशः प्रियः प्रियतमस्य ह
उद्धवः सात्यकिश्चैव व्यजने परमाद्भुते
विकीर्यमाणः कुसुमै रेजे मधुपतिः पथि
अश्रूयन्ताशिषः सत्यास्तत्र तत्र द्विजेरिताः
नानुरूपानुरूपाश्च निर्गुणस्य गुणात्मनः
अन्योन्यमासीत्सञ्जल्प उत्तमश्लोकचेतसाम्‌
कौरवेन्द्रपुरस्त्रीणां सर्वश्रुतिमनोहरः
स वै किलायं पुरुषः पुरातनो य एक आसीदविशेष आत्मनि
अग्रे गुणेभ्यो जगदात्मनीश्वरे निमीलितात्मन्निशि सुप्तशक्तिषु
स एव भूयो निजवीर्यचोदितां स्वजीवमायां प्रकृतिं सिसृक्षतीम्‌
अनामरूपात्मनि रूपनामनी विधित्समानोऽनुससार शास्त्रकृत्‌
स वा अयं यत्पदमत्र सूरयो जितेन्द्रिया निर्जितमातरिश्वनः
पश्यन्ति भक्त्युत्कलितामलात्मना नन्वेष सत्त्वं परिमार्ष्टुमर्हति
स वा अयं सख्यनुगीतसत्कथो वेदेषु गुह्येषु च गुह्यवादिभिः
य एक ईशो जगदात्मलीलया सृजत्यवत्यत्ति न तत्र सज्जते
यदा ह्यधर्मेण तमोधियो नृपा जीवन्ति तत्रैष हि सत्त्वतः किल
धत्ते भगं सत्यमृतं दयां यशो भवाय रूपाणि दधद्युगे युगे
अहो अलं श्लाघ्यतमं यदोः कुलमहो अलं पुण्यतमं मधोर्वनम्‌
यदेष पुंसामृषभः श्रियः पतिः स्वजन्मना चङ्क्रमणेन चाञ्चति
अहो बत स्वर्यशसस्तिरस्करी कुशस्थली पुण्ययशस्करी भुवः
पश्यन्ति नित्यं यदनुग्रहेषितं स्मितावलोकं स्वपतिं स्म यत्प्रजाः
नूनं व्रतस्नानहुतादिनेश्वरः समर्चितो ह्यस्य गृहीतपाणिभिः
पिबन्ति याः सख्यधरामृतं मुहुर्व्रजस्त्रियः सम्मुमुहुर्यदाशयाः
या वीर्यशुल्केन हृताः स्वयंवरे प्रमथ्य चैद्यप्रमुखान्हि शुष्मिणः
प्रद्युम्नसाम्बाम्बसुतादयोऽपरा याश्चाहृता भौमवधे सहस्रशः
एताः परं स्त्रीत्वमपास्तपेशलं निरस्तशौचं बत साधु कुर्वते
यासां गृहात्पुष्करलोचनः पतिर्न जात्वपैत्याहृतिभिर्हृदि स्पृशन्‌
एवंविधा गदन्तीनां स गिरः पुरयोषिताम्‌
निरीक्षणेनाभिनन्दन्सस्मितेन ययौ हरिः
अजातशत्रुः पृतनां गोपीथाय मधुद्विषः
परेभ्यः शङ्कितः स्नेहात्प्रायुङ्क्त चतुरङ्गिणीम्‌
अथ दूरागतान्शौरिः कौरवान्विरहातुरान्‌
सन्निवर्त्य दृढं स्निग्धान्प्रायात्स्वनगरीं प्रियैः
कुरुजाङ्गलपाञ्चालान्शूरसेनान्सयामुनान्‌
ब्रह्मावर्तं कुरुक्षेत्रं मत्स्यान्सारस्वतानथ
मरुधन्वमतिक्रम्य सौवीराभीरयोः परान्‌
आनर्तान्भार्गवोपागाच्छ्रान्तवाहो मनाग्विभुः
तत्र तत्र ह तत्रत्यैर्हरिः प्रत्युद्यतार्हणः
सायं भेजे दिशं पश्चाद्गविष्ठो गां गतस्तदा
सूत उवाच
आनर्तान्स उपव्रज्य स्वृद्धाञ्जनपदान्स्वकान्‌
दध्मौ दरवरं तेषां विषादं शमयन्निव
स उच्चकाशे धवलोदरो दरोऽप्युरुक्रमस्याधरशोणशोणिमा
दाध्मायमानः करकञ्जसम्पुटे यथाब्जखण्डे कलहंस उत्स्वनः
तमुपश्रुत्य निनदं जगद्भयभयावहम्‌
प्रत्युद्ययुः प्रजाः सर्वा भर्तृदर्शनलालसाः
तत्रोपनीतबलयो रवेर्दीपमिवादृताः
आत्मारामं पूर्णकामं निजलाभेन नित्यदा
प्रीत्युत्फुल्लमुखाः प्रोचुर्हर्षगद्गदया गिरा
पितरं सर्वसुहृदमवितारमिवार्भकाः
नताः स्म ते नाथ सदाङ्घ्रिपङ्कजं विरिञ्चवैरिञ्च्यसुरेन्द्रवन्दितम्‌
परायणं क्षेममिहेच्छतां परं न यत्र कालः प्रभवेत्परः प्रभुः
भवाय नस्त्वं भव विश्वभावन त्वमेव माताथ सुहृत्पतिः पिता
त्वं सद्गुरुर्नः परमं च दैवतं यस्यानुवृत्त्या कृतिनो बभूविम
अहो सनाथा भवता स्म यद्वयं त्रैविष्टपानामपि दूरदर्शनम्‌
प्रेमस्मितस्निग्धनिरीक्षणाननं पश्येम रूपं तव सर्वसौभगम्‌
यर्ह्यम्बुजाक्षापससार भो भवान्कुरून्मधून्वाथ सुहृद्दिदृक्षया
तत्राब्दकोटिप्रतिमः क्षणो भवेद्रविं विनाक्ष्णोरिव नस्तवाच्युत
कथं वयं नाथ चिरोषिते त्वयि प्रसन्नदृष्ट्याखिलतापशोषणम्‌
जीवेम ते सुन्दरहासशोभितमपश्यमाना वदनं मनोहरम्‌
इति चोदीरिता वाचः प्रजानां भक्तवत्सलः
शृण्वानोऽनुग्रहं दृष्ट्या वितन्वन्प्राविशत्पुरम्‌
मधुभोजदशार्हार्हकुकुरान्धकवृष्णिभिः
आत्मतुल्यबलैर्गुप्तां नागैर्भोगवतीमिव
सर्वर्तुसर्वविभवपुण्यवृक्षलताश्रमैः
उद्यानोपवनारामैर्वृतपद्माकरश्रियम्‌
गोपुरद्वारमार्गेषु कृतकौतुकतोरणाम्‌
चित्रध्वजपताकाग्रैरन्तः प्रतिहतातपाम्‌
सम्मार्जितमहामार्ग रथ्यापणकचत्वराम्‌
सिक्तां गन्धजलैरुप्तां फलपुष्पाक्षताङ्कुरैः
द्वारि द्वारि गृहाणां च दध्यक्षतफलेक्षुभिः
अलङ्कृतां पूर्णकुम्भैर्बलिभिर्धूपदीपकैः
निशम्य प्रेष्ठमायान्तं वसुदेवो महामनाः
अक्रूरश्चोग्रसेनश्च रामश्चाद्भुतविक्रमः
प्रद्युम्नश्चारुदेष्णश्च साम्बो जाम्बवतीसुतः
प्रहर्षवेगोच्छशितशयनासनभोजनाः
वारणेन्द्रं पुरस्कृत्य ब्राह्मणैः ससुमङ्गलैः
शङ्खतूर्यनिनादेन ब्रह्मघोषेण चादृताः
प्रत्युज्जग्मू रथैर्हृष्टाः प्रणयागतसाध्वसाः
वारमुख्याश्च शतशो यानैस्तद्दर्शनोत्सुकाः
लसत्कुण्डलनिर्भातकपोलवदनश्रियः
नटनर्तकगन्धर्वाः सूतमागधवन्दिनः
गायन्ति चोत्तमश्लोकचरितान्यद्भुतानि च
भगवांस्तत्र बन्धूनां पौराणामनुवर्तिनाम्‌
यथाविध्युपसङ्गम्य सर्वेषां मानमादधे
प्रह्वाभिवादनाश्लेषकरस्पर्शस्मितेक्षणैः
आश्वास्य चाश्वपाकेभ्यो वरैश्चाभिमतैर्विभुः
स्वयं च गुरुभिर्विप्रैः सदारैः स्थविरैरपि
आशीर्भिर्युज्यमानोऽन्यैर्वन्दिभिश्चाविशत्पुरम्‌
राजमार्गं गते कृष्णे द्वारकायाः कुलस्त्रियः
हर्म्याण्यारुरुहुर्विप्र तदीक्षणमहोत्सवाः
नित्यं निरीक्षमाणानां यदपि द्वारकौकसाम्‌
न वितृप्यन्ति हि दृशः श्रियो धामाङ्गमच्युतम्‌
श्रियो निवासो यस्योरः पानपात्रं मुखं दृशाम्‌
बाहवो लोकपालानां सारङ्गाणां पदाम्बुजम्‌
सितातपत्रव्यजनैरुपस्कृतः प्रसूनवर्षैरभिवर्षितः पथि
पिशङ्गवासा वनमालया बभौ घनो यथार्कोडुपचापवैद्युतैः
प्रविष्टस्तु गृहं पित्रोः परिष्वक्तः स्वमातृभिः
ववन्दे शिरसा सप्त देवकीप्रमुखा मुदा
ताः पुत्रमङ्कमारोप्य स्नेहस्नुतपयोधराः
हर्षविह्वलितात्मानः सिषिचुर्नेत्रजैर्जलैः
अथाविशत्स्वभवनं सर्वकाममनुत्तमम्‌
प्रासादा यत्र पत्नीनां सहस्राणि च षोडश
पत्न्यः पतिं प्रोष्य गृहानुपागतं विलोक्य सञ्जातमनोमहोत्सवाः
उत्तस्थुरारात्सहसासनाशयात्साकं व्रतैर्व्रीडितलोचनाननाः
तमात्मजैर्दृष्टिभिरन्तरात्मना दुरन्तभावाः परिरेभिरे पतिम्‌
निरुद्धमप्यास्रवदम्बु नेत्रयोर्विलज्जतीनां भृगुवर्य वैक्लवात्‌
यद्यप्यसौ पार्श्वगतो रहोगतस्तथापि तस्याङ्घ्रियुगं नवं नवम्‌
पदे पदे का विरमेत तत्पदाच्चलापि यच्छ्रीर्न जहाति कर्हिचित्‌
एवं नृपाणां क्षितिभारजन्मनामक्षौहिणीभिः परिवृत्ततेजसाम्‌
विधाय वैरं श्वसनो यथानलं मिथो वधेनोपरतो निरायुधः
स एष नरलोकेऽस्मिन्नवतीर्णः स्वमायया
रेमे स्त्रीरत्नकूटस्थो भगवान्प्राकृतो यथा
उद्दामभावपिशुनामलवल्गुहास
व्रीडावलोकनिहतो मदनोऽपि यासाम्‌
सम्मुह्य चापमजहात्प्रमदोत्तमास्ता
यस्येन्द्रियं विमथितुं कुहकैर्न शेकुः
तमयं मन्यते लोको ह्यसङ्गमपि सङ्गिनम्‌
आत्मौपम्येन मनुजं व्यापृण्वानं यतोऽबुधः
एतदीशनमीशस्य प्रकृतिस्थोऽपि तद्गुणैः
न युज्यते सदात्मस्थैर्यथा बुद्धिस्तदाश्रया
तं मेनिरेऽबला मूढाः स्त्रैणं चानुव्रतं रहः
अप्रमाणविदो भर्तुरीश्वरं मतयो यथा
शौनक उवाच
अश्वत्थाम्नोपसृष्टेन ब्रह्मशीर्ष्णोरुतेजसा
उत्तराया हतो गर्भ ईशेनाजीवितः पुनः
तस्य जन्म महाबुद्धेः कर्माणि च महात्मनः
निधनं च यथैवासीत्स प्रेत्य गतवान्यथा
तदिदं श्रोतुमिच्छामो गदितुं यदि मन्यसे
ब्रूहि नः श्रद्दधानानां यस्य ज्ञानमदाच्छुकः
सूत उवाच
अपीपलद्धर्मराजः पितृवद्रञ्जयन्प्रजाः
निःस्पृहः सर्वकामेभ्यः कृष्णपादानुसेवया
सम्पदः क्रतवो लोका महिषी भ्रातरो मही
जम्बूद्वीपाधिपत्यं च यशश्च त्रिदिवं गतम्‌
किं ते कामाः सुरस्पार्हा मुकुन्दमनसो द्विजाः
अधिजह्रुर्मुदं राज्ञः क्षुधितस्य यथेतरे
मातुर्गर्भगतो वीरः स तदा भृगुनन्दन
ददर्श पुरुषं कञ्चिद्दह्यमानोऽस्त्रतेजसा
अङ्गुष्ठमात्रममलं स्फुरत्पुरटमौलिनम्‌
अपीव्यदर्शनं श्यामं तडिद्वाससमच्युतम्‌
श्रीमद्दीर्घचतुर्बाहुं तप्तकाञ्चनकुण्डलम्‌
क्षतजाक्षं गदापाणिमात्मनः सर्वतो दिशम्‌
परिभ्रमन्तमुल्काभां भ्रामयन्तं गदां मुहुः
अस्त्रतेजः स्वगदया नीहारमिव गोपतिः
विधमन्तं सन्निकर्षे पर्यैक्षत क इत्यसौ
विधूय तदमेयात्मा भगवान्धर्मगुब्विभुः
मिषतो दशमासस्य तत्रैवान्तर्दधे हरिः
ततः सर्वगुणोदर्के सानुकूलग्रहोदये
जज्ञे वंशधरः पाण्डोर्भूयः पाण्डुरिवौजसा
तस्य प्रीतमना राजा विप्रैर्धौम्यकृपादिभिः
जातकं कारयामास वाचयित्वा च मङ्गलम्‌
हिरण्यं गां महीं ग्रामान्हस्त्यश्वान्नृपतिर्वरान्‌
प्रादात्स्वन्नं च विप्रेभ्यः प्रजातीर्थे स तीर्थवित्‌
तमूचुर्ब्राह्मणास्तुष्टा राजानं प्रश्रयान्वितम्‌
एष ह्यस्मिन्प्रजातन्तौ पुरूणां पौरवर्षभ
दैवेनाप्रतिघातेन शुक्ले संस्थामुपेयुषि
रातो वोऽनुग्रहार्थाय विष्णुना प्रभविष्णुना
तस्मान्नाम्ना विष्णुरात इति लोके भविष्यति
न सन्देहो महाभाग महाभागवतो महान्‌
श्रीराजोवाच
अप्येष वंश्यान्राजर्षीन्पुण्यश्लोकान्महात्मनः
अनुवर्तिता स्विद्यशसा साधुवादेन सत्तमाः
ब्राह्मणा ऊचुः
पार्थ प्रजाविता साक्षादिक्ष्वाकुरिव मानवः
ब्रह्मण्यः सत्यसन्धश्च रामो दाशरथिर्यथा
एष दाता शरण्यश्च यथा ह्यौशीनरः शिबिः
यशो वितनिता स्वानां दौष्यन्तिरिव यज्वनाम्‌
धन्विनामग्रणीरेष तुल्यश्चार्जुनयोर्द्वयोः
हुताश इव दुर्धर्षः समुद्र इव दुस्तरः
मृगेन्द्र इव विक्रान्तो निषेव्यो हिमवानिव
तितिक्षुर्वसुधेवासौ सहिष्णुः पितराविव
पितामहसमः साम्ये प्रसादे गिरिशोपमः
आश्रयः सर्वभूतानां यथा देवो रमाश्रयः
सर्वसद्गुणमाहात्म्ये एष कृष्णमनुव्रतः
रन्तिदेव इवोदारो ययातिरिव धार्मिकः
हृत्या बलिसमः कृष्णे प्रह्राद इव सद्ग्रहः
आहर्तैषोऽश्वमेधानां वृद्धानां पर्युपासकः
राजर्षीणां जनयिता शास्ता चोत्पथगामिनाम्‌
निग्रहीता कलेरेष भुवो धर्मस्य कारणात्‌
तक्षकादात्मनो मृत्युं द्विजपुत्रोपसर्जितात्‌
प्रपत्स्यत उपश्रुत्य मुक्तसङ्गः पदं हरेः
जिज्ञासितात्मयाथार्थ्यो मुनेर्व्याससुतादसौ
हित्वेदं नृप गङ्गायां यास्यत्यद्धाकुतोभयम्‌
इति राज्ञ उपादिश्य विप्रा जातककोविदाः
लब्धापचितयः सर्वे प्रतिजग्मुः स्वकान्गृहान्‌
स एष लोके विख्यातः परीक्षिदिति यत्प्रभुः
पूर्वं दृष्टमनुध्यायन्परीक्षेत नरेष्विह
स राजपुत्रो ववृधे आशु शुक्ल इवोडुपः
आपूर्यमाणः पितृभिः काष्ठाभिरिव सोऽन्वहम्‌
यक्ष्यमाणोऽश्वमेधेन ज्ञातिद्रोहजिहासया
राजा लब्धधनो दध्यौ नान्यत्र करदण्डयोः
तदभिप्रेतमालक्ष्य भ्रातरो ञ्च्युतचोदिताः
धनं प्रहीणमाजह्रुरुदीच्यां दिशि भूरिशः
तेन सम्भृतसम्भारो धर्मपुत्रो युधिष्ठिरः
वाजिमेधैस्त्रिभिर्भीतो यज्ञैः समयजद्धरिम्‌
आहूतो भगवान्राज्ञा याजयित्वा द्विजैर्नृपम्‌
उवास कतिचिन्मासान्सुहृदां प्रियकाम्यया
ततो राज्ञाभ्यनुज्ञातः कृष्णया सहबन्धुभिः
ययौ द्वारवतीं ब्रह्मन्सार्जुनो यदुभिर्वृतः
सूत उवाच
विदुरस्तीर्थयात्रायां मैत्रेयादात्मनो गतिम्‌
ज्ञात्वागाद्धास्तिनपुरं तयावाप्तविवित्सितः
यावतः कृतवान्प्रश्नान्क्षत्ता कौषारवाग्रतः
जातैकभक्तिर्गोविन्दे तेभ्यश्चोपरराम ह
तं बन्धुमागतं दृष्ट्वा धर्मपुत्रः सहानुजः
धृतराष्ट्रो युयुत्सुश्च सूतः शारद्वतः पृथा
गान्धारी द्रौपदी ब्रह्मन्सुभद्रा चोत्तरा कृपी
अन्याश्च जामयः पाण्डोर्ज्ञातयः ससुताः स्त्रियः
प्रत्युज्जग्मुः प्रहर्षेण प्राणं तन्व इवागतम्‌
अभिसङ्गम्य विधिवत्परिष्वङ्गाभिवादनैः
मुमुचुः प्रेमबाष्पौघं विरहौत्कण्ठ्यकातराः
राजा तमर्हयां चक्रे कृतासनपरिग्रहम्‌
तं भुक्तवन्तं विश्रान्तमासीनं सुखमासने
प्रश्रयावनतो राजा प्राह तेषां च शृण्वताम्‌
युधिष्ठिर उवाच
अपि स्मरथ नो युष्मत्पक्षच्छायासमेधितान्‌
विपद्गणाद्विषाग्न्यादेर्मोचिता यत्समातृकाः
कया वृत्त्या वर्तितं वश्चरद्भिः क्षितिमण्डलम्‌
तीर्थानि क्षेत्रमुख्यानि सेवितानीह भूतले
भवद्विधा भागवतास्तीर्थभूताः स्वयं विभो
तीर्थीकुर्वन्ति तीर्थानि स्वान्तःस्थेन गदाभृता
अपि नः सुहृदस्तात बान्धवाः कृष्णदेवताः
दृष्टाः श्रुता वा यदवः स्वपुर्यां सुखमासते
इत्युक्तो धर्मराजेन सर्वं तत्समवर्णयत्‌
यथानुभूतं क्रमशो विना यदुकुलक्षयम्‌
नन्वप्रियं दुर्विषहं नृणां स्वयमुपस्थितम्‌
नावेदयत्सकरुणो दुःखितान्द्रष्टुमक्षमः
कञ्चित्कालमथावात्सीत्सत्कृतो देववत्सुखम्‌
भ्रातुर्ज्येष्ठस्य श्रेयस्कृत्सर्वेषां सुखमावहन्‌
अबिभ्रदर्यमा दण्डं यथावदघकारिषु
यावद्दधार शूद्रत्वं शापाद्वर्षशतं यमः
युधिष्ठिरो लब्धराज्यो दृष्ट्वा पौत्रं कुलन्धरम्‌
भ्रातृभिर्लोकपालाभैर्मुमुदे परया श्रिया
एवं गृहेषु सक्तानां प्रमत्तानां तदीहया
अत्यक्रामदविज्ञातः कालः परमदुस्तरः
विदुरस्तदभिप्रेत्य धृतराष्ट्रमभाषत
राजन्निर्गम्यतां शीघ्रं पश्येदं भयमागतम्‌
प्रतिक्रिया न यस्येह कुतश्चित्कर्हिचित्प्रभो
स एष भगवान्कालः सर्वेषां नः समागतः
येन चैवाभिपन्नोऽयं प्राणैः प्रियतमैरपि
जनः सद्यो वियुज्येत किमुतान्यैर्धनादिभिः
पितृभ्रातृसुहृत्पुत्रा हतास्ते विगतं वयम्‌
आत्मा च जरया ग्रस्तः परगेहमुपाससे
अन्धः पुरैव वधिरो मन्दप्रज्ञाश्च साम्प्रतम्‌
विशीर्णदन्तो मन्दाग्निः सरागः कफमुद्वहन्‌
अहो महीयसी जन्तोर्जीविताशा यथा भवान्‌
भीमापवर्जितं पिण्डमादत्ते गृहपालवत्‌
अग्निर्निसृष्टो दत्तश्च गरो दाराश्च दूषिताः
हृतं क्षेत्रं धनं येषां तद्दत्तैरसुभिः कियत्‌
तस्यापि तव देहोऽयं कृपणस्य जिजीविषोः
परैत्यनिच्छतो जीर्णो जरया वाससी इव
गतस्वार्थमिमं देहं विरक्तो मुक्तबन्धनः
अविज्ञातगतिर्जह्यात्स वै धीर उदाहृतः
यः स्वकात्परतो वेह जातनिर्वेद आत्मवान्‌
हृदि कृत्वा हरिं गेहात्प्रव्रजेत्स नरोत्तमः
अथोदीचीं दिशं यातु स्वैरज्ञातगतिर्भवान्‌
इतोऽर्वाक्प्रायशः कालः पुंसां गुणविकर्षणः
एवं राजा विदुरेणानुजेन प्रज्ञाचक्षुर्बोधित आजमीढः
छित्त्वा स्वेषु स्नेहपाशान्द्रढिम्नो निश्चक्राम भ्रातृसन्दर्शिताध्वा
पतिं प्रयान्तं सुबलस्य पुत्री पतिव्रता चानुजगाम साध्वी
हिमालयं न्यस्तदण्डप्रहर्षं मनस्विनामिव सत्सम्प्रहारः
अजातशत्रुः कृतमैत्रो हुताग्निर्विप्रान्नत्वा तिलगोभूमिरुक्मैः
गृहं प्रविष्टो गुरुवन्दनाय न चापश्यत्पितरौ सौबलीं च
तत्र सञ्जयमासीनं पप्रच्छोद्विग्नमानसः
गावल्गणे क्व नस्तातो वृद्धो हीनश्च नेत्रयोः
अम्बा च हतपुत्रार्ता पितृव्यः क्व गतः सुहृत्‌
अपि मय्यकृतप्रज्ञे हतबन्धुः स भार्यया
आशंसमानः शमलं गङ्गायां दुःखितोऽपतत्‌
पितर्युपरते पाण्डौ सर्वान्नः सुहृदः शिशून्‌
अरक्षतां व्यसनतः पितृव्यौ क्व गतावितः
सूत उवाच
कृपया स्नेहवैक्लव्यात्सूतो विरहकर्शितः
आत्मेश्वरमचक्षाणो न प्रत्याहातिपीडितः
विमृज्याश्रूणि पाणिभ्यां विष्टभ्यात्मानमात्मना
अजातशत्रुं प्रत्यूचे प्रभोः पादावनुस्मरन्‌
सञ्जय उवाच
नाहं वेद व्यवसितं पित्रोर्वः कुलनन्दन
गान्धार्या वा महाबाहो मुषितोऽस्मि महात्मभिः
अथाजगाम भगवान्नारदः सहतुम्बुरुः
प्रत्युत्थायाभिवाद्याह सानुजोऽभ्यर्चयन्मुनिम्‌
युधिष्ठिर उवाच
नाहं वेद गतिं पित्रोर्भगवन्क्व गतावितः
अम्बा वा हतपुत्रार्ता क्व गता च तपस्विनी
कर्णधार इवापारे भगवान्पारदर्शकः
अथाबभाषे भगवान्नारदो मुनिसत्तमः
नारद उवाच
मा कञ्चन शुचो राजन्यदीश्वरवशं जगत्‌
लोकाः सपाला यस्येमे वहन्ति बलिमीशितुः
स संयुनक्ति भूतानि स एव वियुनक्ति च
यथा गावो नसि प्रोतास्तन्त्यां बद्धाश्च दामभिः
वाक्तन्त्यां नामभिर्बद्धा वहन्ति बलिमीशितुः
यथा क्रीडोपस्कराणां संयोगविगमाविह
इच्छया क्रीडितुः स्यातां तथैवेशेच्छया नृणाम्‌
यन्मन्यसे ध्रुवं लोकमध्रुवं वा न चोभयम्‌
सर्वथा न हि शोच्यास्ते स्नेहादन्यत्र मोहजात्‌
तस्माज्जह्यङ्ग वैक्लव्यमज्ञानकृतमात्मनः
कथं त्वनाथाः कृपणा वर्तेरंस्ते च मां विना
कालकर्मगुणाधीनो देहोऽयं पाञ्चभौतिकः
कथमन्यांस्तु गोपायेत्सर्पग्रस्तो यथा परम्‌
अहस्तानि सहस्तानामपदानि चतुष्पदाम्‌
फल्गूनि तत्र महतां जीवो जीवस्य जीवनम्‌
तदिदं भगवान्राजन्नेक आत्मात्मनां स्वदृक्‌
अन्तरोऽनन्तरो भाति पश्य तं माययोरुधा
सोऽयमद्य महाराज भगवान्भूतभावनः
कालरूपोऽवतीर्णोऽस्यामभावाय सुरद्विषाम्‌
निष्पादितं देवकृत्यमवशेषं प्रतीक्षते
तावद्यूयमवेक्षध्वं भवेद्यावदिहेश्वरः
धृतराष्ट्रः सह भ्रात्रा गान्धार्या च स्वभार्यया
दक्षिणेन हिमवत ऋषीणामाश्रमं गतः
स्रोतोभिः सप्तभिर्या वै स्वर्धुनी सप्तधा व्यधात्‌
सप्तानां प्रीतये नाना सप्तस्रोतः प्रचक्षते
स्नात्वानुसवनं तस्मिन्हुत्वा चाग्नीन्यथाविधि
अब्भक्ष उपशान्तात्मा स आस्ते विगतैषणः
जितासनो जितश्वासः प्रत्याहृतषडिन्द्रियः
हरिभावनया ध्वस्तरजःसत्त्वतमोमलः
विज्ञानात्मनि संयोज्य क्षेत्रज्ञे प्रविलाप्य तम्‌
ब्रह्मण्यात्मानमाधारे घटाम्बरमिवाम्बरे
ध्वस्तमायागुणोदर्को निरुद्धकरणाशयः
निवर्तिताखिलाहार आस्ते स्थाणुरिवाचलः
तस्यान्तरायो मैवाभूः सन्न्यस्ताखिलकर्मणः
स वा अद्यतनाद्राजन्परतः पञ्चमेऽहनि
कलेवरं हास्यति स्वं तच्च भस्मीभविष्यति
दह्यमानेऽग्निभिर्देहे पत्युः पत्नी सहोटजे
बहिः स्थिता पतिं साध्वी तमग्निमनु वेक्ष्यति
विदुरस्तु तदाश्चर्यं निशाम्य कुरुनन्दन
हर्षशोकयुतस्तस्माद्गन्ता तीर्थनिषेवकः
इत्युक्त्वाथारुहत्स्वर्गं नारदः सहतुम्बुरुः
युधिष्ठिरो वचस्तस्य हृदि कृत्वाजहाच्छुचः
सूत उवाच
सम्प्रस्थिते द्वारकायांजिष्णौ बन्धुदिदृक्षया
ज्ञातुं च पुण्यश्लोकस्य कृष्णस्य च विचेष्टितम्‌
व्यतीताः कतिचिन्मासास्तदा नायात्ततोऽर्जुनः
ददर्श घोररूपाणि निमित्तानि कुरूद्वहः
कालस्य च गतिं रौद्रां विपर्यस्तर्तुधर्मिणः
पापीयसीं नृणां वार्तां क्रोधलोभानृतात्मनाम्‌
जिह्मप्रायं व्यवहृतं शाठ्यमिश्रं च सौहृदम्‌
पितृमातृसुहृद्भ्रातृदम्पतीनां च कल्कनम्‌
निमित्तान्यत्यरिष्टानि काले त्वनुगते नृणाम्‌
लोभाद्यधर्मप्रकृतिं दृष्ट्वोवाचानुजं नृपः
युधिष्ठिर उवाच
सम्प्रेषितो द्वारकायां जिष्णुर्बन्धुदिदृक्षयाज्‌
ञातुं च पुण्यश्लोकस्य कृष्णस्य च विचेष्टितम्‌
गताः सप्ताधुना मासा भीमसेन तवानुजः
नायाति कस्य वा हेतोर्नाहं वेदेदमञ्जसा
अपि देवर्षिणादिष्टः स कालोऽयमुपस्थितः
यदात्मनोऽङ्गमाक्रीडं भगवानुत्सिसृक्षति
यस्मान्नः सम्पदो राज्यं दाराः प्राणाः कुलं प्रजाः
आसन्सपत्नविजयो लोकाश्च यदनुग्रहात्‌
पश्योत्पातान्नरव्याघ्र दिव्यान्भौमान्सदैहिकान्‌
दारुणान्शंसतोऽदूराद्भयं नो बुद्धिमोहनम्‌
ऊर्वक्षिबाहवो मह्यं स्फुरन्त्यङ्ग पुनः पुनः
वेपथुश्चापि हृदये आराद्दास्यन्ति विप्रियम्‌
शिवैषोद्यन्तमादित्यमभिरौत्यनलानना
मामङ्ग सारमेयोऽयमभिरेभत्यभीरुवत्‌
शस्ताः कुर्वन्ति मां सव्यं दक्षिणं पशवोऽपरे
वाहांश्च पुरुषव्याघ्र लक्षये रुदतो मम
मृत्युदूतः कपोतोऽयमुलूकः कम्पयन्मनः
प्रत्युलूकश्च कुह्वानैर्विश्वं वै शून्यमिच्छतः
धूम्रा दिशः परिधयः कम्पते भूः सहाद्रिभिः
निर्घातश्च महांस्तात साकं च स्तनयित्नुभिः
वायुर्वाति खरस्पर्शो रजसा विसृजंस्तमः
असृग्वर्षन्ति जलदा बीभत्समिव सर्वतः
सूर्यं हतप्रभं पश्य ग्रहमर्दं मिथो दिवि
ससङ्कुलैर्भूतगणैर्ज्वलिते इव रोदसी
नद्यो नदाश्च क्षुभिताः सरांसि च मनांसि च
न ज्वलत्यग्निराज्येन कालोऽयं किं विधास्यति
न पिबन्ति स्तनं वत्सा न दुह्यन्ति च मातरः
रुदन्त्यश्रुमुखा गावो न हृष्यन्त्यृषभा व्रजे
दैवतानि रुदन्तीव स्विद्यन्ति ह्युच्चलन्ति च
इमे जनपदा ग्रामाः पुरोद्यानाकराश्रमाः
भ्रष्टश्रियो निरानन्दाः किमघं दर्शयन्ति नः
मन्य एतैर्महोत्पातैर्नूनं भगवतः पदैः
अनन्यपुरुषश्रीभिर्हीना भूर्हतसौभगा
इति चिन्तयतस्तस्य दृष्टारिष्टेन चेतसा
राज्ञः प्रत्यागमद्ब्रह्मन्यदुपुर्याः कपिध्वजः
तं पादयोर्निपतितमयथापूर्वमातुरम्‌
अधोवदनमब्बिन्दून्सृजन्तं नयनाब्जयोः
विलोक्योद्विग्नहृदयो विच्छायमनुजं नृपः
पृच्छति स्म सुहृन्मध्ये संस्मरन्नारदेरितम्‌
युधिष्ठिर उवाच
कच्चिदानर्तपुर्यां नः स्वजनाः सुखमासते
मधुभोजदशार्हार्ह सात्वतान्धकवृष्णयः
शूरो मातामहः कच्चित्स्वस्त्यास्ते वाथ मारिषः
मातुलः सानुजः कच्चित्कुशल्यानकदुन्दुभिः
सप्त स्वसारस्तत्पत्न्यो मातुलान्यः सहात्मजाः
आसते सस्नुषाः क्षेमंदेवकीप्रमुखाः स्वयम्‌
कच्चिद्राजाहुको जीवत्यसत्पुत्रोऽस्य चानुजः
हृदीकः ससुतोऽक्रूरो जयन्तगदसारणाः
आसते कुशलं कच्चिद्ये च शत्रुजिदादयः
कच्चिदास्ते सुखं रामो भगवान्सात्वतां प्रभुः
प्रद्युम्नः सर्ववृष्णीनां सुखमास्ते महारथः
गम्भीररयोऽनिरुद्धो वर्धते भगवानुत
सुषेणश्चारुदेष्णश्च साम्बो जाम्बवतीसुतः
अन्ये च कार्ष्णिप्रवराः सपुत्रा ऋषभादयः
तथैवानुचराः शौरेः श्रुतदेवोद्धवादयः
सुनन्दनन्दशीर्षण्या ये चान्ये सात्वतर्षभाः
अपि स्वस्त्यासते सर्वे रामकृष्णभुजाश्रयाः
अपि स्मरन्ति कुशलमस्माकं बद्धसौहृदाः
भगवानपि गोविन्दो ब्रह्मण्यो भक्तवत्सलः
कच्चित्पुरे सुधर्मायां सुखमास्ते सुहृद्वृतः
मङ्गलाय च लोकानां क्षेमाय च भवाय च
आस्ते यदुकुलाम्भोधावाद्योऽनन्तसखः पुमान्‌
यद्बाहुदण्डगुप्तायां स्वपुर्यां यदवोऽर्चिताः
क्रीडन्ति परमानन्दं महापौरुषिका इव
यत्पादशुश्रूषणमुख्यकर्मणा सत्यादयो द्व्यष्टसहस्रयोषितः
निर्जित्य सङ्ख्ये त्रिदशांस्तदाशिषो हरन्ति वज्रायुधवल्लभोचिताः
यद्बाहुदण्डाभ्युदयानुजीविनो यदुप्रवीरा ह्यकुतोभया मुहुः
अधिक्रमन्त्यङ्घ्रिभिराहृतां बलात्सभां सुधर्मां सुरसत्तमोचिताम्‌
कच्चित्तेऽनामयं तात भ्रष्टतेजा विभासि मे
अलब्धमानोऽवज्ञातः किं वा तात चिरोषितः
कच्चिन्नाभिहतोऽभावैः शब्दादिभिरमङ्गलैः
न दत्तमुक्तमर्थिभ्य आशया यत्प्रतिश्रुतम्‌
कच्चित्त्वं ब्राह्मणं बालं गां वृद्धं रोगिणं स्त्रियम्‌
शरणोपसृतं सत्त्वं नात्याक्षीः शरणप्रदः
कच्चित्त्वं नागमोऽगम्यां गम्यां वासत्कृतां स्त्रियम्‌
पराजितो वाथ भवान्नोत्तमैर्नासमैः पथि
अपि स्वित्पर्यभुङ्क्थास्त्वं सम्भोज्यान्वृद्धबालकान्‌
जुगुप्सितं कर्म किञ्चित्कृतवान्न यदक्षमम्‌
कच्चित्प्रेष्ठतमेनाथ हृदयेनात्मबन्धुना
शून्योऽस्मि रहितो नित्यं मन्यसे तेऽन्यथा न रुक्‌
सूत उवाच
एवं कृष्णसखः कृष्णो भ्रात्रा राज्ञा विकल्पितः
नानाशङ्कास्पदं रूपं कृष्णविश्लेषकर्शितः
शोकेन शुष्यद्वदन हृत्सरोजो हतप्रभः
विभुं तमेवानुस्मरन्नाशक्नोत्प्रतिभाषितुम्‌
कृच्छ्रेण संस्तभ्य शुचः पाणिनामृज्य नेत्रयोः
परोक्षेण समुन्नद्ध प्रणयौत्कण्ठ्यकातरः
सख्यं मैत्रीं सौहृदं च सारथ्यादिषु संस्मरन्‌
नृपमग्रजमित्याह बाष्पगद्गदया गिरा
अर्जुन उवाच
वञ्चितोऽहं महाराज हरिणा बन्धुरूपिणा
येन मेऽपहृतं तेजो देवविस्मापनं महत्‌
यस्य क्षणवियोगेन लोको ह्यप्रियदर्शनः
उक्थेन रहितो ह्येष मृतकः प्रोच्यते यथा
यत्संश्रयाद्द्रुपदगेहमुपागतानां राज्ञां स्वयंवरमुखे स्मरदुर्मदानाम्‌
तेजो हृतं खलु मयाभिहतश्च मत्स्यः सज्जीकृतेन धनुषाधिगता च कृष्णा
यत्सन्निधावहमु खाण्डवमग्नयेऽदामिन्द्रं च सामरगणं तरसा विजित्य
लब्धा सभा मयकृताद्भुतशिल्पमाया दिग्भ्योऽहरन्नृपतयो बलिमध्वरे ते
यत्तेजसा नृपशिरोऽङ्घ्रिमहन्मखार्थमार्योऽनुजस्तव गजायुतसत्त्ववीर्यः
तेनाहृताः प्रमथनाथमखाय भूपा यन्मोचितास्तदनयन्बलिमध्वरे ते
पत्न्यास्तवाधिमखकॢप्तमहाभिषेक श्लाघिष्ठचारुकबरं कितवैः सभायाम्‌
स्पृष्टं विकीर्य पदयोः पतिताश्रुमुख्या यस्तत्स्त्रियोऽकृतहतेशविमुक्तकेशाः
यो नो जुगोप वन एत्य दुरन्तकृच्छ्राद्दुर्वाससोऽरिरचितादयुताग्रभुग्यः
शाकान्नशिष्टमुपयुज्य यतस्त्रिलोकीं तृप्ताममंस्त सलिले विनिमग्नसङ्घः
यत्तेजसाथ भगवान्युधि शूलपाणिर्विस्मापितः सगिरिजोऽस्त्रमदान्निजं मे
अन्येऽपि चाहममुनैव कलेवरेण प्राप्तो महेन्द्रभवने महदासनार्धम्‌
तत्रैव मे विहरतो भुजदण्डयुग्मं गाण्डीवलक्षणमरातिवधाय देवाः
सेन्द्राः श्रिता यदनुभावितमाजमीढ तेनाहमद्य मुषितः पुरुषेण भूम्ना
यद्बान्धवः कुरुबलाब्धिमनन्तपारमेको रथेन ततरेऽहमतीर्यसत्त्वम्‌
प्रत्याहृतं बहु धनं च मया परेषां तेजास्पदं मणिमयं च हृतं शिरोभ्यः
यो भीष्मकर्णगुरुशल्यचमूष्वदभ्र राजन्यवर्यरथमण्डलमण्डितासु
अग्रेचरो मम विभो रथयूथपानामायुर्मनांसि च दृशा सह ओज आर्च्छत्‌
यद्दोःषु मा प्रणिहितं गुरुभीष्मकर्ण नप्तृत्रिगर्तशल्यसैन्धवबाह्लिकाद्यैः
अस्त्राण्यमोघमहिमानि निरूपितानि नोपस्पृशुर्नृहरिदासमिवासुराणि
सौत्ये वृतः कुमतिनात्मद ईश्वरो मे यत्पादपद्ममभवाय भजन्ति भव्याः
मां श्रान्तवाहमरयो रथिनो भुविष्ठं न प्राहरन्यदनुभावनिरस्तचित्ताः
नर्माण्युदाररुचिरस्मितशोभितानि हे पार्थ हेऽर्जुन सखे कुरुनन्दनेति
सञ्जल्पितानि नरदेव हृदिस्पृशानि स्मर्तुर्लुठन्ति हृदयं मम माधवस्य
शय्यासनाटनविकत्थनभोजनादिष्वैक्याद्वयस्य ऋतवानिति विप्रलब्धः
सख्युः सखेव पितृवत्तनयस्य सर्वं सेहे महान्महितया कुमतेरघं मे
सोऽहं नृपेन्द्र रहितः पुरुषोत्तमेन सख्या प्रियेण सुहृदा हृदयेन शून्यः
अध्वन्युरुक्रमपरिग्रहमङ्ग रक्षन्गोपैरसद्भिरबलेव विनिर्जितोऽस्मि
तद्वै धनुस्त इषवः स रथो हयास्ते सोऽहं रथी नृपतयो यत आनमन्ति
सर्वं क्षणेन तदभूदसदीशरिक्तं भस्मन्हुतं कुहकराद्धमिवोप्तमूष्याम्‌
राजंस्त्वयानुपृष्टानां सुहृदां नः सुहृत्पुरे
विप्रशापविमूढानां निघ्नतां मुष्टिभिर्मिथः
वारुणीं मदिरां पीत्वा मदोन्मथितचेतसाम्‌
अजानतामिवान्योन्यं चतुःपञ्चावशेषिताः
प्रायेणैतद्भगवत ईश्वरस्य विचेष्टितम्‌
मिथो निघ्नन्ति भूतानि भावयन्ति च यन्मिथः
जलौकसां जले यद्वन्महान्तोऽदन्त्यणीयसः
दुर्बलान्बलिनो राजन्महान्तो बलिनो मिथः
एवं बलिष्ठैर्यदुभिर्महद्भिरितरान्विभुः
यदून्यदुभिरन्योन्यं भूभारान्सञ्जहार ह
देशकालार्थयुक्तानि हृत्तापोपशमानि च
हरन्ति स्मरतश्चित्तं गोविन्दाभिहितानि मे
सूत उवाच
एवं चिन्तयतो जिष्णोः कृष्णपादसरोरुहम्‌
सौहार्देनातिगाढेन शान्तासीद्विमला मतिः
वासुदेवाङ्घ्र्यनुध्यान परिबृंहितरंहसा
भक्त्या निर्मथिताशेष कषायधिषणोऽर्जुनः
गीतं भगवता ज्ञानं यत्तत्सङ्ग्राममूर्धनि
कालकर्मतमोरुद्धं पुनरध्यगमत्प्रभुः
विशोको ब्रह्मसम्पत्त्या सञ्छिन्नद्वैतसंशयः
लीनप्रकृतिनैर्गुण्यादलिङ्गत्वादसम्भवः
निशम्य भगवन्मार्गं संस्थां यदुकुलस्य च
स्वःपथाय मतिं चक्रे निभृतात्मा युधिष्ठिरः
पृथाप्यनुश्रुत्य धनञ्जयोदितं नाशं यदूनां भगवद्गतिं च ताम्‌
एकान्तभक्त्या भगवत्यधोक्षजे निवेशितात्मोपरराम संसृतेः
ययाहरद्भुवो भारं तां तनुं विजहावजः
कण्टकं कण्टकेनेव द्वयं चापीशितुः समम्‌
यथा मत्स्यादिरूपाणि धत्ते जह्याद्यथा नटः
भूभारः क्षपितो येनजहौ तच्च कलेवरम्‌
यदा मुकुन्दो भगवानिमां महीं जहौ स्वतन्वा श्रवणीयसत्कथः
तदाहरेवाप्रतिबुद्धचेतसामभद्रहेतुः कलिरन्ववर्तत
युधिष्ठिरस्तत्परिसर्पणं बुधः पुरे च राष्ट्रे च गृहे तथात्मनि
विभाव्य लोभानृतजिह्महिंसनाद्यधर्मचक्रं गमनाय पर्यधात्‌
स्वराट्पौत्रं विनयिनमात्मनः सुसमं गुणैः
तोयनीव्याः पतिं भूमेरभ्यषिञ्चद्गजाह्वये
मथुरायां तथा वज्रं शूरसेनपतिं ततः
प्राजापत्यां निरूप्येष्टिमग्नीनपिबदीश्वरः
विसृज्य तत्र तत्सर्वं दुकूलवलयादिकम्‌
निर्ममो निरहङ्कारः सञ्छिन्नाशेषबन्धनः
वाचं जुहाव मनसि तत्प्राण इतरे च तम्‌
मृत्यावपानं सोत्सर्गं तं पञ्चत्वे ह्यजोहवीत्‌
त्रित्वे हुत्वा च पञ्चत्वं तच्चैकत्वे ञ्जुहोन्मुनिः
सर्वमात्मन्यजुहवीद्ब्रह्मण्यात्मानमव्यये
चीरवासा निराहारो बद्धवाङ्मुक्तमूर्धजः
दर्शयन्नात्मनो रूपं जडोन्मत्तपिशाचवत्‌
अनवेक्षमाणो निरगादशृण्वन्बधिरो यथा
उदीचीं प्रविवेशाशां गतपूर्वां महात्मभिः
हृदि ब्रह्म परं ध्यायन्नावर्तेत यतो गतः
सर्वे तमनुनिर्जग्मुर्भ्रातरः कृतनिश्चयाः
कलिनाधर्ममित्रेण दृष्ट्वा स्पृष्टाः प्रजा भुवि
ते साधुकृतसर्वार्था ज्ञात्वात्यन्तिकमात्मनः
मनसा धारयामासुर्वैकुण्ठचरणाम्बुजम्‌
तद्ध्यानोद्रिक्तया भक्त्या विशुद्धधिषणाः परे
तस्मिन्नारायणपदे एकान्तमतयो गतिम्‌
अवापुर्दुरवापां ते असद्भिर्विषयात्मभिः
विधूतकल्मषा स्थानं विरजेनात्मनैव हि
विदुरोऽपि परित्यज्य प्रभासे देहमात्मनः
कृष्णावेशेन तच्चित्तः पितृभिः स्वक्षयं ययौ
द्रौपदी च तदाज्ञाय पतीनामनपेक्षताम्‌
वासुदेवे भगवति ह्येकान्तमतिराप तम्‌
यः श्रद्धयैतद्भगवत्प्रियाणां पाण्डोः सुतानामिति सम्प्रयाणम्‌
शृणोत्यलं स्वस्त्ययनं पवित्रं लब्ध्वा हरौ भक्तिमुपैति सिद्धिम्‌
सूत उवाच
ततः परीक्षिद्द्विजवर्यशिक्षया महीं महाभागवतः शशास ह
यथा हि सूत्यामभिजातकोविदाः समादिशन्विप्र महद्गुणस्तथा
स उत्तरस्य तनयामुपयेम इरावतीम्‌
जनमेजयादींश्चतुरस्तस्यामुत्पादयत्सुतान्‌
आजहाराश्वमेधांस्त्रीन्गङ्गायां भूरिदक्षिणान्‌
शारद्वतं गुरुं कृत्वा देवा यत्राक्षिगोचराः
निजग्राहौजसा वीरः कलिं दिग्विजये क्वचित्‌
नृपलिङ्गधरं शूद्रं घ्नन्तं गोमिथुनं पदा
शौनक उवाच
कस्य हेतोर्निजग्राह कलिं दिग्विजये नृपः
नृदेवचिह्नधृक्षूद्र कोऽसौ गां यः पदाहनत्‌
तत्कथ्यतां महाभाग यदि कृष्णकथाश्रयम्‌
अथवास्य पदाम्भोज मकरन्दलिहां सताम्‌
किमन्यैरसदालापैरायुषो यदसद्व्ययः
क्षुद्रायुषां नृणामङ्ग मर्त्यानामृतमिच्छताम्‌
इहोपहूतो भगवान्मृत्युः शामित्रकर्मणि
न कश्चिन्म्रियते तावद्यावदास्त इहान्तकः
एतदर्थं हि भगवानाहूतः परमर्षिभिः
अहो नृलोके पीयेत हरिलीलामृतं वचः
मन्दस्य मन्दप्रज्ञस्य वयो मन्दायुषश्च वै
निद्रया ह्रियते नक्तं दिवा च व्यर्थकर्मभिः
सूत उवाच
यदा परीक्षित्कुरुजाङ्गलेऽवसत्कलिं प्रविष्टं निजचक्रवर्तिते
निशम्य वार्तामनतिप्रियां ततः शरासनं संयुगशौण्डिराददे
स्वलङ्कृतं श्यामतुरङ्गयोजितं रथं मृगेन्द्रध्वजमाश्रितः पुरात्‌
वृतो रथाश्वद्विपपत्तियुक्तया स्वसेनया दिग्विजयाय निर्गतः
भद्राश्वं केतुमालं च भारतं चोत्तरान्कुरून्‌
किम्पुरुषादीनि वर्षाणि विजित्य जगृहे बलिम्‌
नगरांश्च वनांश्चैव नदीश्च विमलोदकाः
पुरुषान्देवकल्पांश्च नारीश्च प्रियदर्शनाः
अदृष्टपूर्वान्सुभगान्स ददर्श धनञ्जयः
सदनानि च शुभ्राणि नारीश्चाप्सरसां निभाः
तत्र तत्रोपशृण्वानः स्वपूर्वेषां महात्मनाम्‌
प्रगीयमाणं च यशः कृष्णमाहात्म्यसूचकम्‌
आत्मानं च परित्रातमश्वत्थाम्नोऽस्त्रतेजसः
स्नेहं च वृष्णिपार्थानां तेषां भक्तिं च केशवे
तेभ्यः परमसन्तुष्टः प्रीत्युज्जृम्भितलोचनः
महाधनानि वासांसि ददौ हारान्महामनाः
सारथ्यपारषदसेवनसख्यदौत्य
वीरासनानुगमनस्तवनप्रणामान्‌
स्निग्धेषु पाण्डुषु जगत्प्रणतिं च विष्णोर्‌
भक्तिं करोति नृपतिश्चरणारविन्दे
तस्यैवं वर्तमानस्य पूर्वेषां वृत्तिमन्वहम्‌
नातिदूरे किलाश्चर्यं यदासीत्तन्निबोध मे
धर्मः पदैकेन चरन्विच्छायामुपलभ्य गाम्‌
पृच्छति स्माश्रुवदनां विवत्सामिव मातरम्‌
धर्म उवाच
कच्चिद्भद्रेऽनामयमात्मनस्ते विच्छायासि म्लायतेषन्मुखेन
आलक्षये भवतीमन्तराधिं दूरे बन्धुं शोचसि कञ्चनाम्ब
पादैर्न्यूनं शोचसि मैकपादमात्मानं वा वृषलैर्भोक्ष्यमाणम्‌
आहो सुरादीन्हृतयज्ञभागान्प्रजा उत स्विन्मघवत्यवर्षति
अरक्ष्यमाणाः स्त्रिय उर्वि बालान्शोचस्यथो पुरुषादैरिवार्तान्‌
वाचं देवीं ब्रह्मकुले कुकर्मण्यब्रह्मण्ये राजकुले कुलाग्र्यान्‌
किं क्षत्रबन्धून्कलिनोपसृष्टान्राष्ट्राणि वा तैरवरोपितानि
इतस्ततो वाशनपानवासः स्नानव्यवायोन्मुखजीवलोकम्‌
यद्वाम्ब ते भूरिभरावतार कृतावतारस्य हरेर्धरित्रि
अन्तर्हितस्य स्मरती विसृष्टा कर्माणि निर्वाणविलम्बितानि
इदं ममाचक्ष्व तवाधिमूलं वसुन्धरे येन विकर्शितासि
कालेन वा ते बलिनां बलीयसा सुरार्चितं किं हृतमम्ब सौभगम्‌
धरण्युवाच
भवान्हि वेद तत्सर्वं यन्मां धर्मानुपृच्छसि
चतुर्भिर्वर्तसे येन पादैर्लोकसुखावहैः
सत्यं शौचं दया क्षान्तिस्त्यागः सन्तोष आर्जवम्‌
शमो दमस्तपः साम्यं तितिक्षोपरतिः श्रुतम्‌
ज्ञानं विरक्तिरैश्वर्यं शौर्यं तेजो बलं स्मृतिः
स्वातन्त्र्यं कौशलं कान्तिर्धैर्यं मार्दवमेव च
प्रागल्भ्यं प्रश्रयः शीलं सह ओजो बलं भगः
गाम्भीर्यं स्थैर्यमास्तिक्यं कीर्तिर्मानोऽनहङ्कृतिः
एते चान्ये च भगवन्नित्या यत्र महागुणाः
प्रार्थ्या महत्त्वमिच्छद्भिर्न वियन्ति स्म कर्हिचित्‌
तेनाहं गुणपात्रेण श्रीनिवासेन साम्प्रतम्‌
शोचामि रहितं लोकं पाप्मना कलिनेक्षितम्‌
आत्मानं चानुशोचामि भवन्तं चामरोत्तमम्‌
देवान्पित्ऱ्‌^ईनृषीन्साधून्सर्वान्वर्णांस्तथाश्रमान्‌
ब्रह्मादयो बहुतिथं यदपाङ्गमोक्ष
कामास्तपः समचरन्भगवत्प्रपन्नाः
सा श्रीः स्ववासमरविन्दवनं विहाय
यत्पादसौभगमलं भजतेऽनुरक्ता
तस्याहमब्जकुलिशाङ्कुशकेतुकेतैः
श्रीमत्पदैर्भगवतः समलङ्कृताङ्गी
त्रीनत्यरोच उपलभ्य ततो विभूतिं
लोकान्स मां व्यसृजदुत्स्मयतीं तदन्ते
यो वै ममातिभरमासुरवंशराज्ञाम्‌
अक्षौहिणीशतमपानुददात्मतन्त्रः
त्वां दुःस्थमूनपदमात्मनि पौरुषेण
सम्पादयन्यदुषु रम्यमबिभ्रदङ्गम्‌
का वा सहेत विरहं पुरुषोत्तमस्य
प्रेमावलोकरुचिरस्मितवल्गुजल्पैः
स्थैर्यं समानमहरन्मधुमानिनीनां
रोमोत्सवो मम यदङ्घ्रिविटङ्कितायाः
तयोरेवं कथयतोः पृथिवीधर्मयोस्तदा
परीक्षिन्नाम राजर्षिः प्राप्तः प्राचीं सरस्वतीम्‌
सूत उवाच
तत्र गोमिथुनं राजा हन्यमानमनाथवत्‌
दण्डहस्तं च वृषलं ददृशे नृपलाञ्छनम्‌
वृषं मृणालधवलं मेहन्तमिव बिभ्यतम्‌
वेपमानं पदैकेन सीदन्तं शूद्रताडितम्‌
गां च धर्मदुघां दीनां भृशं शूद्रपदाहताम्‌
विवत्सामाश्रुवदनां क्षामां यवसमिच्छतीम्‌
पप्रच्छ रथमारूढः कार्तस्वरपरिच्छदम्‌
मेघगम्भीरया वाचा समारोपितकार्मुकः
कस्त्वं मच्छरणे लोके बलाद्धंस्यबलान्बली
नरदेवोऽसि वेषेण नटवत्कर्मणाद्विजः
यस्त्वं कृष्णे गते दूरं सहगाण्डीवधन्वना
शोच्योऽस्यशोच्यान्रहसि प्रहरन्वधमर्हसि
त्वं वा मृणालधवलः पादैर्न्यूनः पदा चरन्‌
वृषरूपेण किं कश्चिद्देवो नः परिखेदयन्‌
न जातु कौरवेन्द्राणां दोर्दण्डपरिरम्भिते
भूतलेऽनुपतन्त्यस्मिन्विना ते प्राणिनां शुचः
मा सौरभेयात्र शुचो व्येतु ते वृषलाद्भयम्‌
मा रोदीरम्ब भद्रं ते खलानां मयि शास्तरि
यस्य राष्ट्रे प्रजाः सर्वास्त्रस्यन्ते साध्व्यसाधुभिः
तस्य मत्तस्य नश्यन्ति कीर्तिरायुर्भगो गतिः
एष राज्ञां परो धर्मो ह्यार्तानामार्तिनिग्रहः
अत एनं वधिष्यामि भूतद्रुहमसत्तमम्‌
कोऽवृश्चत्तव पादांस्त्रीन्सौरभेय चतुष्पद
मा भूवंस्त्वादृशा राष्ट्रे राज्ञां कृष्णानुवर्तिनाम्‌
आख्याहि वृष भद्रं वः साधूनामकृतागसाम्‌
आत्मवैरूप्यकर्तारं पार्थानां कीर्तिदूषणम्‌
जनेऽनागस्यघं युञ्जन्सर्वतोऽस्य च मद्भयम्‌
साधूनां भद्रमेव स्यादसाधुदमने कृते
अनागःस्विह भूतेषु य आगस्कृन्निरङ्कुशः
आहर्तास्मि भुजं साक्षादमर्त्यस्यापि साङ्गदम्‌
राज्ञो हि परमो धर्मः स्वधर्मस्थानुपालनम्‌
शासतोऽन्यान्यथाशास्त्रमनापद्युत्पथानिह
धर्म उवाच
एतद्वः पाण्डवेयानां युक्तमार्ताभयं वचः
येषां गुणगणैः कृष्णो दौत्यादौ भगवान्कृतः
न वयं क्लेशबीजानि यतः स्युः पुरुषर्षभ
पुरुषं तं विजानीमो वाक्यभेदविमोहिताः
केचिद्विकल्पवसना आहुरात्मानमात्मनः
दैवमन्येऽपरे कर्म स्वभावमपरे प्रभुम्‌
अप्रतर्क्यादनिर्देश्यादिति केष्वपि निश्चयः
अत्रानुरूपं राजर्षे विमृश स्वमनीषया
सूत उवाच
एवं धर्मे प्रवदति स सम्राड्द्विजसत्तमाः
समाहितेन मनसा विखेदः पर्यचष्ट तम्‌
राजोवाच
धर्मं ब्रवीषि धर्मज्ञ धर्मोऽसि वृषरूपधृक्‌
यदधर्मकृतः स्थानं सूचकस्यापि तद्भवेत्‌
अथवा देवमायाया नूनं गतिरगोचरा
चेतसो वचसश्चापि भूतानामिति निश्चयः
तपः शौचं दया सत्यमिति पादाः कृते कृताः
अधर्मांशैस्त्रयो भग्नाः स्मयसङ्गमदैस्तव
इदानीं धर्म पादस्ते सत्यं निर्वर्तयेद्यतः
तं जिघृक्षत्यधर्मोऽयमनृतेनैधितः कलिः
इयं च भूमिर्भगवता न्यासितोरुभरा सती
श्रीमद्भिस्तत्पदन्यासैः सर्वतः कृतकौतुका
शोचत्यश्रुकला साध्वी दुर्भगेवोज्झिता सती
अब्रह्मण्या नृपव्याजाः शूद्रा भोक्ष्यन्ति मामिति
इति धर्मं महीं चैव सान्त्वयित्वा महारथः
निशातमाददे खड्गं कलयेऽधर्महेतवे
तं जिघांसुमभिप्रेत्य विहाय नृपलाञ्छनम्‌
तत्पादमूलं शिरसा समगाद्भयविह्वलः
पतितं पादयोर्वीरः कृपया दीनवत्सलः
शरण्यो नावधीच्छ्लोक्य आह चेदं हसन्निव
राजोवाच
न ते गुडाकेशयशोधराणां बद्धाञ्जलेर्वै भयमस्ति किञ्चित्‌
न वर्तितव्यं भवता कथञ्चन क्षेत्रे मदीये त्वमधर्मबन्धुः
त्वां वर्तमानं नरदेवदेहेष्वनुप्रवृत्तोऽयमधर्मपूगः
लोभोऽनृतं चौर्यमनार्यमंहो ज्येष्ठा च माया कलहश्च दम्भः
न वर्तितव्यं तदधर्मबन्धो धर्मेण सत्येन च वर्तितव्ये
ब्रह्मावर्ते यत्र यजन्ति यज्ञैर्यज्ञेश्वरं यज्ञवितानविज्ञाः
यस्मिन्हरिर्भगवानिज्यमान इज्यात्ममूर्तिर्यजतां शं तनोति
कामानमोघान्स्थिरजङ्गमानामन्तर्बहिर्वायुरिवैष आत्मा
सूत उवाच
परीक्षितैवमादिष्टः स कलिर्जातवेपथुः
तमुद्यतासिमाहेदं दण्डपाणिमिवोद्यतम्‌
कलिरुवाच
यत्र क्व वाथ वत्स्यामि सार्वभौम तवाज्ञया
लक्षये तत्र तत्रापि त्वामात्तेषुशरासनम्‌
तन्मे धर्मभृतां श्रेष्ठ स्थानं निर्देष्टुमर्हसि
यत्रैव नियतो वत्स्य आतिष्ठंस्तेऽनुशासनम्‌
सूत उवाच
अभ्यर्थितस्तदा तस्मै स्थानानि कलये ददौ
द्यूतं पानं स्त्रियः सूना यत्राधर्मश्चतुर्विधः
पुनश्च याचमानाय जातरूपमदात्प्रभुः
ततोऽनृतं मदं कामं रजो वैरं च पञ्चमम्‌
अमूनि पञ्च स्थानानि ह्यधर्मप्रभवः कलिः
औत्तरेयेण दत्तानि न्यवसत्तन्निदेशकृत्‌
अथैतानि न सेवेत बुभूषुः पुरुषः क्वचित्‌
विशेषतो धर्मशीलो राजा लोकपतिर्गुरुः
वृषस्य नष्टांस्त्रीन्पादान्तपः शौचं दयामिति
प्रतिसन्दध आश्वास्य महीं च समवर्धयत्‌
स एष एतर्ह्यध्यास्त आसनं पार्थिवोचितम्‌
पितामहेनोपन्यस्तं राज्ञारण्यं विविक्षता
आस्तेऽधुना स राजर्षिः कौरवेन्द्रश्रियोल्लसन्‌
गजाह्वये महाभागश्चक्रवर्ती बृहच्छ्रवाः
इत्थम्भूतानुभावोऽयमभिमन्युसुतो नृपः
यस्य पालयतः क्षौणीं यूयं सत्राय दीक्षिताः
सूत उवाच
यो वै द्रौण्यस्त्रविप्लुष्टो न मातुरुदरे मृतः
अनुग्रहाद्भगवतः कृष्णस्याद्भुतकर्मणः
ब्रह्मकोपोत्थिताद्यस्तु तक्षकात्प्राणविप्लवात्‌
न सम्मुमोहोरुभयाद्भगवत्यर्पिताशयः
उत्सृज्य सर्वतः सङ्गं विज्ञाताजितसंस्थितिः
वैयासकेर्जहौ शिष्यो गङ्गायां स्वं कलेवरम्‌
नोत्तमश्लोकवार्तानां जुषतां तत्कथामृतम्‌
स्यात्सम्भ्रमोऽन्तकालेऽपि स्मरतां तत्पदाम्बुजम्‌
तावत्कलिर्न प्रभवेत्प्रविष्टोऽपीह सर्वतः
यावदीशो महानुर्व्यामाभिमन्यव एकराट्‌
यस्मिन्नहनि यर्ह्येव भगवानुत्ससर्ज गाम्‌
तदैवेहानुवृत्तोऽसावधर्मप्रभवः कलिः
नानुद्वेष्टि कलिं सम्राट्सारङ्ग इव सारभुक्‌
कुशलान्याशु सिद्ध्यन्ति नेतराणि कृतानि यत्‌
किं नु बालेषु शूरेण कलिना धीरभीरुणा
अप्रमत्तः प्रमत्तेषु यो वृको नृषु वर्तते
उपवर्णितमेतद्वः पुण्यं पारीक्षितं मया
वासुदेवकथोपेतमाख्यानं यदपृच्छत
या याः कथा भगवतः कथनीयोरुकर्मणः
गुणकर्माश्रयाः पुम्भिः संसेव्यास्ता बुभूषुभिः
ऋषय ऊचुः
सूत जीव समाः सौम्य शाश्वतीर्विशदं यशः
यस्त्वं शंससि कृष्णस्य मर्त्यानाममृतं हि नः
कर्मण्यस्मिन्ननाश्वासे धूमधूम्रात्मनां भवान्‌
आपाययति गोविन्द पादपद्मासवं मधु
तुलयाम लवेनापि न स्वर्गं नापुनर्भवम्‌
भगवत्सङ्गिसङ्गस्य मर्त्यानां किमुताशिषः
को नाम तृप्येद्रसवित्कथायां महत्तमैकान्तपरायणस्य
नान्तं गुणानामगुणस्य जग्मुर्योगेश्वरा ये भवपाद्ममुख्याः
तन्नो भवान्वै भगवत्प्रधानो महत्तमैकान्तपरायणस्य
हरेरुदारं चरितं विशुद्धं शुश्रूषतां नो वितनोतु विद्वन्‌
स वै महाभागवतः परीक्षिद्येनापवर्गाख्यमदभ्रबुद्धिः
ज्ञानेन वैयासकिशब्दितेन भेजे खगेन्द्रध्वजपादमूलम्‌
तन्नः परं पुण्यमसंवृतार्थमाख्यानमत्यद्भुतयोगनिष्ठम्‌
आख्याह्यनन्ताचरितोपपन्नं पारीक्षितं भागवताभिरामम्‌
सूत उवाच
अहो वयं जन्मभृतोऽद्य हास्म वृद्धानुवृत्त्यापि विलोमजाताः
दौष्कुल्यमाधिं विधुनोति शीघ्रं महत्तमानामभिधानयोगः
कुतः पुनर्गृणतो नाम तस्य महत्तमैकान्तपरायणस्य
योऽनन्तशक्तिर्भगवाननन्तो महद्गुणत्वाद्यमनन्तमाहुः
एतावतालं ननु सूचितेन गुणैरसाम्यानतिशायनस्य
हित्वेतरान्प्रार्थयतो विभूतिर्यस्याङ्घ्रिरेणुं जुषतेऽनभीप्सोः
अथापि यत्पादनखावसृष्टं जगद्विरिञ्चोपहृतार्हणाम्भः
सेशं पुनात्यन्यतमो मुकुन्दात्को नाम लोके भगवत्पदार्थः
यत्रानुरक्ताः सहसैव धीरा व्यपोह्य देहादिषु सङ्गमूढम्‌
व्रजन्ति तत्पारमहंस्यमन्त्यं यस्मिन्नहिंसोपशमः स्वधर्मः
अहं हि पृष्टोऽर्यमणो भवद्भिराचक्ष आत्मावगमोऽत्र यावान्‌
नभः पतन्त्यात्मसमं पतत्त्रिणस्तथा समं विष्णुगतिं विपश्चितः
एकदा धनुरुद्यम्य विचरन्मृगयां वने
मृगाननुगतः श्रान्तः क्षुधितस्तृषितो भृशम्‌
जलाशयमचक्षाणः प्रविवेश तमाश्रमम्‌
ददर्श मुनिमासीनं शान्तं मीलितलोचनम्‌
प्रतिरुद्धेन्द्रियप्राण मनोबुद्धिमुपारतम्‌
स्थानत्रयात्परं प्राप्तं ब्रह्मभूतमविक्रियम्‌
विप्रकीर्णजटाच्छन्नं रौरवेणाजिनेन च
विशुष्यत्तालुरुदकं तथाभूतमयाचत
अलब्धतृणभूम्यादिरसम्प्राप्तार्घ्यसूनृतः
अवज्ञातमिवात्मानं मन्यमानश्चुकोप ह
अभूतपूर्वः सहसा क्षुत्तृड्भ्यामर्दितात्मनः
ब्राह्मणं प्रत्यभूद्ब्रह्मन्मत्सरो मन्युरेव च
स तु ब्रह्मऋषेरंसे गतासुमुरगं रुषा
विनिर्गच्छन्धनुष्कोट्या निधाय पुरमागतः
एष किं निभृताशेष करणो मीलितेक्षणः
मृषासमाधिराहोस्वित्किं नु स्यात्क्षत्रबन्धुभिः
तस्य पुत्रोऽतितेजस्वी विहरन्बालकोऽर्भकैः
राज्ञाघं प्रापितं तातं श्रुत्वा तत्रेदमब्रवीत्‌
अहो अधर्मः पालानां पीव्नां बलिभुजामिव
स्वामिन्यघं यद्दासानां द्वारपानां शुनामिव
ब्राह्मणैः क्षत्रबन्धुर्हि गृहपालो निरूपितः
स कथं तद्गृहे द्वाःस्थः सभाण्डं भोक्तुमर्हति
कृष्णे गते भगवति शास्तर्युत्पथगामिनाम्‌
तद्भिन्नसेतूनद्याहं शास्मि पश्यत मे बलम्‌
इत्युक्त्वा रोषताम्राक्षो वयस्यानृषिबालकः
कौशिक्याप उपस्पृश्य वाग्वज्रं विससर्ज ह
इति लङ्घितमर्यादं तक्षकः सप्तमेऽहनि
दङ्क्ष्यति स्म कुलाङ्गारं चोदितो मे ततद्रुहम्‌
ततोऽभ्येत्याश्रमं बालो गले सर्पकलेवरम्‌
पितरं वीक्ष्य दुःखार्तो मुक्तकण्ठो रुरोद ह
स वा आङ्गिरसो ब्रह्मन्श्रुत्वा सुतविलापनम्‌
उन्मील्य शनकैर्नेत्रे दृष्ट्वा चांसे मृतोरगम्‌
विसृज्य तं च पप्रच्छ वत्स कस्माद्धि रोदिषि
केन वा तेऽपकृतमित्युक्तः स न्यवेदयत्‌
निशम्य शप्तमतदर्हं नरेन्द्रं स ब्राह्मणो नात्मजमभ्यनन्दत्‌
अहो बतांहो महदद्य ते कृतमल्पीयसि द्रोह उरुर्दमो धृतः
न वै नृभिर्नरदेवं पराख्यं सम्मातुमर्हस्यविपक्वबुद्धे
यत्तेजसा दुर्विषहेण गुप्ता विन्दन्ति भद्राण्यकुतोभयाः प्रजाः
अलक्ष्यमाणे नरदेवनाम्नि रथाङ्गपाणावयमङ्ग लोकः
तदा हि चौरप्रचुरो विनङ्क्ष्यत्यरक्ष्यमाणोऽविवरूथवत्क्षणात्‌
तदद्य नः पापमुपैत्यनन्वयं यन्नष्टनाथस्य वसोर्विलुम्पकात्‌
परस्परं घ्नन्ति शपन्ति वृञ्जते पशून्स्त्रियोऽर्थान्पुरुदस्यवो जनाः
तदार्यधर्मः प्रविलीयते नृणां वर्णाश्रमाचारयुतस्त्रयीमयः
ततोऽर्थकामाभिनिवेशितात्मनां शुनां कपीनामिव वर्णसङ्करः
धर्मपालो नरपतिः स तु सम्राड्बृहच्छ्रवाः
साक्षान्महाभागवतो राजर्षिर्हयमेधयाट्‌
क्षुत्तृट्श्रमयुतो दीनो नैवास्मच्छापमर्हति
अपापेषु स्वभृत्येषु बालेनापक्वबुद्धिना
पापं कृतं तद्भगवान्सर्वात्मा क्षन्तुमर्हति
तिरस्कृता विप्रलब्धाः शप्ताः क्षिप्ता हता अपि
नास्य तत्प्रतिकुर्वन्ति तद्भक्ताः प्रभवोऽपि हि
इति पुत्रकृताघेन सोऽनुतप्तो महामुनिः
स्वयं विप्रकृतो राज्ञा नैवाघं तदचिन्तयत्‌
प्रायशः साधवो लोके परैर्द्वन्द्वेषु योजिताः
न व्यथन्ति न हृष्यन्ति यत आत्मागुणाश्रयः
सूत उवाच
महीपतिस्त्वथ तत्कर्म गर्ह्यं विचिन्तयन्नात्मकृतं सुदुर्मनाः
अहो मया नीचमनार्यवत्कृतं निरागसि ब्रह्मणि गूढतेजसि
ध्रुवं ततो मे कृतदेवहेलनाद्दुरत्ययं व्यसनं नातिदीर्घात्‌
तदस्तु कामं ह्यघनिष्कृताय मे यथा न कुर्यां पुनरेवमद्धा
अद्यैव राज्यं बलमृद्धकोशं प्रकोपितब्रह्मकुलानलो मे
दहत्वभद्रस्य पुनर्न मेऽभूत्पापीयसी धीर्द्विजदेवगोभ्यः
स चिन्तयन्नित्थमथाशृणोद्यथा मुनेः सुतोक्तो निरृतिस्तक्षकाख्यः
स साधु मेने न चिरेण तक्षका नलं प्रसक्तस्य विरक्तिकारणम्‌
अथो विहायेमममुं च लोकं विमर्शितौ हेयतया पुरस्तात्‌
कृष्णाङ्घ्रिसेवामधिमन्यमान उपाविशत्प्रायममर्त्यनद्याम्‌
या वै लसच्छ्रीतुलसीविमिश्र कृष्णाङ्घ्रिरेण्वभ्यधिकाम्बुनेत्री
पुनाति लोकानुभयत्र सेशान्कस्तां न सेवेत मरिष्यमाणः
इति व्यवच्छिद्य स पाण्डवेयः प्रायोपवेशं प्रति विष्णुपद्याम्‌
दधौ मुकुन्दाङ्घ्रिमनन्यभावो मुनिव्रतो मुक्तसमस्तसङ्गः
तत्रोपजग्मुर्भुवनं पुनाना महानुभावा मुनयः सशिष्याः
प्रायेण तीर्थाभिगमापदेशैः स्वयं हि तीर्थानि पुनन्ति सन्तः
अत्रिर्वसिष्ठश्च्यवनः शरद्वानरिष्टनेमिर्भृगुरङ्गिराश्च
पराशरो गाधिसुतोऽथ राम उतथ्य इन्द्रप्रमदेध्मवाहौ
मेधातिथिर्देवल आर्ष्टिषेणो भारद्वाजो गौतमः पिप्पलादः
मैत्रेय और्वः कवषः कुम्भयोनिर्द्वैपायनो भगवान्नारदश्च
अन्ये च देवर्षिब्रह्मर्षिवर्या राजर्षिवर्या अरुणादयश्च
नानार्षेयप्रवरान्समेतानभ्यर्च्य राजा शिरसा ववन्दे
सुखोपविष्टेष्वथ तेषु भूयः कृतप्रणामः स्वचिकीर्षितं यत्‌
विज्ञापयामास विविक्तचेता उपस्थितोऽग्रेऽभिगृहीतपाणिः
राजोवाच
अहो वयं धन्यतमा नृपाणां महत्तमानुग्रहणीयशीलाः
राज्ञां कुलं ब्राह्मणपादशौचाद्दूराद्विसृष्टं बत गर्ह्यकर्म
तस्यैव मेऽघस्य परावरेशो व्यासक्तचित्तस्य गृहेष्वभीक्ष्णम्‌
निर्वेदमूलो द्विजशापरूपो यत्र प्रसक्तो भयमाशु धत्ते
तं मोपयातं प्रतियन्तु विप्रा गङ्गा च देवी धृतचित्तमीशे
द्विजोपसृष्टः कुहकस्तक्षको वा दशत्वलं गायत विष्णुगाथाः
पुनश्च भूयाद्भगवत्यनन्ते रतिः प्रसङ्गश्च तदाश्रयेषु
महत्सु यां यामुपयामि सृष्टिं मैत्र्यस्तु सर्वत्र नमो द्विजेभ्यः
इति स्म राजाध्यवसाययुक्तः प्राचीनमूलेषु कुशेषु धीरः
उदङ्मुखो दक्षिणकूल आस्ते समुद्रपत्न्याः स्वसुतन्यस्तभारः
एवं च तस्मिन्नरदेवदेवे प्रायोपविष्टे दिवि देवसङ्घाः
प्रशस्य भूमौ व्यकिरन्प्रसूनैर्मुदा मुहुर्दुन्दुभयश्च नेदुः
महर्षयो वै समुपागता ये प्रशस्य साध्वित्यनुमोदमानाः
ऊचुः प्रजानुग्रहशीलसारा यदुत्तमश्लोकगुणाभिरूपम्‌
न वा इदं राजर्षिवर्य चित्रं भवत्सु कृष्णं समनुव्रतेषु
येऽध्यासनं राजकिरीटजुष्टं सद्यो जहुर्भगवत्पार्श्वकामाः
सर्वे वयं तावदिहास्महेऽथ कलेवरं यावदसौ विहाय
लोकं परं विरजस्कं विशोकं यास्यत्ययं भागवतप्रधानः
आश्रुत्य तदृषिगणवचः परीक्षित्समं मधुच्युद्गुरु चाव्यलीकम्‌
आभाषतैनानभिनन्द्य युक्तान्शुश्रूषमाणश्चरितानि विष्णोः
समागताः सर्वत एव सर्वे वेदा यथा मूर्तिधरास्त्रिपृष्ठे
नेहाथ नामुत्र च कश्चनार्थ ऋते परानुग्रहमात्मशीलम्‌
ततश्च वः पृच्छ्यमिमं विपृच्छे विश्रभ्य विप्रा इति कृत्यतायाम्‌
सर्वात्मना म्रियमाणैश्च कृत्यं शुद्धं च तत्रामृशताभियुक्ताः
तत्राभवद्भगवान्व्यासपुत्रो यदृच्छया गामटमानोऽनपेक्षः
अलक्ष्यलिङ्गो निजलाभतुष्टो वृतश्च बालैरवधूतवेषः
तं द्व्यष्टवर्षं सुकुमारपाद करोरुबाह्वंसकपोलगात्रम्‌
चार्वायताक्षोन्नसतुल्यकर्ण सुभ्र्वाननं कम्बुसुजातकण्ठम्‌
निगूढजत्रुं पृथुतुङ्गवक्षसमावर्तनाभिं वलिवल्गूदरं च
दिगम्बरं वक्त्रविकीर्णकेशं प्रलम्बबाहुं स्वमरोत्तमाभम्‌
श्यामं सदापीव्यवयोऽङ्गलक्ष्म्या स्त्रीणां मनोज्ञं रुचिरस्मितेन
प्रत्युत्थितास्ते मुनयः स्वासनेभ्यस्तल्लक्षणज्ञा अपि गूढवर्चसम्‌
स विष्णुरातोऽतिथय आगताय तस्मै सपर्यां शिरसाजहार
ततो निवृत्ता ह्यबुधाः स्त्रियोऽर्भका महासने सोपविवेश पूजितः
स संवृतस्तत्र महान्महीयसां ब्रह्मर्षिराजर्षिदेवर्षिसङ्घैः
व्यरोचतालं भगवान्यथेन्दुर्ग्रहर्क्षतारानिकरैः परीतः
प्रशान्तमासीनमकुण्ठमेधसं मुनिं नृपो भागवतोऽभ्युपेत्य
प्रणम्य मूर्ध्नावहितः कृताञ्जलिर्नत्वा गिरा सूनृतयान्वपृच्छत्‌
परीक्षिदुवाच
अहो अद्य वयं ब्रह्मन्सत्सेव्याः क्षत्रबन्धवः
कृपयातिथिरूपेण भवद्भिस्तीर्थकाः कृताः
येषां संस्मरणात्पुंसां सद्यः शुद्ध्यन्ति वै गृहाः
किं पुनर्दर्शनस्पर्श पादशौचासनादिभिः
सान्निध्यात्ते महायोगिन्पातकानि महान्त्यपि
सद्यो नश्यन्ति वै पुंसां विष्णोरिव सुरेतराः
अपि मे भगवान्प्रीतः कृष्णः पाण्डुसुतप्रियः
पैतृष्वसेयप्रीत्यर्थं तद्गोत्रस्यात्तबान्धवः
अन्यथा तेऽव्यक्तगतेर्दर्शनं नः कथं नृणाम्‌
नितरां म्रियमाणानां संसिद्धस्य वनीयसः
अतः पृच्छामि संसिद्धिं योगिनां परमं गुरुम्‌
पुरुषस्येह यत्कार्यं म्रियमाणस्य सर्वथा
यच्छ्रोतव्यमथो जप्यं यत्कर्तव्यं नृभिः प्रभो
स्मर्तव्यं भजनीयं वा ब्रूहि यद्वा विपर्ययम्‌
नूनं भगवतो ब्रह्मन्गृहेषु गृहमेधिनाम्‌
न लक्ष्यते ह्यवस्थानमपि गोदोहनं क्वचित्‌
सूत उवाच
एवमाभाषितः पृष्टः स राज्ञा श्लक्ष्णया गिरा
प्रत्यभाषत धर्मज्ञो भगवान्बादरायणिः
श्रीशुक उवाच
वरीयानेष ते प्रश्नः कृतो लोकहितं नृप
आत्मवित्सम्मतः पुंसां श्रोतव्यादिषु यः परः
श्रोतव्यादीनि राजेन्द्र नृणां सन्ति सहस्रशः
अपश्यतामात्मतत्त्वं गृहेषु गृहमेधिनाम्‌
निद्रया ह्रियते नक्तं व्यवायेन च वा वयः
दिवा चार्थेहया राजन्कुटुम्बभरणेन वा
देहापत्यकलत्रादिष्वात्मसैन्येष्वसत्स्वपि
तेषां प्रमत्तो निधनं पश्यन्नपि न पश्यति
तस्माद्भारत सर्वात्मा भगवानीश्वरो हरिः
श्रोतव्यः कीर्तितव्यश्च स्मर्तव्यश्चेच्छताभयम्‌
एतावान्साङ्ख्ययोगाभ्यां स्वधर्मपरिनिष्ठया
जन्मलाभः परः पुंसामन्ते नारायणस्मृतिः
प्रायेण मुनयो राजन्निवृत्ता विधिषेधतः
नैर्गुण्यस्था रमन्ते स्म गुणानुकथने हरेः
इदं भागवतं नाम पुराणं ब्रह्मसम्मितम्‌
अधीतवान्द्वापरादौ पितुर्द्वैपायनादहम्‌
परिनिष्ठितोऽपि नैर्गुण्य उत्तमश्लोकलीलया
गृहीतचेता राजर्षे आख्यानं यदधीतवान्‌
तदहं तेऽभिधास्यामि महापौरुषिको भवान्‌
यस्य श्रद्दधतामाशु स्यान्मुकुन्दे मतिः सती
एतन्निर्विद्यमानानामिच्छतामकुतोभयम्‌
योगिनां नृप निर्णीतं हरेर्नामानुकीर्तनम्‌
किं प्रमत्तस्य बहुभिः परोक्षैर्हायनैरिह
वरं मुहूर्तं विदितं घटते श्रेयसे यतः
खट्वाङ्गो नाम राजर्षिर्ज्ञात्वेयत्तामिहायुषः
मुहूर्तात्सर्वमुत्सृज्य गतवानभयं हरिम्‌
तवाप्येतर्हि कौरव्य सप्ताहं जीवितावधिः
उपकल्पय तत्सर्वं तावद्यत्साम्परायिकम्‌
अन्तकाले तु पुरुष आगते गतसाध्वसः
छिन्द्यादसङ्गशस्त्रेण स्पृहां देहेऽनु ये च तम्‌
गृहात्प्रव्रजितो धीरः पुण्यतीर्थजलाप्लुतः
शुचौ विविक्त आसीनो विधिवत्कल्पितासने
अभ्यसेन्मनसा शुद्धं त्रिवृद्ब्रह्माक्षरं परम्‌
मनो यच्छेज्जितश्वासो ब्रह्मबीजमविस्मरन्‌
नियच्छेद्विषयेभ्योऽक्षान्मनसा बुद्धिसारथिः
मनः कर्मभिराक्षिप्तं शुभार्थे धारयेद्धिया
तत्रैकावयवं ध्यायेदव्युच्छिन्नेन चेतसा
मनो निर्विषयं युक्त्वा ततः किञ्चन न स्मरेत्‌
पदं तत्परमं विष्णोर्मनो यत्र प्रसीदति
रजस्तमोभ्यामाक्षिप्तं विमूढं मन आत्मनः
यच्छेद्धारणया धीरो हन्ति या तत्कृतं मलम्‌
यस्यां सन्धार्यमाणायां योगिनो भक्तिलक्षणः
आशु सम्पद्यते योग आश्रयं भद्रमीक्षतः
राजोवाच
यथा सन्धार्यते ब्रह्मन्धारणा यत्र सम्मता
यादृशी वा हरेदाशु पुरुषस्य मनोमलम्‌
श्रीशुक उवाच
जितासनो जितश्वासो जितसङ्गो जितेन्द्रियः
स्थूले भगवतो रूपे मनः सन्धारयेद्धिया
विशेषस्तस्य देहोऽयं स्थविष्ठश्च स्थवीयसाम्‌
यत्रेदं व्यज्यते विश्वं भूतं भव्यं भवच्च सत्‌
अण्डकोशे शरीरेऽस्मिन्सप्तावरणसंयुते
वैराजः पुरुषो योऽसौ भगवान्धारणाश्रयः
पातालमेतस्य हि पादमूलं पठन्ति पार्ष्णिप्रपदे रसातलम्‌
महातलं विश्वसृजोऽथ गुल्फौ तलातलं वै पुरुषस्य जङ्घे
द्वे जानुनी सुतलं विश्वमूर्तेरूरुद्वयं वितलं चातलं च
महीतलं तज्जघनं महीपते नभस्तलं नाभिसरो गृणन्ति
उरःस्थलं ज्योतिरनीकमस्य ग्रीवा महर्वदनं वै जनोऽस्य
तपो वराटीं विदुरादिपुंसः सत्यं तु शीर्षाणि सहस्रशीर्ष्णः
इन्द्रादयो बाहव आहुरुस्राः कर्णौ दिशः श्रोत्रममुष्य शब्दः
नासत्यदस्रौ परमस्य नासे घ्राणोऽस्य गन्धो मुखमग्निरिद्धः
द्यौरक्षिणी चक्षुरभूत्पतङ्गः पक्ष्माणि विष्णोरहनी उभे च
तद्भ्रूविजृम्भः परमेष्ठिधिष्ण्यमापोऽस्य तालू रस एव जिह्वा
छन्दांस्यनन्तस्य शिरो गृणन्ति दंष्ट्रा यमः स्नेहकला द्विजानि
हासो जनोन्मादकरी च माया दुरन्तसर्गो यदपाङ्गमोक्षः
व्रीडोत्तरौष्ठोऽधर एव लोभो धर्मः स्तनोऽधर्मपथोऽस्य पृष्ठम्‌
कस्तस्य मेढ्रं वृषणौ च मित्रौ कुक्षिः समुद्रा गिरयोऽस्थिसङ्घाः
नाड्योऽस्य नद्योऽथ तनूरुहाणि महीरुहा विश्वतनोर्नृपेन्द्र
अनन्तवीर्यः श्वसितं मातरिश्वा गतिर्वयः कर्म गुणप्रवाहः
ईशस्य केशान्विदुरम्बुवाहान्वासस्तु सन्ध्यां कुरुवर्य भूम्नः
अव्यक्तमाहुर्हृदयं मनश्चस चन्द्रमाः सर्वविकारकोशः
विज्ञानशक्तिं महिमामनन्ति सर्वात्मनोऽन्तःकरणं गिरित्रम्‌
अश्वाश्वतर्युष्ट्रगजा नखानि सर्वे मृगाः पशवः श्रोणिदेशे
वयांसि तद्व्याकरणं विचित्रं मनुर्मनीषा मनुजो निवासः
गन्धर्वविद्याधरचारणाप्सरः स्वरस्मृतीरसुरानीकवीर्यः
ब्रह्माननं क्षत्रभुजो महात्मा विडूरुरङ्घ्रिश्रितकृष्णवर्णः
नानाभिधाभीज्यगणोपपन्नो द्रव्यात्मकः कर्म वितानयोगः
इयानसावीश्वरविग्रहस्य यः सन्निवेशः कथितो मया ते
सन्धार्यतेऽस्मिन्वपुषि स्थविष्ठे मनः स्वबुद्ध्या न यतोऽस्ति किञ्चित्‌
स सर्वधीवृत्त्यनुभूतसर्व आत्मा यथा स्वप्नजनेक्षितैकः
तं सत्यमानन्दनिधिं भजेत नान्यत्र सज्जेद्यत आत्मपातः
श्रीशुक उवाच
एवं पुरा धारणयात्मयोनिर्नष्टां स्मृतिं प्रत्यवरुध्य तुष्टात्‌
तथा ससर्जेदममोघदृष्टिर्यथाप्ययात्प्राग्व्यवसायबुद्धिः
शाब्दस्य हि ब्रह्मण एष पन्था यन्नामभिर्ध्यायति धीरपार्थैः
परिभ्रमंस्तत्र न विन्दतेऽर्थान्मायामये वासनया शयानः
अतः कविर्नामसु यावदर्थः स्यादप्रमत्तो व्यवसायबुद्धिः
सिद्धेऽन्यथार्थे न यतेत तत्र परिश्रमं तत्र समीक्षमाणः
सत्यां क्षितौ किं कशिपोः प्रयासैर्बाहौ स्वसिद्धे ह्युपबर्हणैः किम्‌
सत्यञ्जलौ किं पुरुधान्नपात्र्या दिग्वल्कलादौ सति किं दुकूलैः
चीराणि किं पथि न सन्ति दिशन्ति भिक्षां
नैवाङ्घ्रिपाः परभृतः सरितोऽप्यशुष्यन्‌
रुद्धा गुहाः किमजितोऽवति नोपसन्नान्‌
कस्माद्भजन्ति कवयो धनदुर्मदान्धान्‌
एवं स्वचित्ते स्वत एव सिद्ध आत्मा प्रियोऽर्थो भगवाननन्तः
तं निर्वृतो नियतार्थो भजेत संसारहेतूपरमश्च यत्र
कस्तां त्वनादृत्य परानुचिन्तामृते पशूनसतीं नाम कुर्यात्‌
पश्यञ्जनं पतितं वैतरण्यां स्वकर्मजान्परितापाञ्जुषाणम्‌
केचित्स्वदेहान्तर्हृदयावकाशे प्रादेशमात्रं पुरुषं वसन्तम्‌
चतुर्भुजं कञ्जरथाङ्गशङ्ख गदाधरं धारणया स्मरन्ति
रसन्नवक्त्रं नलिनायतेक्षणं कदम्बकिञ्जल्कपिशङ्गवाससम्‌
लसन्महारत्नहिरण्मयाङ्गदं स्फुरन्महारत्नकिरीटकुण्डलम्‌
उन्निद्रहृत्पङ्कजकर्णिकालये योगेश्वरास्थापितपादपल्लवम्‌
श्रीलक्षणं कौस्तुभरत्नकन्धरमम्लानलक्ष्म्या वनमालयाचितम्‌
विभूषितं मेखलयाङ्गुलीयकैर्महाधनैर्नूपुरकङ्कणादिभिः
स्निग्धामलाकुञ्चितनीलकुन्तलैर्विरोचमानाननहासपेशलम्‌
अदीनलीलाहसितेक्षणोल्लसद्भ्रूभङ्गसंसूचितभूर्यनुग्रहम्‌
ईक्षेत चिन्तामयमेनमीश्वरं यावन्मनो धारणयावतिष्ठते
एकैकशोऽङ्गानि धियानुभावयेत्पादादि यावद्धसितं गदाभृतः
जितं जितं स्थानमपोह्य धारयेत्परं परं शुद्ध्यति धीर्यथा यथा
यावन्न जायेत परावरेऽस्मिन्विश्वेश्वरे द्रष्टरि भक्तियोगः
तावत्स्थवीयः पुरुषस्य रूपं क्रियावसाने प्रयतः स्मरेत
स्थिरं सुखं चासनमास्थितो यतिर्यदा जिहासुरिममङ्ग लोकम्‌
काले च देशे च मनो न सज्जयेत्प्राणान्नियच्छेन्मनसा जितासुः
मनः स्वबुद्ध्यामलया नियम्य क्षेत्रज्ञ एतां निनयेत्तमात्मनि
आत्मानमात्मन्यवरुध्य धीरो लब्धोपशान्तिर्विरमेत कृत्यात्‌
न यत्र कालोऽनिमिषां परः प्रभुः कुतो नु देवा जगतां य ईशिरे
न यत्र सत्त्वं न रजस्तमश्च न वै विकारो न महान्प्रधानम्‌
परं पदं वैष्णवमामनन्ति तद्यन्नेति नेतीत्यतदुत्सिसृक्षवः
विसृज्य दौरात्म्यमनन्यसौहृदा हृदोपगुह्यार्हपदं पदे पदे
इत्थं मुनिस्तूपरमेद्व्यवस्थितो विज्ञानदृग्वीर्यसुरन्धिताशयः
स्वपार्ष्णिनापीड्य गुदं ततोऽनिलं स्थानेषु षट्सून्नमयेज्जितक्लमः
नाभ्यां स्थितं हृद्यधिरोप्य तस्मादुदानगत्योरसि तं नयेन्मुनिः
ततोऽनुसन्धाय धिया मनस्वी स्वतालुमूलं शनकैर्नयेत
तस्माद्भ्रुवोरन्तरमुन्नयेत निरुद्धसप्तायतनोऽनपेक्षः
स्थित्वा मुहूर्तार्धमकुण्ठदृष्टिर्निर्भिद्य मूर्धन्विसृजेत्परं गतः
यदि प्रयास्यन्नृप पारमेष्ठ्यं वैहायसानामुत यद्विहारम्‌
अष्टाधिपत्यं गुणसन्निवाये सहैव गच्छेन्मनसेन्द्रियैश्च
योगेश्वराणां गतिमाहुरन्तर्बहिस्त्रिलोक्याः पवनान्तरात्मनाम्‌
न कर्मभिस्तां गतिमाप्नुवन्ति विद्यातपोयोगसमाधिभाजाम्‌
वैश्वानरं याति विहायसा गतः सुषुम्णया ब्रह्मपथेन शोचिषा
विधूतकल्कोऽथ हरेरुदस्तात्प्रयाति चक्रं नृप शैशुमारम्‌
तद्विश्वनाभिं त्वतिवर्त्य विष्णोरणीयसा विरजेनात्मनैकः
नमस्कृतं ब्रह्मविदामुपैति कल्पायुषो यद्विबुधा रमन्ते
अथो अनन्तस्य मुखानलेन दन्दह्यमानं स निरीक्ष्य विश्वम्‌
निर्याति सिद्धेश्वरयुष्टधिष्ण्यं यद्द्वैपरार्ध्यं तदु पारमेष्ठ्यम्‌
न यत्र शोको न जरा न मृत्युर्नार्तिर्न चोद्वेग ऋते कुतश्चित्‌
यच्चित्ततोऽदः कृपयानिदंविदां दुरन्तदुःखप्रभवानुदर्शनात्‌
ततो विशेषं प्रतिपद्य निर्भयस्तेनात्मनापोऽनलमूर्तिरत्वरन्‌
ज्योतिर्मयो वायुमुपेत्य काले वाय्वात्मना खं बृहदात्मलिङ्गम्‌
घ्राणेन गन्धं रसनेन वै रसं रूपं च दृष्ट्या श्वसनं त्वचैव
श्रोत्रेण चोपेत्य नभोगुणत्वं प्राणेन चाकूतिमुपैति योगी
स भूतसूक्ष्मेन्द्रियसन्निकर्षं मनोमयं देवमयं विकार्यम्‌
संसाद्य गत्या सह तेन याति विज्ञानतत्त्वं गुणसन्निरोधम्‌
तेनात्मनात्मानमुपैति शान्तमानन्दमानन्दमयोऽवसाने
एतां गतिं भागवतीं गतो यः स वै पुनर्नेह विषज्जतेऽङ्ग
एते सृती ते नृप वेदगीते त्वयाभिपृष्टे च सनातने च
ये वै पुरा ब्रह्मण आह तुष्ट आराधितो भगवान्वासुदेवः
न ह्यतोऽन्यः शिवः पन्था विशतः संसृताविह
वासुदेवे भगवति भक्तियोगो यतो भवेत्‌
भगवान्ब्रह्म कार्त्स्न्येन त्रिरन्वीक्ष्य मनीषया
तदध्यवस्यत्कूटस्थो रतिरात्मन्यतो भवेत्‌
भगवान्सर्वभूतेषु लक्षितः स्वात्मना हरिः
दृश्यैर्बुद्ध्यादिभिर्द्रष्टा लक्षणैरनुमापकैः
तस्मात्सर्वात्मना राजन्हरिः सर्वत्र सर्वदा
श्रोतव्यः कीर्तितव्यश्च स्मर्तव्यो भगवान्नृणाम्‌
पिबन्ति ये भगवत आत्मनः सतां कथामृतं श्रवणपुटेषु सम्भृतम्‌
पुनन्ति ते विषयविदूषिताशयं व्रजन्ति तच्चरणसरोरुहान्तिकम्‌
श्रीशुक उवाच
एवमेतन्निगदितं पृष्टवान्यद्भवान्मम
नृणां यन्म्रियमाणानां मनुष्येषु मनीषिणाम्‌
ब्रह्मवर्चसकामस्तु यजेत ब्रह्मणः पतिम्‌
इन्द्रमिन्द्रियकामस्तु प्रजाकामः प्रजापतीन्‌
देवीं मायां तु श्रीकामस्तेजस्कामो विभावसुम्‌
वसुकामो वसून्रुद्रान्वीर्यकामोऽथ वीर्यवान्‌
अन्नाद्यकामस्त्वदितिं स्वर्गकामोऽदितेः सुतान्‌
विश्वान्देवान्राज्यकामः साध्यान्संसाधको विशाम्‌
आयुष्कामोऽश्विनौ देवौ पुष्टिकाम इलां यजेत्‌
प्रतिष्ठाकामः पुरुषो रोदसी लोकमातरौ
रूपाभिकामो गन्धर्वान्स्त्रीकामोऽप्सर उर्वशीम्‌
आधिपत्यकामः सर्वेषां यजेत परमेष्ठिनम्‌
यज्ञं यजेद्यशस्कामः कोशकामः प्रचेतसम्‌
विद्याकामस्तु गिरिशं दाम्पत्यार्थ उमां सतीम्‌
धर्मार्थ उत्तमश्लोकं तन्तुः तन्वन्पित्न्यजेत्‌
रक्षाकामः पुण्यजनानोजस्कामो मरुद्गणान्‌
राज्यकामो मनून्देवान्निरृतिं त्वभिचरन्यजेत्‌
कामकामो यजेत्सोममकामः पुरुषं परम्‌
अकामः सर्वकामो वा मोक्षकाम उदारधीः
तीव्रेण भक्तियोगेन यजेत पुरुषं परम्‌
एतावानेव यजतामिह निःश्रेयसोदयः
भगवत्यचलो भावो यद्भागवतसङ्गतः
ज्ञानं यदाप्रतिनिवृत्तगुणोर्मिचक्रम्‌
आत्मप्रसाद उत यत्र गुणेष्वसङ्गः
कैवल्यसम्मतपथस्त्वथ भक्तियोगः
को निर्वृतो हरिकथासु रतिं न कुर्यात्‌
शौनक उवाच
इत्यभिव्याहृतं राजा निशम्य भरतर्षभः
किमन्यत्पृष्टवान्भूयो वैयासकिमृषिं कविम्‌
एतच्छुश्रूषतां विद्वन्सूत नोऽर्हसि भाषितुम्‌
कथा हरिकथोदर्काः सतां स्युः सदसि ध्रुवम्‌
स वै भागवतो राजा पाण्डवेयो महारथः
बालक्रीडनकैः क्रीडन्कृष्णक्रीडां य आददे
वैयासकिश्च भगवान्वासुदेवपरायणः
उरुगायगुणोदाराः सतां स्युर्हि समागमे
आयुर्हरति वै पुंसामुद्यन्नस्तं च यन्नसौ
तस्यर्ते यत्क्षणो नीत उत्तमश्लोकवार्तया
तरवः किं न जीवन्ति भस्त्राः किं न श्वसन्त्युत
न खादन्ति न मेहन्ति किं ग्रामे पशवोऽपरे
श्वविड्वराहोष्ट्रखरैः संस्तुतः पुरुषः पशुः
न यत्कर्णपथोपेतो जातु नाम गदाग्रजः
बिले बतोरुक्रमविक्रमान्ये न शृण्वतः कर्णपुटे नरस्य
जिह्वासती दार्दुरिकेव सूत न चोपगायत्युरुगायगाथाः
भारः परं पट्टकिरीटजुष्टमप्युत्तमाङ्गं न नमेन्मुकुन्दम्‌
शावौ करौ नो कुरुते सपर्यां हरेर्लसत्काञ्चनकङ्कणौ वा
बर्हायिते ते नयने नराणां लिङ्गानि विष्णोर्न निरीक्षतो ये
पादौ नृणां तौ द्रुमजन्मभाजौ क्षेत्राणि नानुव्रजतो हरेर्यौ
जीवञ्छवो भागवताङ्घ्रिरेणुं न जातु मर्त्योऽभिलभेत यस्तु
श्रीविष्णुपद्या मनुजस्तुलस्याः श्वसञ्छवो यस्तु न वेद गन्धम्‌
तदश्मसारं हृदयं बतेदं यद्गृह्यमाणैर्हरिनामधेयैः
न विक्रियेताथ यदा विकारो नेत्रे जलं गात्ररुहेषु हर्षः
अथाभिधेह्यङ्ग मनोऽनुकूलं प्रभाषसे भागवतप्रधानः
यदाह वैयासकिरात्मविद्या विशारदो नृपतिं साधु पृष्टः
सूत उवाच
वैयासकेरिति वचस्तत्त्वनिश्चयमात्मनः
उपधार्य मतिं कृष्णे औत्तरेयः सतीं व्यधात्‌
आत्मजायासुतागार पशुद्रविणबन्धुषु
राज्ये चाविकले नित्यं विरूढां ममतां जहौ
पप्रच्छ चेममेवार्थं यन्मां पृच्छथ सत्तमाः
कृष्णानुभावश्रवणे श्रद्दधानो महामनाः
संस्थां विज्ञाय सन्न्यस्य कर्म त्रैवर्गिकं च यत्‌
वासुदेवे भगवति आत्मभावं दृढं गतः
राजोवाच
समीचीनं वचो ब्रह्मन्सर्वज्ञस्य तवानघ
तमो विशीर्यते मह्यं हरेः कथयतः कथाम्‌
भूय एव विवित्सामि भगवानात्ममायया
यथेदं सृजते विश्वं दुर्विभाव्यमधीश्वरैः
यथा गोपायति विभुर्यथा संयच्छते पुनः
यां यां शक्तिमुपाश्रित्य पुरुशक्तिः परः पुमान्‌
आत्मानं क्रीडयन्क्रीडन्करोति विकरोति च
नूनं भगवतो ब्रह्मन्हरेरद्भुतकर्मणः
दुर्विभाव्यमिवाभाति कविभिश्चापि चेष्टितम्‌
यथा गुणांस्तु प्रकृतेर्युगपत्क्रमशोऽपि वा
बिभर्ति भूरिशस्त्वेकः कुर्वन्कर्माणि जन्मभिः
विचिकित्सितमेतन्मे ब्रवीतु भगवान्यथा
शाब्दे ब्रह्मणि निष्णातः परस्मिंश्च भवान्खलु
सूत उवाच
इत्युपामन्त्रितो राज्ञा गुणानुकथने हरेः
हृषीकेशमनुस्मृत्य प्रतिवक्तुं प्रचक्रमे
श्रीशुक उवाच
नमः परस्मै पुरुषाय भूयसे सदुद्भवस्थाननिरोधलीलया
गृहीतशक्तित्रितयाय देहिनामन्तर्भवायानुपलक्ष्यवर्त्मने
भूयो नमः सद्वृजिनच्छिदेऽसतामसम्भवायाखिलसत्त्वमूर्तये
पुंसां पुनः पारमहंस्य आश्रमे व्यवस्थितानामनुमृग्यदाशुषे
नमो नमस्तेऽस्त्वृषभाय सात्वतां विदूरकाष्ठाय मुहुः कुयोगिनाम्‌
निरस्तसाम्यातिशयेन राधसा स्वधामनि ब्रह्मणि रंस्यते नमः
यत्कीर्तनं यत्स्मरणं यदीक्षणं यद्वन्दनं यच्छ्रवणं यदर्हणम्‌
लोकस्य सद्यो विधुनोति कल्मषं तस्मै सुभद्रश्रवसे नमो नमः
विचक्षणा यच्चरणोपसादनात्सङ्गं व्युदस्योभयतोऽन्तरात्मनः
विन्दन्ति हि ब्रह्मगतिं गतक्लमास्तस्मै सुभद्रश्रवसे नमो नमः
तपस्विनो दानपरा यशस्विनो मनस्विनो मन्त्रविदः सुमङ्गलाः
क्षेमं न विन्दन्ति विना यदर्पणं तस्मै सुभद्रश्रवसे नमो नमः
किरातहूणान्ध्रपुलिन्दपुल्कशा आभीरशुम्भा यवनाः खसादयः
येऽन्ये च पापा यदपाश्रयाश्रयाः शुध्यन्ति तस्मै प्रभविष्णवे नमः
स एष आत्मात्मवतामधीश्वरस्त्रयीमयो धर्ममयस्तपोमयः
गतव्यलीकैरजशङ्करादिभिर्वितर्क्यलिङ्गो भगवान्प्रसीदताम्‌
श्रियः पतिर्यज्ञपतिः प्रजापतिर्धियां पतिर्लोकपतिर्धरापतिः
पतिर्गतिश्चान्धकवृष्णिसात्वतां प्रसीदतां मे भगवान्सतां पतिः
यदङ्घ्र्यभिध्यानसमाधिधौतया धियानुपश्यन्ति हि तत्त्वमात्मनः
वदन्ति चैतत्कवयो यथारुचं स मे मुकुन्दो भगवान्प्रसीदताम्‌
प्रचोदिता येन पुरा सरस्वती वितन्वताजस्य सतीं स्मृतिं हृदि
स्वलक्षणा प्रादुरभूत्किलास्यतः स मे ऋषीणामृषभः प्रसीदताम्‌
भूतैर्महद्भिर्य इमाः पुरो विभुर्निर्माय शेते यदमूषु पूरुषः
भुङ्क्ते गुणान्षोडश षोडशात्मकः सोऽलङ्कृषीष्ट भगवान्वचांसि मे
नमस्तस्मै भगवते वासुदेवाय वेधसे
पपुर्ज्ञानमयं सौम्या यन्मुखाम्बुरुहासवम्‌
एतदेवात्मभू राजन्नारदाय विपृच्छते
वेदगर्भोऽभ्यधात्साक्षाद्यदाह हरिरात्मनः
नारद उवाच
देवदेव नमस्तेऽस्तु भूतभावन पूर्वज
तद्विजानीहि यज्ज्ञानमात्मतत्त्वनिदर्शनम्‌
यद्रूपं यदधिष्ठानं यतः सृष्टमिदं प्रभो
यत्संस्थं यत्परं यच्च तत्तत्त्वं वद तत्त्वतः
सर्वं ह्येतद्भवान्वेद भूतभव्यभवत्प्रभुः
करामलकवद्विश्वं विज्ञानावसितं तव
यद्विज्ञानो यदाधारो यत्परस्त्वं यदात्मकः
एकः सृजसि भूतानि भूतैरेवात्ममायया
आत्मन्भावयसे तानि न पराभावयन्स्वयम्‌
आत्मशक्तिमवष्टभ्य ऊर्णनाभिरिवाक्लमः
नाहं वेद परं ह्यस्मिन्नापरं न समं विभो
नामरूपगुणैर्भाव्यं सदसत्किञ्चिदन्यतः
स भवानचरद्घोरं यत्तपः सुसमाहितः
तेन खेदयसे नस्त्वं पराशङ्कां च यच्छसि
एतन्मे पृच्छतः सर्वं सर्वज्ञ सकलेश्वर
विजानीहि यथैवेदमहं बुध्येऽनुशासितः
ब्रह्मोवाच
सम्यक्कारुणिकस्येदं वत्स ते विचिकित्सितम्‌
यदहं चोदितः सौम्य भगवद्वीर्यदर्शने
नानृतं तव तच्चापि यथा मां प्रब्रवीषि भोः
अविज्ञाय परं मत्त एतावत्त्वं यतो हि मे
येन स्वरोचिषा विश्वं रोचितं रोचयाम्यहम्‌
यथार्कोऽग्निर्यथा सोमो यथर्क्षग्रहतारकाः
तस्मै नमो भगवते वासुदेवाय धीमहि
यन्मायया दुर्जयया मां वदन्ति जगद्गुरुम्‌
विलज्जमानया यस्य स्थातुमीक्षापथेऽमुया
विमोहिता विकत्थन्ते ममाहमिति दुर्धियः
द्रव्यं कर्म च कालश्च स्वभावो जीव एव च
वासुदेवात्परो ब्रह्मन्न चान्योऽर्थोऽस्ति तत्त्वतः
नारायणपरा वेदा देवा नारायणाङ्गजाः
नारायणपरा लोका नारायणपरा मखाः
नारायणपरो योगो नारायणपरं तपः
नारायणपरं ज्ञानं नारायणपरा गतिः
तस्यापि द्रष्टुरीशस्य कूटस्थस्याखिलात्मनः
सृज्यं सृजामि सृष्टोऽहमीक्षयैवाभिचोदितः
सत्त्वं रजस्तम इति निर्गुणस्य गुणास्त्रयः
स्थितिसर्गनिरोधेषु गृहीता मायया विभोः
कार्यकारणकर्तृत्वे द्रव्यज्ञानक्रियाश्रयाः
बध्नन्ति नित्यदा मुक्तं मायिनं पुरुषं गुणाः
स एष भगवांल्लिङ्गैस्त्रिभिरेतैरधोक्षजः
स्वलक्षितगतिर्ब्रह्मन्सर्वेषां मम चेश्वरः
कालं कर्म स्वभावं च मायेशो मायया स्वया
आत्मन्यदृच्छया प्राप्तं विबुभूषुरुपाददे
कालाद्गुणव्यतिकरः परिणामः स्वभावतः
कर्मणो जन्म महतः पुरुषाधिष्ठितादभूत्‌
महतस्तु विकुर्वाणाद्रजःसत्त्वोपबृंहितात्‌
तमःप्रधानस्त्वभवद्द्रव्यज्ञानक्रियात्मकः
सोऽहङ्कार इति प्रोक्तो विकुर्वन्समभूत्त्रिधा
वैकारिकस्तैजसश्च तामसश्चेति यद्भिदा
द्रव्यशक्तिः क्रियाशक्तिर्ज्ञानशक्तिरिति प्रभो
तामसादपि भूतादेर्विकुर्वाणादभून्नभः
तस्य मात्रा गुणः शब्दो लिङ्गं यद्द्रष्टृदृश्ययोः
नभसोऽथ विकुर्वाणादभूत्स्पर्शगुणोऽनिलः
परान्वयाच्छब्दवांश्च प्राण ओजः सहो बलम्‌
वायोरपि विकुर्वाणात्कालकर्मस्वभावतः
उदपद्यत तेजो वै रूपवत्स्पर्शशब्दवत्‌
तेजसस्तु विकुर्वाणादासीदम्भो रसात्मकम्‌
रूपवत्स्पर्शवच्चाम्भो घोषवच्च परान्वयात्‌
विशेषस्तु विकुर्वाणादम्भसो गन्धवानभूत्‌
परान्वयाद्रसस्पर्श शब्दरूपगुणान्वितः
वैकारिकान्मनो जज्ञे देवा वैकारिका दश
दिग्वातार्कप्रचेतोऽश्वि वह्नीन्द्रोपेन्द्रमित्रकाः
तैजसात्तु विकुर्वाणादिन्द्रियाणि दशाभवन्‌
ज्ञानशक्तिः क्रियाशक्तिर्बुद्धिः प्राणश्च तैजसौ
श्रोत्रं त्वग्घ्राणदृग्जिह्वा वाग्दोर्मेढ्राङ्घ्रिपायवः
यदैतेऽसङ्गता भावा भूतेन्द्रियमनोगुणाः
यदायतननिर्माणे न शेकुर्ब्रह्मवित्तम
तदा संहत्य चान्योन्यं भगवच्छक्तिचोदिताः
सदसत्त्वमुपादाय चोभयं ससृजुर्ह्यदः
वर्षपूगसहस्रान्ते तदण्डमुदके शयम्‌
कालकर्मस्वभावस्थो जीवो ञ्जीवमजीवयत्‌
स एव पुरुषस्तस्मादण्डं निर्भिद्य निर्गतः
सहस्रोर्वङ्घ्रिबाह्वक्षः सहस्राननशीर्षवान्‌
यस्येहावयवैर्लोकान्कल्पयन्ति मनीषिणः
कट्यादिभिरधः सप्त सप्तोर्ध्वं जघनादिभिः
पुरुषस्य मुखं ब्रह्म क्षत्रमेतस्य बाहवः
ऊर्वोर्वैश्यो भगवतः पद्भ्यां शूद्रो व्यजायत
भूर्लोकः कल्पितः पद्भ्यां भुवर्लोकोऽस्य नाभितः
हृदा स्वर्लोक उरसा महर्लोको महात्मनः
ग्रीवायां जनलोकोऽस्य तपोलोकः स्तनद्वयात्‌
मूर्धभिः सत्यलोकस्तु ब्रह्मलोकः सनातनः
तत्कट्यां चातलं कॢप्तमूरुभ्यां वितलं विभोः
जानुभ्यां सुतलं शुद्धं जङ्घाभ्यां तु तलातलम्‌
महातलं तु गुल्फाभ्यां प्रपदाभ्यां रसातलम्‌
पातालं पादतलत इति लोकमयः पुमान्‌
भूर्लोकः कल्पितः पद्भ्यां भुवर्लोकोऽस्य नाभितः
स्वर्लोकः कल्पितो मूर्ध्ना इति वा लोककल्पना
ब्रह्मोवाच
वाचां वह्नेर्मुखं क्षेत्रं छन्दसां सप्त धातवः
हव्यकव्यामृतान्नानां जिह्वा सर्वरसस्य च
सर्वासूनां च वायोश्च तन्नासे परमायणे
अश्विनोरोषधीनां च घ्राणो मोदप्रमोदयोः
रूपाणां तेजसां चक्षुर्दिवः सूर्यस्य चाक्षिणी
कर्णौ दिशां च तीर्थानां श्रोत्रमाकाशशब्दयोः
तद्गात्रं वस्तुसाराणां सौभगस्य च भाजनम्‌
त्वगस्य स्पर्शवायोश्च सर्वमेधस्य चैव हि
रोमाण्युद्भिज्जजातीनां यैर्वा यज्ञस्तु सम्भृतः
केशश्मश्रुनखान्यस्य शिलालोहाभ्रविद्युताम्‌
बाहवो लोकपालानां प्रायशः क्षेमकर्मणाम्‌
विक्रमो भूर्भुवः स्वश्च क्षेमस्य शरणस्य च
सर्वकामवरस्यापि हरेश्चरण आस्पदम्‌
अपां वीर्यस्य सर्गस्य पर्जन्यस्य प्रजापतेः
पुंसः शिश्न उपस्थस्तु प्रजात्यानन्दनिर्वृतेः
पायुर्यमस्य मित्रस्य परिमोक्षस्य नारद
हिंसाया निरृतेर्मृत्योर्निरयस्य गुदं स्मृतः
पराभूतेरधर्मस्य तमसश्चापि पश्चिमः
नाड्यो नदनदीनां च गोत्राणामस्थिसंहतिः
अव्यक्तरससिन्धूनां भूतानां निधनस्य च
उदरं विदितं पुंसो हृदयं मनसः पदम्‌
धर्मस्य मम तुभ्यं च कुमाराणां भवस्य च
विज्ञानस्य च सत्त्वस्य परस्यात्मा परायणम्‌
अहं भवान्भवश्चैव त इमे मुनयोऽग्रजाः
सुरासुरनरा नागाः खगा मृगसरीसृपाः
गन्धर्वाप्सरसो यक्षा रक्षोभूतगणोरगाः
पशवः पितरः सिद्धा विद्याध्राश्चारणा द्रुमाः
अन्ये च विविधा जीवाजलस्थलनभौकसः
ग्रहर्क्षकेतवस्तारास्तडितः स्तनयित्नवः
सर्वं पुरुष एवेदं भूतं भव्यं भवच्च यत्‌
तेनेदमावृतं विश्वं वितस्तिमधितिष्ठति
स्वधिष्ण्यं प्रतपन्प्राणो बहिश्च प्रतपत्यसौ
एवं विराजं प्रतपंस्तपत्यन्तर्बहिः पुमान्‌
सोऽमृतस्याभयस्येशो मर्त्यमन्नं यदत्यगात्‌
महिमैष ततो ब्रह्मन्पुरुषस्य दुरत्ययः
पादेषु सर्वभूतानि पुंसः स्थितिपदो विदुः
अमृतं क्षेममभयं त्रिमूर्ध्नोऽधायि मूर्धसु
पादास्त्रयो बहिश्चासन्नप्रजानां य आश्रमाः
अन्तस्त्रिलोक्यास्त्वपरो गृहमेधोऽबृहद्व्रतः
सृती विचक्रमे विश्वम्साशनानशने उभे
यदविद्या च विद्या च पुरुषस्तूभयाश्रयः
यस्मादण्डं विराड्जज्ञे भूतेन्द्रियगुणात्मकः
तद्द्रव्यमत्यगाद्विश्वं गोभिः सूर्य इवातपन्‌
यदास्य नाभ्यान्नलिनादहमासं महात्मनः
नाविदं यज्ञसम्भारान्पुरुषावयवानृते
तेषु यज्ञस्य पशवः सवनस्पतयः कुशाः
इदं च देवयजनं कालश्चोरुगुणान्वितः
वस्तून्योषधयः स्नेहा रसलोहमृदो जलम्‌
ऋचो यजूंषि सामानि चातुर्होत्रं च सत्तम
नामधेयानि मन्त्राश्च दक्षिणाश्च व्रतानि च
देवतानुक्रमः कल्पः सङ्कल्पस्तन्त्रमेव च
गतयो मतयश्चैव प्रायश्चित्तं समर्पणम्‌
पुरुषावयवैरेते सम्भाराः सम्भृता मया
इति सम्भृतसम्भारः पुरुषावयवैरहम्‌
तमेव पुरुषं यज्ञं तेनैवायजमीश्वरम्‌
ततस्ते भ्रातर इमे प्रजानां पतयो नव
अयजन्व्यक्तमव्यक्तं पुरुषं सुसमाहिताः
ततश्च मनवः काले ईजिरे ऋषयोऽपरे
पितरो विबुधा दैत्या मनुष्याः क्रतुभिर्विभुम्‌
नारायणे भगवति तदिदं विश्वमाहितम्‌
गृहीतमायोरुगुणः सर्गादावगुणः स्वतः
सृजामि तन्नियुक्तोऽहं हरो हरति तद्वशः
विश्वं पुरुषरूपेण परिपाति त्रिशक्तिधृक्‌
इति तेऽभिहितं तात यथेदमनुपृच्छसि
नान्यद्भगवतः किञ्चिद्भाव्यं सदसदात्मकम्‌
न भारती मेऽङ्ग मृषोपलक्ष्यते न वै क्वचिन्मे मनसो मृषा गतिः
न मे हृषीकाणि पतन्त्यसत्पथे यन्मे हृदौत्कण्ठ्यवता धृतो हरिः
सोऽहं समाम्नायमयस्तपोमयः प्रजापतीनामभिवन्दितः पतिः
आस्थाय योगं निपुणं समाहितस्तं नाध्यगच्छं यत आत्मसम्भवः
नतोऽस्म्यहं तच्चरणं समीयुषां भवच्छिदं स्वस्त्ययनं सुमङ्गलम्‌
यो ह्यात्ममायाविभवं स्म पर्यगाद्यथा नभः स्वान्तमथापरे कुतः
नाहं न यूयं यदृतां गतिं विदुर्न वामदेवः किमुतापरे सुराः
तन्मायया मोहितबुद्धयस्त्विदं विनिर्मितं चात्मसमं विचक्ष्महे
यस्यावतारकर्माणि गायन्ति ह्यस्मदादयः
न यं विदन्ति तत्त्वेन तस्मै भगवते नमः
स एष आद्यः पुरुषः कल्पे कल्पे सृजत्यजः
आत्मात्मन्यात्मनात्मानं स संयच्छति पाति च
विशुद्धं केवलं ज्ञानं प्रत्यक्सम्यगवस्थितम्‌
सत्यं पूर्णमनाद्यन्तं निर्गुणं नित्यमद्वयम्‌
ऋषे विदन्ति मुनयः प्रशान्तात्मेन्द्रियाशयाः
यदा तदेवासत्तर्कैस्तिरोधीयेत विप्लुतम्‌
आद्योऽवतारः पुरुषः परस्य कालः स्वभावः सदसन्मनश्च
द्रव्यं विकारो गुण इन्द्रियाणि विराट्स्वराट्स्थास्नु चरिष्णु भूम्नः
अहं भवो यज्ञ इमे प्रजेशा दक्षादयो ये भवदादयश्च
स्वर्लोकपालाः खगलोकपाला नृलोकपालास्तललोकपालाः
गन्धर्वविद्याधरचारणेशा ये यक्षरक्षोरगनागनाथाः
ये वा ऋषीणामृषभाः पित्णां दैत्येन्द्रसिद्धेश्वरदानवेन्द्राः
अन्ये च ये प्रेतपिशाचभूत कूष्माण्डयादोमृगपक्ष्यधीशाः
यत्किञ्च लोके भगवन्महस्वदोजःसहस्वद्बलवत्क्षमावत्‌
श्रीह्रीविभूत्यात्मवदद्भुतार्णं तत्त्वं परं रूपवदस्वरूपम्‌
प्राधान्यतो यानृष आमनन्ति लीलावतारान्पुरुषस्य भूम्नः
आपीयतां कर्णकषायशोषाननुक्रमिष्ये त इमान्सुपेशान्‌
ब्रह्मोवाच
यत्रोद्यतः क्षितितलोद्धरणाय बिभ्रत्‌
क्रौडीं तनुं सकलयज्ञमयीमनन्तः
अन्तर्महार्णव उपागतमादिदैत्यं
तं दंष्ट्रयाद्रिमिव वज्रधरो ददार
जातो रुचेरजनयत्सुयमान्सुयज्ञ
आकूतिसूनुरमरानथ दक्षिणायाम्‌
लोकत्रयस्य महतीमहरद्यदार्तिं
स्वायम्भुवेन मनुना हरिरित्यनूक्तः
जज्ञे च कर्दमगृहे द्विज देवहूत्यां
स्त्रीभिः समं नवभिरात्मगतिं स्वमात्रे
ऊचे ययात्मशमलं गुणसङ्गपङ्कम्‌
अस्मिन्विधूय कपिलस्य गतिं प्रपेदे
अत्रेरपत्यमभिकाङ्क्षत आह तुष्टो
दत्तो मयाहमिति यद्भगवान्स दत्तः
यत्पादपङ्कजपरागपवित्रदेहा
योगर्द्धिमापुरुभयीं यदुहैहयाद्याः
तप्तं तपो विविधलोकसिसृक्षया मे
आदौ सनात्स्वतपसः स चतुःसनोऽभूत्‌
प्राक्कल्पसम्प्लवविनष्टमिहात्मतत्त्वं
सम्यग्जगाद मुनयो यदचक्षतात्मन्‌
धर्मस्य दक्षदुहितर्यजनिष्ट मूर्त्यां
नारायणो नर इति स्वतपःप्रभावः
दृष्ट्वात्मनो भगवतो नियमावलोपं
देव्यस्त्वनङ्गपृतना घटितुं न शेकुः
कामं दहन्ति कृतिनो ननु रोषदृष्ट्या
रोषं दहन्तमुत ते न दहन्त्यसह्यम्‌
सोऽयं यदन्तरमलं प्रविशन्बिभेति
कामः कथं नु पुनरस्य मनः श्रयेत
विद्धः सपत्न्युदितपत्रिभिरन्ति राज्ञो
बालोऽपि सन्नुपगतस्तपसे वनानि
तस्मा अदाद्ध्रुवगतिं गृणते प्रसन्नो
दिव्याः स्तुवन्ति मुनयो यदुपर्यधस्तात्‌
यद्वेनमुत्पथगतं द्विजवाक्यवज्र
निष्प्लुष्टपौरुषभगं निरये पतन्तम्‌
त्रात्वार्थितो जगति पुत्रपदं च लेभे
दुग्धा वसूनि वसुधा सकलानि येन
नाभेरसावृषभ आस सुदेविसूनुर्‌
यो वै चचार समदृग्जडयोगचर्याम्‌
यत्पारमहंस्यमृषयः पदमामनन्ति
स्वस्थः प्रशान्तकरणः परिमुक्तसङ्गः
सत्रे ममास भगवान्हयशीरषाथो
साक्षात्स यज्ञपुरुषस्तपनीयवर्णः
छन्दोमयो मखमयोऽखिलदेवतात्मा
वाचो बभूवुरुशतीः श्वसतोऽस्य नस्तः
मत्स्यो युगान्तसमये मनुनोपलब्धः
क्षोणीमयो निखिलजीवनिकायकेतः
विस्रंसितानुरुभये सलिले मुखान्मे
आदाय तत्र विजहार ह वेदमार्गान्‌
क्षीरोदधावमरदानवयूथपानाम्‌
उन्मथ्नताममृतलब्धय आदिदेवः
पृष्ठेन कच्छपवपुर्विदधार गोत्रं
निद्राक्षणोऽद्रिपरिवर्तकषाणकण्डूः
त्रैपिष्टपोरुभयहा स नृसिंहरूपं
कृत्वा भ्रमद्भ्रुकुटिदंष्ट्रकरालवक्त्रम्‌
दैत्येन्द्रमाशु गदयाभिपतन्तमाराद्‌
ऊरौ निपात्य विददार नखैः स्फुरन्तम्‌
अन्तःसरस्युरुबलेन पदे गृहीतो
ग्राहेण यूथपतिरम्बुजहस्त आर्तः
आहेदमादिपुरुषाखिललोकनाथ
तीर्थश्रवः श्रवणमङ्गलनामधेय
श्रुत्वा हरिस्तमरणार्थिनमप्रमेयश्‌
चक्रायुधः पतगराजभुजाधिरूढः
चक्रेण नक्रवदनं विनिपाट्य तस्माद्‌
धस्ते प्रगृह्य भगवान्कृपयोज्जहार
ज्यायान्गुणैरवरजोऽप्यदितेः सुतानां
लोकान्विचक्रम इमान्यदथाधियज्ञः
क्ष्मां वामनेन जगृहे त्रिपदच्छलेन
याच्ञामृते पथि चरन्प्रभुभिर्न चाल्यः
नार्थो बलेरयमुरुक्रमपादशौचम्‌
आपः शिखाधृतवतो विबुधाधिपत्यम्‌
यो वै प्रतिश्रुतमृते न चिकीर्षदन्यद्‌
आत्मानमङ्ग मनसा हरयेऽभिमेने
तुभ्यं च नारद भृशं भगवान्विवृद्ध
भावेन साधु परितुष्ट उवाच योगम्‌
ज्ञानं च भागवतमात्मसतत्त्वदीपं
यद्वासुदेवशरणा विदुरञ्जसैव
चक्रं च दिक्ष्वविहतं दशसु स्वतेजो
मन्वन्तरेषु मनुवंशधरो बिभर्ति
दुष्टेषु राजसु दमं व्यदधात्स्वकीर्तिं
सत्ये त्रिपृष्ठ उशतीं प्रथयंश्चरित्रैः
धन्वन्तरिश्च भगवान्स्वयमेव कीर्तिर्‌
नाम्ना नृणां पुरुरुजां रुज आशु हन्ति
यज्ञे च भागममृतायुरवावरुन्ध
आयुष्यवेदमनुशास्त्यवतीर्य लोके
क्षत्रं क्षयाय विधिनोपभृतं महात्मा
ब्रह्मध्रुगुज्झितपथं नरकार्तिलिप्सु
उद्धन्त्यसाववनिकण्टकमुग्रवीर्यस्‌
त्रिःसप्तकृत्व उरुधारपरश्वधेन
अस्मत्प्रसादसुमुखः कलया कलेश
इक्ष्वाकुवंश अवतीर्य गुरोर्निदेशे
तिष्ठन्वनं सदयितानुज आविवेश
यस्मिन्विरुध्य दशकन्धर आर्तिमार्च्छत्‌
यस्मा अदादुदधिरूढभयाङ्गवेपो
मार्गं सपद्यरिपुरं हरवद्दिधक्षोः
दूरे सुहृन्मथितरोषसुशोणदृष्ट्या
तातप्यमानमकरोरगनक्रचक्रः
वक्षःस्थलस्पर्शरुग्नमहेन्द्रवाह
दन्तैर्विडम्बितककुब्जुष ऊढहासम्‌
सद्योऽसुभिः सह विनेष्यति दारहर्तुर्‌
विस्फूर्जितैर्धनुष उच्चरतोऽधिसैन्ये
भूमेः सुरेतरवरूथविमर्दितायाः
क्लेशव्ययाय कलया सितकृष्णकेशः
जातः करिष्यति जनानुपलक्ष्यमार्गः
कर्माणि चात्ममहिमोपनिबन्धनानि
तोकेन जीवहरणं यदुलूकिकायास्‌
त्रैमासिकस्य च पदा शकटोऽपवृत्तः
यद्रिङ्गतान्तरगतेन दिविस्पृशोर्वा
उन्मूलनं त्वितरथार्जुनयोर्न भाव्यम्‌
यद्वै व्रजे व्रजपशून्विषतोयपीतान्‌
पालांस्त्वजीवयदनुग्रहदृष्टिवृष्ट्या
तच्छुद्धयेऽतिविषवीर्यविलोलजिह्वम्‌
उच्चाटयिष्यदुरगं विहरन्ह्रदिन्याम्‌
तत्कर्म दिव्यमिव यन्निशि निःशयानं
दावाग्निना शुचिवने परिदह्यमाने
उन्नेष्यति व्रजमतोऽवसितान्तकालं
नेत्रे पिधाप्य सबलोऽनधिगम्यवीर्यः
गृह्णीत यद्यदुपबन्धममुष्य माता
शुल्बं सुतस्य न तु तत्तदमुष्य माति
यज्जृम्भतोऽस्य वदने भुवनानि गोपी
संवीक्ष्य शङ्कितमनाः प्रतिबोधितासीत्‌
नन्दं च मोक्ष्यति भयाद्वरुणस्य पाशाद्‌
गोपान्बिलेषु पिहितान्मयसूनुना च
अह्न्यापृतं निशि शयानमतिश्रमेण
लोकं विकुण्ठमुपनेष्यति गोकुलं स्म
गोपैर्मखे प्रतिहते व्रजविप्लवाय
देवेऽभिवर्षति पशून्कृपया रिरक्षुः
धर्तोच्छिलीन्ध्रमिव सप्तदिनानि सप्त
वर्षो महीध्रमनघैककरे सलीलम्‌
क्रीडन्वने निशि निशाकररश्मिगौर्यां
रासोन्मुखः कलपदायतमूर्च्छितेन
उद्दीपितस्मररुजां व्रजभृद्वधूनां
हर्तुर्हरिष्यति शिरो धनदानुगस्य
ये च प्रलम्बखरदर्दुरकेश्यरिष्ट
मल्लेभकंसयवनाः कपिपौण्ड्रकाद्याः
अन्ये च शाल्वकुजबल्वलदन्तवक्र
सप्तोक्षशम्बरविदूरथरुक्मिमुख्याः
ये वा मृधे समितिशालिन आत्तचापाः
काम्बोजमत्स्यकुरुसृञ्जयकैकयाद्याः
यास्यन्त्यदर्शनमलं बलपार्थभीम
व्याजाह्वयेन हरिणा निलयं तदीयम्‌
कालेन मीलितधियामवमृश्य न्णां
स्तोकायुषां स्वनिगमो बत दूरपारः
आविर्हितस्त्वनुयुगं स हि सत्यवत्यां
वेदद्रुमं विटपशो विभजिष्यति स्म
देवद्विषां निगमवर्त्मनि निष्ठितानां
पूर्भिर्मयेन विहिताभिरदृश्यतूर्भिः
लोकान्घ्नतां मतिविमोहमतिप्रलोभं
वेषं विधाय बहु भाष्यत औपधर्म्यम्‌
यर्ह्यालयेष्वपि सतां न हरेः कथाः स्युः
पाषण्डिनो द्विजजना वृषला नृदेवाः
स्वाहा स्वधा वषडिति स्म गिरो न यत्र
शास्ता भविष्यति कलेर्भगवान्युगान्ते
सर्गे तपोऽहमृषयो नव ये प्रजेशाः
स्थानेऽथ धर्ममखमन्वमरावनीशाः
अन्ते त्वधर्महरमन्युवशासुराद्या
मायाविभूतय इमाः पुरुशक्तिभाजः
विष्णोर्नु वीर्यगणनां कतमोऽर्हतीह
यः पार्थिवान्यपि कविर्विममे रजांसि
चस्कम्भ यः स्वरहसास्खलता त्रिपृष्ठं
यस्मात्त्रिसाम्यसदनादुरुकम्पयानम्‌
नान्तं विदाम्यहममी मुनयोऽग्रजास्ते
मायाबलस्य पुरुषस्य कुतोऽवरा ये
गायन्गुणान्दशशतानन आदिदेवः
शेषोऽधुनापि समवस्यति नास्य पारम्‌
येषां स एष भगवान्दययेदनन्तः
सर्वात्मनाश्रितपदो यदि निर्व्यलीकम्‌
ते दुस्तरामतितरन्ति च देवमायां
नैषां ममाहमिति धीः श्वशृगालभक्ष्ये
वेदाहमङ्ग परमस्य हि योगमायां
यूयं भवश्च भगवानथ दैत्यवर्यः
पत्नी मनोः स च मनुश्च तदात्मजाश्च
प्राचीनबर्हिरृभुरङ्ग उत ध्रुवश्च
इक्ष्वाकुरैलमुचुकुन्दविदेहगाधि
रघ्वम्बरीषसगरा गयनाहुषाद्याः
मान्धात्रलर्कशतधन्वनुरन्तिदेवा
देवव्रतो बलिरमूर्त्तरयो दिलीपः
सौभर्युतङ्कशिबिदेवलपिप्पलाद
सारस्वतोद्धवपराशरभूरिषेणाः
येऽन्ये विभीषणहनूमदुपेन्द्रदत्त
पार्थार्ष्टिषेणविदुरश्रुतदेववर्याः
ते वै विदन्त्यतितरन्ति च देवमायां
स्त्रीशूद्रहूणशबरा अपि पापजीवाः
यद्यद्भुतक्रमपरायणशीलशिक्षास्‌
तिर्यग्जना अपि किमु श्रुतधारणा ये
शश्वत्प्रशान्तमभयं प्रतिबोधमात्रं
शुद्धं समं सदसतः परमात्मतत्त्वम्‌
शब्दो न यत्र पुरुकारकवान्क्रियार्थो
माया परैत्यभिमुखे च विलज्जमाना
तद्वै पदं भगवतः परमस्य पुंसो
ब्रह्मेति यद्विदुरजस्रसुखं विशोकम्‌
सध्र्यङ्नियम्य यतयो यमकर्तहेतिं
जह्युः स्वराडिव निपानखनित्रमिन्द्रः
स श्रेयसामपि विभुर्भगवान्यतोऽस्य
भावस्वभावविहितस्य सतः प्रसिद्धिः
देहे स्वधातुविगमेऽनुविशीर्यमाणे
व्योमेव तत्र पुरुषो न विशीर्यते ञ्जः
सोऽयं तेऽभिहितस्तात भगवान्विश्वभावनः
समासेन हरेर्नान्यदन्यस्मात्सदसच्च यत्‌
इदं भागवतं नाम यन्मे भगवतोदितम्‌
सङ्ग्रहोऽयं विभूतीनां त्वमेतद्विपुली कुरु
यथा हरौ भगवति नृणां भक्तिर्भविष्यति
सर्वात्मन्यखिलाधारे इति सङ्कल्प्य वर्णय
मायां वर्णयतोऽमुष्य ईश्वरस्यानुमोदतः
शृण्वतः श्रद्धया नित्यं माययात्मा न मुह्यति
राजोवाच
ब्रह्मणा चोदितो ब्रह्मन्गुणाख्यानेऽगुणस्य च
यस्मै यस्मै यथा प्राह नारदो देवदर्शनः
एतद्वेदितुमिच्छामि तत्त्वं तत्त्वविदां वर
हरेरद्भुतवीर्यस्य कथा लोकसुमङ्गलाः
कथयस्व महाभाग यथाहमखिलात्मनि
कृष्णे निवेश्य निःसङ्गं मनस्त्यक्ष्ये कलेवरम्‌
शृण्वतः श्रद्धया नित्यं गृणतश्च स्वचेष्टितम्‌
कालेन नातिदीर्घेण भगवान्विशते हृदि
प्रविष्टः कर्णरन्ध्रेण स्वानां भावसरोरुहम्‌
धुनोति शमलं कृष्णः सलिलस्य यथा शरत्‌
धौतात्मा पुरुषः कृष्ण पादमूलं न मुञ्चति
मुक्तसर्वपरिक्लेशः पान्थः स्वशरणं यथा
यदधातुमतो ब्रह्मन्देहारम्भोऽस्य धातुभिः
यदृच्छया हेतुना वा भवन्तो जानते यथा
आसीद्यदुदरात्पद्मं लोकसंस्थानलक्षणम्‌
यावानयं वै पुरुष इयत्तावयवैः पृथक्‌
तावानसाविति प्रोक्तः संस्थावयववानिव
अजः सृजति भूतानि भूतात्मा यदनुग्रहात्‌
ददृशे येन तद्रूपं नाभिपद्मसमुद्भवः
स चापि यत्र पुरुषो विश्वस्थित्युद्भवाप्ययः
मुक्त्वात्ममायां मायेशः शेते सर्वगुहाशयः
पुरुषावयवैर्लोकाः सपालाः पूर्वकल्पिताः
लोकैरमुष्यावयवाः सपालैरिति शुश्रुम
यावान्कल्पो विकल्पो वा यथा कालोऽनुमीयते
भूतभव्यभवच्छब्द आयुर्मानं च यत्सतः
कालस्यानुगतिर्या तु लक्ष्यतेऽण्वी बृहत्यपि
यावत्यः कर्मगतयो यादृशीर्द्विजसत्तम
यस्मिन्कर्मसमावायो यथा येनोपगृह्यते
गुणानां गुणिनां चैव परिणाममभीप्सताम्‌
भूपातालककुब्व्योम ग्रहनक्षत्रभूभृताम्‌
सरित्समुद्रद्वीपानां सम्भवश्चैतदोकसाम्‌
प्रमाणमण्डकोशस्य बाह्याभ्यन्तरभेदतः
महतां चानुचरितं वर्णाश्रमविनिश्चयः
युगानि युगमानं च धर्मो यश्च युगे युगे
अवतारानुचरितं यदाश्चर्यतमं हरेः
नृणां साधारणो धर्मः सविशेषश्च यादृशः
श्रेणीनां राजर्षीणां च धर्मः कृच्छ्रेषु जीवताम्‌
तत्त्वानां परिसङ्ख्यानं लक्षणं हेतुलक्षणम्‌
पुरुषाराधनविधिर्योगस्याध्यात्मिकस्य च
योगेश्वरैश्वर्यगतिर्लिङ्गभङ्गस्तु योगिनाम्‌
वेदोपवेदधर्माणामितिहासपुराणयोः
सम्प्लवः सर्वभूतानां विक्रमः प्रतिसङ्क्रमः
इष्टापूर्तस्य काम्यानां त्रिवर्गस्य च यो विधिः
यो वानुशायिनां सर्गः पाषण्डस्य च सम्भवः
आत्मनो बन्धमोक्षौ च व्यवस्थानं स्वरूपतः
यथात्मतन्त्रो भगवान्विक्रीडत्यात्ममायया
विसृज्य वा यथा मायामुदास्ते साक्षिवद्विभुः
सर्वमेतच्च भगवन्पृच्छतो मेऽनुपूर्वशः
तत्त्वतोऽर्हस्युदाहर्तुं प्रपन्नाय महामुने
अत्र प्रमाणं हि भवान्परमेष्ठी यथात्मभूः
अपरे चानुतिष्ठन्ति पूर्वेषां पूर्वजैः कृतम्‌
न मेऽसवः परायन्ति ब्रह्मन्ननशनादमी
पिबतो ञ्च्युतपीयूषम्तद्वाक्याब्धिविनिःसृतम्‌
सूत उवाच
स उपामन्त्रितो राज्ञा कथायामिति सत्पतेः
ब्रह्मरातो भृशं प्रीतो विष्णुरातेन संसदि
प्राह भागवतं नाम पुराणं ब्रह्मसम्मितम्‌
ब्रह्मणे भगवत्प्रोक्तं ब्रह्मकल्प उपागते
यद्यत्परीक्षिदृषभः पाण्डूनामनुपृच्छति
आनुपूर्व्येण तत्सर्वमाख्यातुमुपचक्रमे
श्रीशुक उवाच
आत्ममायामृते राजन्परस्यानुभवात्मनः
न घटेतार्थसम्बन्धः स्वप्नद्रष्टुरिवाञ्जसा
बहुरूप इवाभाति मायया बहुरूपया
रममाणो गुणेष्वस्या ममाहमिति मन्यते
यर्हि वाव महिम्नि स्वे परस्मिन्कालमाययोः
रमेत गतसम्मोहस्त्यक्त्वोदास्ते तदोभयम्‌
आत्मतत्त्वविशुद्ध्यर्थं यदाह भगवानृतम्‌
ब्रह्मणे दर्शयन्रूपमव्यलीकव्रतादृतः
स आदिदेवो जगतां परो गुरुः स्वधिष्ण्यमास्थाय सिसृक्षयैक्षत
तां नाध्यगच्छद्दृशमत्र सम्मतां प्रपञ्चनिर्माणविधिर्यया भवेत्‌
स चिन्तयन्द्व्यक्षरमेकदाम्भस्युपाशृणोद्द्विर्गदितं वचो विभुः
स्पर्शेषु यत्षोडशमेकविंशं निष्किञ्चनानां नृप यद्धनं विदुः
निशम्य तद्वक्तृदिदृक्षया दिशो विलोक्य तत्रान्यदपश्यमानः
स्वधिष्ण्यमास्थाय विमृश्य तद्धितं तपस्युपादिष्ट इवादधे मनः
दिव्यं सहस्राब्दममोघदर्शनो जितानिलात्मा विजितोभयेन्द्रियः
अतप्यत स्माखिललोकतापनं तपस्तपीयांस्तपतां समाहितः
तस्मै स्वलोकं भगवान्सभाजितः सन्दर्शयामास परं न यत्परम्‌
व्यपेतसङ्क्लेशविमोहसाध्वसं स्वदृष्टवद्भिर्पुरुषैरभिष्टुतम्‌
प्रवर्तते यत्र रजस्तमस्तयोः सत्त्वं च मिश्रं न च कालविक्रमः
न यत्र माया किमुतापरे हरेरनुव्रता यत्र सुरासुरार्चिताः
श्यामावदाताः शतपत्रलोचनाः पिशङ्गवस्त्राः सुरुचः सुपेशसः
सर्वे चतुर्बाहव उन्मिषन्मणि प्रवेकनिष्काभरणाः सुवर्चसः
प्रवालवैदूर्यमृणालवर्चसः परिस्फुरत्कुण्डलमौलिमालिनः
भ्राजिष्णुभिर्यः परितो विराजते लसद्विमानावलिभिर्महात्मनाम्‌
विद्योतमानः प्रमदोत्तमाद्युभिः सविद्युदभ्रावलिभिर्यथा नभः
श्रीर्यत्र रूपिण्युरुगायपादयोः करोति मानं बहुधा विभूतिभिः
प्रेङ्खं श्रिता या कुसुमाकरानुगैर्विगीयमाना प्रियकर्म गायती
ददर्श तत्राखिलसात्वतां पतिं श्रियः पतिं यज्ञपतिं जगत्पतिम्‌
सुनन्दनन्दप्रबलार्हणादिभिः स्वपार्षदाग्रैः परिसेवितं विभुम्‌
भृत्यप्रसादाभिमुखं दृगासवं प्रसन्नहासारुणलोचनाननम्‌
किरीटिनं कुण्डलिनं चतुर्भुजं पीतांशुकं वक्षसि लक्षितं श्रिया
अध्यर्हणीयासनमास्थितं परं वृतं चतुःषोडशपञ्चशक्तिभिः
युक्तं भगैः स्वैरितरत्र चाध्रुवैः स्व एव धामन्रममाणमीश्वरम्‌
तद्दर्शनाह्लादपरिप्लुतान्तरो हृष्यत्तनुः प्रेमभराश्रुलोचनः
ननाम पादाम्बुजमस्य विश्वसृग्यत्पारमहंस्येन पथाधिगम्यते
तं प्रीयमाणं समुपस्थितं कविं प्रजाविसर्गे निजशासनार्हणम्‌
बभाष ईषत्स्मितशोचिषा गिरा प्रियः प्रियं प्रीतमनाः करे स्पृशन्‌
श्रीभगवानुवाच
त्वयाहं तोषितः सम्यग्वेदगर्भ सिसृक्षया
चिरं भृतेन तपसा दुस्तोषः कूटयोगिनाम्‌
वरं वरय भद्रं ते वरेशं माभिवाञ्छितम्‌
ब्रह्मञ्छ्रेयःपरिश्रामः पुंसां मद्दर्शनावधिः
मनीषितानुभावोऽयं मम लोकावलोकनम्‌
यदुपश्रुत्य रहसि चकर्थ परमं तपः
प्रत्यादिष्टं मया तत्र त्वयि कर्मविमोहिते
तपो मे हृदयं साक्षादात्माहं तपसोऽनघ
सृजामि तपसैवेदं ग्रसामि तपसा पुनः
बिभर्मि तपसा विश्वं वीर्यं मे दुश्चरं तपः
ब्रह्मोवाच
भगवन्सर्वभूतानामध्यक्षोऽवस्थितो गुहाम्‌
वेद ह्यप्रतिरुद्धेन प्रज्ञानेन चिकीर्षितम्‌
तथापि नाथमानस्य नाथ नाथय नाथितम्‌
परावरे यथा रूपेजानीयां ते त्वरूपिणः
यथात्ममायायोगेन नानाशक्त्युपबृंहितम्‌
विलुम्पन्विसृजन्गृह्णन्बिभ्रदात्मानमात्मना
क्रीडस्यमोघसङ्कल्प ऊर्णनाभिर्यथोर्णुते
तथा तद्विषयां धेहि मनीषां मयि माधव
भगवच्छिक्षितमहं करवाणि ह्यतन्द्रितः
नेहमानः प्रजासर्गं बध्येयं यदनुग्रहात्‌
यावत्सखा सख्युरिवेश ते कृतः प्रजाविसर्गे विभजामि भो जनम्‌
अविक्लवस्ते परिकर्मणि स्थितो मा मे समुन्नद्धमदो ञ्ज मानिनः
श्रीभगवानुवाच
ज्ञानं परमगुह्यं मे यद्विज्ञानसमन्वितम्‌
सरहस्यं तदङ्गं च गृहाण गदितं मया
यावानहं यथाभावो यद्रूपगुणकर्मकः
तथैव तत्त्वविज्ञानमस्तु ते मदनुग्रहात्‌
अहमेवासमेवाग्रे नान्यद्यत्सदसत्परम्‌
पश्चादहं यदेतच्च योऽवशिष्येत सोऽस्म्यहम्‌
ऋतेऽर्थं यत्प्रतीयेत न प्रतीयेत चात्मनि
तद्विद्यादात्मनो मायां यथाभासो यथा तमः
यथा महान्ति भूतानि भूतेषूच्चावचेष्वनु
प्रविष्टान्यप्रविष्टानि तथा तेषु न तेष्वहम्‌
एतावदेव जिज्ञास्यं तत्त्वजिज्ञासुनात्मनः
अन्वयव्यतिरेकाभ्यां यत्स्यात्सर्वत्र सर्वदा
एतन्मतं समातिष्ठ परमेण समाधिना
भवान्कल्पविकल्पेषु न विमुह्यति कर्हिचित्‌
श्रीशुक उवाच
सम्प्रदिश्यैवमजनो जनानां परमेष्ठिनम्‌
पश्यतस्तस्य तद्रूपमात्मनो न्यरुणद्धरिः
अन्तर्हितेन्द्रियार्थाय हरये विहिताञ्जलिः
सर्वभूतमयो विश्वं ससर्जेदं स पूर्ववत्‌
प्रजापतिर्धर्मपतिरेकदा नियमान्यमान्‌
भद्रं प्रजानामन्विच्छन्नातिष्ठत्स्वार्थकाम्यया
तं नारदः प्रियतमो रिक्थादानामनुव्रतः
शुश्रूषमाणः शीलेन प्रश्रयेण दमेन च
मायां विविदिषन्विष्णोर्मायेशस्य महामुनिः
महाभागवतो राजन्पितरं पर्यतोषयत्‌
तुष्टं निशाम्य पितरं लोकानां प्रपितामहम्‌
देवर्षिः परिपप्रच्छ भवान्यन्मानुपृच्छति
तस्मा इदं भागवतं पुराणं दशलक्षणम्‌
प्रोक्तं भगवता प्राह प्रीतः पुत्राय भूतकृत्‌
नारदः प्राह मुनये सरस्वत्यास्तटे नृप
ध्यायते ब्रह्म परमं व्यासायामिततेजसे
यदुताहं त्वया पृष्टो वैराजात्पुरुषादिदम्‌
यथासीत्तदुपाख्यास्ते प्रश्नानन्यांश्च कृत्स्नशः
श्रीशुक उवाच
अत्र सर्गो विसर्गश्च स्थानं पोषणमूतयः
मन्वन्तरेशानुकथा निरोधो मुक्तिराश्रयः
दशमस्य विशुद्ध्यर्थं नवानामिह लक्षणम्‌
वर्णयन्ति महात्मानः श्रुतेनार्थेन चाञ्जसा
भूतमात्रेन्द्रियधियां जन्म सर्ग उदाहृतः
ब्रह्मणो गुणवैषम्याद्विसर्गः पौरुषः स्मृतः
स्थितिर्वैकुण्ठविजयः पोषणं तदनुग्रहः
मन्वन्तराणि सद्धर्म ऊतयः कर्मवासनाः
अवतारानुचरितं हरेश्चास्यानुवर्तिनाम्‌
पुंसामीशकथाः प्रोक्ता नानाख्यानोपबृंहिताः
निरोधोऽस्यानुशयनमात्मनः सह शक्तिभिः
मुक्तिर्हित्वान्यथा रूपं स्वरूपेण व्यवस्थितिः
आभासश्च निरोधश्च यतोऽस्त्यध्यवसीयते
स आश्रयः परं ब्रह्म परमात्मेति शब्द्यते
योऽध्यात्मिकोऽयं पुरुषः सोऽसावेवाधिदैविकः
यस्तत्रोभयविच्छेदः पुरुषो ह्याधिभौतिकः
एकमेकतराभावे यदा नोपलभामहे
त्रितयं तत्र यो वेद स आत्मा स्वाश्रयाश्रयः
पुरुषोऽण्डं विनिर्भिद्य यदासौ स विनिर्गतः
आत्मनोऽयनमन्विच्छन्नपोऽस्राक्षीच्छुचिः शुचीः
तास्ववात्सीत्स्वसृष्टासु सहस्रं परिवत्सरान्‌
तेन नारायणो नाम यदापः पुरुषोद्भवाः
द्रव्यं कर्म च कालश्च स्वभावो जीव एव च
यदनुग्रहतः सन्ति न सन्ति यदुपेक्षया
एको नानात्वमन्विच्छन्योगतल्पात्समुत्थितः
वीर्यं हिरण्मयं देवो मायया व्यसृजत्त्रिधा
अधिदैवमथाध्यात्ममधिभूतमिति प्रभुः
अथैकं पौरुषं वीर्यं त्रिधाभिद्यत तच्छृणु
अन्तः शरीर आकाशात्पुरुषस्य विचेष्टतः
ओजः सहो बलं जज्ञे ततः प्राणो महानसुः
अनुप्राणन्ति यं प्राणाः प्राणन्तं सर्वजन्तुषु
अपानन्तमपानन्ति नरदेवमिवानुगाः
प्राणेनाक्षिपता क्षुत्तृडन्तरा जायते विभोः
पिपासतो जक्षतश्च प्राङ्मुखं निरभिद्यत
मुखतस्तालु निर्भिन्नंजिह्वा तत्रोपजायते
ततो नानारसो जज्ञे जिह्वया योऽधिगम्यते
विवक्षोर्मुखतो भूम्नो वह्निर्वाग्व्याहृतं तयोः
जले चैतस्य सुचिरं निरोधः समजायत
नासिके निरभिद्येतां दोधूयति नभस्वति
तत्र वायुर्गन्धवहो घ्राणो नसि जिघृक्षतः
यदात्मनि निरालोकमात्मानं च दिदृक्षतः
निर्भिन्ने ह्यक्षिणी तस्य ज्योतिश्चक्षुर्गुणग्रहः
बोध्यमानस्य ऋषिभिरात्मनस्तज्जिघृक्षतः
कर्णौ च निरभिद्येतां दिशः श्रोत्रं गुणग्रहः
वस्तुनो मृदुकाठिन्य लघुगुर्वोष्णशीतताम्‌
जिघृक्षतस्त्वङ्निर्भिन्ना तस्यां रोममहीरुहाः
तत्र चान्तर्बहिर्वातस्त्वचा लब्धगुणो वृतः
हस्तौ रुरुहतुस्तस्य नानाकर्मचिकीर्षया
तयोस्तु बलवानिन्द्र आदानमुभयाश्रयम्‌
गतिं जिगीषतः पादौ रुरुहातेऽभिकामिकाम्‌
पद्भ्यां यज्ञः स्वयं हव्यं कर्मभिः क्रियते नृभिः
निरभिद्यत शिश्नो वै प्रजानन्दामृतार्थिनः
उपस्थ आसीत्कामानां प्रियं तदुभयाश्रयम्‌
उत्सिसृक्षोर्धातुमलं निरभिद्यत वै गुदम्‌
ततः पायुस्ततो मित्र उत्सर्ग उभयाश्रयः
आसिसृप्सोः पुरः पुर्या नाभिद्वारमपानतः
तत्रापानस्ततो मृत्युः पृथक्त्वमुभयाश्रयम्‌
आदित्सोरन्नपानानामासन्कुक्ष्यन्त्रनाडयः
नद्यः समुद्राश्च तयोस्तुष्टिः पुष्टिस्तदाश्रये
निदिध्यासोरात्ममायां हृदयं निरभिद्यत
ततो मनश्चन्द्र इति सङ्कल्पः काम एव च
त्वक्चर्ममांसरुधिर मेदोमज्जास्थिधातवः
भूम्यप्तेजोमयाः सप्त प्राणो व्योमाम्बुवायुभिः
गुणात्मकानीन्द्रियाणि भूतादिप्रभवा गुणाः
मनः सर्वविकारात्मा बुद्धिर्विज्ञानरूपिणी
एतद्भगवतो रूपं स्थूलं ते व्याहृतं मया
मह्यादिभिश्चावरणैरष्टभिर्बहिरावृतम्‌
अतः परं सूक्ष्मतममव्यक्तं निर्विशेषणम्‌
अनादिमध्यनिधनं नित्यं वाङ्मनसः परम्‌
अमुनी भगवद्रूपे मया ते ह्यनुवर्णिते
उभे अपि न गृह्णन्ति मायासृष्टे विपश्चितः
स वाच्यवाचकतया भगवान्ब्रह्मरूपधृक्‌
नामरूपक्रिया धत्ते सकर्माकर्मकः परः
प्रजापतीन्मनून्देवानृषीन्पितृगणान्पृथक्‌
सिद्धचारणगन्धर्वान्विद्याध्रासुरगुह्यकान्‌
किन्नराप्सरसो नागान्सर्पान्किम्पुरुषान्नरान्‌
मात्रक्षःपिशाचांश्च प्रेतभूतविनायकान्‌
कूष्माण्डोन्मादवेतालान्यातुधानान्ग्रहानपि
खगान्मृगान्पशून्वृक्षान्गिरीन्नृप सरीसृपान्‌
द्विविधाश्चतुर्विधा येऽन्ये जलस्थलनभौकसः
कुशलाकुशला मिश्राः कर्मणां गतयस्त्विमाः
सत्त्वं रजस्तम इति तिस्रः सुरनृनारकाः
तत्राप्येकैकशो राजन्भिद्यन्ते गतयस्त्रिधा
यदैकैकतरोऽन्याभ्यां स्वभाव उपहन्यते
स एवेदं जगद्धाता भगवान्धर्मरूपधृक्‌
पुष्णाति स्थापयन्विश्वं तिर्यङ्नरसुरादिभिः
ततः कालाग्निरुद्रात्मा यत्सृष्टमिदमात्मनः
सन्नियच्छति तत्काले घनानीकमिवानिलः
इत्थम्भावेन कथितो भगवान्भगवत्तमः
नेत्थम्भावेन हि परं द्रष्टुमर्हन्ति सूरयः
नास्य कर्मणि जन्मादौ परस्यानुविधीयते
कर्तृत्वप्रतिषेधार्थं माययारोपितं हि तत्‌
अयं तु ब्रह्मणः कल्पः सविकल्प उदाहृतः
विधिः साधारणो यत्र सर्गाः प्राकृतवैकृताः
परिमाणं च कालस्य कल्पलक्षणविग्रहम्‌
यथा पुरस्ताद्व्याख्यास्ये पाद्मं कल्पमथो शृणु
शौनक उवाच
यदाह नो भवान्सूत क्षत्ता भागवतोत्तमः
चचार तीर्थानि भुवस्त्यक्त्वा बन्धून्सुदुस्त्यजान्‌
क्षत्तुः कौशारवेस्तस्य संवादोऽध्यात्मसंश्रितः
यद्वा स भगवांस्तस्मै पृष्टस्तत्त्वमुवाच ह
ब्रूहि नस्तदिदं सौम्य विदुरस्य विचेष्टितम्‌
बन्धुत्यागनिमित्तं च यथैवागतवान्पुनः
सूत उवाच
राज्ञा परीक्षिता पृष्टो यदवोचन्महामुनिः
तद्वोऽभिधास्ये शृणुत राअज्ञः प्रश्नानुसारतः
श्रीशुक उवाच
एवमेतत्पुरा पृष्टो मैत्रेयो भगवान्किल
क्षत्त्रा वनं प्रविष्टेन त्यक्त्वा स्वगृहमृद्धिमत्‌
यद्वा अयं मन्त्रकृद्वो भगवानखिलेश्वरः
पौरवेन्द्रगृहं हित्वा प्रविवेशात्मसात्कृतम्‌
राजोवाच
कुत्र क्षत्तुर्भगवता मैत्रेयेणास सङ्गमः
कदा वा सहसंवाद एतद्वर्णय नः प्रभो
न ह्यल्पार्थोदयस्तस्य विदुरस्यामलात्मनः
तस्मिन्वरीयसि प्रश्नः साधुवादोपबृंहितः
सूत उवाच
स एवमृषिवर्योऽयं पृष्टो राज्ञा परीक्षिता
प्रत्याह तं सुबहुवित्प्रीतात्मा श्रूयतामिति
श्रीशुक उवाच
यदा तु राजा स्वसुतानसाधून्पुष्णन्न धर्मेण विनष्टदृष्टिः
भ्रातुर्यविष्ठस्य सुतान्विबन्धून्प्रवेश्य लाक्षाभवने ददाह
यदा सभायां कुरुदेवदेव्याः केशाभिमर्शं सुतकर्म गर्ह्यम्‌
न वारयामास नृपः स्नुषायाः स्वास्रैर्हरन्त्याः कुचकुङ्कुमानि
द्यूते त्वधर्मेण जितस्य साधोः सत्यावलम्बस्य वनं गतस्य
न याचतोऽदात्समयेन दायं तमोजुषाणो यदजातशत्रोः
यदा च पार्थप्रहितः सभायां जगद्गुरुर्यानि जगाद कृष्णः
न तानि पुंसाममृतायनानि राजोरु मेने क्षतपुण्यलेशः
यदोपहूतो भवनं प्रविष्टो मन्त्राय पृष्टः किल पूर्वजेन
अथाह तन्मन्त्रदृशां वरीयान्यन्मन्त्रिणो वैदुरिकं वदन्ति
अजातशत्रोः प्रतियच्छ दायं तितिक्षतो दुर्विषहं तवागः
सहानुजो यत्र वृकोदराहिः श्वसन्रुषा यत्त्वमलं बिभेषि
पार्थांस्तु देवो भगवान्मुकुन्दो गृहीतवान्सक्षितिदेवदेवः
आस्ते स्वपुर्यां यदुदेवदेवो विनिर्जिताशेषनृदेवदेवः
स एष दोषः पुरुषद्विडास्ते गृहान्प्रविष्टो यमपत्यमत्या
पुष्णासि कृष्णाद्विमुखो गतश्रीस्त्यजाश्वशैवं कुलकौशलाय
इत्यूचिवांस्तत्र सुयोधनेन प्रवृद्धकोपस्फुरिताधरेण
असत्कृतः सत्स्पृहणीयशीलः क्षत्ता सकर्णानुजसौबलेन
क एनमत्रोपजुहाव जिह्मं दास्याः सुतं यद्बलिनैव पुष्टः
तस्मिन्प्रतीपः परकृत्य आस्ते निर्वास्यतामाशु पुराच्छ्वसानः
स्वयं धनुर्द्वारि निधाय मायां भ्रातुः पुरो मर्मसु ताडितोऽपि
स इत्थमत्युल्बणकर्णबाणैर्गतव्यथोऽयादुरु मानयानः
स निर्गतः कौरवपुण्यलब्धो गजाह्वयात्तीर्थपदः पदानि
अन्वाक्रमत्पुण्यचिकीर्षयोर्व्यामधिष्ठितो यानि सहस्रमूर्तिः
पुरेषु पुण्योपवनाद्रिकुञ्जेष्वपङ्कतोयेषु सरित्सरःसु
अनन्तलिङ्गैः समलङ्कृतेषु चचार तीर्थायतनेष्वनन्यः
गां पर्यटन्मेध्यविविक्तवृत्तिः सदाप्लुतोऽधः शयनोऽवधूतः
अलक्षितः स्वैरवधूतवेषो व्रतानि चेरे हरितोषणानि
इत्थं व्रजन्भारतमेव वर्षं कालेन यावद्गतवान्प्रभासम्‌
तावच्छशास क्षितिमेक चक्राम्लेकातपत्रामजितेन पार्थः
तत्राथ शुश्राव सुहृद्विनष्टिं वनं यथा वेणुजवह्निसंश्रयम्‌
संस्पर्धया दग्धमथानुशोचन्सरस्वतीं प्रत्यगियाय तूष्णीम्‌
तस्यां त्रितस्योशनसो मनोश्च पृथोरथाग्नेरसितस्य वायोः
तीर्थं सुदासस्य गवां गुहस्य यच्छ्राद्धदेवस्य स आसिषेवे
अन्यानि चेह द्विजदेवदेवैः कृतानि नानायतनानि विष्णोः
प्रत्यङ्गमुख्याङ्कितमन्दिराणि यद्दर्शनात्कृष्णमनुस्मरन्ति
ततस्त्वतिव्रज्य सुराष्ट्रमृद्धं सौवीरमत्स्यान्कुरुजाङ्गलांश्च
कालेन तावद्यमुनामुपेत्य तत्रोद्धवं भागवतं ददर्श
स वासुदेवानुचरं प्रशान्तं बृहस्पतेः प्राक्तनयं प्रतीतम्‌
आलिङ्ग्य गाढं प्रणयेन भद्रं स्वानामपृच्छद्भगवत्प्रजानाम्‌
कच्चित्पुराणौ पुरुषौ स्वनाभ्य पाद्मानुवृत्त्येह किलावतीर्णौ
आसात उर्व्याः कुशलं विधाय कृतक्षणौ कुशलं शूरगेहे
कच्चित्कुरूणां परमः सुहृन्नो भामः स आस्ते सुखमङ्ग शौरिः
यो वै स्वस्णां पितृवद्ददाति वरान्वदान्यो वरतर्पणेन
कच्चिद्वरूथाधिपतिर्यदूनां प्रद्युम्न आस्ते सुखमङ्ग वीरः
यं रुक्मिणी भगवतोऽभिलेभे आराध्य विप्रान्स्मरमादिसर्गे
कच्चित्सुखं सात्वतवृष्णिभोज दाशार्हकाणामधिपः स आस्ते
यमभ्यषिञ्चच्छतपत्रनेत्रो नृपासनाशां परिहृत्य दूरात्‌
कच्चिद्धरेः सौम्य सुतः सदृक्ष आस्तेऽग्रणी रथिनां साधु साम्बः
असूत यं जाम्बवती व्रताढ्या देवं गुहं योऽम्बिकया धृतोऽग्रे
क्षेमं स कच्चिद्युयुधान आस्ते यः फाल्गुनाल्लब्धधनूरहस्यः
लेभेऽञ्जसाधोक्षजसेवयैव गतिं तदीयां यतिभिर्दुरापाम्‌
कच्चिद्बुधः स्वस्त्यनमीव आस्ते श्वफल्कपुत्रो भगवत्प्रपन्नः
यः कृष्णपादाङ्कितमार्गपांसुष्वचेष्टत प्रेमविभिन्नधैर्यः
कच्चिच्छिवं देवकभोजपुत्र्या विष्णुप्रजाया इव देवमातुः
या वै स्वगर्भेण दधार देवं त्रयी यथा यज्ञवितानमर्थम्‌
अपिस्विदास्ते भगवान्सुखं वो यः सात्वतां कामदुघोऽनिरुद्धः
यमामनन्ति स्म हि शब्दयोनिं मनोमयं सत्त्वतुरीयतत्त्वम्‌
अपिस्विदन्ये च निजात्मदैवमनन्यवृत्त्या समनुव्रता ये
हृदीकसत्यात्मजचारुदेष्ण गदादयः स्वस्ति चरन्ति सौम्य
अपि स्वदोर्भ्यां विजयाच्युताभ्यां धर्मेण धर्मः परिपाति सेतुम्‌
दुर्योधनोऽतप्यत यत्सभायां साम्राज्यलक्ष्म्या विजयानुवृत्त्या
किं वा कृताघेष्वघमत्यमर्षी भीमोऽहिवद्दीर्घतमं व्यमुञ्चत्‌
यस्याङ्घ्रिपातं रणभूर्न सेहे मार्गं गदायाश्चरतो विचित्रम्‌
कच्चिद्यशोधा रथयूथपानां गाण्डीवधन्वोपरतारिरास्ते
अलक्षितो यच्छरकूटगूढो मायाकिरातो गिरिशस्तुतोष
यमावुतस्वित्तनयौ पृथायाः पार्थैर्वृतौ पक्ष्मभिरक्षिणीव
रेमात उद्दाय मृधे स्वरिक्थं परात्सुपर्णाविव वज्रिवक्त्रात्‌
अहो पृथापि ध्रियतेऽर्भकार्थे राजर्षिवर्येण विनापि तेन
यस्त्वेकवीरोऽधिरथो विजिग्ये धनुर्द्वितीयः ककुभश्चतस्रः
सौम्यानुशोचे तमधःपतन्तं भ्रात्रे परेताय विदुद्रुहे यः
निर्यापितो येन सुहृत्स्वपुर्या अहं स्वपुत्रान्समनुव्रतेन
सोऽहं हरेर्मर्त्यविडम्बनेन दृशो नृणां चालयतो विधातुः
नान्योपलक्ष्यः पदवीं प्रसादाच्चरामि पश्यन्गतविस्मयोऽत्र
नूनं नृपाणां त्रिमदोत्पथानां महीं मुहुश्चालयतां चमूभिः
वधात्प्रपन्नार्तिजिहीर्षयेशोऽप्युपैक्षताघं भगवान्कुरूणाम्‌
अजस्य जन्मोत्पथनाशनाय कर्माण्यकर्तुर्ग्रहणाय पुंसाम्‌
नन्वन्यथा कोऽर्हति देहयोगं परो गुणानामुत कर्मतन्त्रम्‌
तस्य प्रपन्नाखिललोकपानामवस्थितानामनुशासने स्वे
अर्थाय जातस्य यदुष्वजस्य वार्तां सखे कीर्तय तीर्थकीर्तेः
श्रीशुक उवाच
इति भागवतः पृष्टः क्षत्त्रा वार्तां प्रियाश्रयाम्‌
प्रतिवक्तुं न चोत्सेह औत्कण्ठ्यात्स्मारितेश्वरः
यः पञ्चहायनो मात्रा प्रातराशाय याचितः
तन्नैच्छद्रचयन्यस्य सपर्यां बाललीलया
स कथं सेवया तस्य कालेन जरसं गतः
पृष्टो वार्तां प्रतिब्रूयाद्भर्तुः पादावनुस्मरन्‌
स मुहूर्तमभूत्तूष्णीं कृष्णाङ्घ्रिसुधया भृशम्‌
तीव्रेण भक्तियोगेन निमग्नः साधु निर्वृतः
पुलकोद्भिन्नसर्वाङ्गो मुञ्चन्मीलद्दृशा शुचः
पूर्णार्थो लक्षितस्तेन स्नेहप्रसरसम्प्लुतः
शनकैर्भगवल्लोकान्नृलोकं पुनरागतः
विमृज्य नेत्रे विदुरं प्रीत्याहोद्धव उत्स्मयन्‌
उद्धव उवाच
कृष्णद्युमणि निम्लोचे गीर्णेष्वजगरेण ह
किं नु नः कुशलं ब्रूयां गतश्रीषु गृहेष्वहम्‌
दुर्भगो बत लोकोऽयं यदवो नितरामपि
ये संवसन्तो न विदुर्हरिं मीना इवोडुपम्‌
इङ्गितज्ञाः पुरुप्रौढा एकारामाश्च सात्वताः
सात्वतामृषभं सर्वे भूतावासममंसत
देवस्य मायया स्पृष्टा ये चान्यदसदाश्रिताः
भ्राम्यते धीर्न तद्वाक्यैरात्मन्युप्तात्मनो हरौ
प्रदर्श्यातप्ततपसामवितृप्तदृशां नृणाम्‌
आदायान्तरधाद्यस्तु स्वबिम्बं लोकलोचनम्‌
यन्मर्त्यलीलौपयिकं स्वयोग मायाबलं दर्शयता गृहीतम्‌
विस्मापनं स्वस्य च सौभगर्द्धेः परं पदं भूषणभूषणाङ्गम्‌
यद्धर्मसूनोर्बत राजसूये निरीक्ष्य दृक्स्वस्त्ययनं त्रिलोकः
कार्त्स्न्येन चाद्येह गतं विधातुरर्वाक्सृतौ कौशलमित्यमन्यत
यस्यानुरागप्लुतहासरास लीलावलोकप्रतिलब्धमानाः
व्रजस्त्रियो दृग्भिरनुप्रवृत्त धियोऽवतस्थुः किल कृत्यशेषाः
स्वशान्तरूपेष्वितरैः स्वरूपैरभ्यर्द्यमानेष्वनुकम्पितात्मा
परावरेशो महदंशयुक्तो ह्यजोऽपि जातो भगवान्यथाग्निः
मां खेदयत्येतदजस्य जन्म विडम्बनं यद्वसुदेवगेहे
व्रजे च वासोऽरिभयादिव स्वयं पुराद्व्यवात्सीद्यदनन्तवीर्यः
दुनोति चेतः स्मरतो ममैतद्यदाह पादावभिवन्द्य पित्रोः
ताताम्ब कंसादुरुशङ्कितानां प्रसीदतं नोऽकृतनिष्कृतीनाम्‌
को वा अमुष्याङ्घ्रिसरोजरेणुं विस्मर्तुमीशीत पुमान्विजिघ्रन्‌
यो विस्फुरद्भ्रूविटपेन भूमेर्भारं कृतान्तेन तिरश्चकार
दृष्टा भवद्भिर्ननु राजसूये चैद्यस्य कृष्णं द्विषतोऽपि सिद्धिः
यां योगिनः संस्पृहयन्ति सम्यग्योगेन कस्तद्विरहं सहेत
तथैव चान्ये नरलोकवीरा य आहवे कृष्णमुखारविन्दम्‌
नेत्रैः पिबन्तो नयनाभिरामं पार्थास्त्रपूतः पदमापुरस्य
स्वयं त्वसाम्यातिशयस्त्र्यधीशः स्वाराज्यलक्ष्म्याप्तसमस्तकामः
बलिं हरद्भिश्चिरलोकपालैः किरीटकोट्येडितपादपीठः
तत्तस्य कैङ्कर्यमलं भृतान्नो विग्लापयत्यङ्ग यदुग्रसेनम्‌
तिष्ठन्निषण्णं परमेष्ठिधिष्ण्ये न्यबोधयद्देव निधारयेति
अहो बकी यं स्तनकालकूटं जिघांसयापाययदप्यसाध्वी
लेभे गतिं धात्र्युचितां ततोऽन्यं कं वा दयालुं शरणं व्रजेम
मन्येऽसुरान्भागवतांस्त्र्यधीशे संरम्भमार्गाभिनिविष्टचित्तान्‌
ये संयुगेऽचक्षत तार्क्ष्यपुत्रमंसे सुनाभायुधमापतन्तम्‌
वसुदेवस्य देवक्यां जातो भोजेन्द्रबन्धने
चिकीर्षुर्भगवानस्याः शमजेनाभियाचितः
ततो नन्दव्रजमितः पित्रा कंसाद्विबिभ्यता
एकादश समास्तत्र गूढार्चिः सबलोऽवसत्‌
परीतो वत्सपैर्वत्सांश्चारयन्व्यहरद्विभुः
यमुनोपवने कूजद्‌ द्विजसङ्कुलिताङ्घ्रिपे
कौमारीं दर्शयंश्चेष्टां प्रेक्षणीयां व्रजौकसाम्‌
रुदन्निव हसन्मुग्ध बालसिंहावलोकनः
स एव गोधनं लक्ष्म्या निकेतं सितगोवृषम्‌
चारयन्ननुगान्गोपान्रणद्वेणुररीरमत्‌
प्रयुक्तान्भोजराजेन मायिनः कामरूपिणः
लीलया व्यनुदत्तांस्तान्बालः क्रीडनकानिव
विपन्नान्विषपानेन निगृह्य भुजगाधिपम्‌
उत्थाप्यापाययद्गावस्तत्तोयं प्रकृतिस्थितम्‌
अयाजयद्गोसवेन गोपराजं द्विजोत्तमैः
वित्तस्य चोरुभारस्य चिकीर्षन्सद्व्ययं विभुः
वर्षतीन्द्रे व्रजः कोपाद्भग्नमानेऽतिविह्वलः
गोत्रलीलातपत्रेण त्रातो भद्रानुगृह्णता
शरच्छशिकरैर्मृष्टं मानयन्रजनीमुखम्‌
गायन्कलपदं रेमे स्त्रीणां मण्डलमण्डनः
उद्धव उवाच
ततः स आगत्य पुरं स्वपित्रोश्चिकीर्षया शं बलदेवसंयुतः
निपात्य तुङ्गाद्रिपुयूथनाथं हतं व्यकर्षद्व्यसुमोजसोर्व्याम्‌
सान्दीपनेः सकृत्प्रोक्तं ब्रह्माधीत्य सविस्तरम्‌
तस्मै प्रादाद्वरं पुत्रं मृतं पञ्चजनोदरात्‌
समाहुता भीष्मककन्यया ये श्रियः सवर्णेन बुभूषयैषाम्‌
गान्धर्ववृत्त्या मिषतां स्वभागं जह्रे पदं मूर्ध्नि दधत्सुपर्णः
ककुद्मिनोऽविद्धनसो दमित्वा स्वयंवरे नाग्नजितीमुवाह
तद्भग्नमानानपि गृध्यतोऽज्ञाञ्जघ्नेऽक्षतः शस्त्रभृतः स्वशस्त्रैः
प्रियं प्रभुर्ग्राम्य इव प्रियाया विधित्सुरार्च्छद्द्युतरुं यदर्थे
वज्र्याद्रवत्तं सगणो रुषान्धः क्रीडामृगो नूनमयं वधूनाम्‌
सुतं मृधे खं वपुषा ग्रसन्तं दृष्ट्वा सुनाभोन्मथितं धरित्र्या
आमन्त्रितस्तत्तनयाय शेषं दत्त्वा तदन्तःपुरमाविवेश
तत्राहृतास्ता नरदेवकन्याः कुजेन दृष्ट्वा हरिमार्तबन्धुम्‌
उत्थाय सद्यो जगृहुः प्रहर्ष व्रीडानुरागप्रहितावलोकैः
आसां मुहूर्त एकस्मिन्नानागारेषु योषिताम्‌
सविधं जगृहे पाणीननुरूपः स्वमायया
तास्वपत्यान्यजनयदात्मतुल्यानि सर्वतः
एकैकस्यां दश दश प्रकृतेर्विबुभूषया
कालमागधशाल्वादीननीकै रुन्धतः पुरम्‌
अजीघनत्स्वयं दिव्यं स्वपुंसां तेज आदिशत्‌
शम्बरं द्विविदं बाणं मुरं बल्वलमेव च
अन्यांश्च दन्तवक्रादीनवधीत्कांश्च घातयत्‌
अथ ते भ्रातृपुत्राणां पक्षयोः पतितान्नृपान्‌
चचाल भूः कुरुक्षेत्रं येषामापततां बलैः
स कर्णदुःशासनसौबलानां कुमन्त्रपाकेन हतश्रियायुषम्‌
सुयोधनं सानुचरं शयानं भग्नोरुमूर्व्यां न ननन्द पश्यन्‌
कियान्भुवोऽयं क्षपितोरुभारो यद्द्रोणभीष्मार्जुनभीममूलैः
अष्टादशाक्षौहिणिको मदंशैरास्ते बलं दुर्विषहं यदूनाम्‌
मिथो यदैषां भविता विवादो मध्वामदाताम्रविलोचनानाम्‌
नैषां वधोपाय इयानतोऽन्यो मय्युद्यतेऽन्तर्दधते स्वयं स्म
एवं सञ्चिन्त्य भगवान्स्वराज्ये स्थाप्य धर्मजम्‌
नन्दयामास सुहृदः साधूनां वर्त्म दर्शयन्‌
उत्तरायां धृतः पूरोर्वंशः साध्वभिमन्युना
स वै द्रौण्यस्त्रसम्प्लुष्टः पुनर्भगवता धृतः
अयाजयद्धर्मसुतमश्वमेधैस्त्रिभिर्विभुः
सोऽपि क्ष्मामनुजै रक्षन्रेमे कृष्णमनुव्रतः
भगवानपि विश्वात्मा लोकवेदपथानुगः
कामान्सिषेवे द्वार्वत्यामसक्तः साङ्ख्यमास्थितः
स्निग्धस्मितावलोकेन वाचा पीयूषकल्पया
चरित्रेणानवद्येन श्रीनिकेतेन चात्मना
इमं लोकममुं चैव रमयन्सुतरां यदून्‌
रेमे क्षणदया दत्त क्षणस्त्रीक्षणसौहृदः
तस्यैवं रममाणस्य संवत्सरगणान्बहून्‌
गृहमेधेषु योगेषु विरागः समजायत
दैवाधीनेषु कामेषु दैवाधीनः स्वयं पुमान्‌
को विश्रम्भेत योगेन योगेश्वरमनुव्रतः
पुर्यां कदाचित्क्रीडद्भिर्यदुभोजकुमारकैः
कोपिता मुनयः शेपुर्भगवन्मतकोविदाः
ततः कतिपयैर्मासैर्वृष्णिभोजान्धकादयः
ययुः प्रभासं संहृष्टा रथैर्देवविमोहिताः
तत्र स्नात्वा पित्न्देवानृषींश्चैव तदम्भसा
तर्पयित्वाथ विप्रेभ्यो गावो बहुगुणा ददुः
हिरण्यं रजतं शय्यां वासांस्यजिनकम्बलान्‌
यानं रथानिभान्कन्या धरां वृत्तिकरीमपि
अन्नं चोरुरसं तेभ्यो दत्त्वा भगवदर्पणम्‌
गोविप्रार्थासवः शूराः प्रणेमुर्भुवि मूर्धभिः
उद्धव उवाच
अथ ते तदनुज्ञाता भुक्त्वा पीत्वा च वारुणीम्‌
तया विभ्रंशितज्ञाना दुरुक्तैर्मर्म पस्पृशुः
तेषां मैरेयदोषेण विषमीकृतचेतसाम्‌
निम्लोचति रवावासीद्वेणूनामिव मर्दनम्‌
भगवान्स्वात्ममायाया गतिं तामवलोक्य सः
सरस्वतीमुपस्पृश्य वृक्षमूलमुपाविशत्‌
अहं चोक्तो भगवता प्रपन्नार्तिहरेण ह
बदरीं त्वं प्रयाहीति स्वकुलं सञ्जिहीर्षुणा
तथापि तदभिप्रेतं जानन्नहमरिन्दम
पृष्ठतोऽन्वगमं भर्तुः पादविश्लेषणाक्षमः
अद्राक्षमेकमासीनं विचिन्वन्दयितं पतिम्‌
श्रीनिकेतं सरस्वत्यां कृतकेतमकेतनम्‌
श्यामावदातं विरजं प्रशान्तारुणलोचनम्‌
दोर्भिश्चतुर्भिर्विदितं पीतकौशाम्बरेण च
वाम ऊरावधिश्रित्य दक्षिणाङ्घ्रिसरोरुहम्‌
अपाश्रितार्भकाश्वत्थमकृशं त्यक्तपिप्पलम्‌
तस्मिन्महाभागवतो द्वैपायनसुहृत्सखा
लोकाननुचरन्सिद्ध आससाद यदृच्छया
तस्यानुरक्तस्य मुनेर्मुकुन्दः प्रमोदभावानतकन्धरस्य
आशृण्वतो मामनुरागहास समीक्षया विश्रमयन्नुवाच
श्रीभगवानुवाच
वेदाहमन्तर्मनसीप्सितं ते ददामि यत्तद्दुरवापमन्यैः
सत्रे पुरा विश्वसृजां वसूनां मत्सिद्धिकामेन वसो त्वयेष्टः
स एष साधो चरमो भवानामासादितस्ते मदनुग्रहो यत्‌
यन्मां नृलोकान्रह उत्सृजन्तं दिष्ट्या ददृश्वान्विशदानुवृत्त्या
पुरा मया प्रोक्तमजाय नाभ्ये पद्मे निषण्णाय ममादिसर्गे
ज्ञानं परं मन्महिमावभासं यत्सूरयो भागवतं वदन्ति
इत्यादृतोक्तः परमस्य पुंसः प्रतिक्षणानुग्रहभाजनोऽहम्‌
स्नेहोत्थरोमा स्खलिताक्षरस्तं मुञ्चञ्छुचः प्राञ्जलिराबभाषे
को न्वीश ते पादसरोजभाजां सुदुर्लभोऽर्थेषु चतुर्ष्वपीह
तथापि नाहं प्रवृणोमि भूमन्भवत्पदाम्भोजनिषेवणोत्सुकः
कर्माण्यनीहस्य भवोऽभवस्य ते दुर्गाश्रयोऽथारिभयात्पलायनम्‌
कालात्मनो यत्प्रमदायुताश्रमः स्वात्मन्रतेः खिद्यति धीर्विदामिह
मन्त्रेषु मां वा उपहूय यत्त्वमकुण्ठिताखण्डसदात्मबोधः
पृच्छेः प्रभो मुग्ध इवाप्रमत्तस्तन्नो मनो मोहयतीव देव
ज्ञानं परं स्वात्मरहःप्रकाशं प्रोवाच कस्मै भगवान्समग्रम्‌
अपि क्षमं नो ग्रहणाय भर्तर्वदाञ्जसा यद्वृजिनं तरेम
इत्यावेदितहार्दाय मह्यं स भगवान्परः
आदिदेशारविन्दाक्ष आत्मनः परमां स्थितिम्‌
स एवमाराधितपादतीर्थादधीततत्त्वात्मविबोधमार्गः
प्रणम्य पादौ परिवृत्य देवमिहागतोऽहं विरहातुरात्मा
सोऽहं तद्दर्शनाह्लाद वियोगार्तियुतः प्रभो
गमिष्ये दयितं तस्य बदर्याश्रममण्डलम्‌
यत्र नारायणो देवो नरश्च भगवानृषिः
मृदु तीव्रं तपो दीर्घं तेपाते लोकभावनौ
श्रीशुक उवाच
इत्युद्धवादुपाकर्ण्य सुहृदां दुःसहं वधम्‌
ज्ञानेनाशमयत्क्षत्ता शोकमुत्पतितं बुधः
स तं महाभागवतं व्रजन्तं कौरवर्षभः
विश्रम्भादभ्यधत्तेदं मुख्यं कृष्णपरिग्रहे
विदुर उवाच
ज्ञानं परं स्वात्मरहःप्रकाशं यदाह योगेश्वर ईश्वरस्ते
वक्तुं भवान्नोऽर्हति यद्धि विष्णोर्भृत्याः स्वभृत्यार्थकृतश्चरन्ति
उद्धव उवाच
ननु ते तत्त्वसंराध्य ऋषिः कौषारवोऽन्तिके
साक्षाद्भगवतादिष्टो मर्त्यलोकं जिहासता
श्रीशुक उवाच
इति सह विदुरेण विश्वमूर्तेर्गुणकथया सुधया प्लावितोरुतापः
क्षणमिव पुलिने यमस्वसुस्तां समुषित औपगविर्निशां ततोऽगात्‌
राजोवाच
निधनमुपगतेषु वृष्णिभोजेष्वधिरथयूथपयूथपेषु मुख्यः
स तु कथमवशिष्ट उद्धवो यद्धरिरपि तत्यज आकृतिं त्र्यधीशः
श्रीशुक उवाच
ब्रह्मशापापदेशेन कालेनामोघवाञ्छितः
संहृत्य स्वकुलं स्फीतं त्यक्ष्यन्देहमचिन्तयत्‌
अस्माल्लोकादुपरते मयि ज्ञानं मदाश्रयम्‌
अर्हत्युद्धव एवाद्धा सम्प्रत्यात्मवतां वरः
नोद्धवोऽण्वपि मन्न्यूनो यद्गुणैर्नार्दितः प्रभुः
अतो मद्वयुनं लोकं ग्राहयन्निह तिष्ठतु
एवं त्रिलोकगुरुणा सन्दिष्टः शब्दयोनिना
बदर्याश्रममासाद्य हरिमीजे समाधिना
विदुरोऽप्युद्धवाच्छ्रुत्वा कृष्णस्य परमात्मनः
क्रीडयोपात्तदेहस्य कर्माणि श्लाघितानि च
देहन्यासं च तस्यैवं धीराणां धैर्यवर्धनम्‌
अन्येषां दुष्करतरं पशूनां विक्लवात्मनाम्‌
आत्मानं च कुरुश्रेष्ठ कृष्णेन मनसेक्षितम्‌
ध्यायन्गते भागवते रुरोद प्रेमविह्वलः
कालिन्द्याः कतिभिः सिद्ध अहोभिर्भरतर्षभ
प्रापद्यत स्वःसरितं यत्र मित्रासुतो मुनिः
श्रीशुक उवाच
द्वारि द्युनद्या ऋषभः कुरूणां मैत्रेयमासीनमगाधबोधम्‌
क्षत्तोपसृत्याच्युतभावसिद्धः पप्रच्छ सौशील्यगुणाभितृप्तः
विदुर उवाच
सुखाय कर्माणि करोति लोको न तैः सुखं वान्यदुपारमं वा
विन्देत भूयस्तत एव दुःखं यदत्र युक्तं भगवान्वदेन्नः
जनस्य कृष्णाद्विमुखस्य दैवादधर्मशीलस्य सुदुःखितस्य
अनुग्रहायेह चरन्ति नूनं भूतानि भव्यानि जनार्दनस्य
तत्साधुवर्यादिश वर्त्म शं नः संराधितो भगवान्येन पुंसाम्‌
हृदि स्थितो यच्छति भक्तिपूते ज्ञानं सतत्त्वाधिगमं पुराणम्‌
करोति कर्माणि कृतावतारो यान्यात्मतन्त्रो भगवांस्त्र्यधीशः
यथा ससर्जाग्र इदं निरीहः संस्थाप्य वृत्तिं जगतो विधत्ते
यथा पुनः स्वे ख इदं निवेश्य शेते गुहायां स निवृत्तवृत्तिः
योगेश्वराधीश्वर एक एतदनुप्रविष्टो बहुधा यथासीत्‌
क्रीडन्विधत्ते द्विजगोसुराणां क्षेमाय कर्माण्यवतारभेदैः
मनो न तृप्यत्यपि शृण्वतां नः सुश्लोकमौलेश्चरितामृतानि
यैस्तत्त्वभेदैरधिलोकनाथो लोकानलोकान्सह लोकपालान्‌
अचीकॢपद्यत्र हि सर्वसत्त्व निकायभेदोऽधिकृतः प्रतीतः
येन प्रजानामुत आत्मकर्म रूपाभिधानां च भिदां व्यधत्त
नारायणो विश्वसृगात्मयोनिरेतच्च नो वर्णय विप्रवर्य
परावरेषां भगवन्व्रतानि श्रुतानि मे व्यासमुखादभीक्ष्णम्‌
अतृप्नुम क्षुल्लसुखावहानां तेषामृते कृष्णकथामृतौघात्‌
कस्तृप्नुयात्तीर्थपदोऽभिधानात्सत्रेषु वः सूरिभिरीड्यमानात्‌
यः कर्णनाडीं पुरुषस्य यातो भवप्रदां गेहरतिं छिनत्ति
मुनिर्विवक्षुर्भगवद्गुणानां सखापि ते भारतमाह कृष्णः
यस्मिन्नृणां ग्राम्यसुखानुवादैर्मतिर्गृहीता नु हरेः कथायाम्‌
सा श्रद्दधानस्य विवर्धमाना विरक्तिमन्यत्र करोति पुंसः
हरेः पदानुस्मृतिनिर्वृतस्य समस्तदुःखाप्ययमाशु धत्ते
ताञ्छोच्यशोच्यानविदोऽनुशोचे हरेः कथायां विमुखानघेन
क्षिणोति देवोऽनिमिषस्तु येषामायुर्वृथावादगतिस्मृतीनाम्‌
तदस्य कौषारव शर्मदातुर्हरेः कथामेव कथासु सारम्‌
उद्धृत्य पुष्पेभ्य इवार्तबन्धो शिवाय नः कीर्तय तीर्थकीर्तेः
स विश्वजन्मस्थितिसंयमार्थे कृतावतारः प्रगृहीतशक्तिः
चकार कर्माण्यतिपूरुषाणि यानीश्वरः कीर्तय तानि मह्यम्‌
श्रीशुक उवाच
स एवं भगवान्पृष्टः क्षत्त्रा कौषारवो मुनिः
पुंसां निःश्रेयसार्थेन तमाह बहुमानयन्‌
मैत्रेय उवाच
साधु पृष्टं त्वया साधो लोकान्साध्वनुगृह्णता
कीर्तिं वितन्वता लोके आत्मनोऽधोक्षजात्मनः
नैतच्चित्रं त्वयि क्षत्तर्बादरायणवीर्यजे
गृहीतोऽनन्यभावेन यत्त्वया हरिरीश्वरः
माण्डव्यशापाद्भगवान्प्रजासंयमनो यमः
भ्रातुः क्षेत्रे भुजिष्यायां जातः सत्यवतीसुतात्‌
भवान्भगवतो नित्यं सम्मतः सानुगस्य ह
यस्य ज्ञानोपदेशाय मादिशद्भगवान्व्रजन्‌
अथ ते भगवल्लीला योगमायोरुबृंहिताः
विश्वस्थित्युद्भवान्तार्था वर्णयाम्यनुपूर्वशः
भगवानेक आसेदमग्र आत्मात्मनां विभुः
आत्मेच्छानुगतावात्मा नानामत्युपलक्षणः
स वा एष तदा द्रष्टा नापश्यद्दृश्यमेकराट्‌
मेनेऽसन्तमिवात्मानं सुप्तशक्तिरसुप्तदृक्‌
सा वा एतस्य संद्रष्टुः शक्तिः सदसदात्मिका
माया नाम महाभाग ययेदं निर्ममे विभुः
कालवृत्त्या तु मायायां गुणमय्यामधोक्षजः
पुरुषेणात्मभूतेन वीर्यमाधत्त वीर्यवान्‌
ततोऽभवन्महत्तत्त्वमव्यक्तात्कालचोदितात्‌
विज्ञानात्मात्मदेहस्थं विश्वं व्यञ्जंस्तमोनुदः
सोऽप्यंशगुणकालात्मा भगवद्दृष्टिगोचरः
आत्मानं व्यकरोदात्मा विश्वस्यास्य सिसृक्षया
महत्तत्त्वाद्विकुर्वाणादहंतत्त्वं व्यजायत
कार्यकारणकर्त्रात्मा भूतेन्द्रियमनोमयः
वैकारिकस्तैजसश्च तामसश्चेत्यहं त्रिधा
अहंतत्त्वाद्विकुर्वाणान्मनो वैकारिकादभूत्‌
वैकारिकाश्च ये देवा अर्थाभिव्यञ्जनं यतः
तैजसानीन्द्रियाण्येव ज्ञानकर्ममयानि च
तामसो भूतसूक्ष्मादिर्यतः खं लिङ्गमात्मनः
कालमायांशयोगेन भगवद्वीक्षितं नभः
नभसोऽनुसृतं स्पर्शं विकुर्वन्निर्ममेऽनिलम्‌
अनिलोऽपि विकुर्वाणो नभसोरुबलान्वितः
ससर्ज रूपतन्मात्रं ज्योतिर्लोकस्य लोचनम्‌
अनिलेनान्वितं ज्योतिर्विकुर्वत्परवीक्षितम्‌
आधत्ताम्भो रसमयं कालमायांशयोगतः
ज्योतिषाम्भोऽनुसंसृष्टं विकुर्वद्ब्रह्मवीक्षितम्‌
महीं गन्धगुणामाधात्कालमायांशयोगतः
भूतानां नभआदीनां यद्यद्भव्यावरावरम्‌
तेषां परानुसंसर्गाद्यथा सङ्ख्यं गुणान्विदुः
एते देवाः कला विष्णोः कालमायांशलिङ्गिनः
नानात्वात्स्वक्रियानीशाः प्रोचुः प्राञ्जलयो विभुम्‌
देवा ऊचुः
नमाम ते देव पदारविन्दं प्रपन्नतापोपशमातपत्रम्‌
यन्मूलकेता यतयोऽञ्जसोरु संसारदुःखं बहिरुत्क्षिपन्ति
धातर्यदस्मिन्भव ईश जीवास्तापत्रयेणाभिहता न शर्म
आत्मन्लभन्ते भगवंस्तवाङ्घ्रि च्छायां सविद्यामत आश्रयेम
मार्गन्ति यत्ते मुखपद्मनीडैश्छन्दःसुपर्णैरृषयो विविक्ते
यस्याघमर्षोदसरिद्वरायाः पदं पदं तीर्थपदः प्रपन्नाः
यच्छ्रद्धया श्रुतवत्या च भक्त्या सम्मृज्यमाने हृदयेऽवधाय
ज्ञानेन वैराग्यबलेन धीरा व्रजेम तत्तेऽङ्घ्रिसरोजपीठम्‌
विश्वस्य जन्मस्थितिसंयमार्थे कृतावतारस्य पदाम्बुजं ते
व्रजेम सर्वे शरणं यदीश स्मृतं प्रयच्छत्यभयं स्वपुंसाम्‌
यत्सानुबन्धेऽसति देहगेहे ममाहमित्यूढदुराग्रहाणाम्‌
पुंसां सुदूरं वसतोऽपि पुर्यां भजेम तत्ते भगवन्पदाब्जम्‌
तान्वै ह्यसद्वृत्तिभिरक्षिभिर्ये पराहृतान्तर्मनसः परेश
अथो न पश्यन्त्युरुगाय नूनं ये ते पदन्यासविलासलक्ष्याः
पानेन ते देव कथासुधायाः प्रवृद्धभक्त्या विशदाशया ये
वैराग्यसारं प्रतिलभ्य बोधं यथाञ्जसान्वीयुरकुण्ठधिष्ण्यम्‌
तथापरे चात्मसमाधियोग बलेन जित्वा प्रकृतिं बलिष्ठाम्‌
त्वामेव धीराः पुरुषं विशन्ति तेषां श्रमः स्यान्न तु सेवया ते
तत्ते वयं लोकसिसृक्षयाद्य त्वयानुसृष्टास्त्रिभिरात्मभिः स्म
सर्वे वियुक्ताः स्वविहारतन्त्रं न शक्नुमस्तत्प्रतिहर्तवे ते
यावद्बलिं तेऽज हराम काले यथा वयं चान्नमदाम यत्र
यथोभयेषां त इमे हि लोका बलिं हरन्तोऽन्नमदन्त्यनूहाः
त्वं नः सुराणामसि सान्वयानां कूटस्थ आद्यः पुरुषः पुराणः
त्वं देव शक्त्यां गुणकर्मयोनौ रेतस्त्वजायां कविमादधेऽजः
ततो वयं मत्प्रमुखा यदर्थे बभूविमात्मन्करवाम किं ते
त्वं नः स्वचक्षुः परिदेहि शक्त्या देव क्रियार्थे यदनुग्रहाणाम्‌
ऋषिरुवाच
इति तासां स्वशक्तीनां सतीनामसमेत्य सः
प्रसुप्तलोकतन्त्राणां निशाम्य गतिमीश्वरः
कालसञ्ज्ञां तदा देवीं बिभ्रच्छक्तिमुरुक्रमः
त्रयोविंशति तत्त्वानां गणं युगपदाविशत्‌
सोऽनुप्रविष्टो भगवांश्चेष्टारूपेण तं गणम्‌
भिन्नं संयोजयामास सुप्तं कर्म प्रबोधयन्‌
प्रबुद्धकर्म दैवेन त्रयोविंशतिको गणः
प्रेरितोऽजनयत्स्वाभिर्मात्राभिरधिपूरुषम्‌
परेण विशता स्वस्मिन्मात्रया विश्वसृग्गणः
चुक्षोभान्योन्यमासाद्य यस्मिन्लोकाश्चराचराः
हिरण्मयः स पुरुषः सहस्रपरिवत्सरान्‌
आण्डकोश उवासाप्सु सर्वसत्त्वोपबृंहितः
स वै विश्वसृजां गर्भो देवकर्मात्मशक्तिमान्‌
विबभाजात्मनात्मानमेकधा दशधा त्रिधा
एष ह्यशेषसत्त्वानामात्मांशः परमात्मनः
आद्योऽवतारो यत्रासौ भूतग्रामो विभाव्यते
साध्यात्मः साधिदैवश्च साधिभूत इति त्रिधा
विराट्प्राणो दशविध एकधा हृदयेन च
स्मरन्विश्वसृजामीशो विज्ञापितमधोक्षजः
विराजमतपत्स्वेन तेजसैषां विवृत्तये
अथ तस्याभितप्तस्य कतिधायतनानि ह
निरभिद्यन्त देवानां तानि मे गदतः शृणु
तस्याग्निरास्यं निर्भिन्नं लोकपालोऽविशत्पदम्‌
वाचा स्वांशेन वक्तव्यं ययासौ प्रतिपद्यते
निर्भिन्नं तालु वरुणो लोकपालोऽविशद्धरेः
जिह्वयांशेन च रसं ययासौ प्रतिपद्यते
निर्भिन्ने अश्विनौ नासे विष्णोराविशतां पदम्‌
घ्राणेनांशेन गन्धस्य प्रतिपत्तिर्यतो भवेत्‌
निर्भिन्ने अक्षिणी त्वष्टा लोकपालोऽविशद्विभोः
चक्षुषांशेन रूपाणां प्रतिपत्तिर्यतो भवेत्‌
निर्भिन्नान्यस्य चर्माणि लोकपालोऽनिलोऽविशत्‌
प्राणेनांशेन संस्पर्शं येनासौ प्रतिपद्यते
कर्णावस्य विनिर्भिन्नौ धिष्ण्यं स्वं विविशुर्दिशः
श्रोत्रेणांशेन शब्दस्य सिद्धिं येन प्रपद्यते
त्वचमस्य विनिर्भिन्नां विविशुर्धिष्ण्यमोषधीः
अंशेन रोमभिः कण्डूं यैरसौ प्रतिपद्यते
मेढ्रं तस्य विनिर्भिन्नं स्वधिष्ण्यं क उपाविशत्‌
रेतसांशेन येनासावानन्दं प्रतिपद्यते
गुदं पुंसो विनिर्भिन्नं मित्रो लोकेश आविशत्‌
पायुनांशेन येनासौ विसर्गं प्रतिपद्यते
हस्तावस्य विनिर्भिन्नाविन्द्रः स्वर्पतिराविशत्‌
वार्तयांशेन पुरुषो यया वृत्तिं प्रपद्यते
पादावस्य विनिर्भिन्नौ लोकेशो विष्णुराविशत्‌
गत्या स्वांशेन पुरुषो यया प्राप्यं प्रपद्यते
बुद्धिं चास्य विनिर्भिन्नां वागीशो धिष्ण्यमाविशत्‌
बोधेनांशेन बोद्धव्यम्प्रतिपत्तिर्यतो भवेत्‌
हृदयं चास्य निर्भिन्नं चन्द्रमा धिष्ण्यमाविशत्‌
मनसांशेन येनासौ विक्रियां प्रतिपद्यते
आत्मानं चास्य निर्भिन्नमभिमानोऽविशत्पदम्‌
कर्मणांशेन येनासौ कर्तव्यं प्रतिपद्यते
सत्त्वं चास्य विनिर्भिन्नं महान्धिष्ण्यमुपाविशत्‌
चित्तेनांशेन येनासौ विज्ञानं प्रतिपद्यते
शीर्ष्णोऽस्य द्यौर्धरा पद्भ्यां खं नाभेरुदपद्यत
गुणानां वृत्तयो येषु प्रतीयन्ते सुरादयः
आत्यन्तिकेन सत्त्वेन दिवं देवाः प्रपेदिरे
धरां रजःस्वभावेन पणयो ये च ताननु
तार्तीयेन स्वभावेन भगवन्नाभिमाश्रिताः
उभयोरन्तरं व्योम ये रुद्रपार्षदां गणाः
मुखतोऽवर्तत ब्रह्म पुरुषस्य कुरूद्वह
यस्तून्मुखत्वाद्वर्णानां मुख्योऽभूद्ब्राह्मणो गुरुः
बाहुभ्योऽवर्तत क्षत्रं क्षत्रियस्तदनुव्रतः
यो जातस्त्रायते वर्णान्पौरुषः कण्टकक्षतात्‌
विशोऽवर्तन्त तस्योर्वोर्लोकवृत्तिकरीर्विभोः
वैश्यस्तदुद्भवो वार्तां नृणां यः समवर्तयत्‌
पद्भ्यां भगवतो जज्ञे शुश्रूषा धर्मसिद्धये
तस्यां जातः पुरा शूद्रो यद्वृत्त्या तुष्यते हरिः
एते वर्णाः स्वधर्मेण यजन्ति स्वगुरुं हरिम्‌
श्रद्धयात्मविशुद्ध्यर्थं यज्जाताः सह वृत्तिभिः
एतत्क्षत्तर्भगवतो दैवकर्मात्मरूपिणः
कः श्रद्दध्यादुपाकर्तुं योगमायाबलोदयम्‌
तथापि कीर्तयाम्यङ्ग यथामति यथाश्रुतम्‌
कीर्तिं हरेः स्वां सत्कर्तुं गिरमन्याभिधासतीम्‌
एकान्तलाभं वचसो नु पुंसां सुश्लोकमौलेर्गुणवादमाहुः
श्रुतेश्च विद्वद्भिरुपाकृतायां कथासुधायामुपसम्प्रयोगम्‌
आत्मनोऽवसितो वत्स महिमा कविनादिना
संवत्सरसहस्रान्ते धिया योगविपक्कया
अतो भगवतो माया मायिनामपि मोहिनी
यत्स्वयं चात्मवर्त्मात्मा न वेद किमुतापरे
यतोऽप्राप्य न्यवर्तन्त वाचश्च मनसा सह
अहं चान्य इमे देवास्तस्मै भगवते नमः
श्रीशुक उवाच
एवं ब्रुवाणं मैत्रेयं द्वैपायनसुतो बुधः
प्रीणयन्निव भारत्या विदुरः प्रत्यभाषत
विदुर उवाच
ब्रह्मन्कथं भगवतश्चिन्मात्रस्याविकारिणः
लीलया चापि युज्येरन्निर्गुणस्य गुणाः क्रियाः
क्रीडायामुद्यमोऽर्भस्य कामश्चिक्रीडिषान्यतः
स्वतस्तृप्तस्य च कथं निवृत्तस्य सदान्यतः
अस्राक्षीद्भगवान्विश्वं गुणमय्यात्ममायया
तया संस्थापयत्येतद्भूयः प्रत्यपिधास्यति
देशतः कालतो योऽसाववस्थातः स्वतोऽन्यतः
अविलुप्तावबोधात्मा स युज्येताजया कथम्‌
भगवानेक एवैष सर्वक्षेत्रेष्ववस्थितः
अमुष्य दुर्भगत्वं वा क्लेशो वा कर्मभिः कुतः
एतस्मिन्मे मनो विद्वन्खिद्यतेऽज्ञानसङ्कटे
तन्नः पराणुद विभो कश्मलं मानसं महत्‌
श्रीशुक उवाच
स इत्थं चोदितः क्षत्त्रा तत्त्वजिज्ञासुना मुनिः
प्रत्याह भगवच्चित्तः स्मयन्निव गतस्मयः
मैत्रेय उवाच
सेयं भगवतो माया यन्नयेन विरुध्यते
ईश्वरस्य विमुक्तस्य कार्पण्यमुत बन्धनम्‌
यदर्थेन विनामुष्य पुंस आत्मविपर्ययः
प्रतीयत उपद्रष्टुः स्वशिरश्छेदनादिकः
यथा जले चन्द्रमसः कम्पादिस्तत्कृतो गुणः
दृश्यतेऽसन्नपि द्रष्टुरात्मनोऽनात्मनो गुणः
स वै निवृत्तिधर्मेण वासुदेवानुकम्पया
भगवद्भक्तियोगेन तिरोधत्ते शनैरिह
यदेन्द्रियोपरामोऽथ द्रष्ट्रात्मनि परे हरौ
विलीयन्ते तदा क्लेशाः संसुप्तस्येव कृत्स्नशः
अशेषसङ्क्लेशशमं विधत्ते गुणानुवादश्रवणं मुरारेः
किं वा पुनस्तच्चरणारविन्द परागसेवारतिरात्मलब्धा
विदुर उवाच
सञ्छिन्नः संशयो मह्यं तव सूक्तासिना विभो
उभयत्रापि भगवन्मनो मे सम्प्रधावति
साध्वेतद्व्याहृतं विद्वन्नात्ममायायनं हरेः
आभात्यपार्थं निर्मूलं विश्वमूलं न यद्बहिः
यश्च मूढतमो लोके यश्च बुद्धेः परं गतः
तावुभौ सुखमेधेते क्लिश्यत्यन्तरितो जनः
अर्थाभावं विनिश्चित्य प्रतीतस्यापि नात्मनः
तां चापि युष्मच्चरण सेवयाहं पराणुदे
यत्सेवया भगवतः कूटस्थस्य मधुद्विषः
रतिरासो भवेत्तीव्रः पादयोर्व्यसनार्दनः
दुरापा ह्यल्पतपसः सेवा वैकुण्ठवर्त्मसु
यत्रोपगीयते नित्यं देवदेवो जनार्दनः
सृष्ट्वाग्रे महदादीनि सविकाराण्यनुक्रमात्‌
तेभ्यो विराजमुद्धृत्य तमनु प्राविशद्विभुः
यमाहुराद्यं पुरुषं सहस्राङ्घ्र्यूरुबाहुकम्‌
यत्र विश्व इमे लोकाः सविकाशं त आसते
यस्मिन्दशविधः प्राणः सेन्द्रियार्थेन्द्रियस्त्रिवृत्‌
त्वयेरितो यतो वर्णास्तद्विभूतीर्वदस्व नः
यत्र पुत्रैश्च पौत्रैश्च नप्तृभिः सह गोत्रजैः
प्रजा विचित्राकृतय आसन्याभिरिदं ततम्‌
प्रजापतीनां स पतिश्चकॢपे कान्प्रजापतीन्‌
सर्गांश्चैवानुसर्गांश्च मनून्मन्वन्तराधिपान्‌
एतेषामपि वेदांश्च वंशानुचरितानि च
उपर्यधश्च ये लोका भूमेर्मित्रात्मजासते
तेषां संस्थां प्रमाणं च भूर्लोकस्य च वर्णय
तिर्यङ्मानुषदेवानां सरीसृपपतत्त्रिणाम्‌
वद नः सर्गसंव्यूहं गार्भस्वेदद्विजोद्भिदाम्‌
गुणावतारैर्विश्वस्य सर्गस्थित्यप्ययाश्रयम्‌
सृजतः श्रीनिवासस्य व्याचक्ष्वोदारविक्रमम्‌
वर्णाश्रमविभागांश्च रूपशीलस्वभावतः
ऋषीणां जन्मकर्माणि वेदस्य च विकर्षणम्‌
यज्ञस्य च वितानानि योगस्य च पथः प्रभो
नैष्कर्म्यस्य च साङ्ख्यस्य तन्त्रं वा भगवत्स्मृतम्‌
पाषण्डपथवैषम्यं प्रतिलोमनिवेशनम्‌
जीवस्य गतयो याश्च यावतीर्गुणकर्मजाः
धर्मार्थकाममोक्षाणां निमित्तान्यविरोधतः
वार्ताया दण्डनीतेश्च श्रुतस्य च विधिं पृथक्‌
श्राद्धस्य च विधिं ब्रह्मन्पित्णां सर्गमेव च
ग्रहनक्षत्रताराणां कालावयवसंस्थितिम्‌
दानस्य तपसो वापि यच्चेष्टापूर्तयोः फलम्‌
प्रवासस्थस्य यो धर्मो यश्च पुंस उतापदि
येन वा भगवांस्तुष्येद्धर्मयोनिर्जनार्दनः
सम्प्रसीदति वा येषामेतदाख्याहि मेऽनघ
अनुव्रतानां शिष्याणां पुत्राणां च द्विजोत्तम
अनापृष्टमपि ब्रूयुर्गुरवो दीनवत्सलाः
तत्त्वानां भगवंस्तेषां कतिधा प्रतिसङ्क्रमः
तत्रेमं क उपासीरन्क उ स्विदनुशेरते
पुरुषस्य च संस्थानं स्वरूपं वा परस्य च
ज्ञानं च नैगमं यत्तद्गुरुशिष्यप्रयोजनम्‌
निमित्तानि च तस्येह प्रोक्तान्यनघसूरिभिः
स्वतो ज्ञानं कुतः पुंसां भक्तिर्वैराग्यमेव वा
एतान्मे पृच्छतः प्रश्नान्हरेः कर्मविवित्सया
ब्रूहि मेऽज्ञस्य मित्रत्वादजया नष्टचक्षुषः
सर्वे वेदाश्च यज्ञाश्च तपो दानानि चानघ
जीवाभयप्रदानस्य न कुर्वीरन्कलामपि
श्रीशुक उवाच
स इत्थमापृष्टपुराणकल्पः कुरुप्रधानेन मुनिप्रधानः
प्रवृद्धहर्षो भगवत्कथायां सञ्चोदितस्तं प्रहसन्निवाह
मैत्रेय उवाच
सत्सेवनीयो बत पूरुवंशो यल्लोकपालो भगवत्प्रधानः
बभूविथेहाजितकीर्तिमालां पदे पदे नूतनयस्यभीक्ष्णम्‌
सोऽहं नृणां क्षुल्लसुखाय दुःखं महद्गतानां विरमाय तस्य
प्रवर्तये भागवतं पुराणं यदाह साक्षाद्भगवानृषिभ्यः
आसीनमुर्व्यां भगवन्तमाद्यं सङ्कर्षणं देवमकुण्ठसत्त्वम्‌
विवित्सवस्तत्त्वमतः परस्य कुमारमुख्या मुनयोऽन्वपृच्छन्‌
स्वमेव धिष्ण्यं बहु मानयन्तं यद्वासुदेवाभिधमामनन्ति
प्रत्यग्धृताक्षाम्बुजकोशमीषदुन्मीलयन्तं विबुधोदयाय
स्वर्धुन्युदार्द्रैः स्वजटाकलापैरुपस्पृशन्तश्चरणोपधानम्‌
पद्मं यदर्चन्त्यहिराजकन्याः सप्रेम नानाबलिभिर्वरार्थाः
मुहुर्गृणन्तो वचसानुराग स्खलत्पदेनास्य कृतानि तज्ज्ञाः
किरीटसाहस्रमणिप्रवेक प्रद्योतितोद्दामफणासहस्रम्‌
प्रोक्तं किलैतद्भगवत्तमेन निवृत्तिधर्माभिरताय तेन
सनत्कुमाराय स चाह पृष्टः साङ्ख्यायनायाङ्ग धृतव्रताय
साङ्ख्यायनः पारमहंस्यमुख्यो विवक्षमाणो भगवद्विभूतीः
जगाद सोऽस्मद्गुरवेऽन्विताय पराशरायाथ बृहस्पतेश्च
प्रोवाच मह्यं स दयालुरुक्तो मुनिः पुलस्त्येन पुराणमाद्यम्‌
सोऽहं तवैतत्कथयामि वत्स श्रद्धालवे नित्यमनुव्रताय
उदाप्लुतं विश्वमिदं तदासीद्यन्निद्रयामीलितदृङ्न्यमीलयत्‌
अहीन्द्रतल्पेऽधिशयान एकः कृतक्षणः स्वात्मरतौ निरीहः
सोऽन्तः शरीरेऽर्पितभूतसूक्ष्मः कालात्मिकां शक्तिमुदीरयाणः
उवास तस्मिन्सलिले पदे स्वे यथानलो दारुणि रुद्धवीर्यः
चतुर्युगानां च सहस्रमप्सु स्वपन्स्वयोदीरितया स्वशक्त्या
कालाख्ययासादितकर्मतन्त्रो लोकानपीतान्ददृशे स्वदेहे
तस्यार्थसूक्ष्माभिनिविष्टदृष्टेरन्तर्गतोऽर्थो रजसा तनीयान्‌
गुणेन कालानुगतेन विद्धः सूष्यंस्तदाभिद्यत नाभिदेशात्‌
स पद्मकोशः सहसोदतिष्ठत्कालेन कर्मप्रतिबोधनेन
स्वरोचिषा तत्सलिलं विशालं विद्योतयन्नर्क इवात्मयोनिः
तल्लोकपद्मं स उ एव विष्णुः प्रावीविशत्सर्वगुणावभासम्‌
तस्मिन्स्वयं वेदमयो विधाता स्वयम्भुवं यं स्म वदन्ति सोऽभूत्‌
तस्यां स चाम्भोरुहकर्णिकायामवस्थितो लोकमपश्यमानः
परिक्रमन्व्योम्नि विवृत्तनेत्रश्चत्वारि लेभेऽनुदिशं मुखानि
तस्माद्युगान्तश्वसनावघूर्ण जलोर्मिचक्रात्सलिलाद्विरूढम्‌
उपाश्रितः कञ्जमु लोकतत्त्वं नात्मानमद्धाविददादिदेवः
क एष योऽसावहमब्जपृष्ठ एतत्कुतो वाब्जमनन्यदप्सु
अस्ति ह्यधस्तादिह किञ्चनैतदधिष्ठितं यत्र सता नु भाव्यम्‌
स इत्थमुद्वीक्ष्य तदब्जनाल नाडीभिरन्तर्जलमाविवेश
नार्वाग्गतस्तत्खरनालनाल नाभिं विचिन्वंस्तदविन्दताजः
तमस्यपारे विदुरात्मसर्गं विचिन्वतोऽभूत्सुमहांस्त्रिणेमिः
यो देहभाजां भयमीरयाणः परिक्षिणोत्यायुरजस्य हेतिः
ततो निवृत्तोऽप्रतिलब्धकामः स्वधिष्ण्यमासाद्य पुनः स देवः
शनैर्जितश्वासनिवृत्तचित्तो न्यषीददारूढसमाधियोगः
कालेन सोऽजः पुरुषायुषाभि प्रवृत्तयोगेन विरूढबोधः
स्वयं तदन्तर्हृदयेऽवभातमपश्यतापश्यत यन्न पूर्वम्‌
मृणालगौरायतशेषभोग पर्यङ्क एकं पुरुषं शयानम्‌
फणातपत्रायुतमूर्धरत्न द्युभिर्हतध्वान्तयुगान्ततोये
प्रेक्षां क्षिपन्तं हरितोपलाद्रेः सन्ध्याभ्रनीवेरुरुरुक्ममूर्ध्नः
रत्नोदधारौषधिसौमनस्य वनस्रजो वेणुभुजाङ्घ्रिपाङ्घ्रेः
आयामतो विस्तरतः स्वमान देहेन लोकत्रयसङ्ग्रहेण
विचित्रदिव्याभरणांशुकानां कृतश्रियापाश्रितवेषदेहम्‌
पुंसां स्वकामाय विविक्तमार्गैरभ्यर्चतां कामदुघाङ्घ्रिपद्मम्‌
प्रदर्शयन्तं कृपया नखेन्दु मयूखभिन्नाङ्गुलिचारुपत्रम्‌
मुखेन लोकार्तिहरस्मितेन परिस्फुरत्कुण्डलमण्डितेन
शोणायितेनाधरबिम्बभासा प्रत्यर्हयन्तं सुनसेन सुभ्र्वा
कदम्बकिञ्जल्कपिशङ्गवाससा स्वलङ्कृतं मेखलया नितम्बे
हारेण चानन्तधनेन वत्स श्रीवत्सवक्षःस्थलवल्लभेन
परार्ध्यकेयूरमणिप्रवेक पर्यस्तदोर्दण्डसहस्रशाखम्‌
अव्यक्तमूलं भुवनाङ्घ्रिपेन्द्रमहीन्द्रभोगैरधिवीतवल्शम्‌
चराचरौको भगवन्महीध्रमहीन्द्रबन्धुं सलिलोपगूढम्‌
किरीटसाहस्रहिरण्यशृङ्गमाविर्भवत्कौस्तुभरत्नगर्भम्‌
निवीतमाम्नायमधुव्रतश्रिया स्वकीर्तिमय्या वनमालया हरिम्‌
सूर्येन्दुवाय्वग्न्यगमं त्रिधामभिः परिक्रमत्प्राधनिकैर्दुरासदम्‌
तर्ह्येव तन्नाभिसरःसरोजमात्मानमम्भः श्वसनं वियच्च
ददर्श देवो जगतो विधाता नातः परं लोकविसर्गदृष्टिः
स कर्मबीजं रजसोपरक्तः प्रजाः सिसृक्षन्नियदेव दृष्ट्वा
अस्तौद्विसर्गाभिमुखस्तमीड्यमव्यक्तवर्त्मन्यभिवेशितात्मा
ब्रह्मोवाच
ज्ञातोऽसि मेऽद्य सुचिरान्ननु देहभाजां
न ज्ञायते भगवतो गतिरित्यवद्यम्‌
नान्यत्त्वदस्ति भगवन्नपि तन्न शुद्धं
मायागुणव्यतिकराद्यदुरुर्विभासि
रूपं यदेतदवबोधरसोदयेन
शश्वन्निवृत्ततमसः सदनुग्रहाय
आदौ गृहीतमवतारशतैकबीजं
यन्नाभिपद्मभवनादहमाविरासम्‌
नातः परं परम यद्भवतः स्वरूपम्‌
आनन्दमात्रमविकल्पमविद्धवर्चः
पश्यामि विश्वसृजमेकमविश्वमात्मन्‌
भूतेन्द्रियात्मकमदस्त उपाश्रितोऽस्मि
तद्वा इदं भुवनमङ्गल मङ्गलाय
ध्याने स्म नो दर्शितं त उपासकानाम्‌
तस्मै नमो भगवतेऽनुविधेम तुभ्यं
योऽनादृतो नरकभाग्भिरसत्प्रसङ्गैः
ये तु त्वदीयचरणाम्बुजकोशगन्धं
जिघ्रन्ति कर्णविवरैः श्रुतिवातनीतम्‌
भक्त्या गृहीतचरणः परया च तेषां
नापैषि नाथ हृदयाम्बुरुहात्स्वपुंसाम्‌
तावद्भयं द्रविणदेहसुहृन्निमित्तं
शोकः स्पृहा परिभवो विपुलश्च लोभः
तावन्ममेत्यसदवग्रह आर्तिमूलं
यावन्न तेऽङ्घ्रिमभयं प्रवृणीत लोकः
दैवेन ते हतधियो भवतः प्रसङ्गात्‌
सर्वाशुभोपशमनाद्विमुखेन्द्रिया ये
कुर्वन्ति कामसुखलेशलवाय दीना
लोभाभिभूतमनसोऽकुशलानि शश्वत्‌
क्षुत्तृट्त्रिधातुभिरिमा मुहुरर्द्यमानाः
शीतोष्णवातवरषैरितरेतराच्च
कामाग्निनाच्युतरुषा च सुदुर्भरेण
सम्पश्यतो मन उरुक्रम सीदते मे
यावत्पृथक्त्वमिदमात्मन इन्द्रियार्थ
मायाबलं भगवतो जन ईश पश्येत्‌
तावन्न संसृतिरसौ प्रतिसङ्क्रमेत
व्यर्थापि दुःखनिवहं वहती क्रियार्था
अह्न्यापृतार्तकरणा निशि निःशयाना
नानामनोरथधिया क्षणभग्ननिद्राः
दैवाहतार्थरचना ऋषयोऽपि देव
युष्मत्प्रसङ्गविमुखा इह संसरन्ति
त्वं भक्तियोगपरिभावितहृत्सरोज
आस्से श्रुतेक्षितपथो ननु नाथ पुंसाम्‌
यद्यद्धिया त उरुगाय विभावयन्ति
तत्तद्वपुः प्रणयसे सदनुग्रहाय
नातिप्रसीदति तथोपचितोपचारैर्‌
आराधितः सुरगणैर्हृदि बद्धकामैः
यत्सर्वभूतदययासदलभ्ययैको
नानाजनेष्ववहितः सुहृदन्तरात्मा
पुंसामतो विविधकर्मभिरध्वराद्यैर्‌
दानेन चोग्रतपसा परिचर्यया च
आराधनं भगवतस्तव सत्क्रियार्थो
धर्मोऽर्पितः कर्हिचिद्म्रियते न यत्र
शश्वत्स्वरूपमहसैव निपीतभेद
मोहाय बोधधिषणाय नमः परस्मै
विश्वोद्भवस्थितिलयेषु निमित्तलीला
रासाय ते नम इदं चकृमेश्वराय
यस्यावतारगुणकर्मविडम्बनानि
नामानि येऽसुविगमे विवशा गृणन्ति
तेऽनैकजन्मशमलं सहसैव हित्वा
संयान्त्यपावृतामृतं तमजं प्रपद्ये
यो वा अहं च गिरिशश्च विभुः स्वयं च
स्थित्युद्भवप्रलयहेतव आत्ममूलम्‌
भित्त्वा त्रिपाद्ववृध एक उरुप्ररोहस्‌
तस्मै नमो भगवते भुवनद्रुमाय
लोको विकर्मनिरतः कुशले प्रमत्तः
कर्मण्ययं त्वदुदिते भवदर्चने स्वे
यस्तावदस्य बलवानिह जीविताशां
सद्यश्छिनत्त्यनिमिषाय नमोऽस्तु तस्मै
यस्माद्बिभेम्यहमपि द्विपरार्धधिष्ण्यम्‌
अध्यासितः सकललोकनमस्कृतं यत्‌
तेपे तपो बहुसवोऽवरुरुत्समानस्‌
तस्मै नमो भगवतेऽधिमखाय तुभ्यम्‌
तिर्यङ्मनुष्यविबुधादिषु जीवयोनिष्व्‌
आत्मेच्छयात्मकृतसेतुपरीप्सया यः
रेमे निरस्तविषयोऽप्यवरुद्धदेहस्‌
तस्मै नमो भगवते पुरुषोत्तमाय
योऽविद्ययानुपहतोऽपि दशार्धवृत्त्या
निद्रामुवाह जठरीकृतलोकयात्रः
अन्तर्जलेऽहिकशिपुस्पर्शानुकूलां
भीमोर्मिमालिनि जनस्य सुखं विवृण्वन्‌
यन्नाभिपद्मभवनादहमासमीड्य
लोकत्रयोपकरणो यदनुग्रहेण
तस्मै नमस्त उदरस्थभवाय योग
निद्रावसानविकसन्नलिनेक्षणाय
सोऽयं समस्तजगतां सुहृदेक आत्मा
सत्त्वेन यन्मृडयते भगवान्भगेन
तेनैव मे दृशमनुस्पृशताद्यथाहं
स्रक्ष्यामि पूर्ववदिदं प्रणतप्रियोऽसौ
एष प्रपन्नवरदो रमयात्मशक्त्या
यद्यत्करिष्यति गृहीतगुणावतारः
तस्मिन्स्वविक्रममिदं सृजतोऽपि चेतो
युञ्जीत कर्मशमलं च यथा विजह्याम्‌
नाभिह्रदादिह सतोऽम्भसि यस्य पुंसो
विज्ञानशक्तिरहमासमनन्तशक्तेः
रूपं विचित्रमिदमस्य विवृण्वतो मे
मा रीरिषीष्ट निगमस्य गिरां विसर्गः
सोऽसावदभ्रकरुणो भगवान्विवृद्ध
प्रेमस्मितेन नयनाम्बुरुहं विजृम्भन्‌
उत्थाय विश्वविजयाय च नो विषादं
माध्व्या गिरापनयतात्पुरुषः पुराणः
मैत्रेय उवाच
स्वसम्भवं निशाम्यैवं तपोविद्यासमाधिभिः
यावन्मनोवचः स्तुत्वा विरराम स खिन्नवत्‌
अथाभिप्रेतमन्वीक्ष्य ब्रह्मणो मधुसूदनः
विषण्णचेतसं तेन कल्पव्यतिकराम्भसा
लोकसंस्थानविज्ञान आत्मनः परिखिद्यतः
तमाहागाधया वाचा कश्मलं शमयन्निव
श्रीभगवानुवाच
मा वेदगर्भ गास्तन्द्रीं सर्ग उद्यममावह
तन्मयापादितं ह्यग्रे यन्मां प्रार्थयते भवान्‌
भूयस्त्वं तप आतिष्ठ विद्यां चैव मदाश्रयाम्‌
ताभ्यामन्तर्हृदि ब्रह्मन्लोकान्द्रक्ष्यस्यपावृतान्‌
तत आत्मनि लोके च भक्तियुक्तः समाहितः
द्रष्टासि मां ततं ब्रह्मन्मयि लोकांस्त्वमात्मनः
यदा तु सर्वभूतेषु दारुष्वग्निमिव स्थितम्‌
प्रतिचक्षीत मां लोको जह्यात्तर्ह्येव कश्मलम्‌
यदा रहितमात्मानं भूतेन्द्रियगुणाशयैः
स्वरूपेण मयोपेतं पश्यन्स्वाराज्यमृच्छति
नानाकर्मवितानेन प्रजा बह्वीः सिसृक्षतः
नात्मावसीदत्यस्मिंस्ते वर्षीयान्मदनुग्रहः
ऋषिमाद्यं न बध्नाति पापीयांस्त्वां रजोगुणः
यन्मनो मयि निर्बद्धं प्रजाः संसृजतोऽपि ते
ज्ञातोऽहं भवता त्वद्य दुर्विज्ञेयोऽपि देहिनाम्‌
यन्मां त्वं मन्यसेऽयुक्तं भूतेन्द्रियगुणात्मभिः
तुभ्यं मद्विचिकित्सायामात्मा मे दर्शितोऽबहिः
नालेन सलिले मूलं पुष्करस्य विचिन्वतः
यच्चकर्थाङ्ग मत्स्तोत्रं मत्कथाभ्युदयाङ्कितम्‌
यद्वा तपसि ते निष्ठा स एष मदनुग्रहः
प्रीतोऽहमस्तु भद्रं ते लोकानां विजयेच्छया
यदस्तौषीर्गुणमयं निर्गुणं मानुवर्णयन्‌
य एतेन पुमान्नित्यं स्तुत्वा स्तोत्रेण मां भजेत्‌
तस्याशु सम्प्रसीदेयं सर्वकामवरेश्वरः
पूर्तेन तपसा यज्ञैर्दानैर्योगसमाधिना
राद्धं निःश्रेयसं पुंसां मत्प्रीतिस्तत्त्वविन्मतम्‌
अहमात्मात्मनां धातः प्रेष्ठः सन्प्रेयसामपि
अतो मयि रतिं कुर्याद्देहादिर्यत्कृते प्रियः
सर्ववेदमयेनेदमात्मनात्मात्मयोनिना
प्रजाः सृज यथापूर्वं याश्च मय्यनुशेरते
मैत्रेय उवाच
तस्मा एवं जगत्स्रष्ट्रे प्रधानपुरुषेश्वरः
व्यज्येदं स्वेन रूपेण कञ्जनाभस्तिरोदधे
विदुर उवाच
अन्तर्हिते भगवति ब्रह्मा लोकपितामहः
प्रजाः ससर्ज कतिधा दैहिकीर्मानसीर्विभुः
ये च मे भगवन्पृष्टास्त्वय्यर्था बहुवित्तम
तान्वदस्वानुपूर्व्येण छिन्धि नः सर्वसंशयान्‌
सूत उवाच
एवं सञ्चोदितस्तेन क्षत्त्रा कौषारविर्मुनिः
प्रीतः प्रत्याह तान्प्रश्नान्हृदिस्थानथ भार्गव
मैत्रेय उवाच
विरिञ्चोऽपि तथा चक्रे दिव्यं वर्षशतं तपः
आत्मन्यात्मानमावेश्य यथाह भगवानजः
तद्विलोक्याब्जसम्भूतो वायुना यदधिष्ठितः
पद्ममम्भश्च तत्काल कृतवीर्येण कम्पितम्‌
तपसा ह्येधमानेन विद्यया चात्मसंस्थया
विवृद्धविज्ञानबलो न्यपाद्वायुं सहाम्भसा
तद्विलोक्य वियद्व्यापि पुष्करं यदधिष्ठितम्‌
अनेन लोकान्प्राग्लीनान्कल्पितास्मीत्यचिन्तयत्‌
पद्मकोशं तदाविश्य भगवत्कर्मचोदितः
एकं व्यभाङ्क्षीदुरुधा त्रिधा भाव्यं द्विसप्तधा
एतावाञ्जीवलोकस्य संस्थाभेदः समाहृतः
धर्मस्य ह्यनिमित्तस्य विपाकः परमेष्ठ्यसौ
विदुर उवाच
यथात्थ बहुरूपस्य हरेरद्भुतकर्मणः
कालाख्यं लक्षणं ब्रह्मन्यथा वर्णय नः प्रभो
मैत्रेय उवाच
गुणव्यतिकराकारो निर्विशेषोऽप्रतिष्ठितः
पुरुषस्तदुपादानमात्मानं लीलयासृजत्‌
विश्वं वै ब्रह्मतन्मात्रं संस्थितं विष्णुमायया
ईश्वरेण परिच्छिन्नं कालेनाव्यक्तमूर्तिना
यथेदानीं तथाग्रे च पश्चादप्येतदीदृशम्‌
सर्गो नवविधस्तस्य प्राकृतो वैकृतस्तु यः
कालद्रव्यगुणैरस्य त्रिविधः प्रतिसङ्क्रमः
आद्यस्तु महतः सर्गो गुणवैषम्यमात्मनः
द्वितीयस्त्वहमो यत्र द्रव्यज्ञानक्रियोदयः
भूतसर्गस्तृतीयस्तु तन्मात्रो द्रव्यशक्तिमान्‌
चतुर्थ ऐन्द्रियः सर्गो यस्तु ज्ञानक्रियात्मकः
वैकारिको देवसर्गः पञ्चमो यन्मयं मनः
षष्ठस्तु तमसः सर्गो यस्त्वबुद्धिकृतः प्रभोः
षडिमे प्राकृताः सर्गा वैकृतानपि मे शृणु
रजोभाजो भगवतो लीलेयं हरिमेधसः
सप्तमो मुख्यसर्गस्तु षड्विधस्तस्थुषां च यः
वनस्पत्योषधिलता त्वक्सारा वीरुधो द्रुमाः
उत्स्रोतसस्तमःप्राया अन्तःस्पर्शा विशेषिणः
तिरश्चामष्टमः सर्गः सोऽष्टाविंशद्विधो मतः
अविदो भूरितमसो घ्राणज्ञा हृद्यवेदिनः
गौरजो महिषः कृष्णः सूकरो गवयो रुरुः
द्विशफाः पशवश्चेमे अविरुष्ट्रश्च सत्तम
खरोऽश्वोऽश्वतरो गौरः शरभश्चमरी तथा
एते चैकशफाः क्षत्तः शृणु पञ्चनखान्पशून्‌
श्वा सृगालो वृको व्याघ्रो मार्जारः शशशल्लकौ
सिंहः कपिर्गजः कूर्मो गोधा च मकरादयः
कङ्कगृध्रबकश्येन भासभल्लूकबर्हिणः
हंससारसचक्राह्व काकोलूकादयः खगाः
अर्वाक्स्रोतस्तु नवमः क्षत्तरेकविधो नृणाम्‌
रजोऽधिकाः कर्मपरा दुःखे च सुखमानिनः
वैकृतास्त्रय एवैते देवसर्गश्च सत्तम
वैकारिकस्तु यः प्रोक्तः कौमारस्तूभयात्मकः
देवसर्गश्चाष्टविधो विबुधाः पितरोऽसुराः
गन्धर्वाप्सरसः सिद्धा यक्षरक्षांसि चारणाः
भूतप्रेतपिशाचाश्च विद्याध्राः किन्नरादयः
दशैते विदुराख्याताः सर्गास्ते विश्वसृक्कृताः
अतः परं प्रवक्ष्यामि वंशान्मन्वन्तराणि च
एवं रजःप्लुतः स्रष्टा कल्पादिष्वात्मभूर्हरिः
सृजत्यमोघसङ्कल्प आत्मैवात्मानमात्मना
मैत्रेय उवाच
चरमः सद्विशेषाणामनेकोऽसंयुतः सदा
परमाणुः स विज्ञेयो नृणामैक्यभ्रमो यतः
सत एव पदार्थस्य स्वरूपावस्थितस्य यत्‌
कैवल्यं परममहानविशेषो निरन्तरः
एवं कालोऽप्यनुमितः सौक्ष्म्ये स्थौल्ये च सत्तम
संस्थानभुक्त्या भगवानव्यक्तो व्यक्तभुग्विभुः
स कालः परमाणुर्वै यो भुङ्क्ते परमाणुताम्‌
सतोऽविशेषभुग्यस्तु स कालः परमो महान्‌
अणुर्द्वौ परमाणू स्यात्त्रसरेणुस्त्रयः स्मृतः
जालार्करश्म्यवगतः खमेवानुपतन्नगात्‌
त्रसरेणुत्रिकं भुङ्क्ते यः कालः स त्रुटिः स्मृतः
शतभागस्तु वेधः स्यात्तैस्त्रिभिस्तु लवः स्मृतः
निमेषस्त्रिलवो ज्ञेय आम्नातस्ते त्रयः क्षणः
क्षणान्पञ्च विदुः काष्ठां लघु ता दश पञ्च च
लघूनि वै समाम्नाता दश पञ्च च नाडिका
ते द्वे मुहूर्तः प्रहरः षड्यामः सप्त वा नृणाम्‌
द्वादशार्धपलोन्मानं चतुर्भिश्चतुरङ्गुलैः
स्वर्णमाषैः कृतच्छिद्रं यावत्प्रस्थजलप्लुतम्‌
यामाश्चत्वारश्चत्वारो मर्त्यानामहनी उभे
पक्षः पञ्चदशाहानि शुक्लः कृष्णश्च मानद
तयोः समुच्चयो मासः पित्ऱ्‌^ईणां तदहर्निशम्‌
द्वौ तावृतुः षडयनं दक्षिणं चोत्तरं दिवि
अयने चाहनी प्राहुर्वत्सरो द्वादश स्मृतः
संवत्सरशतं न्णां परमायुर्निरूपितम्‌
ग्रहर्क्षताराचक्रस्थः परमाण्वादिना जगत्‌
संवत्सरावसानेन पर्येत्यनिमिषो विभुः
संवत्सरः परिवत्सर इडावत्सर एव च
अनुवत्सरो वत्सरश्च विदुरैवं प्रभाष्यते
यः सृज्यशक्तिमुरुधोच्छ्वसयन्स्वशक्त्या
पुंसोऽभ्रमाय दिवि धावति भूतभेदः
कालाख्यया गुणमयं क्रतुभिर्वितन्वंस्‌
तस्मै बलिं हरत वत्सरपञ्चकाय
विदुर उवाच
पितृदेवमनुष्याणामायुः परमिदं स्मृतम्‌
परेषां गतिमाचक्ष्व ये स्युः कल्पाद्बहिर्विदः
भगवान्वेद कालस्य गतिं भगवतो ननु
विश्वं विचक्षते धीरा योगराद्धेन चक्षुषा
मैत्रेय उवाच
कृतं त्रेता द्वापरं च कलिश्चेति चतुर्युगम्‌
दिव्यैर्द्वादशभिर्वर्षैः सावधानं निरूपितम्‌
चत्वारि त्रीणि द्वे चैकं कृतादिषु यथाक्रमम्‌
सङ्ख्यातानि सहस्राणि द्विगुणानि शतानि च
सन्ध्यासन्ध्यांशयोरन्तर्यः कालः शतसङ्ख्ययोः
तमेवाहुर्युगं तज्ज्ञा यत्र धर्मो विधीयते
धर्मश्चतुष्पान्मनुजान्कृते समनुवर्तते
स एवान्येष्वधर्मेण व्येति पादेन वर्धता
त्रिलोक्या युगसाहस्रं बहिराब्रह्मणो दिनम्‌
तावत्येव निशा तात यन्निमीलति विश्वसृक्‌
निशावसान आरब्धो लोककल्पोऽनुवर्तते
यावद्दिनं भगवतो मनून्भुञ्जंश्चतुर्दश
स्वं स्वं कालं मनुर्भुङ्क्ते साधिकां ह्येकसप्ततिम्‌
मन्वन्तरेषु मनवस्तद्वंश्या ऋषयः सुराः
भवन्ति चैव युगपत्सुरेशाश्चानु ये च तान्‌
एष दैनन्दिनः सर्गो ब्राह्मस्त्रैलोक्यवर्तनः
तिर्यङ्नृपितृदेवानां सम्भवो यत्र कर्मभिः
मन्वन्तरेषु भगवान्बिभ्रत्सत्त्वं स्वमूर्तिभिः
मन्वादिभिरिदं विश्वमवत्युदितपौरुषः
तमोमात्रामुपादाय प्रतिसंरुद्धविक्रमः
कालेनानुगताशेष आस्ते तूष्णीं दिनात्यये
तमेवान्वपि धीयन्ते लोका भूरादयस्त्रयः
निशायामनुवृत्तायां निर्मुक्तशशिभास्करम्‌
त्रिलोक्यां दह्यमानायां शक्त्या सङ्कर्षणाग्निना
यान्त्यूष्मणा महर्लोकाज्जनं भृग्वादयोऽर्दिताः
तावत्त्रिभुवनं सद्यः कल्पान्तैधितसिन्धवः
प्लावयन्त्युत्कटाटोप चण्डवातेरितोर्मयः
अन्तः स तस्मिन्सलिल आस्तेऽनन्तासनो हरिः
योगनिद्रानिमीलाक्षः स्तूयमानो जनालयैः
एवंविधैरहोरात्रैः कालगत्योपलक्षितैः
अपक्षितमिवास्यापि परमायुर्वयःशतम्‌
यदर्धमायुषस्तस्य परार्धमभिधीयते
पूर्वः परार्धोऽपक्रान्तो ह्यपरोऽद्य प्रवर्तते
पूर्वस्यादौ परार्धस्य ब्राह्मो नाम महानभूत्‌
कल्पो यत्राभवद्ब्रह्मा शब्दब्रह्मेति यं विदुः
तस्यैव चान्ते कल्पोऽभूद्यं पाद्ममभिचक्षते
यद्धरेर्नाभिसरस आसील्लोकसरोरुहम्‌
अयं तु कथितः कल्पो द्वितीयस्यापि भारत
वाराह इति विख्यातो यत्रासीच्छूकरो हरिः
कालोऽयं द्विपरार्धाख्यो निमेष उपचर्यते
अव्याकृतस्यानन्तस्य ह्यनादेर्जगदात्मनः
कालोऽयं परमाण्वादिर्द्विपरार्धान्त ईश्वरः
नैवेशितुं प्रभुर्भूम्न ईश्वरो धाममानिनाम्‌
विकारैः सहितो युक्तैर्विशेषादिभिरावृतः
आण्डकोशो बहिरयं पञ्चाशत्कोटिविस्तृतः
दशोत्तराधिकैर्यत्र प्रविष्टः परमाणुवत्‌
लक्ष्यतेऽन्तर्गताश्चान्ये कोटिशो ह्यण्डराशयः
तदाहुरक्षरं ब्रह्म सर्वकारणकारणम्‌
विष्णोर्धाम परं साक्षात्पुरुषस्य महात्मनः
मैत्रेय उवाच
इति ते वर्णितः क्षत्तः कालाख्यः परमात्मनः
महिमा वेदगर्भोऽथ यथास्राक्षीन्निबोध मे
ससर्जाग्रेऽन्धतामिस्रमथ तामिस्रमादिकृत्‌
महामोहं च मोहं च तमश्चाज्ञानवृत्तयः
दृष्ट्वा पापीयसीं सृष्टिं नात्मानं बह्वमन्यत
भगवद्ध्यानपूतेन मनसान्यां ततोऽसृजत्‌
सनकं च सनन्दं च सनातनमथात्मभूः
सनत्कुमारं च मुनीन्निष्क्रियानूर्ध्वरेतसः
तान्बभाषे स्वभूः पुत्रान्प्रजाः सृजत पुत्रकाः
तन्नैच्छन्मोक्षधर्माणो वासुदेवपरायणाः
सोऽवध्यातः सुतैरेवं प्रत्याख्यातानुशासनैः
क्रोधं दुर्विषहं जातं नियन्तुमुपचक्रमे
धिया निगृह्यमाणोऽपि भ्रुवोर्मध्यात्प्रजापतेः
सद्योऽजायत तन्मन्युः कुमारो नीललोहितः
स वै रुरोद देवानां पूर्वजो भगवान्भवः
नामानि कुरु मे धातः स्थानानि च जगद्गुरो
इति तस्य वचः पाद्मो भगवान्परिपालयन्‌
अभ्यधाद्भद्रया वाचा मा रोदीस्तत्करोमि ते
यदरोदीः सुरश्रेष्ठ सोद्वेग इव बालकः
ततस्त्वामभिधास्यन्ति नाम्ना रुद्र इति प्रजाः
हृदिन्द्रियाण्यसुर्व्योम वायुरग्निर्जलं मही
सूर्यश्चन्द्रस्तपश्चैव स्थानान्यग्रे कृतानि ते
मन्युर्मनुर्महिनसो महाञ्छिव ऋतध्वजः
उग्ररेता भवः कालो वामदेवो धृतव्रतः
धीर्धृतिरसलोमा च नियुत्सर्पिरिलाम्बिका
इरावती स्वधा दीक्षा रुद्राण्यो रुद्र ते स्त्रियः
गृहाणैतानि नामानि स्थानानि च सयोषणः
एभिः सृज प्रजा बह्वीः प्रजानामसि यत्पतिः
इत्यादिष्टः स्वगुरुणा भगवान्नीललोहितः
सत्त्वाकृतिस्वभावेन ससर्जात्मसमाः प्रजाः
रुद्राणां रुद्रसृष्टानां समन्ताद्ग्रसतां जगत्‌
निशाम्यासङ्ख्यशो यूथान्प्रजापतिरशङ्कत
अलं प्रजाभिः सृष्टाभिरीदृशीभिः सुरोत्तम
मया सह दहन्तीभिर्दिशश्चक्षुर्भिरुल्बणैः
तप आतिष्ठ भद्रं ते सर्वभूतसुखावहम्‌
तपसैव यथा पूर्वं स्रष्टा विश्वमिदं भवान्‌
तपसैव परं ज्योतिर्भगवन्तमधोक्षजम्‌
सर्वभूतगुहावासमञ्जसा विन्दते पुमान्‌
मैत्रेय उवाच
एवमात्मभुवादिष्टः परिक्रम्य गिरां पतिम्‌
बाढमित्यमुमामन्त्र्य विवेश तपसे वनम्‌
अथाभिध्यायतः सर्गं दश पुत्राः प्रजज्ञिरे
भगवच्छक्तियुक्तस्य लोकसन्तानहेतवः
मरीचिरत्र्यङ्गिरसौ पुलस्त्यः पुलहः क्रतुः
भृगुर्वसिष्ठो दक्षश्च दशमस्तत्र नारदः
उत्सङ्गान्नारदो जज्ञे दक्षोऽङ्गुष्ठात्स्वयम्भुवः
प्राणाद्वसिष्ठः सञ्जातो भृगुस्त्वचि करात्क्रतुः
पुलहो नाभितो जज्ञे पुलस्त्यः कर्णयोरृषिः
अङ्गिरा मुखतोऽक्ष्णोऽत्रिर्मरीचिर्मनसोऽभवत्‌
धर्मः स्तनाद्दक्षिणतो यत्र नारायणः स्वयम्‌
अधर्मः पृष्ठतो यस्मान्मृत्युर्लोकभयङ्करः
हृदि कामो भ्रुवः क्रोधो लोभश्चाधरदच्छदात्‌
आस्याद्वाक्सिन्धवो मेढ्रान्निरृतिः पायोरघाश्रयः
छायायाः कर्दमो जज्ञे देवहूत्याः पतिः प्रभुः
मनसो देहतश्चेदं जज्ञे विश्वकृतो जगत्‌
वाचं दुहितरं तन्वीं स्वयम्भूर्हरतीं मनः
अकामां चकमे क्षत्तः सकाम इति नः श्रुतम्‌
तमधर्मे कृतमतिं विलोक्य पितरं सुताः
मरीचिमुख्या मुनयो विश्रम्भात्प्रत्यबोधयन्‌
नैतत्पूर्वैः कृतं त्वद्ये न करिष्यन्ति चापरे
यस्त्वं दुहितरं गच्छेरनिगृह्याङ्गजं प्रभुः
तेजीयसामपि ह्येतन्न सुश्लोक्यं जगद्गुरो
यद्वृत्तमनुतिष्ठन्वै लोकः क्षेमाय कल्पते
तस्मै नमो भगवते य इदं स्वेन रोचिषा
आत्मस्थं व्यञ्जयामास स धर्मं पातुमर्हति
स इत्थं गृणतः पुत्रान्पुरो दृष्ट्वा प्रजापतीन्‌
प्रजापतिपतिस्तन्वं तत्याज व्रीडितस्तदा
तां दिशो जगृहुर्घोरां नीहारं यद्विदुस्तमः
कदाचिद्ध्यायतः स्रष्टुर्वेदा आसंश्चतुर्मुखात्‌
कथं स्रक्ष्याम्यहं लोकान्समवेतान्यथा पुरा
चातुर्होत्रं कर्मतन्त्रमुपवेदनयैः सह
धर्मस्य पादाश्चत्वारस्तथैवाश्रमवृत्तयः
विदुर उवाच
स वै विश्वसृजामीशो वेदादीन्मुखतोऽसृजत्‌
यद्यद्येनासृजद्देवस्तन्मे ब्रूहि तपोधन
मैत्रेय उवाच
ऋग्यजुःसामाथर्वाख्यान्वेदान्पूर्वादिभिर्मुखैः
शास्त्रमिज्यां स्तुतिस्तोमं प्रायश्चित्तं व्यधात्क्रमात्‌
आयुर्वेदं धनुर्वेदं गान्धर्वं वेदमात्मनः
स्थापत्यं चासृजद्वेदं क्रमात्पूर्वादिभिर्मुखैः
इतिहासपुराणानि पञ्चमं वेदमीश्वरः
सर्वेभ्य एव वक्त्रेभ्यः ससृजे सर्वदर्शनः
षोडश्युक्थौ पूर्ववक्त्रात्पुरीष्यग्निष्टुतावथ
आप्तोर्यामातिरात्रौ च वाजपेयं सगोसवम्‌
विद्या दानं तपः सत्यं धर्मस्येति पदानि च
आश्रमांश्च यथासङ्ख्यमसृजत्सह वृत्तिभिः
सावित्रं प्राजापत्यं च ब्राह्मं चाथ बृहत्तथा
वार्ता सञ्चयशालीन शिलोञ्छ इति वै गृहे
वैखानसा वालखिल्यौ दुम्बराः फेनपा वने
न्यासे कुटीचकः पूर्वं बह्वोदो हंसनिष्क्रियौ
आन्वीक्षिकी त्रयी वार्ता दण्डनीतिस्तथैव च
एवं व्याहृतयश्चासन्प्रणवो ह्यस्य दह्रतः
तस्योष्णिगासील्लोमभ्यो गायत्री च त्वचो विभोः
त्रिष्टुम्मांसात्स्नुतोऽनुष्टुब्जगत्यस्थ्नः प्रजापतेः
मज्जायाः पङ्क्तिरुत्पन्ना बृहती प्राणतोऽभवत्‌
स्पर्शस्तस्याभवज्जीवः स्वरो देह उदाहृत
ऊष्माणमिन्द्रियाण्याहुरन्तःस्था बलमात्मनः
स्वराः सप्त विहारेण भवन्ति स्म प्रजापतेः
शब्दब्रह्मात्मनस्तस्य व्यक्ताव्यक्तात्मनः परः
ब्रह्मावभाति विततो नानाशक्त्युपबृंहितः
ततोऽपरामुपादाय स सर्गाय मनो दधे
ऋषीणां भूरिवीर्याणामपि सर्गमविस्तृतम्‌
ज्ञात्वा तद्धृदये भूयश्चिन्तयामास कौरव
अहो अद्भुतमेतन्मे व्यापृतस्यापि नित्यदा
न ह्येधन्ते प्रजा नूनं दैवमत्र विघातकम्‌
एवं युक्तकृतस्तस्य दैवं चावेक्षतस्तदा
कस्य रूपमभूद्द्वेधा यत्कायमभिचक्षते
ताभ्यां रूपविभागाभ्यां मिथुनं समपद्यत
यस्तु तत्र पुमान्सोऽभून्मनुः स्वायम्भुवः स्वराट्‌
स्त्री यासीच्छतरूपाख्या महिष्यस्य महात्मनः
तदा मिथुनधर्मेण प्रजा ह्येधाम्बभूविरे
स चापि शतरूपायां पञ्चापत्यान्यजीजनत्‌
प्रियव्रतोत्तानपादौ तिस्रः कन्याश्च भारत
आकूतिर्देवहूतिश्च प्रसूतिरिति सत्तम
आकूतिं रुचये प्रादात्कर्दमाय तु मध्यमाम्‌
दक्षायादात्प्रसूतिं च यत आपूरितं जगत्‌
श्रीशुक उवाच
निशम्य वाचं वदतो मुनेः पुण्यतमां नृप
भूयः पप्रच्छ कौरव्यो वासुदेवकथादृतः
विदुर उवाच
स वै स्वायम्भुवः सम्राट्प्रियः पुत्रः स्वयम्भुवः
प्रतिलभ्य प्रियां पत्नीं किं चकार ततो मुने
चरितं तस्य राजर्षेरादिराजस्य सत्तम
ब्रूहि मे श्रद्दधानाय विष्वक्सेनाश्रयो ह्यसौ
श्रुतस्य पुंसां सुचिरश्रमस्य नन्वञ्जसा सूरिभिरीडितोऽर्थः
तत्तद्गुणानुश्रवणं मुकुन्द पादारविन्दं हृदयेषु येषाम्‌
श्रीशुक उवाच
इति ब्रुवाणं विदुरं विनीतं सहस्रशीर्ष्णश्चरणोपधानम्‌
प्रहृष्टरोमा भगवत्कथायां प्रणीयमानो मुनिरभ्यचष्ट
मैत्रेय उवाच
यदा स्वभार्यया सार्धं जातः स्वायम्भुवो मनुः
प्राञ्जलिः प्रणतश्चेदं वेदगर्भमभाषत
त्वमेकः सर्वभूतानां जन्मकृद्वृत्तिदः पिता
तथापि नः प्रजानां ते शुश्रूषा केन वा भवेत्‌
तद्विधेहि नमस्तुभ्यं कर्मस्वीड्यात्मशक्तिषु
यत्कृत्वेह यशो विष्वगमुत्र च भवेद्गतिः
ब्रह्मोवाच
प्रीतस्तुभ्यमहं तात स्वस्ति स्ताद्वां क्षितीश्वर
यन्निर्व्यलीकेन हृदा शाधि मेत्यात्मनार्पितम्‌
एतावत्यात्मजैर्वीर कार्या ह्यपचितिर्गुरौ
शक्त्याप्रमत्तैर्गृह्येत सादरं गतमत्सरैः
स त्वमस्यामपत्यानि सदृशान्यात्मनो गुणैः
उत्पाद्य शास धर्मेण गां यज्ञैः पुरुषं यज
परं शुश्रूषणं मह्यं स्यात्प्रजारक्षया नृप
भगवांस्ते प्रजाभर्तुर्हृषीकेशोऽनुतुष्यति
येषां न तुष्टो भगवान्यज्ञलिङ्गो जनार्दनः
तेषां श्रमो ह्यपार्थाय यदात्मा नादृतः स्वयम्‌
मनुरुवाच
आदेशेऽहं भगवतो वर्तेयामीवसूदन
स्थानं त्विहानुजानीहि प्रजानां मम च प्रभो
यदोकः सर्वभूतानां मही मग्ना महाम्भसि
अस्या उद्धरणे यत्नो देव देव्या विधीयताम्‌
मैत्रेय उवाच
परमेष्ठी त्वपां मध्ये तथा सन्नामवेक्ष्य गाम्‌
कथमेनां समुन्नेष्य इति दध्यौ धिया चिरम्‌
सृजतो मे क्षितिर्वार्भिः प्लाव्यमाना रसां गता
अथात्र किमनुष्ठेयमस्माभिः सर्गयोजितैः
यस्याहं हृदयादासं स ईशो विदधातु मे
इत्यभिध्यायतो नासा विवरात्सहसानघ
वराहतोको निरगादङ्गुष्ठपरिमाणकः
तस्याभिपश्यतः खस्थः क्षणेन किल भारत
गजमात्रः प्रववृधे तदद्भुतमभून्महत्‌
मरीचिप्रमुखैर्विप्रैः कुमारैर्मनुना सह
दृष्ट्वा तत्सौकरं रूपं तर्कयामास चित्रधा
किमेतत्सूकरव्याजं सत्त्वं दिव्यमवस्थितम्‌
अहो बताश्चर्यमिदं नासाया मे विनिःसृतम्‌
दृष्टोऽङ्गुष्ठशिरोमात्रः क्षणाद्गण्डशिलासमः
अपि स्विद्भगवानेष यज्ञो मे खेदयन्मनः
इति मीमांसतस्तस्य ब्रह्मणः सह सूनुभिः
भगवान्यज्ञपुरुषो जगर्जागेन्द्रसन्निभः
ब्रह्माणं हर्षयामास हरिस्तांश्च द्विजोत्तमान्‌
स्वगर्जितेन ककुभः प्रतिस्वनयता विभुः
निशम्य ते घर्घरितं स्वखेद क्षयिष्णु मायामयसूकरस्य
जनस्तपःसत्यनिवासिनस्ते त्रिभिः पवित्रैर्मुनयोऽगृणन्स्म
तेषां सतां वेदवितानमूर्तिर्ब्रह्मावधार्यात्मगुणानुवादम्‌
विनद्य भूयो विबुधोदयाय गजेन्द्रलीलो जलमाविवेश
उत्क्षिप्तवालः खचरः कठोरः सटा विधुन्वन्खररोमशत्वक्‌
खुराहताभ्रः सितदंष्ट्र ईक्षा ज्योतिर्बभासे भगवान्महीध्रः
घ्राणेन पृथ्व्याः पदवीं विजिघ्रन्क्रोडापदेशः स्वयमध्वराङ्गः
करालदंष्ट्रोऽप्यकरालदृग्भ्यामुद्वीक्ष्य विप्रान्गृणतोऽविशत्कम्‌
स वज्रकूटाङ्गनिपातवेग विशीर्णकुक्षिः स्तनयन्नुदन्वान्‌
उत्सृष्टदीर्घोर्मिभुजैरिवार्तश्चुक्रोश यज्ञेश्वर पाहि मेति
खुरैः क्षुरप्रैर्दरयंस्तदाप उत्पारपारं त्रिपरू रसायाम्‌
ददर्श गां तत्र सुषुप्सुरग्रे यां जीवधानीं स्वयमभ्यधत्त
पातालमूलेश्वरभोगसंहतौ विन्यस्य पादौ पृथिवीं च बिभ्रतः
यस्योपमानो न बभूव सोऽच्युतो ममास्तु माङ्गल्यविवृद्धये हरिः
स्वदंष्ट्रयोद्धृत्य महीं निमग्नां स उत्थितः संरुरुचे रसायाः
तत्रापि दैत्यं गदयापतन्तं सुनाभसन्दीपिततीव्रमन्युः
जघान रुन्धानमसह्यविक्रमं स लीलयेभं मृगराडिवाम्भसि
तद्रक्तपङ्काङ्कितगण्डतुण्डो यथा गजेन्द्रो जगतीं विभिन्दन्‌
तमालनीलं सितदन्तकोट्या क्ष्मामुत्क्षिपन्तं गजलीलयाङ्ग
प्रज्ञाय बद्धाञ्जलयोऽनुवाकैर्विरिञ्चिमुख्या उपतस्थुरीशम्‌
ऋषय ऊचुः
जितं जितं तेऽजित यज्ञभावन त्रयीं तनुं स्वां परिधुन्वते नमः
यद्रोमगर्तेषु निलिल्युरद्धयस्तस्मै नमः कारणसूकराय ते
रूपं तवैतन्ननु दुष्कृतात्मनां दुर्दर्शनं देव यदध्वरात्मकम्‌
छन्दांसि यस्य त्वचि बर्हिरोमस्वाज्यं दृशि त्वङ्घ्रिषु चातुर्होत्रम्‌
स्रक्तुण्ड आसीत्स्रुव ईश नासयोरिडोदरे चमसाः कर्णरन्ध्रे
प्राशित्रमास्ये ग्रसने ग्रहास्तु ते यच्चर्वणं ते भगवन्नग्निहोत्रम्‌
दीक्षानुजन्मोपसदः शिरोधरं त्वं प्रायणीयोदयनीयदंष्ट्रः
जिह्वा प्रवर्ग्यस्तव शीर्षकं क्रतोः सत्यावसथ्यं चितयोऽसवो हि ते
सोमस्तु रेतः सवनान्यवस्थितिः संस्थाविभेदास्तव देव धातवः
सत्राणि सर्वाणि शरीरसन्धिस्त्वं सर्वयज्ञक्रतुरिष्टिबन्धनः
नमो नमस्तेऽखिलमन्त्रदेवता द्रव्याय सर्वक्रतवे क्रियात्मने
वैराग्यभक्त्यात्मजयानुभावित ज्ञानाय विद्यागुरवे नमो नमः
दंष्ट्राग्रकोट्या भगवंस्त्वया धृता विराजते भूधर भूः सभूधरा
यथा वनान्निःसरतो दता धृता मतङ्गजेन्द्रस्य सपत्रपद्मिनी
त्रयीमयं रूपमिदं च सौकरं भूमण्डलेनाथ दता धृतेन ते
चकास्ति शृङ्गोढघनेन भूयसा कुलाचलेन्द्रस्य यथैव विभ्रमः
संस्थापयैनां जगतां सतस्थुषां लोकाय पत्नीमसि मातरं पिता
विधेम चास्यै नमसा सह त्वया यस्यां स्वतेजोऽग्निमिवारणावधाः
कः श्रद्दधीतान्यतमस्तव प्रभो रसां गताया भुव उद्विबर्हणम्‌
न विस्मयोऽसौ त्वयि विश्वविस्मये यो माययेदं ससृजेऽतिविस्मयम्‌
विधुन्वता वेदमयं निजं वपुर्जनस्तपःसत्यनिवासिनो वयम्‌
सटाशिखोद्धूतशिवाम्बुबिन्दुभिर्विमृज्यमाना भृशमीश पाविताः
स वै बत भ्रष्टमतिस्तवैषते यः कर्मणां पारमपारकर्मणः
यद्योगमायागुणयोगमोहितं विश्वं समस्तं भगवन्विधेहि शम्‌
मैत्रेय उवाच
इत्युपस्थीयमानोऽसौ मुनिभिर्ब्रह्मवादिभिः
सलिले स्वखुराक्रान्त उपाधत्तावितावनिम्‌
स इत्थं भगवानुर्वीं विष्वक्सेनः प्रजापतिः
रसाया लीलयोन्नीतामप्सु न्यस्य ययौ हरिः
य एवमेतां हरिमेधसो हरेः कथां सुभद्रां कथनीयमायिनः
शृण्वीत भक्त्या श्रवयेत वोशतीं जनार्दनोऽस्याशु हृदि प्रसीदति
तस्मिन्प्रसन्ने सकलाशिषां प्रभौ किं दुर्लभं ताभिरलं लवात्मभिः
अनन्यदृष्ट्या भजतां गुहाशयः स्वयं विधत्ते स्वगतिं परः पराम्‌
को नाम लोके पुरुषार्थसारवित्पुराकथानां भगवत्कथासुधाम्‌
आपीय कर्णाञ्जलिभिर्भवापहामहो विरज्येत विना नरेतरम्‌
श्रीशुक उवाच
निशम्य कौषारविणोपवर्णितां हरेः कथां कारणसूकरात्मनः
पुनः स पप्रच्छ तमुद्यताञ्जलिर्न चातितृप्तो विदुरो धृतव्रतः
विदुर उवाच
तेनैव तु मुनिश्रेष्ठ हरिणा यज्ञमूर्तिना
आदिदैत्यो हिरण्याक्षो हत इत्यनुशुश्रुम
तस्य चोद्धरतः क्षौणीं स्वदंष्ट्राग्रेण लीलया
दैत्यराजस्य च ब्रह्मन्कस्माद्धेतोरभून्मृधः
श्रद्दधानाय भक्ताय ब्रूहि तज्जन्मविस्तरम्‌
ऋषे न तृप्यति मनः परं कौतूहलं हि मे
मैत्रेय उवाच
साधु वीर त्वया पृष्टमवतारकथां हरेः
यत्त्वं पृच्छसि मर्त्यानां मृत्युपाशविशातनीम्‌
ययोत्तानपदः पुत्रो मुनिना गीतयार्भकः
मृत्योः कृत्वैव मूर्ध्न्यङ्घ्रिमारुरोह हरेः पदम्‌
अथात्रापीतिहासोऽयं श्रुतो मे वर्णितः पुरा
ब्रह्मणा देवदेवेन देवानामनुपृच्छताम्‌
दितिर्दाक्षायणी क्षत्तर्मारीचं कश्यपं पतिम्‌
अपत्यकामा चकमे सन्ध्यायां हृच्छयार्दिता
इष्ट्वाग्निजिह्वं पयसा पुरुषं यजुषां पतिम्‌
निम्लोचत्यर्क आसीनमग्न्यगारे समाहितम्‌
दितिरुवाच
एष मां त्वत्कृते विद्वन्काम आत्तशरासनः
दुनोति दीनां विक्रम्य रम्भामिव मतङ्गजः
तद्भवान्दह्यमानायां सपत्नीनां समृद्धिभिः
प्रजावतीनां भद्रं ते मय्यायुङ्क्तामनुग्रहम्‌
भर्तर्याप्तोरुमानानां लोकानाविशते यशः
पतिर्भवद्विधो यासां प्रजया ननु जायते
पुरा पिता नो भगवान्दक्षो दुहितृवत्सलः
कं वृणीत वरं वत्सा इत्यपृच्छत नः पृथक्‌
स विदित्वात्मजानां नो भावं सन्तानभावनः
त्रयोदशाददात्तासां यास्ते शीलमनुव्रताः
अथ मे कुरु कल्याणं कामं कमललोचन
आर्तोपसर्पणं भूमन्नमोघं हि महीयसि
इति तां वीर मारीचः कृपणां बहुभाषिणीम्‌
प्रत्याहानुनयन्वाचा प्रवृद्धानङ्गकश्मलाम्‌
एष तेऽहं विधास्यामि प्रियं भीरु यदिच्छसि
तस्याः कामं न कः कुर्यात्सिद्धिस्त्रैवर्गिकी यतः
सर्वाश्रमानुपादाय स्वाश्रमेण कलत्रवान्‌
व्यसनार्णवमत्येति जलयानैर्यथार्णवम्‌
यामाहुरात्मनो ह्यर्धं श्रेयस्कामस्य मानिनि
यस्यां स्वधुरमध्यस्य पुमांश्चरति विज्वरः
यामाश्रित्येन्द्रियारातीन्दुर्जयानितराश्रमैः
वयं जयेम हेलाभिर्दस्यून्दुर्गपतिर्यथा
न वयं प्रभवस्तां त्वामनुकर्तुं गृहेश्वरि
अप्यायुषा वा कार्त्स्न्येन ये चान्ये गुणगृध्नवः
अथापि काममेतं ते प्रजात्यै करवाण्यलम्‌
यथा मां नातिरोचन्ति मुहूर्तं प्रतिपालय
एषा घोरतमा वेला घोराणां घोरदर्शना
चरन्ति यस्यां भूतानि भूतेशानुचराणि ह
एतस्यां साध्वि सन्ध्यायां भगवान्भूतभावनः
परीतो भूतपर्षद्भिर्वृषेणाटति भूतराट्‌
श्मशानचक्रानिलधूलिधूम्र विकीर्णविद्योतजटाकलापः
भस्मावगुण्ठामलरुक्मदेहो देवस्त्रिभिः पश्यति देवरस्ते
न यस्य लोके स्वजनः परो वा नात्यादृतो नोत कश्चिद्विगर्ह्यः
वयं व्रतैर्यच्चरणापविद्धामाशास्महेऽजां बत भुक्तभोगाम्‌
यस्यानवद्याचरितं मनीषिणो गृणन्त्यविद्यापटलं बिभित्सवः
निरस्तसाम्यातिशयोऽपि यत्स्वयं पिशाचचर्यामचरद्गतिः सताम्‌
हसन्ति यस्याचरितं हि दुर्भगाः स्वात्मन्रतस्याविदुषः समीहितम्‌
यैर्वस्त्रमाल्याभरणानुलेपनैः श्वभोजनं स्वात्मतयोपलालितम्‌
ब्रह्मादयो यत्कृतसेतुपाला यत्कारणं विश्वमिदं च माया
आज्ञाकरी यस्य पिशाचचर्या अहो विभूम्नश्चरितं विडम्बनम्‌
मैत्रेय उवाच
सैवं संविदिते भर्त्रा मन्मथोन्मथितेन्द्रिया
जग्राह वासो ब्रह्मर्षेर्वृषलीव गतत्रपा
स विदित्वाथ भार्यायास्तं निर्बन्धं विकर्मणि
नत्वा दिष्टाय रहसि तयाथोपविवेश हि
अथोपस्पृश्य सलिलं प्राणानायम्य वाग्यतः
ध्यायञ्जजाप विरजं ब्रह्म ज्योतिः सनातनम्‌
दितिस्तु व्रीडिता तेन कर्मावद्येन भारत
उपसङ्गम्य विप्रर्षिमधोमुख्यभ्यभाषत
दितिरुवाच
न मे गर्भमिमं ब्रह्मन्भूतानामृषभोऽवधीत्‌
रुद्रः पतिर्हि भूतानां यस्याकरवमंहसम्‌
नमो रुद्राय महते देवायोग्राय मीढुषे
शिवाय न्यस्तदण्डाय धृतदण्डाय मन्यवे
स नः प्रसीदतां भामो भगवानुर्वनुग्रहः
व्याधस्याप्यनुकम्प्यानां स्त्रीणां देवः सतीपतिः
मैत्रेय उवाच
स्वसर्गस्याशिषं लोक्यामाशासानां प्रवेपतीम्‌
निवृत्तसन्ध्यानियमो भार्यामाह प्रजापतिः
कश्यप उवाच
अप्रायत्यादात्मनस्ते दोषान्मौहूर्तिकादुत
मन्निदेशातिचारेण देवानां चातिहेलनात्‌
भविष्यतस्तवाभद्रावभद्रे जाठराधमौ
लोकान्सपालांस्त्रींश्चण्डि मुहुराक्रन्दयिष्यतः
प्राणिनां हन्यमानानां दीनानामकृतागसाम्‌
स्त्रीणां निगृह्यमाणानां कोपितेषु महात्मसु
तदा विश्वेश्वरः क्रुद्धो भगवाल्लोकभावनः
हनिष्यत्यवतीर्यासौ यथाद्रीन्शतपर्वधृक्‌
दितिरुवाच
वधं भगवता साक्षात्सुनाभोदारबाहुना
आशासे पुत्रयोर्मह्यं मा क्रुद्धाद्ब्राह्मणाद्प्रभो
न ब्रह्मदण्डदग्धस्य न भूतभयदस्य च
नारकाश्चानुगृह्णन्ति यां यां योनिमसौ गतः
कश्यप उवाच
कृतशोकानुतापेन सद्यः प्रत्यवमर्शनात्‌
भगवत्युरुमानाच्च भवे मय्यपि चादरात्‌
पुत्रस्यैव च पुत्राणां भवितैकः सतां मतः
गास्यन्ति यद्यशः शुद्धं भगवद्यशसा समम्‌
योगैर्हेमेव दुर्वर्णं भावयिष्यन्ति साधवः
निर्वैरादिभिरात्मानं यच्छीलमनुवर्तितुम्‌
यत्प्रसादादिदं विश्वं प्रसीदति यदात्मकम्‌
स स्वदृग्भगवान्यस्य तोष्यतेऽनन्यया दृशा
स वै महाभागवतो महात्मा महानुभावो महतां महिष्ठः
प्रवृद्धभक्त्या ह्यनुभाविताशये निवेश्य वैकुण्ठमिमं विहास्यति
अलम्पटः शीलधरो गुणाकरो हृष्टः परर्द्ध्या व्यथितो दुःखितेषु
अभूतशत्रुर्जगतः शोकहर्ता नैदाघिकं तापमिवोडुराजः
अन्तर्बहिश्चामलमब्जनेत्रं स्वपूरुषेच्छानुगृहीतरूपम्‌
पौत्रस्तव श्रीललनाललामं द्रष्टा स्फुरत्कुण्डलमण्डिताननम्‌
मैत्रेय उवाच
श्रुत्वा भागवतं पौत्रममोदत दितिर्भृशम्‌
पुत्रयोश्च वधं कृष्णाद्विदित्वासीन्महामनाः
मैत्रेय उवाच
प्राजापत्यं तु तत्तेजः परतेजोहनं दितिः
दधार वर्षाणि शतं शङ्कमाना सुरार्दनात्‌
लोके तेनाहतालोके लोकपाला हतौजसः
न्यवेदयन्विश्वसृजे ध्वान्तव्यतिकरं दिशाम्‌
देवा ऊचुः
तम एतद्विभो वेत्थ संविग्ना यद्वयं भृशम्‌
न ह्यव्यक्तं भगवतः कालेनास्पृष्टवर्त्मनः
देवदेव जगद्धातर्लोकनाथशिखामणे
परेषामपरेषां त्वं भूतानामसि भाववित्‌
नमो विज्ञानवीर्याय माययेदमुपेयुषे
गृहीतगुणभेदाय नमस्तेऽव्यक्तयोनये
ये त्वानन्येन भावेन भावयन्त्यात्मभावनम्‌
आत्मनि प्रोतभुवनं परं सदसदात्मकम्‌
तेषां सुपक्वयोगानां जितश्वासेन्द्रियात्मनाम्‌
लब्धयुष्मत्प्रसादानां न कुतश्चित्पराभवः
यस्य वाचा प्रजाः सर्वा गावस्तन्त्येव यन्त्रिताः
हरन्ति बलिमायत्तास्तस्मै मुख्याय ते नमः
स त्वं विधत्स्व शं भूमंस्तमसा लुप्तकर्मणाम्‌
अदभ्रदयया दृष्ट्या आपन्नानर्हसीक्षितुम्‌
एष देव दितेर्गर्भ ओजः काश्यपमर्पितम्‌
दिशस्तिमिरयन्सर्वा वर्धतेऽग्निरिवैधसि
मैत्रेय उवाच
स प्रहस्य महाबाहो भगवान्शब्दगोचरः
प्रत्याचष्टात्मभूर्देवान्प्रीणन्रुचिरया गिरा
ब्रह्मोवाच
मानसा मे सुता युष्मत्‌ पूर्वजाः सनकादयः
चेरुर्विहायसा लोकाल्लोकेषु विगतस्पृहाः
त एकदा भगवतो वैकुण्ठस्यामलात्मनः
ययुर्वैकुण्ठनिलयं सर्वलोकनमस्कृतम्‌
वसन्ति यत्र पुरुषाः सर्वे वैकुण्ठमूर्तयः
येऽनिमित्तनिमित्तेन धर्मेणाराधयन्हरिम्‌
यत्र चाद्यः पुमानास्ते भगवान्शब्दगोचरः
सत्त्वं विष्टभ्य विरजं स्वानां नो मृडयन्वृषः
यत्र नैःश्रेयसं नाम वनं कामदुघैर्द्रुमैः
सर्वर्तुश्रीभिर्विभ्राजत्कैवल्यमिव मूर्तिमत्‌
वैमानिकाः सललनाश्चरितानि शश्वद्‌
गायन्ति यत्र शमलक्षपणानि भर्तुः
अन्तर्जलेऽनुविकसन्मधुमाधवीनां
गन्धेन खण्डितधियोऽप्यनिलं क्षिपन्तः
पारावतान्यभृतसारसचक्रवाक
दात्यूहहंसशुकतित्तिरिबर्हिणां यः
कोलाहलो विरमतेऽचिरमात्रमुच्चैर्‌
भृङ्गाधिपे हरिकथामिव गायमाने
मन्दारकुन्दकुरबोत्पलचम्पकार्ण
पुन्नागनागबकुलाम्बुजपारिजाताः
गन्धेऽर्चिते तुलसिकाभरणेन तस्या
यस्मिंस्तपः सुमनसो बहु मानयन्ति
यत्सङ्कुलं हरिपदानतिमात्रदृष्टैर्‌
वैदूर्यमारकतहेममयैर्विमानैः
येषां बृहत्कटितटाः स्मितशोभिमुख्यः
कृष्णात्मनां न रज आदधुरुत्स्मयाद्यैः
श्री रूपिणी क्वणयती चरणारविन्दं
लीलाम्बुजेन हरिसद्मनि मुक्तदोषा
संलक्ष्यते स्फटिककुड्य उपेतहेम्नि
सम्मार्जतीव यदनुग्रहणेऽन्ययत्नः
वापीषु विद्रुमतटास्वमलामृताप्सु
प्रेष्यान्विता निजवने तुलसीभिरीशम्‌
अभ्यर्चती स्वलकमुन्नसमीक्ष्य वक्त्रम्‌
उच्छेषितं भगवतेत्यमताङ्ग यच्छ्रीः
यन्न व्रजन्त्यघभिदो रचनानुवादाच्‌
छृण्वन्ति येऽन्यविषयाः कुकथा मतिघ्नीः
यास्तु श्रुता हतभगैर्नृभिरात्तसारास्‌
तांस्तान्क्षिपन्त्यशरणेषु तमःसु हन्त
येऽभ्यर्थितामपि च नो नृगतिं प्रपन्ना
ज्ञानं च तत्त्वविषयं सहधर्मं यत्र
नाराधनं भगवतो वितरन्त्यमुष्य
सम्मोहिता विततया बत मायया ते
यच्च व्रजन्त्यनिमिषामृषभानुवृत्त्या
दूरे यमा ह्युपरि नः स्पृहणीयशीलाः
भर्तुर्मिथः सुयशसः कथनानुराग
वैक्लव्यबाष्पकलया पुलकीकृताङ्गाः
तद्विश्वगुर्वधिकृतं भुवनैकवन्द्यं
दिव्यं विचित्रविबुधाग्र्यविमानशोचिः
आपुः परां मुदमपूर्वमुपेत्य योग
मायाबलेन मुनयस्तदथो विकुण्ठम्‌
तस्मिन्नतीत्य मुनयः षडसज्जमानाः
कक्षाः समानवयसावथ सप्तमायाम्‌
देवावचक्षत गृहीतगदौ परार्ध्य
केयूरकुण्डलकिरीटविटङ्कवेषौ
मत्तद्विरेफवनमालिकया निवीतौ
विन्यस्तयासितचतुष्टयबाहुमध्ये
वक्त्रं भ्रुवा कुटिलया स्फुटनिर्गमाभ्यां
रक्तेक्षणेन च मनाग्रभसं दधानौ
द्वार्येतयोर्निविविशुर्मिषतोरपृष्ट्वा
पूर्वा यथा पुरटवज्रकपाटिका याः
सर्वत्र तेऽविषमया मुनयः स्वदृष्ट्या
ये सञ्चरन्त्यविहता विगताभिशङ्काः
तान्वीक्ष्य वातरशनांश्चतुरः कुमारान्‌
वृद्धान्दशार्धवयसो विदितात्मतत्त्वान्‌
वेत्रेण चास्खलयतामतदर्हणांस्तौ
तेजो विहस्य भगवत्प्रतिकूलशीलौ
ताभ्यां मिषत्स्वनिमिषेषु निषिध्यमानाः
स्वर्हत्तमा ह्यपि हरेः प्रतिहारपाभ्याम्‌
ऊचुः सुहृत्तमदिदृक्षितभङ्ग ईषत्‌
कामानुजेन सहसा त उपप्लुताक्षाः
मुनय ऊचुः
को वामिहैत्य भगवत्परिचर्ययोच्चैस्‌
तद्धर्मिणां निवसतां विषमः स्वभावः
तस्मिन्प्रशान्तपुरुषे गतविग्रहे वां
को वात्मवत्कुहकयोः परिशङ्कनीयः
न ह्यन्तरं भगवतीह समस्तकुक्षाव्‌
आत्मानमात्मनि नभो नभसीव धीराः
पश्यन्ति यत्र युवयोः सुरलिङ्गिनोः किं
व्युत्पादितं ह्युदरभेदि भयं यतोऽस्य
तद्वाममुष्य परमस्य विकुण्ठभर्तुः
कर्तुं प्रकृष्टमिह धीमहि मन्दधीभ्याम्‌
लोकानितो व्रजतमन्तरभावदृष्ट्या
पापीयसस्त्रय इमे रिपवोऽस्य यत्र
तेषामितीरितमुभाववधार्य घोरं
तं ब्रह्मदण्डमनिवारणमस्त्रपूगैः
सद्यो हरेरनुचरावुरु बिभ्यतस्तत्‌
पादग्रहावपततामतिकातरेण
भूयादघोनि भगवद्भिरकारि दण्डो
यो नौ हरेत सुरहेलनमप्यशेषम्‌
मा वोऽनुतापकलया भगवत्स्मृतिघ्नो
मोहो भवेदिह तु नौ व्रजतोरधोऽधः
एवं तदैव भगवानरविन्दनाभः
स्वानां विबुध्य सदतिक्रममार्यहृद्यः
तस्मिन्ययौ परमहंसमहामुनीनाम्‌
अन्वेषणीयचरणौ चलयन्सहश्रीः
तं त्वागतं प्रतिहृतौपयिकं स्वपुम्भिस्‌
तेऽचक्षताक्षविषयं स्वसमाधिभाग्यम्‌
हंसश्रियोर्व्यजनयोः शिववायुलोलच्‌
छुभ्रातपत्रशशिकेसरशीकराम्बुम्‌
कृत्स्नप्रसादसुमुखं स्पृहणीयधाम
स्नेहावलोककलया हृदि संस्पृशन्तम्‌
श्यामे पृथावुरसि शोभितया श्रिया स्वश्‌
चूडामणिं सुभगयन्तमिवात्मधिष्ण्यम्‌
पीतांशुके पृथुनितम्बिनि विस्फुरन्त्या
काञ्च्यालिभिर्विरुतया वनमालया च
वल्गुप्रकोष्ठवलयं विनतासुतांसे
विन्यस्तहस्तमितरेण धुनानमब्जम्‌
विद्युत्क्षिपन्मकरकुण्डलमण्डनार्ह
गण्डस्थलोन्नसमुखं मणिमत्किरीटम्‌
दोर्दण्डषण्डविवरे हरता परार्ध्य
हारेण कन्धरगतेन च कौस्तुभेन
अत्रोपसृष्टमिति चोत्स्मितमिन्दिरायाः
स्वानां धिया विरचितं बहुसौष्ठवाढ्यम्‌
मह्यं भवस्य भवतां च भजन्तमङ्गं
नेमुर्निरीक्ष्य न वितृप्तदृशो मुदा कैः
तस्यारविन्दनयनस्य पदारविन्द
किञ्जल्कमिश्रतुलसीमकरन्दवायुः
अन्तर्गतः स्वविवरेण चकार तेषां
सङ्क्षोभमक्षरजुषामपि चित्ततन्वोः
ते वा अमुष्य वदनासितपद्मकोशम्‌
उद्वीक्ष्य सुन्दरतराधरकुन्दहासम्‌
लब्धाशिषः पुनरवेक्ष्य तदीयमङ्घ्रि
द्वन्द्वं नखारुणमणिश्रयणं निदध्युः
पुंसां गतिं मृगयतामिह योगमार्गैर्‌
ध्यानास्पदं बहुमतं नयनाभिरामम्‌
पौंस्नं वपुर्दर्शयानमनन्यसिद्धैर्‌
औत्पत्तिकैः समगृणन्युतमष्टभोगैः
कुमारा ऊचुः
योऽन्तर्हितो हृदि गतोऽपि दुरात्मनां त्वं
सोऽद्यैव नो नयनमूलमनन्त राद्धः
यर्ह्येव कर्णविवरेण गुहां गतो नः
पित्रानुवर्णितरहा भवदुद्भवेन
तं त्वां विदाम भगवन्परमात्मतत्त्वं
सत्त्वेन सम्प्रति रतिं रचयन्तमेषाम्‌
यत्तेऽनुतापविदितैर्दृढभक्तियोगैर्‌
उद्ग्रन्थयो हृदि विदुर्मुनयो विरागाः
नात्यन्तिकं विगणयन्त्यपि ते प्रसादं
किम्वन्यदर्पितभयं भ्रुव उन्नयैस्ते
येऽङ्ग त्वदङ्घ्रिशरणा भवतः कथायाः
कीर्तन्यतीर्थयशसः कुशला रसज्ञाः
कामं भवः स्ववृजिनैर्निरयेषु नः स्ताच्‌
चेतोऽलिवद्यदि नु ते पदयो रमेत
वाचश्च नस्तुलसिवद्यदि तेऽङ्घ्रिशोभाः
पूर्येत ते गुणगणैर्यदि कर्णरन्ध्रः
प्रादुश्चकर्थ यदिदं पुरुहूत रूपं
तेनेश निर्वृतिमवापुरलं दृशो नः
तस्मा इदं भगवते नम इद्विधेम
योऽनात्मनां दुरुदयो भगवान्प्रतीतः
ब्रह्मोवाच
इति तद्गृणतां तेषां मुनीनां योगधर्मिणाम्‌
प्रतिनन्द्य जगादेदं विकुण्ठनिलयो विभुः
श्रीभगवानुवाच
एतौ तौ पार्षदौ मह्यं जयो विजय एव च
कदर्थीकृत्य मां यद्वो बह्वक्रातामतिक्रमम्‌
यस्त्वेतयोर्धृतो दण्डो भवद्भिर्मामनुव्रतैः
स एवानुमतोऽस्माभिर्मुनयो देवहेलनात्‌
तद्वः प्रसादयाम्यद्य ब्रह्म दैवं परं हि मे
तद्धीत्यात्मकृतं मन्ये यत्स्वपुम्भिरसत्कृताः
यन्नामानि च गृह्णाति लोको भृत्ये कृतागसि
सोऽसाधुवादस्तत्कीर्तिं हन्ति त्वचमिवामयः
यस्यामृतामलयशःश्रवणावगाहः
सद्यः पुनाति जगदाश्वपचाद्विकुण्ठः
सोऽहं भवद्भ्य उपलब्धसुतीर्थकीर्तिश्‌
छिन्द्यां स्वबाहुमपि वः प्रतिकूलवृत्तिम्‌
यत्सेवया चरणपद्मपवित्ररेणुं
सद्यः क्षताखिलमलं प्रतिलब्धशीलम्‌
न श्रीर्विरक्तमपि मां विजहाति यस्याः
प्रेक्षालवार्थ इतरे नियमान्वहन्ति
नाहं तथाद्मि यजमानहविर्विताने
श्च्योतद्घृतप्लुतमदन्हुतभुङ्मुखेन
यद्ब्राह्मणस्य मुखतश्चरतोऽनुघासं
तुष्टस्य मय्यवहितैर्निजकर्मपाकैः
येषां बिभर्म्यहमखण्डविकुण्ठयोग
मायाविभूतिरमलाङ्घ्रिरजः किरीटैः
विप्रांस्तु को न विषहेत यदर्हणाम्भः
सद्यः पुनाति सहचन्द्रललामलोकान्‌
ये मे तनूर्द्विजवरान्दुहतीर्मदीया
भूतान्यलब्धशरणानि च भेदबुद्ध्या
द्रक्ष्यन्त्यघक्षतदृशो ह्यहिमन्यवस्तान्‌
गृध्रा रुषा मम कुषन्त्यधिदण्डनेतुः
ये ब्राह्मणान्मयि धिया क्षिपतोऽर्चयन्तस्‌
तुष्यद्धृदः स्मितसुधोक्षितपद्मवक्त्राः
वाण्यानुरागकलयात्मजवद्गृणन्तः
सम्बोधयन्त्यहमिवाहमुपाहृतस्तैः
तन्मे स्वभर्तुरवसायमलक्षमाणौ
युष्मद्व्यतिक्रमगतिं प्रतिपद्य सद्यः
भूयो ममान्तिकमितां तदनुग्रहो मे
यत्कल्पतामचिरतो भृतयोर्विवासः
ब्रह्मोवाच
अथ तस्योशतीं देवीमृषिकुल्यां सरस्वतीम्‌
नास्वाद्य मन्युदष्टानां तेषामात्माप्यतृप्यत
सतीं व्यादाय शृण्वन्तो लघ्वीं गुर्वर्थगह्वराम्‌
विगाह्यागाधगम्भीरां न विदुस्तच्चिकीर्षितम्‌
ते योगमाययारब्ध पारमेष्ठ्यमहोदयम्‌
प्रोचुः प्राञ्जलयो विप्राः प्रहृष्टाः क्षुभितत्वचः
ऋषय ऊचुः
न वयं भगवन्विद्मस्तव देव चिकीर्षितम्‌
कृतो मेऽनुग्रहश्चेति यदध्यक्षः प्रभाषसे
ब्रह्मण्यस्य परं दैवं ब्राह्मणाः किल ते प्रभो
विप्राणां देवदेवानां भगवानात्मदैवतम्‌
त्वत्तः सनातनो धर्मो रक्ष्यते तनुभिस्तव
धर्मस्य परमो गुह्यो निर्विकारो भवान्मतः
तरन्ति ह्यञ्जसा मृत्युं निवृत्ता यदनुग्रहात्‌
योगिनः स भवान्किं स्विदनुगृह्येत यत्परैः
यं वै विभूतिरुपयात्यनुवेलमन्यैर्‌
अर्थार्थिभिः स्वशिरसा धृतपादरेणुः
धन्यार्पिताङ्घ्रितुलसीनवदामधाम्नो
लोकं मधुव्रतपतेरिव कामयाना
यस्तां विविक्तचरितैरनुवर्तमानां
नात्याद्रियत्परमभागवतप्रसङ्गः
स त्वं द्विजानुपथपुण्यरजःपुनीतः
श्रीवत्सलक्ष्म किमगा भगभाजनस्त्वम्‌
धर्मस्य ते भगवतस्त्रियुग त्रिभिः स्वैः
पद्भिश्चराचरमिदं द्विजदेवतार्थम्‌
नूनं भृतं तदभिघाति रजस्तमश्च
सत्त्वेन नो वरदया तनुवा निरस्य
न त्वं द्विजोत्तमकुलं यदि हात्मगोपं
गोप्ता वृषः स्वर्हणेन ससूनृतेन
तर्ह्येव नङ्क्ष्यति शिवस्तव देव पन्था
लोकोऽग्रहीष्यदृषभस्य हि तत्प्रमाणम्‌
तत्तेऽनभीष्टमिव सत्त्वनिधेर्विधित्सोः
क्षेमं जनाय निजशक्तिभिरुद्धृतारेः
नैतावता त्र्यधिपतेर्बत विश्वभर्तुस्‌
तेजः क्षतं त्ववनतस्य स ते विनोदः
यं वानयोर्दममधीश भवान्विधत्ते
वृत्तिं नु वा तदनुमन्महि निर्व्यलीकम्‌
अस्मासु वा य उचितो ध्रियतां स दण्डो
येऽनागसौ वयमयुङ्क्ष्महि किल्बिषेण
श्रीभगवानुवाच
एतौ सुरेतरगतिं प्रतिपद्य सद्यः
संरम्भसम्भृतसमाध्यनुबद्धयोगौ
भूयः सकाशमुपयास्यत आशु यो वः
शापो मयैव निमितस्तदवेत विप्राः
ब्रह्मोवाच
अथ ते मुनयो दृष्ट्वा नयनानन्दभाजनम्‌
वैकुण्ठं तदधिष्ठानं विकुण्ठं च स्वयंप्रभम्‌
भगवन्तं परिक्रम्य प्रणिपत्यानुमान्य च
प्रतिजग्मुः प्रमुदिताः शंसन्तो वैष्णवीं श्रियम्‌
भगवाननुगावाह यातं मा भैष्टमस्तु शम्‌
ब्रह्मतेजः समर्थोऽपि हन्तुं नेच्छे मतं तु मे
एतत्पुरैव निर्दिष्टं रमया क्रुद्धया यदा
पुरापवारिता द्वारि विशन्ती मय्युपारते
मयि संरम्भयोगेन निस्तीर्य ब्रह्महेलनम्‌
प्रत्येष्यतं निकाशं मे कालेनाल्पीयसा पुनः
द्वाःस्थावादिश्य भगवान्विमानश्रेणिभूषणम्‌
सर्वातिशयया लक्ष्म्या जुष्टं स्वं धिष्ण्यमाविशत्‌
तौ तु गीर्वाणऋषभौ दुस्तराद्धरिलोकतः
हतश्रियौ ब्रह्मशापादभूतां विगतस्मयौ
तदा विकुण्ठधिषणात्तयोर्निपतमानयोः
हाहाकारो महानासीद्विमानाग्र्येषु पुत्रकाः
तावेव ह्यधुना प्राप्तौ पार्षदप्रवरौ हरेः
दितेर्जठरनिर्विष्टं काश्यपं तेज उल्बणम्‌
तयोरसुरयोरद्य तेजसा यमयोर्हि वः
आक्षिप्तं तेज एतर्हि भगवांस्तद्विधित्सति
विश्वस्य यः स्थितिलयोद्भवहेतुराद्यो
योगेश्वरैरपि दुरत्यययोगमायः
क्षेमं विधास्यति स नो भगवांस्त्र्यधीशस्‌
तत्रास्मदीयविमृशेन कियानिहार्थः
मैत्रेय उवाच
निशम्यात्मभुवा गीतं कारणं शङ्कयोज्झिताः
ततः सर्वे न्यवर्तन्त त्रिदिवाय दिवौकसः
दितिस्तु भर्तुरादेशादपत्यपरिशङ्किनी
पूर्णे वर्षशते साध्वी पुत्रौ प्रसुषुवे यमौ
उत्पाता बहवस्तत्र निपेतुर्जायमानयोः
दिवि भुव्यन्तरिक्षे च लोकस्योरुभयावहाः
सहाचला भुवश्चेलुर्दिशः सर्वाः प्रजज्वलुः
सोल्काश्चाशनयः पेतुः केतवश्चार्तिहेतवः
ववौ वायुः सुदुःस्पर्शः फूत्कारानीरयन्मुहुः
उन्मूलयन्नगपतीन्वात्यानीको रजोध्वजः
उद्धसत्तडिदम्भोद घटया नष्टभागणे
व्योम्नि प्रविष्टतमसा न स्म व्यादृश्यते पदम्‌
चुक्रोश विमना वार्धिरुदूर्मिः क्षुभितोदरः
सोदपानाश्च सरितश्चुक्षुभुः शुष्कपङ्कजाः
मुहुः परिधयोऽभूवन्सराह्वोः शशिसूर्ययोः
निर्घाता रथनिर्ह्रादा विवरेभ्यः प्रजज्ञिरे
अन्तर्ग्रामेषु मुखतो वमन्त्यो वह्निमुल्बणम्‌
सृगालोलूकटङ्कारैः प्रणेदुरशिवं शिवाः
सङ्गीतवद्रोदनवदुन्नमय्य शिरोधराम्‌
व्यमुञ्चन्विविधा वाचो ग्रामसिंहास्ततस्ततः
खराश्च कर्कशैः क्षत्तः खुरैर्घ्नन्तो धरातलम्‌
खार्काररभसा मत्ताः पर्यधावन्वरूथशः
रुदन्तो रासभत्रस्ता नीडादुदपतन्खगाः
घोषेऽरण्ये च पशवः शकृन्मूत्रमकुर्वत
गावोऽत्रसन्नसृग्दोहास्तोयदाः पूयवर्षिणः
व्यरुदन्देवलिङ्गानि द्रुमाः पेतुर्विनानिलम्‌
ग्रहान्पुण्यतमानन्ये भगणांश्चापि दीपिताः
अतिचेरुर्वक्रगत्या युयुधुश्च परस्परम्‌
दृष्ट्वान्यांश्च महोत्पातानतत्तत्त्वविदः प्रजाः
ब्रह्मपुत्रानृते भीता मेनिरे विश्वसम्प्लवम्‌
तावादिदैत्यौ सहसा व्यज्यमानात्मपौरुषौ
ववृधातेऽश्मसारेण कायेनाद्रिपती इव
दिविस्पृशौ हेमकिरीटकोटिभिर्निरुद्धकाष्ठौ स्फुरदङ्गदाभुजौ
गां कम्पयन्तौ चरणैः पदे पदे कट्या सुकाञ्च्यार्कमतीत्य तस्थतुः
प्रजापतिर्नाम तयोरकार्षीद्यः प्राक्स्वदेहाद्यमयोरजायत
तं वै हिरण्यकशिपुं विदुः प्रजा यं तं हिरण्याक्षमसूत साग्रतः
चक्रे हिरण्यकशिपुर्दोर्भ्यां ब्रह्मवरेण च
वशे सपालान्लोकांस्त्रीनकुतोमृत्युरुद्धतः
हिरण्याक्षोऽनुजस्तस्य प्रियः प्रीतिकृदन्वहम्‌
गदापाणिर्दिवं यातो युयुत्सुर्मृगयन्रणम्‌
तं वीक्ष्य दुःसहजवं रणत्काञ्चननूपुरम्‌
वैजयन्त्या स्रजा जुष्टमंसन्यस्तमहागदम्‌
मनोवीर्यवरोत्सिक्तमसृण्यमकुतोभयम्‌
भीता निलिल्यिरे देवास्तार्क्ष्यत्रस्ता इवाहयः
स वै तिरोहितान्दृष्ट्वा महसा स्वेन दैत्यराट्‌
सेन्द्रान्देवगणान्क्षीबानपश्यन्व्यनदद्भृशम्‌
ततो निवृत्तः क्रीडिष्यन्गम्भीरं भीमनिस्वनम्‌
विजगाहे महासत्त्वो वार्धिं मत्त इव द्विपः
तस्मिन्प्रविष्टे वरुणस्य सैनिका यादोगणाः सन्नधियः ससाध्वसाः
अहन्यमाना अपि तस्य वर्चसा प्रधर्षिता दूरतरं प्रदुद्रुवुः
स वर्षपूगानुदधौ महाबलश्चरन्महोर्मीञ्छ्वसनेरितान्मुहुः
मौर्व्याभिजघ्ने गदया विभावरीमासेदिवांस्तात पुरीं प्रचेतसः
तत्रोपलभ्यासुरलोकपालकं यादोगणानामृषभं प्रचेतसम्‌
स्मयन्प्रलब्धुं प्रणिपत्य नीचवज्जगाद मे देह्यधिराज संयुगम्‌
त्वं लोकपालोऽधिपतिर्बृहच्छ्रवा वीर्यापहो दुर्मदवीरमानिनाम्‌
विजित्य लोकेऽखिलदैत्यदानवान्यद्राजसूयेन पुरायजत्प्रभो
स एवमुत्सिक्तमदेन विद्विषा दृढं प्रलब्धो भगवानपां पतिः
रोषं समुत्थं शमयन्स्वया धिया व्यवोचदङ्गोपशमं गता वयम्‌
पश्यामि नान्यं पुरुषात्पुरातनाद्यः संयुगे त्वां रणमार्गकोविदम्‌
आराधयिष्यत्यसुरर्षभेहि तं मनस्विनो यं गृणते भवादृशाः
तं वीरमारादभिपद्य विस्मयः शयिष्यसे वीरशये श्वभिर्वृतः
यस्त्वद्विधानामसतां प्रशान्तये रूपाणि धत्ते सदनुग्रहेच्छया
मैत्रेय उवाच
तदेवमाकर्ण्य जलेशभाषितं महामनास्तद्विगणय्य दुर्मदः
हरेर्विदित्वा गतिमङ्ग नारदाद्रसातलं निर्विविशे त्वरान्वितः
ददर्श तत्राभिजितं धराधरं प्रोन्नीयमानावनिमग्रदंष्ट्रया
मुष्णन्तमक्ष्णा स्वरुचोऽरुणश्रिया जहास चाहो वनगोचरो मृगः
आहैनमेह्यज्ञ महीं विमुञ्च नो रसौकसां विश्वसृजेयमर्पिता
न स्वस्ति यास्यस्यनया ममेक्षतः सुराधमासादितसूकराकृते
त्वं नः सपत्नैरभवाय किं भृतो यो मायया हन्त्यसुरान्परोक्षजित्‌
त्वां योगमायाबलमल्पपौरुषं संस्थाप्य मूढ प्रमृजे सुहृच्छुचः
त्वयि संस्थिते गदया शीर्णशीर्षण्यस्मद्भुजच्युतया ये च तुभ्यम्‌
बलिं हरन्त्यृषयो ये च देवाः स्वयं सर्वे न भविष्यन्त्यमूलाः
स तुद्यमानोऽरिदुरुक्ततोमरैर्दंष्ट्राग्रगां गामुपलक्ष्य भीताम्‌
तोदं मृषन्निरगादम्बुमध्याद्ग्राहाहतः सकरेणुर्यथेभः
तं निःसरन्तं सलिलादनुद्रुतो हिरण्यकेशो द्विरदं यथा झषः
करालदंष्ट्रोऽशनिनिस्वनोऽब्रवीद्गतह्रियां किं त्वसतां विगर्हितम्‌
स गामुदस्तात्सलिलस्य गोचरे विन्यस्य तस्यामदधात्स्वसत्त्वम्‌
अभिष्टुतो विश्वसृजा प्रसूनैरापूर्यमाणो विबुधैः पश्यतोऽरेः
परानुषक्तं तपनीयोपकल्पं महागदं काञ्चनचित्रदंशम्‌
मर्माण्यभीक्ष्णं प्रतुदन्तं दुरुक्तैः प्रचण्डमन्युः प्रहसंस्तं बभाषे
श्रीभगवानुवाच
सत्यं वयं भो वनगोचरा मृगा युष्मद्विधान्मृगये ग्रामसिंहान्‌
न मृत्युपाशैः प्रतिमुक्तस्य वीरा विकत्थनं तव गृह्णन्त्यभद्र
एते वयं न्यासहरा रसौकसां गतह्रियो गदया द्रावितास्ते
तिष्ठामहेऽथापि कथञ्चिदाजौ स्थेयं क्व यामो बलिनोत्पाद्य वैरम्‌
त्वं पद्रथानां किल यूथपाधिपो घटस्व नोऽस्वस्तय आश्वनूहः
संस्थाप्य चास्मान्प्रमृजाश्रु स्वकानां यः स्वां प्रतिज्ञां नातिपिपर्त्यसभ्यः
मैत्रेय उवाच
सोऽधिक्षिप्तो भगवता प्रलब्धश्च रुषा भृशम्‌
आजहारोल्बणं क्रोधं क्रीड्यमानोऽहिराडिव
सृजन्नमर्षितः श्वासान्मन्युप्रचलितेन्द्रियः
आसाद्य तरसा दैत्यो गदया न्यहनद्धरिम्‌
भगवांस्तु गदावेगं विसृष्टं रिपुणोरसि
अवञ्चयत्तिरश्चीनो योगारूढ इवान्तकम्‌
पुनर्गदां स्वामादाय भ्रामयन्तमभीक्ष्णशः
अभ्यधावद्धरिः क्रुद्धः संरम्भाद्दष्टदच्छदम्‌
ततश्च गदयारातिं दक्षिणस्यां भ्रुवि प्रभुः
आजघ्ने स तु तां सौम्य गदया कोविदोऽहनत्‌
एवं गदाभ्यां गुर्वीभ्यां हर्यक्षो हरिरेव च
जिगीषया सुसंरब्धावन्योन्यमभिजघ्नतुः
तयोः स्पृधोस्तिग्मगदाहताङ्गयोः क्षतास्रवघ्राणविवृद्धमन्य्वोः
विचित्रमार्गांश्चरतोर्जिगीषया व्यभादिलायामिव शुष्मिणोर्मृधः
दैत्यस्य यज्ञावयवस्य माया गृहीतवाराहतनोर्महात्मनः
कौरव्य मह्यां द्विषतोर्विमर्दनं दिदृक्षुरागादृषिभिर्वृतः स्वराट्‌
आसन्नशौण्डीरमपेतसाध्वसं कृतप्रतीकारमहार्यविक्रमम्‌
विलक्ष्य दैत्यं भगवान्सहस्रणीर्जगाद नारायणमादिसूकरम्‌
ब्रह्मोवाच
एष ते देव देवानामङ्घ्रिमूलमुपेयुषाम्‌
विप्राणां सौरभेयीणां भूतानामप्यनागसाम्‌
आगस्कृद्भयकृद्दुष्कृदस्मद्राद्धवरोऽसुरः
अन्वेषन्नप्रतिरथो लोकानटति कण्टकः
मैनं मायाविनं दृप्तं निरङ्कुशमसत्तमम्‌
आक्रीड बालवद्देव यथाशीविषमुत्थितम्‌
न यावदेष वर्धेत स्वां वेलां प्राप्य दारुणः
स्वां देव मायामास्थाय तावज्जह्यघमच्युत
एषा घोरतमा सन्ध्या लोकच्छम्बट्करी प्रभो
उपसर्पति सर्वात्मन्सुराणां जयमावह
अधुनैषोऽभिजिन्नाम योगो मौहूर्तिको ह्यगात्‌
शिवाय नस्त्वं सुहृदामाशु निस्तर दुस्तरम्‌
दिष्ट्या त्वां विहितं मृत्युमयमासादितः स्वयम्‌
विक्रम्यैनं मृधे हत्वा लोकानाधेहि शर्मणि
मैत्रेय उवाच
अवधार्य विरिञ्चस्य निर्व्यलीकामृतं वचः
प्रहस्य प्रेमगर्भेण तदपाङ्गेन सोऽग्रहीत्‌
ततः सपत्नं मुखतश्चरन्तमकुतोभयम्‌
जघानोत्पत्य गदया हनावसुरमक्षजः
सा हता तेन गदया विहता भगवत्करात्‌
विघूर्णितापतद्रेजे तदद्भुतमिवाभवत्‌
स तदा लब्धतीर्थोऽपि न बबाधे निरायुधम्‌
मानयन्स मृधे धर्मं विष्वक्सेनं प्रकोपयन्‌
गदायामपविद्धायां हाहाकारे विनिर्गते
मानयामास तद्धर्मं सुनाभं चास्मरद्विभुः
तं व्यग्रचक्रं दितिपुत्राधमेन स्वपार्षदमुख्येन विषज्जमानम्‌
चित्रा वाचोऽतद्विदां खेचराणां तत्र स्मासन्स्वस्ति तेऽमुं जहीति
स तं निशाम्यात्तरथाङ्गमग्रतो व्यवस्थितं पद्मपलाशलोचनम्‌
विलोक्य चामर्षपरिप्लुतेन्द्रियो रुषा स्वदन्तच्छदमादशच्छ्वसन्‌
करालदंष्ट्रश्चक्षुर्भ्यां सञ्चक्षाणो दहन्निव
अभिप्लुत्य स्वगदया हतोऽसीत्याहनद्धरिम्‌
पदा सव्येन तां साधो भगवान्यज्ञसूकरः
लीलया मिषतः शत्रोः प्राहरद्वातरंहसम्‌
आह चायुधमाधत्स्व घटस्व त्वं जिगीषसि
इत्युक्तः स तदा भूयस्ताडयन्व्यनदद्भृशम्‌
तां स आपततीं वीक्ष्य भगवान्समवस्थितः
जग्राह लीलया प्राप्तां गरुत्मानिव पन्नगीम्‌
स्वपौरुषे प्रतिहते हतमानो महासुरः
नैच्छद्गदां दीयमानां हरिणा विगतप्रभः
जग्राह त्रिशिखं शूलं ज्वलज्ज्वलनलोलुपम्‌
यज्ञाय धृतरूपाय विप्रायाभिचरन्यथा
तदोजसा दैत्यमहाभटार्पितं चकासदन्तःख उदीर्णदीधिति
चक्रेण चिच्छेद निशातनेमिना हरिर्यथा तार्क्ष्यपतत्रमुज्झितम्‌
वृक्णे स्वशूले बहुधारिणा हरेः प्रत्येत्य विस्तीर्णमुरो विभूतिमत्‌
प्रवृद्धरोषः स कठोरमुष्टिना नदन्प्रहृत्यान्तरधीयतासुरः
तेनेत्थमाहतः क्षत्तर्भगवानादिसूकरः
नाकम्पत मनाक्क्वापि स्रजा हत इव द्विपः
अथोरुधासृजन्मायां योगमायेश्वरे हरौ
यां विलोक्य प्रजास्त्रस्ता मेनिरेऽस्योपसंयमम्‌
प्रववुर्वायवश्चण्डास्तमः पांसवमैरयन्‌
दिग्भ्यो निपेतुर्ग्रावाणः क्षेपणैः प्रहिता इव
द्यौर्नष्टभगणाभ्रौघैः सविद्युत्स्तनयित्नुभिः
वर्षद्भिः पूयकेशासृग्‌ विण्मूत्रास्थीनि चासकृत्‌
गिरयः प्रत्यदृश्यन्त नानायुधमुचोऽनघ
दिग्वाससो यातुधान्यः शूलिन्यो मुक्तमूर्धजाः
बहुभिर्यक्षरक्षोभिः पत्त्यश्वरथकुञ्जरैः
आततायिभिरुत्सृष्टा हिंस्रा वाचोऽतिवैशसाः
प्रादुष्कृतानां मायानामासुरीणां विनाशयत्‌
सुदर्शनास्त्रं भगवान्प्रायुङ्क्त दयितं त्रिपात्‌
तदा दितेः समभवत्सहसा हृदि वेपथुः
स्मरन्त्या भर्तुरादेशं स्तनाच्चासृक्प्रसुस्रुवे
विनष्टासु स्वमायासु भूयश्चाव्रज्य केशवम्‌
रुषोपगूहमानोऽमुं ददृशेऽवस्थितं बहिः
तं मुष्टिभिर्विनिघ्नन्तं वज्रसारैरधोक्षजः
करेण कर्णमूलेऽहन्यथा त्वाष्ट्रं मरुत्पतिः
स आहतो विश्वजिता ह्यवज्ञया परिभ्रमद्गात्र उदस्तलोचनः
विशीर्णबाह्वङ्घ्रिशिरोरुहोऽपतद्यथा नगेन्द्रो लुलितो नभस्वता
क्षितौ शयानं तमकुण्ठवर्चसं करालदंष्ट्रं परिदष्टदच्छदम्‌
अजादयो वीक्ष्य शशंसुरागता अहो इमं को नु लभेत संस्थितिम्‌
यं योगिनो योगसमाधिना रहो ध्यायन्ति लिङ्गादसतो मुमुक्षया
तस्यैष दैत्यऋषभः पदाहतो मुखं प्रपश्यंस्तनुमुत्ससर्ज ह
एतौ तौ पार्षदावस्य शापाद्यातावसद्गतिम्‌
पुनः कतिपयैः स्थानं प्रपत्स्येते ह जन्मभिः
देवा ऊचुः
नमो नमस्तेऽखिलयज्ञतन्तवे स्थितौ गृहीतामलसत्त्वमूर्तये
दिष्ट्या हतोऽयं जगतामरुन्तुदस्त्वत्पादभक्त्या वयमीश निर्वृताः
मैत्रेय उवाच
एवं हिरण्याक्षमसह्यविक्रमं स सादयित्वा हरिरादिसूकरः
जगाम लोकं स्वमखण्डितोत्सवं समीडितः पुष्करविष्टरादिभिः
मया यथानूक्तमवादि ते हरेः कृतावतारस्य सुमित्र चेष्टितम्‌
यथा हिरण्याक्ष उदारविक्रमो महामृधे क्रीडनवन्निराकृतः
सूत उवाच
इति कौषारवाख्यातामाश्रुत्य भगवत्कथाम्‌
क्षत्तानन्दं परं लेभे महाभागवतो द्विज
अन्येषां पुण्यश्लोकानामुद्दामयशसां सताम्‌
उपश्रुत्य भवेन्मोदः श्रीवत्साङ्कस्य किं पुनः
यो गजेन्द्रं झषग्रस्तं ध्यायन्तं चरणाम्बुजम्‌
क्रोशन्तीनां करेणूनां कृच्छ्रतोऽमोचयद्द्रुतम्‌
तं सुखाराध्यमृजुभिरनन्यशरणैर्नृभिः
कृतज्ञः को न सेवेत दुराराध्यमसाधुभिः
यो वै हिरण्याक्षवधं महाद्भुतं विक्रीडितं कारणसूकरात्मनः
शृणोति गायत्यनुमोदतेऽञ्जसा विमुच्यते ब्रह्मवधादपि द्विजाः
एतन्महापुण्यमलं पवित्रं धन्यं यशस्यं पदमायुराशिषाम्‌
प्राणेन्द्रियाणां युधि शौर्यवर्धनं नारायणोऽन्ते गतिरङ्ग शृण्वताम्‌
शौनक उवाच
महीं प्रतिष्ठामध्यस्य सौते स्वायम्भुवो मनुः
कान्यन्वतिष्ठद्द्वाराणि मार्गायावरजन्मनाम्‌
क्षत्ता महाभागवतः कृष्णस्यैकान्तिकः सुहृत्‌
यस्तत्याजाग्रजं कृष्णे सापत्यमघवानिति
द्वैपायनादनवरो महित्वे तस्य देहजः
सर्वात्मना श्रितः कृष्णं तत्परांश्चाप्यनुव्रतः
किमन्वपृच्छन्मैत्रेयं विरजास्तीर्थसेवया
उपगम्य कुशावर्त आसीनं तत्त्ववित्तमम्‌
तयोः संवदतोः सूत प्रवृत्ता ह्यमलाः कथाः
आपो गाङ्गा इवाघघ्नीर्हरेः पादाम्बुजाश्रयाः
ता नः कीर्तय भद्रं ते कीर्तन्योदारकर्मणः
रसज्ञः को नु तृप्येत हरिलीलामृतं पिबन्‌
एवमुग्रश्रवाः पृष्ट ऋषिभिर्नैमिषायनैः
भगवत्यर्पिताध्यात्मस्तानाह श्रूयतामिति
सूत उवाच
हरेर्धृतक्रोडतनोः स्वमायया निशम्य गोरुद्धरणं रसातलात्‌
लीलां हिरण्याक्षमवज्ञया हतं सञ्जातहर्षो मुनिमाह भारतः
विदुर उवाच
प्रजापतिपतिः सृष्ट्वा प्रजासर्गे प्रजापतीन्‌
किमारभत मे ब्रह्मन्प्रब्रूह्यव्यक्तमार्गवित्‌
ये मरीच्यादयो विप्रा यस्तु स्वायम्भुवो मनुः
ते वै ब्रह्मण आदेशात्कथमेतदभावयन्‌
सद्वितीयाः किमसृजन्स्वतन्त्रा उत कर्मसु
आहो स्वित्संहताः सर्व इदं स्म समकल्पयन्‌
मैत्रेय उवाच
दैवेन दुर्वितर्क्येण परेणानिमिषेण च
जातक्षोभाद्भगवतो महानासीद्गुणत्रयात्‌
रजःप्रधानान्महतस्त्रिलिङ्गो दैवचोदितात्‌
जातः ससर्ज भूतादिर्वियदादीनि पञ्चशः
तानि चैकैकशः स्रष्टुमसमर्थानि भौतिकम्‌
संहत्य दैवयोगेन हैममण्डमवासृजन्‌
सोऽशयिष्टाब्धिसलिले आण्डकोशो निरात्मकः
साग्रं वै वर्षसाहस्रमन्ववात्सीत्तमीश्वरः
तस्य नाभेरभूत्पद्मं सहस्रार्कोरुदीधिति
सर्वजीवनिकायौको यत्र स्वयमभूत्स्वराट्‌
सोऽनुविष्टो भगवता यः शेते सलिलाशये
लोकसंस्थां यथा पूर्वं निर्ममे संस्थया स्वया
ससर्ज च्छाययाविद्यां पञ्चपर्वाणमग्रतः
तामिस्रमन्धतामिस्रं तमो मोहो महातमः
विससर्जात्मनः कायं नाभिनन्दंस्तमोमयम्‌
जगृहुर्यक्षरक्षांसि रात्रिं क्षुत्तृट्समुद्भवाम्‌
क्षुत्तृड्भ्यामुपसृष्टास्ते तं जग्धुमभिदुद्रुवुः
मा रक्षतैनं जक्षध्वमित्यूचुः क्षुत्तृडर्दिताः
देवस्तानाह संविग्नो मा मां जक्षत रक्षत
अहो मे यक्षरक्षांसि प्रजा यूयं बभूविथ
देवताः प्रभया या या दीव्यन्प्रमुखतोऽसृजत्‌
ते अहार्षुर्देवयन्तो विसृष्टां तां प्रभामहः
देवोऽदेवाञ्जघनतः सृजति स्मातिलोलुपान्‌
त एनं लोलुपतया मैथुनायाभिपेदिरे
ततो हसन्स भगवानसुरैर्निरपत्रपैः
अन्वीयमानस्तरसा क्रुद्धो भीतः परापतत्‌
स उपव्रज्य वरदं प्रपन्नार्तिहरं हरिम्‌
अनुग्रहाय भक्तानामनुरूपात्मदर्शनम्‌
पाहि मां परमात्मंस्ते प्रेषणेनासृजं प्रजाः
ता इमा यभितुं पापा उपाक्रामन्ति मां प्रभो
त्वमेकः किल लोकानां क्लिष्टानां क्लेशनाशनः
त्वमेकः क्लेशदस्तेषामनासन्नपदां तव
सोऽवधार्यास्य कार्पण्यं विविक्ताध्यात्मदर्शनः
विमुञ्चात्मतनुं घोरामित्युक्तो विमुमोच ह
तां क्वणच्चरणाम्भोजां मदविह्वललोचनाम्‌
काञ्चीकलापविलसद्‌ दुकूलच्छन्नरोधसम्‌
अन्योन्यश्लेषयोत्तुङ्ग निरन्तरपयोधराम्‌
सुनासां सुद्विजां स्निग्ध हासलीलावलोकनाम्‌
गूहन्तीं व्रीडयात्मानं नीलालकवरूथिनीम्‌
उपलभ्यासुरा धर्म सर्वे सम्मुमुहुः स्त्रियम्‌
अहो रूपमहो धैर्यमहो अस्या नवं वयः
मध्ये कामयमानानामकामेव विसर्पति
वितर्कयन्तो बहुधा तां सन्ध्यां प्रमदाकृतिम्‌
अभिसम्भाव्य विश्रम्भात्पर्यपृच्छन्कुमेधसः
कासि कस्यासि रम्भोरु को वार्थस्तेऽत्र भामिनि
रूपद्रविणपण्येन दुर्भगान्नो विबाधसे
या वा काचित्त्वमबले दिष्ट्या सन्दर्शनं तव
उत्सुनोषीक्षमाणानां कन्दुकक्रीडया मनः
नैकत्र ते जयति शालिनि पादपद्मं
घ्नन्त्या मुहुः करतलेन पतत्पतङ्गम्‌
मध्यं विषीदति बृहत्स्तनभारभीतं
शान्तेव दृष्टिरमला सुशिखासमूहः
इति सायन्तनीं सन्ध्यामसुराः प्रमदायतीम्‌
प्रलोभयन्तीं जगृहुर्मत्वा मूढधियः स्त्रियम्‌
प्रहस्य भावगम्भीरं जिघ्रन्त्यात्मानमात्मना
कान्त्या ससर्ज भगवान्गन्धर्वाप्सरसां गणान्‌
विससर्ज तनुं तां वै ज्योत्स्नां कान्तिमतीं प्रियाम्‌
त एव चाददुः प्रीत्या विश्वावसुपुरोगमाः
सृष्ट्वा भूतपिशाचांश्च भगवानात्मतन्द्रिणा
दिग्वाससो मुक्तकेशान्वीक्ष्य चामीलयद्दृशौ
जगृहुस्तद्विसृष्टां तां जृम्भणाख्यां तनुं प्रभोः
निद्रामिन्द्रियविक्लेदो यया भूतेषु दृश्यते
येनोच्छिष्टान्धर्षयन्ति तमुन्मादं प्रचक्षते
ऊर्जस्वन्तं मन्यमान आत्मानं भगवानजः
साध्यान्गणान्पितृगणान्परोक्षेणासृजत्प्रभुः
त आत्मसर्गं तं कायं पितरः प्रतिपेदिरे
साध्येभ्यश्च पितृभ्यश्च कवयो यद्वितन्वते
सिद्धान्विद्याधरांश्चैव तिरोधानेन सोऽसृजत्‌
तेभ्योऽददात्तमात्मानमन्तर्धानाख्यमद्भुतम्‌
स किन्नरान्किम्पुरुषान्प्रत्यात्म्येनासृजत्प्रभुः
मानयन्नात्मनात्मानमात्माभासं विलोकयन्‌
ते तु तज्जगृहू रूपं त्यक्तं यत्परमेष्ठिना
मिथुनीभूय गायन्तस्तमेवोषसि कर्मभिः
देहेन वै भोगवता शयानो बहुचिन्तया
सर्गेऽनुपचिते क्रोधादुत्ससर्ज ह तद्वपुः
येऽहीयन्तामुतः केशा अहयस्तेऽङ्ग जज्ञिरे
सर्पाः प्रसर्पतः क्रूरा नागा भोगोरुकन्धराः
स आत्मानं मन्यमानः कृतकृत्यमिवात्मभूः
तदा मनून्ससर्जान्ते मनसा लोकभावनान्‌
तेभ्यः सोऽसृजत्स्वीयं पुरं पुरुषमात्मवान्‌
तान्दृष्ट्वा ये पुरा सृष्टाः प्रशशंसुः प्रजापतिम्‌
अहो एतज्जगत्स्रष्टः सुकृतं बत ते कृतम्‌
प्रतिष्ठिताः क्रिया यस्मिन्साकमन्नमदाम हे
तपसा विद्यया युक्तो योगेन सुसमाधिना
ऋषीनृषिर्हृषीकेशः ससर्जाभिमताः प्रजाः
तेभ्यश्चैकैकशः स्वस्य देहस्यांशमदादजः
यत्तत्समाधियोगर्द्धि तपोविद्याविरक्तिमत्‌
विदुर उवाच
स्वायम्भुवस्य च मनोरंशः परमसम्मतः
कथ्यतां भगवन्यत्र मैथुनेनैधिरे प्रजाः
प्रियव्रतोत्तानपादौ सुतौ स्वायम्भुवस्य वै
यथाधर्मं जुगुपतुः सप्तद्वीपवतीं महीम्‌
तस्य वै दुहिता ब्रह्मन्देवहूतीति विश्रुता
पत्नी प्रजापतेरुक्ता कर्दमस्य त्वयानघ
तस्यां स वै महायोगी युक्तायां योगलक्षणैः
ससर्ज कतिधा वीर्यं तन्मे शुश्रूषवे वद
रुचिर्यो भगवान्ब्रह्मन्दक्षो वा ब्रह्मणः सुतः
यथा ससर्ज भूतानि लब्ध्वा भार्यां च मानवीम्‌
मैत्रेय उवाच
प्रजाः सृजेति भगवान्कर्दमो ब्रह्मणोदितः
सरस्वत्यां तपस्तेपे सहस्राणां समा दश
ततः समाधियुक्तेन क्रियायोगेन कर्दमः
सम्प्रपेदे हरिं भक्त्या प्रपन्नवरदाशुषम्‌
तावत्प्रसन्नो भगवान्पुष्कराक्षः कृते युगे
दर्शयामास तं क्षत्तः शाब्दं ब्रह्म दधद्वपुः
स तं विरजमर्काभं सितपद्मोत्पलस्रजम्‌
स्निग्धनीलालकव्रात वक्त्राब्जं विरजोऽम्बरम्‌
किरीटिनं कुण्डलिनं शङ्खचक्रगदाधरम्‌
श्वेतोत्पलक्रीडनकं मनःस्पर्शस्मितेक्षणम्‌
विन्यस्तचरणाम्भोजमंसदेशे गरुत्मतः
दृष्ट्वा खेऽवस्थितं वक्षः श्रियं कौस्तुभकन्धरम्‌
जातहर्षोऽपतन्मूर्ध्ना क्षितौ लब्धमनोरथः
गीर्भिस्त्वभ्यगृणात्प्रीति स्वभावात्मा कृताञ्जलिः
ऋषिरुवाच
जुष्टं बताद्याखिलसत्त्वराशेः सांसिद्ध्यमक्ष्णोस्तव दर्शनान्नः
यद्दर्शनं जन्मभिरीड्य सद्भिराशासते योगिनो रूढयोगाः
ये मायया ते हतमेधसस्त्वत्‌ पादारविन्दं भवसिन्धुपोतम्‌
उपासते कामलवाय तेषां रासीश कामान्निरयेऽपि ये स्युः
तथा स चाहं परिवोढुकामः समानशीलां गृहमेधधेनुम्‌
उपेयिवान्मूलमशेषमूलं दुराशयः कामदुघाङ्घ्रिपस्य
प्रजापतेस्ते वचसाधीश तन्त्या लोकः किलायं कामहतोऽनुबद्धः
अहं च लोकानुगतो वहामि बलिं च शुक्लानिमिषाय तुभ्यम्‌
लोकांश्च लोकानुगतान्पशूंश्च हित्वा श्रितास्ते चरणातपत्रम्‌
परस्परं त्वद्गुणवादसीधु पीयूषनिर्यापितदेहधर्माः
न तेऽजराक्षभ्रमिरायुरेषां त्रयोदशारं त्रिशतं षष्टिपर्व
षण्नेम्यनन्तच्छदि यत्त्रिणाभि करालस्रोतो जगदाच्छिद्य धावत्‌
एकः स्वयं सन्जगतः सिसृक्षया द्वितीययात्मन्नधियोगमायया
सृजस्यदः पासि पुनर्ग्रसिष्यसे यथोर्णनाभिर्भगवन्स्वशक्तिभिः
नैतद्बताधीश पदं तवेप्सितं यन्मायया नस्तनुषे भूतसूक्ष्मम्‌
अनुग्रहायास्त्वपि यर्हि मायया लसत्तुलस्या भगवान्विलक्षितः
तं त्वानुभूत्योपरतक्रियार्थं स्वमायया वर्तितलोकतन्त्रम्‌
नमाम्यभीक्ष्णं नमनीयपाद सरोजमल्पीयसि कामवर्षम्‌
ऋषिरुवाच
इत्यव्यलीकं प्रणुतोऽब्जनाभस्तमाबभाषे वचसामृतेन
सुपर्णपक्षोपरि रोचमानः प्रेमस्मितोद्वीक्षणविभ्रमद्भ्रूः
श्रीभगवानुवाच
विदित्वा तव चैत्यं मे पुरैव समयोजि तत्‌
यदर्थमात्मनियमैस्त्वयैवाहं समर्चितः
न वै जातु मृषैव स्यात्प्रजाध्यक्ष मदर्हणम्‌
भवद्विधेष्वतितरां मयि सङ्गृभितात्मनाम्‌
प्रजापतिसुतः सम्राण्मनुर्विख्यातमङ्गलः
ब्रह्मावर्तं योऽधिवसन्शास्ति सप्तार्णवां महीम्‌
स चेह विप्र राजर्षिर्महिष्या शतरूपया
आयास्यति दिदृक्षुस्त्वां परश्वो धर्मकोविदः
आत्मजामसितापाङ्गीं वयःशीलगुणान्विताम्‌
मृगयन्तीं पतिं दास्यत्यनुरूपाय ते प्रभो
समाहितं ते हृदयं यत्रेमान्परिवत्सरान्‌
सा त्वां ब्रह्मन्नृपवधूः काममाशु भजिष्यति
या त आत्मभृतं वीर्यं नवधा प्रसविष्यति
वीर्ये त्वदीये ऋषय आधास्यन्त्यञ्जसात्मनः
त्वं च सम्यगनुष्ठाय निदेशं म उशत्तमः
मयि तीर्थीकृताशेष क्रियार्थो मां प्रपत्स्यसे
कृत्वा दयां च जीवेषु दत्त्वा चाभयमात्मवान्‌
मय्यात्मानं सह जगद्द्रक्ष्यस्यात्मनि चापि माम्‌
सहाहं स्वांशकलया त्वद्वीर्येण महामुने
तव क्षेत्रे देवहूत्यां प्रणेष्ये तत्त्वसंहिताम्‌
मैत्रेय उवाच
एवं तमनुभाष्याथ भगवान्प्रत्यगक्षजः
जगाम बिन्दुसरसः सरस्वत्या परिश्रितात्‌
निरीक्षतस्तस्य ययावशेष सिद्धेश्वराभिष्टुतसिद्धमार्गः
आकर्णयन्पत्ररथेन्द्रपक्षैरुच्चारितं स्तोममुदीर्णसाम
अथ सम्प्रस्थिते शुक्ले कर्दमो भगवानृषिः
आस्ते स्म बिन्दुसरसि तं कालं प्रतिपालयन्‌
मनुः स्यन्दनमास्थाय शातकौम्भपरिच्छदम्‌
आरोप्य स्वां दुहितरं सभार्यः पर्यटन्महीम्‌
तस्मिन्सुधन्वन्नहनि भगवान्यत्समादिशत्‌
उपायादाश्रमपदं मुनेः शान्तव्रतस्य तत्‌
यस्मिन्भगवतो नेत्रान्न्यपतन्नश्रुबिन्दवः
कृपया सम्परीतस्य प्रपन्नेऽर्पितया भृशम्‌
तद्वै बिन्दुसरो नाम सरस्वत्या परिप्लुतम्‌
पुण्यं शिवामृतजलं महर्षिगणसेवितम्‌
पुण्यद्रुमलताजालैः कूजत्पुण्यमृगद्विजैः
सर्वर्तुफलपुष्पाढ्यं वनराजिश्रियान्वितम्‌
मत्तद्विजगणैर्घुष्टं मत्तभ्रमरविभ्रमम्‌
मत्तबर्हिनटाटोपमाह्वयन्मत्तकोकिलम्‌
कदम्बचम्पकाशोक करञ्जबकुलासनैः
कुन्दमन्दारकुटजैश्चूतपोतैरलङ्कृतम्‌
कारण्डवैः प्लवैर्हंसैः कुररैर्जलकुक्कुटैः
सारसैश्चक्रवाकैश्च चकोरैर्वल्गु कूजितम्‌
तथैव हरिणैः क्रोडैः श्वाविद्गवयकुञ्जरैः
गोपुच्छैर्हरिभिर्मर्कैर्नकुलैर्नाभिभिर्वृतम्‌
प्रविश्य तत्तीर्थवरमादिराजः सहात्मजः
ददर्श मुनिमासीनं तस्मिन्हुतहुताशनम्‌
विद्योतमानं वपुषा तपस्युग्रयुजा चिरम्‌
नातिक्षामं भगवतः स्निग्धापाङ्गावलोकनात्‌
तद्व्याहृतामृतकला पीयूषश्रवणेन च
प्रांशुं पद्मपलाशाक्षं जटिलं चीरवाससम्‌
उपसंश्रित्य मलिनं यथार्हणमसंस्कृतम्‌
अथोटजमुपायातं नृदेवं प्रणतं पुरः
सपर्यया पर्यगृह्णात्प्रतिनन्द्यानुरूपया
गृहीतार्हणमासीनं संयतं प्रीणयन्मुनिः
स्मरन्भगवदादेशमित्याह श्लक्ष्णया गिरा
नूनं चङ्क्रमणं देव सतां संरक्षणाय ते
वधाय चासतां यस्त्वं हरेः शक्तिर्हि पालिनी
योऽर्केन्द्वग्नीन्द्रवायूनां यमधर्मप्रचेतसाम्‌
रूपाणि स्थान आधत्से तस्मै शुक्लाय ते नमः
न यदा रथमास्थाय जैत्रं मणिगणार्पितम्‌
विस्फूर्जच्चण्डकोदण्डो रथेन त्रासयन्नघान्‌
स्वसैन्यचरणक्षुण्णं वेपयन्मण्डलं भुवः
विकर्षन्बृहतीं सेनां पर्यटस्यंशुमानिव
तदैव सेतवः सर्वे वर्णाश्रमनिबन्धनाः
भगवद्रचिता राजन्भिद्येरन्बत दस्युभिः
अधर्मश्च समेधेत लोलुपैर्व्यङ्कुशैर्नृभिः
शयाने त्वयि लोकोऽयं दस्युग्रस्तो विनङ्क्ष्यति
अथापि पृच्छे त्वां वीर यदर्थं त्वमिहागतः
तद्वयं निर्व्यलीकेन प्रतिपद्यामहे हृदा
मैत्रेय उवाच
एवमाविष्कृताशेष गुणकर्मोदयो मुनिम्‌
सव्रीड इव तं सम्राडुपारतमुवाच ह
मनुरुवाच
ब्रह्मासृजत्स्वमुखतो युष्मानात्मपरीप्सया
छन्दोमयस्तपोविद्या योगयुक्तानलम्पटान्‌
तत्त्राणायासृजच्चास्मान्दोःसहस्रात्सहस्रपात्‌
हृदयं तस्य हि ब्रह्म क्षत्रमङ्गं प्रचक्षते
अतो ह्यन्योन्यमात्मानं ब्रह्म क्षत्रं च रक्षतः
रक्षति स्माव्ययो देवः स यः सदसदात्मकः
तव सन्दर्शनादेव च्छिन्ना मे सर्वसंशयाः
यत्स्वयं भगवान्प्रीत्या धर्ममाह रिरक्षिषोः
दिष्ट्या मे भगवान्दृष्टो दुर्दर्शो योऽकृतात्मनाम्‌
दिष्ट्या पादरजः स्पृष्टं शीर्ष्णा मे भवतः शिवम्‌
दिष्ट्या त्वयानुशिष्टोऽहं कृतश्चानुग्रहो महान्‌
अपावृतैः कर्णरन्ध्रैर्जुष्टा दिष्ट्योशतीर्गिरः
स भवान्दुहितृस्नेह परिक्लिष्टात्मनो मम
श्रोतुमर्हसि दीनस्य श्रावितं कृपया मुने
प्रियव्रतोत्तानपदोः स्वसेयं दुहिता मम
अन्विच्छति पतिं युक्तं वयःशीलगुणादिभिः
यदा तु भवतः शील श्रुतरूपवयोगुणान्‌
अशृणोन्नारदादेषा त्वय्यासीत्कृतनिश्चया
तत्प्रतीच्छ द्विजाग्र्येमां श्रद्धयोपहृतां मया
सर्वात्मनानुरूपां ते गृहमेधिषु कर्मसु
उद्यतस्य हि कामस्य प्रतिवादो न शस्यते
अपि निर्मुक्तसङ्गस्य कामरक्तस्य किं पुनः
य उद्यतमनादृत्य कीनाशमभियाचते
क्षीयते तद्यशः स्फीतं मानश्चावज्ञया हतः
अहं त्वाशृणवं विद्वन्विवाहार्थं समुद्यतम्‌
अतस्त्वमुपकुर्वाणः प्रत्तां प्रतिगृहाण मे
ऋषिरुवाच
बाढमुद्वोढुकामोऽहमप्रत्ता च तवात्मजा
आवयोरनुरूपोऽसावाद्यो वैवाहिको विधिः
कामः स भूयान्नरदेव तेऽस्याः पुत्र्याः समाम्नायविधौ प्रतीतः
क एव ते तनयां नाद्रियेत स्वयैव कान्त्या क्षिपतीमिव श्रियम्‌
यां हर्म्यपृष्ठे क्वणदङ्घ्रिशोभां विक्रीडतीं कन्दुकविह्वलाक्षीम्‌
विश्वावसुर्न्यपतत्स्वाद्विमानाद्विलोक्य सम्मोहविमूढचेताः
तां प्रार्थयन्तीं ललनाललाममसेवितश्रीचरणैरदृष्टाम्‌
वत्सां मनोरुच्चपदः स्वसारं को नानुमन्येत बुधोऽभियाताम्‌
अतो भजिष्ये समयेन साध्वीं यावत्तेजो बिभृयादात्मनो मे
अतो धर्मान्पारमहंस्यमुख्यान्शुक्लप्रोक्तान्बहु मन्येऽविहिंस्रान्‌
यतोऽभवद्विश्वमिदं विचित्रं संस्थास्यते यत्र च वावतिष्ठते
प्रजापतीनां पतिरेष मह्यं परं प्रमाणं भगवाननन्तः
मैत्रेय उवाच
स उग्रधन्वन्नियदेवाबभाषे आसीच्च तूष्णीमरविन्दनाभम्‌
धियोपगृह्णन्स्मितशोभितेन मुखेन चेतो लुलुभे देवहूत्याः
सोऽनु ज्ञात्वा व्यवसितं महिष्या दुहितुः स्फुटम्‌
तस्मै गुणगणाढ्याय ददौ तुल्यां प्रहर्षितः
शतरूपा महाराज्ञी पारिबर्हान्महाधनान्‌
दम्पत्योः पर्यदात्प्रीत्या भूषावासः परिच्छदान्‌
प्रत्तां दुहितरं सम्राट्सदृक्षाय गतव्यथः
उपगुह्य च बाहुभ्यामौत्कण्ठ्योन्मथिताशयः
अशक्नुवंस्तद्विरहं मुञ्चन्बाष्पकलां मुहुः
आसिञ्चदम्ब वत्सेति नेत्रोदैर्दुहितुः शिखाः
आमन्त्र्य तं मुनिवरमनुज्ञातः सहानुगः
प्रतस्थे रथमारुह्य सभार्यः स्वपुरं नृपः
उभयोरृषिकुल्यायाः सरस्वत्याः सुरोधसोः
ऋषीणामुपशान्तानां पश्यन्नाश्रमसम्पदः
तमायान्तमभिप्रेत्य ब्रह्मावर्तात्प्रजाः पतिम्‌
गीतसंस्तुतिवादित्रैः प्रत्युदीयुः प्रहर्षिताः
बर्हिष्मती नाम पुरी सर्वसम्पत्समन्विता
न्यपतन्यत्र रोमाणि यज्ञस्याङ्गं विधुन्वतः
कुशाः काशास्त एवासन्शश्वद्धरितवर्चसः
ऋषयो यैः पराभाव्य यज्ञघ्नान्यज्ञमीजिरे
कुशकाशमयं बर्हिरास्तीर्य भगवान्मनुः
अयजद्यज्ञपुरुषं लब्धा स्थानं यतो भुवम्‌
बर्हिष्मतीं नाम विभुर्यां निर्विश्य समावसत्‌
तस्यां प्रविष्टो भवनं तापत्रयविनाशनम्‌
सभार्यः सप्रजः कामान्बुभुजेऽन्याविरोधतः
सङ्गीयमानसत्कीर्तिः सस्त्रीभिः सुरगायकैः
प्रत्यूषेष्वनुबद्धेन हृदा शृण्वन्हरेः कथाः
निष्णातं योगमायासु मुनिं स्वायम्भुवं मनुम्‌
यदाभ्रंशयितुं भोगा न शेकुर्भगवत्परम्‌
अयातयामास्तस्यासन्यामाः स्वान्तरयापनाः
शृण्वतो ध्यायतो विष्णोः कुर्वतो ब्रुवतः कथाः
स एवं स्वान्तरं निन्ये युगानामेकसप्ततिम्‌
वासुदेवप्रसङ्गेन परिभूतगतित्रयः
शारीरा मानसा दिव्या वैयासे ये च मानुषाः
भौतिकाश्च कथं क्लेशा बाधन्ते हरिसंश्रयम्‌
यः पृष्टो मुनिभिः प्राह धर्मान्नानाविधान्छुभान्‌
नृणां वर्णाश्रमाणां च सर्वभूतहितः सदा
एतत्त आदिराजस्य मनोश्चरितमद्भुतम्‌
वर्णितं वर्णनीयस्य तदपत्योदयं शृणु
मैत्रेय उवाच
पितृभ्यां प्रस्थिते साध्वी पतिमिङ्गितकोविदा
नित्यं पर्यचरत्प्रीत्या भवानीव भवं प्रभुम्‌
विश्रम्भेणात्मशौचेन गौरवेण दमेन च
शुश्रूषया सौहृदेन वाचा मधुरया च भोः
विसृज्य कामं दम्भं च द्वेषं लोभमघं मदम्‌
अप्रमत्तोद्यता नित्यं तेजीयांसमतोषयत्‌
स वै देवर्षिवर्यस्तां मानवीं समनुव्रताम्‌
दैवाद्गरीयसः पत्युराशासानां महाशिषः
कालेन भूयसा क्षामां कर्शितां व्रतचर्यया
प्रेमगद्गदया वाचा पीडितः कृपयाब्रवीत्‌
कर्दम उवाच
तुष्टोऽहमद्य तव मानवि मानदायाः
शुश्रूषया परमया परया च भक्त्या
यो देहिनामयमतीव सुहृत्स देहो
नावेक्षितः समुचितः क्षपितुं मदर्थे
ये मे स्वधर्मनिरतस्य तपःसमाधि
विद्यात्मयोगविजिता भगवत्प्रसादाः
तानेव ते मदनुसेवनयावरुद्धान्‌
दृष्टिं प्रपश्य वितराम्यभयानशोकान्‌
अन्ये पुनर्भगवतो भ्रुव उद्विजृम्भ
विभ्रंशितार्थरचनाः किमुरुक्रमस्य
सिद्धासि भुङ्क्ष्व विभवान्निजधर्मदोहान्‌
दिव्यान्नरैर्दुरधिगान्नृपविक्रियाभिः
एवं ब्रुवाणमबलाखिलयोगमाया
विद्याविचक्षणमवेक्ष्य गताधिरासीत्‌
सम्प्रश्रयप्रणयविह्वलया गिरेषद्‌
व्रीडावलोकविलसद्धसिताननाह
देवहूतिरुवाच
राद्धं बत द्विजवृषैतदमोघयोग
मायाधिपे त्वयि विभो तदवैमि भर्तः
यस्तेऽभ्यधायि समयः सकृदङ्गसङ्गो
भूयाद्गरीयसि गुणः प्रसवः सतीनाम्‌
तत्रेतिकृत्यमुपशिक्ष यथोपदेशं
येनैष मे कर्शितोऽतिरिरंसयात्मा
सिद्ध्येत ते कृतमनोभवधर्षिताया
दीनस्तदीश भवनं सदृशं विचक्ष्व
मैत्रेय उवाच
प्रियायाः प्रियमन्विच्छन्कर्दमो योगमास्थितः
विमानं कामगं क्षत्तस्तर्ह्येवाविरचीकरत्‌
सर्वकामदुघं दिव्यं सर्वरत्नसमन्वितम्‌
सर्वर्द्ध्युपचयोदर्कं मणिस्तम्भैरुपस्कृतम्‌
दिव्योपकरणोपेतं सर्वकालसुखावहम्‌
पट्टिकाभिः पताकाभिर्विचित्राभिरलङ्कृतम्‌
स्रग्भिर्विचित्रमाल्याभिर्मञ्जुशिञ्जत्षडङ्घ्रिभिः
दुकूलक्षौमकौशेयैर्नानावस्त्रैर्विराजितम्‌
उपर्युपरि विन्यस्त निलयेषु पृथक्पृथक्‌
क्षिप्तैः कशिपुभिः कान्तं पर्यङ्कव्यजनासनैः
तत्र तत्र विनिक्षिप्त नानाशिल्पोपशोभितम्‌
महामरकतस्थल्या जुष्टं विद्रुमवेदिभिः
द्वाःसु विद्रुमदेहल्या भातं वज्रकपाटवत्‌
शिखरेष्विन्द्रनीलेषु हेमकुम्भैरधिश्रितम्‌
चक्षुष्मत्पद्मरागाग्र्यैर्वज्रभित्तिषु निर्मितैः
जुष्टं विचित्रवैतानैर्महार्हैर्हेमतोरणैः
हंसपारावतव्रातैस्तत्र तत्र निकूजितम्‌
कृत्रिमान्मन्यमानैः स्वानधिरुह्याधिरुह्य च
विहारस्थानविश्राम संवेशप्राङ्गणाजिरैः
यथोपजोषं रचितैर्विस्मापनमिवात्मनः
ईदृग्गृहं तत्पश्यन्तीं नातिप्रीतेन चेतसा
सर्वभूताशयाभिज्ञः प्रावोचत्कर्दमः स्वयम्‌
निमज्ज्यास्मिन्ह्रदे भीरु विमानमिदमारुह
इदं शुक्लकृतं तीर्थमाशिषां यापकं नृणाम्‌
सा तद्भर्तुः समादाय वचः कुवलयेक्षणा
सरजं बिभ्रती वासो वेणीभूतांश्च मूर्धजान्‌
अङ्गं च मलपङ्केन सञ्छन्नं शबलस्तनम्‌
आविवेश सरस्वत्याः सरः शिवजलाशयम्‌
सान्तः सरसि वेश्मस्थाः शतानि दश कन्यकाः
सर्वाः किशोरवयसो ददर्शोत्पलगन्धयः
तां दृष्ट्वा सहसोत्थाय प्रोचुः प्राञ्जलयः स्त्रियः
वयं कर्मकरीस्तुभ्यं शाधि नः करवाम किम्‌
स्नानेन तां महार्हेण स्नापयित्वा मनस्विनीम्‌
दुकूले निर्मले नूत्ने ददुरस्यै च मानदाः
भूषणानि परार्ध्यानि वरीयांसि द्युमन्ति च
अन्नं सर्वगुणोपेतं पानं चैवामृतासवम्‌
अथादर्शे स्वमात्मानं स्रग्विणं विरजाम्बरम्‌
विरजं कृतस्वस्त्ययनं कन्याभिर्बहुमानितम्‌
स्नातं कृतशिरःस्नानं सर्वाभरणभूषितम्‌
निष्कग्रीवं वलयिनं कूजत्काञ्चननूपुरम्‌
श्रोण्योरध्यस्तया काञ्च्या काञ्चन्या बहुरत्नया
हारेण च महार्हेण रुचकेन च भूषितम्‌
सुदता सुभ्रुवा श्लक्ष्ण स्निग्धापाङ्गेन चक्षुषा
पद्मकोशस्पृधा नीलैरलकैश्च लसन्मुखम्‌
यदा सस्मार ऋषभमृषीणां दयितं पतिम्‌
तत्र चास्ते सह स्त्रीभिर्यत्रास्ते स प्रजापतिः
भर्तुः पुरस्तादात्मानं स्त्रीसहस्रवृतं तदा
निशाम्य तद्योगगतिं संशयं प्रत्यपद्यत
स तां कृतमलस्नानां विभ्राजन्तीमपूर्ववत्‌
आत्मनो बिभ्रतीं रूपं संवीतरुचिरस्तनीम्‌
विद्याधरीसहस्रेण सेव्यमानां सुवाससम्‌
जातभावो विमानं तदारोहयदमित्रहन्‌
तस्मिन्नलुप्तमहिमा प्रिययानुरक्तो
विद्याधरीभिरुपचीर्णवपुर्विमाने
बभ्राज उत्कचकुमुद्गणवानपीच्यस्‌
ताराभिरावृत इवोडुपतिर्नभःस्थः
तेनाष्टलोकपविहारकुलाचलेन्द्र
द्रोणीष्वनङ्गसखमारुतसौभगासु
सिद्धैर्नुतो द्युधुनिपातशिवस्वनासु
रेमे चिरं धनदवल्ललनावरूथी
वैश्रम्भके सुरसने नन्दने पुष्पभद्रके
मानसे चैत्ररथ्ये च स रेमे रामया रतः
भ्राजिष्णुना विमानेन कामगेन महीयसा
वैमानिकानत्यशेत चरल्लोकान्यथानिलः
किं दुरापादनं तेषां पुंसामुद्दामचेतसाम्‌
यैराश्रितस्तीर्थपदश्चरणो व्यसनात्ययः
प्रेक्षयित्वा भुवो गोलं पत्न्यै यावान्स्वसंस्थया
बह्वाश्चर्यं महायोगी स्वाश्रमाय न्यवर्तत
विभज्य नवधात्मानं मानवीं सुरतोत्सुकाम्‌
रामां निरमयन्रेमे वर्षपूगान्मुहूर्तवत्‌
तस्मिन्विमान उत्कृष्टां शय्यां रतिकरीं श्रिता
न चाबुध्यत तं कालं पत्यापीच्येन सङ्गता
एवं योगानुभावेन दम्पत्यो रममाणयोः
शतं व्यतीयुः शरदः कामलालसयोर्मनाक्‌
तस्यामाधत्त रेतस्तां भावयन्नात्मनात्मवित्‌
नोधा विधाय रूपं स्वं सर्वसङ्कल्पविद्विभुः
अतः सा सुषुवे सद्यो देवहूतिः स्त्रियः प्रजाः
सर्वास्ताश्चारुसर्वाङ्ग्यो लोहितोत्पलगन्धयः
पतिं सा प्रव्रजिष्यन्तं तदालक्ष्योशती बहिः
स्मयमाना विक्लवेन हृदयेन विदूयता
लिखन्त्यधोमुखी भूमिं पदा नखमणिश्रिया
उवाच ललितां वाचं निरुध्याश्रुकलां शनैः
देवहूतिरुवाच
सर्वं तद्भगवान्मह्यमुपोवाह प्रतिश्रुतम्‌
अथापि मे प्रपन्नाया अभयं दातुमर्हसि
ब्रह्मन्दुहितृभिस्तुभ्यं विमृग्याः पतयः समाः
कश्चित्स्यान्मे विशोकाय त्वयि प्रव्रजिते वनम्‌
एतावतालं कालेन व्यतिक्रान्तेन मे प्रभो
इन्द्रियार्थप्रसङ्गेन परित्यक्तपरात्मनः
इन्द्रियार्थेषु सज्जन्त्या प्रसङ्गस्त्वयि मे कृतः
अजानन्त्या परं भावं तथाप्यस्त्वभयाय मे
सङ्गो यः संसृतेर्हेतुरसत्सु विहितोऽधिया
स एव साधुषु कृतो निःसङ्गत्वाय कल्पते
नेह यत्कर्म धर्माय न विरागाय कल्पते
न तीर्थपदसेवायै जीवन्नपि मृतो हि सः
साहं भगवतो नूनं वञ्चिता मायया दृढम्‌
यत्त्वां विमुक्तिदं प्राप्य न मुमुक्षेय बन्धनात्‌
मैत्रेय उवाच
निर्वेदवादिनीमेवं मनोर्दुहितरं मुनिः
दयालुः शालिनीमाह शुक्लाभिव्याहृतं स्मरन्‌
ऋषिरुवाच
मा खिदो राजपुत्रीत्थमात्मानं प्रत्यनिन्दिते
भगवांस्तेऽक्षरो गर्भमदूरात्सम्प्रपत्स्यते
धृतव्रतासि भद्रं ते दमेन नियमेन च
तपोद्रविणदानैश्च श्रद्धया चेश्वरं भज
स त्वयाराधितः शुक्लो वितन्वन्मामकं यशः
छेत्ता ते हृदयग्रन्थिमौदर्यो ब्रह्मभावनः
मैत्रेय उवाच
देवहूत्यपि सन्देशं गौरवेण प्रजापतेः
सम्यक्ष्रद्धाय पुरुषं कूटस्थमभजद्गुरुम्‌
तस्यां बहुतिथे काले भगवान्मधुसूदनः
कार्दमं वीर्यमापन्नो जज्ञेऽग्निरिव दारुणि
अवादयंस्तदा व्योम्नि वादित्राणि घनाघनाः
गायन्ति तं स्म गन्धर्वा नृत्यन्त्यप्सरसो मुदा
पेतुः सुमनसो दिव्याः खेचरैरपवर्जिताः
प्रसेदुश्च दिशः सर्वा अम्भांसि च मनांसि च
तत्कर्दमाश्रमपदं सरस्वत्या परिश्रितम्‌
स्वयम्भूः साकमृषिभिर्मरीच्यादिभिरभ्ययात्‌
भगवन्तं परं ब्रह्म सत्त्वेनांशेन शत्रुहन्‌
तत्त्वसङ्ख्यानविज्ञप्त्यै जातं विद्वानजः स्वराट्‌
सभाजयन्विशुद्धेन चेतसा तच्चिकीर्षितम्‌
प्रहृष्यमाणैरसुभिः कर्दमं चेदमभ्यधात्‌
ब्रह्मोवाच
त्वया मेऽपचितिस्तात कल्पिता निर्व्यलीकतः
यन्मे सञ्जगृहे वाक्यं भवान्मानद मानयन्‌
एतावत्येव शुश्रूषा कार्या पितरि पुत्रकैः
बाढमित्यनुमन्येत गौरवेण गुरोर्वचः
इमा दुहितरः सत्यस्तव वत्स सुमध्यमाः
सर्गमेतं प्रभावैः स्वैर्बृंहयिष्यन्त्यनेकधा
अतस्त्वमृषिमुख्येभ्यो यथाशीलं यथारुचि
आत्मजाः परिदेह्यद्य विस्तृणीहि यशो भुवि
वेदाहमाद्यं पुरुषमवतीर्णं स्वमायया
भूतानां शेवधिं देहं बिभ्राणं कपिलं मुने
ज्ञानविज्ञानयोगेन कर्मणामुद्धरन्जटाः
हिरण्यकेशः पद्माक्षः पद्ममुद्रापदाम्बुजः
एष मानवि ते गर्भं प्रविष्टः कैटभार्दनः
अविद्यासंशयग्रन्थिं छित्त्वा गां विचरिष्यति
अयं सिद्धगणाधीशः साङ्ख्याचार्यैः सुसम्मतः
लोके कपिल इत्याख्यां गन्ता ते कीर्तिवर्धनः
मैत्रेय उवाच
तावाश्वास्य जगत्स्रष्टा कुमारैः सहनारदः
हंसो हंसेन यानेन त्रिधामपरमं ययौ
गते शतधृतौ क्षत्तः कर्दमस्तेन चोदितः
यथोदितं स्वदुहित्‌ः प्रादाद्विश्वसृजां ततः
मरीचये कलां प्रादादनसूयामथात्रये
श्रद्धामङ्गिरसेऽयच्छत्पुलस्त्याय हविर्भुवम्‌
पुलहाय गतिं युक्तां क्रतवे च क्रियां सतीम्‌
ख्यातिं च भृगवेऽयच्छद्वसिष्ठायाप्यरुन्धतीम्‌
अथर्वणेऽददाच्छान्तिं यया यज्ञो वितन्यते
विप्रर्षभान्कृतोद्वाहान्सदारान्समलालयत्‌
ततस्त ऋषयः क्षत्तः कृतदारा निमन्त्र्य तम्‌
प्रातिष्ठन्नन्दिमापन्नाः स्वं स्वमाश्रममण्डलम्‌
स चावतीर्णं त्रियुगमाज्ञाय विबुधर्षभम्‌
विविक्त उपसङ्गम्य प्रणम्य समभाषत
अहो पापच्यमानानां निरये स्वैरमङ्गलैः
कालेन भूयसा नूनं प्रसीदन्तीह देवताः
बहुजन्मविपक्वेन सम्यग्योगसमाधिना
द्रष्टुं यतन्ते यतयः शून्यागारेषु यत्पदम्‌
स एव भगवानद्य हेलनं न गणय्य नः
गृहेषु जातो ग्राम्याणां यः स्वानां पक्षपोषणः
स्वीयं वाक्यमृतं कर्तुमवतीर्णोऽसि मे गृहे
चिकीर्षुर्भगवान्ज्ञानं भक्तानां मानवर्धनः
तान्येव तेऽभिरूपाणि रूपाणि भगवंस्तव
यानि यानि च रोचन्ते स्वजनानामरूपिणः
त्वां सूरिभिस्तत्त्वबुभुत्सयाद्धा सदाभिवादार्हणपादपीठम्‌
ऐश्वर्यवैराग्ययशोऽवबोध वीर्यश्रिया पूर्तमहं प्रपद्ये
परं प्रधानं पुरुषं महान्तं कालं कविं त्रिवृतं लोकपालम्‌
आत्मानुभूत्यानुगतप्रपञ्चं स्वच्छन्दशक्तिं कपिलं प्रपद्ये
अ स्माभिपृच्छेऽद्य पतिं प्रजानां त्वयावतीर्णर्ण उताप्तकामः
परिव्रजत्पदवीमास्थितोऽहं चरिष्ये त्वां हृदि युञ्जन्विशोकः
श्रीभगवानुवाच
मया प्रोक्तं हि लोकस्य प्रमाणं सत्यलौकिके
अथाजनि मया तुभ्यं यदवोचमृतं मुने
एतन्मे जन्म लोकेऽस्मिन्मुमुक्षूणां दुराशयात्‌
प्रसङ्ख्यानाय तत्त्वानां सम्मतायात्मदर्शने
एष आत्मपथोऽव्यक्तो नष्टः कालेन भूयसा
तं प्रवर्तयितुं देहमिमं विद्धि मया भृतम्‌
गच्छ कामं मयापृष्टो मयि सन्न्यस्तकर्मणा
जित्वा सुदुर्जयं मृत्युममृतत्वाय मां भज
मामात्मानं स्वयंज्योतिः सर्वभूतगुहाशयम्‌
आत्मन्येवात्मना वीक्ष्य विशोकोऽभयमृच्छसि
मात्र आध्यात्मिकीं विद्यां शमनीं सर्वकर्मणाम्‌
वितरिष्ये यया चासौ भयं चातितरिष्यति
मैत्रेय उवाच
एवं समुदितस्तेन कपिलेन प्रजापतिः
दक्षिणीकृत्य तं प्रीतो वनमेव जगाम ह
व्रतं स आस्थितो मौनमात्मैकशरणो मुनिः
निःसङ्गो व्यचरत्क्षोणीमनग्निरनिकेतनः
मनो ब्रह्मणि युञ्जानो यत्तत्सदसतः परम्‌
गुणावभासे विगुण एकभक्त्यानुभाविते
निरहङ्कृतिर्निर्ममश्च निर्द्वन्द्वः समदृक्स्वदृक्‌
प्रत्यक्प्रशान्तधीर्धीरः प्रशान्तोर्मिरिवोदधिः
वासुदेवे भगवति सर्वज्ञे प्रत्यगात्मनि
परेण भक्तिभावेन लब्धात्मा मुक्तबन्धनः
आत्मानं सर्वभूतेषु भगवन्तमवस्थितम्‌
अपश्यत्सर्वभूतानि भगवत्यपि चात्मनि
इच्छाद्वेषविहीनेन सर्वत्र समचेतसा
भगवद्भक्तियुक्तेन प्राप्ता भागवती गतिः
शौनक उवाच
कपिलस्तत्त्वसङ्ख्याता भगवानात्ममायया
जातः स्वयमजः साक्षादात्मप्रज्ञप्तये नृणाम्‌
न ह्यस्य वर्ष्मणः पुंसां वरिम्णः सर्वयोगिनाम्‌
विश्रुतौ श्रुतदेवस्य भूरि तृप्यन्ति मेऽसवः
यद्यद्विधत्ते भगवान्स्वच्छन्दात्मात्ममायया
तानि मे श्रद्दधानस्य कीर्तन्यान्यनुकीर्तय
सूत उवाच
द्वैपायनसखस्त्वेवं मैत्रेयो भगवांस्तथा
प्राहेदं विदुरं प्रीत आन्वीक्षिक्यां प्रचोदितः
मैत्रेय उवाच
पितरि प्रस्थितेऽरण्यं मातुः प्रियचिकीर्षया
तस्मिन्बिन्दुसरेऽवात्सीद्भगवान्कपिलः किल
तमासीनमकर्माणं तत्त्वमार्गाग्रदर्शनम्‌
स्वसुतं देवहूत्याह धातुः संस्मरती वचः
देवहूतिरुवाच
निर्विण्णा नितरां भूमन्नसदिन्द्रियतर्षणात्‌
येन सम्भाव्यमानेन प्रपन्नान्धं तमः प्रभो
तस्य त्वं तमसोऽन्धस्य दुष्पारस्याद्य पारगम्‌
सच्चक्षुर्जन्मनामन्ते लब्धं मे त्वदनुग्रहात्‌
य आद्यो भगवान्पुंसामीश्वरो वै भवान्किल
लोकस्य तमसान्धस्य चक्षुः सूर्य इवोदितः
अथ मे देव सम्मोहमपाक्रष्टुं त्वमर्हसि
योऽवग्रहोऽहं ममेतीत्येतस्मिन्योजितस्त्वया
तं त्वा गताहं शरणं शरण्यं स्वभृत्यसंसारतरोः कुठारम्‌
जिज्ञासयाहं प्रकृतेः पूरुषस्य नमामि सद्धर्मविदां वरिष्ठम्‌
मैत्रेय उवाच
इति स्वमातुर्निरवद्यमीप्सितं निशम्य पुंसामपवर्गवर्धनम्‌
धियाभिनन्द्यात्मवतां सतां गतिर्बभाष ईषत्स्मितशोभिताननः
श्रीभगवानुवाच
योग आध्यात्मिकः पुंसां मतो निःश्रेयसाय मे
अत्यन्तोपरतिर्यत्र दुःखस्य च सुखस्य च
तमिमं ते प्रवक्ष्यामि यमवोचं पुरानघे
ऋषीणां श्रोतुकामानां योगं सर्वाङ्गनैपुणम्‌
चेतः खल्वस्य बन्धाय मुक्तये चात्मनो मतम्‌
गुणेषु सक्तं बन्धाय रतं वा पुंसि मुक्तये
अहं ममाभिमानोत्थैः कामलोभादिभिर्मलैः
वीतं यदा मनः शुद्धमदुःखमसुखं समम्‌
तदा पुरुष आत्मानं केवलं प्रकृतेः परम्‌
निरन्तरं स्वयंज्योतिरणिमानमखण्डितम्‌
ज्ञानवैराग्ययुक्तेन भक्तियुक्तेन चात्मना
परिपश्यत्युदासीनं प्रकृतिं च हतौजसम्‌
न युज्यमानया भक्त्या भगवत्यखिलात्मनि
सदृशोऽस्ति शिवः पन्था योगिनां ब्रह्मसिद्धये
प्रसङ्गमजरं पाशमात्मनः कवयो विदुः
स एव साधुषु कृतो मोक्षद्वारमपावृतम्‌
तितिक्षवः कारुणिकाः सुहृदः सर्वदेहिनाम्‌
अजातशत्रवः शान्ताः साधवः साधुभूषणाः
मय्यनन्येन भावेन भक्तिं कुर्वन्ति ये दृढाम्‌
मत्कृते त्यक्तकर्माणस्त्यक्तस्वजनबान्धवाः
मदाश्रयाः कथा मृष्टाः शृण्वन्ति कथयन्ति च
तपन्ति विविधास्तापा नैतान्मद्गतचेतसः
त एते साधवः साध्वि सर्वसङ्गविवर्जिताः
सङ्गस्तेष्वथ ते प्रार्थ्यः सङ्गदोषहरा हि ते
सतां प्रसङ्गान्मम वीर्यसंविदो भवन्ति हृत्कर्णरसायनाः कथाः
तज्जोषणादाश्वपवर्गवर्त्मनि श्रद्धा रतिर्भक्तिरनुक्रमिष्यति
भक्त्या पुमान्जातविराग ऐन्द्रियाद्दृष्टश्रुतान्मद्रचनानुचिन्तया
चित्तस्य यत्तो ग्रहणे योगयुक्तो यतिष्यते ऋजुभिर्योगमार्गैः
असेवयायं प्रकृतेर्गुणानां ज्ञानेन वैराग्यविजृम्भितेन
योगेन मय्यर्पितया च भक्त्या मां प्रत्यगात्मानमिहावरुन्धे
देवहूतिरुवाच
काचित्त्वय्युचिता भक्तिः कीदृशी मम गोचरा
यया पदं ते निर्वाणमञ्जसान्वाश्नवा अहम्‌
यो योगो भगवद्बाणो निर्वाणात्मंस्त्वयोदितः
कीदृशः कति चाङ्गानि यतस्तत्त्वावबोधनम्‌
तदेतन्मे विजानीहि यथाहं मन्दधीर्हरे
सुखं बुद्ध्येय दुर्बोधं योषा भवदनुग्रहात्‌
मैत्रेय उवाच
विदित्वार्थं कपिलो मातुरित्थं जातस्नेहो यत्र तन्वाभिजातः
तत्त्वाम्नायं यत्प्रवदन्ति साङ्ख्यं प्रोवाच वै भक्तिवितानयोगम्‌
श्रीभगवानुवाच
देवानां गुणलिङ्गानामानुश्रविककर्मणाम्‌
सत्त्व एवैकमनसो वृत्तिः स्वाभाविकी तु या
अनिमित्ता भागवती भक्तिः सिद्धेर्गरीयसी
जरयत्याशु या कोशं निगीर्णमनलो यथा
नैकात्मतां मे स्पृहयन्ति केचिन्मत्पादसेवाभिरता मदीहाः
येऽन्योन्यतो भागवताः प्रसज्य सभाजयन्ते मम पौरुषाणि
पश्यन्ति ते मे रुचिराण्यम्ब सन्तः प्रसन्नवक्त्रारुणलोचनानि
रूपाणि दिव्यानि वरप्रदानि साकं वाचं स्पृहणीयां वदन्ति
तैर्दर्शनीयावयवैरुदार विलासहासेक्षितवामसूक्तैः
हृतात्मनो हृतप्राणांश्च भक्तिरनिच्छतो मे गतिमण्वीं प्रयुङ्क्ते
अथो विभूतिं मम मायाविनस्तामैश्वर्यमष्टाङ्गमनुप्रवृत्तम्‌
श्रियं भागवतीं वास्पृहयन्ति भद्रां परस्य मे तेऽश्नुवते तु लोके
न कर्हिचिन्मत्पराः शान्तरूपे नङ्क्ष्यन्ति नो मेऽनिमिषो लेढि हेतिः
येषामहं प्रिय आत्मा सुतश्च सखा गुरुः सुहृदो दैवमिष्टम्‌
इमं लोकं तथैवामुमात्मानमुभयायिनम्‌
आत्मानमनु ये चेह ये रायः पशवो गृहाः
विसृज्य सर्वानन्यांश्च मामेवं विश्वतोमुखम्‌
भजन्त्यनन्यया भक्त्या तान्मृत्योरतिपारये
नान्यत्र मद्भगवतः प्रधानपुरुषेश्वरात्‌
आत्मनः सर्वभूतानां भयं तीव्रं निवर्तते
मद्भयाद्वाति वातोऽयं सूर्यस्तपति मद्भयात्‌
वर्षतीन्द्रो दहत्यग्निर्मृत्युश्चरति मद्भयात्‌
ज्ञानवैराग्ययुक्तेन भक्तियोगेन योगिनः
क्षेमाय पादमूलं मे प्रविशन्त्यकुतोभयम्‌
एतावानेव लोकेऽस्मिन्पुंसां निःश्रेयसोदयः
तीव्रेण भक्तियोगेन मनो मय्यर्पितं स्थिरम्‌
श्रीभगवानुवाच
अथ ते सम्प्रवक्ष्यामि तत्त्वानां लक्षणं पृथक्‌
यद्विदित्वा विमुच्येत पुरुषः प्राकृतैर्गुणैः
ज्ञानं निःश्रेयसार्थाय पुरुषस्यात्मदर्शनम्‌
यदाहुर्वर्णये तत्ते हृदयग्रन्थिभेदनम्‌
अनादिरात्मा पुरुषो निर्गुणः प्रकृतेः परः
प्रत्यग्धामा स्वयंज्योतिर्विश्वं येन समन्वितम्‌
स एष प्रकृतिं सूक्ष्मां दैवीं गुणमयीं विभुः
यदृच्छयैवोपगतामभ्यपद्यत लीलया
गुणैर्विचित्राः सृजतीं सरूपाः प्रकृतिं प्रजाः
विलोक्य मुमुहे सद्यः स इह ज्ञानगूहया
एवं पराभिध्यानेन कर्तृत्वं प्रकृतेः पुमान्‌
कर्मसु क्रियमाणेषु गुणैरात्मनि मन्यते
तदस्य संसृतिर्बन्धः पारतन्त्र्यं च तत्कृतम्‌
भवत्यकर्तुरीशस्य साक्षिणो निर्वृतात्मनः
कार्यकारणकर्तृत्वे कारणं प्रकृतिं विदुः
भोक्तृत्वे सुखदुःखानां पुरुषं प्रकृतेः परम्‌
देवहूतिरुवाच
प्रकृतेः पुरुषस्यापि लक्षणं पुरुषोत्तम
ब्रूहि कारणयोरस्य सदसच्च यदात्मकम्‌
श्रीभगवानुवाच
यत्तत्त्रिगुणमव्यक्तं नित्यं सदसदात्मकम्‌
प्रधानं प्रकृतिं प्राहुरविशेषं विशेषवत्‌
पञ्चभिः पञ्चभिर्ब्रह्म चतुर्भिर्दशभिस्तथा
एतच्चतुर्विंशतिकं गणं प्राधानिकं विदुः
महाभूतानि पञ्चैव भूरापोऽग्निर्मरुन्नभः
तन्मात्राणि च तावन्ति गन्धादीनि मतानि मे
इन्द्रियाणि दश श्रोत्रं त्वग्दृग्रसननासिकाः
वाक्करौ चरणौ मेढ्रं पायुर्दशम उच्यते
मनो बुद्धिरहङ्कारश्चित्तमित्यन्तरात्मकम्‌
चतुर्धा लक्ष्यते भेदो वृत्त्या लक्षणरूपया
एतावानेव सङ्ख्यातो ब्रह्मणः सगुणस्य ह
सन्निवेशो मया प्रोक्तो यः कालः पञ्चविंशकः
प्रभावं पौरुषं प्राहुः कालमेके यतो भयम्‌
अहङ्कारविमूढस्य कर्तुः प्रकृतिमीयुषः
प्रकृतेर्गुणसाम्यस्य निर्विशेषस्य मानवि
चेष्टा यतः स भगवान्काल इत्युपलक्षितः
अन्तः पुरुषरूपेण कालरूपेण यो बहिः
समन्वेत्येष सत्त्वानां भगवानात्ममायया
दैवात्क्षुभितधर्मिण्यां स्वस्यां योनौ परः पुमान्‌
आधत्त वीर्यं सासूत महत्तत्त्वं हिरण्मयम्‌
विश्वमात्मगतं व्यञ्जन्कूटस्थो जगदङ्कुरः
स्वतेजसापिबत्तीव्रमात्मप्रस्वापनं तमः
यत्तत्सत्त्वगुणं स्वच्छं शान्तं भगवतः पदम्‌
यदाहुर्वासुदेवाख्यं चित्तं तन्महदात्मकम्‌
स्वच्छत्वमविकारित्वं शान्तत्वमिति चेतसः
वृत्तिभिर्लक्षणं प्रोक्तं यथापां प्रकृतिः परा
महत्तत्त्वाद्विकुर्वाणाद्भगवद्वीर्यसम्भवात्‌
क्रियाशक्तिरहङ्कारस्त्रिविधः समपद्यत
वैकारिकस्तैजसश्च तामसश्च यतो भवः
मनसश्चेन्द्रियाणां च भूतानां महतामपि
सहस्रशिरसं साक्षाद्यमनन्तं प्रचक्षते
सङ्कर्षणाख्यं पुरुषं भूतेन्द्रियमनोमयम्‌
कर्तृत्वं करणत्वं च कार्यत्वं चेति लक्षणम्‌
शान्तघोरविमूढत्वमिति वा स्यादहङ्कृतेः
वैकारिकाद्विकुर्वाणान्मनस्तत्त्वमजायत
यत्सङ्कल्पविकल्पाभ्यां वर्तते कामसम्भवः
यद्विदुर्ह्यनिरुद्धाख्यं हृषीकाणामधीश्वरम्‌
शारदेन्दीवरश्यामं संराध्यं योगिभिः शनैः
तैजसात्तु विकुर्वाणाद्बुद्धितत्त्वमभूत्सति
द्रव्यस्फुरणविज्ञानमिन्द्रियाणामनुग्रहः
संशयोऽथ विपर्यासो निश्चयः स्मृतिरेव च
स्वाप इत्युच्यते बुद्धेर्लक्षणं वृत्तितः पृथक्‌
तैजसानीन्द्रियाण्येव क्रियाज्ञानविभागशः
प्राणस्य हि क्रियाशक्तिर्बुद्धेर्विज्ञानशक्तिता
तामसाच्च विकुर्वाणाद्भगवद्वीर्यचोदितात्‌
शब्दमात्रमभूत्तस्मान्नभः श्रोत्रं तु शब्दगम्‌
अर्थाश्रयत्वं शब्दस्य द्रष्टुर्लिङ्गत्वमेव च
तन्मात्रत्वं च नभसो लक्षणं कवयो विदुः
भूतानां छिद्रदातृत्वं बहिरन्तरमेव च
प्राणेन्द्रियात्मधिष्ण्यत्वं नभसो वृत्तिलक्षणम्‌
नभसः शब्दतन्मात्रात्कालगत्या विकुर्वतः
स्पर्शोऽभवत्ततो वायुस्त्वक्स्पर्शस्य च सङ्ग्रहः
मृदुत्वं कठिनत्वं च शैत्यमुष्णत्वमेव च
एतत्स्पर्शस्य स्पर्शत्वं तन्मात्रत्वं नभस्वतः
चालनं व्यूहनं प्राप्तिर्नेतृत्वं द्रव्यशब्दयोः
सर्वेन्द्रियाणामात्मत्वं वायोः कर्माभिलक्षणम्‌
वायोश्च स्पर्शतन्मात्राद्रूपं दैवेरितादभूत्‌
समुत्थितं ततस्तेजश्चक्षू रूपोपलम्भनम्‌
द्रव्याकृतित्वं गुणता व्यक्तिसंस्थात्वमेव च
तेजस्त्वं तेजसः साध्वि रूपमात्रस्य वृत्तयः
द्योतनं पचनं पानमदनं हिममर्दनम्‌
तेजसो वृत्तयस्त्वेताः शोषणं क्षुत्तृडेव च
रूपमात्राद्विकुर्वाणात्तेजसो दैवचोदितात्‌
रसमात्रमभूत्तस्मादम्भो जिह्वा रसग्रहः
कषायो मधुरस्तिक्तः कट्वम्ल इति नैकधा
भौतिकानां विकारेण रस एको विभिद्यते
क्लेदनं पिण्डनं तृप्तिः प्राणनाप्यायनोन्दनम्‌
तापापनोदो भूयस्त्वमम्भसो वृत्तयस्त्विमाः
रसमात्राद्विकुर्वाणादम्भसो दैवचोदितात्‌
गन्धमात्रमभूत्तस्मात्पृथ्वी घ्राणस्तु गन्धगः
करम्भपूतिसौरभ्य शान्तोग्राम्लादिभिः पृथक्‌
द्रव्यावयववैषम्याद्गन्ध एको विभिद्यते
भावनं ब्रह्मणः स्थानं धारणं सद्विशेषणम्‌
सर्वसत्त्वगुणोद्भेदः पृथिवीवृत्तिलक्षणम्‌
नभोगुणविशेषोऽर्थो यस्य तच्छ्रोत्रमुच्यते
वायोर्गुणविशेषोऽर्थो यस्य तत्स्पर्शनं विदुः
तेजोगुणविशेषोऽर्थो यस्य तच्चक्षुरुच्यते
अम्भोगुणविशेषोऽर्थो यस्य तद्रसनं विदुः
भूमेर्गुणविशेषोऽर्थो यस्य स घ्राण उच्यते
परस्य दृश्यते धर्मो ह्यपरस्मिन्समन्वयात्‌
अतो विशेषो भावानां भूमावेवोपलक्ष्यते
एतान्यसंहत्य यदा महदादीनि सप्त वै
कालकर्मगुणोपेतो जगदादिरुपाविशत्‌
ततस्तेनानुविद्धेभ्यो युक्तेभ्योऽण्डमचेतनम्‌
उत्थितं पुरुषो यस्मादुदतिष्ठदसौ विराट्‌
एतदण्डं विशेषाख्यं क्रमवृद्धैर्दशोत्तरैः
तोयादिभिः परिवृतं प्रधानेनावृतैर्बहिः
यत्र लोकवितानोऽयं रूपं भगवतो हरेः
हिरण्मयादण्डकोशादुत्थाय सलिले शयात्‌
तमाविश्य महादेवो बहुधा निर्बिभेद खम्‌
निरभिद्यतास्य प्रथमं मुखं वाणी ततोऽभवत्‌
वाण्या वह्निरथो नासे प्राणोतो घ्राण एतयोः
घ्राणाद्वायुरभिद्येतामक्षिणी चक्षुरेतयोः
तस्मात्सूर्यो न्यभिद्येतां कर्णौ श्रोत्रं ततो दिशः
निर्बिभेद विराजस्त्वग्‌ रोमश्मश्र्वादयस्ततः
तत ओषधयश्चासन्शिश्नं निर्बिभिदे ततः
रेतस्तस्मादाप आसन्निरभिद्यत वै गुदम्‌
गुदादपानोऽपानाच्च मृत्युर्लोकभयङ्करः
हस्तौ च निरभिद्येतां बलं ताभ्यां ततः स्वराट्‌
पादौ च निरभिद्येतां गतिस्ताभ्यां ततो हरिः
नाड्योऽस्य निरभिद्यन्त ताभ्यो लोहितमाभृतम्‌
नद्यस्ततः समभवन्नुदरं निरभिद्यत
क्षुत्पिपासे ततः स्यातां समुद्रस्त्वेतयोरभूत्‌
अथास्य हृदयं भिन्नं हृदयान्मन उत्थितम्‌
मनसश्चन्द्रमा जातो बुद्धिर्बुद्धेर्गिरां पतिः
अहङ्कारस्ततो रुद्रश्चित्तं चैत्यस्ततोऽभवत्‌
एते ह्यभ्युत्थिता देवा नैवास्योत्थापनेऽशकन्‌
पुनराविविशुः खानि तमुत्थापयितुं क्रमात्‌
वह्निर्वाचा मुखं भेजे नोदतिष्ठत्तदा विराट्‌
घ्राणेन नासिके वायुर्नोदतिष्ठत्तदा विराट्‌
अक्षिणी चक्षुषादित्यो नोदतिष्ठत्तदा विराट्‌
श्रोत्रेण कर्णौ च दिशो नोदतिष्ठत्तदा विराट्‌
त्वचं रोमभिरोषध्यो नोदतिष्ठत्तदा विराट्‌
रेतसा शिश्नमापस्तु नोदतिष्ठत्तदा विराट्‌
गुदं मृत्युरपानेन नोदतिष्ठत्तदा विराट्‌
हस्ताविन्द्रो बलेनैव नोदतिष्ठत्तदा विराट्‌
विष्णुर्गत्यैव चरणौ नोदतिष्ठत्तदा विराट्‌
नाडीर्नद्यो लोहितेन नोदतिष्ठत्तदा विराट्‌
क्षुत्तृड्भ्यामुदरं सिन्धुर्नोदतिष्ठत्तदा विराट्‌
हृदयं मनसा चन्द्रो नोदतिष्ठत्तदा विराट्‌
बुद्ध्या ब्रह्मापि हृदयं नोदतिष्ठत्तदा विराट्‌
रुद्रोऽभिमत्या हृदयं नोदतिष्ठत्तदा विराट्‌
चित्तेन हृदयं चैत्यः क्षेत्रज्ञः प्राविशद्यदा
विराट्तदैव पुरुषः सलिलादुदतिष्ठत
यथा प्रसुप्तं पुरुषं प्राणेन्द्रियमनोधियः
प्रभवन्ति विना येन नोत्थापयितुमोजसा
तमस्मिन्प्रत्यगात्मानं धिया योगप्रवृत्तया
भक्त्या विरक्त्या ज्ञानेन विविच्यात्मनि चिन्तयेत्‌
श्रीभगवानुवाच
प्रकृतिस्थोऽपि पुरुषो नाज्यते प्राकृतैर्गुणैः
अविकारादकर्तृत्वान्निर्गुणत्वाज्जलार्कवत्‌
स एष यर्हि प्रकृतेर्गुणेष्वभिविषज्जते
अहङ्क्रियाविमूढात्मा कर्तास्मीत्यभिमन्यते
तेन संसारपदवीमवशोऽभ्येत्यनिर्वृतः
प्रासङ्गिकैः कर्मदोषैः सदसन्मिश्रयोनिषु
अर्थे ह्यविद्यमानेऽपि संसृतिर्न निवर्तते
ध्यायतो विषयानस्य स्वप्नेऽनर्थागमो यथा
अत एव शनैश्चित्तं प्रसक्तमसतां पथि
भक्तियोगेन तीव्रेण विरक्त्या च नयेद्वशम्‌
यमादिभिर्योगपथैरभ्यसञ्श्रद्धयान्वितः
मयि भावेन सत्येन मत्कथाश्रवणेन च
सर्वभूतसमत्वेन निर्वैरेणाप्रसङ्गतः
ब्रह्मचर्येण मौनेन स्वधर्मेण बलीयसा
यदृच्छयोपलब्धेन सन्तुष्टो मितभुङ्मुनिः
विविक्तशरणः शान्तो मैत्रः करुण आत्मवान्‌
सानुबन्धे च देहेऽस्मिन्नकुर्वन्नसदाग्रहम्‌
ज्ञानेन दृष्टतत्त्वेन प्रकृतेः पुरुषस्य च
निवृत्तबुद्ध्यवस्थानो दूरीभूतान्यदर्शनः
उपलभ्यात्मनात्मानं चक्षुषेवार्कमात्मदृक्‌
मुक्तलिङ्गं सदाभासमसति प्रतिपद्यते
सतो बन्धुमसच्चक्षुः सर्वानुस्यूतमद्वयम्‌
यथा जलस्थ आभासः स्थलस्थेनावदृश्यते
स्वाभासेन तथा सूर्यो जलस्थेन दिवि स्थितः
एवं त्रिवृदहङ्कारो भूतेन्द्रियमनोमयैः
स्वाभासैर्लक्षितोऽनेन सदाभासेन सत्यदृक्‌
भूतसूक्ष्मेन्द्रियमनो बुद्ध्यादिष्विह निद्रया
लीनेष्वसति यस्तत्र विनिद्रो निरहङ्क्रियः
मन्यमानस्तदात्मानमनष्टो नष्टवन्मृषा
नष्टेऽहङ्करणे द्रष्टा नष्टवित्त इवातुरः
एवं प्रत्यवमृश्यासावात्मानं प्रतिपद्यते
साहङ्कारस्य द्रव्यस्य योऽवस्थानमनुग्रहः
देवहूतिरुवाच
पुरुषं प्रकृतिर्ब्रह्मन्न विमुञ्चति कर्हिचित्‌
अन्योन्यापाश्रयत्वाच्च नित्यत्वादनयोः प्रभो
यथा गन्धस्य भूमेश्च न भावो व्यतिरेकतः
अपां रसस्य च यथा तथा बुद्धेः परस्य च
अकर्तुः कर्मबन्धोऽयं पुरुषस्य यदाश्रयः
गुणेषु सत्सु प्रकृतेः कैवल्यं तेष्वतः कथम्‌
क्वचित्तत्त्वावमर्शेन निवृत्तं भयमुल्बणम्‌
अनिवृत्तनिमित्तत्वात्पुनः प्रत्यवतिष्ठते
श्रीभगवानुवाच
अनिमित्तनिमित्तेन स्वधर्मेणामलात्मना
तीव्रया मयि भक्त्या च श्रुतसम्भृतया चिरम्‌
ज्ञानेन दृष्टतत्त्वेन वैराग्येण बलीयसा
तपोयुक्तेन योगेन तीव्रेणात्मसमाधिना
प्रकृतिः पुरुषस्येह दह्यमाना त्वहर्निशम्‌
तिरोभवित्री शनकैरग्नेर्योनिरिवारणिः
भुक्तभोगा परित्यक्ता दृष्टदोषा च नित्यशः
नेश्वरस्याशुभं धत्ते स्वे महिम्नि स्थितस्य च
यथा ह्यप्रतिबुद्धस्य प्रस्वापो बह्वनर्थभृत्‌
स एव प्रतिबुद्धस्य न वै मोहाय कल्पते
एवं विदिततत्त्वस्य प्रकृतिर्मयि मानसम्‌
युञ्जतो नापकुरुत आत्मारामस्य कर्हिचित्‌
यदैवमध्यात्मरतः कालेन बहुजन्मना
सर्वत्र जातवैराग्य आब्रह्मभुवनान्मुनिः
मद्भक्तः प्रतिबुद्धार्थो मत्प्रसादेन भूयसा
निःश्रेयसं स्वसंस्थानं कैवल्याख्यं मदाश्रयम्‌
प्राप्नोतीहाञ्जसा धीरः स्वदृशा च्छिन्नसंशयः
यद्गत्वा न निवर्तेत योगी लिङ्गाद्विनिर्गमे
यदा न योगोपचितासु चेतो मायासु सिद्धस्य विषज्जतेऽङ्ग
अनन्यहेतुष्वथ मे गतिः स्यादात्यन्तिकी यत्र न मृत्युहासः
श्रीभगवानुवाच
योगस्य लक्षणं वक्ष्ये सबीजस्य नृपात्मजे
मनो येनैव विधिना प्रसन्नं याति सत्पथम्‌
स्वधर्माचरणं शक्त्या विधर्माच्च निवर्तनम्‌
दैवाल्लब्धेन सन्तोष आत्मविच्चरणार्चनम्‌
ग्राम्यधर्मनिवृत्तिश्च मोक्षधर्मरतिस्तथा
मितमेध्यादनं शश्वद्विविक्तक्षेमसेवनम्‌
अहिंसा सत्यमस्तेयं यावदर्थपरिग्रहः
ब्रह्मचर्यं तपः शौचं स्वाध्यायः पुरुषार्चनम्‌
मौनं सदासनजयः स्थैर्यं प्राणजयः शनैः
प्रत्याहारश्चेन्द्रियाणां विषयान्मनसा हृदि
स्वधिष्ण्यानामेकदेशे मनसा प्राणधारणम्‌
वैकुण्ठलीलाभिध्यानं समाधानं तथात्मनः
एतैरन्यैश्च पथिभिर्मनो दुष्टमसत्पथम्‌
बुद्ध्या युञ्जीत शनकैर्जितप्राणो ह्यतन्द्रितः
शुचौ देशे प्रतिष्ठाप्य विजितासन आसनम्‌
तस्मिन्स्वस्ति समासीन ऋजुकायः समभ्यसेत्‌
प्राणस्य शोधयेन्मार्गं पूरकुम्भकरेचकैः
प्रतिकूलेन वा चित्तं यथा स्थिरमचञ्चलम्‌
मनोऽचिरात्स्याद्विरजं जितश्वासस्य योगिनः
वाय्वग्निभ्यां यथा लोहं ध्मातं त्यजति वै मलम्‌
प्राणायामैर्दहेद्दोषान्धारणाभिश्च किल्बिषान्‌
प्रत्याहारेण संसर्गान्ध्यानेनानीश्वरान्गुणान्‌
यदा मनः स्वं विरजं योगेन सुसमाहितम्‌
काष्ठां भगवतो ध्यायेत्स्वनासाग्रावलोकनः
प्रसन्नवदनाम्भोजं पद्मगर्भारुणेक्षणम्‌
नीलोत्पलदलश्यामं शङ्खचक्रगदाधरम्‌
लसत्पङ्कजकिञ्जल्क पीतकौशेयवाससम्‌
श्रीवत्सवक्षसं भ्राजत्कौस्तुभामुक्तकन्धरम्‌
मत्तद्विरेफकलया परीतं वनमालया
परार्ध्यहारवलय किरीटाङ्गदनूपुरम्‌
काञ्चीगुणोल्लसच्छ्रोणिं हृदयाम्भोजविष्टरम्‌
दर्शनीयतमं शान्तं मनोनयनवर्धनम्‌
अपीच्यदर्शनं शश्वत्सर्वलोकनमस्कृतम्‌
सन्तं वयसि कैशोरे भृत्यानुग्रहकातरम्‌
कीर्तन्यतीर्थयशसं पुण्यश्लोकयशस्करम्‌
ध्यायेद्देवं समग्राङ्गं यावन्न च्यवते मनः
स्थितं व्रजन्तमासीनं शयानं वा गुहाशयम्‌
प्रेक्षणीयेहितं ध्यायेच्छुद्धभावेन चेतसा
तस्मिन्लब्धपदं चित्तं सर्वावयवसंस्थितम्‌
विलक्ष्यैकत्र संयुज्यादङ्गे भगवतो मुनिः
सञ्चिन्तयेद्भगवतश्चरणारविन्दं
वज्राङ्कुशध्वजसरोरुहलाञ्छनाढ्यम्‌
उत्तुङ्गरक्तविलसन्नखचक्रवाल
ज्योत्स्नाभिराहतमहद्धृदयान्धकारम्‌
यच्छौचनिःसृतसरित्प्रवरोदकेन
तीर्थेन मूर्ध्न्यधिकृतेन शिवः शिवोऽभूत्‌
ध्यातुर्मनःशमलशैलनिसृष्टवज्रं
ध्यायेच्चिरं भगवतश्चरणारविन्दम्‌
जानुद्वयं जलजलोचनया जनन्या
लक्ष्म्याखिलस्य सुरवन्दितया विधातुः
ऊर्वोर्निधाय करपल्लवरोचिषा यत्‌
संलालितं हृदि विभोरभवस्य कुर्यात्‌
ऊरू सुपर्णभुजयोरधि शोभमानाव्‌
ओजोनिधी अतसिकाकुसुमावभासौ
व्यालम्बिपीतवरवाससि वर्तमान
काञ्चीकलापपरिरम्भि नितम्बबिम्बम्‌
नाभिह्रदं भुवनकोशगुहोदरस्थं
यत्रात्मयोनिधिषणाखिललोकपद्मम्‌
व्यूढं हरिन्मणिवृषस्तनयोरमुष्य
ध्यायेद्द्वयं विशदहारमयूखगौरम्‌
वक्षोऽधिवासमृषभस्य महाविभूतेः
पुंसां मनोनयननिर्वृतिमादधानम्‌
कण्ठं च कौस्तुभमणेरधिभूषणार्थं
कुर्यान्मनस्यखिललोकनमस्कृतस्य
बाहूंश्च मन्दरगिरेः परिवर्तनेन
निर्णिक्तबाहुवलयानधिलोकपालान्‌
सञ्चिन्तयेद्दशशतारमसह्यतेजः
शङ्खं च तत्करसरोरुहराजहंसम्‌
कौमोदकीं भगवतो दयितां स्मरेत
दिग्धामरातिभटशोणितकर्दमेन
मालां मधुव्रतवरूथगिरोपघुष्टां
चैत्यस्य तत्त्वममलं मणिमस्य कण्ठे
भृत्यानुकम्पितधियेह गृहीतमूर्तेः
सञ्चिन्तयेद्भगवतो वदनारविन्दम्‌
यद्विस्फुरन्मकरकुण्डलवल्गितेन
विद्योतितामलकपोलमुदारनासम्‌
यच्छ्रीनिकेतमलिभिः परिसेव्यमानं
भूत्या स्वया कुटिलकुन्तलवृन्दजुष्टम्‌
मीनद्वयाश्रयमधिक्षिपदब्जनेत्रं
ध्यायेन्मनोमयमतन्द्रित उल्लसद्भ्रु
तस्यावलोकमधिकं कृपयातिघोर
तापत्रयोपशमनाय निसृष्टमक्ष्णोः
स्निग्धस्मितानुगुणितं विपुलप्रसादं
ध्यायेच्चिरं विपुलभावनया गुहायाम्‌
हासं हरेरवनताखिललोकतीव्र
शोकाश्रुसागरविशोषणमत्युदारम्‌
सम्मोहनाय रचितं निजमाययास्य
भ्रूमण्डलं मुनिकृते मकरध्वजस्य
ध्यानायनं प्रहसितं बहुलाधरोष्ठ
भासारुणायिततनुद्विजकुन्दपङ्क्ति
ध्यायेत्स्वदेहकुहरेऽवसितस्य विष्णोर्‌
भक्त्यार्द्रयार्पितमना न पृथग्दिदृक्षेत्‌
एवं हरौ भगवति प्रतिलब्धभावो
भक्त्या द्रवद्धृदय उत्पुलकः प्रमोदात्‌
औत्कण्ठ्यबाष्पकलया मुहुरर्द्यमानस्‌
तच्चापि चित्तबडिशं शनकैर्वियुङ्क्ते
मुक्ताश्रयं यर्हि निर्विषयं विरक्तं
निर्वाणमृच्छति मनः सहसा यथार्चिः
आत्मानमत्र पुरुषोऽव्यवधानमेकम्‌
अन्वीक्षते प्रतिनिवृत्तगुणप्रवाहः
सोऽप्येतया चरमया मनसो निवृत्त्या
तस्मिन्महिम्न्यवसितः सुखदुःखबाह्ये
हेतुत्वमप्यसति कर्तरि दुःखयोर्यत्‌
स्वात्मन्विधत्त उपलब्धपरात्मकाष्ठः
देहं च तं न चरमः स्थितमुत्थितं वा
सिद्धो विपश्यति यतोऽध्यगमत्स्वरूपम्‌
दैवादुपेतमथ दैववशादपेतं
वासो यथा परिकृतं मदिरामदान्धः
देहोऽपि दैववशगः खलु कर्म यावत्‌
स्वारम्भकं प्रतिसमीक्षत एव सासुः
तं सप्रपञ्चमधिरूढसमाधियोगः
स्वाप्नं पुनर्न भजते प्रतिबुद्धवस्तुः
यथा पुत्राच्च वित्ताच्च पृथङ्मर्त्यः प्रतीयते
अप्यात्मत्वेनाभिमताद्देहादेः पुरुषस्तथा
यथोल्मुकाद्विस्फुलिङ्गाद्धूमाद्वापि स्वसम्भवात्‌
अप्यात्मत्वेनाभिमताद्यथाग्निः पृथगुल्मुकात्‌
भूतेन्द्रियान्तःकरणात्प्रधानाज्जीवसंज्ञितात्‌
आत्मा तथा पृथग्द्रष्टा भगवान्ब्रह्मसंज्ञितः
सर्वभूतेषु चात्मानं सर्वभूतानि चात्मनि
ईक्षेतानन्यभावेन भूतेष्विव तदात्मताम्‌
स्वयोनिषु यथा ज्योतिरेकं नाना प्रतीयते
योनीनां गुणवैषम्यात्तथात्मा प्रकृतौ स्थितः
तस्मादिमां स्वां प्रकृतिं दैवीं सदसदात्मिकाम्‌
दुर्विभाव्यां पराभाव्य स्वरूपेणावतिष्ठते
देवहूतिरुवाच
लक्षणं महदादीनां प्रकृतेः पुरुषस्य च
स्वरूपं लक्ष्यतेऽमीषां येन तत्पारमार्थिकम्‌
यथा साङ्ख्येषु कथितं यन्मूलं तत्प्रचक्षते
भक्तियोगस्य मे मार्गं ब्रूहि विस्तरशः प्रभो
विरागो येन पुरुषो भगवन्सर्वतो भवेत्‌
आचक्ष्व जीवलोकस्य विविधा मम संसृतीः
कालस्येश्वररूपस्य परेषां च परस्य ते
स्वरूपं बत कुर्वन्ति यद्धेतोः कुशलं जनाः
लोकस्य मिथ्याभिमतेरचक्षुषश्चिरं प्रसुप्तस्य तमस्यनाश्रये
श्रान्तस्य कर्मस्वनुविद्धया धिया त्वमाविरासीः किल योगभास्करः
मैत्रेय उवाच
इति मातुर्वचः श्लक्ष्णं प्रतिनन्द्य महामुनिः
आबभाषे कुरुश्रेष्ठ प्रीतस्तां करुणार्दितः
श्रीभगवानुवाच
भक्तियोगो बहुविधो मार्गैर्भामिनि भाव्यते
स्वभावगुणमार्गेण पुंसां भावो विभिद्यते
अभिसन्धाय यो हिंसां दम्भं मात्सर्यमेव वा
संरम्भी भिन्नदृग्भावं मयि कुर्यात्स तामसः
विषयानभिसन्धाय यश ऐश्वर्यमेव वा
अर्चादावर्चयेद्यो मां पृथग्भावः स राजसः
कर्मनिर्हारमुद्दिश्य परस्मिन्वा तदर्पणम्‌
यजेद्यष्टव्यमिति वा पृथग्भावः स सात्त्विकः
मद्गुणश्रुतिमात्रेण मयि सर्वगुहाशये
मनोगतिरविच्छिन्ना यथा गङ्गाम्भसोऽम्बुधौ
लक्षणं भक्तियोगस्य निर्गुणस्य ह्युदाहृतम्‌
अहैतुक्यव्यवहिता या भक्तिः पुरुषोत्तमे
सालोक्यसार्ष्टिसामीप्य सारूप्यैकत्वमप्युत
दीयमानं न गृह्णन्ति विना मत्सेवनं जनाः
स एव भक्तियोगाख्य आत्यन्तिक उदाहृतः
येनातिव्रज्य त्रिगुणं मद्भावायोपपद्यते
निषेवितेनानिमित्तेन स्वधर्मेण महीयसा
क्रियायोगेन शस्तेन नातिहिंस्रेण नित्यशः
मद्धिष्ण्यदर्शनस्पर्श पूजास्तुत्यभिवन्दनैः
भूतेषु मद्भावनया सत्त्वेनासङ्गमेन च
महतां बहुमानेन दीनानामनुकम्पया
मैत्र्या चैवात्मतुल्येषु यमेन नियमेन च
आध्यात्मिकानुश्रवणान्नामसङ्कीर्तनाच्च मे
आर्जवेनार्यसङ्गेन निरहङ्क्रियया तथा
मद्धर्मणो गुणैरेतैः परिसंशुद्ध आशयः
पुरुषस्याञ्जसाभ्येति श्रुतमात्रगुणं हि माम्‌
यथा वातरथो घ्राणमावृङ्क्ते गन्ध आशयात्‌
एवं योगरतं चेत आत्मानमविकारि यत्‌
अहं सर्वेषु भूतेषु भूतात्मावस्थितः सदा
तमवज्ञाय मां मर्त्यः कुरुतेऽर्चाविडम्बनम्‌
यो मां सर्वेषु भूतेषु सन्तमात्मानमीश्वरम्‌
हित्वार्चां भजते मौढ्याद्भस्मन्येव जुहोति सः
द्विषतः परकाये मां मानिनो भिन्नदर्शिनः
भूतेषु बद्धवैरस्य न मनः शान्तिमृच्छति
अहमुच्चावचैर्द्रव्यैः क्रिययोत्पन्नयानघे
नैव तुष्येऽर्चितोऽर्चायां भूतग्रामावमानिनः
अर्चादावर्चयेत्तावदीश्वरं मां स्वकर्मकृत्‌
यावन्न वेद स्वहृदि सर्वभूतेष्ववस्थितम्‌
आत्मनश्च परस्यापि यः करोत्यन्तरोदरम्‌
तस्य भिन्नदृशो मृत्युर्विदधे भयमुल्बणम्‌
अथ मां सर्वभूतेषु भूतात्मानं कृतालयम्‌
अर्हयेद्दानमानाभ्यां मैत्र्याभिन्नेन चक्षुषा
जीवाः श्रेष्ठा ह्यजीवानां ततः प्राणभृतः शुभे
ततः सचित्ताः प्रवरास्ततश्चेन्द्रियवृत्तयः
तत्रापि स्पर्शवेदिभ्यः प्रवरा रसवेदिनः
तेभ्यो गन्धविदः श्रेष्ठास्ततः शब्दविदो वराः
रूपभेदविदस्तत्र ततश्चोभयतोदतः
तेषां बहुपदाः श्रेष्ठाश्चतुष्पादस्ततो द्विपात्‌
ततो वर्णाश्च चत्वारस्तेषां ब्राह्मण उत्तमः
ब्राह्मणेष्वपि वेदज्ञो ह्यर्थज्ञोऽभ्यधिकस्ततः
अर्थज्ञात्संशयच्छेत्ता ततः श्रेयान्स्वकर्मकृत्‌
मुक्तसङ्गस्ततो भूयानदोग्धा धर्ममात्मनः
तस्मान्मय्यर्पिताशेष क्रियार्थात्मा निरन्तरः
मय्यर्पितात्मनः पुंसो मयि सन्न्यस्तकर्मणः
न पश्यामि परं भूतमकर्तुः समदर्शनात्‌
मनसैतानि भूतानि प्रणमेद्बहुमानयन्‌
ईश्वरो जीवकलया प्रविष्टो भगवानिति
भक्तियोगश्च योगश्च मया मानव्युदीरितः
ययोरेकतरेणैव पुरुषः पुरुषं व्रजेत्‌
एतद्भगवतो रूपं ब्रह्मणः परमात्मनः
परं प्रधानं पुरुषं दैवं कर्मविचेष्टितम्‌
रूपभेदास्पदं दिव्यं काल इत्यभिधीयते
भूतानां महदादीनां यतो भिन्नदृशां भयम्‌
योऽन्तः प्रविश्य भूतानि भूतैरत्त्यखिलाश्रयः
स विष्ण्वाख्योऽधियज्ञोऽसौ कालः कलयतां प्रभुः
न चास्य कश्चिद्दयितो न द्वेष्यो न च बान्धवः
आविशत्यप्रमत्तोऽसौ प्रमत्तं जनमन्तकृत्‌
यद्भयाद्वाति वातोऽयं सूर्यस्तपति यद्भयात्‌
यद्भयाद्वर्षते देवो भगणो भाति यद्भयात्‌
यद्वनस्पतयो भीता लताश्चौषधिभिः सह
स्वे स्वे कालेऽभिगृह्णन्ति पुष्पाणि च फलानि च
स्रवन्ति सरितो भीता नोत्सर्पत्युदधिर्यतः
अग्निरिन्धे सगिरिभिर्भूर्न मज्जति यद्भयात्‌
नभो ददाति श्वसतां पदं यन्नियमाददः
लोकं स्वदेहं तनुते महान्सप्तभिरावृतम्‌
गुणाभिमानिनो देवाः सर्गादिष्वस्य यद्भयात्‌
वर्तन्तेऽनुयुगं येषां वश एतच्चराचरम्‌
सोऽनन्तोऽन्तकरः कालोऽनादिरादिकृदव्ययः
जनं जनेन जनयन्मारयन्मृत्युनान्तकम्‌
कपिल उवाच
तस्यैतस्य जनो नूनं नायं वेदोरुविक्रमम्‌
काल्यमानोऽपि बलिनो वायोरिव घनावलिः
यं यमर्थमुपादत्ते दुःखेन सुखहेतवे
तं तं धुनोति भगवान्पुमान्छोचति यत्कृते
यदध्रुवस्य देहस्य सानुबन्धस्य दुर्मतिः
ध्रुवाणि मन्यते मोहाद्गृहक्षेत्रवसूनि च
जन्तुर्वै भव एतस्मिन्यां यां योनिमनुव्रजेत्‌
तस्यां तस्यां स लभते निर्वृतिं न विरज्यते
नरकस्थोऽपि देहं वै न पुमांस्त्यक्तुमिच्छति
नारक्यां निर्वृतौ सत्यां देवमायाविमोहितः
आत्मजायासुतागार पशुद्रविणबन्धुषु
निरूढमूलहृदय आत्मानं बहु मन्यते
सन्दह्यमानसर्वाङ्ग एषामुद्वहनाधिना
करोत्यविरतं मूढो दुरितानि दुराशयः
आक्षिप्तात्मेन्द्रियः स्त्रीणामसतीनां च मायया
रहो रचितयालापैः शिशूनां कलभाषिणाम्‌
गृहेषु कूटधर्मेषु दुःखतन्त्रेष्वतन्द्रितः
कुर्वन्दुःखप्रतीकारं सुखवन्मन्यते गृही
अर्थैरापादितैर्गुर्व्या हिंसयेतस्ततश्च तान्‌
पुष्णाति येषां पोषेण शेषभुग्यात्यधः स्वयम्‌
वार्तायां लुप्यमानायामारब्धायां पुनः पुनः
लोभाभिभूतो निःसत्त्वः परार्थे कुरुते स्पृहाम्‌
कुटुम्बभरणाकल्पो मन्दभाग्यो वृथोद्यमः
श्रिया विहीनः कृपणो ध्यायन्छ्वसिति मूढधीः
एवं स्वभरणाकल्पं तत्कलत्रादयस्तथा
नाद्रियन्ते यथा पूर्वं कीनाशा इव गोजरम्‌
तत्राप्यजातनिर्वेदो भ्रियमाणः स्वयम्भृतैः
जरयोपात्तवैरूप्यो मरणाभिमुखो गृहे
आस्तेऽवमत्योपन्यस्तं गृहपाल इवाहरन्‌
आमयाव्यप्रदीप्ताग्निरल्पाहारोऽल्पचेष्टितः
वायुनोत्क्रमतोत्तारः कफसंरुद्धनाडिकः
कासश्वासकृतायासः कण्ठे घुरघुरायते
शयानः परिशोचद्भिः परिवीतः स्वबन्धुभिः
वाच्यमानोऽपि न ब्रूते कालपाशवशं गतः
एवं कुटुम्बभरणे व्यापृतात्माजितेन्द्रियः
म्रियते रुदतां स्वानामुरुवेदनयास्तधीः
यमदूतौ तदा प्राप्तौ भीमौ सरभसेक्षणौ
स दृष्ट्वा त्रस्तहृदयः शकृन्मूत्रं विमुञ्चति
यातनादेह आवृत्य पाशैर्बद्ध्वा गले बलात्‌
नयतो दीर्घमध्वानं दण्ड्यं राजभटा यथा
तयोर्निर्भिन्नहृदयस्तर्जनैर्जातवेपथुः
पथि श्वभिर्भक्ष्यमाण आर्तोऽघं स्वमनुस्मरन्‌
क्षुत्तृट्परीतोऽर्कदवानलानिलैः सन्तप्यमानः पथि तप्तवालुके
कृच्छ्रेण पृष्ठे कशया च ताडितश्चलत्यशक्तोऽपि निराश्रमोदके
तत्र तत्र पतन्छ्रान्तो मूर्च्छितः पुनरुत्थितः
पथा पापीयसा नीतस्तरसा यमसादनम्‌
योजनानां सहस्राणि नवतिं नव चाध्वनः
त्रिभिर्मुहूर्तैर्द्वाभ्यां वा नीतः प्राप्नोति यातनाः
आदीपनं स्वगात्राणां वेष्टयित्वोल्मुकादिभिः
आत्ममांसादनं क्वापि स्वकृत्तं परतोऽपि वा
जीवतश्चान्त्राभ्युद्धारः श्वगृध्रैर्यमसादने
सर्पवृश्चिकदंशाद्यैर्दशद्भिश्चात्मवैशसम्‌
कृन्तनं चावयवशो गजादिभ्यो भिदापनम्‌
पातनं गिरिशृङ्गेभ्यो रोधनं चाम्बुगर्तयोः
यास्तामिस्रान्धतामिस्रा रौरवाद्याश्च यातनाः
भुङ्क्ते नरो वा नारी वा मिथः सङ्गेन निर्मिताः
अत्रैव नरकः स्वर्ग इति मातः प्रचक्षते
या यातना वै नारक्यस्ता इहाप्युपलक्षिताः
एवं कुटुम्बं बिभ्राण उदरम्भर एव वा
विसृज्येहोभयं प्रेत्य भुङ्क्ते तत्फलमीदृशम्‌
एकः प्रपद्यते ध्वान्तं हित्वेदं स्वकलेवरम्‌
कुशलेतरपाथेयो भूतद्रोहेण यद्भृतम्‌
दैवेनासादितं तस्य शमलं निरये पुमान्‌
भुङ्क्ते कुटुम्बपोषस्य हृतवित्त इवातुरः
केवलेन ह्यधर्मेण कुटुम्बभरणोत्सुकः
याति जीवोऽन्धतामिस्रं चरमं तमसः पदम्‌
अधस्तान्नरलोकस्य यावतीर्यातनादयः
क्रमशः समनुक्रम्य पुनरत्राव्रजेच्छुचिः
श्रीभगवानुवाच
कर्मणा दैवनेत्रेण जन्तुर्देहोपपत्तये
स्त्रियाः प्रविष्ट उदरं पुंसो रेतःकणाश्रयः
कललं त्वेकरात्रेण पञ्चरात्रेण बुद्बुदम्‌
दशाहेन तु कर्कन्धूः पेश्यण्डं वा ततः परम्‌
मासेन तु शिरो द्वाभ्यां बाह्वङ्घ्र्याद्यङ्गविग्रहः
नखलोमास्थिचर्माणि लिङ्गच्छिद्रोद्भवस्त्रिभिः
चतुर्भिर्धातवः सप्त पञ्चभिः क्षुत्तृडुद्भवः
षड्भिर्जरायुणा वीतः कुक्षौ भ्राम्यति दक्षिणे
मातुर्जग्धान्नपानाद्यैरेधद्धातुरसम्मते
शेते विण्मूत्रयोर्गर्ते स जन्तुर्जन्तुसम्भवे
कृमिभिः क्षतसर्वाङ्गः सौकुमार्यात्प्रतिक्षणम्‌
मूर्च्छामाप्नोत्युरुक्लेशस्तत्रत्यैः क्षुधितैर्मुहुः
कटुतीक्ष्णोष्णलवण रूक्षाम्लादिभिरुल्बणैः
मातृभुक्तैरुपस्पृष्टः सर्वाङ्गोत्थितवेदनः
उल्बेन संवृतस्तस्मिन्नन्त्रैश्च बहिरावृतः
आस्ते कृत्वा शिरः कुक्षौ भुग्नपृष्ठशिरोधरः
अकल्पः स्वाङ्गचेष्टायां शकुन्त इव पञ्जरे
तत्र लब्धस्मृतिर्दैवात्कर्म जन्मशतोद्भवम्‌
स्मरन्दीर्घमनुच्छ्वासं शर्म किं नाम विन्दते
आरभ्य सप्तमान्मासाल्लब्धबोधोऽपि वेपितः
नैकत्रास्ते सूतिवातैर्विष्ठाभूरिव सोदरः
नाथमान ऋषिर्भीतः सप्तवध्रिः कृताञ्जलिः
स्तुवीत तं विक्लवया वाचा येनोदरेऽर्पितः
जन्तुरुवाच
तस्योपसन्नमवितुं जगदिच्छयात्त
नानातनोर्भुवि चलच्चरणारविन्दम्‌
सोऽहं व्रजामि शरणं ह्यकुतोभयं मे
येनेदृशी गतिरदर्श्यसतोऽनुरूपा
यस्त्वत्र बद्ध इव कर्मभिरावृतात्मा
भूतेन्द्रियाशयमयीमवलम्ब्य मायाम्‌
आस्ते विशुद्धमविकारमखण्डबोधम्‌
आतप्यमानहृदयेऽवसितं नमामि
यः पञ्चभूतरचिते रहितः शरीरे
च्छन्नोऽयथेन्द्रियगुणार्थचिदात्मकोऽहम्‌
तेनाविकुण्ठमहिमानमृषिं तमेनं
वन्दे परं प्रकृतिपूरुषयोः पुमांसम्‌
यन्माययोरुगुणकर्मनिबन्धनेऽस्मिन्‌
सांसारिके पथि चरंस्तदभिश्रमेण
नष्टस्मृतिः पुनरयं प्रवृणीत लोकं
युक्त्या कया महदनुग्रहमन्तरेण
ज्ञानं यदेतददधात्कतमः स देवस्‌
त्रैकालिकं स्थिरचरेष्वनुवर्तितांशः
तं जीवकर्मपदवीमनुवर्तमानास्‌
तापत्रयोपशमनाय वयं भजेम
देह्यन्यदेहविवरे जठराग्निनासृग्‌
विण्मूत्रकूपपतितो भृशतप्तदेहः
इच्छन्नितो विवसितुं गणयन्स्वमासान्‌
निर्वास्यते कृपणधीर्भगवन्कदा नु
येनेदृशीं गतिमसौ दशमास्य ईश
सङ्ग्राहितः पुरुदयेन भवादृशेन
स्वेनैव तुष्यतु कृतेन स दीननाथः
को नाम तत्प्रति विनाञ्जलिमस्य कुर्यात्‌
पश्यत्ययं धिषणया ननु सप्तवध्रिः
शारीरके दमशरीर्यपरः स्वदेहे
यत्सृष्टयासं तमहं पुरुषं पुराणं
पश्ये बहिर्हृदि च चैत्यमिव प्रतीतम्‌
सोऽहं वसन्नपि विभो बहुदुःखवासं
गर्भान्न निर्जिगमिषे बहिरन्धकूपे
यत्रोपयातमुपसर्पति देवमाया
मिथ्या मतिर्यदनु संसृतिचक्रमेतत्‌
तस्मादहं विगतविक्लव उद्धरिष्य
आत्मानमाशु तमसः सुहृदात्मनैव
भूयो यथा व्यसनमेतदनेकरन्ध्रं
मा मे भविष्यदुपसादितविष्णुपादः
कपिल उवाच
एवं कृतमतिर्गर्भे दशमास्यः स्तुवन्नृषिः
सद्यः क्षिपत्यवाचीनं प्रसूत्यै सूतिमारुतः
तेनावसृष्टः सहसा कृत्वावाक्षिर आतुरः
विनिष्क्रामति कृच्छ्रेण निरुच्छ्वासो हतस्मृतिः
पतितो भुव्यसृङ्मिश्रः विष्ठाभूरिव चेष्टते
रोरूयति गते ज्ञाने विपरीतां गतिं गतः
परच्छन्दं न विदुषा पुष्यमाणो जनेन सः
अनभिप्रेतमापन्नः प्रत्याख्यातुमनीश्वरः
शायितोऽशुचिपर्यङ्के जन्तुः स्वेदजदूषिते
नेशः कण्डूयनेऽङ्गानामासनोत्थानचेष्टने
तुदन्त्यामत्वचं दंशा मशका मत्कुणादयः
रुदन्तं विगतज्ञानं कृमयः कृमिकं यथा
इत्येवं शैशवं भुक्त्वा दुःखं पौगण्डमेव च
अलब्धाभीप्सितोऽज्ञानादिद्धमन्युः शुचार्पितः
सह देहेन मानेन वर्धमानेन मन्युना
करोति विग्रहं कामी कामिष्वन्ताय चात्मनः
भूतैः पञ्चभिरारब्धे देहे देह्यबुधोऽसकृत्‌
अहं ममेत्यसद्ग्राहः करोति कुमतिर्मतिम्‌
तदर्थं कुरुते कर्म यद्बद्धो याति संसृतिम्‌
योऽनुयाति ददत्क्लेशमविद्याकर्मबन्धनः
यद्यसद्भिः पथि पुनः शिश्नोदरकृतोद्यमैः
आस्थितो रमते जन्तुस्तमो विशति पूर्ववत्‌
सत्यं शौचं दया मौनं बुद्धिः श्रीर्ह्रीर्यशः क्षमा
शमो दमो भगश्चेति यत्सङ्गाद्याति सङ्क्षयम्‌
तेष्वशान्तेषु मूढेषु खण्डितात्मस्वसाधुषु
सङ्गं न कुर्याच्छोच्येषु योषित्क्रीडामृगेषु च
न तथास्य भवेन्मोहो बन्धश्चान्यप्रसङ्गतः
योषित्सङ्गाद्यथा पुंसो यथा तत्सङ्गिसङ्गतः
प्रजापतिः स्वां दुहितरं दृष्ट्वा तद्रूपधर्षितः
रोहिद्भूतां सोऽन्वधावदृक्षरूपी हतत्रपः
तत्सृष्टसृष्टसृष्टेषु को न्वखण्डितधीः पुमान्‌
ऋषिं नारायणमृते योषिन्मय्येह मायया
बलं मे पश्य मायायाः स्त्रीमय्या जयिनो दिशाम्‌
या करोति पदाक्रान्तान्भ्रूविजृम्भेण केवलम्‌
सङ्गं न कुर्यात्प्रमदासु जातु योगस्य पारं परमारुरुक्षुः
मत्सेवया प्रतिलब्धात्मलाभो वदन्ति या निरयद्वारमस्य
योपयाति शनैर्माया योषिद्देवविनिर्मिता
तामीक्षेतात्मनो मृत्युं तृणैः कूपमिवावृतम्‌
यां मन्यते पतिं मोहान्मन्मायामृषभायतीम्‌
स्त्रीत्वं स्त्रीसङ्गतः प्राप्तो वित्तापत्यगृहप्रदम्‌
तामात्मनो विजानीयात्पत्यपत्यगृहात्मकम्‌
दैवोपसादितं मृत्युं मृगयोर्गायनं यथा
देहेन जीवभूतेन लोकाल्लोकमनुव्रजन्‌
भुञ्जान एव कर्माणि करोत्यविरतं पुमान्‌
जीवो ह्यस्यानुगो देहो भूतेन्द्रियमनोमयः
तन्निरोधोऽस्य मरणमाविर्भावस्तु सम्भवः
द्रव्योपलब्धिस्थानस्य द्रव्येक्षायोग्यता यदा
तत्पञ्चत्वमहंमानादुत्पत्तिर्द्रव्यदर्शनम्‌
यथाक्ष्णोर्द्रव्यावयव दर्शनायोग्यता यदा
तदैव चक्षुषो द्रष्टुर्द्रष्टृत्वायोग्यतानयोः
तस्मान्न कार्यः सन्त्रासो न कार्पण्यं न सम्भ्रमः
बुद्ध्वा जीवगतिं धीरो मुक्तसङ्गश्चरेदिह
सम्यग्दर्शनया बुद्ध्या योगवैराग्ययुक्तया
मायाविरचिते लोके चरेन्न्यस्य कलेवरम्‌
कपिल उवाच
अथ यो गृहमेधीयान्धर्मानेवावसन्गृहे
काममर्थं च धर्मान्स्वान्दोग्धि भूयः पिपर्ति तान्‌
स चापि भगवद्धर्मात्काममूढः पराङ्मुखः
यजते क्रतुभिर्देवान्पित्‌ंश्च श्रद्धयान्वितः
तच्छ्रद्धयाक्रान्तमतिः पितृदेवव्रतः पुमान्‌
गत्वा चान्द्रमसं लोकं सोमपाः पुनरेष्यति
यदा चाहीन्द्रशय्यायां शेतेऽनन्तासनो हरिः
तदा लोका लयं यान्ति त एते गृहमेधिनाम्‌
ये स्वधर्मान्न दुह्यन्ति धीराः कामार्थहेतवे
निःसङ्गा न्यस्तकर्माणः प्रशान्ताः शुद्धचेतसः
निवृत्तिधर्मनिरता निर्ममा निरहङ्कृताः
स्वधर्माप्तेन सत्त्वेन परिशुद्धेन चेतसा
सूर्यद्वारेण ते यान्ति पुरुषं विश्वतोमुखम्‌
परावरेशं प्रकृतिमस्योत्पत्त्यन्तभावनम्‌
द्विपरार्धावसाने यः प्रलयो ब्रह्मणस्तु ते
तावदध्यासते लोकं परस्य परचिन्तकाः
क्ष्माम्भोऽनलानिलवियन्मनैन्द्रियार्थ
भूतादिभिः परिवृतं प्रतिसञ्जिहीर्षुः
अव्याकृतं विशति यर्हि गुणत्रयात्माकालं
पराख्यमनुभूय परः स्वयम्भूः
एवं परेत्य भगवन्तमनुप्रविष्टाये
योगिनो जितमरुन्मनसो विरागाः
तेनैव साकममृतं पुरुषं पुराणं
ब्रह्म प्रधानमुपयान्त्यगताभिमानाः
अथ तं सर्वभूतानां हृत्पद्मेषु कृतालयम्‌
श्रुतानुभावं शरणं व्रज भावेन भामिनि
आद्यः स्थिरचराणां यो वेदगर्भः सहर्षिभिः
योगेश्वरैः कुमाराद्यैः सिद्धैर्योगप्रवर्तकैः
भेददृष्ट्याभिमानेन निःसङ्गेनापि कर्मणा
कर्तृत्वात्सगुणं ब्रह्म पुरुषं पुरुषर्षभम्‌
स संसृत्य पुनः काले कालेनेश्वरमूर्तिना
जाते गुणव्यतिकरे यथापूर्वं प्रजायते
ऐश्वर्यं पारमेष्ठ्यं च तेऽपि धर्मविनिर्मितम्‌
निषेव्य पुनरायान्ति गुणव्यतिकरे सति
ये त्विहासक्तमनसः कर्मसु श्रद्धयान्विताः
कुर्वन्त्यप्रतिषिद्धानि नित्यान्यपि च कृत्स्नशः
रजसा कुण्ठमनसः कामात्मानोऽजितेन्द्रियाः
पित्न्यजन्त्यनुदिनं गृहेष्वभिरताशयाः
त्रैवर्गिकास्ते पुरुषा विमुखा हरिमेधसः
कथायां कथनीयोरु विक्रमस्य मधुद्विषः
नूनं दैवेन विहता ये चाच्युतकथासुधाम्‌
हित्वा शृण्वन्त्यसद्गाथाः पुरीषमिव विड्भुजः
दक्षिणेन पथार्यम्णः पितृलोकं व्रजन्ति ते
प्रजामनु प्रजायन्ते श्मशानान्तक्रियाकृतः
ततस्ते क्षीणसुकृताः पुनर्लोकमिमं सति
पतन्ति विवशा देवैः सद्यो विभ्रंशितोदयाः
तस्मात्त्वं सर्वभावेन भजस्व परमेष्ठिनम्‌
तद्गुणाश्रयया भक्त्या भजनीयपदाम्बुजम्‌
वासुदेवे भगवति भक्तियोगः प्रयोजितः
जनयत्याशु वैराग्यं ज्ञानं यद्ब्रह्मदर्शनम्‌
यदास्य चित्तमर्थेषु समेष्विन्द्रियवृत्तिभिः
न विगृह्णाति वैषम्यं प्रियमप्रियमित्युत
स तदैवात्मनात्मानं निःसङ्गं समदर्शनम्‌
हेयोपादेयरहितमारूढं पदमीक्षते
ज्ञानमात्रं परं ब्रह्म परमात्मेश्वरः पुमान्‌
दृश्यादिभिः पृथग्भावैर्भगवानेक ईयते
एतावानेव योगेन समग्रेणेह योगिनः
युज्यतेऽभिमतो ह्यर्थो यदसङ्गस्तु कृत्स्नशः
ज्ञानमेकं पराचीनैरिन्द्रियैर्ब्रह्म निर्गुणम्‌
अवभात्यर्थरूपेण भ्रान्त्या शब्दादिधर्मिणा
यथा महानहंरूपस्त्रिवृत्पञ्चविधः स्वराट्‌
एकादशविधस्तस्य वपुरण्डं जगद्यतः
एतद्वै श्रद्धया भक्त्या योगाभ्यासेन नित्यशः
समाहितात्मा निःसङ्गो विरक्त्या परिपश्यति
इत्येतत्कथितं गुर्वि ज्ञानं तद्ब्रह्मदर्शनम्‌
येनानुबुद्ध्यते तत्त्वं प्रकृतेः पुरुषस्य च
ज्ञानयोगश्च मन्निष्ठो नैर्गुण्यो भक्तिलक्षणः
द्वयोरप्येक एवार्थो भगवच्छब्दलक्षणः
यथेन्द्रियैः पृथग्द्वारैरर्थो बहुगुणाश्रयः
एको नानेयते तद्वद्भगवान्शास्त्रवर्त्मभिः
क्रियया क्रतुभिर्दानैस्तपःस्वाध्यायमर्शनैः
आत्मेन्द्रियजयेनापि सन्न्यासेन च कर्मणाम्‌
योगेन विविधाङ्गेन भक्तियोगेन चैव हि
धर्मेणोभयचिह्नेन यः प्रवृत्तिनिवृत्तिमान्‌
आत्मतत्त्वावबोधेन वैराग्येण दृढेन च
ईयते भगवानेभिः सगुणो निर्गुणः स्वदृक्‌
प्रावोचं भक्तियोगस्य स्वरूपं ते चतुर्विधम्‌
कालस्य चाव्यक्तगतेर्योऽन्तर्धावति जन्तुषु
जीवस्य संसृतीर्बह्वीरविद्याकर्मनिर्मिताः
यास्वङ्ग प्रविशन्नात्मा न वेद गतिमात्मनः
नैतत्खलायोपदिशेन्नाविनीताय कर्हिचित्‌
न स्तब्धाय न भिन्नाय नैव धर्मध्वजाय च
न लोलुपायोपदिशेन्न गृहारूढचेतसे
नाभक्ताय च मे जातु न मद्भक्तद्विषामपि
श्रद्दधानाय भक्ताय विनीतायानसूयवे
भूतेषु कृतमैत्राय शुश्रूषाभिरताय च
बहिर्जातविरागाय शान्तचित्ताय दीयताम्‌
निर्मत्सराय शुचये यस्याहं प्रेयसां प्रियः
य इदं शृणुयादम्ब श्रद्धया पुरुषः सकृत्‌
यो वाभिधत्ते मच्चित्तः स ह्येति पदवीं च मे
मैत्रेय उवाच
एवं निशम्य कपिलस्य वचो जनित्रीसा कर्दमस्य दयिता किल देवहूतिः
विस्रस्तमोहपटला तमभिप्रणम्यतुष्टाव तत्त्वविषयाङ्कितसिद्धिभूमिम्‌
देवहूतिरुवाच
अथाप्यजोऽन्तःसलिले शयानं भूतेन्द्रियार्थात्ममयं वपुस्ते
गुणप्रवाहं सदशेषबीजं दध्यौ स्वयं यज्जठराब्जजातः
स एव विश्वस्य भवान्विधत्ते गुणप्रवाहेण विभक्तवीर्यः
सर्गाद्यनीहोऽवितथाभिसन्धिरात्मेश्वरोऽतर्क्यसहस्रशक्तिः
स त्वं भृतो मे जठरेण नाथ कथं नु यस्योदर एतदासीत्‌
विश्वं युगान्ते वटपत्र एकः शेते स्म मायाशिशुरङ्घ्रिपानः
त्वं देहतन्त्रः प्रशमाय पाप्मनां निदेशभाजां च विभो विभूतये
यथावतारास्तव सूकरादयस्तथायमप्यात्मपथोपलब्धये
यन्नामधेयश्रवणानुकीर्तनाद्यत्प्रह्वणाद्यत्स्मरणादपि क्वचित्‌
श्वादोऽपि सद्यः सवनाय कल्पते कुतः पुनस्ते भगवन्नु दर्शनात्‌
अहो बत श्वपचोऽतो गरीयान्यज्जिह्वाग्रे वर्तते नाम तुभ्यम्‌
तेपुस्तपस्ते जुहुवुः सस्नुरार्या ब्रह्मानूचुर्नाम गृणन्ति ये ते
तं त्वामहं ब्रह्म परं पुमांसं प्रत्यक्स्रोतस्यात्मनि संविभाव्यम्‌
स्वतेजसा ध्वस्तगुणप्रवाहं वन्दे विष्णुं कपिलं वेदगर्भम्‌
मैत्रेय उवाच
ईडितो भगवानेवं कपिलाख्यः परः पुमान्‌
वाचाविक्लवयेत्याह मातरं मातृवत्सलः
कपिल उवाच
मार्गेणानेन मातस्ते सुसेव्येनोदितेन मे
आस्थितेन परां काष्ठामचिरादवरोत्स्यसि
श्रद्धत्स्वैतन्मतं मह्यं जुष्टं यद्ब्रह्मवादिभिः
येन मामभयं याया मृत्युमृच्छन्त्यतद्विदः
मैत्रेय उवाच
इति प्रदर्श्य भगवान्सतीं तामात्मनो गतिम्‌
स्वमात्रा ब्रह्मवादिन्या कपिलोऽनुमतो ययौ
सा चापि तनयोक्तेन योगादेशेन योगयुक्‌
तस्मिन्नाश्रम आपीडे सरस्वत्याः समाहिता
अभीक्ष्णावगाहकपिशान्जटिलान्कुटिलालकान्‌
आत्मानं चोग्रतपसा बिभ्रती चीरिणं कृशम्‌
प्रजापतेः कर्दमस्य तपोयोगविजृम्भितम्‌
स्वगार्हस्थ्यमनौपम्यं प्रार्थ्यं वैमानिकैरपि
पयःफेननिभाः शय्या दान्ता रुक्मपरिच्छदाः
आसनानि च हैमानि सुस्पर्शास्तरणानि च
स्वच्छस्फटिककुड्येषु महामारकतेषु च
रत्नप्रदीपा आभान्ति ललना रत्नसंयुताः
गृहोद्यानं कुसुमितै रम्यं बह्वमरद्रुमैः
कूजद्विहङ्गमिथुनं गायन्मत्तमधुव्रतम्‌
यत्र प्रविष्टमात्मानं विबुधानुचरा जगुः
वाप्यामुत्पलगन्धिन्यां कर्दमेनोपलालितम्‌
हित्वा तदीप्सिततममप्याखण्डलयोषिताम्‌
किञ्चिच्चकार वदनं पुत्रविश्लेषणातुरा
वनं प्रव्रजिते पत्यावपत्यविरहातुरा
ज्ञाततत्त्वाप्यभून्नष्टे वत्से गौरिव वत्सला
तमेव ध्यायती देवमपत्यं कपिलं हरिम्‌
बभूवाचिरतो वत्स निःस्पृहा तादृशे गृहे
ध्यायती भगवद्रूपं यदाह ध्यानगोचरम्‌
सुतः प्रसन्नवदनं समस्तव्यस्तचिन्तया
भक्तिप्रवाहयोगेन वैराग्येण बलीयसा
युक्तानुष्ठानजातेन ज्ञानेन ब्रह्महेतुना
विशुद्धेन तदात्मानमात्मना विश्वतोमुखम्‌
स्वानुभूत्या तिरोभूत मायागुणविशेषणम्‌
ब्रह्मण्यवस्थितमतिर्भगवत्यात्मसंश्रये
निवृत्तजीवापत्तित्वात्क्षीणक्लेशाप्तनिर्वृतिः
नित्यारूढसमाधित्वात्परावृत्तगुणभ्रमा
न सस्मार तदात्मानं स्वप्ने दृष्टमिवोत्थितः
तद्देहः परतः पोषोऽप्यकृशश्चाध्यसम्भवात्‌
बभौ मलैरवच्छन्नः सधूम इव पावकः
स्वाङ्गं तपोयोगमयं मुक्तकेशं गताम्बरम्‌
दैवगुप्तं न बुबुधे वासुदेवप्रविष्टधीः
एवं सा कपिलोक्तेन मार्गेणाचिरतः परम्‌
आत्मानं ब्रह्मनिर्वाणं भगवन्तमवाप ह
तद्वीरासीत्पुण्यतमं क्षेत्रं त्रैलोक्यविश्रुतम्‌
नाम्ना सिद्धपदं यत्र सा संसिद्धिमुपेयुषी
तस्यास्तद्योगविधुत मार्त्यं मर्त्यमभूत्सरित्‌
स्रोतसां प्रवरा सौम्य सिद्धिदा सिद्धसेविता
कपिलोऽपि महायोगी भगवान्पितुराश्रमात्‌
मातरं समनुज्ञाप्य प्रागुदीचीं दिशं ययौ
सिद्धचारणगन्धर्वैर्मुनिभिश्चाप्सरोगणैः
स्तूयमानः समुद्रेण दत्तार्हणनिकेतनः
आस्ते योगं समास्थाय साङ्ख्याचार्यैरभिष्टुतः
त्रयाणामपि लोकानामुपशान्त्यै समाहितः
एतन्निगदितं तात यत्पृष्टोऽहं तवानघ
कपिलस्य च संवादो देवहूत्याश्च पावनः
य इदमनुशृणोति योऽभिधत्ते कपिलमुनेर्मतमात्मयोगगुह्यम्‌
भगवति कृतधीः सुपर्णकेतावुपलभते भगवत्पदारविन्दम्‌
मैत्रेय उवाच
मनोस्तु शतरूपायां तिस्रः कन्याश्च जज्ञिरे
आकूतिर्देवहूतिश्च प्रसूतिरिति विश्रुताः
आकूतिं रुचये प्रादादपि भ्रातृमतीं नृपः
पुत्रिकाधर्ममाश्रित्य शतरूपानुमोदितः
प्रजापतिः स भगवान्रुचिस्तस्यामजीजनत्‌
मिथुनं ब्रह्मवर्चस्वी परमेण समाधिना
यस्तयोः पुरुषः साक्षाद्विष्णुर्यज्ञस्वरूपधृक्‌
या स्त्री सा दक्षिणा भूतेरंशभूतानपायिनी
आनिन्ये स्वगृहं पुत्र्याः पुत्रं विततरोचिषम्‌
स्वायम्भुवो मुदा युक्तो रुचिर्जग्राह दक्षिणाम्‌
तां कामयानां भगवानुवाह यजुषां पतिः
तुष्टायां तोषमापन्नोऽ जनयद्द्वादशात्मजान्‌
तोषः प्रतोषः सन्तोषो भद्रः शान्तिरिडस्पतिः
इध्मः कविर्विभुः स्वह्नः सुदेवो रोचनो द्विषट्‌
तुषिता नाम ते देवा आसन्स्वायम्भुवान्तरे
मरीचिमिश्रा ऋषयो यज्ञः सुरगणेश्वरः
प्रियव्रतोत्तानपादौ मनुपुत्रौ महौजसौ
तत्पुत्रपौत्रनप्त्ऱ्‌^ईणामनुवृत्तं तदन्तरम्‌
देवहूतिमदात्तात कर्दमायात्मजां मनुः
तत्सम्बन्धि श्रुतप्रायं भवता गदतो मम
दक्षाय ब्रह्मपुत्राय प्रसूतिं भगवान्मनुः
प्रायच्छद्यत्कृतः सर्गस्त्रिलोक्यां विततो महान्‌
याः कर्दमसुताः प्रोक्ता नव ब्रह्मर्षिपत्नयः
तासां प्रसूतिप्रसवं प्रोच्यमानं निबोध मे
पत्नी मरीचेस्तु कला सुषुवे कर्दमात्मजा
कश्यपं पूर्णिमानं च ययोरापूरितं जगत्‌
पूर्णिमासूत विरजं विश्वगं च परन्तप
देवकुल्यां हरेः पाद शौचाद्याभूत्सरिद्दिवः
अत्रेः पत्न्यनसूया त्रीञ्जज्ञे सुयशसः सुतान्‌
दत्तं दुर्वाससं सोममात्मेशब्रह्मसम्भवान्‌
विदुर उवाच
अत्रेर्गृहे सुरश्रेष्ठाः स्थित्युत्पत्त्यन्तहेतवः
किञ्चिच्चिकीर्षवो जाता एतदाख्याहि मे गुरो
मैत्रेय उवाच
ब्रह्मणा चोदितः सृष्टावत्रिर्ब्रह्मविदां वरः
सह पत्न्या ययावृक्षं कुलाद्रिं तपसि स्थितः
तस्मिन्प्रसूनस्तबक पलाशाशोककानने
वार्भिः स्रवद्भिरुद्घुष्टे निर्विन्ध्यायाः समन्ततः
प्राणायामेन संयम्य मनो वर्षशतं मुनिः
अतिष्ठदेकपादेन निर्द्वन्द्वोऽनिलभोजनः
शरणं तं प्रपद्येऽहं य एव जगदीश्वरः
प्रजामात्मसमां मह्यं प्रयच्छत्विति चिन्तयन्‌
तप्यमानं त्रिभुवनं प्राणायामैधसाग्निना
निर्गतेन मुनेर्मूर्ध्नः समीक्ष्य प्रभवस्त्रयः
अप्सरोमुनिगन्धर्व सिद्धविद्याधरोरगैः
वितायमानयशसस्तदाश्रमपदं ययुः
तत्प्रादुर्भावसंयोग विद्योतितमना मुनिः
उत्तिष्ठन्नेकपादेन ददर्श विबुधर्षभान्‌
प्रणम्य दण्डवद्भूमावुपतस्थेऽर्हणाञ्जलिः
वृषहंससुपर्णस्थान्स्वैः स्वैश्चिह्नैश्च चिह्नितान्‌
कृपावलोकेन हसद्‌ वदनेनोपलम्भितान्‌
तद्रोचिषा प्रतिहते निमील्य मुनिरक्षिणी
चेतस्तत्प्रवणं युञ्जन्नस्तावीत्संहताञ्जलिः
श्लक्ष्णया सूक्तया वाचा सर्वलोकगरीयसः
अत्रिरुवाच
विश्वोद्भवस्थितिलयेषु विभज्यमानैर्‌
मायागुणैरनुयुगं विगृहीतदेहाः
ते ब्रह्मविष्णुगिरिशाः प्रणतोऽस्म्यहं वस्‌
तेभ्यः क एव भवतां म इहोपहूतः
एको मयेह भगवान्विविधप्रधानैश्‌
चित्तीकृतः प्रजननाय कथं नु यूयम्‌
अत्रागतास्तनुभृतां मनसोऽपि दूराद्‌
ब्रूत प्रसीदत महानिह विस्मयो मे
मैत्रेय उवाच
इति तस्य वचः श्रुत्वा त्रयस्ते विबुधर्षभाः
प्रत्याहुः श्लक्ष्णया वाचा प्रहस्य तमृषिं प्रभो
देवा ऊचुः
यथा कृतस्ते सङ्कल्पो भाव्यं तेनैव नान्यथा
सत्सङ्कल्पस्य ते ब्रह्मन्यद्वै ध्यायति ते वयम्‌
अथास्मदंशभूतास्ते आत्मजा लोकविश्रुताः
भवितारोऽङ्ग भद्रं ते विस्रप्स्यन्ति च ते यशः
एवं कामवरं दत्त्वा प्रतिजग्मुः सुरेश्वराः
सभाजितास्तयोः सम्यग्दम्पत्योर्मिषतोस्ततः
सोमोऽभूद्ब्रह्मणोऽंशेन दत्तो विष्णोस्तु योगवित्‌
दुर्वासाः शङ्करस्यांशो निबोधाङ्गिरसः प्रजाः
श्रद्धा त्वङ्गिरसः पत्नी चतस्रोऽसूत कन्यकाः
सिनीवाली कुहू राका चतुर्थ्यनुमतिस्तथा
तत्पुत्रावपरावास्तां ख्यातौ स्वारोचिषेऽन्तरे
उतथ्यो भगवान्साक्षाद्ब्रह्मिष्ठश्च बृहस्पतिः
पुलस्त्योऽजनयत्पत्न्यामगस्त्यं च हविर्भुवि
सोऽन्यजन्मनि दह्राग्निर्विश्रवाश्च महातपाः
तस्य यक्षपतिर्देवः कुबेरस्त्विडविडासुतः
रावणः कुम्भकर्णश्च तथान्यस्यां विभीषणः
पुलहस्य गतिर्भार्या त्रीनसूत सती सुतान्‌
कर्मश्रेष्ठं वरीयांसं सहिष्णुं च महामते
क्रतोरपि क्रिया भार्या वालखिल्यानसूयत
ऋषीन्षष्टिसहस्राणि ज्वलतो ब्रह्मतेजसा
ऊर्जायां जज्ञिरे पुत्रा वसिष्ठस्य परन्तप
चित्रकेतुप्रधानास्ते सप्त ब्रह्मर्षयोऽमलाः
चित्रकेतुः सुरोचिश्च विरजा मित्र एव च
उल्बणो वसुभृद्यानो द्युमान्शक्त्यादयोऽपरे
चित्तिस्त्वथर्वणः पत्नी लेभे पुत्रं धृतव्रतम्‌
दध्यञ्चमश्वशिरसं भृगोर्वंशं निबोध मे
भृगुः ख्यात्यां महाभागः पत्न्यां पुत्रानजीजनत्‌
धातारं च विधातारं श्रियं च भगवत्पराम्‌
आयतिं नियतिं चैव सुते मेरुस्तयोरदात्‌
ताभ्यां तयोरभवतां मृकण्डः प्राण एव च
मार्कण्डेयो मृकण्डस्य प्राणाद्वेदशिरा मुनिः
कविश्च भार्गवो यस्य भगवानुशना सुतः
त एते मुनयः क्षत्तर्लोकान्सर्गैरभावयन्‌
एष कर्दमदौहित्र सन्तानः कथितस्तव
शृण्वतः श्रद्दधानस्य सद्यः पापहरः परः
प्रसूतिं मानवीं दक्ष उपयेमे ह्यजात्मजः
तस्यां ससर्ज दुहित्ऱ्‌^ईः षोडशामललोचनाः
त्रयोदशादाद्धर्माय तथैकामग्नये विभुः
पितृभ्य एकां युक्तेभ्यो भवायैकां भवच्छिदे
श्रद्धा मैत्री दया शान्तिस्तुष्टिः पुष्टिः क्रियोन्नतिः
बुद्धिर्मेधा तितिक्षा ह्रीर्मूर्तिर्धर्मस्य पत्नयः
श्रद्धासूत शुभं मैत्री प्रसादमभयं दया
शान्तिः सुखं मुदं तुष्टिः स्मयं पुष्टिरसूयत
योगं क्रियोन्नतिर्दर्पमर्थं बुद्धिरसूयत
मेधा स्मृतिं तितिक्षा तु क्षेमं ह्रीः प्रश्रयं सुतम्‌
मूर्तिः सर्वगुणोत्पत्तिर्नरनारायणावृषी
ययोर्जन्मन्यदो विश्वमभ्यनन्दत्सुनिर्वृतम्‌
मनांसि ककुभो वाताः प्रसेदुः सरितोऽद्रयः
दिव्यवाद्यन्त तूर्याणि पेतुः कुसुमवृष्टयः
मुनयस्तुष्टुवुस्तुष्टा जगुर्गन्धर्वकिन्नराः
नृत्यन्ति स्म स्त्रियो देव्य आसीत्परममङ्गलम्‌
देवा ब्रह्मादयः सर्वे उपतस्थुरभिष्टवैः
देवा ऊचुः
यो मायया विरचितं निजयात्मनीदं
खे रूपभेदमिव तत्प्रतिचक्षणाय
एतेन धर्मसदने ऋषिमूर्तिनाद्य
प्रादुश्चकार पुरुषाय नमः परस्मै
सोऽयं स्थितिव्यतिकरोपशमाय सृष्टान्‌
सत्त्वेन नः सुरगणाननुमेयतत्त्वः
दृश्याददभ्रकरुणेन विलोकनेन
यच्छ्रीनिकेतममलं क्षिपतारविन्दम्‌
एवं सुरगणैस्तात भगवन्तावभिष्टुतौ
लब्धावलोकैर्ययतुरर्चितौ गन्धमादनम्‌
ताविमौ वै भगवतो हरेरंशाविहागतौ
भारव्ययाय च भुवः कृष्णौ यदुकुरूद्वहौ
स्वाहाभिमानिनश्चाग्नेरात्मजांस्त्रीनजीजनत्‌
पावकं पवमानं च शुचिं च हुतभोजनम्‌
तेभ्योऽग्नयः समभवन्चत्वारिंशच्च पञ्च च
त एवैकोनपञ्चाशत्साकं पितृपितामहैः
वैतानिके कर्मणि यन्‌ नामभिर्ब्रह्मवादिभिः
आग्नेय्य इष्टयो यज्ञे निरूप्यन्तेऽग्नयस्तु ते
अग्निष्वात्ता बर्हिषदः सौम्याः पितर आज्यपाः
साग्नयोऽनग्नयस्तेषां पत्नी दाक्षायणी स्वधा
तेभ्यो दधार कन्ये द्वे वयुनां धारिणीं स्वधा
उभे ते ब्रह्मवादिन्यौ ज्ञानविज्ञानपारगे
भवस्य पत्नी तु सती भवं देवमनुव्रता
आत्मनः सदृशं पुत्रं न लेभे गुणशीलतः
पितर्यप्रतिरूपे स्वे भवायानागसे रुषा
अप्रौढैवात्मनात्मानमजहाद्योगसंयुता
विदुर उवाच
भवे शीलवतां श्रेष्ठे दक्षो दुहितृवत्सलः
विद्वेषमकरोत्कस्मादनादृत्यात्मजां सतीम्‌
कस्तं चराचरगुरुं निर्वैरं शान्तविग्रहम्‌
आत्मारामं कथं द्वेष्टि जगतो दैवतं महत्‌
एतदाख्याहि मे ब्रह्मन्जामातुः श्वशुरस्य च
विद्वेषस्तु यतः प्राणांस्तत्यजे दुस्त्यजान्सती
मैत्रेय उवाच
पुरा विश्वसृजां सत्रे समेताः परमर्षयः
तथामरगणाः सर्वे सानुगा मुनयोऽग्नयः
तत्र प्रविष्टमृषयो दृष्ट्वार्कमिव रोचिषा
भ्राजमानं वितिमिरं कुर्वन्तं तन्महत्सदः
उदतिष्ठन्सदस्यास्ते स्वधिष्ण्येभ्यः सहाग्नयः
ऋते विरिञ्चां शर्वं च तद्भासाक्षिप्तचेतसः
सदसस्पतिभिर्दक्षो भगवान्साधु सत्कृतः
अजं लोकगुरुं नत्वा निषसाद तदाज्ञया
प्राङ्निषण्णं मृडं दृष्ट्वा नामृष्यत्तदनादृतः
उवाच वामं चक्षुर्भ्यामभिवीक्ष्य दहन्निव
श्रूयतां ब्रह्मर्षयो मे सहदेवाः सहाग्नयः
साधूनां ब्रुवतो वृत्तं नाज्ञानान्न च मत्सरात्‌
अयं तु लोकपालानां यशोघ्नो निरपत्रपः
सद्भिराचरितः पन्था येन स्तब्धेन दूषितः
एष मे शिष्यतां प्राप्तो यन्मे दुहितुरग्रहीत्‌
पाणिं विप्राग्निमुखतः सावित्र्या इव साधुवत्‌
गृहीत्वा मृगशावाक्ष्याः पाणिं मर्कटलोचनः
प्रत्युत्थानाभिवादार्हे वाचाप्यकृत नोचितम्‌
लुप्तक्रियायाशुचये मानिने भिन्नसेतवे
अनिच्छन्नप्यदां बालां शूद्रायेवोशतीं गिरम्‌
प्रेतावासेषु घोरेषु प्रेतैर्भूतगणैर्वृतः
अटत्युन्मत्तवन्नग्नो व्युप्तकेशो हसन्रुदन्‌
चिताभस्मकृतस्नानः प्रेतस्रङ्न्रस्थिभूषणः
शिवापदेशो ह्यशिवो मत्तो मत्तजनप्रियः
पतिः प्रमथनाथानां तमोमात्रात्मकात्मनाम्‌
तस्मा उन्मादनाथाय नष्टशौचाय दुर्हृदे
दत्ता बत मया साध्वी चोदिते परमेष्ठिना
मैत्रेय उवाच
विनिन्द्यैवं स गिरिशमप्रतीपमवस्थितम्‌
दक्षोऽथाप उपस्पृश्य क्रुद्धः शप्तुं प्रचक्रमे
अयं तु देवयजन इन्द्रोपेन्द्रादिभिर्भवः
सह भागं न लभतां देवैर्देवगणाधमः
निषिध्यमानः स सदस्यमुख्यैर्दक्षो गिरित्राय विसृज्य शापम्‌
तस्माद्विनिष्क्रम्य विवृद्धमन्युर्जगाम कौरव्य निजं निकेतनम्‌
विज्ञाय शापं गिरिशानुगाग्रणीर्नन्दीश्वरो रोषकषायदूषितः
दक्षाय शापं विससर्ज दारुणं ये चान्वमोदंस्तदवाच्यतां द्विजाः
य एतन्मर्त्यमुद्दिश्य भगवत्यप्रतिद्रुहि
द्रुह्यत्यज्ञः पृथग्दृष्टिस्तत्त्वतो विमुखो भवेत्‌
गृहेषु कूटधर्मेषु सक्तो ग्राम्यसुखेच्छया
कर्मतन्त्रं वितनुते वेदवादविपन्नधीः
बुद्ध्या पराभिध्यायिन्या विस्मृतात्मगतिः पशुः
स्त्रीकामः सोऽस्त्वतितरां दक्षो बस्तमुखोऽचिरात्‌
विद्याबुद्धिरविद्यायां कर्ममय्यामसौ जडः
संसरन्त्विह ये चामुमनु शर्वावमानिनम्‌
गिरः श्रुतायाः पुष्पिण्या मधुगन्धेन भूरिणा
मथ्ना चोन्मथितात्मानः सम्मुह्यन्तु हरद्विषः
सर्वभक्षा द्विजा वृत्त्यै धृतविद्यातपोव्रताः
वित्तदेहेन्द्रियारामा याचका विचरन्त्विह
तस्यैवं वदतः शापं श्रुत्वा द्विजकुलाय वै
भृगुः प्रत्यसृजच्छापं ब्रह्मदण्डं दुरत्ययम्‌
भवव्रतधरा ये च ये च तान्समनुव्रताः
पाषण्डिनस्ते भवन्तु सच्छास्त्रपरिपन्थिनः
नष्टशौचा मूढधियो जटाभस्मास्थिधारिणः
विशन्तु शिवदीक्षायां यत्र दैवं सुरासवम्‌
ब्रह्म च ब्राह्मणांश्चैव यद्यूयं परिनिन्दथ
सेतुं विधारणं पुंसामतः पाषण्डमाश्रिताः
एष एव हि लोकानां शिवः पन्थाः सनातनः
यं पूर्वे चानुसन्तस्थुर्यत्प्रमाणं जनार्दनः
तद्ब्रह्म परमं शुद्धं सतां वर्त्म सनातनम्‌
विगर्ह्य यात पाषण्डं दैवं वो यत्र भूतराट्‌
मैत्रेय उवाच
तस्यैवं वदतः शापं भृगोः स भगवान्भवः
निश्चक्राम ततः किञ्चिद्विमना इव सानुगः
तेऽपि विश्वसृजः सत्रं सहस्रपरिवत्सरान्‌
संविधाय महेष्वास यत्रेज्य ऋषभो हरिः
आप्लुत्यावभृथं यत्र गङ्गा यमुनयान्विता
विरजेनात्मना सर्वे स्वं स्वं धाम ययुस्ततः
मैत्रेय उवाच
सदा विद्विषतोरेवं कालो वै ध्रियमाणयोः
जामातुः श्वशुरस्यापि सुमहानतिचक्रमे
यदाभिषिक्तो दक्षस्तु ब्रह्मणा परमेष्ठिना
प्रजापतीनां सर्वेषामाधिपत्ये स्मयोऽभवत्‌
इष्ट्वा स वाजपेयेन ब्रह्मिष्ठानभिभूय च
बृहस्पतिसवं नाम समारेभे क्रतूत्तमम्‌
तस्मिन्ब्रह्मर्षयः सर्वे देवर्षिपितृदेवताः
आसन्कृतस्वस्त्ययनास्तत्पत्न्यश्च सभर्तृकाः
तदुपश्रुत्य नभसि खेचराणां प्रजल्पताम्‌
सती दाक्षायणी देवी पितृयज्ञमहोत्सवम्‌
व्रजन्तीः सर्वतो दिग्भ्य उपदेववरस्त्रियः
विमानयानाः सप्रेष्ठा निष्ककण्ठीः सुवाससः
दृष्ट्वा स्वनिलयाभ्याशे लोलाक्षीर्मृष्टकुण्डलाः
पतिं भूतपतिं देवमौत्सुक्यादभ्यभाषत
सत्युवाच
प्रजापतेस्ते श्वशुरस्य साम्प्रतं निर्यापितो यज्ञमहोत्सवः किल
वयं च तत्राभिसराम वाम ते यद्यर्थितामी विबुधा व्रजन्ति हि
तस्मिन्भगिन्यो मम भर्तृभिः स्वकैर्ध्रुवं गमिष्यन्ति सुहृद्दिदृक्षवः
अहं च तस्मिन्भवताभिकामये सहोपनीतं परिबर्हमर्हितुम्‌
तत्र स्वस्ऱ्‌^ईर्मे ननु भर्तृसम्मिता मातृष्वस्ऱ्‌^ईः क्लिन्नधियं च मातरम्‌
द्रक्ष्ये चिरोत्कण्ठमना महर्षिभिरुन्नीयमानं च मृडाध्वरध्वजम्‌
त्वय्येतदाश्चर्यमजात्ममायया विनिर्मितं भाति गुणत्रयात्मकम्‌
तथाप्यहं योषिदतत्त्वविच्च ते दीना दिदृक्षे भव मे भवक्षितिम्‌
पश्य प्रयान्तीरभवान्ययोषितोऽप्यलङ्कृताः कान्तसखा वरूथशः
यासां व्रजद्भिः शितिकण्ठ मण्डितं नभो विमानैः कलहंसपाण्डुभिः
कथं सुतायाः पितृगेहकौतुकं निशम्य देहः सुरवर्य नेङ्गते
अनाहुता अप्यभियन्ति सौहृदं भर्तुर्गुरोर्देहकृतश्च केतनम्‌
तन्मे प्रसीदेदममर्त्य वाञ्छितं कर्तुं भवान्कारुणिको बतार्हति
त्वयात्मनोऽर्धेऽहमदभ्रचक्षुषा निरूपिता मानुगृहाण याचितः
ऋषिरुवाच
एवं गिरित्रः प्रिययाभिभाषितः प्रत्यभ्यधत्त प्रहसन्सुहृत्प्रियः
संस्मारितो मर्मभिदः कुवागिषून्यानाह को विश्वसृजां समक्षतः
श्रीभगवानुवाच
त्वयोदितं शोभनमेव शोभने अनाहुता अप्यभियन्ति बन्धुषु
ते यद्यनुत्पादितदोषदृष्टयो बलीयसानात्म्यमदेन मन्युना
विद्यातपोवित्तवपुर्वयःकुलैः सतां गुणैः षड्भिरसत्तमेतरैः
स्मृतौ हतायां भृतमानदुर्दृशः स्तब्धा न पश्यन्ति हि धाम भूयसाम्‌
नैतादृशानां स्वजनव्यपेक्षया गृहान्प्रतीयादनवस्थितात्मनाम्‌
येऽभ्यागतान्वक्रधियाभिचक्षते आरोपितभ्रूभिरमर्षणाक्षिभिः
तथारिभिर्न व्यथते शिलीमुखैः शेतेऽर्दिताङ्गो हृदयेन दूयता
स्वानां यथा वक्रधियां दुरुक्तिभिर्दिवानिशं तप्यति मर्मताडितः
व्यक्तं त्वमुत्कृष्टगतेः प्रजापतेः प्रियात्मजानामसि सुभ्रु मे मता
तथापि मानं न पितुः प्रपत्स्यसे मदाश्रयात्कः परितप्यते यतः
पापच्यमानेन हृदातुरेन्द्रियः समृद्धिभिः पूरुषबुद्धिसाक्षिणाम्‌
अकल्प एषामधिरोढुमञ्जसा परं पदं द्वेष्टि यथासुरा हरिम्‌
प्रत्युद्गमप्रश्रयणाभिवादनं विधीयते साधु मिथः सुमध्यमे
प्राज्ञैः परस्मै पुरुषाय चेतसा गुहाशयायैव न देहमानिने
सत्त्वं विशुद्धं वसुदेवशब्दितं यदीयते तत्र पुमानपावृतः
सत्त्वे च तस्मिन्भगवान्वासुदेवो ह्यधोक्षजो मे नमसा विधीयते
तत्ते निरीक्ष्यो न पितापि देहकृद्दक्षो मम द्विट्तदनुव्रताश्च ये
यो विश्वसृग्यज्ञगतं वरोरु मामनागसं दुर्वचसाकरोत्तिरः
यदि व्रजिष्यस्यतिहाय मद्वचो भद्रं भवत्या न ततो भविष्यति
सम्भावितस्य स्वजनात्पराभवो यदा स सद्यो मरणाय कल्पते
मैत्रेय उवाच
एतावदुक्त्वा विरराम शङ्करः पत्न्यङ्गनाशं ह्युभयत्र चिन्तयन्‌
सुहृद्दिदृक्षुः परिशङ्किता भवान्निष्क्रामती निर्विशती द्विधास सा
सुहृद्दिदृक्षाप्रतिघातदुर्मनाः स्नेहाद्रुदत्यश्रुकलातिविह्वला
भवं भवान्यप्रतिपूरुषं रुषा प्रधक्ष्यतीवैक्षत जातवेपथुः
ततो विनिःश्वस्य सती विहाय तं शोकेन रोषेण च दूयता हृदा
पित्रोरगात्स्त्रैणविमूढधीर्गृहान्प्रेम्णात्मनो योऽर्धमदात्सतां प्रियः
तामन्वगच्छन्द्रुतविक्रमां सतीमेकां त्रिनेत्रानुचराः सहस्रशः
सपार्षदयक्षा मणिमन्मदादयः पुरोवृषेन्द्रास्तरसा गतव्यथाः
तां सारिकाकन्दुकदर्पणाम्बुज श्वेतातपत्रव्यजनस्रगादिभिः
गीतायनैर्दुन्दुभिशङ्खवेणुभिर्वृषेन्द्रमारोप्य विटङ्किता ययुः
आब्रह्मघोषोर्जितयज्ञवैशसं विप्रर्षिजुष्टं विबुधैश्च सर्वशः
मृद्दार्वयःकाञ्चनदर्भचर्मभिर्निसृष्टभाण्डं यजनं समाविशत्‌
तामागतां तत्र न कश्चनाद्रियद्विमानितां यज्ञकृतो भयाज्जनः
ऋते स्वस्ऱ्‌^ईर्वै जननीं च सादराः प्रेमाश्रुकण्ठ्यः परिषस्वजुर्मुदा
सौदर्यसम्प्रश्नसमर्थवार्तया मात्रा च मातृष्वसृभिश्च सादरम्‌
दत्तां सपर्यां वरमासनं च सा नादत्त पित्राप्रतिनन्दिता सती
अरुद्रभागं तमवेक्ष्य चाध्वरं पित्रा च देवे कृतहेलनं विभौ
अनादृता यज्ञसदस्यधीश्वरी चुकोप लोकानिव धक्ष्यती रुषा
जगर्ह सामर्षविपन्नया गिरा शिवद्विषं धूमपथश्रमस्मयम्‌
स्वतेजसा भूतगणान्समुत्थितान्निगृह्य देवी जगतोऽभिशृण्वतः
देव्युवाच
न यस्य लोकेऽस्त्यतिशायनः प्रियस्तथाप्रियो देहभृतां प्रियात्मनः
तस्मिन्समस्तात्मनि मुक्तवैरके ऋते भवन्तं कतमः प्रतीपयेत्‌
दोषान्परेषां हि गुणेषु साधवो गृह्णन्ति केचिन्न भवादृशो द्विज
गुणांश्च फल्गून्बहुलीकरिष्णवो महत्तमास्तेष्वविदद्भवानघम्‌
नाश्चर्यमेतद्यदसत्सु सर्वदा महद्विनिन्दा कुणपात्मवादिषु
सेर्ष्यं महापूरुषपादपांसुभिर्निरस्ततेजःसु तदेव शोभनम्‌
यद्द्व्यक्षरं नाम गिरेरितं नृणां सकृत्प्रसङ्गादघमाशु हन्ति तत्‌
पवित्रकीर्तिं तमलङ्घ्यशासनं भवानहो द्वेष्टि शिवं शिवेतरः
यत्पादपद्मं महतां मनोऽलिभिर्निषेवितं ब्रह्मरसासवार्थिभिः
लोकस्य यद्वर्षति चाशिषोऽर्थिनस्तस्मै भवान्द्रुह्यति विश्वबन्धवे
किं वा शिवाख्यमशिवं न विदुस्त्वदन्ये ब्रह्मादयस्तमवकीर्य जटाः श्मशाने
तन्माल्यभस्मनृकपाल्यवसत्पिशाचैर्ये मूर्धभिर्दधति तच्चरणावसृष्टम्‌
कर्णौ पिधाय निरयाद्यदकल्प ईशे धर्मावितर्यसृणिभिर्नृभिरस्यमाने
छिन्द्यात्प्रसह्य रुशतीमसतीं प्रभुश्चेज्जिह्वामसूनपि ततो विसृजेत्स धर्मः
अतस्तवोत्पन्नमिदं कलेवरं न धारयिष्ये शितिकण्ठगर्हिणः
जग्धस्य मोहाद्धि विशुद्धिमन्धसो जुगुप्सितस्योद्धरणं प्रचक्षते
न वेदवादाननुवर्तते मतिः स्व एव लोके रमतो महामुनेः
यथा गतिर्देवमनुष्ययोः पृथक्स्व एव धर्मे न परं क्षिपेत्स्थितः
कर्म प्रवृत्तं च निवृत्तमप्यृतं वेदे विविच्योभयलिङ्गमाश्रितम्‌
विरोधि तद्यौगपदैककर्तरि द्वयं तथा ब्रह्मणि कर्म नर्च्छति
मा वः पदव्यः पितरस्मदास्थिता या यज्ञशालासु न धूमवर्त्मभिः
तदन्नतृप्तैरसुभृद्भिरीडिता अव्यक्तलिङ्गा अवधूतसेविताः
नैतेन देहेन हरे कृतागसो देहोद्भवेनालमलं कुजन्मना
व्रीडा ममाभूत्कुजनप्रसङ्गतस्तज्जन्म धिग्यो महतामवद्यकृत्‌
गोत्रं त्वदीयं भगवान्वृषध्वजो दाक्षायणीत्याह यदा सुदुर्मनाः
व्यपेतनर्मस्मितमाशु तदाहं व्युत्स्रक्ष्य एतत्कुणपं त्वदङ्गजम्‌
मैत्रेय उवाच
इत्यध्वरे दक्षमनूद्य शत्रुहन्क्षितावुदीचीं निषसाद शान्तवाक्‌
स्पृष्ट्वा जलं पीतदुकूलसंवृता निमील्य दृग्योगपथं समाविशत्‌
कृत्वा समानावनिलौ जितासना सोदानमुत्थाप्य च नाभिचक्रतः
शनैर्हृदि स्थाप्य धियोरसि स्थितं कण्ठाद्भ्रुवोर्मध्यमनिन्दितानयत्‌
एवं स्वदेहं महतां महीयसा मुहुः समारोपितमङ्कमादरात्‌
जिहासती दक्षरुषा मनस्विनी दधार गात्रेष्वनिलाग्निधारणाम्‌
ततः स्वभर्तुश्चरणाम्बुजासवं जगद्गुरोश्चिन्तयती न चापरम्‌
ददर्श देहो हतकल्मषः सती सद्यः प्रजज्वाल समाधिजाग्निना
तत्पश्यतां खे भुवि चाद्भुतं महधा हेति वादः सुमहानजायत
हन्त प्रिया दैवतमस्य देवी जहावसून्केन सती प्रकोपिता
अहो अनात्म्यं महदस्य पश्यत प्रजापतेर्यस्य चराचरं प्रजाः
जहावसून्यद्विमतात्मजा सती मनस्विनी मानमभीक्ष्णमर्हति
सोऽयं दुर्मर्षहृदयो ब्रह्मध्रुक्च लोकेऽपकीर्तिं महतीमवाप्स्यति
यदङ्गजां स्वां पुरुषद्विडुद्यतां न प्रत्यषेधन्मृतयेऽपराधतः
वदत्येवं जने सत्या दृष्ट्वासुत्यागमद्भुतम्‌
दक्षं तत्पार्षदा हन्तुमुदतिष्ठन्नुदायुधाः
तेषामापततां वेगं निशाम्य भगवान्भृगुः
यज्ञघ्नघ्नेन यजुषा दक्षिणाग्नौ जुहाव ह
अध्वर्युणा हूयमाने देवा उत्पेतुरोजसा
ऋभवो नाम तपसा सोमं प्राप्ताः सहस्रशः
तैरलातायुधैः सर्वे प्रमथाः सहगुह्यकाः
हन्यमाना दिशो भेजुरुशद्भिर्ब्रह्मतेजसा
मैत्रेय उवाच
भवो भवान्या निधनं प्रजापतेरसत्कृताया अवगम्य नारदात्‌
स्वपार्षदसैन्यं च तदध्वरर्भुभिर्विद्रावितं क्रोधमपारमादधे
क्रुद्धः सुदष्टौष्ठपुटः स धूर्जटिर्जटां तडिद्वह्निसटोग्ररोचिषम्‌
उत्कृत्य रुद्रः सहसोत्थितो हसन्गम्भीरनादो विससर्ज तां भुवि
ततोऽतिकायस्तनुवा स्पृशन्दिवं सहस्रबाहुर्घनरुक्त्रिसूर्यदृक्‌
करालदंष्ट्रो ज्वलदग्निमूर्धजः कपालमाली विविधोद्यतायुधः
तं किं करोमीति गृणन्तमाह बद्धाञ्जलिं भगवान्भूतनाथः
दक्षं सयज्ञं जहि मद्भटानां त्वमग्रणी रुद्र भटांशको मे
आज्ञप्त एवं कुपितेन मन्युना स देवदेवं परिचक्रमे विभुम्‌
मेनेतदात्मानमसङ्गरंहसा महीयसां तात सहः सहिष्णुम्‌
अन्वीयमानः स तु रुद्रपार्षदैर्भृशं नदद्भिर्व्यनदत्सुभैरवम्‌
उद्यम्य शूलं जगदन्तकान्तकं सम्प्राद्रवद्घोषणभूषणाङ्घ्रिः
अथर्त्विजो यजमानः सदस्याः ककुभ्युदीच्यां प्रसमीक्ष्य रेणुम्‌
तमः किमेतत्कुत एतद्रजोऽभूदिति द्विजा द्विजपत्न्यश्च दध्युः
वाता न वान्ति न हि सन्ति दस्यवः प्राचीनबर्हिर्जीवति होग्रदण्डः
गावो न काल्यन्त इदं कुतो रजो लोकोऽधुना किं प्रलयाय कल्पते
प्रसूतिमिश्राः स्त्रिय उद्विग्नचित्ता ऊचुर्विपाको वृजिनस्यैव तस्य
यत्पश्यन्तीनां दुहित्ऱ्‌^ईणां प्रजेशः सुतां सतीमवदध्यावनागाम्‌
यस्त्वन्तकाले व्युप्तजटाकलापः स्वशूलसूच्यर्पितदिग्गजेन्द्रः
वितत्य नृत्यत्युदितास्त्रदोर्ध्वजानुच्चाट्टहासस्तनयित्नुभिन्नदिक्‌
अमर्षयित्वा तमसह्यतेजसं मन्युप्लुतं दुर्निरीक्ष्यं भ्रुकुट्या
करालदंष्ट्राभिरुदस्तभागणं स्यात्स्वस्ति किं कोपयतो विधातुः
बह्वेवमुद्विग्नदृशोच्यमाने जनेन दक्षस्य मुहुर्महात्मनः
उत्पेतुरुत्पाततमाः सहस्रशो भयावहा दिवि भूमौ च पर्यक्‌
तावत्स रुद्रानुचरैर्महामखो नानायुधैर्वामनकैरुदायुधैः
पिङ्गैः पिशङ्गैर्मकरोदराननैः पर्याद्रवद्भिर्विदुरान्वरुध्यत
केचिद्बभञ्जुः प्राग्वंशं पत्नीशालां तथापरे
सद आग्नीध्रशालां च तद्विहारं महानसम्‌
रुरुजुर्यज्ञपात्राणि तथैकेऽग्नीननाशयन्‌
कुण्डेष्वमूत्रयन्केचिद्बिभिदुर्वेदिमेखलाः
अबाधन्त मुनीनन्ये एके पत्नीरतर्जयन्‌
अपरे जगृहुर्देवान्प्रत्यासन्नान्पलायितान्‌
भृगुं बबन्ध मणिमान्वीरभद्रः प्रजापतिम्‌
चण्डेशः पूषणं देवं भगं नन्दीश्वरोऽग्रहीत्‌
सर्व एवर्त्विजो दृष्ट्वा सदस्याः सदिवौकसः
तैरर्द्यमानाः सुभृशं ग्रावभिर्नैकधाद्रवन्‌
जुह्वतः स्रुवहस्तस्य श्मश्रूणि भगवान्भवः
भृगोर्लुलुञ्चे सदसि योऽहसच्छ्मश्रु दर्शयन्‌
भगस्य नेत्रे भगवान्पातितस्य रुषा भुवि
उज्जहार सदस्थोऽक्ष्णा यः शपन्तमसूसुचत्‌
पूष्णो ह्यपातयद्दन्तान्कालिङ्गस्य यथा बलः
शप्यमाने गरिमणि योऽहसद्दर्शयन्दतः
आक्रम्योरसि दक्षस्य शितधारेण हेतिना
छिन्दन्नपि तदुद्धर्तुं नाशक्नोत्त्र्यम्बकस्तदा
शस्त्रैरस्त्रान्वितैरेवमनिर्भिन्नत्वचं हरः
विस्मयं परमापन्नो दध्यौ पशुपतिश्चिरम्‌
दृष्ट्वा संज्ञपनं योगं पशूनां स पतिर्मखे
यजमानपशोः कस्य कायात्तेनाहरच्छिरः
साधुवादस्तदा तेषां कर्म तत्तस्य पश्यताम्‌
भूतप्रेतपिशाचानां अन्येषां तद्विपर्ययः
जुहावैतच्छिरस्तस्मिन्दक्षिणाग्नावमर्षितः
तद्देवयजनं दग्ध्वा प्रातिष्ठद्गुह्यकालयम्‌
मैत्रेय उवाच
अथ देवगणाः सर्वे रुद्रानीकैः पराजिताः
शूलपट्टिशनिस्त्रिंश गदापरिघमुद्गरैः
सञ्छिन्नभिन्नसर्वाङ्गाः सर्त्विक्सभ्या भयाकुलाः
स्वयम्भुवे नमस्कृत्य कार्त्स्न्येनैतन्न्यवेदयन्‌
उपलभ्य पुरैवैतद्भगवानब्जसम्भवः
नारायणश्च विश्वात्मा न कस्याध्वरमीयतुः
तदाकर्ण्य विभुः प्राह तेजीयसि कृतागसि
क्षेमाय तत्र सा भूयान्न प्रायेण बुभूषताम्‌
अथापि यूयं कृतकिल्बिषा भवं ये बर्हिषो भागभाजं परादुः
प्रसादयध्वं परिशुद्धचेतसा क्षिप्रप्रसादं प्रगृहीताङ्घ्रिपद्मम्‌
आशासाना जीवितमध्वरस्य लोकः सपालः कुपिते न यस्मिन्‌
तमाशु देवं प्रियया विहीनं क्षमापयध्वं हृदि विद्धं दुरुक्तैः
नाहं न यज्ञो न च यूयमन्ये ये देहभाजो मुनयश्च तत्त्वम्‌
विदुः प्रमाणं बलवीर्ययोर्वा यस्यात्मतन्त्रस्य क उपायं विधित्सेत्‌
स इत्थमादिश्य सुरानजस्तु तैः समन्वितः पितृभिः सप्रजेशैः
ययौ स्वधिष्ण्यान्निलयं पुरद्विषः कैलासमद्रिप्रवरं प्रियं प्रभोः
जन्मौषधितपोमन्त्र योगसिद्धैर्नरेतरैः
जुष्टं किन्नरगन्धर्वैरप्सरोभिर्वृतं सदा
नानामणिमयैः शृङ्गैर्नानाधातुविचित्रितैः
नानाद्रुमलतागुल्मैर्नानामृगगणावृतैः
नानामलप्रस्रवणैर्नानाकन्दरसानुभिः
रमणं विहरन्तीनां रमणैः सिद्धयोषिताम्‌
मयूरकेकाभिरुतं मदान्धालिविमूर्च्छितम्‌
प्लावितै रक्तकण्ठानां कूजितैश्च पतत्त्रिणाम्‌
आह्वयन्तमिवोद्धस्तैर्द्विजान्कामदुघैर्द्रुमैः
व्रजन्तमिव मातङ्गैर्गृणन्तमिव निर्झरैः
मन्दारैः पारिजातैश्च सरलैश्चोपशोभितम्‌
तमालैः शालतालैश्च कोविदारासनार्जुनैः
चूतैः कदम्बैर्नीपैश्च नागपुन्नागचम्पकैः
पाटलाशोकबकुलैः कुन्दैः कुरबकैरपि
स्वर्णार्णशतपत्रैश्च वररेणुकजातिभिः
कुब्जकैर्मल्लिकाभिश्च माधवीभिश्च मण्डितम्‌
पनसोदुम्बराश्वत्थ प्लक्षन्यग्रोधहिङ्गुभिः
भूर्जैरोषधिभिः पूगै राजपूगैश्च जम्बुभिः
खर्जूराम्रातकाम्राद्यैः प्रियालमधुकेङ्गुदैः
द्रुमजातिभिरन्यैश्च राजितं वेणुकीचकैः
कुमुदोत्पलकह्लार शतपत्रवनर्द्धिभिः
नलिनीषु कलं कूजत्‌ खगवृन्दोपशोभितम्‌
मृगैः शाखामृगैः क्रोडैर्मृगेन्द्रैरृक्षशल्यकैः
गवयैः शरभैर्व्याघ्रै रुरुभिर्महिषादिभिः
कर्णान्त्रैकपदाश्वास्यैर्निर्जुष्टं वृकनाभिभिः
कदलीखण्डसंरुद्ध नलिनीपुलिनश्रियम्‌
पर्यस्तं नन्दया सत्याः स्नानपुण्यतरोदया
विलोक्य भूतेशगिरिं विबुधा विस्मयं ययुः
ददृशुस्तत्र ते रम्यामलकां नाम वै पुरीम्‌
वनं सौगन्धिकं चापि यत्र तन्नाम पङ्कजम्‌
नन्दा चालकनन्दा च सरितौ बाह्यतः पुरः
तीर्थपादपदाम्भोज रजसातीव पावने
ययोः सुरस्त्रियः क्षत्तरवरुह्य स्वधिष्ण्यतः
क्रीडन्ति पुंसः सिञ्चन्त्यो विगाह्य रतिकर्शिताः
ययोस्तत्स्नानविभ्रष्ट नवकुङ्कुमपिञ्जरम्‌
वितृषोऽपि पिबन्त्यम्भः पाययन्तो गजा गजीः
तारहेममहारत्न विमानशतसङ्कुलाम्‌
जुष्टां पुण्यजनस्त्रीभिर्यथा खं सतडिद्घनम्‌
हित्वा यक्षेश्वरपुरीं वनं सौगन्धिकं च तत्‌
द्रुमैः कामदुघैर्हृद्यं चित्रमाल्यफलच्छदैः
रक्तकण्ठखगानीक स्वरमण्डितषट्पदम्‌
कलहंसकुलप्रेष्ठं खरदण्डजलाशयम्‌
वनकुञ्जरसङ्घृष्ट हरिचन्दनवायुना
अधि पुण्यजनस्त्रीणां मुहुरुन्मथयन्मनः
वैदूर्यकृतसोपाना वाप्य उत्पलमालिनीः
प्राप्तं किम्पुरुषैर्दृष्ट्वा त आराद्ददृशुर्वटम्‌
स योजनशतोत्सेधः पादोनविटपायतः
पर्यक्कृताचलच्छायो निर्नीडस्तापवर्जितः
तस्मिन्महायोगमये मुमुक्षुशरणे सुराः
ददृशुः शिवमासीनं त्यक्तामर्षमिवान्तकम्‌
सनन्दनाद्यैर्महासिद्धैः शान्तैः संशान्तविग्रहम्‌
उपास्यमानं सख्या च भर्त्रा गुह्यकरक्षसाम्‌
विद्यातपोयोगपथमास्थितं तमधीश्वरम्‌
चरन्तं विश्वसुहृदं वात्सल्याल्लोकमङ्गलम्‌
लिङ्गं च तापसाभीष्टं भस्मदण्डजटाजिनम्‌
अङ्गेन सन्ध्याभ्ररुचा चन्द्रलेखां च बिभ्रतम्‌
उपविष्टं दर्भमय्यां बृस्यां ब्रह्म सनातनम्‌
नारदाय प्रवोचन्तं पृच्छते शृण्वतां सताम्‌
कृत्वोरौ दक्षिणे सव्यं पादपद्मं च जानुनि
बाहुं प्रकोष्ठेऽक्षमालामासीनं तर्कमुद्रया
तं ब्रह्मनिर्वाणसमाधिमाश्रितं व्युपाश्रितं गिरिशं योगकक्षाम्‌
सलोकपाला मुनयो मनूनामाद्यं मनुं प्राञ्जलयः प्रणेमुः
स तूपलभ्यागतमात्मयोनिं सुरासुरेशैरभिवन्दिताङ्घ्रिः
उत्थाय चक्रे शिरसाभिवन्दनमर्हत्तमः कस्य यथैव विष्णुः
तथापरे सिद्धगणा महर्षिभिर्ये वै समन्तादनु नीललोहितम्‌
नमस्कृतः प्राह शशाङ्कशेखरं कृतप्रणामं प्रहसन्निवात्मभूः
ब्रह्मोवाच
आने त्वामीशं विश्वस्य जगतो योनिबीजयोः
शक्तेः शिवस्य च परं यत्तद्ब्रह्मा निरन्तरम्‌
त्वमेव भगवन्नेतच्छिवशक्त्योः स्वरूपयोः
विश्वं सृजसि पास्यत्सि क्रीडन्नूर्णपटो यथा
त्वमेव धर्मार्थदुघाभिपत्तये दक्षेण सूत्रेण ससर्जिथाध्वरम्‌
त्वयैव लोकेऽवसिताश्च सेतवो यान्ब्राह्मणाः श्रद्दधते धृतव्रताः
त्वं कर्मणां मङ्गल मङ्गलानां कर्तुः स्वलोकं तनुषे स्वः परं वा
अमङ्गलानां च तमिस्रमुल्बणं विपर्ययः केन तदेव कस्यचित्‌
न वै सतां त्वच्चरणार्पितात्मनां भूतेषु सर्वेष्वभिपश्यतां तव
भूतानि चात्मन्यपृथग्दिदृक्षतां प्रायेण रोषोऽभिभवेद्यथा पशुम्‌
पृथग्धियः कर्मदृशो दुराशयाः परोदयेनार्पितहृद्रुजोऽनिशम्‌
परान्दुरुक्तैर्वितुदन्त्यरुन्तुदास्तान्मावधीद्दैववधान्भवद्विधः
यस्मिन्यदा पुष्करनाभमायया दुरन्तया स्पृष्टधियः पृथग्दृशः
कुर्वन्ति तत्र ह्यनुकम्पया कृपां न साधवो दैवबलात्कृते क्रमम्‌
भवांस्तु पुंसः परमस्य मायया दुरन्तयास्पृष्टमतिः समस्तदृक्‌
तया हतात्मस्वनुकर्मचेतःस्वनुग्रहं कर्तुमिहार्हसि प्रभो
कुर्वध्वरस्योद्धरणं हतस्य भोः त्वयासमाप्तस्य मनो प्रजापतेः
न यत्र भागं तव भागिनो ददुः कुयाजिनो येन मखो निनीयते
जीवताद्यजमानोऽयं प्रपद्येताक्षिणी भगः
भृगोः श्मश्रूणि रोहन्तु पूष्णो दन्ताश्च पूर्ववत्‌
देवानां भग्नगात्राणामृत्विजां चायुधाश्मभिः
भवतानुगृहीतानामाशु मन्योऽस्त्वनातुरम्‌
एष ते रुद्र भागोऽस्तु यदुच्छिष्टोऽध्वरस्य वै
यज्ञस्ते रुद्र भागेन कल्पतामद्य यज्ञहन्‌
मैत्रेय उवाच
इत्यजेनानुनीतेन भवेन परितुष्यता
अभ्यधायि महाबाहो प्रहस्य श्रूयतामिति
महादेव उवाच
नाघं प्रजेश बालानां वर्णये नानुचिन्तये
देवमायाभिभूतानां दण्डस्तत्र धृतो मया
प्रजापतेर्दग्धशीर्ष्णो भवत्वजमुखं शिरः
मित्रस्य चक्षुषेक्षेत भागं स्वं बर्हिषो भगः
पूषा तु यजमानस्य दद्भिर्जक्षतु पिष्टभुक्‌
देवाः प्रकृतसर्वाङ्गा ये म उच्छेषणं ददुः
बाहुभ्यामश्विनोः पूष्णो हस्ताभ्यां कृतबाहवः
भवन्त्वध्वर्यवश्चान्ये बस्तश्मश्रुर्भृगुर्भवेत्‌
मैत्रेय उवाच
तदा सर्वाणि भूतानि श्रुत्वा मीढुष्टमोदितम्‌
परितुष्टात्मभिस्तात साधु साध्वित्यथाब्रुवन्‌
ततो मीढ्वांसमामन्त्र्य शुनासीराः सहर्षिभिः
भूयस्तद्देवयजनं समीढ्वद्वेधसो ययुः
विधाय कार्त्स्न्येन च तद्यदाह भगवान्भवः
सन्दधुः कस्य कायेन सवनीयपशोः शिरः
सन्धीयमाने शिरसि दक्षो रुद्राभिवीक्षितः
सद्यः सुप्त इवोत्तस्थौ ददृशे चाग्रतो मृडम्‌
तदा वृषध्वजद्वेष कलिलात्मा प्रजापतिः
शिवावलोकादभवच्छरद्ध्रद इवामलः
भवस्तवाय कृतधीर्नाशक्नोदनुरागतः
औत्कण्ठ्याद्बाष्पकलया सम्परेतां सुतां स्मरन्‌
कृच्छ्रात्संस्तभ्य च मनः प्रेमविह्वलितः सुधीः
शशंस निर्व्यलीकेन भावेनेशं प्रजापतिः
दक्ष उवाच
भूयाननुग्रह अहो भवता कृतो मे
दण्डस्त्वया मयि भृतो यदपि प्रलब्धः
न ब्रह्मबन्धुषु च वां भगवन्नवज्ञा
तुभ्यं हरेश्च कुत एव धृतव्रतेषु
विद्यातपोव्रतधरान्मुखतः स्म विप्रान्‌
ब्रह्मात्मतत्त्वमवितुं प्रथमं त्वमस्राक्‌
तद्ब्राह्मणान्परम सर्वविपत्सु पासि
पालः पशूनिव विभो प्रगृहीतदण्डः
योऽसौ मयाविदिततत्त्वदृशा सभायां
क्षिप्तो दुरुक्तिविशिखैर्विगणय्य तन्माम्‌
अर्वाक्पतन्तमर्हत्तमनिन्दयापाद्‌
दृष्ट्यार्द्रया स भगवान्स्वकृतेन तुष्येत्‌
मैत्रेय उवाच
क्षमाप्यैवं स मीढ्वांसं ब्रह्मणा चानुमन्त्रितः
कर्म सन्तानयामास सोपाध्यायर्त्विगादिभिः
वैष्णवं यज्ञसन्तत्यै त्रिकपालं द्विजोत्तमाः
पुरोडाशं निरवपन्वीरसंसर्गशुद्धये
अध्वर्युणात्तहविषा यजमानो विशाम्पते
धिया विशुद्धया दध्यौ तथा प्रादुरभूद्धरिः
तदा स्वप्रभया तेषां द्योतयन्त्या दिशो दश
मुष्णंस्तेज उपानीतस्तार्क्ष्येण स्तोत्रवाजिना
श्यामो हिरण्यरशनोऽर्ककिरीटजुष्टो
नीलालकभ्रमरमण्डितकुण्डलास्यः
शङ्खाब्जचक्रशरचापगदासिचर्म
व्यग्रैर्हिरण्मयभुजैरिव कर्णिकारः
वक्षस्यधिश्रितवधूर्वनमाल्युदार
हासावलोककलया रमयंश्च विश्वम्‌
पार्श्वभ्रमद्व्यजनचामरराजहंसः
श्वेतातपत्रशशिनोपरि रज्यमानः
तमुपागतमालक्ष्य सर्वे सुरगणादयः
प्रणेमुः सहसोत्थाय ब्रह्मेन्द्रत्र्यक्षनायकाः
तत्तेजसा हतरुचः सन्नजिह्वाः ससाध्वसाः
मूर्ध्ना धृताञ्जलिपुटा उपतस्थुरधोक्षजम्‌
अप्यर्वाग्वृत्तयो यस्य महि त्वात्मभुवादयः
यथामति गृणन्ति स्म कृतानुग्रहविग्रहम्‌
दक्षो गृहीतार्हणसादनोत्तमं
यज्ञेश्वरं विश्वसृजां परं गुरुम्‌
सुनन्दनन्दाद्यनुगैर्वृतं मुदा
गृणन्प्रपेदे प्रयतः कृताञ्जलिः
दक्ष उवाच
शुद्धं स्वधाम्न्युपरताखिलबुद्ध्यवस्थं
चिन्मात्रमेकमभयं प्रतिषिध्य मायाम्‌
तिष्ठंस्तयैव पुरुषत्वमुपेत्य तस्याम्‌
आस्ते भवानपरिशुद्ध इवात्मतन्त्रः
ऋत्विज ऊचुः
तत्त्वं न ते वयमनञ्जन रुद्रशापात्‌
कर्मण्यवग्रहधियो भगवन्विदामः
धर्मोपलक्षणमिदं त्रिवृदध्वराख्यं
ज्ञातं यदर्थमधिदैवमदो व्यवस्थाः
सदस्या ऊचुः
उत्पत्त्यध्वन्यशरण उरुक्लेशदुर्गेऽन्तकोग्र
व्यालान्विष्टे विषयमृगतृष्यात्मगेहोरुभारः
द्वन्द्वश्वभ्रे खलमृगभये शोकदावेऽज्ञसार्थः
पादौकस्ते शरणद कदा याति कामोपसृष्टः
रुद्र उवाच
तव वरद वराङ्घ्रावाशिषेहाखिलार्थे
ह्यपि मुनिभिरसक्तैरादरेणार्हणीये
यदि रचितधियं माविद्यलोकोऽपविद्धं
जपति न गणये तत्त्वत्परानुग्रहेण
भृगुरुवाच
यन्मायया गहनयापहृतात्मबोधा
ब्रह्मादयस्तनुभृतस्तमसि स्वपन्तः
नात्मन्श्रितं तव विदन्त्यधुनापि तत्त्वं
सोऽयं प्रसीदतु भवान्प्रणतात्मबन्धुः
ब्रह्मोवाच
नैतत्स्वरूपं भवतोऽसौ पदार्थ भेदग्रहैः पुरुषो यावदीक्षेत्‌
ज्ञानस्य चार्थस्य गुणस्य चाश्रयो मायामयाद्व्यतिरिक्तो मतस्त्वम्‌
इन्द्र उवाच
इदमप्यच्युत विश्वभावनं वपुरानन्दकरं मनोदृशाम्‌
सुरविद्विट्क्षपणैरुदायुधैर्भुजदण्डैरुपपन्नमष्टभिः
पत्न्य ऊचुः
यज्ञोऽयं तव यजनाय केन सृष्टो विध्वस्तः पशुपतिनाद्य दक्षकोपात्‌
तं नस्त्वं शवशयनाभशान्तमेधं यज्ञात्मन्नलिनरुचा दृशा पुनीहि
ऋषय ऊचुः
अनन्वितं ते भगवन्विचेष्टितं यदात्मना चरसि हि कर्म नाज्यसे
विभूतये यत उपसेदुरीश्वरीं न मन्यते स्वयमनुवर्ततीं भवान्‌
सिद्धा ऊचुः
अयं त्वत्कथामृष्टपीयूषनद्यां मनोवारणः क्लेशदावाग्निदग्धः
तृषार्तोऽवगाढो न सस्मार दावं न निष्क्रामति ब्रह्मसम्पन्नवन्नः
यजमान्युवाच
स्वागतं ते प्रसीदेश तुभ्यं नमः श्रीनिवास श्रिया कान्तया त्राहि नः
त्वामृतेऽधीश नाङ्गैर्मखः शोभते शीर्षहीनः कबन्धो यथा पुरुषः
लोकपाला ऊचुः
दृष्टः किं नो दृग्भिरसद्ग्रहैस्त्वं प्रत्यग्द्रष्टा दृश्यते येन विश्वम्‌
माया ह्येषा भवदीया हि भूमन्यस्त्वं षष्ठः पञ्चभिर्भासि भूतैः
योगेश्वरा ऊचुः
प्रेयान्न तेऽन्योऽस्त्यमुतस्त्वयि प्रभो विश्वात्मनीक्षेन्न पृथग्य आत्मनः
अथापि भक्त्येश तयोपधावतामनन्यवृत्त्यानुगृहाण वत्सल
जगदुद्भवस्थितिलयेषु दैवतो बहुभिद्यमानगुणयात्ममायया
रचितात्मभेदमतये स्वसंस्थया विनिवर्तितभ्रमगुणात्मने नमः
ब्रह्मोवाच
नमस्ते श्रितसत्त्वाय धर्मादीनां च सूतये
निर्गुणाय च यत्काष्ठां नाहं वेदापरेऽपि च
अग्निरुवाच
यत्तेजसाहं सुसमिद्धतेजा हव्यं वहे स्वध्वर आज्यसिक्तम्‌
तं यज्ञियं पञ्चविधं च पञ्चभिः स्विष्टं यजुर्भिः प्रणतोऽस्मि यज्ञम्‌
देवा ऊचुः
पुरा कल्पापाये स्वकृतमुदरीकृत्य विकृतं
त्वमेवाद्यस्तस्मिन्सलिल उरगेन्द्राधिशयने
पुमान्शेषे सिद्धैर्हृदि विमृशिताध्यात्मपदविः
स एवाद्याक्ष्णोर्यः पथि चरसि भृत्यानवसि नः
गन्धर्वा ऊचुः
अंशांशास्ते देव मरीच्यादय एते ब्रह्मेन्द्राद्या देवगणा रुद्रपुरोगाः
क्रीडाभाण्डं विश्वमिदं यस्य विभूमन्तस्मै नित्यं नाथ नमस्ते करवाम
विद्याधरा ऊचुः
त्वन्माययार्थमभिपद्य कलेवरेऽस्मिन्‌
कृत्वा ममाहमिति दुर्मतिरुत्पथैः स्वैः
क्षिप्तोऽप्यसद्विषयलालस आत्ममोहं
युष्मत्कथामृतनिषेवक उद्व्युदस्येत्‌
ब्राह्मणा ऊचुः
त्वं क्रतुस्त्वं हविस्त्वं हुताशः स्वयं त्वं हि मन्त्रः समिद्दर्भपात्राणि च
त्वं सदस्यर्त्विजो दम्पती देवता अग्निहोत्रं स्वधा सोम आज्यं पशुः
त्वं पुरा गां रसाया महासूकरो दंष्ट्रया पद्मिनीं वारणेन्द्रो यथा
स्तूयमानो नदल्लीलया योगिभिर्व्युज्जहर्थ त्रयीगात्र यज्ञक्रतुः
स प्रसीद त्वमस्माकमाकाङ्क्षतां दर्शनं ते परिभ्रष्टसत्कर्मणाम्‌
कीर्त्यमाने नृभिर्नाम्नि यज्ञेश ते यज्ञविघ्नाः क्षयं यान्ति तस्मै नमः
मैत्रेय उवाच
इति दक्षः कविर्यज्ञं भद्र रुद्राभिमर्शितम्‌
कीर्त्यमाने हृषीकेशे सन्निन्ये यज्ञभावने
भगवान्स्वेन भागेन सर्वात्मा सर्वभागभुक्‌
दक्षं बभाष आभाष्य प्रीयमाण इवानघ
श्रीभगवानुवाच
अहं ब्रह्मा च शर्वश्च जगतः कारणं परम्‌
आत्मेश्वर उपद्रष्टा स्वयन्दृगविशेषणः
आत्ममायां समाविश्य सोऽहं गुणमयीं द्विज
सृजन्रक्षन्हरन्विश्वं दध्रे संज्ञां क्रियोचिताम्‌
तस्मिन्ब्रह्मण्यद्वितीये केवले परमात्मनि
ब्रह्मरुद्रौ च भूतानि भेदेनाज्ञोऽनुपश्यति
यथा पुमान्न स्वाङ्गेषु शिरःपाण्यादिषु क्वचित्‌
पारक्यबुद्धिं कुरुते एवं भूतेषु मत्परः
त्रयाणामेकभावानां यो न पश्यति वै भिदाम्‌
सर्वभूतात्मनां ब्रह्मन्स शान्तिमधिगच्छति
मैत्रेय उवाच
एवं भगवतादिष्टः प्रजापतिपतिर्हरिम्‌
अर्चित्वा क्रतुना स्वेन देवानुभयतोऽयजत्‌
रुद्रं च स्वेन भागेन ह्युपाधावत्समाहितः
कर्मणोदवसानेन सोमपानितरानपि
उदवस्य सहर्त्विग्भिः सस्नाववभृथं ततः
तस्मा अप्यनुभावेन स्वेनैवावाप्तराधसे
धर्म एव मतिं दत्त्वा त्रिदशास्ते दिवं ययुः
एवं दाक्षायणी हित्वा सती पूर्वकलेवरम्‌
जज्ञे हिमवतः क्षेत्रे मेनायामिति शुश्रुम
तमेव दयितं भूय आवृङ्क्ते पतिमम्बिका
अनन्यभावैकगतिं शक्तिः सुप्तेव पूरुषम्‌
एतद्भगवतः शम्भोः कर्म दक्षाध्वरद्रुहः
श्रुतं भागवताच्छिष्यादुद्धवान्मे बृहस्पतेः
इदं पवित्रं परमीशचेष्टितं यशस्यमायुष्यमघौघमर्षणम्‌
यो नित्यदाकर्ण्य नरोऽनुकीर्तयेद्धुनोत्यघं कौरव भक्तिभावतः
मैत्रेय उवाच
सनकाद्या नारदश्च ऋभुर्हंसोऽरुणिर्यतिः
नैते गृहान्ब्रह्मसुता ह्यावसन्नूर्ध्वरेतसः
मृषाधर्मस्य भार्यासीद्दम्भं मायां च शत्रुहन्‌
असूत मिथुनं तत्तु निरृतिर्जगृहेऽप्रजः
तयोः समभवल्लोभो निकृतिश्च महामते
ताभ्यां क्रोधश्च हिंसा च यद्दुरुक्तिः स्वसा कलिः
दुरुक्तौ कलिराधत्त भयं मृत्युं च सत्तम
तयोश्च मिथुनं जज्ञे यातना निरयस्तथा
सङ्ग्रहेण मयाख्यातः प्रतिसर्गस्तवानघ
त्रिः श्रुत्वैतत्पुमान्पुण्यं विधुनोत्यात्मनो मलम्‌
अथातः कीर्तये वंशं पुण्यकीर्तेः कुरूद्वह
स्वायम्भुवस्यापि मनोर्हरेरंशांशजन्मनः
प्रियव्रतोत्तानपादौ शतरूपापतेः सुतौ
वासुदेवस्य कलया रक्षायां जगतः स्थितौ
जाये उत्तानपादस्य सुनीतिः सुरुचिस्तयोः
सुरुचिः प्रेयसी पत्युर्नेतरा यत्सुतो ध्रुवः
एकदा सुरुचेः पुत्रमङ्कमारोप्य लालयन्‌
उत्तमं नारुरुक्षन्तं ध्रुवं राजाभ्यनन्दत
तथा चिकीर्षमाणं तं सपत्न्यास्तनयं ध्रुवम्‌
सुरुचिः शृण्वतो राज्ञः सेर्ष्यमाहातिगर्विता
न वत्स नृपतेर्धिष्ण्यं भवानारोढुमर्हति
न गृहीतो मया यत्त्वं कुक्षावपि नृपात्मजः
बालोऽसि बत नात्मानमन्यस्त्रीगर्भसम्भृतम्‌
नूनं वेद भवान्यस्य दुर्लभेऽर्थे मनोरथः
तपसाराध्य पुरुषं तस्यैवानुग्रहेण मे
गर्भे त्वं साधयात्मानं यदीच्छसि नृपासनम्‌
मैत्रेय उवाच
मातुः सपत्न्याः स दुरुक्तिविद्धः श्वसन्रुषा दण्डहतो यथाहिः
हित्वा मिषन्तं पितरं सन्नवाचं जगाम मातुः प्ररुदन्सकाशम्‌
तं निःश्वसन्तं स्फुरिताधरोष्ठं सुनीतिरुत्सङ्ग उदूह्य बालम्‌
निशम्य तत्पौरमुखान्नितान्तं सा विव्यथे यद्गदितं सपत्न्या
सोत्सृज्य धैर्यं विललाप शोक दावाग्निना दावलतेव बाला
वाक्यं सपत्न्याः स्मरती सरोज श्रिया दृशा बाष्पकलामुवाह
दीर्घं श्वसन्ती वृजिनस्य पारमपश्यती बालकमाह बाला
मामङ्गलं तात परेषु मंस्था भुङ्क्ते जनो यत्परदुःखदस्तत्‌
सत्यं सुरुच्याभिहितं भवान्मे यद्दुर्भगाया उदरे गृहीतः
स्तन्येन वृद्धश्च विलज्जते यां भार्येति वा वोढुमिडस्पतिर्माम्‌
आतिष्ठ तत्तात विमत्सरस्त्वमुक्तं समात्रापि यदव्यलीकम्‌
आराधयाधोक्षजपादपद्मं यदीच्छसेऽध्यासनमुत्तमो यथा
यस्याङ्घ्रिपद्मं परिचर्य विश्व विभावनायात्तगुणाभिपत्तेः
अजोऽध्यतिष्ठत्खलु पारमेष्ठ्यं पदं जितात्मश्वसनाभिवन्द्यम्‌
तथा मनुर्वो भगवान्पितामहो यमेकमत्या पुरुदक्षिणैर्मखैः
इष्ट्वाभिपेदे दुरवापमन्यतो भौमं सुखं दिव्यमथापवर्ग्यम्‌
तमेव वत्साश्रय भृत्यवत्सलं मुमुक्षुभिर्मृग्यपदाब्जपद्धतिम्‌
अनन्यभावे निजधर्मभाविते मनस्यवस्थाप्य भजस्व पूरुषम्‌
नान्यं ततः पद्मपलाशलोचनाद्दुःखच्छिदं ते मृगयामि कञ्चन
यो मृग्यते हस्तगृहीतपद्मया श्रियेतरैरङ्ग विमृग्यमाणया
मैत्रेय उवाच
एवं सञ्जल्पितं मातुराकर्ण्यार्थागमं वचः
सन्नियम्यात्मनात्मानं निश्चक्राम पितुः पुरात्‌
नारदस्तदुपाकर्ण्य ज्ञात्वा तस्य चिकीर्षितम्‌
स्पृष्ट्वा मूर्धन्यघघ्नेन पाणिना प्राह विस्मितः
अहो तेजः क्षत्रियाणां मानभङ्गममृष्यताम्‌
बालोऽप्ययं हृदा धत्ते यत्समातुरसद्वचः
नारद उवाच
नाधुनाप्यवमानं ते सम्मानं वापि पुत्रक
लक्षयामः कुमारस्य सक्तस्य क्रीडनादिषु
विकल्पे विद्यमानेऽपि न ह्यसन्तोषहेतवः
पुंसो मोहमृते भिन्ना यल्लोके निजकर्मभिः
परितुष्येत्ततस्तात तावन्मात्रेण पूरुषः
दैवोपसादितं यावद्वीक्ष्येश्वरगतिं बुधः
अथ मात्रोपदिष्टेन योगेनावरुरुत्ससि
यत्प्रसादं स वै पुंसां दुराराध्यो मतो मम
मुनयः पदवीं यस्य निःसङ्गेनोरुजन्मभिः
न विदुर्मृगयन्तोऽपि तीव्रयोगसमाधिना
अतो निवर्ततामेष निर्बन्धस्तव निष्फलः
यतिष्यति भवान्काले श्रेयसां समुपस्थिते
यस्य यद्दैवविहितं स तेन सुखदुःखयोः
आत्मानं तोषयन्देही तमसः पारमृच्छति
गुणाधिकान्मुदं लिप्सेदनुक्रोशं गुणाधमात्‌
मैत्रीं समानादन्विच्छेन्न तापैरभिभूयते
ध्रुव उवाच
सोऽयं शमो भगवता सुखदुःखहतात्मनाम्‌
दर्शितः कृपया पुंसां दुर्दर्शोऽस्मद्विधैस्तु यः
अथापि मेऽविनीतस्य क्षात्त्रं घोरमुपेयुषः
सुरुच्या दुर्वचोबाणैर्न भिन्ने श्रयते हृदि
पदं त्रिभुवनोत्कृष्टं जिगीषोः साधु वर्त्म मे
ब्रूह्यस्मत्पितृभिर्ब्रह्मन्नन्यैरप्यनधिष्ठितम्‌
नूनं भवान्भगवतो योऽङ्गजः परमेष्ठिनः
वितुदन्नटते वीणां हिताय जगतोऽर्कवत्‌
मैत्रेय उवाच
इत्युदाहृतमाकर्ण्य भगवान्नारदस्तदा
प्रीतः प्रत्याह तं बालं सद्वाक्यमनुकम्पया
नारद उवाच
जनन्याभिहितः पन्थाः स वै निःश्रेयसस्य ते
भगवान्वासुदेवस्तं भज तं प्रवणात्मना
धर्मार्थकाममोक्षाख्यं य इच्छेच्छ्रेय आत्मनः
एकं ह्येव हरेस्तत्र कारणं पादसेवनम्‌
तत्तात गच्छ भद्रं ते यमुनायास्तटं शुचि
पुण्यं मधुवनं यत्र सान्निध्यं नित्यदा हरेः
स्नात्वानुसवनं तस्मिन्कालिन्द्याः सलिले शिवे
कृत्वोचितानि निवसन्नात्मनः कल्पितासनः
प्राणायामेन त्रिवृता प्राणेन्द्रियमनोमलम्‌
शनैर्व्युदस्याभिध्यायेन्मनसा गुरुणा गुरुम्‌
प्रसादाभिमुखं शश्वत्प्रसन्नवदनेक्षणम्‌
सुनासं सुभ्रुवं चारु कपोलं सुरसुन्दरम्‌
तरुणं रमणीयाङ्गमरुणोष्ठेक्षणाधरम्‌
प्रणताश्रयणं नृम्णं शरण्यं करुणार्णवम्‌
श्रीवत्साङ्कं घनश्यामं पुरुषं वनमालिनम्‌
शङ्खचक्रगदापद्मैरभिव्यक्तचतुर्भुजम्‌
किरीटिनं कुण्डलिनं केयूरवलयान्वितम्‌
कौस्तुभाभरणग्रीवं पीतकौशेयवाससम्‌
काञ्चीकलापपर्यस्तं लसत्काञ्चननूपुरम्‌
दर्शनीयतमं शान्तं मनोनयनवर्धनम्‌
पद्भ्यां नखमणिश्रेण्या विलसद्भ्यां समर्चताम्‌
हृत्पद्मकर्णिकाधिष्ण्यमाक्रम्यात्मन्यवस्थितम्‌
स्मयमानमभिध्यायेत्सानुरागावलोकनम्‌
नियतेनैकभूतेन मनसा वरदर्षभम्‌
एवं भगवतो रूपं सुभद्रं ध्यायतो मनः
निर्वृत्या परया तूर्णं सम्पन्नं न निवर्तते
जपश्च परमो गुह्यः श्रूयतां मे नृपात्मज
यं सप्तरात्रं प्रपठन्पुमान्पश्यति खेचरान्‌
ओं नमो भगवते वासुदेवाय
मन्त्रेणानेन देवस्य कुर्याद्द्रव्यमयीं बुधः
सपर्यां विविधैर्द्रव्यैर्देशकालविभागवित्‌
सलिलैः शुचिभिर्माल्यैर्वन्यैर्मूलफलादिभिः
शस्ताङ्कुरांशुकैश्चार्चेत्तुलस्या प्रियया प्रभुम्‌
लब्ध्वा द्रव्यमयीमर्चां क्षित्यम्ब्वादिषु वार्चयेत्‌
आभृतात्मा मुनिः शान्तो यतवाङ्मितवन्यभुक्‌
स्वेच्छावतारचरितैरचिन्त्यनिजमायया
करिष्यत्युत्तमश्लोकस्तद्ध्यायेद्धृदयङ्गमम्‌
परिचर्या भगवतो यावत्यः पूर्वसेविताः
ता मन्त्रहृदयेनैव प्रयुञ्ज्यान्मन्त्रमूर्तये
एवं कायेन मनसा वचसा च मनोगतम्‌
परिचर्यमाणो भगवान्भक्तिमत्परिचर्यया
पुंसाममायिनां सम्यग्भजतां भाववर्धनः
श्रेयो दिशत्यभिमतं यद्धर्मादिषु देहिनाम्‌
विरक्तश्चेन्द्रियरतौ भक्तियोगेन भूयसा
तं निरन्तरभावेन भजेताद्धा विमुक्तये
इत्युक्तस्तं परिक्रम्य प्रणम्य च नृपार्भकः
ययौ मधुवनं पुण्यं हरेश्चरणचर्चितम्‌
तपोवनं गते तस्मिन्प्रविष्टोऽन्तःपुरं मुनिः
अर्हितार्हणको राज्ञा सुखासीन उवाच तम्‌
नारद उवाच
राजन्किं ध्यायसे दीर्घं मुखेन परिशुष्यता
किं वा न रिष्यते कामो धर्मो वार्थेन संयुतः
राजोवाच
सुतो मे बालको ब्रह्मन्स्त्रैणेनाकरुणात्मना
निर्वासितः पञ्चवर्षः सह मात्रा महान्कविः
अप्यनाथं वने ब्रह्मन्मा स्मादन्त्यर्भकं वृकाः
श्रान्तं शयानं क्षुधितं परिम्लानमुखाम्बुजम्‌
अहो मे बत दौरात्म्यं स्त्रीजितस्योपधारय
योऽङ्कं प्रेम्णारुरुक्षन्तं नाभ्यनन्दमसत्तमः
नारद उवाच
मा मा शुचः स्वतनयं देवगुप्तं विशाम्पते
तत्प्रभावमविज्ञाय प्रावृङ्क्ते यद्यशो जगत्‌
सुदुष्करं कर्म कृत्वा लोकपालैरपि प्रभुः
ऐष्यत्यचिरतो राजन्यशो विपुलयंस्तव
मैत्रेय उवाच
इति देवर्षिणा प्रोक्तं विश्रुत्य जगतीपतिः
राजलक्ष्मीमनादृत्य पुत्रमेवान्वचिन्तयत्‌
तत्राभिषिक्तः प्रयतस्तामुपोष्य विभावरीम्‌
समाहितः पर्यचरदृष्यादेशेन पूरुषम्‌
त्रिरात्रान्ते त्रिरात्रान्ते कपित्थबदराशनः
आत्मवृत्त्यनुसारेण मासं निन्येऽर्चयन्हरिम्‌
द्वितीयं च तथा मासं षष्ठे षष्ठेऽर्भको दिने
तृणपर्णादिभिः शीर्णैः कृतान्नोऽभ्यर्चयन्विभुम्‌
तृतीयं चानयन्मासं नवमे नवमेऽहनि
अब्भक्ष उत्तमश्लोकमुपाधावत्समाधिना
चतुर्थमपि वै मासं द्वादशे द्वादशेऽहनि
वायुभक्षो जितश्वासो ध्यायन्देवमधारयत्‌
पञ्चमे मास्यनुप्राप्ते जितश्वासो नृपात्मजः
ध्यायन्ब्रह्म पदैकेन तस्थौ स्थाणुरिवाचलः
सर्वतो मन आकृष्य हृदि भूतेन्द्रियाशयम्‌
ध्यायन्भगवतो रूपं नाद्राक्षीत्किञ्चनापरम्‌
आधारं महदादीनां प्रधानपुरुषेश्वरम्‌
ब्रह्म धारयमाणस्य त्रयो लोकाश्चकम्पिरे
यदैकपादेन स पार्थिवार्भकस्तस्थौ तदङ्गुष्ठनिपीडिता मही
ननाम तत्रार्धमिभेन्द्रधिष्ठिता तरीव सव्येतरतः पदे पदे
तस्मिन्नभिध्यायति विश्वमात्मनो द्वारं निरुध्यासुमनन्यया धिया
लोका निरुच्छ्वासनिपीडिता भृशं सलोकपालाः शरणं ययुर्हरिम्‌
देवा ऊचुः
नैवं विदामो भगवन्प्राणरोधं चराचरस्याखिलसत्त्वधाम्नः
विधेहि तन्नो वृजिनाद्विमोक्षं प्राप्ता वयं त्वां शरणं शरण्यम्‌
श्रीभगवानुवाच
मा भैष्ट बालं तपसो दुरत्ययान्निवर्तयिष्ये प्रतियात स्वधाम
यतो हि वः प्राणनिरोध आसीदौत्तानपादिर्मयि सङ्गतात्मा
मैत्रेय उवाच
त एवमुत्सन्नभया उरुक्रमे कृतावनामाः प्रययुस्त्रिविष्टपम्‌
सहस्रशीर्षापि ततो गरुत्मता मधोर्वनं भृत्यदिदृक्षया गतः
स वै धिया योगविपाकतीव्रया हृत्पद्मकोशे स्फुरितं तडित्प्रभम्‌
तिरोहितं सहसैवोपलक्ष्य बहिःस्थितं तदवस्थं ददर्श
तद्दर्शनेनागतसाध्वसः क्षिताववन्दताङ्गं विनमय्य दण्डवत्‌
दृग्भ्यां प्रपश्यन्प्रपिबन्निवार्भकश्चुम्बन्निवास्येन भुजैरिवाश्लिषन्‌
स तं विवक्षन्तमतद्विदं हरिर्ज्ञात्वास्य सर्वस्य च हृद्यवस्थितः
कृताञ्जलिं ब्रह्ममयेन कम्बुना पस्पर्श बालं कृपया कपोले
स वै तदैव प्रतिपादितां गिरं दैवीं परिज्ञातपरात्मनिर्णयः
तं भक्तिभावोऽभ्यगृणादसत्वरं परिश्रुतोरुश्रवसं ध्रुवक्षितिः
ध्रुव उवाच
योऽन्तः प्रविश्य मम वाचमिमां प्रसुप्तां
सञ्जीवयत्यखिलशक्तिधरः स्वधाम्ना
अन्यांश्च हस्तचरणश्रवणत्वगादीन्‌
प्राणान्नमो भगवते पुरुषाय तुभ्यम्‌
एकस्त्वमेव भगवन्निदमात्मशक्त्या
मायाख्ययोरुगुणया महदाद्यशेषम्‌
सृष्ट्वानुविश्य पुरुषस्तदसद्गुणेषु
नानेव दारुषु विभावसुवद्विभासि
त्वद्दत्तया वयुनयेदमचष्ट विश्वं
सुप्तप्रबुद्ध इव नाथ भवत्प्रपन्नः
तस्यापवर्ग्यशरणं तव पादमूलं
विस्मर्यते कृतविदा कथमार्तबन्धो
नूनं विमुष्टमतयस्तव मायया ते
ये त्वां भवाप्ययविमोक्षणमन्यहेतोः
अर्चन्ति कल्पकतरुं कुणपोपभोग्यम्‌
इच्छन्ति यत्स्पर्शजं निरयेऽपि न्णाम्‌
या निर्वृतिस्तनुभृतां तव पादपद्म
ध्यानाद्भवज्जनकथाश्रवणेन वा स्यात्‌
सा ब्रह्मणि स्वमहिमन्यपि नाथ मा भूत्‌
किं त्वन्तकासिलुलितात्पततां विमानात्‌
भक्तिं मुहुः प्रवहतां त्वयि मे प्रसङ्गो
भूयादनन्त महताममलाशयानाम्‌
येनाञ्जसोल्बणमुरुव्यसनं भवाब्धिं
नेष्ये भवद्गुणकथामृतपानमत्तः
ते न स्मरन्त्यतितरां प्रियमीश मर्त्यं
ये चान्वदः सुतसुहृद्गृहवित्तदाराः
ये त्वब्जनाभ भवदीयपदारविन्द
सौगन्ध्यलुब्धहृदयेषु कृतप्रसङ्गाः
तिर्यङ्नगद्विजसरीसृपदेवदैत्य
मर्त्यादिभिः परिचितं सदसद्विशेषम्‌
रूपं स्थविष्ठमज ते महदाद्यनेकं
नातः परं परम वेद्मि न यत्र वादः
कल्पान्त एतदखिलं जठरेण गृह्णन्‌
शेते पुमान्स्वदृगनन्तसखस्तदङ्के
यन्नाभिसिन्धुरुहकाञ्चनलोकपद्म
गर्भे द्युमान्भगवते प्रणतोऽस्मि तस्मै
त्वं नित्यमुक्तपरिशुद्धविबुद्ध आत्मा
कूटस्थ आदिपुरुषो भगवांस्त्र्यधीशः
यद्बुद्ध्यवस्थितिमखण्डितया स्वदृष्ट्या
द्रष्टा स्थितावधिमखो व्यतिरिक्त आस्से
यस्मिन्विरुद्धगतयो ह्यनिशं पतन्ति
विद्यादयो विविधशक्तय आनुपूर्व्यात्‌
तद्ब्रह्म विश्वभवमेकमनन्तमाद्यम्‌
आनन्दमात्रमविकारमहं प्रपद्ये
सत्याशिषो हि भगवंस्तव पादपद्मम्‌
आशीस्तथानुभजतः पुरुषार्थमूर्तेः
अप्येवमर्य भगवान्परिपाति दीनान्‌
वाश्रेव वत्सकमनुग्रहकातरोऽस्मान्‌
मैत्रेय उवाच
अथाभिष्टुत एवं वै सत्सङ्कल्पेन धीमता
भृत्यानुरक्तो भगवान्प्रतिनन्द्येदमब्रवीत्‌
श्रीभगवानुवाच
वेदाहं ते व्यवसितं हृदि राजन्यबालक
तत्प्रयच्छामि भद्रं ते दुरापमपि सुव्रत
नान्यैरधिष्ठितं भद्र यद्भ्राजिष्णु ध्रुवक्षिति
यत्र ग्रहर्क्षताराणां ज्योतिषां चक्रमाहितम्‌
मेढ्यां गोचक्रवत्स्थास्नु परस्तात्कल्पवासिनाम्‌
धर्मोऽग्निः कश्यपः शुक्रो मुनयो ये वनौकसः
चरन्ति दक्षिणीकृत्य भ्रमन्तो यत्सतारकाः
प्रस्थिते तु वनं पित्रा दत्त्वा गां धर्मसंश्रयः
षट्त्रिंशद्वर्षसाहस्रं रक्षिताव्याहतेन्द्रियः
त्वद्भ्रातर्युत्तमे नष्टे मृगयायां तु तन्मनाः
अन्वेषन्ती वनं माता दावाग्निं सा प्रवेक्ष्यति
इष्ट्वा मां यज्ञहृदयं यज्ञैः पुष्कलदक्षिणैः
भुक्त्वा चेहाशिषः सत्या अन्ते मां संस्मरिष्यसि
ततो गन्तासि मत्स्थानं सर्वलोकनमस्कृतम्‌
उपरिष्टादृषिभ्यस्त्वं यतो नावर्तते गतः
मैत्रेय उवाच
इत्यर्चितः स भगवानतिदिश्यात्मनः पदम्‌
बालस्य पश्यतो धाम स्वमगाद्गरुडध्वजः
सोऽपि सङ्कल्पजं विष्णोः पादसेवोपसादितम्‌
प्राप्य सङ्कल्पनिर्वाणं नातिप्रीतोऽभ्यगात्पुरम्‌
विदुर उवाच
सुदुर्लभं यत्परमं पदं हरेर्मायाविनस्तच्चरणार्चनार्जितम्‌
लब्ध्वाप्यसिद्धार्थमिवैकजन्मना कथं स्वमात्मानममन्यतार्थवित्‌
मैत्रेय उवाच
मातुः सपत्न्या वाग्बाणैर्हृदि विद्धस्तु तान्स्मरन्‌
नैच्छन्मुक्तिपतेर्मुक्तिं तस्मात्तापमुपेयिवान्‌
ध्रुव उवाच
समाधिना नैकभवेन यत्पदं विदुः सनन्दादय ऊर्ध्वरेतसः
मासैरहं षड्भिरमुष्य पादयोश्छायामुपेत्यापगतः पृथङ्मतिः
अहो बत ममानात्म्यं मन्दभाग्यस्य पश्यत
भवच्छिदः पादमूलं गत्वा याचे यदन्तवत्‌
मतिर्विदूषिता देवैः पतद्भिरसहिष्णुभिः
यो नारदवचस्तथ्यं नाग्राहिषमसत्तमः
दैवीं मायामुपाश्रित्य प्रसुप्त इव भिन्नदृक्‌
तप्ये द्वितीयेऽप्यसति भ्रातृभ्रातृव्यहृद्रुजा
मयैतत्प्रार्थितं व्यर्थं चिकित्सेव गतायुषि
प्रसाद्य जगदात्मानं तपसा दुष्प्रसादनम्‌
भवच्छिदमयाचेऽहं भवं भाग्यविवर्जितः
स्वाराज्यं यच्छतो मौढ्यान्मानो मे भिक्षितो बत
ईश्वरात्क्षीणपुण्येन फलीकारानिवाधनः
मैत्रेय उवाच
न वै मुकुन्दस्य पदारविन्दयो रजोजुषस्तात भवादृशा जनाः
वाञ्छन्ति तद्दास्यमृतेऽर्थमात्मनो यदृच्छया लब्धमनःसमृद्धयः
आकर्ण्यात्मजमायान्तं सम्परेत्य यथागतम्‌
राजा न श्रद्दधे भद्रमभद्रस्य कुतो मम
श्रद्धाय वाक्यं देवर्षेर्हर्षवेगेन धर्षितः
वार्ताहर्तुरतिप्रीतो हारं प्रादान्महाधनम्‌
सदश्वं रथमारुह्य कार्तस्वरपरिष्कृतम्‌
ब्राह्मणैः कुलवृद्धैश्च पर्यस्तोऽमात्यबन्धुभिः
शङ्खदुन्दुभिनादेन ब्रह्मघोषेण वेणुभिः
निश्चक्राम पुरात्तूर्णमात्मजाभीक्षणोत्सुकः
सुनीतिः सुरुचिश्चास्य महिष्यौ रुक्मभूषिते
आरुह्य शिबिकां सार्धमुत्तमेनाभिजग्मतुः
तं दृष्ट्वोपवनाभ्याश आयान्तं तरसा रथात्‌
अवरुह्य नृपस्तूर्णमासाद्य प्रेमविह्वलः
परिरेभेऽङ्गजं दोर्भ्यां दीर्घोत्कण्ठमनाः श्वसन्‌
विष्वक्सेनाङ्घ्रिसंस्पर्श हताशेषाघबन्धनम्‌
अथाजिघ्रन्मुहुर्मूर्ध्नि शीतैर्नयनवारिभिः
स्नापयामास तनयं जातोद्दाममनोरथः
अभिवन्द्य पितुः पादावाशीर्भिश्चाभिमन्त्रितः
ननाम मातरौ शीर्ष्णा सत्कृतः सज्जनाग्रणीः
सुरुचिस्तं समुत्थाप्य पादावनतमर्भकम्‌
परिष्वज्याह जीवेति बाष्पगद्गदया गिरा
यस्य प्रसन्नो भगवान्गुणैर्मैत्र्यादिभिर्हरिः
तस्मै नमन्ति भूतानि निम्नमाप इव स्वयम्‌
उत्तमश्च ध्रुवश्चोभावन्योन्यं प्रेमविह्वलौ
अङ्गसङ्गादुत्पुलकावस्रौघं मुहुरूहतुः
सुनीतिरस्य जननी प्राणेभ्योऽपि प्रियं सुतम्‌
उपगुह्य जहावाधिं तदङ्गस्पर्शनिर्वृता
पयः स्तनाभ्यां सुस्राव नेत्रजैः सलिलैः शिवैः
तदाभिषिच्यमानाभ्यां वीर वीरसुवो मुहुः
तां शशंसुर्जना राज्ञीं दिष्ट्या ते पुत्र आर्तिहा
प्रतिलब्धश्चिरं नष्टो रक्षिता मण्डलं भुवः
अभ्यर्चितस्त्वया नूनं भगवान्प्रणतार्तिहा
यदनुध्यायिनो धीरा मृत्युं जिग्युः सुदुर्जयम्‌
लाल्यमानं जनैरेवं ध्रुवं सभ्रातरं नृपः
आरोप्य करिणीं हृष्टः स्तूयमानोऽविशत्पुरम्‌
तत्र तत्रोपसङ्कॢप्तैर्लसन्मकरतोरणैः
सवृन्दैः कदलीस्तम्भैः पूगपोतैश्च तद्विधैः
चूतपल्लववासःस्रङ्‌ मुक्तादामविलम्बिभिः
उपस्कृतं प्रतिद्वारमपां कुम्भैः सदीपकैः
प्राकारैर्गोपुरागारैः शातकुम्भपरिच्छदैः
सर्वतोऽलङ्कृतं श्रीमद्‌ विमानशिखरद्युभिः
मृष्टचत्वररथ्याट्ट मार्गं चन्दनचर्चितम्‌
लाजाक्षतैः पुष्पफलैस्तण्डुलैर्बलिभिर्युतम्‌
ध्रुवाय पथि दृष्टाय तत्र तत्र पुरस्त्रियः
सिद्धार्थाक्षतदध्यम्बु दूर्वापुष्पफलानि च
उपजह्रुः प्रयुञ्जाना वात्सल्यादाशिषः सतीः
शृण्वंस्तद्वल्गुगीतानि प्राविशद्भवनं पितुः
महामणिव्रातमये स तस्मिन्भवनोत्तमे
लालितो नितरां पित्रा न्यवसद्दिवि देववत्‌
पयःफेननिभाः शय्या दान्ता रुक्मपरिच्छदाः
आसनानि महार्हाणि यत्र रौक्मा उपस्कराः
यत्र स्फटिककुड्येषु महामारकतेषु च
मणिप्रदीपा आभान्ति ललनारत्नसंयुताः
उद्यानानि च रम्याणि विचित्रैरमरद्रुमैः
कूजद्विहङ्गमिथुनैर्गायन्मत्तमधुव्रतैः
वाप्यो वैदूर्यसोपानाः पद्मोत्पलकुमुद्वतीः
हंसकारण्डवकुलैर्जुष्टाश्चक्राह्वसारसैः
उत्तानपादो राजर्षिः प्रभावं तनयस्य तम्‌
श्रुत्वा दृष्ट्वाद्भुततमं प्रपेदे विस्मयं परम्‌
वीक्ष्योढवयसं तं च प्रकृतीनां च सम्मतम्‌
अनुरक्तप्रजं राजा ध्रुवं चक्रे भुवः पतिम्‌
आत्मानं च प्रवयसमाकलय्य विशाम्पतिः
वनं विरक्तः प्रातिष्ठद्विमृशन्नात्मनो गतिम्‌
मैत्रेय उवाच
प्रजापतेर्दुहितरं शिशुमारस्य वै ध्रुवः
उपयेमे भ्रमिं नाम तत्सुतौ कल्पवत्सरौ
इलायामपि भार्यायां वायोः पुत्र्यां महाबलः
पुत्रमुत्कलनामानं योषिद्रत्नमजीजनत्‌
उत्तमस्त्वकृतोद्वाहो मृगयायां बलीयसा
हतः पुण्यजनेनाद्रौ तन्मातास्य गतिं गता
ध्रुवो भ्रातृवधं श्रुत्वा कोपामर्षशुचार्पितः
जैत्रं स्यन्दनमास्थाय गतः पुण्यजनालयम्‌
गत्वोदीचीं दिशं राजा रुद्रानुचरसेविताम्‌
ददर्श हिमवद्द्रोण्यां पुरीं गुह्यकसङ्कुलाम्‌
दध्मौ शङ्खं बृहद्बाहुः खं दिशश्चानुनादयन्‌
येनोद्विग्नदृशः क्षत्तरुपदेव्योऽत्रसन्भृशम्‌
ततो निष्क्रम्य बलिन उपदेवमहाभटाः
असहन्तस्तन्निनादमभिपेतुरुदायुधाः
स तानापततो वीर उग्रधन्वा महारथः
एकैकं युगपत्सर्वानहन्बाणैस्त्रिभिस्त्रिभिः
ते वै ललाटलग्नैस्तैरिषुभिः सर्व एव हि
मत्वा निरस्तमात्मानमाशंसन्कर्म तस्य तत्‌
तेऽपि चामुममृष्यन्तः पादस्पर्शमिवोरगाः
शरैरविध्यन्युगपद्द्विगुणं प्रचिकीर्षवः
ततः परिघनिस्त्रिंशैः प्रासशूलपरश्वधैः
शक्त्यृष्टिभिर्भुशुण्डीभिश्चित्रवाजैः शरैरपि
अभ्यवर्षन्प्रकुपिताः सरथं सहसारथिम्‌
इच्छन्तस्तत्प्रतीकर्तुमयुतानां त्रयोदश
औत्तानपादिः स तदा शस्त्रवर्षेण भूरिणा
न एवादृश्यताच्छन्न आसारेण यथा गिरिः
हाहाकारस्तदैवासीत्सिद्धानां दिवि पश्यताम्‌
हतोऽयं मानवः सूर्यो मग्नः पुण्यजनार्णवे
नदत्सु यातुधानेषु जयकाशिष्वथो मृधे
उदतिष्ठद्रथस्तस्य नीहारादिव भास्करः
धनुर्विस्फूर्जयन्दिव्यं द्विषतां खेदमुद्वहन्‌
अस्त्रौघं व्यधमद्बाणैर्घनानीकमिवानिलः
तस्य ते चापनिर्मुक्ता भित्त्वा वर्माणि रक्षसाम्‌
कायानाविविशुस्तिग्मा गिरीनशनयो यथा
भल्लैः सञ्छिद्यमानानां शिरोभिश्चारुकुण्डलैः
ऊरुभिर्हेमतालाभैर्दोर्भिर्वलयवल्गुभिः
हारकेयूरमुकुटैरुष्णीषैश्च महाधनैः
आस्तृतास्ता रणभुवो रेजुर्वीरमनोहराः
हतावशिष्टा इतरे रणाजिराद्रक्षोगणाः क्षत्रियवर्यसायकैः
प्रायो विवृक्णावयवा विदुद्रुवुर्मृगेन्द्रविक्रीडितयूथपा इव
अपश्यमानः स तदाततायिनं महामृधे कञ्चन मानवोत्तमः
पुरीं दिदृक्षन्नपि नाविशद्द्विषां न मायिनां वेद चिकीर्षितं जनः
इति ब्रुवंश्चित्ररथः स्वसारथिं यत्तः परेषां प्रतियोगशङ्कितः
शुश्राव शब्दं जलधेरिवेरितं नभस्वतो दिक्षु रजोऽन्वदृश्यत
क्षणेनाच्छादितं व्योम घनानीकेन सर्वतः
विस्फुरत्तडिता दिक्षु त्रासयत्स्तनयित्नुना
ववृषू रुधिरौघासृक्‌ पूयविण्मूत्रमेदसः
निपेतुर्गगनादस्य कबन्धान्यग्रतोऽनघ
ततः खेऽदृश्यत गिरिर्निपेतुः सर्वतोदिशम्‌
गदापरिघनिस्त्रिंश मुसलाः साश्मवर्षिणः
अहयोऽशनिनिःश्वासा वमन्तोऽग्निं रुषाक्षिभिः
अभ्यधावन्गजा मत्ताः सिंहव्याघ्राश्च यूथशः
समुद्र ऊर्मिभिर्भीमः प्लावयन्सर्वतो भुवम्‌
आससाद महाह्रादः कल्पान्त इव भीषणः
एवंविधान्यनेकानि त्रासनान्यमनस्विनाम्‌
ससृजुस्तिग्मगतय आसुर्या माययासुराः
ध्रुवे प्रयुक्तामसुरैस्तां मायामतिदुस्तराम्‌
निशम्य तस्य मुनयः शमाशंसन्समागताः
मुनय ऊचुः
औत्तानपाद भगवांस्तव शार्ङ्गधन्वा
देवः क्षिणोत्ववनतार्तिहरो विपक्षान्‌
यन्नामधेयमभिधाय निशम्य चाद्धा
लोकोऽञ्जसा तरति दुस्तरमङ्ग मृत्युम्‌
मैत्रेय उवाच
निशम्य गदतामेवमृषीणां धनुषि ध्रुवः
सन्दधेऽस्त्रमुपस्पृश्य यन्नारायणनिर्मितम्‌
सन्धीयमान एतस्मिन्माया गुह्यकनिर्मिताः
क्षिप्रं विनेशुर्विदुर क्लेशा ज्ञानोदये यथा
तस्यार्षास्त्रं धनुषि प्रयुञ्जतः सुवर्णपुङ्खाः कलहंसवाससः
विनिःसृता आविविशुर्द्विषद्बलं यथा वनं भीमरवाः शिखण्डिनः
तैस्तिग्मधारैः प्रधने शिलीमुखैरितस्ततः पुण्यजना उपद्रुताः
तमभ्यधावन्कुपिता उदायुधाः सुपर्णमुन्नद्धफणा इवाहयः
स तान्पृषत्कैरभिधावतो मृधे निकृत्तबाहूरुशिरोधरोदरान्‌
निनाय लोकं परमर्कमण्डलं व्रजन्ति निर्भिद्य यमूर्ध्वरेतसः
तान्हन्यमानानभिवीक्ष्य गुह्यकाननागसश्चित्ररथेन भूरिशः
औत्तानपादिं कृपया पितामहो मनुर्जगादोपगतः सहर्षिभिः
मनुरुवाच
अलं वत्सातिरोषेण तमोद्वारेण पाप्मना
येन पुण्यजनानेतानवधीस्त्वमनागसः
नास्मत्कुलोचितं तात कर्मैतत्सद्विगर्हितम्‌
वधो यदुपदेवानामारब्धस्तेऽकृतैनसाम्‌
नन्वेकस्यापराधेन प्रसङ्गाद्बहवो हताः
भ्रातुर्वधाभितप्तेन त्वयाङ्ग भ्रातृवत्सल
नायं मार्गो हि साधूनां हृषीकेशानुवर्तिनाम्‌
यदात्मानं पराग्गृह्य पशुवद्भूतवैशसम्‌
सर्वभूतात्मभावेन भूतावासं हरिं भवान्‌
आराध्याप दुराराध्यं विष्णोस्तत्परमं पदम्‌
स त्वं हरेरनुध्यातस्तत्पुंसामपि सम्मतः
कथं त्ववद्यं कृतवाननुशिक्षन्सतां व्रतम्‌
तितिक्षया करुणया मैत्र्या चाखिलजन्तुषु
समत्वेन च सर्वात्मा भगवान्सम्प्रसीदति
सम्प्रसन्ने भगवति पुरुषः प्राकृतैर्गुणैः
विमुक्तो जीवनिर्मुक्तो ब्रह्म निर्वाणमृच्छति
भूतैः पञ्चभिरारब्धैर्योषित्पुरुष एव हि
तयोर्व्यवायात्सम्भूतिर्योषित्पुरुषयोरिह
एवं प्रवर्तते सर्गः स्थितिः संयम एव च
गुणव्यतिकराद्राजन्मायया परमात्मनः
निमित्तमात्रं तत्रासीन्निर्गुणः पुरुषर्षभः
व्यक्ताव्यक्तमिदं विश्वं यत्र भ्रमति लोहवत्‌
स खल्विदं भगवान्कालशक्त्या गुणप्रवाहेण विभक्तवीर्यः
करोत्यकर्तैव निहन्त्यहन्ता चेष्टा विभूम्नः खलु दुर्विभाव्या
सोऽनन्तोऽन्तकरः कालोऽनादिरादिकृदव्ययः
जनं जनेन जनयन्मारयन्मृत्युनान्तकम्‌
न वै स्वपक्षोऽस्य विपक्ष एव वा परस्य मृत्योर्विशतः समं प्रजाः
तं धावमानमनुधावन्त्यनीशा यथा रजांस्यनिलं भूतसङ्घाः
आयुषोऽपचयं जन्तोस्तथैवोपचयं विभुः
उभाभ्यां रहितः स्वस्थो दुःस्थस्य विदधात्यसौ
केचित्कर्म वदन्त्येनं स्वभावमपरे नृप
एके कालं परे दैवं पुंसः काममुतापरे
अव्यक्तस्याप्रमेयस्य नानाशक्त्युदयस्य च
न वै चिकीर्षितं तात को वेदाथ स्वसम्भवम्‌
न चैते पुत्रक भ्रातुर्हन्तारो धनदानुगाः
विसर्गादानयोस्तात पुंसो दैवं हि कारणम्‌
स एव विश्वं सृजति स एवावति हन्ति च
अथापि ह्यनहङ्कारान्नाज्यते गुणकर्मभिः
एष भूतानि भूतात्मा भूतेशो भूतभावनः
स्वशक्त्या मायया युक्तः सृजत्यत्ति च पाति च
तमेव मृत्युममृतं तात दैवं सर्वात्मनोपेहि जगत्परायणम्‌
यस्मै बलिं विश्वसृजो हरन्ति गावो यथा वै नसि दामयन्त्रिताः
यः पञ्चवर्षो जननीं त्वं विहाय मातुः सपत्न्या वचसा भिन्नमर्मा
वनं गतस्तपसा प्रत्यगक्षमाराध्य लेभे मूर्ध्नि पदं त्रिलोक्याः
तमेनमङ्गात्मनि मुक्तविग्रहे व्यपाश्रितं निर्गुणमेकमक्षरम्‌
आत्मानमन्विच्छ विमुक्तमात्मदृग्यस्मिन्निदं भेदमसत्प्रतीयते
त्वं प्रत्यगात्मनि तदा भगवत्यनन्त आनन्दमात्र उपपन्नसमस्तशक्तौ
भक्तिं विधाय परमां शनकैरविद्या ग्रन्थिं विभेत्स्यसि ममाहमिति प्ररूढम्‌
संयच्छ रोषं भद्रं ते प्रतीपं श्रेयसां परम्‌
श्रुतेन भूयसा राजन्नगदेन यथामयम्‌
येनोपसृष्टात्पुरुषाल्लोक उद्विजते भृशम्‌
न बुधस्तद्वशं गच्छेदिच्छन्नभयमात्मनः
हेलनं गिरिशभ्रातुर्धनदस्य त्वया कृतम्‌
यज्जघ्निवान्पुण्यजनान्भ्रातृघ्नानित्यमर्षितः
तं प्रसादय वत्साशु सन्नत्या प्रश्रयोक्तिभिः
न यावन्महतां तेजः कुलं नोऽभिभविष्यति
एवं स्वायम्भुवः पौत्रमनुशास्य मनुर्ध्रुवम्‌
तेनाभिवन्दितः साकमृषिभिः स्वपुरं ययौ
मैत्रेय उवाच
ध्रुवं निवृत्तं प्रतिबुद्ध्य वैशसादपेतमन्युं भगवान्धनेश्वरः
तत्रागतश्चारणयक्षकिन्नरैः संस्तूयमानो न्यवदत्कृताञ्जलिम्‌
धनद उवाच
भो भोः क्षत्रियदायाद परितुष्टोऽस्मि तेऽनघ
यत्त्वं पितामहादेशाद्वैरं दुस्त्यजमत्यजः
न भवानवधीद्यक्षान्न यक्षा भ्रातरं तव
काल एव हि भूतानां प्रभुरप्ययभावयोः
अहं त्वमित्यपार्था धीरज्ञानात्पुरुषस्य हि
स्वाप्नीवाभात्यतद्ध्यानाद्यया बन्धविपर्ययौ
तद्गच्छ ध्रुव भद्रं ते भगवन्तमधोक्षजम्‌
सर्वभूतात्मभावेन सर्वभूतात्मविग्रहम्‌
भजस्व भजनीयाङ्घ्रिमभवाय भवच्छिदम्‌
युक्तं विरहितं शक्त्या गुणमय्यात्ममायया
वृणीहि कामं नृप यन्मनोगतं मत्तस्त्वमौत्तानपदेऽविशङ्कितः
वरं वरार्होऽम्बुजनाभपादयोरनन्तरं त्वां वयमङ्ग शुश्रुम
मैत्रेय उवाच
स राजराजेन वराय चोदितो ध्रुवो महाभागवतो महामतिः
हरौ स वव्रेऽचलितां स्मृतिं यया तरत्ययत्नेन दुरत्ययं तमः
तस्य प्रीतेन मनसा तां दत्त्वैडविडस्ततः
पश्यतोऽन्तर्दधे सोऽपि स्वपुरं प्रत्यपद्यत
अथायजत यज्ञेशं क्रतुभिर्भूरिदक्षिणैः
द्रव्यक्रियादेवतानां कर्म कर्मफलप्रदम्‌
सर्वात्मन्यच्युतेऽसर्वे तीव्रौघां भक्तिमुद्वहन्‌
ददर्शात्मनि भूतेषु तमेवावस्थितं विभुम्‌
तमेवं शीलसम्पन्नं ब्रह्मण्यं दीनवत्सलम्‌
गोप्तारं धर्मसेतूनां मेनिरे पितरं प्रजाः
षट्त्रिंशद्वर्षसाहस्रं शशास क्षितिमण्डलम्‌
भोगैः पुण्यक्षयं कुर्वन्नभोगैरशुभक्षयम्‌
एवं बहुसवं कालं महात्माविचलेन्द्रियः
त्रिवर्गौपयिकं नीत्वा पुत्रायादान्नृपासनम्‌
मन्यमान इदं विश्वं मायारचितमात्मनि
अविद्यारचितस्वप्नगन्धर्वनगरोपमम्‌
आत्मस्त्र्यपत्यसुहृदो बलमृद्धकोशम्‌
अन्तःपुरं परिविहारभुवश्च रम्याः
भूमण्डलं जलधिमेखलमाकलय्य
कालोपसृष्टमिति स प्रययौ विशालाम्‌
तस्यां विशुद्धकरणः शिववार्विगाह्य
बद्ध्वासनं जितमरुन्मनसाहृताक्षः
स्थूले दधार भगवत्प्रतिरूप एतद्‌
ध्यायंस्तदव्यवहितो व्यसृजत्समाधौ
भक्तिं हरौ भगवति प्रवहन्नजस्रम्‌
आनन्दबाष्पकलया मुहुरर्द्यमानः
विक्लिद्यमानहृदयः पुलकाचिताङ्गो
नात्मानमस्मरदसाविति मुक्तलिङ्गः
स ददर्श विमानाग्र्यं नभसोऽवतरद्ध्रुवः
विभ्राजयद्दश दिशो राकापतिमिवोदितम्‌
तत्रानु देवप्रवरौ चतुर्भुजौ
श्यामौ किशोरावरुणाम्बुजेक्षणौ
स्थिताववष्टभ्य गदां सुवाससौ
किरीटहाराङ्गदचारुकुण्डलौ
विज्ञाय तावुत्तमगायकिङ्कराव्‌
अभ्युत्थितः साध्वसविस्मृतक्रमः
ननाम नामानि गृणन्मधुद्विषः
पार्षत्प्रधानाविति संहताञ्जलिः
तं कृष्णपादाभिनिविष्टचेतसं
बद्धाञ्जलिं प्रश्रयनम्रकन्धरम्‌
सुनन्दनन्दावुपसृत्य सस्मितं
प्रत्यूचतुः पुष्करनाभसम्मतौ
सुनन्दनन्दावूचतुः
भो भो राजन्सुभद्रं ते वाचं नोऽवहितः शृणु
यः पञ्चवर्षस्तपसा भवान्देवमतीतृपत्‌
तस्याखिलजगद्धातुरावां देवस्य शार्ङ्गिणः
पार्षदाविह सम्प्राप्तौ नेतुं त्वां भगवत्पदम्‌
सुदुर्जयं विष्णुपदं जितं त्वया यत्सूरयोऽप्राप्य विचक्षते परम्‌
आतिष्ठ तच्चन्द्रदिवाकरादयो ग्रहर्क्षताराः परियन्ति दक्षिणम्‌
अनास्थितं ते पितृभिरन्यैरप्यङ्ग कर्हिचित्‌
आतिष्ठ जगतां वन्द्यं तद्विष्णोः परमं पदम्‌
एतद्विमानप्रवरमुत्तमश्लोकमौलिना
उपस्थापितमायुष्मन्नधिरोढुं त्वमर्हसि
मैत्रेय उवाच
निशम्य वैकुण्ठनियोज्यमुख्ययोर्मधुच्युतं वाचमुरुक्रमप्रियः
कृताभिषेकः कृतनित्यमङ्गलो मुनीन्प्रणम्याशिषमभ्यवादयत्‌
परीत्याभ्यर्च्य धिष्ण्याग्र्यं पार्षदावभिवन्द्य च
इयेष तदधिष्ठातुं बिभ्रद्रूपं हिरण्मयम्‌
तदोत्तानपदः पुत्रो ददर्शान्तकमागतम्‌
मृत्योर्मूर्ध्नि पदं दत्त्वा आरुरोहाद्भुतं गृहम्‌
तदा दुन्दुभयो नेदुर्मृदङ्गपणवादयः
गन्धर्वमुख्याः प्रजगुः पेतुः कुसुमवृष्टयः
स च स्वर्लोकमारोक्ष्यन्सुनीतिं जननीं ध्रुवः
अन्वस्मरदगं हित्वा दीनां यास्ये त्रिविष्टपम्‌
इति व्यवसितं तस्य व्यवसाय सुरोत्तमौ
दर्शयामासतुर्देवीं पुरो यानेन गच्छतीम्‌
तत्र तत्र प्रशंसद्भिः पथि वैमानिकैः सुरैः
अवकीर्यमाणो ददृशे कुसुमैः क्रमशो ग्रहान्‌
त्रिलोकीं देवयानेन सोऽतिव्रज्य मुनीनपि
परस्ताद्यद्ध्रुवगतिर्विष्णोः पदमथाभ्यगात्‌
यद्भ्राजमानं स्वरुचैव सर्वतो लोकास्त्रयो ह्यनु विभ्राजन्त एते
यन्नाव्रजन्जन्तुषु येऽननुग्रहा व्रजन्ति भद्राणि चरन्ति येऽनिशम्‌
शान्ताः समदृशः शुद्धाः सर्वभूतानुरञ्जनाः
यान्त्यञ्जसाच्युतपदमच्युतप्रियबान्धवाः
इत्युत्तानपदः पुत्रो ध्रुवः कृष्णपरायणः
अभूत्त्रयाणां लोकानां चूडामणिरिवामलः
गम्भीरवेगोऽनिमिषं ज्योतिषां चक्रमाहितम्‌
यस्मिन्भ्रमति कौरव्य मेढ्यामिव गवां गणः
महिमानं विलोक्यास्य नारदो भगवानृषिः
आतोद्यं वितुदञ्श्लोकान्सत्रेऽगायत्प्रचेतसाम्‌
नारद उवाच
नूनं सुनीतेः पतिदेवतायास्तपःप्रभावस्य सुतस्य तां गतिम्‌
दृष्ट्वाभ्युपायानपि वेदवादिनो नैवाधिगन्तुं प्रभवन्ति किं नृपाः
यः पञ्चवर्षो गुरुदारवाक्षरैर्भिन्नेन यातो हृदयेन दूयता
वनं मदादेशकरोऽजितं प्रभुं जिगाय तद्भक्तगुणैः पराजितम्‌
यः क्षत्रबन्धुर्भुवि तस्याधिरूढमन्वारुरुक्षेदपि वर्षपूगैः
षट्पञ्चवर्षो यदहोभिरल्पैः प्रसाद्य वैकुण्ठमवाप तत्पदम्‌
मैत्रेय उवाच
एतत्तेऽभिहितं सर्वं यत्पृष्टोऽहमिह त्वया
ध्रुवस्योद्दामयशसश्चरितं सम्मतं सताम्‌
धन्यं यशस्यमायुष्यं पुण्यं स्वस्त्ययनं महत्‌
स्वर्ग्यं ध्रौव्यं सौमनस्यं प्रशस्यमघमर्षणम्‌
श्रुत्वैतच्छ्रद्धयाभीक्ष्णमच्युतप्रियचेष्टितम्‌
भवेद्भक्तिर्भगवति यया स्यात्क्लेशसङ्क्षयः
महत्त्वमिच्छतां तीर्थं श्रोतुः शीलादयो गुणाः
यत्र तेजस्तदिच्छूनां मानो यत्र मनस्विनाम्‌
प्रयतः कीर्तयेत्प्रातः समवाये द्विजन्मनाम्‌
सायं च पुण्यश्लोकस्य ध्रुवस्य चरितं महत्‌
पौर्णमास्यां सिनीवाल्यां द्वादश्यां श्रवणेऽथवा
दिनक्षये व्यतीपाते सङ्क्रमेऽर्कदिनेऽपि वा
श्रावयेच्छ्रद्दधानानां तीर्थपादपदाश्रयः
नेच्छंस्तत्रात्मनात्मानं सन्तुष्ट इति सिध्यति
ज्ञानमज्ञाततत्त्वाय यो दद्यात्सत्पथेऽमृतम्‌
कृपालोर्दीननाथस्य देवास्तस्यानुगृह्णते
इदं मया तेऽभिहितं कुरूद्वह ध्रुवस्य विख्यातविशुद्धकर्मणः
हित्वार्भकः क्रीडनकानि मातुर्गृहं च विष्णुं शरणं यो जगाम
सूत उवाच
निशम्य कौषारविणोपवर्णितं ध्रुवस्य वैकुण्ठपदाधिरोहणम्‌
प्ररूढभावो भगवत्यधोक्षजे प्रष्टुं पुनस्तं विदुरः प्रचक्रमे
विदुर उवाच
के ते प्रचेतसो नाम कस्यापत्यानि सुव्रत
कस्यान्ववाये प्रख्याताः कुत्र वा सत्रमासत
मन्ये महाभागवतं नारदं देवदर्शनम्‌
येन प्रोक्तः क्रियायोगः परिचर्याविधिर्हरेः
स्वधर्मशीलैः पुरुषैर्भगवान्यज्ञपूरुषः
इज्यमानो भक्तिमता नारदेनेरितः किल
यास्ता देवर्षिणा तत्र वर्णिता भगवत्कथाः
मह्यं शुश्रूषवे ब्रह्मन्कार्त्स्न्येनाचष्टुमर्हसि
मैत्रेय उवाच
ध्रुवस्य चोत्कलः पुत्रः पितरि प्रस्थिते वनम्‌
सार्वभौमश्रियं नैच्छदधिराजासनं पितुः
स जन्मनोपशान्तात्मा निःसङ्गः समदर्शनः
ददर्श लोके विततमात्मानं लोकमात्मनि
आत्मानं ब्रह्म निर्वाणं प्रत्यस्तमितविग्रहम्‌
अवबोधरसैकात्म्यमानन्दमनुसन्ततम्‌
अव्यवच्छिन्नयोगाग्नि दग्धकर्ममलाशयः
स्वरूपमवरुन्धानो नात्मनोऽन्यं तदैक्षत
जडान्धबधिरोन्मत्त मूकाकृतिरतन्मतिः
लक्षितः पथि बालानां प्रशान्तार्चिरिवानलः
मत्वा तं जडमुन्मत्तं कुलवृद्धाः समन्त्रिणः
वत्सरं भूपतिं चक्रुर्यवीयांसं भ्रमेः सुतम्‌
स्वर्वीथिर्वत्सरस्येष्टा भार्यासूत षडात्मजान्‌
पुष्पार्णं तिग्मकेतुं च इषमूर्जं वसुं जयम्‌
पुष्पार्णस्य प्रभा भार्या दोषा च द्वे बभूवतुः
प्रातर्मध्यन्दिनं सायमिति ह्यासन्प्रभासुताः
प्रदोषो निशिथो व्युष्ट इति दोषासुतास्त्रयः
व्युष्टः सुतं पुष्करिण्यां सर्वतेजसमादधे
स चक्षुः सुतमाकूत्यां पत्न्यां मनुमवाप ह
मनोरसूत महिषी विरजान्नड्वला सुतान्‌
पुरुं कुत्सं त्रितं द्युम्नं सत्यवन्तमृतं व्रतम्‌
अग्निष्टोममतीरात्रं प्रद्युम्नं शिबिमुल्मुकम्‌
उल्मुकोऽजनयत्पुत्रान्पुष्करिण्यां षडुत्तमान्‌
अङ्गं सुमनसं ख्यातिं क्रतुमङ्गिरसं गयम्‌
सुनीथाङ्गस्य या पत्नी सुषुवे वेनमुल्बणम्‌
यद्दौःशील्यात्स राजर्षिर्निर्विण्णो निरगात्पुरात्‌
यमङ्ग शेपुः कुपिता वाग्वज्रा मुनयः किल
गतासोस्तस्य भूयस्ते ममन्थुर्दक्षिणं करम्‌
अराजके तदा लोके दस्युभिः पीडिताः प्रजाः
जातो नारायणांशेन पृथुराद्यः क्षितीश्वरः
विदुर उवाच
तस्य शीलनिधेः साधोर्ब्रह्मण्यस्य महात्मनः
राज्ञः कथमभूद्दुष्टा प्रजा यद्विमना ययौ
किं वांहो वेन उद्दिश्य ब्रह्मदण्डमयूयुजन्‌
दण्डव्रतधरे राज्ञि मुनयो धर्मकोविदाः
नावध्येयः प्रजापालः प्रजाभिरघवानपि
यदसौ लोकपालानां बिभर्त्योजः स्वतेजसा
एतदाख्याहि मे ब्रह्मन्सुनीथात्मजचेष्टितम्‌
श्रद्दधानाय भक्ताय त्वं परावरवित्तमः
मैत्रेय उवाच
अङ्गोऽश्वमेधं राजर्षिराजहार महाक्रतुम्‌
नाजग्मुर्देवतास्तस्मिन्नाहूता ब्रह्मवादिभिः
तमूचुर्विस्मितास्तत्र यजमानमथर्त्विजः
हवींषि हूयमानानि न ते गृह्णन्ति देवताः
राजन्हवींष्यदुष्टानि श्रद्धयासादितानि ते
छन्दांस्ययातयामानि योजितानि धृतव्रतैः
न विदामेह देवानां हेलनं वयमण्वपि
यन्न गृह्णन्ति भागान्स्वान्ये देवाः कर्मसाक्षिणः
मैत्रेय उवाच
अङ्गो द्विजवचः श्रुत्वा यजमानः सुदुर्मनाः
तत्प्रष्टुं व्यसृजद्वाचं सदस्यांस्तदनुज्ञया
नागच्छन्त्याहुता देवा न गृह्णन्ति ग्रहानिह
सदसस्पतयो ब्रूत किमवद्यं मया कृतम्‌
सदसस्पतय ऊचुः
नरदेवेह भवतो नाघं तावन्मनाक्स्थितम्‌
अस्त्येकं प्राक्तनमघं यदिहेदृक्त्वमप्रजः
तथा साधय भद्रं ते आत्मानं सुप्रजं नृप
इष्टस्ते पुत्रकामस्य पुत्रं दास्यति यज्ञभुक्‌
तथा स्वभागधेयानि ग्रहीष्यन्ति दिवौकसः
यद्यज्ञपुरुषः साक्षादपत्याय हरिर्वृतः
तांस्तान्कामान्हरिर्दद्याद्यान्यान्कामयते जनः
आराधितो यथैवैष तथा पुंसां फलोदयः
इति व्यवसिता विप्रास्तस्य राज्ञः प्रजातये
पुरोडाशं निरवपन्शिपिविष्टाय विष्णवे
तस्मात्पुरुष उत्तस्थौ हेममाल्यमलाम्बरः
हिरण्मयेन पात्रेण सिद्धमादाय पायसम्‌
स विप्रानुमतो राजा गृहीत्वाञ्जलिनौदनम्‌
अवघ्राय मुदा युक्तः प्रादात्पत्न्या उदारधीः
सा तत्पुंसवनं राज्ञी प्राश्य वै पत्युरादधे
गर्भं काल उपावृत्ते कुमारं सुषुवेऽप्रजा
स बाल एव पुरुषो मातामहमनुव्रतः
अधर्मांशोद्भवं मृत्युं तेनाभवदधार्मिकः
स शरासनमुद्यम्य मृगयुर्वनगोचरः
हन्त्यसाधुर्मृगान्दीनान्वेनोऽसावित्यरौज्जनः
आक्रीडे क्रीडतो बालान्वयस्यानतिदारुणः
प्रसह्य निरनुक्रोशः पशुमारममारयत्‌
तं विचक्ष्य खलं पुत्रं शासनैर्विविधैर्नृपः
यदा न शासितुं कल्पो भृशमासीत्सुदुर्मनाः
प्रायेणाभ्यर्चितो देवो येऽप्रजा गृहमेधिनः
कदपत्यभृतं दुःखं ये न विन्दन्ति दुर्भरम्‌
यतः पापीयसी कीर्तिरधर्मश्च महान्नृणाम्‌
यतो विरोधः सर्वेषां यत आधिरनन्तकः
कस्तं प्रजापदेशं वै मोहबन्धनमात्मनः
पण्डितो बहु मन्येत यदर्थाः क्लेशदा गृहाः
कदपत्यं वरं मन्ये सदपत्याच्छुचां पदात्‌
निर्विद्येत गृहान्मर्त्यो यत्क्लेशनिवहा गृहाः
एवं स निर्विण्णमना नृपो गृहान्निशीथ उत्थाय महोदयोदयात्‌
अलब्धनिद्रोऽनुपलक्षितो नृभिर्हित्वा गतो वेनसुवं प्रसुप्ताम्‌
विज्ञाय निर्विद्य गतं पतिं प्रजाः पुरोहितामात्यसुहृद्गणादयः
विचिक्युरुर्व्यामतिशोककातरा यथा निगूढं पुरुषं कुयोगिनः
अलक्षयन्तः पदवीं प्रजापतेर्हतोद्यमाः प्रत्युपसृत्य ते पुरीम्‌
ऋषीन्समेतानभिवन्द्य साश्रवो न्यवेदयन्पौरव भर्तृविप्लवम्‌
मैत्रेय उवाच
भृग्वादयस्ते मुनयो लोकानां क्षेमदर्शिनः
गोप्तर्यसति वै न्ऱ्‌^ईणां पश्यन्तः पशुसाम्यताम्‌
वीरमातरमाहूय सुनीथां ब्रह्मवादिनः
प्रकृत्यसम्मतं वेनमभ्यषिञ्चन्पतिं भुवः
श्रुत्वा नृपासनगतं वेनमत्युग्रशासनम्‌
निलिल्युर्दस्यवः सद्यः सर्पत्रस्ता इवाखवः
स आरूढनृपस्थान उन्नद्धोऽष्टविभूतिभिः
अवमेने महाभागान्स्तब्धः सम्भावितः स्वतः
एवं मदान्ध उत्सिक्तो निरङ्कुश इव द्विपः
पर्यटन्रथमास्थाय कम्पयन्निव रोदसी
न यष्टव्यं न दातव्यं न होतव्यं द्विजाः क्वचित्‌
इति न्यवारयद्धर्मं भेरीघोषेण सर्वशः
वेनस्यावेक्ष्य मुनयो दुर्वृत्तस्य विचेष्टितम्‌
विमृश्य लोकव्यसनं कृपयोचुः स्म सत्रिणः
अहो उभयतः प्राप्तं लोकस्य व्यसनं महत्‌
दारुण्युभयतो दीप्ते इव तस्करपालयोः
अराजकभयादेष कृतो राजातदर्हणः
ततोऽप्यासीद्भयं त्वद्य कथं स्यात्स्वस्ति देहिनाम्‌
अहेरिव पयःपोषः पोषकस्याप्यनर्थभृत्‌
वेनः प्रकृत्यैव खलः सुनीथागर्भसम्भवः
निरूपितः प्रजापालः स जिघांसति वै प्रजाः
तथापि सान्त्वयेमामुं नास्मांस्तत्पातकं स्पृशेत्‌
तद्विद्वद्भिरसद्वृत्तो वेनोऽस्माभिः कृतो नृपः
सान्त्वितो यदि नो वाचं न ग्रहीष्यत्यधर्मकृत्‌
लोकधिक्कारसन्दग्धं दहिष्यामः स्वतेजसा
एवमध्यवसायैनं मुनयो गूढमन्यवः
उपव्रज्याब्रुवन्वेनं सान्त्वयित्वा च सामभिः
मुनय ऊचुः
नृपवर्य निबोधैतद्यत्ते विज्ञापयाम भोः
आयुःश्रीबलकीर्तीनां तव तात विवर्धनम्‌
धर्म आचरितः पुंसां वाङ्मनःकायबुद्धिभिः
लोकान्विशोकान्वितरत्यथानन्त्यमसङ्गिनाम्‌
स ते मा विनशेद्वीर प्रजानां क्षेमलक्षणः
यस्मिन्विनष्टे नृपतिरैश्वर्यादवरोहति
राजन्नसाध्वमात्येभ्यश्चोरादिभ्यः प्रजा नृपः
रक्षन्यथा बलिं गृह्णन्निह प्रेत्य च मोदते
यस्य राष्ट्रे पुरे चैव भगवान्यज्ञपूरुषः
इज्यते स्वेन धर्मेण जनैर्वर्णाश्रमान्वितैः
तस्य राज्ञो महाभाग भगवान्भूतभावनः
परितुष्यति विश्वात्मा तिष्ठतो निजशासने
तस्मिंस्तुष्टे किमप्राप्यंजगतामीश्वरेश्वरे
लोकाः सपाला ह्येतस्मै हरन्ति बलिमादृताः
तं सर्वलोकामरयज्ञसङ्ग्रहं त्रयीमयं द्रव्यमयं तपोमयम्‌
यज्ञैर्विचित्रैर्यजतो भवाय ते राजन्स्वदेशाननुरोद्धुमर्हसि
यज्ञेन युष्मद्विषये द्विजातिभिर्वितायमानेन सुराः कला हरेः
स्विष्टाः सुतुष्टाः प्रदिशन्ति वाञ्छितं तद्धेलनं नार्हसि वीर चेष्टितुम्‌
वेन उवाच
बालिशा बत यूयं वा अधर्मे धर्ममानिनः
ये वृत्तिदं पतिं हित्वा जारं पतिमुपासते
अवजानन्त्यमी मूढा नृपरूपिणमीश्वरम्‌
नानुविन्दन्ति ते भद्रमिह लोके परत्र च
को यज्ञपुरुषो नाम यत्र वो भक्तिरीदृशी
भर्तृस्नेहविदूराणां यथा जारे कुयोषिताम्‌
विष्णुर्विरिञ्चो गिरिश इन्द्रो वायुर्यमो रविः
पर्जन्यो धनदः सोमः क्षितिरग्निरपाम्पतिः
एते चान्ये च विबुधाः प्रभवो वरशापयोः
देहे भवन्ति नृपतेः सर्वदेवमयो नृपः
तस्मान्मां कर्मभिर्विप्रा यजध्वं गतमत्सराः
बलिं च मह्यं हरत मत्तोऽन्यः कोऽग्रभुक्पुमान्‌
मैत्रेय उवाच
इत्थं विपर्ययमतिः पापीयानुत्पथं गतः
अनुनीयमानस्तद्याच्ञां न चक्रे भ्रष्टमङ्गलः
इति तेऽसत्कृतास्तेन द्विजाः पण्डितमानिना
भग्नायां भव्ययाच्ञायां तस्मै विदुर चुक्रुधुः
हन्यतां हन्यतामेष पापः प्रकृतिदारुणः
जीवन्जगदसावाशु कुरुते भस्मसाद्ध्रुवम्‌
नायमर्हत्यसद्वृत्तो नरदेववरासनम्‌
योऽधियज्ञपतिं विष्णुं विनिन्दत्यनपत्रपः
को वैनं परिचक्षीत वेनमेकमृतेऽशुभम्‌
प्राप्त ईदृशमैश्वर्यं यदनुग्रहभाजनः
इत्थं व्यवसिता हन्तुमृषयो रूढमन्यवः
निजघ्नुर्हुङ्कृतैर्वेनं हतमच्युतनिन्दया
ऋषिभिः स्वाश्रमपदं गते पुत्रकलेवरम्‌
सुनीथा पालयामास विद्यायोगेन शोचती
एकदा मुनयस्ते तु सरस्वत्सलिलाप्लुताः
हुत्वाग्नीन्सत्कथाश्चक्रुरुपविष्टाः सरित्तटे
वीक्ष्योत्थितांस्तदोत्पातानाहुर्लोकभयङ्करान्‌
अप्यभद्रमनाथाया दस्युभ्यो न भवेद्भुवः
एवं मृशन्त ऋषयो धावतां सर्वतोदिशम्‌
पांसुः समुत्थितो भूरिश्चोराणामभिलुम्पताम्‌
तदुपद्रवमाज्ञाय लोकस्य वसु लुम्पताम्‌
भर्तर्युपरते तस्मिन्नन्योन्यं च जिघांसताम्‌
चोरप्रायं जनपदं हीनसत्त्वमराजकम्‌
लोकान्नावारयञ्छक्ता अपि तद्दोषदर्शिनः
ब्राह्मणः समदृक्षान्तो दीनानां समुपेक्षकः
स्रवते ब्रह्म तस्यापि भिन्नभाण्डात्पयो यथा
नाङ्गस्य वंशो राजर्षेरेष संस्थातुमर्हति
अमोघवीर्या हि नृपा वंशेऽस्मिन्केशवाश्रयाः
विनिश्चित्यैवमृषयो विपन्नस्य महीपतेः
ममन्थुरूरुं तरसा तत्रासीद्बाहुको नरः
काककृष्णोऽतिह्रस्वाङ्गो ह्रस्वबाहुर्महाहनुः
ह्रस्वपान्निम्ननासाग्रो रक्ताक्षस्ताम्रमूर्धजः
तं तु तेऽवनतं दीनं किं करोमीति वादिनम्‌
निषीदेत्यब्रुवंस्तात स निषादस्ततोऽभवत्‌
तस्य वंश्यास्तु नैषादा गिरिकाननगोचराः
येनाहरज्जायमानो वेनकल्मषमुल्बणम्‌
मैत्रेय उवाच
अथ तस्य पुनर्विप्रैरपुत्रस्य महीपतेः
बाहुभ्यां मथ्यमानाभ्यां मिथुनं समपद्यत
तद्दृष्ट्वा मिथुनं जातमृषयो ब्रह्मवादिनः
ऊचुः परमसन्तुष्टा विदित्वा भगवत्कलाम्‌
ऋषय ऊचुः
एष विष्णोर्भगवतः कला भुवनपालिनी
इयं च लक्ष्म्याः सम्भूतिः पुरुषस्यानपायिनी
अयं तु प्रथमो राज्ञां पुमान्प्रथयिता यशः
पृथुर्नाम महाराजो भविष्यति पृथुश्रवाः
इयं च सुदती देवी गुणभूषणभूषणा
अर्चिर्नाम वरारोहा पृथुमेवावरुन्धती
एष साक्षाद्धरेरंशोजातो लोकरिरक्षया
इयं च तत्परा हि श्रीरनुजज्ञेऽनपायिनी
मैत्रेय उवाच
प्रशंसन्ति स्म तं विप्रा गन्धर्वप्रवरा जगुः
मुमुचुः सुमनोधाराः सिद्धा नृत्यन्ति स्वःस्त्रियः
शङ्खतूर्यमृदङ्गाद्या नेदुर्दुन्दुभयो दिवि
तत्र सर्व उपाजग्मुर्देवर्षिपित्ऱ्‌^ईणां गणाः
ब्रह्मा जगद्गुरुर्देवैः सहासृत्य सुरेश्वरैः
वैन्यस्य दक्षिणे हस्ते दृष्ट्वा चिह्नं गदाभृतः
पादयोररविन्दं च तं वै मेने हरेः कलाम्‌
यस्याप्रतिहतं चक्रमंशः स परमेष्ठिनः
तस्याभिषेक आरब्धो ब्राह्मणैर्ब्रह्मवादिभिः
आभिषेचनिकान्यस्मै आजह्रुः सर्वतो जनाः
सरित्समुद्रा गिरयो नागा गावः खगा मृगाः
द्यौः क्षितिः सर्वभूतानि समाजह्रुरुपायनम्‌
सोऽभिषिक्तो महाराजः सुवासाः साध्वलङ्कृतः
पत्न्यार्चिषालङ्कृतया विरेजेऽग्निरिवापरः
तस्मै जहार धनदो हैमं वीर वरासनम्‌
वरुणः सलिलस्रावमातपत्रं शशिप्रभम्‌
वायुश्च वालव्यजने धर्मः कीर्तिमयीं स्रजम्‌
इन्द्रः किरीटमुत्कृष्टं दण्डं संयमनं यमः
ब्रह्मा ब्रह्ममयं वर्म भारती हारमुत्तमम्‌
हरिः सुदर्शनं चक्रं तत्पत्न्यव्याहतां श्रियम्‌
दशचन्द्रमसिं रुद्रः शतचन्द्रं तथाम्बिका
सोमोऽमृतमयानश्वांस्त्वष्टा रूपाश्रयं रथम्‌
अग्निराजगवं चापं सूर्यो रश्मिमयानिषून्‌
भूः पादुके योगमय्यौ द्यौः पुष्पावलिमन्वहम्‌
नाट्यं सुगीतं वादित्रमन्तर्धानं च खेचराः
ऋषयश्चाशिषः सत्याः समुद्रः शङ्खमात्मजम्‌
सिन्धवः पर्वता नद्यो रथवीथीर्महात्मनः
सूतोऽथ मागधो वन्दी तं स्तोतुमुपतस्थिरे
स्तावकांस्तानभिप्रेत्य पृथुर्वैन्यः प्रतापवान्‌
मेघनिर्ह्रादया वाचा प्रहसन्निदमब्रवीत्‌
पृथुरुवाच
भोः सूत हे मागध सौम्य वन्दिन्लोकेऽधुनास्पष्टगुणस्य मे स्यात्‌
किमाश्रयो मे स्तव एष योज्यतां मा मय्यभूवन्वितथा गिरो वः
तस्मात्परोक्षेऽस्मदुपश्रुतान्यलं करिष्यथ स्तोत्रमपीच्यवाचः
सत्युत्तमश्लोकगुणानुवादे जुगुप्सितं न स्तवयन्ति सभ्याः
महद्गुणानात्मनि कर्तुमीशः कः स्तावकैः स्तावयतेऽसतोऽपि
तेऽस्याभविष्यन्निति विप्रलब्धो जनावहासं कुमतिर्न वेद
प्रभवो ह्यात्मनः स्तोत्रंजुगुप्सन्त्यपि विश्रुताः
ह्रीमन्तः परमोदाराः पौरुषं वा विगर्हितम्‌
वयं त्वविदिता लोके सूताद्यापि वरीमभिः
कर्मभिः कथमात्मानं गापयिष्याम बालवत्‌
मैत्रेय उवाच
इति ब्रुवाणं नृपतिं गायका मुनिचोदिताः
तुष्टुवुस्तुष्टमनसस्तद्वागमृतसेवया
नालं वयं ते महिमानुवर्णने यो देववर्योऽवततार मायया
वेनाङ्गजातस्य च पौरुषाणि ते वाचस्पतीनामपि बभ्रमुर्धियः
अथाप्युदारश्रवसः पृथोर्हरेः कलावतारस्य कथामृतादृताः
यथोपदेशं मुनिभिः प्रचोदिताः श्लाघ्यानि कर्माणि वयं वितन्महि
एष धर्मभृतां श्रेष्ठो लोकं धर्मेऽनुवर्तयन्‌
गोप्ता च धर्मसेतूनां शास्ता तत्परिपन्थिनाम्‌
एष वै लोकपालानां बिभर्त्येकस्तनौ तनूः
काले काले यथाभागं लोकयोरुभयोर्हितम्‌
वसु काल उपादत्ते काले चायं विमुञ्चति
समः सर्वेषु भूतेषु प्रतपन्सूर्यवद्विभुः
तितिक्षत्यक्रमं वैन्य उपर्याक्रमतामपि
भूतानां करुणः शश्वदार्तानां क्षितिवृत्तिमान्‌
देवेऽवर्षत्यसौ देवो नरदेववपुर्हरिः
कृच्छ्रप्राणाः प्रजा ह्येष रक्षिष्यत्यञ्जसेन्द्रवत्‌
आप्याययत्यसौ लोकं वदनामृतमूर्तिना
सानुरागावलोकेन विशदस्मितचारुणा
अव्यक्तवर्त्मैष निगूढकार्यो गम्भीरवेधा उपगुप्तवित्तः
अनन्तमाहात्म्यगुणैकधामा पृथुः प्रचेता इव संवृतात्मा
दुरासदो दुर्विषह आसन्नोऽपि विदूरवत्‌
नैवाभिभवितुं शक्यो वेनारण्युत्थितोऽनलः
अन्तर्बहिश्च भूतानां पश्यन्कर्माणि चारणैः
उदासीन इवाध्यक्षो वायुरात्मेव देहिनाम्‌
नादण्ड्यं दण्डयत्येष सुतमात्मद्विषामपि
दण्डयत्यात्मजमपि दण्ड्यं धर्मपथे स्थितः
अस्याप्रतिहतं चक्रं पृथोरामानसाचलात्‌
वर्तते भगवानर्को यावत्तपति गोगणैः
रञ्जयिष्यति यल्लोकमयमात्मविचेष्टितैः
अथामुमाहू राजानं मनोरञ्जनकैः प्रजाः
दृढव्रतः सत्यसन्धो ब्रह्मण्यो वृद्धसेवकः
शरण्यः सर्वभूतानां मानदो दीनवत्सलः
मातृभक्तिः परस्त्रीषु पत्न्यामर्ध इवात्मनः
प्रजासु पितृवत्स्निग्धः किङ्करो ब्रह्मवादिनाम्‌
देहिनामात्मवत्प्रेष्ठः सुहृदां नन्दिवर्धनः
मुक्तसङ्गप्रसङ्गोऽयं दण्डपाणिरसाधुषु
अयं तु साक्षाद्भगवांस्त्र्यधीशः कूटस्थ आत्मा कलयावतीर्णः
यस्मिन्नविद्यारचितं निरर्थकं पश्यन्ति नानात्वमपि प्रतीतम्‌
अयं भुवो मण्डलमोदयाद्रेर्गोप्तैकवीरो नरदेवनाथः
आस्थाय जैत्रं रथमात्तचापः पर्यस्यते दक्षिणतो यथार्कः
अस्मै नृपालाः किल तत्र तत्र बलिं हरिष्यन्ति सलोकपालाः
मंस्यन्त एषां स्त्रिय आदिराजं चक्रायुधं तद्यश उद्धरन्त्यः
अयं महीं गां दुदुहेऽधिराजः प्रजापतिर्वृत्तिकरः प्रजानाम्‌
यो लीलयाद्रीन्स्वशरासकोट्या भिन्दन्समां गामकरोद्यथेन्द्रः
विस्फूर्जयन्नाजगवं धनुः स्वयं यदाचरत्क्ष्मामविषह्यमाजौ
तदा निलिल्युर्दिशि दिश्यसन्तो लाङ्गूलमुद्यम्य यथा मृगेन्द्रः
एषोऽश्वमेधाञ्शतमाजहार सरस्वती प्रादुरभावि यत्र
अहार्षीद्यस्य हयं पुरन्दरः शतक्रतुश्चरमे वर्तमाने
एष स्वसद्मोपवने समेत्य सनत्कुमारं भगवन्तमेकम्‌
आराध्य भक्त्यालभतामलं तज्ज्ञानं यतो ब्रह्म परं विदन्ति
तत्र तत्र गिरस्तास्ता इति विश्रुतविक्रमः
श्रोष्यत्यात्माश्रिता गाथाः पृथुः पृथुपराक्रमः
दिशो विजित्याप्रतिरुद्धचक्रः स्वतेजसोत्पाटितलोकशल्यः
सुरासुरेन्द्रैरुपगीयमान महानुभावो भविता पतिर्भुवः
मैत्रेय उवाच
एवं स भगवान्वैन्यः ख्यापितो गुणकर्मभिः
छन्दयामास तान्कामैः प्रतिपूज्याभिनन्द्य च
ब्राह्मणप्रमुखान्वर्णान्भृत्यामात्यपुरोधसः
पौरान्जानपदान्श्रेणीः प्रकृतीः समपूजयत्‌
विदुर उवाच
कस्माद्दधार गोरूपं धरित्री बहुरूपिणी
यां दुदोह पृथुस्तत्र को वत्सो दोहनं च किम्‌
प्रकृत्या विषमा देवी कृता तेन समा कथम्‌
तस्य मेध्यं हयं देवः कस्य हेतोरपाहरत्‌
सनत्कुमाराद्भगवतो ब्रह्मन्ब्रह्मविदुत्तमात्‌
लब्ध्वा ज्ञानं सविज्ञानं राजर्षिः कां गतिं गतः
यच्चान्यदपि कृष्णस्य भवान्भगवतः प्रभोः
श्रवः सुश्रवसः पुण्यं पूर्वदेहकथाश्रयम्‌
भक्ताय मेऽनुरक्ताय तव चाधोक्षजस्य च
वक्तुमर्हसि योऽदुह्यद्वैन्यरूपेण गामिमाम्‌
सूत उवाच
चोदितो विदुरेणैवं वासुदेवकथां प्रति
प्रशस्य तं प्रीतमना मैत्रेयः प्रत्यभाषत
मैत्रेय उवाच
यदाभिषिक्तः पृथुरङ्ग विप्रैरामन्त्रितो जनतायाश्च पालः
प्रजा निरन्ने क्षितिपृष्ठ एत्य क्षुत्क्षामदेहाः पतिमभ्यवोचन्‌
वयं राजञ्जाठरेणाभितप्ता यथाग्निना कोटरस्थेन वृक्षाः
त्वामद्य याताः शरणं शरण्यं यः साधितो वृत्तिकरः पतिर्नः
तन्नो भवानीहतु रातवेऽन्नं क्षुधार्दितानां नरदेवदेव
यावन्न नङ्क्ष्यामह उज्झितोर्जा वार्तापतिस्त्वं किल लोकपालः
मैत्रेय उवाच
पृथुः प्रजानां करुणं निशम्य परिदेवितम्‌
दीर्घं दध्यौ कुरुश्रेष्ठ निमित्तं सोऽन्वपद्यत
इति व्यवसितो बुद्ध्या प्रगृहीतशरासनः
सन्दधे विशिखं भूमेः क्रुद्धस्त्रिपुरहा यथा
प्रवेपमाना धरणी निशाम्योदायुधं च तम्‌
गौः सत्यपाद्रवद्भीता मृगीव मृगयुद्रुता
तामन्वधावत्तद्वैन्यः कुपितोऽत्यरुणेक्षणः
शरं धनुषि सन्धाय यत्र यत्र पलायते
सा दिशो विदिशो देवी रोदसी चान्तरं तयोः
धावन्ती तत्र तत्रैनं ददर्शानूद्यतायुधम्‌
लोके नाविन्दत त्राणं वैन्यान्मृत्योरिव प्रजाः
त्रस्ता तदा निववृते हृदयेन विदूयता
उवाच च महाभागं धर्मज्ञापन्नवत्सल
त्राहि मामपि भूतानां पालनेऽवस्थितो भवान्‌
स त्वं जिघांससे कस्माद्दीनामकृतकिल्बिषाम्‌
अहनिष्यत्कथं योषां धर्मज्ञ इति यो मतः
प्रहरन्ति न वै स्त्रीषु कृतागःस्वपि जन्तवः
किमुत त्वद्विधा राजन्करुणा दीनवत्सलाः
मां विपाट्याजरां नावं यत्र विश्वं प्रतिष्ठितम्‌
आत्मानं च प्रजाश्चेमाः कथमम्भसि धास्यसि
पृथुरुवाच
वसुधे त्वां वधिष्यामि मच्छासनपराङ्मुखीम्‌
भागं बर्हिषि या वृङ्क्ते न तनोति च नो वसु
यवसं जग्ध्यनुदिनं नैव दोग्ध्यौधसं पयः
तस्यामेवं हि दुष्टायां दण्डो नात्र न शस्यते
त्वं खल्वोषधिबीजानि प्राक्सृष्टानि स्वयम्भुवा
न मुञ्चस्यात्मरुद्धानि मामवज्ञाय मन्दधीः
अमूषां क्षुत्परीतानामार्तानां परिदेवितम्‌
शमयिष्यामि मद्बाणैर्भिन्नायास्तव मेदसा
पुमान्योषिदुत क्लीब आत्मसम्भावनोऽधमः
भूतेषु निरनुक्रोशो नृपाणां तद्वधोऽवधः
त्वां स्तब्धां दुर्मदां नीत्वा मायागां तिलशः शरैः
आत्मयोगबलेनेमा धारयिष्याम्यहं प्रजाः
एवं मन्युमयीं मूर्तिं कृतान्तमिव बिभ्रतम्‌
प्रणता प्राञ्जलिः प्राह मही सञ्जातवेपथुः
धरोवाच
नमः परस्मै पुरुषाय मायया विन्यस्तनानातनवे गुणात्मने
नमः स्वरूपानुभवेन निर्धुत द्रव्यक्रियाकारकविभ्रमोर्मये
येनाहमात्मायतनं विनिर्मिता धात्रा यतोऽयं गुणसर्गसङ्ग्रहः
स एव मां हन्तुमुदायुधः स्वराडुपस्थितोऽन्यं शरणं कमाश्रये
य एतदादावसृजच्चराचरं स्वमाययात्माश्रययावितर्क्यया
तयैव सोऽयं किल गोप्तुमुद्यतः कथं नु मां धर्मपरो जिघांसति
नूनं बतेशस्य समीहितं जनैस्तन्मायया दुर्जययाकृतात्मभिः
न लक्ष्यते यस्त्वकरोदकारयद्योऽनेक एकः परतश्च ईश्वरः
सर्गादि योऽस्यानुरुणद्धि शक्तिभिर्द्रव्यक्रियाकारकचेतनात्मभिः
तस्मै समुन्नद्धनिरुद्धशक्तये नमः परस्मै पुरुषाय वेधसे
स वै भवानात्मविनिर्मितं जगद्भूतेन्द्रियान्तःकरणात्मकं विभो
संस्थापयिष्यन्नज मां रसातलादभ्युज्जहाराम्भस आदिसूकरः
अपामुपस्थे मयि नाव्यवस्थिताः प्रजा भवानद्य रिरक्षिषुः किल
स वीरमूर्तिः समभूद्धराधरो यो मां पयस्युग्रशरो जिघांससि
नूनं जनैरीहितमीश्वराणामस्मद्विधैस्तद्गुणसर्गमायया
न ज्ञायते मोहितचित्तवर्त्मभिस्तेभ्यो नमो वीरयशस्करेभ्यः
मैत्रेय उवाच
इत्थं पृथुमभिष्टूय रुषा प्रस्फुरिताधरम्‌
पुनराहावनिर्भीता संस्तभ्यात्मानमात्मना
सन्नियच्छाभिभो मन्युं निबोध श्रावितं च मे
सर्वतः सारमादत्ते यथा मधुकरो बुधः
अस्मिन्लोकेऽथवामुष्मिन्मुनिभिस्तत्त्वदर्शिभिः
दृष्टा योगाः प्रयुक्ताश्च पुंसां श्रेयःप्रसिद्धये
तानातिष्ठति यः सम्यगुपायान्पूर्वदर्शितान्‌
अवरः श्रद्धयोपेत उपेयान्विन्दतेऽञ्जसा
ताननादृत्य योऽविद्वानर्थानारभते स्वयम्‌
तस्य व्यभिचरन्त्यर्था आरब्धाश्च पुनः पुनः
पुरा सृष्टा ह्योषधयो ब्रह्मणा या विशाम्पते
भुज्यमाना मया दृष्टा असद्भिरधृतव्रतैः
अपालितानादृता च भवद्भिर्लोकपालकैः
चोरीभूतेऽथ लोकेऽहं यज्ञार्थेऽग्रसमोषधीः
नूनं ता वीरुधः क्षीणा मयि कालेन भूयसा
तत्र योगेन दृष्टेन भवानादातुमर्हति
वत्सं कल्पय मे वीर येनाहं वत्सला तव
धोक्ष्ये क्षीरमयान्कामाननुरूपं च दोहनम्‌
दोग्धारं च महाबाहो भूतानां भूतभावन
अन्नमीप्सितमूर्जस्वद्भगवान्वाञ्छते यदि
समां च कुरु मां राजन्देववृष्टं यथा पयः
अपर्तावपि भद्रं ते उपावर्तेत मे विभो
इति प्रियं हितं वाक्यं भुव आदाय भूपतिः
वत्सं कृत्वा मनुं पाणावदुहत्सकलौषधीः
तथापरे च सर्वत्र सारमाददते बुधाः
ततोऽन्ये च यथाकामं दुदुहुः पृथुभाविताम्‌
ऋषयो दुदुहुर्देवीमिन्द्रियेष्वथ सत्तम
वत्सं बृहस्पतिं कृत्वा पयश्छन्दोमयं शुचि
कृत्वा वत्सं सुरगणा इन्द्रं सोममदूदुहन्‌
हिरण्मयेन पात्रेण वीर्यमोजो बलं पयः
दैतेया दानवा वत्सं प्रह्लादमसुरर्षभम्‌
विधायादूदुहन्क्षीरमयःपात्रे सुरासवम्‌
गन्धर्वाप्सरसोऽधुक्षन्पात्रे पद्ममये पयः
वत्सं विश्वावसुं कृत्वा गान्धर्वं मधु सौभगम्‌
वत्सेन पितरोऽर्यम्णा कव्यं क्षीरमधुक्षत
आमपात्रे महाभागाः श्रद्धया श्राद्धदेवताः
प्रकल्प्य वत्सं कपिलं सिद्धाः सङ्कल्पनामयीम्‌
सिद्धिं नभसि विद्यां च ये च विद्याधरादयः
अन्ये च मायिनो मायामन्तर्धानाद्भुतात्मनाम्‌
मयं प्रकल्प्य वत्सं ते दुदुहुर्धारणामयीम्‌
यक्षरक्षांसि भूतानि पिशाचाः पिशिताशनाः
भूतेशवत्सा दुदुहुः कपाले क्षतजासवम्‌
तथाहयो दन्दशूकाः सर्पा नागाश्च तक्षकम्‌
विधाय वत्सं दुदुहुर्बिलपात्रे विषं पयः
पशवो यवसं क्षीरं वत्सं कृत्वा च गोवृषम्‌
अरण्यपात्रे चाधुक्षन्मृगेन्द्रेण च दंष्ट्रिणः
क्रव्यादाः प्राणिनः क्रव्यं दुदुहुः स्वे कलेवरे
सुपर्णवत्सा विहगाश्चरं चाचरमेव च
वटवत्सा वनस्पतयः पृथग्रसमयं पयः
गिरयो हिमवद्वत्सा नानाधातून्स्वसानुषु
सर्वे स्वमुख्यवत्सेन स्वे स्वे पात्रे पृथक्पयः
सर्वकामदुघां पृथ्वीं दुदुहुः पृथुभाविताम्‌
एवं पृथ्वादयः पृथ्वीमन्नादाः स्वन्नमात्मनः
दोहवत्सादिभेदेन क्षीरभेदं कुरूद्वह
ततो महीपतिः प्रीतः सर्वकामदुघां पृथुः
दुहितृत्वे चकारेमां प्रेम्णा दुहितृवत्सलः
चूर्णयन्स्वधनुष्कोट्या गिरिकूटानि राजराट्‌
भूमण्डलमिदं वैन्यः प्रायश्चक्रे समं विभुः
अथास्मिन्भगवान्वैन्यः प्रजानां वृत्तिदः पिता
निवासान्कल्पयां चक्रे तत्र तत्र यथार्हतः
ग्रामान्पुरः पत्तनानि दुर्गाणि विविधानि च
घोषान्व्रजान्सशिबिरानाकरान्खेटखर्वटान्‌
प्राक्पृथोरिह नैवैषा पुरग्रामादिकल्पना
यथासुखं वसन्ति स्म तत्र तत्राकुतोभयाः
मैत्रेय उवाच
अथादीक्षत राजा तु हयमेधशतेन सः
ब्रह्मावर्ते मनोः क्षेत्रे यत्र प्राची सरस्वती
तदभिप्रेत्य भगवान्कर्मातिशयमात्मनः
शतक्रतुर्न ममृषे पृथोर्यज्ञमहोत्सवम्‌
यत्र यज्ञपतिः साक्षाद्भगवान्हरिरीश्वरः
अन्वभूयत सर्वात्मा सर्वलोकगुरुः प्रभुः
अन्वितो ब्रह्मशर्वाभ्यां लोकपालैः सहानुगैः
उपगीयमानो गन्धर्वैर्मुनिभिश्चाप्सरोगणैः
सिद्धा विद्याधरा दैत्या दानवा गुह्यकादयः
सुनन्दनन्दप्रमुखाः पार्षदप्रवरा हरेः
कपिलो नारदो दत्तो योगेशाः सनकादयः
तमन्वीयुर्भागवता ये च तत्सेवनोत्सुकाः
यत्र धर्मदुघा भूमिः सर्वकामदुघा सती
दोग्धि स्माभीप्सितानर्थान्यजमानस्य भारत
ऊहुः सर्वरसान्नद्यः क्षीरदध्यन्नगोरसान्‌
तरवो भूरिवर्ष्माणः प्रासूयन्त मधुच्युतः
सिन्धवो रत्ननिकरान्गिरयोऽन्नं चतुर्विधम्‌
उपायनमुपाजह्रुः सर्वे लोकाः सपालकाः
इति चाधोक्षजेशस्य पृथोस्तु परमोदयम्‌
असूयन्भगवानिन्द्रः प्रतिघातमचीकरत्‌
चरमेणाश्वमेधेन यजमाने यजुष्पतिम्‌
वैन्ये यज्ञपशुं स्पर्धन्नपोवाह तिरोहितः
तमत्रिर्भगवानैक्षत्त्वरमाणं विहायसा
आमुक्तमिव पाखण्डं योऽधर्मे धर्मविभ्रमः
अत्रिणा चोदितो हन्तुं पृथुपुत्रो महारथः
अन्वधावत सङ्क्रुद्धस्तिष्ठ तिष्ठेति चाब्रवीत्‌
तं तादृशाकृतिं वीक्ष्य मेने धर्मं शरीरिणम्‌
जटिलं भस्मनाच्छन्नं तस्मै बाणं न मुञ्चति
वधान्निवृत्तं तं भूयो हन्तवेऽत्रिरचोदयत्‌
जहि यज्ञहनं तात महेन्द्रं विबुधाधमम्‌
एवं वैन्यसुतः प्रोक्तस्त्वरमाणं विहायसा
अन्वद्रवदभिक्रुद्धो रावणं गृध्रराडिव
सोऽश्वं रूपं च तद्धित्वा तस्मा अन्तर्हितः स्वराट्‌
वीरः स्वपशुमादाय पितुर्यज्ञमुपेयिवान्‌
तत्तस्य चाद्भुतं कर्म विचक्ष्य परमर्षयः
नामधेयं ददुस्तस्मै विजिताश्व इति प्रभो
उपसृज्य तमस्तीव्रं जहाराश्वं पुनर्हरिः
चषालयूपतश्छन्नो हिरण्यरशनं विभुः
अत्रिः सन्दर्शयामास त्वरमाणं विहायसा
कपालखट्वाङ्गधरं वीरो नैनमबाधत
अत्रिणा चोदितस्तस्मै सन्दधे विशिखं रुषा
सोऽश्वं रूपं च तद्धित्वा तस्थावन्तर्हितः स्वराट्‌
वीरश्चाश्वमुपादाय पितृयज्ञमथाव्रजत्‌
तदवद्यं हरे रूपं जगृहुर्ज्ञानदुर्बलाः
यानि रूपाणि जगृहे इन्द्रो हयजिहीर्षया
तानि पापस्य खण्डानि लिङ्गं खण्डमिहोच्यते
एवमिन्द्रे हरत्यश्वं वैन्ययज्ञजिघांसया
तद्गृहीतविसृष्टेषु पाखण्डेषु मतिर्नृणाम्‌
धर्म इत्युपधर्मेषु नग्नरक्तपटादिषु
प्रायेण सज्जते भ्रान्त्या पेशलेषु च वाग्मिषु
तदभिज्ञाय भगवान्पृथुः पृथुपराक्रमः
इन्द्राय कुपितो बाणमादत्तोद्यतकार्मुकः
तमृत्विजः शक्रवधाभिसन्धितं विचक्ष्य दुष्प्रेक्ष्यमसह्यरंहसम्‌
निवारयामासुरहो महामते न युज्यतेऽत्रान्यवधः प्रचोदितात्‌
वयं मरुत्वन्तमिहार्थनाशनं ह्वयामहे त्वच्छ्रवसा हतत्विषम्‌
अयातयामोपहवैरनन्तरं प्रसह्य राजन्जुहवाम तेऽहितम्‌
इत्यामन्त्र्य क्रतुपतिं विदुरास्यर्त्विजो रुषा
स्रुग्घस्तान्जुह्वतोऽभ्येत्य स्वयम्भूः प्रत्यषेधत
न वध्यो भवतामिन्द्रो यद्यज्ञो भगवत्तनुः
यं जिघांसथ यज्ञेन यस्येष्टास्तनवः सुराः
तदिदं पश्यत महद्‌ धर्मव्यतिकरं द्विजाः
इन्द्रेणानुष्ठितं राज्ञः कर्मैतद्विजिघांसता
पृथुकीर्तेः पृथोर्भूयात्तर्ह्येकोनशतक्रतुः
अलं ते क्रतुभिः स्विष्टैर्यद्भवान्मोक्षधर्मवित्‌
नैवात्मने महेन्द्राय रोषमाहर्तुमर्हसि
उभावपि हि भद्रं ते उत्तमश्लोकविग्रहौ
मास्मिन्महाराज कृथाः स्म चिन्तां निशामयास्मद्वच आदृतात्मा
यद्ध्यायतो दैवहतं नु कर्तुं मनोऽतिरुष्टं विशते तमोऽन्धम्‌
क्रतुर्विरमतामेष देवेषु दुरवग्रहः
धर्मव्यतिकरो यत्र पाखण्डैरिन्द्रनिर्मितैः
एभिरिन्द्रोपसंसृष्टैः पाखण्डैर्हारिभिर्जनम्‌
ह्रियमाणं विचक्ष्वैनं यस्ते यज्ञध्रुगश्वमुट्‌
भवान्परित्रातुमिहावतीर्णो धर्मं जनानां समयानुरूपम्‌
वेनापचारादवलुप्तमद्य तद्देहतो विष्णुकलासि वैन्य
स त्वं विमृश्यास्य भवं प्रजापते सङ्कल्पनं विश्वसृजां पिपीपृहि
ऐन्द्रीं च मायामुपधर्ममातरं प्रचण्डपाखण्डपथं प्रभो जहि
मैत्रेय उवाच
इत्थं स लोकगुरुणा समादिष्टो विशाम्पतिः
तथा च कृत्वा वात्सल्यं मघोनापि च सन्दधे
कृतावभृथस्नानाय पृथवे भूरिकर्मणे
वरान्ददुस्ते वरदा ये तद्बर्हिषि तर्पिताः
विप्राः सत्याशिषस्तुष्टाः श्रद्धया लब्धदक्षिणाः
आशिषो युयुजुः क्षत्तरादिराजाय सत्कृताः
त्वयाहूता महाबाहो सर्व एव समागताः
पूजिता दानमानाभ्यां पितृदेवर्षिमानवाः
मैत्रेय उवाच
भगवानपि वैकुण्ठः साकं मघवता विभुः
यज्ञैर्यज्ञपतिस्तुष्टो यज्ञभुक्तमभाषत
श्रीभगवानुवाच
एष तेऽकार्षीद्भङ्गं हयमेधशतस्य ह
क्षमापयत आत्मानममुष्य क्षन्तुमर्हसि
सुधियः साधवो लोके नरदेव नरोत्तमाः
नाभिद्रुह्यन्ति भूतेभ्यो यर्हि नात्मा कलेवरम्‌
पुरुषा यदि मुह्यन्ति त्वादृशा देवमायया
श्रम एव परं जातो दीर्घया वृद्धसेवया
अतः कायमिमं विद्वानविद्याकामकर्मभिः
आरब्ध इति नैवास्मिन्प्रतिबुद्धोऽनुषज्जते
असंसक्तः शरीरेऽस्मिन्नमुनोत्पादिते गृहे
अपत्ये द्रविणे वापि कः कुर्यान्ममतां बुधः
एकः शुद्धः स्वयंज्योतिर्निर्गुणोऽसौ गुणाश्रयः
सर्वगोऽनावृतः साक्षी निरात्मात्मात्मनः परः
य एवं सन्तमात्मानमात्मस्थं वेद पूरुषः
नाज्यते प्रकृतिस्थोऽपि तद्गुणैः स मयि स्थितः
यः स्वधर्मेण मां नित्यं निराशीः श्रद्धयान्वितः
भजते शनकैस्तस्य मनो राजन्प्रसीदति
परित्यक्तगुणः सम्यग्दर्शनो विशदाशयः
शान्तिं मे समवस्थानं ब्रह्म कैवल्यमश्नुते
उदासीनमिवाध्यक्षं द्रव्यज्ञानक्रियात्मनाम्‌
कूटस्थमिममात्मानं यो वेदाप्नोति शोभनम्‌
भिन्नस्य लिङ्गस्य गुणप्रवाहो द्रव्यक्रियाकारकचेतनात्मनः
दृष्टासु सम्पत्सु विपत्सु सूरयो न विक्रियन्ते मयि बद्धसौहृदाः
समः समानोत्तममध्यमाधमः सुखे च दुःखे च जितेन्द्रियाशयः
मयोपकॢप्ताखिललोकसंयुतो विधत्स्व वीराखिललोकरक्षणम्‌
श्रेयः प्रजापालनमेव राज्ञो यत्साम्पराये सुकृतात्षष्ठमंशम्‌
हर्तान्यथा हृतपुण्यः प्रजानामरक्षिता करहारोऽघमत्ति
एवं द्विजाग्र्यानुमतानुवृत्त धर्मप्रधानोऽन्यतमोऽवितास्याः
ह्रस्वेन कालेन गृहोपयातान्द्रष्टासि सिद्धाननुरक्तलोकः
वरं च मत्कञ्चन मानवेन्द्र वृणीष्व तेऽहं गुणशीलयन्त्रितः
नाहं मखैर्वै सुलभस्तपोभिर्योगेन वा यत्समचित्तवर्ती
मैत्रेय उवाच
स इत्थं लोकगुरुणा विष्वक्सेनेन विश्वजित्‌
अनुशासित आदेशं शिरसा जगृहे हरेः
स्पृशन्तं पादयोः प्रेम्णा व्रीडितं स्वेन कर्मणा
शतक्रतुं परिष्वज्य विद्वेषं विससर्ज ह
भगवानथ विश्वात्मा पृथुनोपहृतार्हणः
समुज्जिहानया भक्त्या गृहीतचरणाम्बुजः
प्रस्थानाभिमुखोऽप्येनमनुग्रहविलम्बितः
पश्यन्पद्मपलाशाक्षो न प्रतस्थे सुहृत्सताम्‌
स आदिराजो रचिताञ्जलिर्हरिं विलोकितुं नाशकदश्रुलोचनः
न किञ्चनोवाच स बाष्पविक्लवो हृदोपगुह्यामुमधादवस्थितः
अथावमृज्याश्रुकला विलोकयन्नतृप्तदृग्गोचरमाह पूरुषम्‌
पदा स्पृशन्तं क्षितिमंस उन्नते विन्यस्तहस्ताग्रमुरङ्गविद्विषः
पृथुरुवाच
वरान्विभो त्वद्वरदेश्वराद्बुधः कथं वृणीते गुणविक्रियात्मनाम्‌
ये नारकाणामपि सन्ति देहिनां तानीश कैवल्यपते वृणे न च
न कामये नाथ तदप्यहं क्वचिन्न यत्र युष्मच्चरणाम्बुजासवः
महत्तमान्तर्हृदयान्मुखच्युतो विधत्स्व कर्णायुतमेष मे वरः
स उत्तमश्लोक महन्मुखच्युतो भवत्पदाम्भोजसुधा कणानिलः
स्मृतिं पुनर्विस्मृततत्त्ववर्त्मनां कुयोगिनां नो वितरत्यलं वरैः
यशः शिवं सुश्रव आर्यसङ्गमे यदृच्छया चोपशृणोति ते सकृत्‌
कथं गुणज्ञो विरमेद्विना पशुं श्रीर्यत्प्रवव्रे गुणसङ्ग्रहेच्छया
अथाभजे त्वाखिलपूरुषोत्तमं गुणालयं पद्मकरेव लालसः
अप्यावयोरेकपतिस्पृधोः कलिर्न स्यात्कृतत्वच्चरणैकतानयोः
जगज्जनन्यां जगदीश वैशसं स्यादेव यत्कर्मणि नः समीहितम्‌
करोषि फल्ग्वप्युरु दीनवत्सलः स्व एव धिष्ण्येऽभिरतस्य किं तया
भजन्त्यथ त्वामत एव साधवो व्युदस्तमायागुणविभ्रमोदयम्‌
भवत्पदानुस्मरणादृते सतां निमित्तमन्यद्भगवन्न विद्महे
मन्ये गिरं ते जगतां विमोहिनीं वरं वृणीष्वेति भजन्तमात्थ यत्‌
वाचा नु तन्त्या यदि ते जनोऽसितः कथं पुनः कर्म करोति मोहितः
त्वन्माययाद्धा जन ईश खण्डितो यदन्यदाशास्त ऋतात्मनोऽबुधः
यथा चरेद्बालहितं पिता स्वयं तथा त्वमेवार्हसि नः समीहितुम्‌
मैत्रेय उवाच
इत्यादिराजेन नुतः स विश्वदृक्तमाह राजन्मयि भक्तिरस्तु ते
दिष्ट्येदृशी धीर्मयि ते कृता यया मायां मदीयां तरति स्म दुस्त्यजाम्‌
तत्त्वं कुरु मयादिष्टमप्रमत्तः प्रजापते
मदादेशकरो लोकः सर्वत्राप्नोति शोभनम्‌
मैत्रेय उवाच
इति वैन्यस्य राजर्षेः प्रतिनन्द्यार्थवद्वचः
पूजितोऽनुगृहीत्वैनं गन्तुं चक्रेऽच्युतो मतिम्‌
देवर्षिपितृगन्धर्व सिद्धचारणपन्नगाः
किन्नराप्सरसो मर्त्याः खगा भूतान्यनेकशः
यज्ञेश्वरधिया राज्ञा वाग्वित्ताञ्जलिभक्तितः
सभाजिता ययुः सर्वे वैकुण्ठानुगतास्ततः
भगवानपि राजर्षेः सोपाध्यायस्य चाच्युतः
हरन्निव मनोऽमुष्य स्वधाम प्रत्यपद्यत
अदृष्टाय नमस्कृत्य नृपः सन्दर्शितात्मने
अव्यक्ताय च देवानां देवाय स्वपुरं ययौ
मैत्रेय उवाच
मौक्तिकैः कुसुमस्रग्भिर्दुकूलैः स्वर्णतोरणैः
महासुरभिभिर्धूपैर्मण्डितं तत्र तत्र वै
चन्दनागुरुतोयार्द्र रथ्याचत्वरमार्गवत्‌
पुष्पाक्षतफलैस्तोक्मैर्लाजैरर्चिर्भिरर्चितम्‌
सवृन्दैः कदलीस्तम्भैः पूगपोतैः परिष्कृतम्‌
तरुपल्लवमालाभिः सर्वतः समलङ्कृतम्‌
प्रजास्तं दीपबलिभिः सम्भृताशेषमङ्गलैः
अभीयुर्मृष्टकन्याश्च मृष्टकुण्डलमण्डिताः
शङ्खदुन्दुभिघोषेण ब्रह्मघोषेण चर्त्विजाम्‌
विवेश भवनं वीरः स्तूयमानो गतस्मयः
पूजितः पूजयामास तत्र तत्र महायशाः
पौराञ्जानपदांस्तांस्तान्प्रीतः प्रियवरप्रदः
स एवमादीन्यनवद्यचेष्टितः कर्माणि भूयांसि महान्महत्तमः
कुर्वन्शशासावनिमण्डलं यशः स्फीतं निधायारुरुहे परं पदम्‌
सूत उवाच
तदादिराजस्य यशो विजृम्भितं गुणैरशेषैर्गुणवत्सभाजितम्‌
क्षत्ता महाभागवतः सदस्पते कौषारविं प्राह गृणन्तमर्चयन्‌
विदुर उवाच
सोऽभिषिक्तः पृथुर्विप्रैर्लब्धाशेषसुरार्हणः
बिभ्रत्स वैष्णवं तेजो बाह्वोर्याभ्यां दुदोह गाम्‌
को न्वस्य कीर्तिं न शृणोत्यभिज्ञो यद्विक्रमोच्छिष्टमशेषभूपाः
लोकाः सपाला उपजीवन्ति काममद्यापि तन्मे वद कर्म शुद्धम्‌
मैत्रेय उवाच
गङ्गायमुनयोर्नद्योरन्तरा क्षेत्रमावसन्‌
आरब्धानेव बुभुजे भोगान्पुण्यजिहासया
सर्वत्रास्खलितादेशः सप्तद्वीपैकदण्डधृक्‌
अन्यत्र ब्राह्मणकुलादन्यत्राच्युतगोत्रतः
एकदासीन्महासत्र दीक्षा तत्र दिवौकसाम्‌
समाजो ब्रह्मर्षीणां च राजर्षीणां च सत्तम
तस्मिन्नर्हत्सु सर्वेषु स्वर्चितेषु यथार्हतः
उत्थितः सदसो मध्ये ताराणामुडुराडिव
प्रांशुः पीनायतभुजो गौरः कञ्जारुणेक्षणः
सुनासः सुमुखः सौम्यः पीनांसः सुद्विजस्मितः
व्यूढवक्षा बृहच्छ्रोणिर्वलिवल्गुदलोदरः
आवर्तनाभिरोजस्वी काञ्चनोरुरुदग्रपात्‌
सूक्ष्मवक्रासितस्निग्ध मूर्धजः कम्बुकन्धरः
महाधने दुकूलाग्र्ये परिधायोपवीय च
व्यञ्जिताशेषगात्रश्रीर्नियमे न्यस्तभूषणः
कृष्णाजिनधरः श्रीमान्कुशपाणिः कृतोचितः
शिशिरस्निग्धताराक्षः समैक्षत समन्ततः
ऊचिवानिदमुर्वीशः सदः संहर्षयन्निव
चारु चित्रपदं श्लक्ष्णं मृष्टं गूढमविक्लवम्‌
सर्वेषामुपकारार्थं तदा अनुवदन्निव
राजोवाच
सभ्याः शृणुत भद्रं वः साधवो य इहागताः
सत्सु जिज्ञासुभिर्धर्ममावेद्यं स्वमनीषितम्‌
अहं दण्डधरो राजा प्रजानामिह योजितः
रक्षिता वृत्तिदः स्वेषु सेतुषु स्थापिता पृथक्‌
तस्य मे तदनुष्ठानाद्यानाहुर्ब्रह्मवादिनः
लोकाः स्युः कामसन्दोहा यस्य तुष्यति दिष्टदृक्‌
य उद्धरेत्करं राजा प्रजा धर्मेष्वशिक्षयन्‌
प्रजानां शमलं भुङ्क्ते भगं च स्वं जहाति सः
तत्प्रजा भर्तृपिण्डार्थं स्वार्थमेवानसूयवः
कुरुताधोक्षजधियस्तर्हि मेऽनुग्रहः कृतः
यूयं तदनुमोदध्वं पितृदेवर्षयोऽमलाः
कर्तुः शास्तुरनुज्ञातुस्तुल्यं यत्प्रेत्य तत्फलम्‌
अस्ति यज्ञपतिर्नाम केषाञ्चिदर्हसत्तमाः
इहामुत्र च लक्ष्यन्ते ज्योत्स्नावत्यः क्वचिद्भुवः
मनोरुत्तानपादस्य ध्रुवस्यापि महीपतेः
प्रियव्रतस्य राजर्षेरङ्गस्यास्मत्पितुः पितुः
ईदृशानामथान्येषामजस्य च भवस्य च
प्रह्लादस्य बलेश्चापि कृत्यमस्ति गदाभृता
दौहित्रादीनृते मृत्योः शोच्यान्धर्मविमोहितान्‌
वर्गस्वर्गापवर्गाणां प्रायेणैकात्म्यहेतुना
यत्पादसेवाभिरुचिस्तपस्विनामशेषजन्मोपचितं मलं धियः
सद्यः क्षिणोत्यन्वहमेधती सती यथा पदाङ्गुष्ठविनिःसृता सरित्‌
विनिर्धुताशेषमनोमलः पुमानसङ्गविज्ञानविशेषवीर्यवान्‌
यदङ्घ्रिमूले कृतकेतनः पुनर्न संसृतिं क्लेशवहां प्रपद्यते
तमेव यूयं भजतात्मवृत्तिभिर्मनोवचःकायगुणैः स्वकर्मभिः
अमायिनः कामदुघाङ्घ्रिपङ्कजं यथाधिकारावसितार्थसिद्धयः
असाविहानेकगुणोऽगुणोऽध्वरः पृथग्विधद्रव्यगुणक्रियोक्तिभिः
सम्पद्यतेऽर्थाशयलिङ्गनामभिर्विशुद्धविज्ञानघनः स्वरूपतः
प्रधानकालाशयधर्मसङ्ग्रहे शरीर एष प्रतिपद्य चेतनाम्‌
क्रियाफलत्वेन विभुर्विभाव्यते यथानलो दारुषु तद्गुणात्मकः
अहो ममामी वितरन्त्यनुग्रहं हरिं गुरुं यज्ञभुजामधीश्वरम्‌
स्वधर्मयोगेन यजन्ति मामका निरन्तरं क्षोणितले दृढव्रताः
मा जातु तेजः प्रभवेन्महर्द्धिभिस्तितिक्षया तपसा विद्यया च
देदीप्यमानेऽजितदेवतानां कुले स्वयं राजकुलाद्द्विजानाम्‌
ब्रह्मण्यदेवः पुरुषः पुरातनो नित्यं हरिर्यच्चरणाभिवन्दनात्‌
अवाप लक्ष्मीमनपायिनीं यशो जगत्पवित्रं च महत्तमाग्रणीः
यत्सेवयाशेषगुहाशयः स्वराड्विप्रप्रियस्तुष्यति काममीश्वरः
तदेव तद्धर्मपरैर्विनीतैः सर्वात्मना ब्रह्मकुलं निषेव्यताम्‌
पुमान्लभेतानतिवेलमात्मनः प्रसीदतोऽत्यन्तशमं स्वतः स्वयम्‌
यन्नित्यसम्बन्धनिषेवया ततः परं किमत्रास्ति मुखं हविर्भुजाम्‌
अश्नात्यनन्तः खलु तत्त्वकोविदैः श्रद्धाहुतं यन्मुख इज्यनामभिः
न वै तथा चेतनया बहिष्कृते हुताशने पारमहंस्यपर्यगुः
यद्ब्रह्म नित्यं विरजं सनातनं श्रद्धातपोमङ्गलमौनसंयमैः
समाधिना बिभ्रति हार्थदृष्टये यत्रेदमादर्श इवावभासते
तेषामहं पादसरोजरेणुमार्या वहेयाधिकिरीटमायुः
यं नित्यदा बिभ्रत आशु पापं नश्यत्यमुं सर्वगुणा भजन्ति
गुणायनं शीलधनं कृतज्ञं वृद्धाश्रयं संवृणतेऽनु सम्पदः
प्रसीदतां ब्रह्मकुलं गवां च जनार्दनः सानुचरश्च मह्यम्‌
मैत्रेय उवाच
इति ब्रुवाणं नृपतिं पितृदेवद्विजातयः
तुष्टुवुर्हृष्टमनसः साधुवादेन साधवः
पुत्रेण जयते लोकानिति सत्यवती श्रुतिः
ब्रह्मदण्डहतः पापो यद्वेनोऽत्यतरत्तमः
हिरण्यकशिपुश्चापि भगवन्निन्दया तमः
विविक्षुरत्यगात्सूनोः प्रह्लादस्यानुभावतः
वीरवर्य पितः पृथ्व्याः समाः सञ्जीव शाश्वतीः
यस्येदृश्यच्युते भक्तिः सर्वलोकैकभर्तरि
अहो वयं ह्यद्य पवित्रकीर्ते त्वयैव नाथेन मुकुन्दनाथाः
य उत्तमश्लोकतमस्य विष्णोर्ब्रह्मण्यदेवस्य कथां व्यनक्ति
नात्यद्भुतमिदं नाथ तवाजीव्यानुशासनम्‌
प्रजानुरागो महतां प्रकृतिः करुणात्मनाम्‌
अद्य नस्तमसः पारस्त्वयोपासादितः प्रभो
भ्राम्यतां नष्टदृष्टीनां कर्मभिर्दैवसंज्ञितैः
नमो विवृद्धसत्त्वाय पुरुषाय महीयसे
यो ब्रह्म क्षत्रमाविश्य बिभर्तीदं स्वतेजसा
मैत्रेय उवाच
जनेषु प्रगृणत्स्वेवं पृथुं पृथुलविक्रमम्‌
तत्रोपजग्मुर्मुनयश्चत्वारः सूर्यवर्चसः
तांस्तु सिद्धेश्वरान्राजा व्योम्नोऽवतरतोऽर्चिषा
लोकानपापान्कुर्वाणान्सानुगोऽचष्ट लक्षितान्‌
तद्दर्शनोद्गतान्प्राणान्प्रत्यादित्सुरिवोत्थितः
ससदस्यानुगो वैन्य इन्द्रियेशो गुणानिव
गौरवाद्यन्त्रितः सभ्यः प्रश्रयानतकन्धरः
विधिवत्पूजयां चक्रे गृहीताध्यर्हणासनान्‌
तत्पादशौचसलिलैर्मार्जितालकबन्धनः
तत्र शीलवतां वृत्तमाचरन्मानयन्निव
हाटकासन आसीनान्स्वधिष्ण्येष्विव पावकान्‌
श्रद्धासंयमसंयुक्तः प्रीतः प्राह भवाग्रजान्‌
पृथुरुवाच
अहो आचरितं किं मे मङ्गलं मङ्गलायनाः
यस्य वो दर्शनं ह्यासीद्दुर्दर्शानां च योगिभिः
किं तस्य दुर्लभतरमिह लोके परत्र च
यस्य विप्राः प्रसीदन्ति शिवो विष्णुश्च सानुगः
नैव लक्षयते लोको लोकान्पर्यटतोऽपि यान्‌
यथा सर्वदृशं सर्व आत्मानं येऽस्य हेतवः
अधना अपि ते धन्याः साधवो गृहमेधिनः
यद्गृहा ह्यर्हवर्याम्बु तृणभूमीश्वरावराः
व्यालालयद्रुमा वै तेष्वरिक्ताखिलसम्पदः
यद्गृहास्तीर्थपादीय पादतीर्थविवर्जिताः
स्वागतं वो द्विजश्रेष्ठा यद्व्रतानि मुमुक्षवः
चरन्ति श्रद्धया धीरा बाला एव बृहन्ति च
कच्चिन्नः कुशलं नाथा इन्द्रियार्थार्थवेदिनाम्‌
व्यसनावाप एतस्मिन्पतितानां स्वकर्मभिः
भवत्सु कुशलप्रश्न आत्मारामेषु नेष्यते
कुशलाकुशला यत्र न सन्ति मतिवृत्तयः
तदहं कृतविश्रम्भः सुहृदो वस्तपस्विनाम्‌
सम्पृच्छे भव एतस्मिन्क्षेमः केनाञ्जसा भवेत्‌
व्यक्तमात्मवतामात्मा भगवानात्मभावनः
स्वानामनुग्रहायेमां सिद्धरूपी चरत्यजः
मैत्रेय उवाच
पृथोस्तत्सूक्तमाकर्ण्य सारं सुष्ठु मितं मधु
स्मयमान इव प्रीत्या कुमारः प्रत्युवाच ह
सनत्कुमार उवाच
साधु पृष्टं महाराज सर्वभूतहितात्मना
भवता विदुषा चापि साधूनां मतिरीदृशी
सङ्गमः खलु साधूनामुभयेषां च सम्मतः
यत्सम्भाषणसम्प्रश्नः सर्वेषां वितनोति शम्‌
अस्त्येव राजन्भवतो मधुद्विषः पादारविन्दस्य गुणानुवादने
रतिर्दुरापा विधुनोति नैष्ठिकी कामं कषायं मलमन्तरात्मनः
शास्त्रेष्वियानेव सुनिश्चितो नृणां क्षेमस्य सध्र्यग्विमृशेषु हेतुः
असङ्ग आत्मव्यतिरिक्त आत्मनि दृढा रतिर्ब्रह्मणि निर्गुणे च या
सा श्रद्धया भगवद्धर्मचर्यया जिज्ञासयाध्यात्मिकयोगनिष्ठया
योगेश्वरोपासनया च नित्यं पुण्यश्रवःकथया पुण्यया च
अर्थेन्द्रियारामसगोष्ठ्यतृष्णया तत्सम्मतानामपरिग्रहेण च
विविक्तरुच्या परितोष आत्मनि विना हरेर्गुणपीयूषपानात्‌
अहिंसया पारमहंस्यचर्यया स्मृत्या मुकुन्दाचरिताग्र्यसीधुना
यमैरकामैर्नियमैश्चाप्यनिन्दया निरीहया द्वन्द्वतितिक्षया च
हरेर्मुहुस्तत्परकर्णपूर गुणाभिधानेन विजृम्भमाणया
भक्त्या ह्यसङ्गः सदसत्यनात्मनि स्यान्निर्गुणे ब्रह्मणि चाञ्जसा रतिः
यदा रतिर्ब्रह्मणि नैष्ठिकी पुमानाचार्यवान्ज्ञानविरागरंहसा
दहत्यवीर्यं हृदयं जीवकोशं पञ्चात्मकं योनिमिवोत्थितोऽग्निः
दग्धाशयो मुक्तसमस्ततद्गुणो नैवात्मनो बहिरन्तर्विचष्टे
परात्मनोर्यद्व्यवधानं पुरस्तात्स्वप्ने यथा पुरुषस्तद्विनाशे
आत्मानमिन्द्रियार्थं च परं यदुभयोरपि
सत्याशय उपाधौ वै पुमान्पश्यति नान्यदा
निमित्ते सति सर्वत्र जलादावपि पूरुषः
आत्मनश्च परस्यापि भिदां पश्यति नान्यदा
इन्द्रियैर्विषयाकृष्टैराक्षिप्तं ध्यायतां मनः
चेतनां हरते बुद्धेः स्तम्बस्तोयमिव ह्रदात्‌
भ्रश्यत्यनुस्मृतिश्चित्तं ज्ञानभ्रंशः स्मृतिक्षये
तद्रोधं कवयः प्राहुरात्मापह्नवमात्मनः
नातः परतरो लोके पुंसः स्वार्थव्यतिक्रमः
यदध्यन्यस्य प्रेयस्त्वमात्मनः स्वव्यतिक्रमात्‌
अर्थेन्द्रियार्थाभिध्यानं सर्वार्थापह्नवो नृणाम्‌
भ्रंशितो ज्ञानविज्ञानाद्येनाविशति मुख्यताम्‌
न कुर्यात्कर्हिचित्सङ्गं तमस्तीव्रं तितीरिषुः
धर्मार्थकाममोक्षाणां यदत्यन्तविघातकम्‌
तत्रापि मोक्ष एवार्थ आत्यन्तिकतयेष्यते
त्रैवर्ग्योऽर्थो यतो नित्यं कृतान्तभयसंयुतः
परेऽवरे च ये भावा गुणव्यतिकरादनु
न तेषां विद्यते क्षेममीशविध्वंसिताशिषाम्‌
तत्त्वं नरेन्द्र जगतामथ तस्थूषां च
देहेन्द्रियासुधिषणात्मभिरावृतानाम्‌
यः क्षेत्रवित्तपतया हृदि विश्वगाविः
प्रत्यक्चकास्ति भगवांस्तमवेहि सोऽस्मि
यस्मिन्निदं सदसदात्मतया विभाति
माया विवेकविधुति स्रजि वाहिबुद्धिः
तं नित्यमुक्तपरिशुद्धविशुद्धतत्त्वं
प्रत्यूढकर्मकलिलप्रकृतिं प्रपद्ये
यत्पादपङ्कजपलाशविलासभक्त्या
कर्माशयं ग्रथितमुद्ग्रथयन्ति सन्तः
तद्वन्न रिक्तमतयो यतयोऽपि रुद्ध
स्रोतोगणास्तमरणं भज वासुदेवम्‌
कृच्छ्रो महानिह भवार्णवमप्लवेशां
षड्वर्गनक्रमसुखेन तितीर्षन्ति
तत्त्वं हरेर्भगवतो भजनीयमङ्घ्रिं
कृत्वोडुपं व्यसनमुत्तर दुस्तरार्णम्‌
मैत्रेय उवाच
स एवं ब्रह्मपुत्रेण कुमारेणात्ममेधसा
दर्शितात्मगतिः सम्यक्प्रशस्योवाच तं नृपः
राजोवाच
कृतो मेऽनुग्रहः पूर्वं हरिणार्तानुकम्पिना
तमापादयितुं ब्रह्मन्भगवन्यूयमागताः
निष्पादितश्च कार्त्स्न्येन भगवद्भिर्घृणालुभिः
साधूच्छिष्टं हि मे सर्वमात्मना सह किं ददे
प्राणा दाराः सुता ब्रह्मन्गृहाश्च सपरिच्छदाः
राज्यं बलं मही कोश इति सर्वं निवेदितम्‌
सैनापत्यं च राज्यं च दण्डनेतृत्वमेव च
सर्व लोकाधिपत्यं च वेदशास्त्रविदर्हति
स्वमेव ब्राह्मणो भुङ्क्ते स्वं वस्ते स्वं ददाति च
तस्यैवानुग्रहेणान्नं भुञ्जते क्षत्रियादयः
यैरीदृशी भगवतो गतिरात्मवाद
एकान्ततो निगमिभिः प्रतिपादिता नः
तुष्यन्त्वदभ्रकरुणाः स्वकृतेन नित्यं
को नाम तत्प्रतिकरोति विनोदपात्रम्‌
मैत्रेय उवाच
त आत्मयोगपतय आदिराजेन पूजिताः
शीलं तदीयं शंसन्तः खेऽभवन्मिषतां नृणाम्‌
वैन्यस्तु धुर्यो महतां संस्थित्याध्यात्मशिक्षया
आप्तकाममिवात्मानं मेन आत्मन्यवस्थितः
कर्माणि च यथाकालं यथादेशं यथाबलम्‌
यथोचितं यथावित्तमकरोद्ब्रह्मसात्कृतम्‌
फलं ब्रह्मणि सन्न्यस्य निर्विषङ्गः समाहितः
कर्माध्यक्षं च मन्वान आत्मानं प्रकृतेः परम्‌
गृहेषु वर्तमानोऽपि स साम्राज्यश्रियान्वितः
नासज्जतेन्द्रियार्थेषु निरहम्मतिरर्कवत्‌
एवमध्यात्मयोगेन कर्माण्यनुसमाचरन्‌
पुत्रानुत्पादयामास पञ्चार्चिष्यात्मसम्मतान्‌
विजिताश्वं धूम्रकेशं हर्यक्षं द्रविणं वृकम्‌
सर्वेषां लोकपालानां दधारैकः पृथुर्गुणान्‌
गोपीथाय जगत्सृष्टेः काले स्वे स्वेऽच्युतात्मकः
मनोवाग्वृत्तिभिः सौम्यैर्गुणैः संरञ्जयन्प्रजाः
राजेत्यधान्नामधेयं सोमराज इवापरः
सूर्यवद्विसृजन्गृह्णन्प्रतपंश्च भुवो वसु
दुर्धर्षस्तेजसेवाग्निर्महेन्द्र इव दुर्जयः
तितिक्षया धरित्रीव द्यौरिवाभीष्टदो नृणाम्‌
वर्षति स्म यथाकामं पर्जन्य इव तर्पयन्‌
समुद्र इव दुर्बोधः सत्त्वेनाचलराडिव
धर्मराडिव शिक्षायामाश्चर्ये हिमवानिव
कुवेर इव कोशाढ्यो गुप्तार्थो वरुणो यथा
मातरिश्वेव सर्वात्मा बलेन महसौजसा
अविषह्यतया देवो भगवान्भूतराडिव
कन्दर्प इव सौन्दर्ये मनस्वी मृगराडिव
वात्सल्ये मनुवन्नृणां प्रभुत्वे भगवानजः
बृहस्पतिर्ब्रह्मवादे आत्मवत्त्वे स्वयं हरिः
भक्त्या गोगुरुविप्रेषु विष्वक्सेनानुवर्तिषु
ह्रिया प्रश्रयशीलाभ्यामात्मतुल्यः परोद्यमे
कीर्त्योर्ध्वगीतया पुम्भिस्त्रैलोक्ये तत्र तत्र ह
प्रविष्टः कर्णरन्ध्रेषु स्त्रीणां रामः सतामिव
मैत्रेय उवाच
दृष्ट्वात्मानं प्रवयसमेकदा वैन्य आत्मवान्‌
आत्मना वर्धिताशेष स्वानुसर्गः प्रजापतिः
जगतस्तस्थुषश्चापि वृत्तिदो धर्मभृत्सताम्‌
निष्पादितेश्वरादेशो यदर्थमिह जज्ञिवान्‌
आत्मजेष्वात्मजां न्यस्य विरहाद्रुदतीमिव
प्रजासु विमनःस्वेकः सदारोऽगात्तपोवनम्‌
तत्राप्यदाभ्यनियमो वैखानससुसम्मते
आरब्ध उग्रतपसि यथा स्वविजये पुरा
कन्दमूलफलाहारः शुष्कपर्णाशनः क्वचित्‌
अब्भक्षः कतिचित्पक्षान्वायुभक्षस्ततः परम्‌
ग्रीष्मे पञ्चतपा वीरो वर्षास्वासारषाण्मुनिः
आकण्ठमग्नः शिशिरे उदके स्थण्डिलेशयः
तितिक्षुर्यतवाग्दान्त ऊर्ध्वरेता जितानिलः
आरिराधयिषुः कृष्णमचरत्तप उत्तमम्‌
तेन क्रमानुसिद्धेन ध्वस्तकर्ममलाशयः
प्राणायामैः सन्निरुद्ध षड्वर्गश्छिन्नबन्धनः
सनत्कुमारो भगवान्यदाहाध्यात्मिकं परम्‌
योगं तेनैव पुरुषमभजत्पुरुषर्षभः
भगवद्धर्मिणः साधोः श्रद्धया यततः सदा
भक्तिर्भगवति ब्रह्मण्यनन्यविषयाभवत्‌
तस्यानया भगवतः परिकर्मशुद्ध
सत्त्वात्मनस्तदनुसंस्मरणानुपूर्त्या
ज्ञानं विरक्तिमदभून्निशितेन येन
चिच्छेद संशयपदं निजजीवकोशम्‌
छिन्नान्यधीरधिगतात्मगतिर्निरीहस्‌
तत्तत्यजेऽच्छिनदिदं वयुनेन येन
तावन्न योगगतिभिर्यतिरप्रमत्तो
यावद्गदाग्रजकथासु रतिं न कुर्यात्‌
एवं स वीरप्रवरः संयोज्यात्मानमात्मनि
ब्रह्मभूतो दृढं काले तत्याज स्वं कलेवरम्‌
सम्पीड्य पायुं पार्ष्णिभ्यां वायुमुत्सारयञ्छनैः
नाभ्यां कोष्ठेष्ववस्थाप्य हृदुरःकण्ठशीर्षणि
उत्सर्पयंस्तु तं मूर्ध्नि क्रमेणावेश्य निःस्पृहः
वायुं वायौ क्षितौ कायं तेजस्तेजस्ययूयुजत्‌
खान्याकाशे द्रवं तोये यथास्थानं विभागशः
क्षितिमम्भसि तत्तेजस्यदो वायौ नभस्यमुम्‌
इन्द्रियेषु मनस्तानि तन्मात्रेषु यथोद्भवम्‌
भूतादिनामून्युत्कृष्य महत्यात्मनि सन्दधे
तं सर्वगुणविन्यासं जीवे मायामये न्यधात्‌
तं चानुशयमात्मस्थमसावनुशयी पुमान्‌
नानवैराग्यवीर्येण स्वरूपस्थोऽजहात्प्रभुः
अर्चिर्नाम महाराज्ञी तत्पत्न्यनुगता वनम्‌
सुकुमार्यतदर्हा च यत्पद्भ्यां स्पर्शनं भुवः
अतीव भर्तुर्व्रतधर्मनिष्ठया शुश्रूषया चार्षदेहयात्रया
नाविन्दतार्तिं परिकर्शितापि सा प्रेयस्करस्पर्शनमाननिर्वृतिः
देहं विपन्नाखिलचेतनादिकं पत्युः पृथिव्या दयितस्य चात्मनः
आलक्ष्य किञ्चिच्च विलप्य सा सती चितामथारोपयदद्रिसानुनि
विधाय कृत्यं ह्रदिनीजलाप्लुता दत्त्वोदकं भर्तुरुदारकर्मणः
नत्वा दिविस्थांस्त्रिदशांस्त्रिः परीत्य विवेश वह्निं ध्यायती भर्तृपादौ
विलोक्यानुगतां साध्वीं पृथुं वीरवरं पतिम्‌
तुष्टुवुर्वरदा देवैर्देवपत्न्यः सहस्रशः
कुर्वत्यः कुसुमासारं तस्मिन्मन्दरसानुनि
नदत्स्वमरतूर्येषु गृणन्ति स्म परस्परम्‌
देव्य ऊचुः
अहो इयं वधूर्धन्या या चैवं भूभुजां पतिम्‌
सर्वात्मना पतिं भेजे यज्ञेशं श्रीर्वधूरिव
सैषा नूनं व्रजत्यूर्ध्वमनु वैन्यं पतिं सती
पश्यतास्मानतीत्यार्चिर्दुर्विभाव्येन कर्मणा
तेषां दुरापं किं त्वन्यन्मर्त्यानां भगवत्पदम्‌
भुवि लोलायुषो ये वै नैष्कर्म्यं साधयन्त्युत
स वञ्चितो बतात्मध्रुक्कृच्छ्रेण महता भुवि
लब्ध्वापवर्ग्यं मानुष्यं विषयेषु विषज्जते
मैत्रेय उवाच
स्तुवतीष्वमरस्त्रीषु पतिलोकं गता वधूः
यं वा आत्मविदां धुर्यो वैन्यः प्रापाच्युताश्रयः
इत्थम्भूतानुभावोऽसौ पृथुः स भगवत्तमः
कीर्तितं तस्य चरितमुद्दामचरितस्य ते
य इदं सुमहत्पुण्यं श्रद्धयावहितः पठेत्‌
श्रावयेच्छृणुयाद्वापि स पृथोः पदवीमियात्‌
ब्राह्मणो ब्रह्मवर्चस्वी राजन्यो जगतीपतिः
वैश्यः पठन्विट्पतिः स्याच्छूद्रः सत्तमतामियात्‌
त्रिः कृत्व इदमाकर्ण्य नरो नार्यथवादृता
अप्रजः सुप्रजतमो निर्धनो धनवत्तमः
अस्पष्टकीर्तिः सुयशा मूर्खो भवति पण्डितः
इदं स्वस्त्ययनं पुंसाममङ्गल्यनिवारणम्‌
धन्यं यशस्यमायुष्यं स्वर्ग्यं कलिमलापहम्‌
धर्मार्थकाममोक्षाणां सम्यक्सिद्धिमभीप्सुभिः
श्रद्धयैतदनुश्राव्यं चतुर्णां कारणं परम्‌
विजयाभिमुखो राजा श्रुत्वैतदभियाति यान्‌
बलिं तस्मै हरन्त्यग्रे राजानः पृथवे यथा
मुक्तान्यसङ्गो भगवत्यमलां भक्तिमुद्वहन्‌
वैन्यस्य चरितं पुण्यं शृणुयाच्छ्रावयेत्पठेत्‌
वैचित्रवीर्याभिहितं महन्माहात्म्यसूचकम्‌
अस्मिन्कृतमतिमर्त्यं पार्थवीं गतिमाप्नुयात्‌
अनुदिनमिदमादरेण शृण्वन्पृथुचरितं प्रथयन्विमुक्तसङ्गः
भगवति भवसिन्धुपोतपादे स च निपुणां लभते रतिं मनुष्यः
मैत्रेय उवाच
विजिताश्वोऽधिराजासीत्पृथुपुत्रः पृथुश्रवाः
यवीयोभ्योऽददात्काष्ठा भ्रातृभ्यो भ्रातृवत्सलः
हर्यक्षायादिशत्प्राचीं धूम्रकेशाय दक्षिणाम्‌
प्रतीचीं वृकसंज्ञाय तुर्यां द्रविणसे विभुः
अन्तर्धानगतिं शक्राल्लब्ध्वान्तर्धानसंज्ञितः
अपत्यत्रयमाधत्त शिखण्डिन्यां सुसम्मतम्‌
पावकः पवमानश्च शुचिरित्यग्नयः पुरा
वसिष्ठशापादुत्पन्नाः पुनर्योगगतिं गताः
अन्तर्धानो नभस्वत्यां हविर्धानमविन्दत
य इन्द्रमश्वहर्तारं विद्वानपि न जघ्निवान्‌
राज्ञां वृत्तिं करादान दण्डशुल्कादिदारुणाम्‌
मन्यमानो दीर्घसत्त्र व्याजेन विससर्ज ह
तत्रापि हंसं पुरुषं परमात्मानमात्मदृक्‌
यजंस्तल्लोकतामाप कुशलेन समाधिना
हविर्धानाद्धविर्धानी विदुरासूत षट्सुतान्‌
बर्हिषदं गयं शुक्लं कृष्णं सत्यं जितव्रतम्‌
बर्हिषत्सुमहाभागो हाविर्धानिः प्रजापतिः
क्रियाकाण्डेषु निष्णातो योगेषु च कुरूद्वह
यस्येदं देवयजनमनुयज्ञं वितन्वतः
प्राचीनाग्रैः कुशैरासीदास्तृतं वसुधातलम्‌
सामुद्रीं देवदेवोक्तामुपयेमे शतद्रुतिम्‌
यां वीक्ष्य चारुसर्वाङ्गीं किशोरीं सुष्ठ्वलङ्कृताम्‌
परिक्रमन्तीमुद्वाहे चकमेऽग्निः शुकीमिव
विबुधासुरगन्धर्व मुनिसिद्धनरोरगाः
विजिताः सूर्यया दिक्षु क्वणयन्त्यैव नूपुरैः
प्राचीनबर्हिषः पुत्राः शतद्रुत्यां दशाभवन्‌
तुल्यनामव्रताः सर्वे धर्मस्नाताः प्रचेतसः
पित्रादिष्टाः प्रजासर्गे तपसेऽर्णवमाविशन्‌
दशवर्षसहस्राणि तपसार्चंस्तपस्पतिम्‌
यदुक्तं पथि दृष्टेन गिरिशेन प्रसीदता
तद्ध्यायन्तो जपन्तश्च पूजयन्तश्च संयताः
विदुर उवाच
प्रचेतसां गिरित्रेण यथासीत्पथि सङ्गमः
यदुताह हरः प्रीतस्तन्नो ब्रह्मन्वदार्थवत्‌
सङ्गमः खलु विप्रर्षे शिवेनेह शरीरिणाम्‌
दुर्लभो मुनयो दध्युरसङ्गाद्यमभीप्सितम्‌
आत्मारामोऽपि यस्त्वस्य लोककल्पस्य राधसे
शक्त्या युक्तो विचरति घोरया भगवान्भवः
मैत्रेय उवाच
प्रचेतसः पितुर्वाक्यं शिरसादाय साधवः
दिशं प्रतीचीं प्रययुस्तपस्यादृतचेतसः
ससमुद्रमुप विस्तीर्णमपश्यन्सुमहत्सरः
महन्मन इव स्वच्छं प्रसन्नसलिलाशयम्‌
नीलरक्तोत्पलाम्भोज कह्लारेन्दीवराकरम्‌
हंससारसचक्राह्व कारण्डवनिकूजितम्‌
मत्तभ्रमरसौस्वर्य हृष्टरोमलताङ्घ्रिपम्‌
पद्मकोशरजो दिक्षु विक्षिपत्पवनोत्सवम्‌
तत्र गान्धर्वमाकर्ण्य दिव्यमार्गमनोहरम्‌
विसिस्म्यू राजपुत्रास्ते मृदङ्गपणवाद्यनु
तर्ह्येव सरसस्तस्मान्निष्क्रामन्तं सहानुगम्‌
उपगीयमानममर प्रवरं विबुधानुगैः
तप्तहेमनिकायाभं शितिकण्ठं त्रिलोचनम्‌
प्रसादसुमुखं वीक्ष्य प्रणेमुर्जातकौतुकाः
स तान्प्रपन्नार्तिहरो भगवान्धर्मवत्सलः
धर्मज्ञान्शीलसम्पन्नान्प्रीतः प्रीतानुवाच ह
श्रीरुद्र उवाच
यूयं वेदिषदः पुत्रा विदितं वश्चिकीर्षितम्‌
अनुग्रहाय भद्रं व एवं मे दर्शनं कृतम्‌
यः परं रंहसः साक्षात्त्रिगुणाज्जीवसंज्ञितात्‌
भगवन्तं वासुदेवं प्रपन्नः स प्रियो हि मे
स्वधर्मनिष्ठः शतजन्मभिः पुमान्विरिञ्चतामेति ततः परं हि माम्‌
अव्याकृतं भागवतोऽथ वैष्णवं पदं यथाहं विबुधाः कलात्यये
अथ भागवता यूयं प्रियाः स्थ भगवान्यथा
न मद्भागवतानां च प्रेयानन्योऽस्ति कर्हिचित्‌
इदं विविक्तं जप्तव्यं पवित्रं मङ्गलं परम्‌
निःश्रेयसकरं चापि श्रूयतां तद्वदामि वः
मैत्रेय उवाच
इत्यनुक्रोशहृदयो भगवानाह ताञ्छिवः
बद्धाञ्जलीन्राजपुत्रान्नारायणपरो वचः
श्रीरुद्र उवाच
जितं त आत्मविद्वर्य स्वस्तये स्वस्तिरस्तु मे
भवताराधसा राद्धं सर्वस्मा आत्मने नमः
नमः पङ्कजनाभाय भूतसूक्ष्मेन्द्रियात्मने
वासुदेवाय शान्ताय कूटस्थाय स्वरोचिषे
सङ्कर्षणाय सूक्ष्माय दुरन्तायान्तकाय च
नमो विश्वप्रबोधाय प्रद्युम्नायान्तरात्मने
नमो नमोऽनिरुद्धाय हृषीकेशेन्द्रियात्मने
नमः परमहंसाय पूर्णाय निभृतात्मने
स्वर्गापवर्गद्वाराय नित्यं शुचिषदे नमः
नमो हिरण्यवीर्याय चातुर्होत्राय तन्तवे
नम ऊर्ज इषे त्रय्याः पतये यज्ञरेतसे
तृप्तिदाय च जीवानां नमः सर्वरसात्मने
सर्वसत्त्वात्मदेहाय विशेषाय स्थवीयसे
नमस्त्रैलोक्यपालाय सह ओजोबलाय च
अर्थलिङ्गाय नभसे नमोऽन्तर्बहिरात्मने
नमः पुण्याय लोकाय अमुष्मै भूरिवर्चसे
प्रवृत्ताय निवृत्ताय पितृदेवाय कर्मणे
नमोऽधर्मविपाकाय मृत्यवे दुःखदाय च
नमस्त आशिषामीश मनवे कारणात्मने
नमो धर्माय बृहते कृष्णायाकुण्ठमेधसे
पुरुषाय पुराणाय साङ्ख्ययोगेश्वराय च
शक्तित्रयसमेताय मीढुषेऽहङ्कृतात्मने
चेतआकूतिरूपाय नमो वाचो विभूतये
दर्शनं नो दिदृक्षूणां देहि भागवतार्चितम्‌
रूपं प्रियतमं स्वानां सर्वेन्द्रियगुणाञ्जनम्‌
स्निग्धप्रावृड्घनश्यामं सर्वसौन्दर्यसङ्ग्रहम्‌
चार्वायतचतुर्बाहु सुजातरुचिराननम्‌
पद्मकोशपलाशाक्षं सुन्दरभ्रु सुनासिकम्‌
सुद्विजं सुकपोलास्यं समकर्णविभूषणम्‌
प्रीतिप्रहसितापाङ्गमलकै रूपशोभितम्‌
लसत्पङ्कजकिञ्जल्क दुकूलं मृष्टकुण्डलम्‌
स्फुरत्किरीटवलय हारनूपुरमेखलम्‌
शङ्खचक्रगदापद्म मालामण्युत्तमर्द्धिमत्‌
सिंहस्कन्धत्विषो बिभ्रत्सौभगग्रीवकौस्तुभम्‌
श्रियानपायिन्या क्षिप्त निकषाश्मोरसोल्लसत्‌
पूररेचकसंविग्न वलिवल्गुदलोदरम्‌
प्रतिसङ्क्रामयद्विश्वं नाभ्यावर्तगभीरया
श्यामश्रोण्यधिरोचिष्णु दुकूलस्वर्णमेखलम्‌
समचार्वङ्घ्रिजङ्घोरु निम्नजानुसुदर्शनम्‌
पदा शरत्पद्मपलाशरोचिषा नखद्युभिर्नोऽन्तरघं विधुन्वता
प्रदर्शय स्वीयमपास्तसाध्वसं पदं गुरो मार्गगुरुस्तमोजुषाम्‌
एतद्रूपमनुध्येयमात्मशुद्धिमभीप्सताम्‌
यद्भक्तियोगोऽभयदः स्वधर्ममनुतिष्ठताम्‌
भवान्भक्तिमता लभ्यो दुर्लभः सर्वदेहिनाम्‌
स्वाराज्यस्याप्यभिमत एकान्तेनात्मविद्गतिः
तं दुराराध्यमाराध्य सतामपि दुरापया
एकान्तभक्त्या को वाञ्छेत्पादमूलं विना बहिः
यत्र निर्विष्टमरणं कृतान्तो नाभिमन्यते
विश्वं विध्वंसयन्वीर्य शौर्यविस्फूर्जितभ्रुवा
क्षणार्धेनापि तुलये न स्वर्गं नापुनर्भवम्‌
भगवत्सङ्गिसङ्गस्य मर्त्यानां किमुताशिषः
अथानघाङ्घ्रेस्तव कीर्तितीर्थयोरन्तर्बहिःस्नानविधूतपाप्मनाम्‌
भूतेष्वनुक्रोशसुसत्त्वशीलिनां स्यात्सङ्गमोऽनुग्रह एष नस्तव
न यस्य चित्तं बहिरर्थविभ्रमं तमोगुहायां च विशुद्धमाविशत्‌
यद्भक्तियोगानुगृहीतमञ्जसा मुनिर्विचष्टे ननु तत्र ते गतिम्‌
यत्रेदं व्यज्यते विश्वं विश्वस्मिन्नवभाति यत्‌
तत्त्वं ब्रह्म परं ज्योतिराकाशमिव विस्तृतम्‌
यो माययेदं पुरुरूपयासृजद्बिभर्ति भूयः क्षपयत्यविक्रियः
यद्भेदबुद्धिः सदिवात्मदुःस्थया त्वमात्मतन्त्रं भगवन्प्रतीमहि
क्रियाकलापैरिदमेव योगिनः श्रद्धान्विताः साधु यजन्ति सिद्धये
भूतेन्द्रियान्तःकरणोपलक्षितं वेदे च तन्त्रे च त एव कोविदाः
त्वमेक आद्यः पुरुषः सुप्तशक्तिस्तया रजःसत्त्वतमो विभिद्यते
महानहं खं मरुदग्निवार्धराः सुरर्षयो भूतगणा इदं यतः
सृष्टं स्वशक्त्येदमनुप्रविष्टश्चतुर्विधं पुरमात्मांशकेन
अथो विदुस्तं पुरुषं सन्तमन्तर्भुङ्क्ते हृषीकैर्मधु सारघं यः
स एष लोकानतिचण्डवेगो विकर्षसि त्वं खलु कालयानः
भूतानि भूतैरनुमेयतत्त्वो घनावलीर्वायुरिवाविषह्यः
प्रमत्तमुच्चैरिति कृत्यचिन्तया प्रवृद्धलोभं विषयेषु लालसम्‌
त्वमप्रमत्तः सहसाभिपद्यसे क्षुल्लेलिहानोऽहिरिवाखुमन्तकः
कस्त्वत्पदाब्जं विजहाति पण्डितो यस्तेऽवमानव्ययमानकेतनः
विशङ्कयास्मद्गुरुरर्चति स्म यद्विनोपपत्तिं मनवश्चतुर्दश
अथ त्वमसि नो ब्रह्मन्परमात्मन्विपश्चिताम्‌
विश्वं रुद्रभयध्वस्तमकुतश्चिद्भया गतिः
इदं जपत भद्रं वो विशुद्धा नृपनन्दनाः
स्वधर्ममनुतिष्ठन्तो भगवत्यर्पिताशयाः
तमेवात्मानमात्मस्थं सर्वभूतेष्ववस्थितम्‌
पूजयध्वं गृणन्तश्च ध्यायन्तश्चासकृद्धरिम्‌
योगादेशमुपासाद्य धारयन्तो मुनिव्रताः
समाहितधियः सर्व एतदभ्यसतादृताः
इदमाह पुरास्माकं भगवान्विश्वसृक्पतिः
भृग्वादीनामात्मजानां सिसृक्षुः संसिसृक्षताम्‌
ते वयं नोदिताः सर्वे प्रजासर्गे प्रजेश्वराः
अनेन ध्वस्ततमसः सिसृक्ष्मो विविधाः प्रजाः
अथेदं नित्यदा युक्तो जपन्नवहितः पुमान्‌
अचिराच्छ्रेय आप्नोति वासुदेवपरायणः
श्रेयसामिह सर्वेषां ज्ञानं निःश्रेयसं परम्‌
सुखं तरति दुष्पारं ज्ञाननौर्व्यसनार्णवम्‌
य इमं श्रद्धया युक्तो मद्गीतं भगवत्स्तवम्‌
अधीयानो दुराराध्यं हरिमाराधयत्यसौ
विन्दते पुरुषोऽमुष्माद्यद्यदिच्छत्यसत्वरम्‌
मद्गीतगीतात्सुप्रीताच्छ्रेयसामेकवल्लभात्‌
इदं यः कल्य उत्थाय प्राञ्जलिः श्रद्धयान्वितः
शृणुयाच्छ्रावयेन्मर्त्यो मुच्यते कर्मबन्धनैः
गीतं मयेदं नरदेवनन्दनाः परस्य पुंसः परमात्मनः स्तवम्‌
जपन्त एकाग्रधियस्तपो महत्चरध्वमन्ते तत आप्स्यथेप्सितम्‌
मैत्रेय उवाच
इति सन्दिश्य भगवान्बार्हिषदैरभिपूजितः
पश्यतां राजपुत्राणां तत्रैवान्तर्दधे हरः
रुद्रगीतं भगवतः स्तोत्रं सर्वे प्रचेतसः
जपन्तस्ते तपस्तेपुर्वर्षाणामयुतं जले
प्राचीनबर्हिषं क्षत्तः कर्मस्वासक्तमानसम्‌
नारदोऽध्यात्मतत्त्वज्ञः कृपालुः प्रत्यबोधयत्‌
श्रेयस्त्वं कतमद्राजन्कर्मणात्मन ईहसे
दुःखहानिः सुखावाप्तिः श्रेयस्तन्नेह चेष्यते
राजोवाच
न जानामि महाभाग परं कर्मापविद्धधीः
ब्रूहि मे विमलं ज्ञानं येन मुच्येय कर्मभिः
गृहेषु कूटधर्मेषु पुत्रदारधनार्थधीः
न परं विन्दते मूढो भ्राम्यन्संसारवर्त्मसु
नारद उवाच
भो भोः प्रजापते राजन्पशून्पश्य त्वयाध्वरे
संज्ञापिताञ्जीवसङ्घान्निर्घृणेन सहस्रशः
एते त्वां सम्प्रतीक्षन्ते स्मरन्तो वैशसं तव
सम्परेतमयःकूटैश्छिन्दन्त्युत्थितमन्यवः
अत्र ते कथयिष्येऽमुमितिहासं पुरातनम्‌
पुरञ्जनस्य चरितं निबोध गदतो मम
आसीत्पुरञ्जनो नाम राजा राजन्बृहच्छ्रवाः
तस्याविज्ञातनामासीत्सखाविज्ञातचेष्टितः
सोऽन्वेषमाणः शरणं बभ्राम पृथिवीं प्रभुः
नानुरूपं यदाविन्ददभूत्स विमना इव
न साधु मेने ताः सर्वा भूतले यावतीः पुरः
कामान्कामयमानोऽसौ तस्य तस्योपपत्तये
स एकदा हिमवतो दक्षिणेष्वथ सानुषु
ददर्श नवभिर्द्वार्भिः पुरं लक्षितलक्षणाम्‌
प्राकारोपवनाट्टाल परिखैरक्षतोरणैः
स्वर्णरौप्यायसैः शृङ्गैः सङ्कुलां सर्वतो गृहैः
नीलस्फटिकवैदूर्य मुक्तामरकतारुणैः
कॢप्तहर्म्यस्थलीं दीप्तां श्रिया भोगवतीमिव
सभाचत्वररथ्याभिराक्रीडायतनापणैः
चैत्यध्वजपताकाभिर्युक्तां विद्रुमवेदिभिः
पुर्यास्तु बाह्योपवने दिव्यद्रुमलताकुले
नदद्विहङ्गालिकुल कोलाहलजलाशये
हिमनिर्झरविप्रुष्मत्‌ कुसुमाकरवायुना
चलत्प्रवालविटप नलिनीतटसम्पदि
नानारण्यमृगव्रातैरनाबाधे मुनिव्रतैः
आहूतं मन्यते पान्थो यत्र कोकिलकूजितैः
यदृच्छयागतां तत्र ददर्श प्रमदोत्तमाम्‌
भृत्यैर्दशभिरायान्तीमेकैकशतनायकैः
अञ्चशीर्षाहिना गुप्तां प्रतीहारेण सर्वतः
अन्वेषमाणामृषभमप्रौढां कामरूपिणीम्‌
सुनासां सुदतीं बालां सुकपोलां वराननाम्‌
समविन्यस्तकर्णाभ्यां बिभ्रतीं कुण्डलश्रियम्‌
पिशङ्गनीवीं सुश्रोणीं श्यामां कनकमेखलाम्‌
पद्भ्यां क्वणद्भ्यां चलन्तीं नूपुरैर्देवतामिव
स्तनौ व्यञ्जितकैशोरौ समवृत्तौ निरन्तरौ
वस्त्रान्तेन निगूहन्तीं व्रीडया गजगामिनीम्‌
तामाह ललितं वीरः सव्रीडस्मितशोभनाम्‌
स्निग्धेनापाङ्गपुङ्खेन स्पृष्टः प्रेमोद्भ्रमद्भ्रुवा
का त्वं कञ्जपलाशाक्षि कस्यासीह कुतः सति
इमामुप पुरीं भीरु किं चिकीर्षसि शंस मे
क एतेऽनुपथा ये त एकादश महाभटाः
एता वा ललनाः सुभ्रु कोऽयं तेऽहिः पुरःसरः
त्वं ह्रीर्भवान्यस्यथ वाग्रमा पतिं विचिन्वती किं मुनिवद्रहो वने
त्वदङ्घ्रिकामाप्तसमस्तकामं क्व पद्मकोशः पतितः कराग्रात्‌
नासां वरोर्वन्यतमा भुविस्पृक्पुरीमिमां वीरवरेण साकम्‌
अर्हस्यलङ्कर्तुमदभ्रकर्मणा लोकं परं श्रीरिव यज्ञपुंसा
यदेष मापाङ्गविखण्डितेन्द्रियं सव्रीडभावस्मितविभ्रमद्भ्रुवा
त्वयोपसृष्टो भगवान्मनोभवः प्रबाधतेऽथानुगृहाण शोभने
त्वदाननं सुभ्रु सुतारलोचनं व्यालम्बिनीलालकवृन्दसंवृतम्‌
उन्नीय मे दर्शय वल्गुवाचकं यद्व्रीडया नाभिमुखं शुचिस्मिते
नारद उवाच
इत्थं पुरञ्जनं नारी याचमानमधीरवत्‌
अभ्यनन्दत तं वीरं हसन्ती वीर मोहिता
न विदाम वयं सम्यक्कर्तारं पुरुषर्षभ
आत्मनश्च परस्यापि गोत्रं नाम च यत्कृतम्‌
इहाद्य सन्तमात्मानं विदाम न ततः परम्‌
येनेयं निर्मिता वीर पुरी शरणमात्मनः
एते सखायः सख्यो मे नरा नार्यश्च मानद
सुप्तायां मयि जागर्ति नागोऽयं पालयन्पुरीम्‌
दिष्ट्यागतोऽसि भद्रं ते ग्राम्यान्कामानभीप्ससे
उद्वहिष्यामि तांस्तेऽहं स्वबन्धुभिररिन्दम
इमां त्वमधितिष्ठस्व पुरीं नवमुखीं विभो
मयोपनीतान्गृह्णानः कामभोगान्शतं समाः
कं नु त्वदन्यं रमये ह्यरतिज्ञमकोविदम्‌
असम्परायाभिमुखमश्वस्तनविदं पशुम्‌
धर्मो ह्यत्रार्थकामौ च प्रजानन्दोऽमृतं यशः
लोका विशोका विरजा यान्न केवलिनो विदुः
पितृदेवर्षिमर्त्यानां भूतानामात्मनश्च ह
क्षेम्यं वदन्ति शरणं भवेऽस्मिन्यद्गृहाश्रमः
का नाम वीर विख्यातं वदान्यं प्रियदर्शनम्‌
न वृणीत प्रियं प्राप्तं मादृशी त्वादृशं पतिम्‌
कस्या मनस्ते भुवि भोगिभोगयोः स्त्रिया न सज्जेद्भुजयोर्महाभुज
योऽनाथवर्गाधिमलं घृणोद्धत स्मितावलोकेन चरत्यपोहितुम्‌
नारद उवाच
इति तौ दम्पती तत्र समुद्य समयं मिथः
तां प्रविश्य पुरीं राजन्मुमुदाते शतं समाः
उपगीयमानो ललितं तत्र तत्र च गायकैः
क्रीडन्परिवृतः स्त्रीभिर्ह्रदिनीमाविशच्छुचौ
सप्तोपरि कृता द्वारः पुरस्तस्यास्तु द्वे अधः
पृथग्विषयगत्यर्थं तस्यां यः कश्चनेश्वरः
पञ्च द्वारस्तु पौरस्त्या दक्षिणैका तथोत्तरा
पश्चिमे द्वे अमूषां ते नामानि नृप वर्णये
खद्योताविर्मुखी च प्राग्द्वारावेकत्र निर्मिते
विभ्राजितं जनपदं याति ताभ्यां द्युमत्सखः
नलिनी नालिनी च प्राग्द्वारावेकत्र निर्मिते
अवधूतसखस्ताभ्यां विषयं याति सौरभम्‌
मुख्या नाम पुरस्ताद्द्वास्तयापणबहूदनौ
विषयौ याति पुरराड्रसज्ञविपणान्वितः
पितृहूर्नृप पुर्या द्वार्दक्षिणेन पुरञ्जनः
राष्ट्रं दक्षिणपञ्चालं याति श्रुतधरान्वितः
देवहूर्नाम पुर्या द्वा उत्तरेण पुरञ्जनः
राष्ट्रमुत्तरपञ्चालं याति श्रुतधरान्वितः
आसुरी नाम पश्चाद्द्वास्तया याति पुरञ्जनः
ग्रामकं नाम विषयं दुर्मदेन समन्वितः
निरृतिर्नाम पश्चाद्द्वास्तया याति पुरञ्जनः
वैशसं नाम विषयं लुब्धकेन समन्वितः
अन्धावमीषां पौराणां निर्वाक्पेशस्कृतावुभौ
अक्षण्वतामधिपतिस्ताभ्यां याति करोति च
स यर्ह्यन्तःपुरगतो विषूचीनसमन्वितः
मोहं प्रसादं हर्षं वा याति जायात्मजोद्भवम्‌
एवं कर्मसु संसक्तः कामात्मा वञ्चितोऽबुधः
महिषी यद्यदीहेत तत्तदेवान्ववर्तत
क्वचित्पिबन्त्यां पिबति मदिरां मदविह्वलः
अश्नन्त्यां क्वचिदश्नाति जक्षत्यां सह जक्षिति
क्वचिद्गायति गायन्त्यां रुदत्यां रुदति क्वचित्‌
क्वचिद्धसन्त्यां हसति जल्पन्त्यामनु जल्पति
क्वचिद्धावति धावन्त्यां तिष्ठन्त्यामनु तिष्ठति
अनु शेते शयानायामन्वास्ते क्वचिदासतीम्‌
क्वचिच्छृणोति शृण्वन्त्यां पश्यन्त्यामनु पश्यति
क्वचिज्जिघ्रति जिघ्रन्त्यां स्पृशन्त्यां स्पृशति क्वचित्‌
क्वचिच्च शोचतीं जायामनु शोचति दीनवत्‌
अनु हृष्यति हृष्यन्त्यां मुदितामनु मोदते
विप्रलब्धो महिष्यैवं सर्वप्रकृतिवञ्चितः
नेच्छन्ननुकरोत्यज्ञः क्लैब्यात्क्रीडामृगो यथा
नारद उवाच
स एकदा महेष्वासो रथं पञ्चाश्वमाशुगम्‌
द्वीषं द्विचक्रमेकाक्षं त्रिवेणुं पञ्चबन्धुरम्‌
एकरश्म्येकदमनमेकनीडं द्विकूबरम्‌
पञ्चप्रहरणं सप्त वरूथं पञ्चविक्रमम्‌
हैमोपस्करमारुह्य स्वर्णवर्माक्षयेषुधिः
एकादशचमूनाथः पञ्चप्रस्थमगाद्वनम्‌
चचार मृगयां तत्र दृप्त आत्तेषुकार्मुकः
विहाय जायामतदर्हां मृगव्यसनलालसः
आसुरीं वृत्तिमाश्रित्य घोरात्मा निरनुग्रहः
न्यहनन्निशितैर्बाणैर्वनेषु वनगोचरान्‌
तीर्थेषु प्रतिदृष्टेषु राजा मेध्यान्पशून्वने
यावदर्थमलं लुब्धो हन्यादिति नियम्यते
य एवं कर्म नियतं विद्वान्कुर्वीत मानवः
कर्मणा तेन राजेन्द्र ज्ञानेन न स लिप्यते
अन्यथा कर्म कुर्वाणो मानारूढो निबध्यते
गुणप्रवाहपतितो नष्टप्रज्ञो व्रजत्यधः
तत्र निर्भिन्नगात्राणां चित्रवाजैः शिलीमुखैः
विप्लवोऽभूद्दुःखितानां दुःसहः करुणात्मनाम्‌
शशान्वराहान्महिषान्गवयान्रुरुशल्यकान्‌
मेध्यानन्यांश्च विविधान्विनिघ्नन्श्रममध्यगात्‌
ततः क्षुत्तृट्परिश्रान्तो निवृत्तो गृहमेयिवान्‌
कृतस्नानोचिताहारः संविवेश गतक्लमः
आत्मानमर्हयां चक्रे धूपालेपस्रगादिभिः
साध्वलङ्कृतसर्वाङ्गो महिष्यामादधे मनः
तृप्तो हृष्टः सुदृप्तश्च कन्दर्पाकृष्टमानसः
न व्यचष्ट वरारोहां गृहिणीं गृहमेधिनीम्‌
अन्तःपुरस्त्रियोऽपृच्छद्विमना इव वेदिषत्‌
अपि वः कुशलं रामाः सेश्वरीणां यथा पुरा
न तथैतर्हि रोचन्ते गृहेषु गृहसम्पदः
यदि न स्याद्गृहे माता पत्नी वा पतिदेवता
व्यङ्गे रथ इव प्राज्ञः को नामासीत दीनवत्‌
क्व वर्तते सा ललना मज्जन्तं व्यसनार्णवे
या मामुद्धरते प्रज्ञां दीपयन्ती पदे पदे
रामा ऊचुः
नरनाथ न जानीमस्त्वत्प्रिया यद्व्यवस्यति
भूतले निरवस्तारे शयानां पश्य शत्रुहन्‌
नारद उवाच
पुरञ्जनः स्वमहिषीं निरीक्ष्यावधुतां भुवि
तत्सङ्गोन्मथितज्ञानो वैक्लव्यं परमं ययौ
सान्त्वयन्श्लक्ष्णया वाचा हृदयेन विदूयता
प्रेयस्याः स्नेहसंरम्भ लिङ्गमात्मनि नाभ्यगात्‌
अनुनिन्येऽथ शनकैर्वीरोऽनुनयकोविदः
पस्पर्श पादयुगलमाह चोत्सङ्गलालिताम्‌
पुरञ्जन उवाच
नूनं त्वकृतपुण्यास्ते भृत्या येष्वीश्वराः शुभे
कृतागःस्वात्मसात्कृत्वा शिक्षादण्डं न युञ्जते
परमोऽनुग्रहो दण्डो भृत्येषु प्रभुणार्पितः
बालो न वेद तत्तन्वि बन्धुकृत्यममर्षणः
सा त्वं मुखं सुदति सुभ्र्वनुरागभार व्रीडाविलम्बविलसद्धसितावलोकम्‌
नीलालकालिभिरुपस्कृतमुन्नसं नः स्वानां प्रदर्शय मनस्विनि वल्गुवाक्यम्‌
तस्मिन्दधे दममहं तव वीरपत्नि योऽन्यत्र भूसुरकुलात्कृतकिल्बिषस्तम्‌
पश्ये न वीतभयमुन्मुदितं त्रिलोक्यामन्यत्र वै मुररिपोरितरत्र दासात्‌
वक्त्रं न ते वितिलकं मलिनं विहर्षं संरम्भभीममविमृष्टमपेतरागम्‌
पश्ये स्तनावपि शुचोपहतौ सुजातौ बिम्बाधरं विगतकुङ्कुमपङ्करागम्‌
तन्मे प्रसीद सुहृदः कृतकिल्बिषस्य स्वैरं गतस्य मृगयां व्यसनातुरस्य
का देवरं वशगतं कुसुमास्त्रवेग विस्रस्तपौंस्नमुशती न भजेत कृत्ये
नारद उवाच
इत्थं पुरञ्जनं सध्र्यग्वशमानीय विभ्रमैः
पुरञ्जनी महाराज रेमे रमयती पतिम्‌
स राजा महिषीं राजन्सुस्नातां रुचिराननाम्‌
कृतस्वस्त्ययनां तृप्तामभ्यनन्ददुपागताम्‌
तयोपगूढः परिरब्धकन्धरो रहोऽनुमन्त्रैरपकृष्टचेतनः
न कालरंहो बुबुधे दुरत्ययं दिवा निशेति प्रमदापरिग्रहः
शयान उन्नद्धमदो महामना महार्हतल्पे महिषीभुजोपधिः
तामेव वीरो मनुते परं यतस्तमोऽभिभूतो न निजं परं च यत्‌
तयैवं रममाणस्य कामकश्मलचेतसः
क्षणार्धमिव राजेन्द्र व्यतिक्रान्तं नवं वयः
तस्यामजनयत्पुत्रान्पुरञ्जन्यां पुरञ्जनः
शतान्येकादश विराडायुषोऽर्धमथात्यगात्‌
दुहित्ऱ्‌^ईर्दशोत्तरशतं पितृमातृयशस्करीः
शीलौदार्यगुणोपेताः पौरञ्जन्यः प्रजापते
स पञ्चालपतिः पुत्रान्पितृवंशविवर्धनान्‌
दारैः संयोजयामास दुहित्ऱ्‌^ईः सदृशैर्वरैः
पुत्राणां चाभवन्पुत्रा एकैकस्य शतं शतम्‌
यैर्वै पौरञ्जनो वंशः पञ्चालेषु समेधितः
तेषु तद्रिक्थहारेषु गृहकोशानुजीविषु
निरूढेन ममत्वेन विषयेष्वन्वबध्यत
ईजे च क्रतुभिर्घोरैर्दीक्षितः पशुमारकैः
देवान्पित्ऱ्‌^ईन्भूतपतीन्नानाकामो यथा भवान्‌
युक्तेष्वेवं प्रमत्तस्य कुटुम्बासक्तचेतसः
आससाद स वै कालो योऽप्रियः प्रिययोषिताम्‌
चण्डवेग इति ख्यातो गन्धर्वाधिपतिर्नृप
गन्धर्वास्तस्य बलिनः षष्ट्युत्तरशतत्रयम्‌
गन्धर्व्यस्तादृशीरस्य मैथुन्यश्च सितासिताः
परिवृत्त्या विलुम्पन्ति सर्वकामविनिर्मिताम्‌
ते चण्डवेगानुचराः पुरञ्जनपुरं यदा
हर्तुमारेभिरे तत्र प्रत्यषेधत्प्रजागरः
स सप्तभिः शतैरेको विंशत्या च शतं समाः
पुरञ्जनपुराध्यक्षो गन्धर्वैर्युयुधे बली
क्षीयमाणे स्वसम्बन्धे एकस्मिन्बहुभिर्युधा
चिन्तां परां जगामार्तः सराष्ट्रपुरबान्धवः
स एव पुर्यां मधुभुक्पञ्चालेषु स्वपार्षदैः
उपनीतं बलिं गृह्णन्स्त्रीजितो नाविदद्भयम्‌
कालस्य दुहिता काचित्त्रिलोकीं वरमिच्छती
पर्यटन्ती न बर्हिष्मन्प्रत्यनन्दत कश्चन
दौर्भाग्येनात्मनो लोके विश्रुता दुर्भगेति सा
या तुष्टा राजर्षये तु वृतादात्पूरवे वरम्‌
कदाचिदटमाना सा ब्रह्मलोकान्महीं गतम्‌
वव्रे बृहद्व्रतं मां तु जानती काममोहिता
मयि संरभ्य विपुल मदाच्छापं सुदुःसहम्‌
स्थातुमर्हसि नैकत्र मद्याच्ञाविमुखो मुने
ततो विहतसङ्कल्पा कन्यका यवनेश्वरम्‌
मयोपदिष्टमासाद्य वव्रे नाम्ना भयं पतिम्‌
ऋषभं यवनानां त्वां वृणे वीरेप्सितं पतिम्‌
सङ्कल्पस्त्वयि भूतानां कृतः किल न रिष्यति
द्वाविमावनुशोचन्ति बालावसदवग्रहौ
यल्लोकशास्त्रोपनतं न राति न तदिच्छति
अथो भजस्व मां भद्र भजन्तीं मे दयां कुरु
एतावान्पौरुषो धर्मो यदार्ताननुकम्पते
कालकन्योदितवचो निशम्य यवनेश्वरः
चिकीर्षुर्देवगुह्यं स सस्मितं तामभाषत
मया निरूपितस्तुभ्यं पतिरात्मसमाधिना
नाभिनन्दति लोकोऽयं त्वामभद्रामसम्मताम्‌
त्वमव्यक्तगतिर्भुङ्क्ष्व लोकं कर्मविनिर्मितम्‌
या हि मे पृतनायुक्ता प्रजानाशं प्रणेष्यसि
प्रज्वारोऽयं मम भ्राता त्वं च मे भगिनी भव
चराम्युभाभ्यां लोकेऽस्मिन्नव्यक्तो भीमसैनिकः
नारद उवाच
सैनिका भयनाम्नो ये बर्हिष्मन्दिष्टकारिणः
प्रज्वारकालकन्याभ्यां विचेरुरवनीमिमाम्‌
त एकदा तु रभसा पुरञ्जनपुरीं नृप
रुरुधुर्भौमभोगाढ्यां जरत्पन्नगपालिताम्‌
कालकन्यापि बुभुजे पुरञ्जनपुरं बलात्‌
ययाभिभूतः पुरुषः सद्यो निःसारतामियात्‌
तयोपभुज्यमानां वै यवनाः सर्वतोदिशम्‌
द्वार्भिः प्रविश्य सुभृशं प्रार्दयन्सकलां पुरीम्‌
तस्यां प्रपीड्यमानायामभिमानी पुरञ्जनः
अवापोरुविधांस्तापान्कुटुम्बी ममताकुलः
कन्योपगूढो नष्टश्रीः कृपणो विषयात्मकः
नष्टप्रज्ञो हृतैश्वर्यो गन्धर्वयवनैर्बलात्‌
विशीर्णां स्वपुरीं वीक्ष्य प्रतिकूलाननादृतान्‌
पुत्रान्पौत्रानुगामात्यान्जायां च गतसौहृदाम्‌
आत्मानं कन्यया ग्रस्तं पञ्चालानरिदूषितान्‌
दुरन्तचिन्तामापन्नो न लेभे तत्प्रतिक्रियाम्‌
कामानभिलषन्दीनो यातयामांश्च कन्यया
विगतात्मगतिस्नेहः पुत्रदारांश्च लालयन्‌
गन्धर्वयवनाक्रान्तां कालकन्योपमर्दिताम्‌
हातुं प्रचक्रमे राजा तां पुरीमनिकामतः
भयनाम्नोऽग्रजो भ्राता प्रज्वारः प्रत्युपस्थितः
ददाह तां पुरीं कृत्स्नां भ्रातुः प्रियचिकीर्षया
तस्यां सन्दह्यमानायां सपौरः सपरिच्छदः
कौटुम्बिकः कुटुम्बिन्या उपातप्यत सान्वयः
यवनोपरुद्धायतनो ग्रस्तायां कालकन्यया
पुर्यां प्रज्वारसंसृष्टः पुरपालोऽन्वतप्यत
न शेके सोऽवितुं तत्र पुरुकृच्छ्रोरुवेपथुः
गन्तुमैच्छत्ततो वृक्ष कोटरादिव सानलात्‌
शिथिलावयवो यर्हि गन्धर्वैर्हृतपौरुषः
यवनैररिभी राजन्नुपरुद्धो रुरोद ह
दुहित्ऱ्‌^ईः पुत्रपौत्रांश्च जामिजामातृपार्षदान्‌
स्वत्वावशिष्टं यत्किञ्चिद्गृहकोशपरिच्छदम्‌
अहं ममेति स्वीकृत्य गृहेषु कुमतिर्गृही
दध्यौ प्रमदया दीनो विप्रयोग उपस्थिते
लोकान्तरं गतवति मय्यनाथा कुटुम्बिनी
वर्तिष्यते कथं त्वेषा बालकाननुशोचती
न मय्यनाशिते भुङ्क्ते नास्नाते स्नाति मत्परा
मयि रुष्टे सुसन्त्रस्ता भर्त्सिते यतवाग्भयात्‌
प्रबोधयति माविज्ञं व्युषिते शोककर्शिता
वर्त्मैतद्गृहमेधीयं वीरसूरपि नेष्यति
कथं नु दारका दीना दारकीर्वापरायणाः
वर्तिष्यन्ते मयि गते भिन्ननाव इवोदधौ
एवं कृपणया बुद्ध्या शोचन्तमतदर्हणम्‌
ग्रहीतुं कृतधीरेनं भयनामाभ्यपद्यत
पशुवद्यवनैरेष नीयमानः स्वकं क्षयम्‌
अन्वद्रवन्ननुपथाः शोचन्तो भृशमातुराः
पुरीं विहायोपगत उपरुद्धो भुजङ्गमः
यदा तमेवानु पुरी विशीर्णा प्रकृतिं गता
विकृष्यमाणः प्रसभं यवनेन बलीयसा
नाविन्दत्तमसाविष्टः सखायं सुहृदं पुरः
तं यज्ञपशवोऽनेन संज्ञप्ता येऽदयालुना
कुठारैश्चिच्छिदुः क्रुद्धाः स्मरन्तोऽमीवमस्य तत्‌
अनन्तपारे तमसि मग्नो नष्टस्मृतिः समाः
शाश्वतीरनुभूयार्तिं प्रमदासङ्गदूषितः
तामेव मनसा गृह्णन्बभूव प्रमदोत्तमा
अनन्तरं विदर्भस्य राजसिंहस्य वेश्मनि
उपयेमे वीर्यपणां वैदर्भीं मलयध्वजः
युधि निर्जित्य राजन्यान्पाण्ड्यः परपुरञ्जयः
तस्यां स जनयां चक्र आत्मजामसितेक्षणाम्‌
यवीयसः सप्त सुतान्सप्त द्रविडभूभृतः
एकैकस्याभवत्तेषां राजन्नर्बुदमर्बुदम्‌
भोक्ष्यते यद्वंशधरैर्मही मन्वन्तरं परम्‌
अगस्त्यः प्राग्दुहितरमुपयेमे धृतव्रताम्‌
यस्यां दृढच्युतो जात इध्मवाहात्मजो मुनिः
विभज्य तनयेभ्यः क्ष्मां राजर्षिर्मलयध्वजः
आरिराधयिषुः कृष्णं स जगाम कुलाचलम्‌
हित्वा गृहान्सुतान्भोगान्वैदर्भी मदिरेक्षणा
अन्वधावत पाण्ड्येशं ज्योत्स्नेव रजनीकरम्‌
तत्र चन्द्रवसा नाम ताम्रपर्णी वटोदका
तत्पुण्यसलिलैर्नित्यमुभयत्रात्मनो मृजन्‌
कन्दाष्टिभिर्मूलफलैः पुष्पपर्णैस्तृणोदकैः
वर्तमानः शनैर्गात्र कर्शनं तप आस्थितः
शीतोष्णवातवर्षाणि क्षुत्पिपासे प्रियाप्रिये
सुखदुःखे इति द्वन्द्वान्यजयत्समदर्शनः
तपसा विद्यया पक्व कषायो नियमैर्यमैः
युयुजे ब्रह्मण्यात्मानं विजिताक्षानिलाशयः
आस्ते स्थाणुरिवैकत्र दिव्यं वर्षशतं स्थिरः
वासुदेवे भगवति नान्यद्वेदोद्वहन्रतिम्‌
स व्यापकतयात्मानं व्यतिरिक्ततयात्मनि
विद्वान्स्वप्न इवामर्श साक्षिणं विरराम ह
साक्षाद्भगवतोक्तेन गुरुणा हरिणा नृप
विशुद्धज्ञानदीपेन स्फुरता विश्वतोमुखम्‌
परे ब्रह्मणि चात्मानं परं ब्रह्म तथात्मनि
वीक्षमाणो विहायेक्षामस्मादुपरराम ह
पतिं परमधर्मज्ञं वैदर्भी मलयध्वजम्‌
प्रेम्णा पर्यचरद्धित्वा भोगान्सा पतिदेवता
चीरवासा व्रतक्षामा वेणीभूतशिरोरुहा
बभावुप पतिं शान्ता शिखा शान्तमिवानलम्‌
अजानती प्रियतमं यदोपरतमङ्गना
सुस्थिरासनमासाद्य यथापूर्वमुपाचरत्‌
यदा नोपलभेताङ्घ्रावूष्माणं पत्युरर्चती
आसीत्संविग्नहृदया यूथभ्रष्टा मृगी यथा
आत्मानं शोचती दीनमबन्धुं विक्लवाश्रुभिः
स्तनावासिच्य विपिने सुस्वरं प्ररुरोद सा
उत्तिष्ठोत्तिष्ठ राजर्षे इमामुदधिमेखलाम्‌
दस्युभ्यः क्षत्रबन्धुभ्यो बिभ्यतीं पातुमर्हसि
एवं विलपन्ती बाला विपिनेऽनुगता पतिम्‌
पतिता पादयोर्भर्तू रुदत्यश्रूण्यवर्तयत्‌
चितिं दारुमयीं चित्वा तस्यां पत्युः कलेवरम्‌
आदीप्य चानुमरणे विलपन्ती मनो दधे
तत्र पूर्वतरः कश्चित्सखा ब्राह्मण आत्मवान्‌
सान्त्वयन्वल्गुना साम्ना तामाह रुदतीं प्रभो
ब्राह्मण उवाच
का त्वं कस्यासि को वायं शयानो यस्य शोचसि
जानासि किं सखायं मां येनाग्रे विचचर्थ ह
अपि स्मरसि चात्मानमविज्ञातसखं सखे
हित्वा मां पदमन्विच्छन्भौमभोगरतो गतः
हंसावहं च त्वं चार्य सखायौ मानसायनौ
अभूतामन्तरा वौकः सहस्रपरिवत्सरान्‌
स त्वं विहाय मां बन्धो गतो ग्राम्यमतिर्महीम्‌
विचरन्पदमद्राक्षीः कयाचिन्निर्मितं स्त्रिया
पञ्चारामं नवद्वारमेकपालं त्रिकोष्ठकम्‌
षट्कुलं पञ्चविपणं पञ्चप्रकृति स्त्रीधवम्‌
पञ्चेन्द्रियार्था आरामा द्वारः प्राणा नव प्रभो
तेजोऽबन्नानि कोष्ठानि कुलमिन्द्रियसङ्ग्रहः
विपणस्तु क्रियाशक्तिर्भूतप्रकृतिरव्यया
शक्त्यधीशः पुमांस्त्वत्र प्रविष्टो नावबुध्यते
तस्मिंस्त्वं रामया स्पृष्टो रममाणोऽश्रुतस्मृतिः
तत्सङ्गादीदृशीं प्राप्तो दशां पापीयसीं प्रभो
न त्वं विदर्भदुहिता नायं वीरः सुहृत्तव
न पतिस्त्वं पुरञ्जन्या रुद्धो नवमुखे यया
माया ह्येषा मया सृष्टा यत्पुमांसं स्त्रियं सतीम्‌
मन्यसे नोभयं यद्वै हंसौ पश्यावयोर्गतिम्‌
अहं भवान्न चान्यस्त्वं त्वमेवाहं विचक्ष्व भोः
न नौ पश्यन्ति कवयश्छिद्रं जातु मनागपि
यथा पुरुष आत्मानमेकमादर्शचक्षुषोः
द्विधाभूतमवेक्षेत तथैवान्तरमावयोः
एवं स मानसो हंसो हंसेन प्रतिबोधितः
स्वस्थस्तद्व्यभिचारेण नष्टामाप पुनः स्मृतिम्‌
बर्हिष्मन्नेतदध्यात्मं पारोक्ष्येण प्रदर्शितम्‌
यत्परोक्षप्रियो देवो भगवान्विश्वभावनः
प्राचीनबर्हिरुवाच
भगवंस्ते वचोऽस्माभिर्न सम्यगवगम्यते
कवयस्तद्विजानन्ति न वयं कर्ममोहिताः
नारद उवाच
पुरुषं पुरञ्जनं विद्याद्यद्व्यनक्त्यात्मनः पुरम्‌
एकद्वित्रिचतुष्पादं बहुपादमपादकम्‌
योऽविज्ञाताहृतस्तस्य पुरुषस्य सखेश्वरः
यन्न विज्ञायते पुम्भिर्नामभिर्वा क्रियागुणैः
यदा जिघृक्षन्पुरुषः कार्त्स्न्येन प्रकृतेर्गुणान्‌
नवद्वारं द्विहस्ताङ्घ्रि तत्रामनुत साध्विति
बुद्धिं तु प्रमदां विद्यान्ममाहमिति यत्कृतम्‌
यामधिष्ठाय देहेऽस्मिन्पुमान्भुङ्क्तेऽक्षभिर्गुणान्‌
सखाय इन्द्रियगणा ज्ञानं कर्म च यत्कृतम्‌
सख्यस्तद्वृत्तयः प्राणः पञ्चवृत्तिर्यथोरगः
बृहद्बलं मनो विद्यादुभयेन्द्रियनायकम्‌
पञ्चालाः पञ्च विषया यन्मध्ये नवखं पुरम्‌
अक्षिणी नासिके कर्णौ मुखं शिश्नगुदाविति
द्वे द्वे द्वारौ बहिर्याति यस्तदिन्द्रियसंयुतः
अक्षिणी नासिके आस्यमिति पञ्च पुरः कृताः
दक्षिणा दक्षिणः कर्ण उत्तरा चोत्तरः स्मृतः
पश्चिमे इत्यधो द्वारौ गुदं शिश्नमिहोच्यते
खद्योताविर्मुखी चात्र नेत्रे एकत्र निर्मिते
रूपं विभ्राजितं ताभ्यां विचष्टे चक्षुषेश्वरः
नलिनी नालिनी नासे गन्धः सौरभ उच्यते
घ्राणोऽवधूतो मुख्यास्यं विपणो वाग्रसविद्रसः
आपणो व्यवहारोऽत्र चित्रमन्धो बहूदनम्‌
पितृहूर्दक्षिणः कर्ण उत्तरो देवहूः स्मृतः
प्रवृत्तं च निवृत्तं च शास्त्रं पञ्चालसंज्ञितम्‌
पितृयानं देवयानं श्रोत्राच्छ्रुतधराद्व्रजेत्‌
आसुरी मेढ्रमर्वाग्द्वार्व्यवायो ग्रामिणां रतिः
उपस्थो दुर्मदः प्रोक्तो निरृतिर्गुद उच्यते
वैशसं नरकं पायुर्लुब्धकोऽन्धौ तु मे शृणु
हस्तपादौ पुमांस्ताभ्यां युक्तो याति करोति च
अन्तःपुरं च हृदयं विषूचिर्मन उच्यते
तत्र मोहं प्रसादं वा हर्षं प्राप्नोति तद्गुणैः
यथा यथा विक्रियते गुणाक्तो विकरोति वा
तथा तथोपद्रष्टात्मा तद्वृत्तीरनुकार्यते
देहो रथस्त्विन्द्रियाश्वः संवत्सररयोऽगतिः
द्विकर्मचक्रस्त्रिगुण ध्वजः पञ्चासुबन्धुरः
मनोरश्मिर्बुद्धिसूतो हृन्नीडो द्वन्द्वकूबरः
पञ्चेन्द्रियार्थप्रक्षेपः सप्तधातुवरूथकः
आकूतिर्विक्रमो बाह्यो मृगतृष्णां प्रधावति
एकादशेन्द्रियचमूः पञ्चसूनाविनोदकृत्‌
संवत्सरश्चण्डवेगः कालो येनोपलक्षितः
तस्याहानीह गन्धर्वा गन्धर्व्यो रात्रयः स्मृताः
हरन्त्यायुः परिक्रान्त्या षष्ट्युत्तरशतत्रयम्‌
कालकन्या जरा साक्षाल्लोकस्तां नाभिनन्दति
स्वसारं जगृहे मृत्युः क्षयाय यवनेश्वरः
आधयो व्याधयस्तस्य सैनिका यवनाश्चराः
भूतोपसर्गाशुरयः प्रज्वारो द्विविधो ज्वरः
एवं बहुविधैर्दुःखैर्दैवभूतात्मसम्भवैः
क्लिश्यमानः शतं वर्षं देहे देही तमोवृतः
प्राणेन्द्रियमनोधर्मानात्मन्यध्यस्य निर्गुणः
शेते कामलवान्ध्यायन्ममाहमिति कर्मकृत्‌
यदात्मानमविज्ञाय भगवन्तं परं गुरुम्‌
पुरुषस्तु विषज्जेत गुणेषु प्रकृतेः स्वदृक्‌
गुणाभिमानी स तदा कर्माणि कुरुतेऽवशः
शुक्लं कृष्णं लोहितं वा यथाकर्माभिजायते
शुक्लात्प्रकाशभूयिष्ठा लोकानाप्नोति कर्हिचित्‌
दुःखोदर्कान्क्रियायासांस्तमःशोकोत्कटान्क्वचित्‌
क्वचित्पुमान्क्वचिच्च स्त्री क्वचिन्नोभयमन्धधीः
देवो मनुष्यस्तिर्यग्वा यथाकर्मगुणं भवः
क्षुत्परीतो यथा दीनः सारमेयो गृहं गृहम्‌
चरन्विन्दति यद्दिष्टं दण्डमोदनमेव वा
तथा कामाशयो जीव उच्चावचपथा भ्रमन्‌
उपर्यधो वा मध्ये वा याति दिष्टं प्रियाप्रियम्‌
दुःखेष्वेकतरेणापि दैवभूतात्महेतुषु
जीवस्य न व्यवच्छेदः स्याच्चेत्तत्तत्प्रतिक्रिया
यथा हि पुरुषो भारं शिरसा गुरुमुद्वहन्‌
तं स्कन्धेन स आधत्ते तथा सर्वाः प्रतिक्रियाः
नैकान्ततः प्रतीकारः कर्मणां कर्म केवलम्‌
द्वयं ह्यविद्योपसृतं स्वप्ने स्वप्न इवानघ
अर्थे ह्यविद्यमानेऽपि संसृतिर्न निवर्तते
मनसा लिङ्गरूपेण स्वप्ने विचरतो यथा
अथात्मनोऽर्थभूतस्य यतोऽनर्थपरम्परा
संसृतिस्तद्व्यवच्छेदो भक्त्या परमया गुरौ
वासुदेवे भगवति भक्तियोगः समाहितः
सध्रीचीनेन वैराग्यं ज्ञानं च जनयिष्यति
सोऽचिरादेव राजर्षे स्यादच्युतकथाश्रयः
शृण्वतः श्रद्दधानस्य नित्यदा स्यादधीयतः
यत्र भागवता राजन्साधवो विशदाशयाः
भगवद्गुणानुकथन श्रवणव्यग्रचेतसः
तस्मिन्महन्मुखरिता मधुभिच्‌ चरित्रपीयूषशेषसरितः परितः स्रवन्ति
ता ये पिबन्त्यवितृषो नृप गाढकर्णैस्तान्न स्पृशन्त्यशनतृड्भयशोकमोहाः
एतैरुपद्रुतो नित्यं जीवलोकः स्वभावजैः
न करोति हरेर्नूनं कथामृतनिधौ रतिम्‌
प्रजापतिपतिः साक्षाद्भगवान्गिरिशो मनुः
दक्षादयः प्रजाध्यक्षा नैष्ठिकाः सनकादयः
मरीचिरत्र्यङ्गिरसौ पुलस्त्यः पुलहः क्रतुः
भृगुर्वसिष्ठ इत्येते मदन्ता ब्रह्मवादिनः
अद्यापि वाचस्पतयस्तपोविद्यासमाधिभिः
पश्यन्तोऽपि न पश्यन्ति पश्यन्तं परमेश्वरम्‌
शब्दब्रह्मणि दुष्पारे चरन्त उरुविस्तरे
मन्त्रलिङ्गैर्व्यवच्छिन्नं भजन्तो न विदुः परम्‌
सर्वेषामेव जन्तूनां सततं देहपोषणे vedabase 4.29.1a)
अस्ति प्रज्ञा समायत्ता को विशेषस्तदा नृणाम्‌ vedabase 4.29.1a)
लब्ध्वेहान्ते मनुष्यत्वं हित्वा देहाद्यसद्ग्रहम्‌ vedabase 4.29.2a)
आत्मसृत्या विहायेदं जीवात्मा स विशिष्यते vedabase 4.29.2a)
यदा यस्यानुगृह्णाति भगवानात्मभावितः
स जहाति मतिं लोके वेदे च परिनिष्ठिताम्‌
तस्मात्कर्मसु बर्हिष्मन्नज्ञानादर्थकाशिषु
मार्थदृष्टिं कृथाः श्रोत्र स्पर्शिष्वस्पृष्टवस्तुषु
स्वं लोकं न विदुस्ते वै यत्र देवो जनार्दनः
आहुर्धूम्रधियो वेदं सकर्मकमतद्विदः
आस्तीर्य दर्भैः प्रागग्रैः कार्त्स्न्येन क्षितिमण्डलम्‌
स्तब्धो बृहद्वधान्मानी कर्म नावैषि यत्परम्‌
तत्कर्म हरितोषं यत्सा विद्या तन्मतिर्यया
हरिर्देहभृतामात्मा स्वयं प्रकृतिरीश्वरः
तत्पादमूलं शरणं यतः क्षेमो नृणामिह
स वै प्रियतमश्चात्मा यतो न भयमण्वपि
इति वेद स वै विद्वान्यो विद्वान्स गुरुर्हरिः
नारद उवाच
प्रश्न एवं हि सञ्छिन्नो भवतः पुरुषर्षभ
अत्र मे वदतो गुह्यं निशामय सुनिश्चितम्‌
क्षुद्रं चरं सुमनसां शरणे मिथित्वा
रक्तं षडङ्घ्रिगणसामसु लुब्धकर्णम्‌
अग्रे वृकानसुतृपोऽविगणय्य यान्तं
पृष्ठे मृगं मृगय लुब्धकबाणभिन्नम्‌
अस्यार्थः सुमनःसमधर्मणां स्त्रीणां शरण आश्रमे पुष्पमधुगन्धवत्क्षुद्रतमं
काम्यकर्मविपाकजं कामसुखलवं जैह्व्यौपस्थ्यादि विचिन्वन्तं मिथुनीभूय तदभिनिवेशित
मनसं षडङ्घ्रिगणसामगीतवदतिमनोहरवनितादिजनालापेष्वतितरामतिप्रलोभितकर्णमग्रे
वृकयूथवदात्मन आयुर्हरतोऽहोरात्रान्तान्काललवविशेषानविगणय्य गृहेषु विहरन्तं पृष्ठत एव
परोक्षमनुप्रवृत्तो लुब्धकः कृतान्तोऽन्तः शरेण यमिह पराविध्यति तमिममात्मानमहो
राजन्भिन्नहृदयं द्रष्टुमर्हसीति
स त्वं विचक्ष्य मृगचेष्टितमात्मनोऽन्तश्‌
चित्तं नियच्छ हृदि कर्णधुनीं च चित्ते
जह्यङ्गनाश्रममसत्तमयूथगाथं
प्रीणीहि हंसशरणं विरम क्रमेण
राजोवाच
श्रुतमन्वीक्षितं ब्रह्मन्भगवान्यदभाषत
नैतज्जानन्त्युपाध्यायाः किं न ब्रूयुर्विदुर्यदि
संशयोऽत्र तु मे विप्र सञ्छिन्नस्तत्कृतो महान्‌
ऋषयोऽपि हि मुह्यन्ति यत्र नेन्द्रियवृत्तयः
कर्माण्यारभते येन पुमानिह विहाय तम्‌
अमुत्रान्येन देहेन जुष्टानि स यदश्नुते
इति वेदविदां वादः श्रूयते तत्र तत्र ह
कर्म यत्क्रियते प्रोक्तं परोक्षं न प्रकाशते
नारद उवाच
येनैवारभते कर्म तेनैवामुत्र तत्पुमान्‌
भुङ्क्ते ह्यव्यवधानेन लिङ्गेन मनसा स्वयम्‌
शयानमिममुत्सृज्य श्वसन्तं पुरुषो यथा
कर्मात्मन्याहितं भुङ्क्ते तादृशेनेतरेण वा
ममैते मनसा यद्यदसावहमिति ब्रुवन्‌
गृह्णीयात्तत्पुमान्राद्धं कर्म येन पुनर्भवः
यथानुमीयते चित्तमुभयैरिन्द्रियेहितैः
एवं प्राग्देहजं कर्म लक्ष्यते चित्तवृत्तिभिः
नानुभूतं क्व चानेन देहेनादृष्टमश्रुतम्‌
कदाचिदुपलभ्येत यद्रूपं यादृगात्मनि
तेनास्य तादृशं राज लिङ्गिनो देहसम्भवम्‌
श्रद्धत्स्वाननुभूतोऽर्थो न मनः स्प्रष्टुमर्हति
मन एव मनुष्यस्य पूर्वरूपाणि शंसति
भविष्यतश्च भद्रं ते तथैव न भविष्यतः
अदृष्टमश्रुतं चात्र क्वचिन्मनसि दृश्यते
यथा तथानुमन्तव्यं देशकालक्रियाश्रयम्‌
सर्वे क्रमानुरोधेन मनसीन्द्रियगोचराः
आयान्ति बहुशो यान्ति सर्वे समनसो जनाः
सत्त्वैकनिष्ठे मनसि भगवत्पार्श्ववर्तिनि
तमश्चन्द्रमसीवेदमुपरज्यावभासते
नाहं ममेति भावोऽयं पुरुषे व्यवधीयते
यावद्बुद्धिमनोऽक्षार्थ गुणव्यूहो ह्यनादिमान्‌
सुप्तिमूर्च्छोपतापेषु प्राणायनविघाततः
नेहतेऽहमिति ज्ञानं मृत्युप्रज्वारयोरपि
गर्भे बाल्येऽप्यपौष्कल्यादेकादशविधं तदा
लिङ्गं न दृश्यते यूनः कुह्वां चन्द्रमसो यथा
अर्थे ह्यविद्यमानेऽपि संसृतिर्न निवर्तते
ध्यायतो विषयानस्य स्वप्नेऽनर्थागमो यथा
एवं पञ्चविधं लिङ्गं त्रिवृत्षोडश विस्तृतम्‌
एष चेतनया युक्तो जीव इत्यभिधीयते
अनेन पुरुषो देहानुपादत्ते विमुञ्चति
हर्षं शोकं भयं दुःखं सुखं चानेन विन्दति
भक्तिः कृष्णे दया जीवेष्वकुण्ठज्ञानमात्मनि vedabase 4.29.1b)
यदि स्यादात्मनो भूयादपवर्गस्तु संसृतेः vedabase 4.29.1b)
यथा तृणजलूकेयं नापयात्यपयाति च
न त्यजेन्म्रियमाणोऽपि प्राग्देहाभिमतिं जनः
अदृष्टं दृष्टवन्नङ्क्षेद्भूतं स्वप्नवदन्यथा vedabase 4.29.2b)
भूतं भवद्भविष्यच्च सुप्तं सर्वरहोरहः vedabase 4.29.2b)
यावदन्यं न विन्देत व्यवधानेन कर्मणाम्‌
मन एव मनुष्येन्द्र भूतानां भवभावनम्‌
यदाक्षैश्चरितान्ध्यायन्कर्माण्याचिनुतेऽसकृत्‌
सति कर्मण्यविद्यायां बन्धः कर्मण्यनात्मनः
अतस्तदपवादार्थं भज सर्वात्मना हरिम्‌
पश्यंस्तदात्मकं विश्वं स्थित्युत्पत्त्यप्यया यतः
मैत्रेय उवाच
भागवतमुख्यो भगवान्नारदो हंसयोर्गतिम्‌
प्रदर्श्य ह्यमुमामन्त्र्य सिद्धलोकं ततोऽगमत्‌
प्राचीनबर्ही राजर्षिः प्रजासर्गाभिरक्षणे
आदिश्य पुत्रानगमत्तपसे कपिलाश्रमम्‌
तत्रैकाग्रमना धीरो गोविन्दचरणाम्बुजम्‌
विमुक्तसङ्गोऽनुभजन्भक्त्या तत्साम्यतामगात्‌
एतदध्यात्मपारोक्ष्यं गीतं देवर्षिणानघ
यः श्रावयेद्यः शृणुयात्स लिङ्गेन विमुच्यते
एतन्मुकुन्दयशसा भुवनं पुनानं
देवर्षिवर्यमुखनिःसृतमात्मशौचम्‌
यः कीर्त्यमानमधिगच्छति पारमेष्ठ्यं
नास्मिन्भवे भ्रमति मुक्तसमस्तबन्धः
अध्यात्मपारोक्ष्यमिदं मयाधिगतमद्भुतम्‌
एवं स्त्रियाश्रमः पुंसश्छिन्नोऽमुत्र च संशयः
विदुर उवाच
ये त्वयाभिहिता ब्रह्मन्सुताः प्राचीनबर्हिषः
ते रुद्रगीतेन हरिं सिद्धिमापुः प्रतोष्य काम्‌
किं बार्हस्पत्येह परत्र वाथ कैवल्यनाथप्रियपार्श्ववर्तिनः
आसाद्य देवं गिरिशं यदृच्छया प्रापुः परं नूनमथ प्रचेतसः
मैत्रेय उवाच
प्रचेतसोऽन्तरुदधौ पितुरादेशकारिणः
अपयज्ञेन तपसा पुरञ्जनमतोषयन्‌
दशवर्षसहस्रान्ते पुरुषस्तु सनातनः
तेषामाविरभूत्कृच्छ्रं शान्तेन शमयन्रुचा
सुपर्णस्कन्धमारूढो मेरुशृङ्गमिवाम्बुदः
पीतवासा मणिग्रीवः कुर्वन्वितिमिरा दिशः
काशिष्णुना कनकवर्णविभूषणेन
भ्राजत्कपोलवदनो विलसत्किरीटः
अष्टायुधैरनुचरैर्मुनिभिः सुरेन्द्रैर्‌
आसेवितो गरुडकिन्नरगीतकीर्तिः
पीनायताष्टभुजमण्डलमध्यलक्ष्म्या
स्पर्धच्छ्रिया परिवृतो वनमालयाद्यः
बर्हिष्मतः पुरुष आह सुतान्प्रपन्नान्‌
पर्जन्यनादरुतया सघृणावलोकः
श्रीभगवानुवाच
वरं वृणीध्वं भद्रं वो यूयं मे नृपनन्दनाः
सौहार्देनापृथग्धर्मास्तुष्टोऽहं सौहृदेन वः
योऽनुस्मरति सन्ध्यायां युष्माननुदिनं नरः
तस्य भ्रातृष्वात्मसाम्यं तथा भूतेषु सौहृदम्‌
ये तु मां रुद्रगीतेन सायं प्रातः समाहिताः
स्तुवन्त्यहं कामवरान्दास्ये प्रज्ञां च शोभनाम्‌
यद्यूयं पितुरादेशमग्रहीष्ट मुदान्विताः
अथो व उशती कीर्तिर्लोकाननु भविष्यति
भविता विश्रुतः पुत्रोऽनवमो ब्रह्मणो गुणैः
य एतामात्मवीर्येण त्रिलोकीं पूरयिष्यति
कण्डोः प्रम्लोचया लब्धा कन्या कमललोचना
तां चापविद्धां जगृहुर्भूरुहा नृपनन्दनाः
क्षुत्क्षामाया मुखे राजा सोमः पीयूषवर्षिणीम्‌
देशिनीं रोदमानाया निदधे स दयान्वितः
प्रजाविसर्ग आदिष्टाः पित्रा मामनुवर्तता
तत्र कन्यां वरारोहां तामुद्वहत मा चिरम्‌
अपृथग्धर्मशीलानां सर्वेषां वः सुमध्यमा
अपृथग्धर्मशीलेयं भूयात्पत्न्यर्पिताशया
दिव्यवर्षसहस्राणां सहस्रमहतौजसः
भौमान्भोक्ष्यथ भोगान्वै दिव्यांश्चानुग्रहान्मम
अथ मय्यनपायिन्या भक्त्या पक्वगुणाशयाः
उपयास्यथ मद्धाम निर्विद्य निरयादतः
गृहेष्वाविशतां चापि पुंसां कुशलकर्मणाम्‌
मद्वार्तायातयामानां न बन्धाय गृहा मताः
नव्यवद्धृदये यज्ज्ञो ब्रह्मैतद्ब्रह्मवादिभिः
न मुह्यन्ति न शोचन्ति न हृष्यन्ति यतो गताः
मैत्रेय उवाच
एवं ब्रुवाणं पुरुषार्थभाजनं जनार्दनं प्राञ्जलयः प्रचेतसः
तद्दर्शनध्वस्ततमोरजोमला गिरागृणन्गद्गदया सुहृत्तमम्‌
प्रचेतस ऊचुः
नमो नमः क्लेशविनाशनाय निरूपितोदारगुणाह्वयाय
मनोवचोवेगपुरोजवाय सर्वाक्षमार्गैरगताध्वने नमः
शुद्धाय शान्ताय नमः स्वनिष्ठया मनस्यपार्थं विलसद्द्वयाय
नमो जगत्स्थानलयोदयेषु गृहीतमायागुणविग्रहाय
नमो विशुद्धसत्त्वाय हरये हरिमेधसे
वासुदेवाय कृष्णाय प्रभवे सर्वसात्वताम्‌
नमः कमलनाभाय नमः कमलमालिने
नमः कमलपादाय नमस्ते कमलेक्षण
नमः कमलकिञ्जल्क पिशङ्गामलवाससे
सर्वभूतनिवासाय नमोऽयुङ्क्ष्महि साक्षिणे
रूपं भगवता त्वेतदशेषक्लेशसङ्क्षयम्‌
आविष्कृतं नः क्लिष्टानां किमन्यदनुकम्पितम्‌
एतावत्त्वं हि विभुभिर्भाव्यं दीनेषु वत्सलैः
यदनुस्मर्यते काले स्वबुद्ध्याभद्ररन्धन
येनोपशान्तिर्भूतानां क्षुल्लकानामपीहताम्‌
अन्तर्हितोऽन्तर्हृदये कस्मान्नो वेद नाशिषः
असावेव वरोऽस्माकमीप्सितो जगतः पते
प्रसन्नो भगवान्येषामपवर्गः गुरुर्गतिः
वरं वृणीमहेऽथापि नाथ त्वत्परतः परात्‌
न ह्यन्तस्त्वद्विभूतीनां सोऽनन्त इति गीयसे
पारिजातेऽञ्जसा लब्धे सारङ्गोऽन्यन्न सेवते
त्वदङ्घ्रिमूलमासाद्य साक्षात्किं किं वृणीमहि
यावत्ते मायया स्पृष्टा भ्रमाम इह कर्मभिः
तावद्भवत्प्रसङ्गानां सङ्गः स्यान्नो भवे भवे
तुलयाम लवेनापि न स्वर्गं नापुनर्भवम्‌
भगवत्सङ्गिसङ्गस्य मर्त्यानां किमुताशिषः
यत्रेड्यन्ते कथा मृष्टास्तृष्णायाः प्रशमो यतः
निर्वैरं यत्र भूतेषु नोद्वेगो यत्र कश्चन
यत्र नारायणः साक्षाद्भगवान्न्यासिनां गतिः
संस्तूयते सत्कथासु मुक्तसङ्गैः पुनः पुनः
तेषां विचरतां पद्भ्यां तीर्थानां पावनेच्छया
भीतस्य किं न रोचेत तावकानां समागमः
वयं तु साक्षाद्भगवन्भवस्य प्रियस्य सख्युः क्षणसङ्गमेन
सुदुश्चिकित्स्यस्य भवस्य मृत्योर्भिषक्तमं त्वाद्य गतिं गताः स्म
यन्नः स्वधीतं गुरवः प्रसादिता विप्राश्च वृद्धाश्च सदानुवृत्त्या
आर्या नताः सुहृदो भ्रातरश्च सर्वाणि भूतान्यनसूययैव
यन्नः सुतप्तं तप एतदीश निरन्धसां कालमदभ्रमप्सु
सर्वं तदेतत्पुरुषस्य भूम्नो वृणीमहे ते परितोषणाय
मनुः स्वयम्भूर्भगवान्भवश्च येऽन्ये तपोज्ञानविशुद्धसत्त्वाः
अदृष्टपारा अपि यन्महिम्नः स्तुवन्त्यथो त्वात्मसमं गृणीमः
नमः समाय शुद्धाय पुरुषाय पराय च
वासुदेवाय सत्त्वाय तुभ्यं भगवते नमः
मैत्रेय उवाच
इति प्रचेतोभिरभिष्टुतो हरिः प्रीतस्तथेत्याह शरण्यवत्सलः
अनिच्छतां यानमतृप्तचक्षुषां ययौ स्वधामानपवर्गवीर्यः
अथ निर्याय सलिलात्प्रचेतस उदन्वतः
वीक्ष्याकुप्यन्द्रुमैश्छन्नां गां गां रोद्धुमिवोच्छ्रितैः
ततोऽग्निमारुतौ राजन्नमुञ्चन्मुखतो रुषा
महीं निर्वीरुधं कर्तुं संवर्तक इवात्यये
भस्मसात्क्रियमाणांस्तान्द्रुमान्वीक्ष्य पितामहः
आगतः शमयामास पुत्रान्बर्हिष्मतो नयैः
तत्रावशिष्टा ये वृक्षा भीता दुहितरं तदा
उज्जह्रुस्ते प्रचेतोभ्य उपदिष्टाः स्वयम्भुवा
ते च ब्रह्मण आदेशान्मारिषामुपयेमिरे
यस्यां महदवज्ञानादजन्यजनयोनिजः
चाक्षुषे त्वन्तरे प्राप्ते प्राक्सर्गे कालविद्रुते
यः ससर्ज प्रजा इष्टाः स दक्षो दैवचोदितः
यो जायमानः सर्वेषां तेजस्तेजस्विनां रुचा
स्वयोपादत्त दाक्ष्याच्च कर्मणां दक्षमब्रुवन्‌
तं प्रजासर्गरक्षायामनादिरभिषिच्य च
युयोज युयुजेऽन्यांश्च स वै सर्वप्रजापतीन्‌
मैत्रेय उवाच
तत उत्पन्नविज्ञाना आश्वधोक्षजभाषितम्‌
स्मरन्त आत्मजे भार्यां विसृज्य प्राव्रजन्गृहात्‌
दीक्षिता ब्रह्मसत्रेण सर्वभूतात्ममेधसा
प्रतीच्यां दिशि वेलायां सिद्धोऽभूद्यत्र जाजलिः
तान्निर्जितप्राणमनोवचोदृशो जितासनान्शान्तसमानविग्रहान्‌
परेऽमले ब्रह्मणि योजितात्मनः सुरासुरेड्यो ददृशे स्म नारदः
तमागतं त उत्थाय प्रणिपत्याभिनन्द्य च
पूजयित्वा यथादेशं सुखासीनमथाब्रुवन्‌
प्रचेतस ऊचुः
स्वागतं ते सुरर्षेऽद्य दिष्ट्या नो दर्शनं गतः
तव चङ्क्रमणं ब्रह्मन्नभयाय यथा रवेः
यदादिष्टं भगवता शिवेनाधोक्षजेन च
तद्गृहेषु प्रसक्तानां प्रायशः क्षपितं प्रभो
तन्नः प्रद्योतयाध्यात्म ज्ञानं तत्त्वार्थदर्शनम्‌
येनाञ्जसा तरिष्यामो दुस्तरं भवसागरम्‌
मैत्रेय उवाच
इति प्रचेतसां पृष्टो भगवान्नारदो मुनिः
भगवत्युत्तमश्लोक आविष्टात्माब्रवीन्नृपान्‌
नारद उवाच
तज्जन्म तानि कर्माणि तदायुस्तन्मनो वचः
नृणां येन हि विश्वात्मा सेव्यते हरिरीश्वरः
किं जन्मभिस्त्रिभिर्वेह शौक्रसावित्रयाज्ञिकैः
कर्मभिर्वा त्रयीप्रोक्तैः पुंसोऽपि विबुधायुषा
श्रुतेन तपसा वा किं वचोभिश्चित्तवृत्तिभिः
बुद्ध्या वा किं निपुणया बलेनेन्द्रियराधसा
किं वा योगेन साङ्ख्येन न्यासस्वाध्याययोरपि
किं वा श्रेयोभिरन्यैश्च न यत्रात्मप्रदो हरिः
श्रेयसामपि सर्वेषामात्मा ह्यवधिरर्थतः
सर्वेषामपि भूतानां हरिरात्मात्मदः प्रियः
यथा तरोर्मूलनिषेचनेन तृप्यन्ति तत्स्कन्धभुजोपशाखाः
प्राणोपहाराच्च यथेन्द्रियाणां तथैव सर्वार्हणमच्युतेज्या
यथैव सूर्यात्प्रभवन्ति वारः पुनश्च तस्मिन्प्रविशन्ति काले
भूतानि भूमौ स्थिरजङ्गमानि तथा हरावेव गुणप्रवाहः
एतत्पदं तज्जगदात्मनः परं सकृद्विभातं सवितुर्यथा प्रभा
यथासवो जाग्रति सुप्तशक्तयो द्रव्यक्रियाज्ञानभिदाभ्रमात्ययः
यथा नभस्यभ्रतमःप्रकाशा भवन्ति भूपा न भवन्त्यनुक्रमात्‌
एवं परे ब्रह्मणि शक्तयस्त्वमू रजस्तमः सत्त्वमिति प्रवाहः
तेनैकमात्मानमशेषदेहिनां कालं प्रधानं पुरुषं परेशम्‌
स्वतेजसा ध्वस्तगुणप्रवाहमात्मैकभावेन भजध्वमद्धा
दयया सर्वभूतेषु सन्तुष्ट्या येन केन वा
सर्वेन्द्रियोपशान्त्या च तुष्यत्याशु जनार्दनः
अपहतसकलैषणामलात्मन्यविरतमेधितभावनोपहूतः
निजजनवशगत्वमात्मनोऽयन्न सरति छिद्रवदक्षरः सतां हि
न भजति कुमनीषिणां स इज्यां हरिरधनात्मधनप्रियो रसज्ञः
श्रुतधनकुलकर्मणां मदैर्ये विदधति पापमकिञ्चनेषु सत्सु
श्रियमनुचरतीं तदर्थिनश्च द्विपदपतीन्विबुधांश्च यत्स्वपूर्णः
न भजति निजभृत्यवर्गतन्त्रः कथममुमुद्विसृजेत्पुमान्कृतज्ञः
मैत्रेय उवाच
इति प्रचेतसो राजन्नन्याश्च भगवत्कथाः
श्रावयित्वा ब्रह्मलोकं ययौ स्वायम्भुवो मुनिः
तेऽपि तन्मुखनिर्यातं यशो लोकमलापहम्‌
हरेर्निशम्य तत्पादं ध्यायन्तस्तद्गतिं ययुः
एतत्तेऽभिहितं क्षत्तर्यन्मां त्वं परिपृष्टवान्‌
प्रचेतसां नारदस्य संवादं हरिकीर्तनम्‌
श्रीशुक उवाच
य एष उत्तानपदो मानवस्यानुवर्णितः
वंशः प्रियव्रतस्यापि निबोध नृपसत्तम
यो नारदादात्मविद्यामधिगम्य पुनर्महीम्‌
भुक्त्वा विभज्य पुत्रेभ्य ऐश्वरं समगात्पदम्‌
इमां तु कौषारविणोपवर्णितां क्षत्ता निशम्याजितवादसत्कथाम्‌
प्रवृद्धभावोऽश्रुकलाकुलो मुनेर्दधार मूर्ध्ना चरणं हृदा हरेः
विदुर उवाच
सोऽयमद्य महायोगिन्भवता करुणात्मना
दर्शितस्तमसः पारो यत्राकिञ्चनगो हरिः
श्रीशुक उवाच
इत्यानम्य तमामन्त्र्य विदुरो गजसाह्वयम्‌
स्वानां दिदृक्षुः प्रययौ ज्ञातीनां निर्वृताशयः
एतद्यः शृणुयाद्राजन्राज्ञां हर्यर्पितात्मनाम्‌
आयुर्धनं यशः स्वस्ति गतिमैश्वर्यमाप्नुयात्‌
राजोवाच
प्रियव्रतो भागवताअत्मारामः कथं मुने
गृहेऽरमत यन्मूलः कर्मबन्धः पराभवः
न नूनं मुक्तसङ्गानां तादृशानां द्विजर्षभ
गृहेष्वभिनिवेशोऽयं पुंसां भवितुमर्हति
महतां खलु विप्रर्षे उत्तमश्लोकपादयोः
छायानिर्वृतचित्तानां न कुटुम्बे स्पृहामतिः
संशयोऽयं महान्ब्रह्मन्दारागारसुतादिषु
सक्तस्य यत्सिद्धिरभूत्कृष्णे च मतिरच्युता
श्रीशुक उवाच
बाढमुक्तं भगवत उत्तमश्लोकस्य श्रीमच्चरणारविन्दमकरन्दरस आवेशितचेतसो
भागवतपरमहंसदयितकथां किञ्चिदन्तरायविहतां स्वां शिवतमां पदवीं न प्रायेण हिन्वन्ति
यर्हि वाव ह राजन्स राजपुत्रः प्रियव्रतः परमभागवतो नारदस्य
चरणोपसेवयाञ्जसावगतपरमार्थसतत्त्वो ब्रह्मसत्रेण दीक्षिष्यमाणोऽवनितलपरिपालनायाम्नात
प्रवरगुणगणैकान्तभाजनतया स्वपित्रोपामन्त्रितो भगवति वासुदेव एवाव्यवधानसमाधियोगेन
समावेशितसकलकारकक्रियाकलापो नैवाभ्यनन्दद्यद्यपि तदप्रत्याम्नातव्यं तदधिकरण
आत्मनोऽन्यस्मादसतोऽपि पराभवमन्वीक्षमाणः
अथ ह भगवानादिदेव एतस्य गुणविसर्गस्य परिबृंहणानुध्यानव्यवसितसकलजगद्‌
अभिप्राय आत्मयोनिरखिलनिगमनिजगणपरिवेष्टितः स्वभवनादवततार
स तत्र तत्र गगनतल उडुपतिरिव विमानावलिभिरनुपथममर
परिवृढैरभिपूज्यमानः पथि पथि च वरूथशः सिद्धगन्धर्वसाध्यचारणमुनि
गणैरुपगीयमानो गन्धमादनद्रोणीमवभासयन्नुपससर्प
तत्र ह वा एनं देवर्षिर्हंसयानेन पितरं भगवन्तं हिरण्यगर्भमुपलभमानः
सहसैवोत्थायार्हणेन सह पितापुत्राभ्यामवहिताञ्जलिरुपतस्थे
भगवानपि भारत तदुपनीतार्हणः सूक्तवाकेनातितरामुदितगुणगणावतारसुजयः
प्रियव्रतमादिपुरुषस्तं सदयहासावलोक इति होवाच
श्रीभगवानुवाच
निबोध तातेदमृतं ब्रवीमि मासूयितुं देवमर्हस्यप्रमेयम्‌
वयं भवस्ते तत एष महर्षिर्वहाम सर्वे विवशा यस्य दिष्टम्‌
न तस्य कश्चित्तपसा विद्यया वा न योगवीर्येण मनीषया वा
नैवार्थधर्मैः परतः स्वतो वा कृतं विहन्तुं तनुभृद्विभूयात्‌
भवाय नाशाय च कर्म कर्तुं शोकाय मोहाय सदा भयाय
सुखाय दुःखाय च देहयोगमव्यक्तदिष्टं जनताङ्ग धत्ते
यद्वाचि तन्त्यां गुणकर्मदामभिः सुदुस्तरैर्वत्स वयं सुयोजिताः
सर्वे वहामो बलिमीश्वराय प्रोता नसीव द्विपदे चतुष्पदः
ईशाभिसृष्टं ह्यवरुन्ध्महेऽङ्ग दुःखं सुखं वा गुणकर्मसङ्गात्‌
आस्थाय तत्तद्यदयुङ्क्त नाथश्चक्षुष्मतान्धा इव नीयमानाः
मुक्तोऽपि तावद्बिभृयात्स्वदेहमारब्धमश्नन्नभिमानशून्यः
यथानुभूतं प्रतियातनिद्रः किं त्वन्यदेहाय गुणान्न वृङ्क्ते
भयं प्रमत्तस्य वनेष्वपि स्याद्यतः स आस्ते सहषट्सपत्नः
जितेन्द्रियस्यात्मरतेर्बुधस्य गृहाश्रमः किं नु करोत्यवद्यम्‌
यः षट्सपत्नान्विजिगीषमाणो गृहेषु निर्विश्य यतेत पूर्वम्‌
अत्येति दुर्गाश्रित ऊर्जितारीन्क्षीणेषु कामं विचरेद्विपश्चित्‌
त्वं त्वब्जनाभाङ्घ्रिसरोजकोश दुर्गाश्रितो निर्जितषट्सपत्नः
भुङ्क्ष्वेह भोगान्पुरुषातिदिष्टान्विमुक्तसङ्गः प्रकृतिं भजस्व
श्रीशुक उवाच
इति समभिहितो महाभागवतो भगवतस्त्रिभुवनगुरोरनुशासनमात्मनो लघुतयावनत
शिरोधरो बाढमिति सबहुमानमुवाह
भगवानपि मनुना यथावदुपकल्पितापचितिः प्रियव्रत
नारदयोरविषममभिसमीक्षमाणयोरात्मसमवस्थानमवाङ्मनसं क्षयमव्यवहृतं
प्रवर्तयन्नगमत्‌
मनुरपि परेणैवं प्रतिसन्धितमनोरथः सुरर्षिवरानुमतेनात्मजमखिलधरामण्डल
स्थितिगुप्तय आस्थाप्य स्वयमतिविषमविषयविषजलाशयाशाया उपरराम
इति ह वाव स जगतीपतिरीश्वरेच्छयाधिनिवेशितकर्माधिकारोऽखिलजगद्बन्धध्वंसन
परानुभावस्य भगवत आदिपुरुषस्याङ्घ्रियुगलानवरतध्यानानुभावेन परिरन्धितकषायाशयो
ऽवदातोऽपि मानवर्धनो महतां महीतलमनुशशास
अथ च दुहितरं प्रजापतेर्विश्वकर्मण उपयेमे बर्हिष्मतीं नाम तस्यामु ह वाव
आत्मजानात्मसमानशीलगुणकर्मरूपवीर्योदारान्दश भावयाम्बभूव कन्यां च
यवीयसीमूर्जस्वतीं नाम
आग्नीध्रेध्मजिह्वयज्ञबाहुमहावीरहिरण्यरेतोघृतपृष्ठसवनमेधातिथिवीतिहोत्रकवय
इति सर्व एवाग्निनामानः
एतेषां कविर्महावीरः सवन इति त्रय आसन्नूर्ध्वरेतसस्त आत्मविद्यायामर्भ
भावादारभ्य कृतपरिचयाः पारमहंस्यमेवाश्रममभजन्‌
तस्मिन्नु ह वा उपशमशीलाः परमर्षयः सकलजीवनिकायावासस्य भगवतो वासुदेवस्य
भीतानां शरणभूतस्य श्रीमच्चरणारविन्दाविरतस्मरणाविगलितपरमभक्तियोगानुभावेन
परिभावितान्तर्हृदयाधिगते भगवति सर्वेषां भूतानामात्मभूते प्रत्यग्‌
आत्मन्येवात्मनस्तादात्म्यमविशेषेण समीयुः
अन्यस्यामपि जायायां त्रयः पुत्रा आसन्नुत्तमस्तामसो रैवत इति मन्वन्तराधिपतयः
एवमुपशमायनेषु स्वतनयेष्वथ जगतीपतिर्जगतीमर्बुदान्येकादश
परिवत्सराणामव्याहताखिलपुरुषकारसारसम्भृतदोर्दण्डयुगलापीडितमौर्वीगुणस्तनितविरमित
धर्मप्रतिपक्षो बर्हिष्मत्याश्चानुदिनमेधमानप्रमोदप्रसरणयौषिण्यव्रीडाप्रमुषित
हासावलोकरुचिरक्ष्वेल्यादिभिः पराभूयमानविवेक इवानवबुध्यमान इव महामना बुभुजे
यावदवभासयति सुरगिरिमनुपरिक्रामन्भगवानादित्यो वसुधातलमर्धेनैव
प्रतपत्यर्धेनावच्छादयति तदा हि भगवदुपासनोपचितातिपुरुष
प्रभावस्तदनभिनन्दन्समजवेन रथेन ज्योतिर्मयेन रजनीमपि दिनं करिष्यामीति सप्त
कृत्वस्तरणिमनुपर्यक्रामद्द्वितीय इव पतङ्गः
ये वा उ ह तद्रथचरणनेमिकृतपरिखातास्ते सप्त सिन्धव आसन्यत एव कृताः सप्त भुवो
द्वीपाः
जम्बूप्लक्षशाल्मलिकुशक्रौञ्चशाकपुष्करसंज्ञास्तेषां परिमाणं
पूर्वस्मात्पूर्वस्मादुत्तर उत्तरो यथासङ्ख्यं द्विगुणमानेन बहिः समन्तत उपकॢप्ताः
दुहितरं चोर्जस्वतीं नामोशनसे प्रायच्छद्यस्यामासीद्देवयानी नाम काव्यसुता
नैवंविधः पुरुषकार उरुक्रमस्य
पुंसां तदङ्घ्रिरजसा जितषड्गुणानाम्‌
चित्रं विदूरविगतः सकृदाददीत
यन्नामधेयमधुना स जहाति बन्धम्‌
स एवमपरिमितबलपराक्रम एकदा तु देवर्षिचरणानुशयनानुपतितगुणविसर्ग
संसर्गेणानिर्वृतमिवात्मानं मन्यमान आत्मनिर्वेद इदमाह
अहो असाध्वनुष्ठितं यदभिनिवेशितोऽहमिन्द्रियैरविद्यारचितविषमविषयान्धकूपे
तदलमलममुष्या वनिताया विनोदमृगं मां धिग्धिगिति गर्हयां चकार
परदेवताप्रसादाधिगतात्मप्रत्यवमर्शेनानुप्रवृत्तेभ्यः पुत्रेभ्य इमां यथादायं
विभज्य भुक्तभोगां च महिषीं मृतकमिव सह महाविभूतिमपहाय स्वयं निहितनिर्वेदो हृदि
गृहीतहरिविहारानुभावो भगवतो नारदस्य पदवीं पुनरेवानुससार
तस्य ह वा एते श्लोकाः
प्रियव्रतकृतं कर्म को नु कुर्याद्विनेश्वरम्‌
यो नेमिनिम्नैरकरोच्छायां घ्नन्सप्त वारिधीन्‌
भूसंस्थानं कृतं येन सरिद्गिरिवनादिभिः
सीमा च भूतनिर्वृत्यै द्वीपे द्वीपे विभागशः
भौमं दिव्यं मानुषं च महित्वं कर्मयोगजम्‌
यश्चक्रे निरयौपम्यं पुरुषानुजनप्रियः
श्रीशुक उवाच
एवं पितरि सम्प्रवृत्ते तदनुशासने वर्तमान आग्नीध्रो जम्बूद्वीपौकसः प्रजा
औरसवद्धर्मावेक्षमाणः पर्यगोपायत्‌
स च कदाचित्पितृलोककामः सुरवरवनिताक्रीडाचलद्रोण्यां भगवन्तं विश्वसृजां
पतिमाभृतपरिचर्योपकरण आत्म इकाग्र्येण तपस्व्याराधयां बभूव
तदुपलभ्य भगवानादिपुरुषः सदसि गायन्तीं पूर्वचित्तिं नामाप्सरसमभियापयामास
सा च तदाश्रमोपवनमतिरमणीयं विविधनिबिडविटपिविटपनिकरसंश्लिष्टपुरट
लतारूढस्थलविहङ्गममिथुनैः प्रोच्यमानश्रुतिभिः प्रतिबोध्यमानसलिलकुक्कुटकारण्डव
कलहंसादिभिर्विचित्रमुपकूजितामलजलाशयकमलाकरमुपबभ्राम
तस्याः सुललितगमनपदविन्यासगतिविलासायाश्चानुपदं खणखणायमानरुचिर
चरणाभरणस्वनमुपाकर्ण्य नरदेवकुमारः समाधियोगेनामीलितनयननलिनमुकुल
युगलमीषद्विकचय्य व्यचष्ट
तामेवाविदूरे मधुकरीमिव सुमनस उपजिघ्रन्तीं दिविजमनुजमनोनयनाह्लाद
दुघैर्गतिविहारव्रीडाविनयावलोकसुस्वराक्षरावयवैर्मनसि नृणां कुसुमायुधस्य विदधतीं
विवरं निजमुखविगलितामृतासवसहासभाषणामोदमदान्धमधुकरनिकरोपरोधेन द्रुतपद
विन्यासेन वल्गुस्पन्दनस्तनकलशकबरभाररशनां देवीं तदवलोकनेन विवृतावसरस्य
भगवतो मकरध्वजस्य वशमुपनीतो जडवदिति होवाच
का त्वं चिकीर्षसि च किं मुनिवर्य शैले
मायासि कापि भगवत्परदेवतायाः
विज्ये बिभर्षि धनुषी सुहृदात्मनोऽर्थे
किं वा मृगान्मृगयसे विपिने प्रमत्तान्‌
बाणाविमौ भगवतः शतपत्रपत्रौ
शान्तावपुङ्खरुचिरावतितिग्मदन्तौ
कस्मै युयुङ्क्षसि वने विचरन्न विद्मः
क्षेमाय नो जडधियां तव विक्रमोऽस्तु
शिष्या इमे भगवतः परितः पठन्ति
गायन्ति साम सरहस्यमजस्रमीशम्‌
युष्मच्छिखाविलुलिताः सुमनोऽभिवृष्टीः
सर्वे भजन्त्यृषिगणा इव वेदशाखाः
वाचं परं चरणपञ्जरतित्तिरीणां
ब्रह्मन्नरूपमुखरां शृणवाम तुभ्यम्‌
लब्धा कदम्बरुचिरङ्कविटङ्कबिम्बे
यस्यामलातपरिधिः क्व च वल्कलं ते
किं सम्भृतं रुचिरयोर्द्विज शृङ्गयोस्ते
मध्ये कृशो वहसि यत्र दृशिः श्रिता मे
पङ्कोऽरुणः सुरभिरात्मविषाण ईदृग्‌
येनाश्रमं सुभग मे सुरभीकरोषि
लोकं प्रदर्शय सुहृत्तम तावकं मे
यत्रत्य इत्थमुरसावयवावपूर्वौ
अस्मद्विधस्य मनौन्नयनौ बिभर्ति
बह्वद्भुतं सरसराससुधादि वक्त्रे
का वात्मवृत्तिरदनाद्धविरङ्ग वाति
विष्णोः कलास्यनिमिषोन्मकरौ च कर्णौ
उद्विग्नमीनयुगलं द्विजपङ्क्तिशोचिर्‌
आसन्नभृङ्गनिकरं सर इन्मुखं ते
योऽसौ त्वया करसरोजहतः पतङ्गो
दिक्षु भ्रमन्भ्रमत एजयतेऽक्षिणी मे
मुक्तं न ते स्मरसि वक्रजटावरूथं
कष्टोऽनिलो हरति लम्पट एष नीवीम्‌
रूपं तपोधन तपश्चरतां तपोघ्नं
ह्येतत्तु केन तपसा भवतोपलब्धम्‌
चर्तुं तपोऽर्हसि मया सह मित्र मह्यं
किं वा प्रसीदति स वै भवभावनो मे
न त्वां त्यजामि दयितं द्विजदेवदत्तं
यस्मिन्मनो दृगपि नो न वियाति लग्नम्‌
मां चारुशृङ्ग्यर्हसि नेतुमनुव्रतं ते
चित्तं यतः प्रतिसरन्तु शिवाः सचिव्यः
श्रीशुक उवाच
इति ललनानुनयातिविशारदो ग्राम्यवैदग्ध्यया परिभाषया तां विबुधवधूं विबुध
मतिरधिसभाजयामास
सा च ततस्तस्य वीरयूथपतेर्बुद्धिशीलरूपवयःश्रियौदार्येण पराक्षिप्तमनास्तेन
सहायुतायुतपरिवत्सरोपलक्षणं कालं जम्बूद्वीपपतिना भौमस्वर्गभोगान्बुभुजे
तस्यामु ह वा आत्मजान्स राजवर आग्नीध्रो नाभिकिम्पुरुषहरिवर्षेलावृतरम्यकहिरण्मय
कुरुभद्राश्वकेतुमालसंज्ञान्नव पुत्रानजनयत्‌
सा सूत्वाथ सुतान्नवानुवत्सरं गृह एवापहाय पूर्वचित्तिर्भूय एवाजं देवमुपतस्थे
आग्नीध्रसुतास्ते मातुरनुग्रहादौत्पत्तिकेनैव संहननबलोपेताः पित्रा विभक्ता आत्मतुल्य
नामानि यथाभागं जम्बूद्वीपवर्षाणि बुभुजुः
आग्नीध्रो राजातृप्तः कामानामप्सरसमेवानुदिनमधिमन्यमानस्तस्याः सलोकतां
श्रुतिभिरवारुन्ध यत्र पितरो मादयन्ते
सम्परेते पितरि नव भ्रातरो मेरुदुहित्र्मेरुदेवीं प्रतिरूपामुग्रदंष्ट्रीं लतां रम्यां
श्यामां नारीं भद्रां देववीतिमिति संज्ञा नवोदवहन्‌
श्रीशुक उवाच
नाभिरपत्यकामोऽप्रजया मेरुदेव्या भगवन्तं यज्ञपुरुषमवहितात्मायजत
तस्य ह वाव श्रद्धया विशुद्धभावेन यजतः प्रवर्ग्येषु प्रचरत्सु द्रव्यदेशकाल
मन्त्रर्त्विग्दक्षिणाविधानयोगोपपत्त्या दुरधिगमोऽपि भगवान्भागवतवात्सल्यतया सुप्रतीक
आत्मानमपराजितं निजजनाभिप्रेतार्थविधित्सया गृहीतहृदयो हृदयङ्गमं मनो
नयनानन्दनावयवाभिराममाविश्चकार
अथ ह तमाविष्कृतभुजयुगलद्वयं हिरण्मयं पुरुषविशेषं कपिशकौशेयाम्बर
धरमुरसि विलसच्छ्रीवत्सललामं दरवरवनरुहवनमालाच्छूर्यमृतमणि
गदादिभिरुपलक्षितं स्फुटकिरणप्रवरमुकुटकुण्डलकटककटिसूत्रहारकेयूरनूपुराद्यङ्ग
भूषणविभूषितमृत्विक्सदस्यगृहपतयोऽधना इवोत्तमधनमुपलभ्य सबहु
मानमर्हणेनावनतशीर्षाण उपतस्थुः
ऋत्विज ऊचुः
अर्हसि मुहुरर्हत्तमार्हणमस्माकमनुपथानां नमो नम इत्येतावत्सदुपशिक्षितं को
ऽर्हति पुमान्प्रकृतिगुणव्यतिकरमतिरनीश ईश्वरस्य परस्य प्रकृतिपुरुषयोरर्वाक्तनाभिर्नाम
रूपाकृतिभी रूपनिरूपणम्सकलजननिकायवृजिननिरसनशिवतमप्रवरगुणगणैकदेश
कथनादृते
परिजनानुरागविरचितशबलसंशब्दसलिलसितकिसलयतुलसिकादूर्वाङ्कुरैरपि सम्भृतया
सपर्यया किल परम परितुष्यसि
अथानयापि न भवत इज्ययोरुभारभरया समुचितमर्थमिहोपलभामहे
आत्मन एवानुसवनमञ्जसाव्यतिरेकेण बोभूयमानाशेषपुरुषार्थस्वरूपस्य किन्तु नाथाशिष
आशासानानामेतदभिसंराधनमात्रं भवितुमर्हति
तद्यथा बालिशानां स्वयमात्मनः श्रेयः परमविदुषां परमपरमपुरुष प्रकर्ष
करुणया स्वमहिमानं चापवर्गाख्यमुपकल्पयिष्यन्स्वयं नापचित एवेतरवदिहोपलक्षितः
अथायमेव वरो ह्यर्हत्तम यर्हि बर्हिषि राजर्षेर्वरदर्षभो भवान्निजपुरुषेक्षणविषय
आसीत्‌
असङ्गनिशितज्ञानानलविधूताशेषमलानां भवत्स्वभावानामात्मारामाणां
मुनीनामनवरतपरिगुणितगुणगण परममङ्गलायनगुणगणकथनोऽसि
अथ कथञ्चित्स्खलनक्षुत्पतनजृम्भणदुरवस्थानादिषु विवशानां नः स्मरणाय ज्वर
मरणदशायामपि सकलकश्मलनिरसनानि तव गुणकृतनामधेयानि वचनगोचराणि भवन्तु
किञ्चायं राजर्षिरपत्यकामः प्रजां भवादृशीमाशासान ईश्वरमाशिषां स्वर्गापवर्गयोरपि
भवन्तमुपधावति प्रजायामर्थप्रत्ययो धनदमिवाधनः फलीकरणम्‌
को वा इह तेऽपराजितोऽपराजितया माययानवसितपदव्यानावृतमतिर्विषयविषरयानावृत
प्रकृतिरनुपासितमहच्चरणः
यदु ह वाव तव पुनरदभ्रकर्तरिह समाहूतस्तत्रार्थधियां मन्दानां
नस्तद्यद्देवहेलनं देवदेवार्हसि साम्येन सर्वान्प्रतिवोढुमविदुषाम्‌
श्रीशुक उवाच
इति निगदेनाभिष्टूयमानो भगवाननिमिषर्षभो वर्षधराभिवादिताभिवन्दितचरणः
सदयमिदमाह
श्रीभगवानुवाच
अहो बताहमृषयो भवद्भिरवितथगीर्भिर्वरमसुलभमभियाचितो यदमुष्यात्मजो मया
सदृशो भूयादिति ममाहमेवाभिरूपः कैवल्यादथापि ब्रह्मवादो न मृषा भवितुमर्हति ममैव हि
मुखं यद्द्विजदेवकुलम्‌
तत आग्नीध्रीयेऽंशकलयावतरिष्याम्यात्मतुल्यमनुपलभमानः
श्रीशुक उवाच
इति निशामयन्त्या मेरुदेव्याः पतिमभिधायान्तर्दधे भगवान्‌
श्रीशुक उवाच
अथ ह तमुत्पत्त्यैवाभिव्यज्यमानभगवल्लक्षणं साम्योपशमवैराग्यैश्वर्यमहा
विभूतिभिरनुदिनमेधमानानुभावं प्रकृतयः प्रजा ब्राह्मणा देवताश्चावनितलसमवनायातितरां
जगृधुः
तस्य ह वा इत्थं वर्ष्मणा वरीयसा बृहच्छ्लोकेन चौजसा बलेन श्रिया यशसा वीर्यशौर्याभ्यां
च पिता ऋषभ इतीदं नाम चकार
यस्य हीन्द्रः स्पर्धमानो भगवान्वर्षे न ववर्ष तदवधार्य भगवानृषभदेवो
योगेश्वरः प्रहस्यात्मयोगमायया स्ववर्षमजनाभं नामाभ्यवर्षत्‌
नाभिस्तु यथाभिलषितं सुप्रजस्त्वमवरुध्यातिप्रमोदभरविह्वलो गद्गदाक्षरया गिरा
स्वैरं गृहीतनरलोकसधर्मं भगवन्तं पुराणपुरुषं मायाविलसितमतिर्वत्स तातेति
सानुरागमुपलालयन्परां निर्वृतिमुपगतः
विदितानुरागमापौरप्रकृति जनपदो राजा नाभिरात्मजं समयसेतुरक्षायामभिषिच्य
ब्राह्मणेषूपनिधाय सह मेरुदेव्या विशालायां प्रसन्ननिपुणेन तपसा समाधियोगेन नर
नारायणाख्यं भगवन्तं वासुदेवमुपासीनः कालेन तन्महिमानमवाप
यस्य ह पाण्डवेय श्लोकावुदाहरन्ति
को नु तत्कर्म राजर्षेर्नाभेरन्वाचरेत्पुमान्‌
अपत्यतामगाद्यस्य हरिः शुद्धेन कर्मणा
ब्रह्मण्योऽन्यः कुतो नाभेर्विप्रा मङ्गलपूजिताः
यस्य बर्हिषि यज्ञेशं दर्शयामासुरोजसा
अथ ह भगवानृषभदेवः स्ववर्षं कर्मक्षेत्रमनुमन्यमानः प्रदर्शितगुरुकुलवासो
लब्धवरैर्गुरुभिरनुज्ञातो गृहमेधिनां धर्माननुशिक्षमाणो जयन्त्यामिन्द्रदत्तायामुभय
लक्षणं कर्म समाम्नायाम्नातमभियुञ्जन्नात्मजानामात्मसमानानां शतं जनयामास
येषां खलु महायोगी भरतो ज्येष्ठः श्रेष्ठगुण आसीद्येनेदं
वर्षं भारतमिति व्यपदिशन्ति
तमनु कुशावर्त इलावर्तो ब्रह्मावर्तो मलयः केतुर्भद्रसेन
इन्द्रस्पृग्विदर्भः कीकट इति
नव नवति प्रधानाः
कविर्हविरन्तरिक्षः प्रबुद्धः पिप्पलायनः
आविर्होत्रोऽथ द्रुमिलश्चमसः करभाजनः
इति भागवतधर्मदर्शना नव महाभागवतास्तेषां सुचरितं
भगवन्महिमोपबृंहितं
वसुदेवनारदसंवादमुपशमायनमुपरिष्टाद्वर्णयिष्यामः
यवीयांस एकाशीतिर्जायन्तेयाः पितुरादेशकरा महाशालीना
महाश्रोत्रिया यज्ञशीलाः कर्मविशुद्धा ब्राह्मणा बभूवुः
भगवानृषभसंज्ञ आत्मतन्त्रः स्वयं नित्यनिवृत्तानर्थपरम्परः केवलानन्दानुभव
ईश्वर एव विपरीतवत्कर्माण्यारभमाणः कालेनानुगतं धर्ममाचरणेनोपशिक्षयन्नतद्विदां
सम उपशान्तो मैत्रः कारुणिको धर्मार्थयशःप्रजानन्दामृतावरोधेन गृहेषु लोकं नियमयत्‌
यद्यच्छीर्षण्याचरितं तत्तदनुवर्तते लोकः
यद्यपि स्वविदितं सकलधर्मं ब्राह्मं गुह्यं ब्राह्मणैर्दर्शितमार्गेण
सामादिभिरुपायैर्जनतामनुशशास
द्रव्यदेशकालवयःश्रद्धर्त्विग्विविधोद्देशोपचितैः सर्वैरपि क्रतुभिर्यथोपदेशं शत
कृत्व इयाज
भगवतर्षभेण परिरक्ष्यमाण एतस्मिन्वर्षे न कश्चन पुरुषो
वाञ्छत्यविद्यमानमिवात्मनोऽन्यस्मात्कथञ्चन किमपि कर्हिचिदवेक्षते भर्तर्यनुसवनं
विजृम्भितस्नेहातिशयमन्तरेण
स कदाचिदटमानो भगवानृषभो ब्रह्मावर्तगतो ब्रह्मर्षिप्रवरसभायां प्रजानां
निशामयन्तीनामात्मजानवहितात्मनः प्रश्रयप्रणयभरसुयन्त्रितानप्युपशिक्षयन्निति होवाच
ऋषभ उवाच
नायं देहो देहभाजां नृलोके कष्टान्कामानर्हते विड्भुजां ये
तपो दिव्यं पुत्रका येन सत्त्वं शुद्ध्येद्यस्माद्ब्रह्मसौख्यं त्वनन्तम्‌
महत्सेवां द्वारमाहुर्विमुक्तेस्तमोद्वारं योषितां सङ्गिसङ्गम्‌
महान्तस्ते समचित्ताः प्रशान्ता विमन्यवः सुहृदः साधवो ये
ये वा मयीशे कृतसौहृदार्था जनेषु देहम्भरवार्तिकेषु
गृहेषु जायात्मजरातिमत्सु न प्रीतियुक्ता यावदर्थाश्च लोके
नूनं प्रमत्तः कुरुते विकर्म यदिन्द्रियप्रीतय आपृणोति
न साधु मन्ये यत आत्मनोऽयमसन्नपि क्लेशद आस देहः
पराभवस्तावदबोधजातो यावन्न जिज्ञासत आत्मतत्त्वम्‌
यावत्क्रियास्तावदिदं मनो वै कर्मात्मकं येन शरीरबन्धः
एवं मनः कर्मवशं प्रयुङ्क्ते अविद्ययात्मन्युपधीयमाने
प्रीतिर्न यावन्मयि वासुदेवे न मुच्यते देहयोगेन तावत्‌
यदा न पश्यत्ययथा गुणेहां स्वार्थे प्रमत्तः सहसा विपश्चित्‌
गतस्मृतिर्विन्दति तत्र तापानासाद्य मैथुन्यमगारमज्ञः
पुंसः स्त्रिया मिथुनीभावमेतं तयोर्मिथो हृदयग्रन्थिमाहुः
अतो गृहक्षेत्रसुताप्तवित्तैर्जनस्य मोहोऽयमहं ममेति
यदा मनोहृदयग्रन्थिरस्य कर्मानुबद्धो दृढ आश्लथेत
तदा जनः सम्परिवर्ततेऽस्माद्मुक्तः परं यात्यतिहाय हेतुम्‌
हंसे गुरौ मयि भक्त्यानुवृत्या वितृष्णया द्वन्द्वतितिक्षया च
सर्वत्र जन्तोर्व्यसनावगत्या जिज्ञासया तपसेहानिवृत्त्या
मत्कर्मभिर्मत्कथया च नित्यं मद्देवसङ्गाद्गुणकीर्तनान्मे
निर्वैरसाम्योपशमेन पुत्रा जिहासया देहगेहात्मबुद्धेः
अध्यात्मयोगेन विविक्तसेवया प्राणेन्द्रियात्माभिजयेन सध्र्यक्‌
सच्छ्रद्धया ब्रह्मचर्येण शश्वदसम्प्रमादेन यमेन वाचाम्‌
सर्वत्र मद्भावविचक्षणेन ज्ञानेन विज्ञानविराजितेन
योगेन धृत्युद्यमसत्त्वयुक्तो लिङ्गं व्यपोहेत्कुशलोऽहमाख्यम्‌
कर्माशयं हृदयग्रन्थिबन्धमविद्ययासादितमप्रमत्तः
अनेन योगेन यथोपदेशं सम्यग्व्यपोह्योपरमेत योगात्‌
पुत्रांश्च शिष्यांश्च नृपो गुरुर्वा मल्लोककामो मदनुग्रहार्थः
इत्थं विमन्युरनुशिष्यादतज्ज्ञान्न योजयेत्कर्मसु कर्ममूढान्‌
कं योजयन्मनुजोऽर्थं लभेत निपातयन्नष्टदृशं हि गर्ते
लोकः स्वयं श्रेयसि नष्टदृष्टिर्योऽर्थान्समीहेत निकामकामः
अन्योन्यवैरः सुखलेशहेतोरनन्तदुःखं च न वेद मूढः
कस्तं स्वयं तदभिज्ञो विपश्चिदविद्यायामन्तरे वर्तमानम्‌
दृष्ट्वा पुनस्तं सघृणः कुबुद्धिं प्रयोजयेदुत्पथगं यथान्धम्‌
गुरुर्न स स्यात्स्वजनो न स स्यात्पिता न स स्याज्जननी न सा स्यात्‌
दैवं न तत्स्यान्न पतिश्च स स्यान्न मोचयेद्यः समुपेतमृत्युम्‌
इदं शरीरं मम दुर्विभाव्यं सत्त्वं हि मे हृदयं यत्र धर्मः
पृष्ठे कृतो मे यदधर्म आरादतो हि मामृषभं प्राहुरार्याः
तस्माद्भवन्तो हृदयेन जाताः सर्वे महीयांसममुं सनाभम्‌
अक्लिष्टबुद्ध्या भरतं भजध्वं शुश्रूषणं तद्भरणं प्रजानाम्‌
भूतेषु वीरुद्भ्य उदुत्तमा ये सरीसृपास्तेषु सबोधनिष्ठाः
ततो मनुष्याः प्रमथास्ततोऽपि गन्धर्वसिद्धा विबुधानुगा ये
देवासुरेभ्यो मघवत्प्रधाना दक्षादयो ब्रह्मसुतास्तु तेषाम्‌
भवः परः सोऽथ विरिञ्चवीर्यः स मत्परोऽहं द्विजदेवदेवः
न ब्राह्मणैस्तुलये भूतमन्यत्पश्यामि विप्राः किमतः परं तु
यस्मिन्नृभिः प्रहुतं श्रद्धयाहमश्नामि कामं न तथाग्निहोत्रे
धृता तनूरुशती मे पुराणी येनेह सत्त्वं परमं पवित्रम्‌
शमो दमः सत्यमनुग्रहश्च तपस्तितिक्षानुभवश्च यत्र
मत्तोऽप्यनन्तात्परतः परस्मात्स्वर्गापवर्गाधिपतेर्न किञ्चित्‌
येषां किमु स्यादितरेण तेषामकिञ्चनानां मयि भक्तिभाजाम्‌
सर्वाणि मद्धिष्ण्यतया भवद्भिश्चराणि भूतानि सुता ध्रुवाणि
सम्भावितव्यानि पदे पदे वो विविक्तदृग्भिस्तदु हार्हणं मे
मनोवचोदृक्करणेहितस्य साक्षात्कृतं मे परिबर्हणं हि
विना पुमान्येन महाविमोहात्कृतान्तपाशान्न विमोक्तुमीशेत्‌
श्रीशुक उवाच
एवमनुशास्यात्मजान्स्वयमनुशिष्टानपि लोकानुशासनार्थं महानुभावः परम
सुहृद्भगवानृषभापदेश उपशमशीलानामुपरतकर्मणां महामुनीनां भक्तिज्ञानवैराग्य
लक्षणं पारमहंस्यधर्ममुपशिक्षमाणः स्वतनयशतज्येष्ठं परमभागवतं भगवज्‌
जनपरायणं भरतं धरणिपालनायाभिषिच्य स्वयं भवन एवोर्वरितशरीरमात्रपरिग्रह उन्मत्त
इव गगनपरिधानः प्रकीर्णकेश आत्मन्यारोपिताहवनीयो ब्रह्मावर्तात्प्रवव्राज
जडान्धमूकबधिरपिशाचोन्मादकवदवधूतवेषोऽभिभाष्यमाणोऽपि जनानां गृहीतमौन
व्रतस्तूष्णीं बभूव
तत्र तत्र पुरग्रामाकरखेटवाटखर्वटशिबिरव्रजघोषसार्थगिरि
वनाश्रमादिष्वनुपथमवनिचरापसदैः परिभूयमानो मक्षिकाभिरिव वनगजस्तर्जन
ताडनावमेहनष्ठीवनग्रावशकृद्रजःप्रक्षेपपूतिवातदुरुक्तैस्तदविगणयन्नेवासत्संस्थान
एतस्मिन्देहोपलक्षणे सदपदेश उभयानुभवस्वरूपेण स्वमहिमावस्थानेनासमारोपिताहं
ममाभिमानत्वादविखण्डितमनाः पृथिवीमेकचरः परिबभ्राम
अतिसुकुमारकरचरणोरःस्थलविपुलबाह्वंसगलवदनाद्यवयवविन्यासः प्रकृति
सुन्दरस्वभावहाससुमुखो नवनलिनदलायमानशिशिरतारारुणायतनयनरुचिरः सदृशसुभग
कपोलकर्णकण्ठनासो विगूढस्मितवदनमहोत्सवेन पुरवनितानां मनसि कुसुम
शरासनमुपदधानः परागवलम्बमानकुटिलजटिलकपिशकेशभूरिभारोऽवधूतमलिननिज
शरीरेण ग्रहगृहीत इवादृश्यत
यर्हि वाव स भगवान्लोकमिमं योगस्याद्धा प्रतीपमिवाचक्षाणस्तत्प्रतिक्रियाकर्म
बीभत्सितमिति व्रतमाजगरमास्थितः शयान एवाश्नाति पिबति खादत्यवमेहति हदति स्म चेष्टमान
उच्चरित आदिग्धोद्देशः
तस्य ह यः पुरीषसुरभिसौगन्ध्यवायुस्तं देशं दशयोजनं समन्तात्सुरभिं चकार
एवं गोमृगकाकचर्यया व्रजंस्तिष्ठन्नासीनः शयानः काकमृगगोचरितः पिबति
खादत्यवमेहति स्म
इति नानायोगचर्याचरणो भगवान्कैवल्यपतिरृषभोऽविरतपरममहानन्दानुभव आत्मनि
सर्वेषां भूतानामात्मभूते भगवति वासुदेव आत्मनोऽव्यवधानानन्तरोदरभावेन सिद्ध
समस्तार्थपरिपूर्णो योगैश्वर्याणि वैहायसमनोजवान्तर्धानपरकायप्रवेशदूरग्रहणादीनि
यदृच्छयोपगतानि नाञ्जसा नृप हृदयेनाभ्यनन्दत्‌
राजोवाच
न नूनं भगव आत्मारामाणां योगसमीरितज्ञानावभर्जितकर्मबीजानामैश्वर्याणि पुनः
क्लेशदानि भवितुमर्हन्ति यदृच्छयोपगतानि
ऋषिरुवाच
सत्यमुक्तं किन्त्विह वा एके न मनसोऽद्धा विश्रम्भमनवस्थानस्य शठकिरात इव
सङ्गच्छन्ते
तथा चोक्तम्‌
न कुर्यात्कर्हिचित्सख्यं मनसि ह्यनवस्थिते
यद्विश्रम्भाच्चिराच्चीर्णं चस्कन्द तप ऐश्वरम्‌
नित्यं ददाति कामस्य च्छिद्रं तमनु येऽरयः
योगिनः कृतमैत्रस्य पत्युर्जायेव पुंश्चली
कामो मन्युर्मदो लोभः शोकमोहभयादयः
कर्मबन्धश्च यन्मूलः स्वीकुर्यात्को नु तद्बुधः
अथैवमखिललोकपालललामोऽपि विलक्षणैर्जडवदवधूतवेषभाषाचरितैरविलक्षित
भगवत्प्रभावो योगिनां साम्परायविधिमनुशिक्षयन्स्वकलेवरं
जिहासुरात्मन्यात्मानमसंव्यवहितमनर्थान्तरभावेनान्वीक्षमाण उपरतानुवृत्तिरुपरराम
तस्य ह वा एवं मुक्तलिङ्गस्य भगवत ऋषभस्य योगमायावासनया देह इमां
जगतीमभिमानाभासेन सङ्क्रममाणः कोङ्कवेङ्ककुटकान्दक्षिण
कर्णाटकान्देशान्यदृच्छयोपगतः कुटकाचलोपवन आस्य कृताश्मकवल उन्माद इव मुक्तमूर्धजो
ऽसंवीत एव विचचार
अथ समीरवेगविधूतवेणुविकर्षणजातोग्रदावानलस्तद्वनमालेलिहानः सह तेन ददाह
यस्य किलानुचरितमुपाकर्ण्य कोङ्कवेङ्ककुटकानां राजार्हन्नामोपशिक्ष्य कलावधर्म
उत्कृष्यमाणे भवितव्येन विमोहितः स्वधर्मपथमकुतोभयमपहाय कुपथ
पाखण्डमसमञ्जसं निजमनीषया मन्दः सम्प्रवर्तयिष्यते
येन ह वाव कलौ मनुजापसदा देवमायामोहिताः स्वविधिनियोगशौचचारित्रविहीना देव
हेलनान्यपव्रतानि निजनिजेच्छया गृह्णाना अस्नानानाचमनाशौचकेशोल्लुञ्चनादीनि कलिनाधर्म
बहुलेनोपहतधियो ब्रह्मब्राह्मणयज्ञपुरुषलोकविदूषकाः प्रायेण भविष्यन्ति
ते च ह्यर्वाक्तनया निजलोकयात्रयान्धपरम्परयाश्वस्तास्तमस्यन्धे स्वयमेव
प्रपतिष्यन्ति
अयमवतारो रजसोपप्लुतकैवल्योपशिक्षणार्थः
तस्यानुगुणान्श्लोकान्गायन्ति
अहो भुवः सप्तसमुद्रवत्या द्वीपेषु वर्षेष्वधिपुण्यमेतत्‌
गायन्ति यत्रत्यजना मुरारेः कर्माणि भद्राण्यवतारवन्ति
अहो नु वंशो यशसावदातः प्रैयव्रतो यत्र पुमान्पुराणः
कृतावतारः पुरुषः स आद्यश्चचार धर्मं यदकर्महेतुम्‌
को न्वस्य काष्ठामपरोऽनुगच्छेन्मनोरथेनाप्यभवस्य योगी
यो योगमायाः स्पृहयत्युदस्ता ह्यसत्तया येन कृतप्रयत्नाः
इति ह स्म सकलवेदलोकदेवब्राह्मणगवां परमगुरोर्भगवत ऋषभाख्यस्य
विशुद्धाचरितमीरितं पुंसां समस्तदुश्चरिताभिहरणं परममहा
मङ्गलायनमिदमनुश्रद्धयोपचितयानुशृणोत्याश्रावयति वावहितो भगवति तस्मिन्वासुदेव एकान्ततो
भक्तिरनयोरपि समनुवर्तते
यस्यामेव कवय आत्मानमविरतं विविधवृजिनसंसारपरितापोपतप्यमानमनुसवनं
स्नापयन्तस्तयैव परया निर्वृत्या ह्यपवर्गमात्यन्तिकं परमपुरुषार्थमपि स्वयमासादितं नो
एवाद्रियन्ते भगवदीयत्वेनैव परिसमाप्तसर्वार्थाः
राजन्पतिर्गुरुरलं भवतां यदूनां
दैवं प्रियः कुलपतिः क्व च किङ्करो वः
अस्त्वेवमङ्ग भगवान्भजतां मुकुन्दो
मुक्तिं ददाति कर्हिचित्स्म न भक्तियोगम्‌
नित्यानुभूतनिजलाभनिवृत्ततृष्णः
श्रेयस्यतद्रचनया चिरसुप्तबुद्धेः
लोकस्य यः करुणयाभयमात्मलोकम्‌
आख्यान्नमो भगवते ऋषभाय तस्मै
श्रीशुक उवाच
भरतस्तु महाभागवतो यदा भगवतावनितलपरिपालनाय सञ्चिन्तितस्तदनुशासनपरः
पञ्चजनीं विश्वरूपदुहितरमुपयेमे
तस्यामु ह वा आत्मजान्कार्त्स्न्येनानुरूपानात्मनः पञ्च जनयामास भूतादिरिव भूत
सूक्ष्माणि सुमतिं राष्ट्रभृतं सुदर्शनमावरणं धूम्रकेतुमिति
अजनाभं नामैतद्वर्षं भारतमिति यत आरभ्य व्यपदिशन्ति
स बहुविन्महीपतिः पितृपितामहवदुरुवत्सलतया स्वे स्वे कर्मणि वर्तमानाः प्रजाः स्व
धर्ममनुवर्तमानः पर्यपालयत्‌
ईजे च भगवन्तं यज्ञक्रतुरूपं क्रतुभिरुच्चावचैः श्रद्धयाहृताग्निहोत्रदर्शपूर्णमास
चातुर्मास्यपशुसोमानां प्रकृतिविकृतिभिरनुसवनं चातुर्होत्रविधिना
सम्प्रचरत्सु नानायागेषु विरचिताङ्गक्रियेष्वपूर्वं यत्तत्क्रियाफलं धर्माख्यं परे
ब्रह्मणि यज्ञपुरुषे सर्वदेवतालिङ्गानां मन्त्राणामर्थनियामकतया साक्षात्कर्तरि परदेवतायां
भगवति वासुदेव एव भावयमान आत्मनैपुण्यमृदितकषायो हविःष्वध्वर्युभिर्गृह्यमाणेषु स
यजमानो यज्ञभाजो देवांस्तान्पुरुषावयवेष्वभ्यध्यायत्‌
एवं कर्मविशुद्ध्या विशुद्धसत्त्वस्यान्तर्हृदयाकाशशरीरे ब्रह्मणि भगवति वासुदेवे महा
पुरुषरूपोपलक्षणे श्रीवत्सकौस्तुभवनमालारिदरगदादिभिरुपलक्षिते निजपुरुषहृल्लिखितेनात्मनि
पुरुषरूपेण विरोचमान उच्चैस्तरां भक्तिरनुदिनमेधमानरयाजायत
एवं वर्षायुतसहस्रपर्यन्तावसितकर्मनिर्वाणावसरोऽधिभुज्यमानं स्वतनयेभ्यो
रिक्थं पितृपैतामहं यथादायं विभज्य स्वयं सकलसम्पन्निकेतात्स्वनिकेतात्पुलहाश्रमं
प्रवव्राज
यत्र ह वाव भगवान्हरिरद्यापि तत्रत्यानां निजजनानां वात्सल्येन सन्निधाप्यत इच्छा
रूपेण
यत्राश्रमपदान्युभयतो नाभिभिर्दृषच्चक्रैश्चक्रनदी नाम सरित्प्रवरा सर्वतः पवित्री
करोति
तस्मिन्वाव किल स एकलः पुलहाश्रमोपवने विविधकुसुमकिसलयतुलसिकाम्बुभिः कन्दमूल
फलोपहारैश्च समीहमानो भगवत आराधनं विविक्त उपरतविषयाभिलाष उपभृतोपशमः परां
निर्वृतिमवाप
तयेत्थमविरतपुरुषपरिचर्यया भगवति प्रवर्धमानानुरागभरद्रुतहृदयशैथिल्यः
प्रहर्षवेगेनात्मन्युद्भिद्यमानरोमपुलककुलक औत्कण्ठ्यप्रवृत्तप्रणयबाष्पनिरुद्धावलोक
नयन एवं निजरमणारुणचरणारविन्दानुध्यानपरिचितभक्तियोगेन परिप्लुतपरमाह्लादगम्भीर
हृदयह्रदावगाढधिषणस्तामपि क्रियमाणां भगवत्सपर्यां न सस्मार
इत्थं धृतभगवद्व्रत ऐणेयाजिनवाससानुसवनाभिषेकार्द्रकपिशकुटिलजटाकलापेन च
विरोचमानः सूर्यर्चा भगवन्तं हिरण्मयं पुरुषमुज्जिहाने सूर्यमण्डलेऽभ्युपतिष्ठन्नेतदु
होवाच
परोरजः सवितुर्जातवेदो देवस्य भर्गो मनसेदं जजान
सुरेतसादः पुनराविश्य चष्टे हंसं गृध्राणं नृषद्रिङ्गिरामिमः
श्रीशुक उवाच
एकदा तु महानद्यां कृताभिषेकनैयमिकावश्यको ब्रह्माक्षरमभिगृणानो मुहूर्त
त्रयमुदकान्त उपविवेश
तत्र तदा राजन्हरिणी पिपासया जलाशयाभ्याशमेकैवोपजगाम
तया पेपीयमान उदके तावदेवाविदूरेण नदतो मृगपतेरुन्नादो लोकभयङ्कर उदपतत्‌
तमुपश्रुत्य सा मृगवधूः प्रकृतिविक्लवा चकितनिरीक्षणा सुतरामपि हरिभयाभिनिवेश
व्यग्रहृदया पारिप्लवदृष्टिरगततृषा भयात्सहसैवोच्चक्राम
तस्या उत्पतन्त्या अन्तर्वत्न्या उरुभयावगलितो योनिनिर्गतो गर्भः स्रोतसि निपपात
तत्प्रसवोत्सर्पणभयखेदातुरा स्वगणेन वियुज्यमाना कस्याञ्चिद्दर्यां कृष्णसारसती
निपपाताथ च ममार
तं त्वेणकुणकं कृपणं स्रोतसानूह्यमानमभिवीक्ष्यापविद्धं बन्धुरिवानुकम्पया
राजर्षिर्भरत आदाय मृतमातरमित्याश्रमपदमनयत्‌
तस्य ह वा एणकुणक उच्चैरेतस्मिन्कृतनिजाभिमानस्याहरहस्तत्पोषणपालनलालन
प्रीणनानुध्यानेनात्मनियमाः सहयमाः पुरुषपरिचर्यादय एकैकशः कतिपयेनाहर्गणेन
वियुज्यमानाः किल सर्व एवोदवसन्‌
अहो बतायं हरिणकुणकः कृपण ईश्वररथचरणपरिभ्रमणरयेण स्वगणसुहृद्‌
बन्धुभ्यः परिवर्जितः शरणं च मोपसादितो मामेव मातापितरौ भ्रातृज्ञातीन्यौथिकांश्चैवोपेयाय
नान्यं कञ्चन वेद मय्यतिविस्रब्धश्चात एव मया मत्परायणस्य पोषणपालनप्रीणन
लालनमनसूयुनानुष्ठेयं शरण्योपेक्षादोषविदुषा
नूनं ह्यार्याः साधव उपशमशीलाः कृपणसुहृद एवंविधार्थे स्वार्थानपि
गुरुतरानुपेक्षन्ते
इति कृतानुषङ्ग आसनशयनाटनस्नानाशनादिषु सह मृगजहुना स्नेहानुबद्धहृदय आसीत्‌
कुशकुसुमसमित्पलाशफलमूलोदकान्याहरिष्यमाणो वृकसालावृकादिभ्यो
भयमाशंसमानो यदा सह हरिणकुणकेन वनं समाविशति
पथिषु च मुग्धभावेन तत्र तत्र विषक्तमतिप्रणयभरहृदयः
कार्पण्यात्स्कन्धेनोद्वहति एवमुत्सङ्ग उरसि चाधायोपलालयन्मुदं परमामवाप
क्रियायां निर्वर्त्यमानायामन्तरालेऽप्युत्थायोत्थाय यदैनमभिचक्षीत तर्हि वाव स वर्ष
पतिः प्रकृतिस्थेन मनसा तस्मा आशिष आशास्ते स्वस्ति स्ताद्वत्स ते सर्वत इति
अन्यदा भृशमुद्विग्नमना नष्टद्रविण इव कृपणः सकरुणमतितर्षेण हरिणकुणकविरह
विह्वलहृदयसन्तापस्तमेवानुशोचन्किल कश्मलं महदभिरम्भित इति होवाच
अपि बत स वै कृपण एणबालको मृतहरिणीसुतोऽहो ममानार्यस्य शठकिरातमतेरकृत
सुकृतस्य कृतविस्रम्भ आत्मप्रत्ययेन तदविगणयन्सुजन इवागमिष्यति
अपि क्षेमेणास्मिन्नाश्रमोपवने शष्पाणि चरन्तं देवगुप्तं द्रक्ष्यामि
अपि च न वृकः सालावृकोऽन्यतमो वा नैकचर एकचरो वा भक्षयति
निम्लोचति ह भगवान्सकलजगत्क्षेमोदयस्त्रय्यात्माद्यापि मम न मृगवधून्यास
आगच्छति
अपि स्विदकृतसुकृतमागत्य मां सुखयिष्यति हरिणराजकुमारो विविधरुचिरदर्शनीयनिज
मृगदारकविनोदैरसन्तोषं स्वानामपनुदन्‌
क्ष्वेलिकायां मां मृषासमाधिनामीलितदृशं प्रेमसंरम्भेण चकितचकित आगत्य पृषद्‌
अपरुषविषाणाग्रेण लुठति
आसादितहविषि बर्हिषि दूषिते मयोपालब्धो भीतभीतः सपद्युपरतरास ऋषिकुमारवदवहित
करणकलाप आस्ते
किं वा अरे आचरितं तपस्तपस्विन्यानया यदियमवनिः सविनयकृष्णसारतनयतनुतर
सुभगशिवतमाखरखुरपदपङ्क्तिभिर्द्रविणविधुरातुरस्य कृपणस्य मम द्रविणपदवीं
सूचयन्त्यात्मानं च सर्वतः कृतकौतुकं द्विजानां स्वर्गापवर्गकामानां देवयजनं करोति
अपि स्विदसौ भगवानुडुपतिरेनं मृगपतिभयान्मृतमातरं मृगबालकं स्वाश्रम
परिभ्रष्टमनुकम्पया कृपणजनवत्सलः परिपाति
किं वात्मजविश्लेषज्वरदवदहनशिखाभिरुपतप्यमानहृदयस्थलनलिनीकं
मामुपसृतमृगीतनयं शिशिरशान्तानुरागगुणितनिजवदनसलिलामृतमयगभस्तिभिः स्वधयतीति च
एवमघटमानमनोरथाकुलहृदयो मृगदारकाभासेन स्वारब्धकर्मणा योगारम्भणतो
विभ्रंशितः स योगतापसो भगवदाराधनलक्षणाच्च कथमितरथा जात्यन्तर एणकुणक आसङ्गः
साक्षान्निःश्रेयसप्रतिपक्षतया प्राक्परित्यक्तदुस्त्यजहृदयाभिजातस्य तस्यैवमन्तरायविहत
योगारम्भणस्य राजर्षेर्भरतस्य तावन्मृगार्भकपोषणपालनप्रीणनलालनानुषङ्गेणाविगणयत
आत्मानमहिरिवाखुबिलं दुरतिक्रमः कालः करालरभस आपद्यत
तदानीमपि पार्श्ववर्तिनमात्मजमिवानुशोचन्तमभिवीक्षमाणो मृग एवाभिनिवेशितमना
विसृज्य लोकमिमं सह मृगेण कलेवरं मृतमनु न मृतजन्मानुस्मृतिरितरवन्मृगशरीरमवाप
तत्रापि ह वा आत्मनो मृगत्वकारणं भगवदाराधनसमीहानुभावेनानुस्मृत्य
भृशमनुतप्यमान आह
अहो कष्टं भ्रष्टोऽहमात्मवतामनुपथाद्यद्विमुक्तसमस्तसङ्गस्य विविक्तपुण्यारण्य
शरणस्यात्मवत आत्मनि सर्वेषामात्मनां भगवति वासुदेवे तदनुश्रवणमनन
सङ्कीर्तनाराधनानुस्मरणाभियोगेनाशून्यसकलयामेन कालेन समावेशितं समाहितं कार्त्स्न्येन
मनस्तत्तु पुनर्ममाबुधस्यारान्मृगसुतमनु परिसुस्राव
इत्येवं निगूढनिर्वेदो विसृज्य मृगीं मातरं पुनर्भगवत्क्षेत्रमुपशमशीलमुनिगण
दयितं शालग्रामं पुलस्त्यपुलहाश्रमं कालञ्जरात्प्रत्याजगाम
तस्मिन्नपि कालं प्रतीक्षमाणः सङ्गाच्च भृशमुद्विग्न आत्मसहचरः शुष्कपर्णतृण
वीरुधा वर्तमानो मृगत्वनिमित्तावसानमेव गणयन्मृगशरीरं तीर्थोदकक्लिन्नमुत्ससर्ज
श्रीशुक उवाच
अथ कस्यचिद्द्विजवरस्याङ्गिरःप्रवरस्य शमदमतपःस्वाध्यायाध्ययनत्यागसन्तोष
तितिक्षाप्रश्रयविद्यानसूयात्मज्ञानानन्दयुक्तस्यात्मसदृशश्रुतशीलाचाररूपौदार्यगुणा नव सोदर्या
अङ्गजा बभूवुर्मिथुनं च यवीयस्यां भार्यायाम्यस्तु तत्र पुमांस्तं परमभागवतं राजर्षि
प्रवरं भरतमुत्सृष्टमृगशरीरं चरमशरीरेण विप्रत्वं गतमाहुः
तत्रापि स्वजनसङ्गाच्च भृशमुद्विजमानो भगवतः कर्मबन्धविध्वंसनश्रवण
स्मरणगुणविवरणचरणारविन्दयुगलं मनसा विदधदात्मनः प्रतिघातमाशङ्कमानो भगवद्‌
अनुग्रहेणानुस्मृतस्वपूर्वजन्मावलिरात्मानमुन्मत्तजडान्धबधिरस्वरूपेण दर्शयामास
लोकस्य
तस्यापि ह वा आत्मजस्य विप्रः पुत्रस्नेहानुबद्धमना आसमावर्तनात्संस्कारान्यथोपदेशं
विदधान उपनीतस्य च पुनः शौचाचमनादीन्कर्मनियमाननभिप्रेतानपि समशिक्षयदनुशिष्टेन हि
भाव्यं पितुः पुत्रेणेति
स चापि तदु ह पितृसन्निधावेवासध्रीचीनमिव स्म करोति छन्दांस्यध्यापयिष्यन्सह
व्याहृतिभिः सप्रणवशिरस्त्रिपदीं सावित्रीं ग्रैष्मवासन्तिकान्मासानधीयानमप्यसमवेतरूपं
ग्राहयामास
एवं स्वतनुज आत्मन्यनुरागावेशितचित्तः शौचाध्ययनव्रतनियमगुर्वनलशुश्रूषणाद्य्‌
औपकुर्वाणककर्माण्यनभियुक्तान्यपि समनुशिष्टेन भाव्यमित्यसदाग्रहः पुत्रमनुशास्य
स्वयं तावदनधिगतमनोरथः कालेनाप्रमत्तेन स्वयं गृह एव प्रमत्त उपसंहृतः
अथ यवीयसी द्विजसती स्वगर्भजातं मिथुनं सपत्न्या उपन्यस्य स्वयमनुसंस्थया
पतिलोकमगात्‌
पितर्युपरते भ्रातर एनमतत्प्रभावविदस्त्रय्यां विद्यायामेव पर्यवसितमतयो न
परविद्यायां जडमतिरिति भ्रातुरनुशासननिर्बन्धान्न्यवृत्सन्त
स च प्राकृतैर्द्विपदपशुभिरुन्मत्तजडबधिरमूकेत्यभिभाष्यमाणो यदा तदनुरूपाणि
प्रभाषते कर्माणि च कार्यमाणः परेच्छया करोति विष्टितो वेतनतो वा याच्ञ्या यदृच्छया वोपसादितमल्पं
बहु मृष्टं कदन्नं वाभ्यवहरति परं नेन्द्रियप्रीतिनिमित्तम्नित्यनिवृत्तनिमित्तस्वसिद्ध
विशुद्धानुभवानन्दस्वात्मलाभाधिगमः सुखदुःखयोर्द्वन्द्वनिमित्तयोरसम्भावित
देहाभिमानः
शीतोष्णवातवर्षेषु वृष इवानावृताङ्गः पीनः संहननाङ्गः स्थण्डिल
संवेशनानुन्मर्दनामज्जनरजसा महामणिरिवानभिव्यक्तब्रह्मवर्चसः कुपटावृत
कटिरुपवीतेनोरुमषिणा द्विजातिरिति ब्रह्मबन्धुरिति संज्ञयातज्ज्ञजनावमतो विचचार
यदा तु परत आहारं कर्मवेतनत ईहमानः स्वभ्रातृभिरपि केदारकर्मणि निरूपितस्तदपि
करोति किन्तु न समं विषमं न्यूनमधिकमिति वेद कणपिण्याकफलीकरणकुल्माष
स्थालीपुरीषादीन्यप्यमृतवदभ्यवहरति
अथ कदाचित्कश्चिद्वृषलपतिर्भद्रकाल्यै पुरुषपशुमालभतापत्यकामः
तस्य ह दैवमुक्तस्य पशोः पदवीं तदनुचराः परिधावन्तो निशि निशीथसमये
तमसावृतायामनधिगतपशव आकस्मिकेन विधिना केदारान्वीरासनेन मृगवराहादिभ्यः
संरक्षमाणमङ्गिरःप्रवरसुतमपश्यन्‌
अथ त एनमनवद्यलक्षणमवमृश्य भर्तृकर्मनिष्पत्तिं मन्यमाना बद्ध्वा रशनया
चण्डिकागृहमुपनिन्युर्मुदा विकसितवदनाः
अथ पणयस्तं स्वविधिनाभिषिच्याहतेन वाससाच्छाद्य भूषणालेपस्रक्तिलकादिभिरुपस्कृतं
भुक्तवन्तं धूपदीपमाल्यलाजकिसलयाङ्कुरफलोपहारोपेतया वैशससंस्थया महता गीतस्तुति
मृदङ्गपणवघोषेण च पुरुषपशुं भद्रकाल्याः पुरत उपवेशयामासुः
अथ वृषलराजपणिः पुरुषपशोरसृगासवेन देवीं भद्रकालीं यक्ष्यमाणस्तद्‌
अभिमन्त्रितमसिमतिकरालनिशितमुपाददे
इति तेषां वृषलानां रजस्तमःप्रकृतीनां धनमदरजौत्सिक्तमनसां भगवत्कलावीर
कुलं कदर्थीकृत्योत्पथेन स्वैरं विहरतां हिंसाविहाराणां कर्मातिदारुणं यद्ब्रह्मभूतस्य
साक्षाद्ब्रह्मर्षिसुतस्य निर्वैरस्य सर्वभूतसुहृदः सूनायामप्यननुमतमालम्भनं
तदुपलभ्य ब्रह्मतेजसातिदुर्विषहेण दन्दह्यमानेन वपुषा सहसोच्चचाट सैव देवी भद्रकाली
भृशममर्षरोषावेशरभसविलसितभ्रुकुटिविटपकुटिलदंष्ट्रारुणेक्षणाटोपातिभयानक
वदना हन्तुकामेवेदं महाट्टहासमतिसंरम्भेण विमुञ्चन्ती तत उत्पत्य पापीयसां दुष्टानां
तेनैवासिना विवृक्णशीर्ष्णां गलात्स्रवन्तमसृगासवमत्युष्णं सह गणेन निपीयातिपानमद
विह्वलोच्चैस्तरां स्वपार्षदैः सह जगौ ननर्त च विजहार च शिरःकन्दुकलीलया
एवमेव खलु महदभिचारातिक्रमः कार्त्स्न्येनात्मने फलति
न वा एतद्विष्णुदत्त महदद्भुतं यदसम्भ्रमः स्वशिरश्छेदन आपतितेऽपि विमुक्त
देहाद्यात्मभावसुदृढहृदयग्रन्थीनां सर्वसत्त्वसुहृदात्मनां निर्वैराणां
साक्षाद्भगवतानिमिषारिवरायुधेनाप्रमत्तेन तैस्तैर्भावैः परिरक्ष्यमाणानां तत्पाद
मूलमकुतश्चिद्भयमुपसृतानां भागवतपरमहंसानाम्‌
श्रीशुक उवाच
अथ सिन्धुसौवीरपते रहूगणस्य व्रजत इक्षुमत्यास्तटे तत्कुलपतिना शिबिकावाह
पुरुषान्वेषणसमये दैवेनोपसादितः स द्विजवर उपलब्ध एष पीवा युवा संहननाङ्गो गो
खरवद्धुरं वोढुमलमिति पूर्वविष्टिगृहीतैः सह गृहीतः प्रसभमतदर्ह उवाह शिबिकां स
महानुभावः
यदा हि द्विजवरस्येषुमात्रावलोकानुगतेर्न समाहिता पुरुषगतिस्तदा विषमगतां स्वशिबिकां
रहूगण उपधार्य पुरुषानधिवहत आह हे वोढारः साध्वतिक्रमत किमिति विषममुह्यते यानमिति
अथ त ईश्वरवचः सोपालम्भमुपाकर्ण्योपायतुरीयाच्छङ्कितमनसस्तं विज्ञापयां
बभूवुः
न वयं नरदेव प्रमत्ता भवन्नियमानुपथाः साध्वेव वहामः अयमधुनैव नियुक्तो
ऽपि न द्रुतं व्रजति नानेन सह वोढुमु ह वयं पारयाम इति
सांसर्गिको दोष एव नूनमेकस्यापि सर्वेषां सांसर्गिकाणां भवितुमर्हतीति निश्चित्य निशम्य
कृपणवचो राजा रहूगण उपासितवृद्धोऽपि निसर्गेण बलात्कृत ईषदुत्थितमन्युरविस्पष्टब्रह्म
तेजसं जातवेदसमिव रजसावृतमतिराह
अहो कष्टं भ्रातर्व्यक्तमुरुपरिश्रान्तो दीर्घमध्वानमेक एव ऊहिवान्सुचिरं नातिपीवा न
संहननाङ्गो जरसा चोपद्रुतो भवान्सखे नो एवापर एते सङ्घट्टिन इति बहुविप्रलब्धोऽप्यविद्यया
रचितद्रव्यगुणकर्माशयस्वचरमकलेवरेऽवस्तुनि संस्थानविशेषेऽहं ममेत्यनध्यारोपित
मिथ्याप्रत्ययो ब्रह्मभूतस्तूष्णीं शिबिकां पूर्ववदुवाह
अथ पुनः स्वशिबिकायां विषमगतायां प्रकुपित उवाच रहूगणः किमिदमरे त्वं जीवन्मृतो
मां कदर्थीकृत्य भर्तृशासनमतिचरसि प्रमत्तस्य च ते करोमि चिकित्सां दण्डपाणिरिव जनताया यथा
प्रकृतिं स्वां भजिष्यस इति
एवं बह्वबद्धमपि भाषमाणं नरदेवाभिमानं रजसा तमसानुविद्धेन मदेन
तिरस्कृताशेषभगवत्प्रियनिकेतं पण्डितमानिनं स भगवान्ब्राह्मणो ब्रह्मभूतसर्वभूत
सुहृदात्मा योगेश्वरचर्यायां नातिव्युत्पन्नमतिं स्मयमान इव विगतस्मय इदमाह
ब्राह्मण उवाच
त्वयोदितं व्यक्तमविप्रलब्धं भर्तुः स मे स्याद्यदि वीर भारः
गन्तुर्यदि स्यादधिगम्यमध्वा पीवेति राशौ न विदां प्रवादः
स्थौल्यं कार्श्यं व्याधय आधयश्च क्षुत्तृड्भयं कलिरिच्छा जरा च
निद्रा रतिर्मन्युरहं मदः शुचो देहेन जातस्य हि मे न सन्ति
जीवन्मृतत्वं नियमेन राजनाद्यन्तवद्यद्विकृतस्य दृष्टम्‌
स्वस्वाम्यभावो ध्रुव ईड्य यत्र तर्ह्युच्यतेऽसौ विधिकृत्ययोगः
विशेषबुद्धेर्विवरं मनाक्च पश्याम यन्न व्यवहारतोऽन्यत्‌
क ईश्वरस्तत्र किमीशितव्यं तथापि राजन्करवाम किं ते
उन्मत्तमत्तजडवत्स्वसंस्थां गतस्य मे वीर चिकित्सितेन
अर्थः कियान्भवता शिक्षितेन स्तब्धप्रमत्तस्य च पिष्टपेषः
श्रीशुक उवाच
एतावदनुवादपरिभाषया प्रत्युदीर्य मुनिवर उपशमशील उपरतानात्म्यनिमित्त उपभोगेन
कर्मारब्धं व्यपनयन्राजयानमपि तथोवाह
स चापि पाण्डवेय सिन्धुसौवीरपतिस्तत्त्वजिज्ञासायां सम्यक्‌
श्रद्धयाधिकृताधिकारस्तद्धृदयग्रन्थिमोचनं द्विजवच आश्रुत्य बहुयोगग्रन्थसम्मतं
त्वरयावरुह्य शिरसा पादमूलमुपसृतः क्षमापयन्विगतनृपदेवस्मय उवाच
कस्त्वं निगूढश्चरसि द्विजानां बिभर्षि सूत्रं कतमोऽवधूतः
कस्यासि कुत्रत्य इहापि कस्मात्क्षेमाय नश्चेदसि नोत शुक्लः
नाहं विशङ्के सुरराजवज्रान्न त्र्यक्षशूलान्न यमस्य दण्डात्‌
नाग्न्यर्कसोमानिलवित्तपास्त्राच्छङ्के भृशं ब्रह्मकुलावमानात्‌
तद्ब्रूह्यसङ्गो जडवन्निगूढ विज्ञानवीर्यो विचरस्यपारः
वचांसि योगग्रथितानि साधो न नः क्षमन्ते मनसापि भेत्तुम्‌
अहं च योगेश्वरमात्मतत्त्व विदां मुनीनां परमं गुरुं वै
प्रष्टुं प्रवृत्तः किमिहारणं तत्साक्षाद्धरिं ज्ञानकलावतीर्णम्‌
स वै भवा लोकनिरीक्षणार्थमव्यक्तलिङ्गो विचरत्यपि स्वित्‌
योगेश्वराणां गतिमन्धबुद्धिः कथं विचक्षीत गृहानुबन्धः
दृष्टः श्रमः कर्मत आत्मनो वै भर्तुर्गन्तुर्भवतश्चानुमन्ये
यथासतोदानयनाद्यभावात्समूल इष्टो व्यवहारमार्गः
स्थाल्यग्नितापात्पयसोऽभितापस्तत्तापतस्तण्डुलगर्भरन्धिः
देहेन्द्रियास्वाशयसन्निकर्षात्तत्संसृतिः पुरुषस्यानुरोधात्‌
शास्ताभिगोप्ता नृपतिः प्रजानां यः किङ्करो वै न पिनष्टि पिष्टम्‌
स्वधर्ममाराधनमच्युतस्य यदीहमानो विजहात्यघौघम्‌
तन्मे भवान्नरदेवाभिमान मदेन तुच्छीकृतसत्तमस्य
कृषीष्ट मैत्रीदृशमार्तबन्धो यथा तरे सदवध्यानमंहः
न विक्रिया विश्वसुहृत्सखस्य साम्येन वीताभिमतेस्तवापि
महद्विमानात्स्वकृताद्धि मादृङ्नङ्क्ष्यत्यदूरादपि शूलपाणिः
ब्राह्मण उवाच
अकोविदः कोविदवादवादान्वदस्यथो नातिविदां वरिष्ठः
न सूरयो हि व्यवहारमेनं तत्त्वावमर्शेन सहामनन्ति
तथैव राजन्नुरुगार्हमेध वितानविद्योरुविजृम्भितेषु
न वेदवादेषु हि तत्त्ववादः प्रायेण शुद्धो नु चकास्ति साधुः
न तस्य तत्त्वग्रहणाय साक्षाद्वरीयसीरपि वाचः समासन्‌
स्वप्ने निरुक्त्या गृहमेधिसौख्यं न यस्य हेयानुमितं स्वयं स्यात्‌
यावन्मनो रजसा पूरुषस्य सत्त्वेन वा तमसा वानुरुद्धम्‌
चेतोभिराकूतिभिरातनोति निरङ्कुशं कुशलं चेतरं वा
स वासनात्मा विषयोपरक्तो गुणप्रवाहो विकृतः षोडशात्मा
बिभ्रत्पृथङ्नामभि रूपभेदमन्तर्बहिष्ट्वं च पुरैस्तनोति
दुःखं सुखं व्यतिरिक्तं च तीव्रं कालोपपन्नं फलमाव्यनक्ति
आलिङ्ग्य मायारचितान्तरात्मा स्वदेहिनं संसृतिचक्रकूटः
तावानयं व्यवहारः सदाविः क्षेत्रज्ञसाक्ष्यो भवति स्थूलसूक्ष्मः
तस्मान्मनो लिङ्गमदो वदन्ति गुणागुणत्वस्य परावरस्य
गुणानुरक्तं व्यसनाय जन्तोः क्षेमाय नैर्गुण्यमथो मनः स्यात्‌
यथा प्रदीपो घृतवर्तिमश्नन्शिखाः सधूमा भजति ह्यन्यदा स्वम्‌
पदं तथा गुणकर्मानुबद्धं वृत्तीर्मनः श्रयतेऽन्यत्र तत्त्वम्‌
एकादशासन्मनसो हि वृत्तय आकूतयः पञ्च धियोऽभिमानः
मात्राणि कर्माणि पुरं च तासां वदन्ति हैकादश वीर भूमीः
गन्धाकृतिस्पर्शरसश्रवांसि विसर्गरत्यर्त्यभिजल्पशिल्पाः
एकादशं स्वीकरणं ममेति शय्यामहं द्वादशमेक आहुः
द्रव्यस्वभावाशयकर्मकालैरेकादशामी मनसो विकाराः
सहस्रशः शतशः कोटिशश्च क्षेत्रज्ञतो न मिथो न स्वतः स्युः
क्षेत्रज्ञ एता मनसो विभूतीर्जीवस्य मायारचितस्य नित्याः
आविर्हिताः क्वापि तिरोहिताश्च शुद्धो विचष्टे ह्यविशुद्धकर्तुः
क्षेत्रज्ञ आत्मा पुरुषः पुराणः साक्षात्स्वयं ज्योतिरजः परेशः
नारायणो भगवान्वासुदेवः स्वमाययात्मन्यवधीयमानः
यथानिलः स्थावरजङ्गमानामात्मस्वरूपेण निविष्ट ईशेत्‌
एवं परो भगवान्वासुदेवः क्षेत्रज्ञ आत्मेदमनुप्रविष्टः
न यावदेतां तनुभृन्नरेन्द्र विधूय मायां वयुनोदयेन
विमुक्तसङ्गो जितषट्सपत्नो वेदात्मतत्त्वं भ्रमतीह तावत्‌
न यावदेतन्मन आत्मलिङ्गं संसारतापावपनं जनस्य
यच्छोकमोहामयरागलोभ वैरानुबन्धं ममतां विधत्ते
भ्रातृव्यमेनं तददभ्रवीर्यमुपेक्षयाध्येधितमप्रमत्तः
गुरोर्हरेश्चरणोपासनास्त्रो जहि व्यलीकं स्वयमात्ममोषम्‌
रहूगण उवाच
नमो नमः कारणविग्रहाय स्वरूपतुच्छीकृतविग्रहाय
नमोऽवधूत द्विजबन्धुलिङ्ग निगूढनित्यानुभवाय तुभ्यम्‌
ज्वरामयार्तस्य यथागदं सत्निदाघदग्धस्य यथा हिमाम्भः
कुदेहमानाहिविदष्टदृष्टेः ब्रह्मन्वचस्तेऽमृतमौषधं मे
तस्माद्भवन्तं मम संशयार्थं प्रक्ष्यामि पश्चादधुना सुबोधम्‌
अध्यात्मयोगग्रथितं तवोक्तमाख्याहि कौतूहलचेतसो मे
यदाह योगेश्वर दृश्यमानं क्रियाफलं सद्व्यवहारमूलम्‌
न ह्यञ्जसा तत्त्वविमर्शनाय भवानमुष्मिन्भ्रमते मनो मे
ब्राह्मण उवाच
अयं जनो नाम चलन्पृथिव्यां यः पार्थिवः पार्थिव कस्य हेतोः
तस्यापि चाङ्घ्र्योरधि गुल्फजङ्घा जानूरुमध्योरशिरोधरांसाः
अंसेऽधि दार्वी शिबिका च यस्यां सौवीरराजेत्यपदेश आस्ते
यस्मिन्भवान्रूढनिजाभिमानो राजास्मि सिन्धुष्विति दुर्मदान्धः
शोच्यानिमांस्त्वमधिकष्टदीनान्विष्ट्या निगृह्णन्निरनुग्रहोऽसि
जनस्य गोप्तास्मि विकत्थमानो न शोभसे वृद्धसभासु धृष्टः
यदा क्षितावेव चराचरस्य विदाम निष्ठां प्रभवं च नित्यम्‌
तन्नामतोऽन्यद्व्यवहारमूलं निरूप्यतां सत्क्रिययानुमेयम्‌
एवं निरुक्तं क्षितिशब्दवृत्तमसन्निधानात्परमाणवो ये
अविद्यया मनसा कल्पितास्ते येषां समूहेन कृतो विशेषः
एवं कृशं स्थूलमणुर्बृहद्यदसच्च सज्जीवमजीवमन्यत्‌
द्रव्यस्वभावाशयकालकर्म नाम्नाजयावेहि कृतं द्वितीयम्‌
ज्ञानं विशुद्धं परमार्थमेकमनन्तरं त्वबहिर्ब्रह्म सत्यम्‌
प्रत्यक्प्रशान्तं भगवच्छब्दसंज्ञं यद्वासुदेवं कवयो वदन्ति
रहूगणैतत्तपसा न याति न चेज्यया निर्वपणाद्गृहाद्वा
न च्छन्दसा नैव जलाग्निसूर्यैर्विना महत्पादरजोऽभिषेकम्‌
यत्रोत्तमश्लोकगुणानुवादः प्रस्तूयते ग्राम्यकथाविघातः
निषेव्यमाणोऽनुदिनं मुमुक्षोर्मतिं सतीं यच्छति वासुदेवे
अहं पुरा भरतो नाम राजा विमुक्तदृष्टश्रुतसङ्गबन्धः
आराधनं भगवत ईहमानो मृगोऽभवं मृगसङ्गाद्धतार्थः
सा मां स्मृतिर्मृगदेहेऽपि वीर कृष्णार्चनप्रभवा नो जहाति
अथो अहं जनसङ्गादसङ्गो विशङ्कमानोऽविवृतश्चरामि
तस्मान्नरोऽसङ्गसुसङ्गजात ज्ञानासिनेहैव विवृक्णमोहः
हरिं तदीहाकथनश्रुताभ्यां लब्धस्मृतिर्यात्यतिपारमध्वनः
ब्राह्मण उवाच
दुरत्ययेऽध्वन्यजया निवेशितो रजस्तमःसत्त्वविभक्तकर्मदृक्‌
स एष सार्थोऽर्थपरः परिभ्रमन्भवाटवीं याति न शर्म विन्दति
यस्यामिमे षण्नरदेव दस्यवः सार्थं विलुम्पन्ति कुनायकं बलात्‌
गोमायवो यत्र हरन्ति सार्थिकं प्रमत्तमाविश्य यथोरणं वृकाः
प्रभूतवीरुत्तृणगुल्मगह्वरे कठोरदंशैर्मशकैरुपद्रुतः
क्वचित्तु गन्धर्वपुरं प्रपश्यति क्वचित्क्वचिच्चाशुरयोल्मुकग्रहम्‌
निवासतोयद्रविणात्मबुद्धिस्ततस्ततो धावति भो अटव्याम्‌
क्वचिच्च वात्योत्थितपांसुधूम्रा दिशो न जानाति रजस्वलाक्षः
अदृश्यझिल्लीस्वनकर्णशूल उलूकवाग्भिर्व्यथितान्तरात्मा
अपुण्यवृक्षान्श्रयते क्षुधार्दितो मरीचितोयान्यभिधावति क्वचित्‌
क्वचिद्वितोयाः सरितोऽभियाति परस्परं चालषते निरन्धः
आसाद्य दावं क्वचिदग्नितप्तो निर्विद्यते क्व च यक्षैर्हृतासुः
शूरैर्हृतस्वः क्व च निर्विण्णचेताः शोचन्विमुह्यन्नुपयाति कश्मलम्‌
क्वचिच्च गन्धर्वपुरं प्रविष्टः प्रमोदते निर्वृतवन्मुहूर्तम्‌
चलन्क्वचित्कण्टकशर्कराङ्घ्रिर्नगारुरुक्षुर्विमना इवास्ते
पदे पदेऽभ्यन्तरवह्निनार्दितः कौटुम्बिकः क्रुध्यति वै जनाय
क्वचिन्निगीर्णोऽजगराहिना जनो नावैति किञ्चिद्विपिनेऽपविद्धः
दष्टः स्म शेते क्व च दन्दशूकैरन्धोऽन्धकूपे पतितस्तमिस्रे
कर्हि स्म चित्क्षुद्ररसान्विचिन्वंस्तन्मक्षिकाभिर्व्यथितो विमानः
तत्रातिकृच्छ्रात्प्रतिलब्धमानो बलाद्विलुम्पन्त्यथ तं ततोऽन्ये
क्वचिच्च शीतातपवातवर्ष प्रतिक्रियां कर्तुमनीश आस्ते
क्वचिन्मिथो विपणन्यच्च किञ्चिद्विद्वेषमृच्छत्युत वित्तशाठ्यात्‌
क्वचित्क्वचित्क्षीणधनस्तु तस्मिन्शय्यासनस्थानविहारहीनः
याचन्परादप्रतिलब्धकामः पारक्यदृष्टिर्लभतेऽवमानम्‌
अन्योन्यवित्तव्यतिषङ्गवृद्ध वैरानुबन्धो विवहन्मिथश्च
अध्वन्यमुष्मिन्नुरुकृच्छ्रवित्त बाधोपसर्गैर्विहरन्विपन्नः
तांस्तान्विपन्नान्स हि तत्र तत्र विहाय जातं परिगृह्य सार्थः
आवर्ततेऽद्यापि न कश्चिदत्र वीराध्वनः पारमुपैति योगम्‌
मनस्विनो निर्जितदिग्गजेन्द्रा ममेति सर्वे भुवि बद्धवैराः
मृधे शयीरन्न तु तद्व्रजन्ति यन्न्यस्तदण्डो गतवैरोऽभियाति
प्रसज्जति क्वापि लताभुजाश्रयस्तदाश्रयाव्यक्तपदद्विजस्पृहः
क्वचित्कदाचिद्धरिचक्रतस्त्रसन्सख्यं विधत्ते बककङ्कगृध्रैः
तैर्वञ्चितो हंसकुलं समाविशन्नरोचयन्शीलमुपैति वानरान्‌
तज्जातिरासेन सुनिर्वृतेन्द्रियः परस्परोद्वीक्षणविस्मृतावधिः
द्रुमेषु रंस्यन्सुतदारवत्सलो व्यवायदीनो विवशः स्वबन्धने
क्वचित्प्रमादाद्गिरिकन्दरे पतन्वल्लीं गृहीत्वा गजभीत आस्थितः
अतः कथञ्चित्स विमुक्त आपदः पुनश्च सार्थं प्रविशत्यरिन्दम
अध्वन्यमुष्मिन्नजया निवेशितो भ्रमञ्जनोऽद्यापि न वेद कश्चन
रहूगण त्वमपि ह्यध्वनोऽस्य सन्न्यस्तदण्डः कृतभूतमैत्रः
असज्जितात्मा हरिसेवया शितं ज्ञानासिमादाय तरातिपारम्‌
राजोवाच
अहो नृजन्माखिलजन्मशोभनं किं जन्मभिस्त्वपरैरप्यमुष्मिन्‌
न यद्धृषीकेशयशःकृतात्मनां महात्मनां वः प्रचुरः समागमः
न ह्यद्भुतं त्वच्चरणाब्जरेणुभिर्हतांहसो भक्तिरधोक्षजेऽमला
मौहूर्तिकाद्यस्य समागमाच्च मे दुस्तर्कमूलोऽपहतोऽविवेकः
नमो महद्भ्योऽस्तु नमः शिशुभ्यो नमो युवभ्यो नम आवटुभ्यः
ये ब्राह्मणा गामवधूतलिङ्गाश्चरन्ति तेभ्यः शिवमस्तु राज्ञाम्‌
श्रीशुक उवाच
इत्येवमुत्तरामातः स वै ब्रह्मर्षिसुतः सिन्धुपतय आत्मसतत्त्वं विगणयतः
परानुभावः परमकारुणिकतयोपदिश्य रहूगणेन सकरुणमभिवन्दितचरण आपूर्णार्णव इव
निभृतकरणोर्म्याशयो धरणिमिमां विचचार
सौवीरपतिरपि सुजनसमवगतपरमात्मसतत्त्व आत्मन्यविद्याध्यारोपितां च देहात्म
मतिं विससर्ज एवं हि नृप भगवदाश्रिताश्रितानुभावः
राजोवाच
यो ह वा इह बहुविदा महाभागवत त्वयाभिहितः परोक्षेण वचसा जीवलोकभवाध्वा स
ह्यार्यमनीषया कल्पितविषयोनाञ्जसाव्युत्पन्नलोकसमधिगमः अथ तदेवैतद्दुरवगमं
समवेतानुकल्पेन निर्दिश्यतामिति
स होवाच
स एष देहात्ममानिनां सत्त्वादिगुणविशेषविकल्पितकुशलाकुशलसमवहारविनिर्मितविविध
देहावलिभिर्वियोगसंयोगाद्यनादिसंसारानुभवस्य द्वारभूतेन षडिन्द्रियवर्गेण
तस्मिन्दुर्गाध्ववदसुगमेऽध्वन्यापतित ईश्वरस्य भगवतो विष्णोर्वशवर्तिन्या मायया जीवलोको
ऽयं यथा वणिक्सार्थोऽर्थपरः स्वदेहनिष्पादितकर्मानुभवः श्मशानवदशिवतमायां
संसाराटव्यां गतो नाद्यापि विफलबहुप्रतियोगेहस्तत्तापोपशमनीं हरिगुरुचरणारविन्द
मधुकरानुपदवीमवरुन्धे
यस्यामु ह वा एते षडिन्द्रियनामानः कर्मणा दस्यव एव ते तद्यथा पुरुषस्य धनं
यत्किञ्चिद्धर्मौपयिकं बहुकृच्छ्राधिगतं साक्षात्परमपुरुषाराधनलक्षणो योऽसौ धर्मस्तं
तु साम्पराय उदाहरन्ति तद्धर्म्यं धनं दर्शनस्पर्शनश्रवणास्वादनावघ्राणसङ्कल्प
व्यवसायगृहग्राम्योपभोगेन कुनाथस्याजितात्मनो यथा सार्थस्य विलुम्पन्ति
अथ च यत्र कौटुम्बिका दारापत्यादयो नाम्ना कर्मणा वृकसृगाला एवानिच्छतोऽपि कदर्यस्य
कुटुम्बिन उरणकवत्संरक्ष्यमाणं मिषतोऽपि हरन्ति
यथा ह्यनुवत्सरं कृष्यमाणमप्यदग्धबीजं क्षेत्रं पुनरेवावपनकाले गुल्मतृण
वीरुद्भिर्गह्वरमिव भवत्येवमेव गृहाश्रमः कर्मक्षेत्रं यस्मिन्न हि कर्माण्युत्सीदन्ति
यदयं कामकरण्ड एष आवसथः
तत्र गतो दंशमशकसमापसदैर्मनुजैः शलभशकुन्ततस्कर
मूषकादिभिरुपरुध्यमानबहिःप्राणः क्वचित्परिवर्तमानोऽस्मिन्नध्वन्यविद्याकाम
कर्मभिरुपरक्तमनसानुपपन्नार्थं नरलोकं गन्धर्वनगरमुपपन्नमिति मिथ्या
दृष्टिरनुपश्यति
तत्र च क्वचिदातपोदकनिभान्विषयानुपधावति पानभोजनव्यवायादिव्यसनलोलुपः
क्वचिच्चाशेषदोषनिषदनं पुरीषविशेषं तद्वर्णगुणनिर्मितमतिः
सुवर्णमुपादित्सत्यग्निकामकातर इवोल्मुकपिशाचम्‌
अथ कदाचिन्निवासपानीयद्रविणाद्यनेकात्मोपजीवनाभिनिवेश एतस्यां संसाराटव्यामितस्ततः
परिधावति
क्वचिच्च वात्यौपम्यया प्रमदयारोहमारोपितस्तत्कालरजसा रजनीभूत इवासाधुमर्यादो
रजस्वलाक्षोऽपि दिग्देवता अतिरजस्वलमतिर्न विजानाति
क्वचित्सकृदवगतविषयवैतथ्यः स्वयं पराभिध्यानेन विभ्रंशितस्मृतिस्तयैव मरीचि
तोयप्रायांस्तानेवाभिधावति
क्वचिदुलूकझिल्लीस्वनवदतिपरुषरभसाटोपं प्रत्यक्षं परोक्षं वा रिपुराजकुल
निर्भर्त्सितेनातिव्यथितकर्णमूलहृदयः
स यदा दुग्धपूर्वसुकृतस्तदा कारस्करकाकतुण्डाद्यपुण्यद्रुमलताविषोद
पानवदुभयार्थशून्यद्रविणान्जीवन्मृतान्स्वयं जीवन्म्रियमाण उपधावति
एकदासत्प्रसङ्गान्निकृतमतिर्व्युदकस्रोतःस्खलनवदुभयतोऽपि दुःखदं
पाखण्डमभियाति
यदा तु परबाधयान्ध आत्मने नोपनमति तदा हि पितृपुत्रबर्हिष्मतः पितृपुत्रान्वा स खलु
भक्षयति
क्वचिदासाद्य गृहं दाववत्प्रियार्थविधुरमसुखोदर्कं शोकाग्निना दह्यमानो भृशं
निर्वेदमुपगच्छति
क्वचित्कालविषमितराजकुलरक्षसापहृतप्रियतमधनासुः प्रमृतक इव विगतजीवलक्षण
आस्ते
कदाचिन्मनोरथोपगतपितृपितामहाद्यसत्सदिति स्वप्ननिर्वृतिलक्षणमनुभवति
क्वचिद्गृहाश्रमकर्मचोदनातिभरगिरिमारुरुक्षमाणो लोकव्यसनकर्षितमनाः कण्टक
शर्कराक्षेत्रं प्रविशन्निव सीदति
क्वचिच्च दुःसहेन कायाभ्यन्तरवह्निना गृहीतसारः स्वकुटुम्बाय क्रुध्यति
स एव पुनर्निद्राजगरगृहीतोऽन्धे तमसि मग्नः शून्यारण्य इव शेते नान्यत्किञ्चन वेद शव
इवापविद्धः
कदाचिद्भग्नमानदंष्ट्रो दुर्जनदन्दशूकैरलब्धनिद्राक्षणो व्यथित
हृदयेनानुक्षीयमाणविज्ञानोऽन्धकूपेऽन्धवत्पतति
कर्हि स्म चित्काममधुलवान्विचिन्वन्यदा परदारपरद्रव्याण्यवरुन्धानो राज्ञा
स्वामिभिर्वा निहतः पतत्यपारे निरये
अथ च तस्मादुभयथापि हि कर्मास्मिन्नात्मनः संसारावपनमुदाहरन्ति
मुक्तस्ततो यदि बन्धाद्देवदत्त उपाच्छिनत्ति तस्मादपि विष्णुमित्र इत्यनवस्थितिः
क्वचिच्च शीतवाताद्यनेकाधिदैविकभौतिकात्मीयानां दशानां प्रतिनिवारणेऽकल्पो दुरन्त
चिन्तया विषण्ण आस्ते
क्वचिन्मिथो व्यवहरन्यत्किञ्चिद्धनमन्येभ्यो वा काकिणिका
मात्रमप्यपहरन्यत्किञ्चिद्वा विद्वेषमेति वित्तशाठ्यात्‌
अध्वन्यमुष्मिन्निम उपसर्गास्तथा सुखदुःखरागद्वेषभयाभिमानप्रमादोन्माद
शोकमोहलोभमात्सर्येर्ष्यावमानक्षुत्पिपासाधिव्याधिजन्मजरामरणादयः
क्वापि देवमायया स्त्रिया भुजलतोपगूढः प्रस्कन्नविवेकविज्ञानो यद्विहारगृहारम्भाकुल
हृदयस्तदाश्रयावसक्तसुतदुहितृकलत्रभाषितावलोकविचेष्टितापहृतहृदय आत्मानमजितात्मापारे
ऽन्धे तमसि प्रहिणोति
कदाचिदीश्वरस्य भगवतो विष्णोश्चक्रात्परमाण्वादिद्विपरार्धापवर्ग
कालोपलक्षणात्परिवर्तितेन वयसा रंहसा हरत आब्रह्मतृणस्तम्बादीनां भूतानामनिमिषतो मिषतां
वित्रस्तहृदयस्तमेवेश्वरं कालचक्रनिजायुधं साक्षाद्भगवन्तं यज्ञपुरुषमनादृत्य
पाखण्डदेवताः कङ्कगृध्रबकवटप्राया आर्यसमयपरिहृताः साङ्केत्येनाभिधत्ते
यदा पाखण्डिभिरात्मवञ्चितैस्तैरुरु वञ्चितो ब्रह्मकुलं समावसंस्तेषां
शीलमुपनयनादिश्रौतस्मार्तकर्मानुष्ठानेन भगवतो यज्ञपुरुषस्याराधनमेव
तदरोचयन्शूद्रकुलं भजते निगमाचारेऽशुद्धितो यस्य मिथुनीभावः कुटुम्बभरणं यथा
वानरजातेः
तत्रापि निरवरोधः स्वैरेण विहरन्नतिकृपणबुद्धिरन्योन्यमुखनिरीक्षणादिना ग्राम्य
कर्मणैव विस्मृतकालावधिः
क्वचिद्द्रुमवदैहिकार्थेषु गृहेषु रंस्यन्यथा वानरः सुतदारवत्सलो व्यवायक्षणः
एवमध्वन्यवरुन्धानो मृत्युगजभयात्तमसि गिरिकन्दरप्राये
क्वचिच्छीतवाताद्यनेकदैविकभौतिकात्मीयानां दुःखानां प्रतिनिवारणेऽकल्पो दुरन्तविषय
विषण्ण आस्ते
क्वचिन्मिथो व्यवहरन्यत्किञ्चिद्धनमुपयाति वित्तशाठ्येन
क्वचित्क्षीणधनः शय्यासनाशनाद्युपभोगविहीनो यावदप्रतिलब्धमनोरथोपगतादाने
ऽवसितमतिस्ततस्ततोऽवमानादीनि जनादभिलभते
एवं वित्तव्यतिषङ्गविवृद्धवैरानुबन्धोऽपि पूर्ववासनया मिथ उद्वहत्यथापवहति
एतस्मिन्संसाराध्वनि नानाक्लेशोपसर्गबाधित आपन्नविपन्नो यत्र यस्तमु ह
वावेतरस्तत्र विसृज्य जातं जातमुपादाय शोचन्मुह्यन्बिभ्यद्‌
विवदन्क्रन्दन्संहृष्यन्गायन्नह्यमानः साधुवर्जितो नैवावर्ततेऽद्यापि यत आरब्ध एष नरलोक
सार्थो यमध्वनः पारमुपदिशन्ति
यदिदं योगानुशासनं न वा एतदवरुन्धते यन्न्यस्तदण्डा मुनय उपशमशीला
उपरतात्मानः समवगच्छन्ति
यदपि दिगिभजयिनो यज्विनो ये वै राजर्षयः किं तु परं मृधे शयीरन्नस्यामेव
ममेयमिति कृतवैरानुबन्धायां विसृज्य स्वयमुपसंहृताः
कर्मवल्लीमवलम्ब्य तत आपदः कथञ्चिन्नरकाद्विमुक्तः पुनरप्येवं संसाराध्वनि
वर्तमानो नरलोकसार्थमुपयाति एवमुपरि गतोऽपि
तस्येदमुपगायन्ति
आर्षभस्येह राजर्षेर्मनसापि महात्मनः
नानुवर्त्मार्हति नृपो मक्षिकेव गरुत्मतः
यो दुस्त्यजान्दारसुतान्सुहृद्राज्यं हृदिस्पृशः
जहौ युवैव मलवदुत्तमश्लोकलालसः
यो दुस्त्यजान्क्षितिसुतस्वजनार्थदारान्‌
प्रार्थ्यां श्रियं सुरवरैः सदयावलोकाम्‌
नैच्छन्नृपस्तदुचितं महतां मधुद्विट्‌
सेवानुरक्तमनसामभवोऽपि फल्गुः
यज्ञाय धर्मपतये विधिनैपुणाय
योगाय साङ्ख्यशिरसे प्रकृतीश्वराय
नारायणाय हरये नम इत्युदारं
हास्यन्मृगत्वमपि यः समुदाजहार
य इदं भागवतसभाजितावदातगुणकर्मणो राजर्षेर्भरतस्यानुचरितं स्वस्त्य्‌
अयनमायुष्यं धन्यं यशस्यं स्वर्ग्यापवर्ग्यं वानुशृणोत्याख्यास्यत्यभिनन्दति च सर्वा एवाशिष
आत्मन आशास्ते न काञ्चन परत इति
श्रीशुक उवाच
भरतस्यात्मजः सुमतिर्नामाभिहितो यमु ह वाव केचित्पाखण्डिन ऋषभ
पदवीमनुवर्तमानं चानार्या अवेदसमाम्नातां देवतां स्वमनीषया पापीयस्या कलौ कल्पयिष्यन्ति
तस्माद्वृद्धसेनायां देवताजिन्नाम पुत्रोऽभवत्‌
अथासुर्यां तत्तनयो देवद्युम्नस्ततो धेनुमत्यां सुतः परमेष्ठी तस्य सुवर्चलायां प्रतीह
उपजातः
य आत्मविद्यामाख्याय स्वयं संशुद्धो महापुरुषमनुसस्मार
प्रतीहात्सुवर्चलायां प्रतिहर्त्रादयस्त्रय आसन्निज्याकोविदाः सूनवः प्रतिहर्तुः स्तुत्यामज
भूमानावजनिषाताम्‌
भूम्न ऋषिकुल्यायामुद्गीथस्ततः प्रस्तावो देवकुल्यायां प्रस्तावान्नियुत्सायां हृदयज
आसीद्विभुर्विभो रत्यां च पृथुषेणस्तस्मान्नक्त आकूत्यां जज्ञे नक्ताद्द्रुतिपुत्रो गयो राजर्षिप्रवर
उदारश्रवा अजायत साक्षाद्भगवतो विष्णोर्जगद्रिरक्षिषया गृहीतसत्त्वस्य कलात्मवत्त्वादिलक्षणेन
महापुरुषतां प्राप्तः
स वै स्वधर्मेण प्रजापालनपोषणप्रीणनोपलालनानुशासनलक्षणेनेज्यादिना च भगवति
महापुरुषे परावरे ब्रह्मणि सर्वात्मनार्पितपरमार्थलक्षणेन ब्रह्मविच्चरणानुसेवयापादित
भगवद्भक्तियोगेन चाभीक्ष्णशः परिभावितातिशुद्धमतिरुपरतानात्म्य आत्मनि
स्वयमुपलभ्यमानब्रह्मात्मानुभवोऽपि निरभिमान एवावनिमजूगुपत्‌
तस्येमां गाथां पाण्डवेय पुराविद उपगायन्ति
गयं नृपः कः प्रतियाति कर्मभिर्यज्वाभिमानी बहुविद्धर्मगोप्ता
समागतश्रीः सदसस्पतिः सतां सत्सेवकोऽन्यो भगवत्कलामृते
यमभ्यषिञ्चन्परया मुदा सतीः सत्याशिषो दक्षकन्याः सरिद्भिः
यस्य प्रजानां दुदुहे धराशिषो निराशिषो गुणवत्सस्नुतोधाः
छन्दांस्यकामस्य च यस्य कामान्दुदूहुराजह्रुरथो बलिं नृपाः
प्रत्यञ्चिता युधि धर्मेण विप्रा यदाशिषां षष्ठमंशं परेत्य
यस्याध्वरे भगवानध्वरात्मा मघोनि माद्यत्युरुसोमपीथे
श्रद्धाविशुद्धाचलभक्तियोग समर्पितेज्याफलमाजहार
यत्प्रीणनाद्बर्हिषि देवतिर्यङ्‌ मनुष्यवीरुत्तृणमाविरिञ्चात्‌
प्रीयेत सद्यः स ह विश्वजीवः प्रीतः स्वयं प्रीतिमगाद्गयस्य
गयाद्गयन्त्यां चित्ररथः सुगतिरवरोधन इति त्रयः पुत्रा बभूवुश्चित्ररथादूर्णायां
सम्राडजनिष्ट तत उत्कलायां मरीचिर्मरीचेर्बिन्दुमत्यां बिन्दुमानुदपद्यत तस्मात्सरघायां
मधुर्नामाभवन्मधोः सुमनसि वीरव्रतस्ततो भोजायां मन्थुप्रमन्थू जज्ञाते मन्थोः
सत्यायां भौवनस्ततो दूषणायां त्वष्टाजनिष्ट त्वष्टुर्विरोचनायां विरजो विरजस्य शतजित्प्रवरं पुत्र
शतं कन्या च विषूच्यां किल जातम्‌
तत्रायं श्लोकः
प्रैयव्रतं वंशमिमं विरजश्चरमोद्भवः
अकरोदत्यलं कीर्त्या विष्णुः सुरगणं यथा
राजोवाच
उक्तस्त्वया भूमण्डलायामविशेषो यावदादित्यस्तपति यत्र चासौ ज्योतिषां गणैश्चन्द्रमा वा
सह दृश्यते
तत्रापि प्रियव्रतरथचरणपरिखातैः सप्तभिः सप्त सिन्धव उपकॢप्ता यत एतस्याः सप्त
द्वीपविशेषविकल्पस्त्वया भगवन्खलु सूचित एतदेवाखिलमहं मानतो लक्षणतश्च सर्वं वि
जिज्ञासामि
भगवतो गुणमये स्थूलरूप आवेशितं मनो ह्यगुणेऽपि सूक्ष्मतम आत्मज्योतिषि परे ब्रह्मणि
भगवति वासुदेवाख्ये क्षममावेशितुं तदु हैतद्गुरोऽर्हस्यनुवर्णयितुमिति
ऋषिरुवाच
न वै महाराज भगवतो मायागुणविभूतेः काष्ठां मनसा वचसा वाधिगन्तुमलं
विबुधायुषापि पुरुषस्तस्मात्प्राधान्येनैव भूगोलकविशेषं नामरूपमानलक्षणतो
व्याख्यास्यामः
यो वायं द्वीपः कुवलयकमलकोशाभ्यन्तरकोशो नियुतयोजनविशालः समवर्तुलो यथा
पुष्करपत्रम्‌
यस्मिन्नव वर्षाणि नवयोजनसहस्रायामान्यष्टभिर्मर्यादागिरिभिः सुविभक्तानि भवन्ति
एषां मध्ये इलावृतं नामाभ्यन्तरवर्षं यस्य नाभ्यामवस्थितः सर्वतः सौवर्णः कुल
गिरिराजो मेरुर्द्वीपायामसमुन्नाहः कर्णिकाभूतः कुवलयकमलस्य मूर्धनि द्वात्रिंशत्सहस्र
योजनविततो मूले षोडशसहस्रं तावतान्तर्भूम्यां प्रविष्टः
उत्तरोत्तरेणेलावृतं नीलः श्वेतः शृङ्गवानिति त्रयो रम्यकहिरण्मयकुरूणां वर्षाणां मर्यादा
गिरयः प्रागायता उभयतः क्षारोदावधयो द्विसहस्रपृथव एकैकशः पूर्वस्मात्पूर्वस्मादुत्तर
उत्तरो दशांशाधिकांशेन दैर्घ्य एव ह्रसन्ति
एवं दक्षिणेनेलावृतं निषधो हेमकूटो हिमालय इति प्रागायता यथा नीलादयोऽयुतयोजनोत्सेधा
हरिवर्षकिम्पुरुषभारतानां यथासङ्ख्यम्‌
तथैवेलावृतमपरेण पूर्वेण च माल्यवद्गन्धमादनावानीलनिषधायतौ द्विसहस्रं
पप्रथतुः केतुमालभद्राश्वयोः सीमानं विदधाते
मन्दरो मेरुमन्दरः सुपार्श्वः कुमुद इत्ययुतयोजनविस्तारोन्नाहा मेरोश्चतुर्‌
दिशमवष्टम्भगिरय उपकॢप्ताः
चतुर्ष्वेतेषु चूतजम्बूकदम्बन्यग्रोधाश्चत्वारः पादपप्रवराः पर्वतकेतव इवाधि
सहस्रयोजनोन्नाहास्तावद्विटपविततयः शतयोजनपरिणाहाः
ह्रदाश्चत्वारः पयोमध्विक्षुरसमृष्टजला यदुपस्पर्शिन उपदेवगणा योगैश्वर्याणि
स्वाभाविकानि भरतर्षभ धारयन्ति
देवोद्यानानि च भवन्ति चत्वारि नन्दनं चैत्ररथं वैभ्राजकं सर्वतोभद्रमिति
येष्वमरपरिवृढाः सह सुरललनाललामयूथपतय उपदेवगणैरुपगीयमान
महिमानः किल विहरन्ति
मन्दरोत्सङ्ग एकादशशतयोजनोत्तुङ्गदेवचूतशिरसो गिरिशिखरस्थूलानि फलान्यमृत
कल्पानि पतन्ति
तेषां विशीर्यमाणानामतिमधुरसुरभिसुगन्धिबहुलारुणरसोदेनारुणोदा नाम नदी
मन्दरगिरिशिखरान्निपतन्ती पूर्वेणेलावृतमुपप्लावयति
यदुपजोषणाद्भवान्या अनुचरीणां पुण्यजनवधूनामवयवस्पर्शसुगन्धवातो दश
योजनं समन्तादनुवासयति
एवं जम्बूफलानामत्युच्चनिपातविशीर्णानामनस्थिप्रायाणामिभकायनिभानां रसेन
जम्बू नाम नदी मेरुमन्दरशिखरादयुतयोजनादवनितले निपतन्ती दक्षिणेनात्मानं
यावदिलावृतमुपस्यन्दयति
तावदुभयोरपि रोधसोर्या मृत्तिका तद्रसेनानुविध्यमाना वाय्वर्कसंयोगविपाकेन
सदामरलोकाभरणं जाम्बूनदं नाम सुवर्णं भवति
यदु ह वाव विबुधादयः सह युवतिभिर्मुकुटकटककटिसूत्राद्याभरणरूपेण खलु
धारयन्ति
यस्तु महाकदम्बः सुपार्श्वनिरूढो यास्तस्य कोटरेभ्यो विनिःसृताः पञ्चायामपरिणाहाः
पञ्च मधुधाराः सुपार्श्वशिखरात्पतन्त्योऽपरेणात्मानमिलावृतमनुमोदयन्ति
या ह्युपयुञ्जानानां मुखनिर्वासितो वायुः समन्ताच्छतयोजनमनुवासयति
एवं कुमुदनिरूढो यः शतवल्शो नाम वटस्तस्य स्कन्धेभ्यो नीचीनाः पयोदधिमधुघृत
गुडान्नाद्यम्बरशय्यासनाभरणादयः सर्व एव कामदुघा नदाः
कुमुदाग्रात्पतन्तस्तमुत्तरेणेलावृतमुपयोजयन्ति
यानुपजुषाणानां न कदाचिदपि प्रजानां वलीपलितक्लमस्वेददौर्गन्ध्यजरामयमृत्यु
शीतोष्णवैवर्ण्योपसर्गादयस्तापविशेषा भवन्ति यावज्जीवं सुखं निरतिशयमेव
कुरङ्गकुररकुसुम्भवैकङ्कत्रिकूटशिशिरपतङ्गरुचकनिषधशिनीवासकपिलशङ्ख
वैदूर्यजारुधिहंसऋषभनागकालञ्जरनारदादयो विंशतिगिरयो मेरोः कर्णिकाया इव केसरभूता
मूलदेशे परित उपकॢप्ताः
जठरदेवकूटौ मेरुं पूर्वेणाष्टादशयोजनसहस्रमुदगायतौ द्विसहस्रं पृथुतुङ्गौ
भवतः एवमपरेण पवनपारियात्रौ दक्षिणेन कैलासकरवीरौ प्रागायतावेवमुत्तरतस्त्रिशृङ्ग
मकरावष्टभिरेतैः परिसृतोऽग्निरिव परितश्चकास्ति काञ्चनगिरिः
मेरोर्मूर्धनि भगवत आत्मयोनेर्मध्यत उपकॢप्तां पुरीमयुतयोजनसाहस्रीं सम
चतुरस्रां शातकौम्भीं वदन्ति
तामनुपरितो लोकपालानामष्टानां यथादिशं यथारूपं तुरीयमानेन पुरोऽष्टावुपकॢप्ताः
श्रीशुक उवाच
तत्र भगवतः साक्षाद्यज्ञलिङ्गस्य विष्णोर्विक्रमतो वामपादाङ्गुष्ठनख
निर्भिन्नोर्ध्वाण्डकटाहविवरेणान्तःप्रविष्टा या बाह्यजलधारा तच्चरणपङ्कजावनेजनारुण
किञ्जल्कोपरञ्जिताखिलजगदघमलापहोपस्पर्शनामला साक्षाद्भगवत्पदीत्यनुपलक्षितवचो
ऽभिधीयमानातिमहता कालेन युगसहस्रोपलक्षणेन दिवो मूर्धन्यवततार यत्तद्विष्णुपदमाहुः
यत्र ह वाव वीरव्रत औत्तानपादिः परमभागवतोऽस्मत्कुलदेवताचरणारविन्दोदकमिति
यामनुसवनमुत्कृष्यमाणभगवद्भक्तियोगेन दृढं क्लिद्यमानान्तर्हृदय औत्कण्ठ्य
विवशामीलितलोचनयुगलकुड्मलविगलितामलबाष्पकलयाभिव्यज्यमानरोमपुलककुलकोऽधुनापि
परमादरेण शिरसा बिभर्ति
ततः सप्त ऋषयस्तत्प्रभावाभिज्ञा यां ननु तपस आत्यन्तिकी सिद्धिरेतावती भगवति
सर्वात्मनि वासुदेवेऽनुपरतभक्तियोगलाभेनैवोपेक्षितान्यार्थात्मगतयो मुक्तिमिवागतां मुमुक्षव
इव सबहुमानमद्यापि जटाजूटैरुद्वहन्ति
ततोऽनेकसहस्रकोटिविमानानीकसङ्कुलदेवयानेनावतरन्तीन्दु मण्डलमावार्य ब्रह्म
सदने निपतति
तत्र चतुर्धा भिद्यमाना चतुर्भिर्नामभिश्चतुर्दिशमभिस्पन्दन्ती नदनदी
पतिमेवाभिनिविशति सीतालकनन्दा चक्षुर्भद्रेति
सीता तु ब्रह्मसदनात्केसराचलादिगिरिशिखरेभ्योऽधोऽधः प्रस्रवन्ती गन्धमादनमूर्धसु
पतित्वान्तरेण भद्राश्ववर्षं प्राच्यां दिशि क्षारसमुद्रमभिप्रविशति
एवं माल्यवच्छिखरान्निष्पतन्ती ततोऽनुपरतवेगा केतुमालमभि चक्षुः प्रतीच्यां दिशि सरित्‌
पतिं प्रविशति
भद्रा चोत्तरतो मेरुशिरसो निपतिता गिरिशिखराद्गिरिशिखरमतिहाय शृङ्गवतः
शृङ्गादवस्यन्दमाना उत्तरांस्तु कुरूनभित उदीच्यां दिशि जलधिमभिप्रविशति
तथैवालकनन्दा दक्षिणेन ब्रह्मसदनाद्बहूनि गिरिकूटान्यतिक्रम्य
हेमकूटाद्धैमकूटान्यतिरभसतररंहसा लुठयन्ती भारतमभिवर्षं दक्षिणस्यां दिशि
जलधिमभिप्रविशति यस्यां स्नानार्थं चागच्छतः पुंसः पदे पदेऽश्वमेधराजसूयादीनां फलं
न दुर्लभमिति
अन्ये च नदा नद्यश्च वर्षे वर्षे सन्ति बहुशो मेर्वादिगिरिदुहितरः शतशः
तत्रापि भारतमेव वर्षं कर्मक्षेत्रमन्यान्यष्ट वर्षाणि स्वर्गिणां पुण्यशेषोपभोग
स्थानानि भौमानि स्वर्गपदानि व्यपदिशन्ति
एषु पुरुषाणामयुतपुरुषायुर्वर्षाणां देवकल्पानां नागायुतप्राणानां वज्रसंहननबल
वयोमोदप्रमुदितमहासौरतमिथुनव्यवायापवर्गवर्षधृतैकगर्भकलत्राणां तत्र तु त्रेता
युगसमः कालो वर्तते
यत्र ह देवपतयः स्वैः स्वैर्गणनायकैर्विहितमहार्हणाः सर्वर्तुकुसुमस्तबकफल
किसलयश्रियानम्यमानविटपलताविटपिभिरुपशुम्भमानरुचिरकाननाश्रमायतनवर्षगिरिद्रोणीषु
तथा चामलजलाशयेषु विकचविविधनववनरुहामोदमुदितराजहंसजलकुक्कुटकारण्डवसारस
चक्रवाकादिभिर्मधुकरनिकराकृतिभिरुपकूजितेषु जलक्रीडादिभिर्विचित्रविनोदैः सुललितसुरसुन्दरीणां
कामकलिलविलासहासलीलावलोकाकृष्टमनोदृष्टयः स्वैरं विहरन्ति
नवस्वपि वर्षेषु भगवान्नारायणो महापुरुषः पुरुषाणां तदनुग्रहायात्मतत्त्व
व्यूहेनात्मनाद्यापि सन्निधीयते
इलावृते तु भगवान्भव एक एव पुमान्न ह्यन्यस्तत्रापरो निर्विशति भवान्याः शापनिमित्तज्ञो
यत्प्रवेक्ष्यतः स्त्रीभावस्तत्पश्चाद्वक्ष्यामि
भवानीनाथैः स्त्रीगणार्बुदसहस्रैरवरुध्यमानो भगवतश्चतुर्मूर्तेर्महापुरुषस्य
तुरीयां तामसीं मूर्तिं प्रकृतिमात्मनः सङ्कर्षणसंज्ञामात्मसमाधिरूपेण
सन्निधाप्यैतदभिगृणन्भव उपधावति
श्रीभगवानुवाच
ओं नमो भगवते महापुरुषाय सर्वगुणसङ्ख्यानायानन्तायाव्यक्ताय नम इति
भजे भजन्यारणपादपङ्कजं भगस्य कृत्स्नस्य परं परायणम्‌
भक्तेष्वलं भावितभूतभावनं भवापहं त्वा भवभावमीश्वरम्‌
न यस्य मायागुणचित्तवृत्तिभिर्निरीक्षतो ह्यण्वपि दृष्टिरज्यते
ईशे यथा नोऽजितमन्युरंहसां कस्तं न मन्येत जिगीषुरात्मनः
असद्दृशो यः प्रतिभाति मायया क्षीबेव मध्वासवताम्रलोचनः
न नागवध्वोऽर्हण ईशिरे ह्रिया यत्पादयोः स्पर्शनधर्षितेन्द्रियाः
यमाहुरस्य स्थितिजन्मसंयमं त्रिभिर्विहीनं यमनन्तमृषयः
न वेद सिद्धार्थमिव क्वचित्स्थितं भूमण्डलं मूर्धसहस्रधामसु
यस्याद्य आसीद्गुणविग्रहो महान्विज्ञानधिष्ण्यो भगवानजः किल
यत्सम्भवोऽहं त्रिवृता स्वतेजसा वैकारिकं तामसमैन्द्रियं सृजे
एते वयं यस्य वशे महात्मनः स्थिताः शकुन्ता इव सूत्रयन्त्रिताः
महानहं वैकृततामसेन्द्रियाः सृजाम सर्वे यदनुग्रहादिदम्‌
यन्निर्मितां कर्ह्यपि कर्मपर्वणीं मायां जनोऽयं गुणसर्गमोहितः
न वेद निस्तारणयोगमञ्जसा तस्मै नमस्ते विलयोदयात्मने
श्रीशुक उवाच
तथा च भद्रश्रवा नाम धर्मसुतस्तत्कुलपतयः पुरुषा भद्राश्ववर्षे साक्षाद्भगवतो
वासुदेवस्य प्रियां तनुं धर्ममयीं हयशीर्षाभिधानां परमेण समाधिना
सन्निधाप्येदमभिगृणन्त उपधावन्ति
भद्रश्रवस ऊचुः
ओं नमो भगवते धर्मायात्मविशोधनाय नम इति
अहो विचित्रं भगवद्विचेष्टितं घ्नन्तं जनोऽयं हि मिषन्न पश्यति
ध्यायन्नसद्यर्हि विकर्म सेवितुं निर्हृत्य पुत्रं पितरं जिजीविषति
वदन्ति विश्वं कवयः स्म नश्वरं पश्यन्ति चाध्यात्मविदो विपश्चितः
तथापि मुह्यन्ति तवाज मायया सुविस्मितं कृत्यमजं नतोऽस्मि तम्‌
विश्वोद्भवस्थाननिरोधकर्म ते ह्यकर्तुरङ्गीकृतमप्यपावृतः
युक्तं न चित्रं त्वयि कार्यकारणे सर्वात्मनि व्यतिरिक्ते च वस्तुतः
वेदान्युगान्ते तमसा तिरस्कृतान्रसातलाद्यो नृतुरङ्गविग्रहः
प्रत्याददे वै कवयेऽभियाचते तस्मै नमस्तेऽवितथेहिताय इति
हरिवर्षे चापि भगवान्नरहरिरूपेणास्ते तद्रूपग्रहणनिमित्तमुत्तरत्राभिधास्ये
तद्दयितं रूपं महापुरुषगुणभाजनो महाभागवतो दैत्यदानवकुलतीर्थीकरणशीलाचरितः
प्रह्लादोऽव्यवधानानन्यभक्तियोगेन सह तद्वर्षपुरुषैरुपास्ते इदं चोदाहरति
ओं नमो भगवते नरसिंहाय नमस्तेजस्तेजसे आविराविर्भव वज्रनख वज्रदंष्ट्र
कर्माशयान्रन्धय रन्धय तमो ग्रस ग्रस ओं स्वाहा अभयमभयमात्मनि भूयिष्ठा ओं क्ष्रौम्‌
स्वस्त्यस्तु विश्वस्य खलः प्रसीदतां ध्यायन्तु भूतानि शिवं मिथो धिया
मनश्च भद्रं भजतादधोक्षजे आवेश्यतां नो मतिरप्यहैतुकी
मागारदारात्मजवित्तबन्धुषु सङ्गो यदि स्याद्भगवत्प्रियेषु नः
यः प्राणवृत्त्या परितुष्ट आत्मवान्सिद्ध्यत्यदूरान्न तथेन्द्रियप्रियः
यत्सङ्गलब्धं निजवीर्यवैभवं तीर्थं मुहुः संस्पृशतां हि मानसम्‌
हरत्यजोऽन्तः श्रुतिभिर्गतोऽङ्गजं को वै न सेवेत मुकुन्दविक्रमम्‌
यस्यास्ति भक्तिर्भगवत्यकिञ्चना सर्वैर्गुणैस्तत्र समासते सुराः
हरावभक्तस्य कुतो महद्गुणा मनोरथेनासति धावतो बहिः
हरिर्हि साक्षाद्भगवान्शरीरिणामात्मा झषाणामिव तोयमीप्सितम्‌
हित्वा महांस्तं यदि सज्जते गृहे तदा महत्त्वं वयसा दम्पतीनाम्‌
तस्माद्रजोरागविषादमन्यु मानस्पृहाभयदैन्याधिमूलम्‌
हित्वा गृहं संसृतिचक्रवालं नृसिंहपादं भजताकुतोभयमिति
केतुमालेऽपि भगवान्कामदेवस्वरूपेण लक्ष्म्याः प्रियचिकीर्षया प्रजापतेर्दुहित्णां पुत्राणां
तद्वर्षपतीनां पुरुषायुषाहोरात्रपरिसङ्ख्यानानां यासां गर्भा महापुरुषमहास्त्रतेजसोद्वेजित
मनसां विध्वस्ता व्यसवः संवत्सरान्ते विनिपतन्ति
अतीव सुललितगतिविलासविलसितरुचिरहासलेशावलोकलीलया
किञ्चिदुत्तम्भितसुन्दरभ्रूमण्डल
सुभगवदनारविन्दश्रिया रमां रमयन्निन्द्रियाणि रमयते
तद्भगवतो मायामयं रूपं परमसमाधियोगेन रमा देवी संवत्सरस्य रात्रिषु
प्रजापतेर्दुहितृभिरुपेताहःसु च तद्भर्तृभिरुपास्ते इदं चोदाहरति
ओं ह्रां ह्रीं ह्रूं ओं नमो भगवते हृषीकेशाय सर्वगुणविशेषैर्विलक्षितात्मने आकूतीनां
चित्तीनां चेतसां विशेषाणां चाधिपतये षोडशकलाय च्छन्दोमयायान्नमयायामृतमयाय सर्वमयाय
सहसे ओजसे बलाय कान्ताय कामाय नमस्ते उभयत्र भूयात्‌
स्त्रियो व्रतैस्त्वा हृषीकेश्वरं स्वतो ह्याराध्य लोके पतिमाशासतेऽन्यम्‌
तासां न ते वै परिपान्त्यपत्यं प्रियं धनायूंषि यतोऽस्वतन्त्राः
स वै पतिः स्यादकुतोभयः स्वयं समन्ततः पाति भयातुरं जनम्‌
स एक एवेतरथा मिथो भयं नैवात्मलाभादधि मन्यते परम्‌
या तस्य ते पादसरोरुहार्हणं निकामयेत्साखिलकामलम्पटा
तदेव रासीप्सितमीप्सितोऽर्चितो यद्भग्नयाच्ञा भगवन्प्रतप्यते
मत्प्राप्तयेऽजेशसुरासुरादयस्तप्यन्त उग्रं तप ऐन्द्रिये धियः
ऋते भवत्पादपरायणान्न मां विन्दन्त्यहं त्वद्धृदया यतोऽजित
स त्वं ममाप्यच्युत शीर्ष्णि वन्दितं कराम्बुजं यत्त्वदधायि सात्वताम्‌
बिभर्षि मां लक्ष्म वरेण्य मायया क ईश्वरस्येहितमूहितुं विभुरिति
रम्यके च भगवतः प्रियतमं मात्स्यमवताररूपं तद्वर्षपुरुषस्य मनोः प्राक्‌
प्रदर्शितं स इदानीमपि महता भक्तियोगेनाराधयतीदं चोदाहरति
ओं नमो भगवते मुख्यतमाय नमः सत्त्वाय प्राणायौजसे सहसे बलाय महामत्स्याय
नम इति
अन्तर्बहिश्चाखिललोकपालकैरदृष्टरूपो विचरस्युरुस्वनः
स ईश्वरस्त्वं य इदं वशेऽनयन्नाम्ना यथा दारुमयीं नरः स्त्रियम्‌
यं लोकपालाः किल मत्सरज्वरा हित्वा यतन्तोऽपि पृथक्समेत्य च
पातुं न शेकुर्द्विपदश्चतुष्पदः सरीसृपं स्थाणु यदत्र दृश्यते
भवान्युगान्तार्णव ऊर्मिमालिनि क्षोणीमिमामोषधिवीरुधां निधिम्‌
मया सहोरु क्रमतेऽज ओजसा तस्मै जगत्प्राणगणात्मने नम इति
हिरण्मयेऽपि भगवान्निवसति कूर्मतनुं बिभ्राणस्तस्य तत्प्रियतमां तनुमर्यमा सह
वर्षपुरुषैः पितृगणाधिपतिरुपधावति मन्त्रमिमं चानुजपति
ओं नमो भगवते अकूपाराय सर्वसत्त्वगुणविशेषणायानुपलक्षितस्थानाय नमो वर्ष्मणे
नमो भूम्ने नमो नमोऽवस्थानाय नमस्ते
यद्रूपमेतन्निजमाययार्पितमर्थस्वरूपं बहुरूपरूपितम्‌
सङ्ख्या न यस्यास्त्ययथोपलम्भनात्तस्मै नमस्तेऽव्यपदेशरूपिणे
जरायुजं स्वेदजमण्डजोद्भिदं चराचरं देवर्षिपितृभूतमैन्द्रियम्‌
द्यौः खं क्षितिः शैलसरित्समुद्र द्वीपग्रहर्क्षेत्यभिधेय एकः
यस्मिन्नसङ्ख्येयविशेषनाम रूपाकृतौ कविभिः कल्पितेयम्‌
सङ्ख्या यया तत्त्वदृशापनीयते तस्मै नमः साङ्ख्यनिदर्शनाय ते इति
उत्तरेषु च कुरुषु भगवान्यज्ञपुरुषः कृतवराहरूप आस्ते तं तु देवी हैषा भूः सह
कुरुभिरस्खलितभक्तियोगेनोपधावति इमां च परमामुपनिषदमावर्तयति
ओं नमो भगवते मन्त्रतत्त्वलिङ्गाय यज्ञक्रतवे महाध्वरावयवाय महापुरुषाय नमः
कर्मशुक्लाय त्रियुगाय नमस्ते
यस्य स्वरूपं कवयो विपश्चितो गुणेषु दारुष्विव जातवेदसम्‌
मथ्नन्ति मथ्ना मनसा दिदृक्षवो गूढं क्रियार्थैर्नम ईरितात्मने
द्रव्यक्रियाहेत्वयनेशकर्तृभिर्मायागुणैर्वस्तुनिरीक्षितात्मने
अन्वीक्षयाङ्गातिशयात्मबुद्धिभिर्निरस्तमायाकृतये नमो नमः
करोति विश्वस्थितिसंयमोदयं यस्येप्सितं नेप्सितमीक्षितुर्गुणैः
माया यथायो भ्रमते तदाश्रयं ग्राव्णो नमस्ते गुणकर्मसाक्षिणे
प्रमथ्य दैत्यं प्रतिवारणं मृधे यो मां रसाया जगदादिसूकरः
कृत्वाग्रदंष्ट्रे निरगादुदन्वतः क्रीडन्निवेभः प्रणतास्मि तं विभुमिति
श्रीशुक उवाच
किम्पुरुषे वर्षे भगवन्तमादिपुरुषं लक्ष्मणाग्रजं सीताभिरामं रामं तच्चरण
सन्निकर्षाभिरतः परमभागवतो हनुमान्सह किम्पुरुषैरविरतभक्तिरुपास्ते
आर्ष्टिषेणेन सह गन्धर्वैरनुगीयमानां परमकल्याणीं भर्तृभगवत्कथां समुपशृणोति
स्वयं चेदं गायति
ओं नमो भगवते उत्तमश्लोकाय नम आर्यलक्षणशीलव्रताय नम उपशिक्षितात्मन उपासित
लोकाय नमः साधुवादनिकषणाय नमो ब्रह्मण्यदेवाय महापुरुषाय महाराजाय नम इति
यत्तद्विशुद्धानुभवमात्रमेकं स्वतेजसा ध्वस्तगुणव्यवस्थम्‌
प्रत्यक्प्रशान्तं सुधियोपलम्भनं ह्यनामरूपं निरहं प्रपद्ये
मर्त्यावतारस्त्विह मर्त्यशिक्षणं रक्षोवधायैव न केवलं विभोः
कुतोऽन्यथा स्याद्रमतः स्व आत्मनः सीताकृतानि व्यसनानीश्वरस्य
न वै स आत्मात्मवतां सुहृत्तमः सक्तस्त्रिलोक्यां भगवान्वासुदेवः
न स्त्रीकृतं कश्मलमश्नुवीत न लक्ष्मणं चापि विहातुमर्हति
न जन्म नूनं महतो न सौभगं न वाङ्न बुद्धिर्नाकृतिस्तोषहेतुः
तैर्यद्विसृष्टानपि नो वनौकसश्चकार सख्ये बत लक्ष्मणाग्रजः
सुरोऽसुरो वाप्यथ वानरो नरः सर्वात्मना यः सुकृतज्ञमुत्तमम्‌
भजेत रामं मनुजाकृतिं हरिं य उत्तराननयत्कोसलान्दिवमिति
भारतेऽपि वर्षे भगवान्नरनारायणाख्य आकल्पान्तमुपचितधर्मज्ञान
वैराग्यैश्वर्योपशमोपरमात्मोपलम्भनमनुग्रहायात्मवतामनुकम्पया तपोऽव्यक्तगतिश्चरति
तं भगवान्नारदो वर्णाश्रमवतीभिर्भारतीभिः प्रजाभिर्भगवत्प्रोक्ताभ्यां साङ्ख्य
योगाभ्यां भगवदनुभावोपवर्णनं सावर्णेरुपदेक्ष्यमाणः परमभक्तिभावेनोपसरति इदं
चाभिगृणाति
ओं नमो भगवते उपशमशीलायोपरतानात्म्याय नमोऽकिञ्चनवित्ताय ऋषिऋषभाय नर
नारायणाय परमहंसपरमगुरवे आत्मारामाधिपतये नमो नम इति
गायति चेदम्‌
कर्तास्य सर्गादिषु यो न बध्यते न हन्यते देहगतोऽपि दैहिकैः
द्रष्टुर्न दृग्यस्य गुणैर्विदूष्यते तस्मै नमोऽसक्तविविक्तसाक्षिणे
इदं हि योगेश्वर योगनैपुणं हिरण्यगर्भो भगवाञ्जगाद यत्‌
यदन्तकाले त्वयि निर्गुणे मनो भक्त्या दधीतोज्झितदुष्कलेवरः
यथैहिकामुष्मिककामलम्पटः सुतेषु दारेषु धनेषु चिन्तयन्‌
शङ्केत विद्वान्कुकलेवरात्ययाद्यस्तस्य यत्नः श्रम एव केवलम्‌
तन्नः प्रभो त्वं कुकलेवरार्पितां त्वन्माययाहंममतामधोक्षज
भिन्द्याम येनाशु वयं सुदुर्भिदां विधेहि योगं त्वयि नः स्वभावमिति
भारतेऽप्यस्मिन्वर्षे सरिच्छैलाः सन्ति बहवो मलयो मङ्गलप्रस्थो मैनाकस्त्रिकूट ऋषभः
कूटकः कोल्लकः सह्यो देवगिरिरृष्यमूकः श्रीशैलो वेङ्कटो महेन्द्रो वारिधारो विन्ध्यः शुक्तिमानृक्षगिरिः
पारियात्रो द्रोणश्चित्रकूटो गोवर्धनो रैवतकः ककुभो नीलो गोकामुख इन्द्रकीलः कामगिरिरिति चान्ये च
शतसहस्रशः शैलास्तेषां नितम्बप्रभवा नदा नद्यश्च सन्त्यसङ्ख्याताः
एतासामपो भारत्यः प्रजा नामभिरेव पुनन्तीनामात्मना चोपस्पृशन्ति
चन्द्रवसा ताम्रपर्णी अवटोदा कृतमाला वैहायसी कावेरी वेणी पयस्विनी शर्करावर्ता तुङ्गभद्रा
कृष्णावेण्या भीमरथी गोदावरी निर्विन्ध्या पयोष्णी तापी रेवा सुरसा नर्मदा चर्मण्वती सिन्धुरन्धः
शोणश्च नदौ महानदी वेदस्मृतिरृषिकुल्या त्रिसामा कौशिकी मन्दाकिनी यमुना सरस्वती दृषद्वती गोमती
सरयू रोधस्वती सप्तवती सुषोमा शतद्रूश्चन्द्रभागा मरुद्वृधा वितस्ता असिक्नी विश्वेति महानद्यः
अस्मिन्नेव वर्षे पुरुषैर्लब्धजन्मभिः शुक्ललोहितकृष्णवर्णेन स्वारब्धेन कर्मणा दिव्य
मानुषनारकगतयो बह्व्य आत्मन आनुपूर्व्येण सर्वा ह्येव सर्वेषां विधीयन्ते यथावर्ण
विधानमपवर्गश्चापि भवति
योऽसौ भगवति सर्वभूतात्मन्यनात्म्येऽनिरुक्तेऽनिलयने परमात्मनि वासुदेवेऽनन्य
निमित्तभक्तियोगलक्षणो नानागतिनिमित्ताविद्याग्रन्थिरन्धनद्वारेण यदा हि महापुरुषपुरुष
प्रसङ्गः
एतदेव हि देवा गायन्ति
अहो अमीषां किमकारि शोभनं प्रसन्न एषां स्विदुत स्वयं हरिः
यैर्जन्म लब्धं नृषु भारताजिरे मुकुन्दसेवौपयिकं स्पृहा हि नः
किं दुष्करैर्नः क्रतुभिस्तपोव्रतैर्दानादिभिर्वा द्युजयेन फल्गुना
न यत्र नारायणपादपङ्कज स्मृतिः प्रमुष्टातिशयेन्द्रियोत्सवात्‌
कल्पायुषां स्थानजयात्पुनर्भवात्क्षणायुषां भारतभूजयो वरम्‌
क्षणेन मर्त्येन कृतं मनस्विनः सन्न्यस्य संयान्त्यभयं पदं हरेः
न यत्र वैकुण्ठकथासुधापगा न साधवो भागवतास्तदाश्रयाः
न यत्र यज्ञेशमखा महोत्सवाः सुरेशलोकोऽपि न वै स सेव्यताम्‌
प्राप्ता नृजातिं त्विह ये च जन्तवो ज्ञानक्रियाद्रव्यकलापसम्भृताम्‌
न वै यतेरन्नपुनर्भवाय ते भूयो वनौका इव यान्ति बन्धनम्‌
यैः श्रद्धया बर्हिषि भागशो हविर्निरुप्तमिष्टं विधिमन्त्रवस्तुतः
एकः पृथङ्नामभिराहुतो मुदा गृह्णाति पूर्णः स्वयमाशिषां प्रभुः
सत्यं दिशत्यर्थितमर्थितो नृणां नैवार्थदो यत्पुनरर्थिता यतः
स्वयं विधत्ते भजतामनिच्छतामिच्छापिधानं निजपादपल्लवम्‌
यद्यत्र नः स्वर्गसुखावशेषितं स्विष्टस्य सूक्तस्य कृतस्य शोभनम्‌
तेनाजनाभे स्मृतिमज्जन्म नः स्याद्वर्षे हरिर्यद्भजतां शं तनोति
श्रीशुक उवाच
जम्बूद्वीपस्य च राजन्नुपद्वीपानष्टौ हैक उपदिशन्ति सगरात्मजैरश्वान्वेषण इमां महीं
परितो निखनद्भिरुपकल्पितान्‌
तद्यथा स्वर्णप्रस्थश्चन्द्रशुक्ल आवर्तनो रमणको मन्दरहरिणः पाञ्चजन्यः सिंहलो
लङ्केति
एवं तव भारतोत्तम जम्बूद्वीपवर्षविभागो यथोपदेशमुपवर्णित इति
श्रीशुक उवाच
अतः परं प्लक्षादीनां प्रमाणलक्षणसंस्थानतो वर्षविभाग उपवर्ण्यते
जम्बूद्वीपोऽयं यावत्प्रमाणविस्तारस्तावता क्षारोदधिना परिवेष्टितो यथा मेरुर्जम्ब्वाख्येन
लवणोदधिरपि ततो द्विगुणविशालेन प्लक्षाख्येन परिक्षिप्तो यथा परिखा बाह्योपवनेन प्लक्षो जम्बू
प्रमाणो द्वीपाख्याकरो हिरण्मय उत्थितो यत्राग्निरुपास्ते सप्तजिह्वस्तस्याधिपतिः प्रियव्रतात्मज
इध्मजिह्वः स्वं द्वीपं सप्तवर्षाणि विभज्य सप्तवर्षनामभ्य आत्मजेभ्य आकलय्य स्वयमात्म
योगेनोपरराम
शिवं यवसं सुभद्रं शान्तं क्षेमममृतमभयमिति वर्षाणि तेषु गिरयो नद्यश्च
सप्तैवाभिज्ञाताः
मणिकूटो वज्रकूट इन्द्रसेनो ज्योतिष्मान्सुपर्णो हिरण्यष्ठीवो मेघमाल इति सेतुशैलाः अरुणा
नृम्णाङ्गिरसी सावित्री सुप्तभाता ऋतम्भरा सत्यम्भरा इति महानद्यः यासां जलोपस्पर्शनविधूतरजस्‌
तमसो हंसपतङ्गोर्ध्वायनसत्याङ्गसंज्ञाश्चत्वारो वर्णाः सहस्रायुषो विबुधोपमसन्दर्शन
प्रजननाः स्वर्गद्वारं त्रय्या विद्यया भगवन्तं त्रयीमयं सूर्यमात्मानं यजन्ते
प्रत्नस्य विष्णो रूपं यत्सत्यस्यर्तस्य ब्रह्मणः
अमृतस्य च मृत्योश्च सूर्यमात्मानमीमहीति
प्लक्षादिषु पञ्चसु पुरुषाणामायुरिन्द्रियमोजः सहो बलं बुद्धिर्विक्रम इति च
सर्वेषामौत्पत्तिकी सिद्धिरविशेषेण वर्तते
प्लक्षः स्वसमानेनेक्षुरसोदेनावृतो यथा तथा द्वीपोऽपि शाल्मलो द्विगुणविशालः समानेन
सुरोदेनावृतः परिवृङ्क्ते
यत्र ह वै शाल्मली प्लक्षायामा यस्यां वाव किल निलयमाहुर्भगवतश्छन्दःस्तुतः पतत्त्रि
राजस्य सा द्वीपहूतये उपलक्ष्यते
तद्द्वीपाधिपतिः प्रियव्रतात्मजो यज्ञबाहुः स्वसुतेभ्यः सप्तभ्यस्तन्नामानि सप्तवर्षाणि
व्यभजत्सुरोचनं सौमनस्यं रमणकं देववर्षं पारिभद्रमाप्यायनमविज्ञातमिति
तेषु वर्षाद्रयो नद्यश्च सप्तैवाभिज्ञाताः स्वरसः शतशृङ्गो वामदेवः कुन्दो मुकुन्दः पुष्प
वर्षः सहस्रश्रुतिरिति अनुमतिः सिनीवाली सरस्वती कुहू रजनी नन्दा राकेति
तद्वर्षपुरुषाः श्रुतधरवीर्यधरवसुन्धरेषन्धरसंज्ञा भगवन्तं वेदमयं
सोममात्मानं वेदेन यजन्ते
स्वगोभिः पितृदेवेभ्यो विभजन्कृष्णशुक्लयोः
प्रजानां सर्वासां राजा न्धः सोमो न आस्त्विति
एवं सुरोदाद्बहिस्तद्द्विगुणः समानेनावृतो घृतोदेन यथापूर्वः कुशद्वीपो यस्मिन्कुश
स्तम्बो देवकृतस्तद्द्वीपाख्याकरो ज्वलन इवापरः स्वशष्परोचिषा दिशो विराजयति
तद्द्वीपपतिः प्रैयव्रतो राजन्हिरण्यरेता नाम स्वं द्वीपं सप्तभ्यः स्वपुत्रेभ्यो यथा
भागं विभज्य स्वयं तप आतिष्ठत वसुवसुदानदृढरुचिनाभिगुप्तस्तुत्यव्रतविविक्तवामदेव
नामभ्यः
तेषां वर्षेषु सीमागिरयो नद्यश्चाभिज्ञाताः सप्त सप्तैव चक्रश्चतुःशृङ्गः कपिलश्चित्रकूटो
देवानीक ऊर्ध्वरोमा द्रविण इति रसकुल्या मधुकुल्या मित्रविन्दा श्रुतविन्दा देवगर्भा घृतच्युता
मन्त्रमालेति
यासां पयोभिः कुशद्वीपौकसः कुशलकोविदाभियुक्तकुलकसंज्ञा भगवन्तं जातवेद
सरूपिणं कर्मकौशलेन यजन्ते
परस्य ब्रह्मणः साक्षाज्जातवेदोऽसि हव्यवाट्‌
देवानां पुरुषाङ्गानां यज्ञेन पुरुषं यजेति
तथा घृतोदाद्बहिः क्रौञ्चद्वीपो द्विगुणः स्वमानेन क्षीरोदेन परित उपकॢप्तो वृतो यथा
कुशद्वीपो घृतोदेन यस्मिन्क्रौञ्चो नाम पर्वतराजो द्वीपनामनिर्वर्तक आस्ते
योऽसौ गुहप्रहरणोन्मथितनितम्बकुञ्जोऽपि क्षीरोदेनासिच्यमानो भगवता वरुणेनाभिगुप्तो
विभयो बभूव
तस्मिन्नपि प्रैयव्रतो घृतपृष्ठो नामाधिपतिः स्वे द्वीपे वर्षाणि सप्त विभज्य तेषु पुत्रनामसु
सप्त रिक्थादान्वर्षपान्निवेश्य स्वयं भगवान्भगवतः परमकल्याणयशस आत्मभूतस्य
हरेश्चरणारविन्दमुपजगाम
आमो मधुरुहो मेघपृष्ठः सुधामा भ्राजिष्ठो लोहितार्णो वनस्पतिरिति घृतपृष्ठसुतास्तेषां
वर्षगिरयः सप्त सप्तैव नद्यश्चाभिख्याताः शुक्लो वर्धमानो भोजन उपबर्हिणो नन्दो नन्दनः
सर्वतोभद्र इति अभया अमृतौघा आर्यका तीर्थवती रूपवती पवित्रवती शुक्लेति
यासामम्भः पवित्रममलमुपयुञ्जानाः पुरुषऋषभद्रविणदेवकसंज्ञा वर्षपुरुषा
आपोमयं देवमपां पूर्णेनाञ्जलिना यजन्ते
आपः पुरुषवीर्याः स्थ पुनन्तीर्भूर्भुवःसुवः
ता नः पुनीतामीवघ्नीः स्पृशतामात्मना भुव इति
एवं पुरस्तात्क्षीरोदात्परित उपवेशितः शाकद्वीपो द्वात्रिंशल्लक्षयोजनायामः समानेन च
दधिमण्डोदेन परीतो यस्मिन्शाको नाम महीरुहः स्वक्षेत्रव्यपदेशको यस्य ह महासुरभि
गन्धस्तं द्वीपमनुवासयति
तस्यापि प्रैयव्रत एवाधिपतिर्नाम्ना मेधातिथिः सोऽपि विभज्य सप्त वर्षाणि पुत्रनामानि तेषु
स्वात्मजान्पुरोजवमनोजवपवमानधूम्रानीकचित्ररेफबहुरूपविश्वधार
संज्ञान्निधाप्याधिपतीन्स्वयं भगवत्यनन्त आवेशितमतिस्तपोवनं प्रविवेश
एतेषां वर्षमर्यादागिरयो नद्यश्च सप्त सप्तैव ईशान उरुशृङ्गो बलभद्रः शतकेसरः
सहस्रस्रोतो देवपालो महानस इति अनघायुर्दा उभयस्पृष्टिरपराजिता पञ्चपदी सहस्रस्रुतिर्निजधृतिरिति
तद्वर्षपुरुषा ऋतव्रतसत्यव्रतदानव्रतानुव्रतनामानो भगवन्तं वाय्वात्मकं
प्राणायामविधूतरजस्तमसः परमसमाधिना यजन्ते
अन्तःप्रविश्य भूतानि यो बिभर्त्यात्मकेतुभिः
अन्तर्यामीश्वरः साक्षात्पातु नो यद्वशे स्फुटम्‌
एवमेव दधिमण्डोदात्परतः पुष्करद्वीपस्ततो द्विगुणायामः समन्तत उपकल्पितः
समानेन स्वादूदकेन समुद्रेण बहिरावृतो यस्मिन्बृहत्पुष्करं ज्वलनशिखामलकनक
पत्रायुतायुतं भगवतः कमलासनस्याध्यासनं परिकल्पितम्‌
तद्द्वीपमध्ये मानसोत्तरनामैक एवार्वाचीनपराचीनवर्षयोर्मर्यादाचलोऽयुत
योजनोच्छ्रायायामो यत्र तु चतसृषु दिक्षु चत्वारि पुराणि लोकपालानामिन्द्रादीनां यदुपरिष्टात्सूर्य
रथस्य मेरुं परिभ्रमतः संवत्सरात्मकं चक्रं देवानामहोरात्राभ्यां परिभ्रमति
तद्द्वीपस्याप्यधिपतिः प्रैयव्रतो वीतिहोत्रो नामैतस्यात्मजौ रमणकधातकिनामानौ वर्ष
पती नियुज्य स स्वयं पूर्वजवद्भगवत्कर्मशील एवास्ते
तद्वर्षपुरुषा भगवन्तं ब्रह्मरूपिणं सकर्मकेण कर्मणाराधयन्तीदं चोदाहरन्ति
यत्तत्कर्ममयं लिङ्गं ब्रह्मलिङ्गं जनोऽर्चयेत्‌
एकान्तमद्वयं शान्तं तस्मै भगवते नम इति
ततः परस्ताल्लोकालोकनामाचलो लोकालोकयोरन्तराले परित उपक्षिप्तः
यावन्मानसोत्तरमेर्वोरन्तरं तावती भूमिः काञ्चन्यन्यादर्शतलोपमा यस्यां प्रहितः
पदार्थो न कथञ्चित्पुनः प्रत्युपलभ्यते तस्मात्सर्वसत्त्वपरिहृतासीत्‌
लोकालोक इति समाख्या यदनेनाचलेन लोकालोकस्यान्तर्वर्तिनावस्थाप्यते
स लोकत्रयान्ते परित ईश्वरेण विहितो यस्मात्सूर्यादीनां ध्रुवापवर्गाणां ज्योतिर्गणानां
गभस्तयोऽर्वाचीनांस्त्रीन्लोकानावितन्वाना न कदाचित्पराचीना भवितुमुत्सहन्ते तावदुन्नहनायामः
एतावान्लोकविन्यासो मानलक्षणसंस्थाभिर्विचिन्तितः कविभिः स तु पञ्चाशत्कोटिगणितस्य भू
गोलस्य तुरीयभागोऽयं लोकालोकाचलः
तदुपरिष्टाच्चतसृष्वाशास्वात्मयोनिनाखिलजगद्गुरुणाधिनिवेशिता ये द्विरदपतय ऋषभः
पुष्करचूडो वामनोऽपराजित इति सकललोकस्थितिहेतवः
तेषां स्वविभूतीनां लोकपालानां च विविधवीर्योपबृंहणाय भगवान्परममहापुरुषो
महाविभूतिपतिरन्तर्याम्यात्मनो विशुद्धसत्त्वं धर्मज्ञानवैराग्यैश्वर्याद्यष्टमहासिद्ध्य्‌
उपलक्षणं विष्वक्सेनादिभिः स्वपार्षदप्रवरैः परिवारितो निजवरायुधोपशोभितैर्निजभुजदण्डैः
सन्धारयमाणस्तस्मिन्गिरिवरे समन्तात्सकललोकस्वस्तय आस्ते
आकल्पमेवं वेषं गत एष भगवानात्मयोगमायया विरचितविविधलोकयात्रा
गोपीयायेत्यर्थः
योऽन्तर्विस्तार एतेन ह्यलोकपरिमाणं च व्याख्यातं यद्बहिर्लोकालोकाचलात्ततः
परस्ताद्योगेश्वरगतिं विशुद्धामुदाहरन्ति
अण्डमध्यगतः सूर्यो द्यावाभूम्योर्यदन्तरम्‌
सूर्याण्डगोलयोर्मध्ये कोट्यः स्युः पञ्चविंशतिः
मृतेऽण्ड एष एतस्मिन्यदभूत्ततो मार्तण्ड इति व्यपदेशः हिरण्यगर्भ इति यद्धिरण्याण्ड
समुद्भवः
सूर्येण हि विभज्यन्ते दिशः खं द्यौर्मही भिदा
स्वर्गापवर्गौ नरका रसौकांसि च सर्वशः
देवतिर्यङ्मनुष्याणां सरीसृपसवीरुधाम्‌
सर्वजीवनिकायानां सूर्य आत्मा दृगीश्वरः
श्रीशुक उवाच
एतावानेव भूवलयस्य सन्निवेशः प्रमाणलक्षणतो व्याख्यातः
एतेन हि दिवो मण्डलमानं तद्विद उपदिशन्ति यथा द्विदलयोर्निष्पावादीनां ते
अन्तरेणान्तरिक्षं तदुभयसन्धितम्‌
यन्मध्यगतो भगवांस्तपतां पतिस्तपन आतपेन त्रिलोकीं प्रतपत्यवभासयत्यात्म
भासा स एष उदगयनदक्षिणायनवैषुवतसंज्ञाभिर्मान्द्यशैघ्र्य
समानाभिर्गतिभिरारोहणावरोहणसमानस्थानेषु यथासवनमभिपद्यमानो मकरादिषु
राशिष्वहोरात्राणि दीर्घह्रस्वसमानानि विधत्ते
यदा मेषतुलयोर्वर्तते तदाहोरात्राणि समानानि भवन्ति यदा वृषभादिषु पञ्चसु च राशिषु
चरति तदाहान्येव वर्धन्ते ह्रसति च मासि मास्येकैका घटिका रात्रिषु
यदा वृश्चिकादिषु पञ्चसु वर्तते तदाहोरात्राणि विपर्ययाणि भवन्ति
यावद्दक्षिणायनमहानि वर्धन्ते यावदुदगयनं रात्रयः
एवं नव कोटय एकपञ्चाशल्लक्षाणि योजनानां मानसोत्तरगिरिपरिवर्तनस्योपदिशन्ति
तस्मिन्नैन्द्रीं पुरीं पूर्वस्मान्मेरोर्देवधानीं नाम दक्षिणतो याम्यां संयमनीं नाम
पश्चाद्वारुणीं निम्लोचनीं नाम उत्तरतः सौम्यां विभावरीं नाम तासूदयमध्याह्नास्तमय
निशीथानीति भूतानां प्रवृत्तिनिवृत्तिनिमित्तानि समयविशेषेण मेरोश्चतुर्दिशम्‌
तत्रत्यानां दिवसमध्यङ्गत एव सदादित्यस्तपति सव्येनाचलं दक्षिणेन करोति
यत्रोदेति तस्य ह समानसूत्रनिपाते निम्लोचति यत्र क्वचन स्यन्देनाभितपति तस्य हैष समान
सूत्रनिपाते प्रस्वापयति तत्र गतं न पश्यन्ति ये तं समनुपश्येरन्‌
यदा चैन्द्र्याः पुर्याः प्रचलते पञ्चदशघटिकाभिर्याम्यां सपादकोटिद्वयं योजनानां
सार्धद्वादशलक्षाणि साधिकानि चोपयाति
एवं ततो वारुणीं सौम्यामैन्द्रीं च पुनस्तथान्ये च ग्रहाः सोमादयो नक्षत्रैः सह ज्योतिश्‌
चक्रे समभ्युद्यन्ति सह वा निम्लोचन्ति
एवं मुहूर्तेन चतुस्त्रिंशल्लक्षयोजनान्यष्टशताधिकानि सौरो रथस्त्रयीमयोऽसौ चतसृषु
परिवर्तते पुरीषु
यस्यैकं चक्रं द्वादशारं षण्नेमि त्रिणाभि संवत्सरात्मकं समामनन्ति तस्याक्षो
मेरोर्मूर्धनि कृतो मानसोत्तरे कृतेतरभागो यत्र प्रोतं रविरथचक्रं तैलयन्त्र
चक्रवद्भ्रमन्मानसोत्तरगिरौ परिभ्रमति
तस्मिन्नक्षे कृतमूलो द्वितीयोऽक्षस्तुर्यमानेन सम्मितस्तैलयन्त्राक्षवद्ध्रुवे कृतोपरि
भागः
रथनीडस्तु षट्त्रिंशल्लक्षयोजनायतस्तत्तुरीयभागविशालस्तावान्रविरथयुगो यत्र
हयाश्छन्दोनामानः सप्तारुणयोजिता वहन्ति देवमादित्यम्‌
पुरस्तात्सवितुररुणः पश्चाच्च नियुक्तः सौत्ये कर्मणि किलास्ते
तथा वालिखिल्या ऋषयोऽङ्गुष्ठपर्वमात्राः षष्टिसहस्राणि पुरतः सूर्यं सूक्तवाकाय नियुक्ताः
संस्तुवन्ति
तथान्ये च ऋषयो गन्धर्वाप्सरसो नागा ग्रामण्यो यातुधाना देवा इत्येकैकशो गणाः सप्त
चतुर्दश मासि मासि भगवन्तं सूर्यमात्मानं नानानामानं पृथङ्नानानामानः पृथक्‌
कर्मभिर्द्वन्द्वश उपासते
राजोवाच
यदेतद्भगवत आदित्यस्य मेरुं ध्रुवं च प्रदक्षिणेन परिक्रामतो राशीनामभिमुखं
प्रचलितं चाप्रदक्षिणं भगवतोपवर्णितममुष्य वयं कथमनुमिमीमहीति
स होवाच
यथा कुलालचक्रेण भ्रमता सह भ्रमतां तदाश्रयाणां पिपीलिकादीनां गतिरन्यैव
प्रदेशान्तरेष्वप्युपलभ्यमानत्वादेवं नक्षत्रराशिभिरुपलक्षितेन कालचक्रेण ध्रुवं मेरुं च
प्रदक्षिणेन परिधावता सह परिधावमानानां तदाश्रयाणां सूर्यादीनां ग्रहाणां गतिरन्यैव
नक्षत्रान्तरे राश्यन्तरे चोपलभ्यमानत्वात्‌
स एष भगवानादिपुरुष एव साक्षान्नारायणो लोकानां स्वस्तय आत्मानं त्रयीमयं कर्म
विशुद्धिनिमित्तं कविभिरपि च वेदेन विजिज्ञास्यमानो द्वादशधा विभज्य षट्सु वसन्तादिष्वृतुषु यथोप
जोषमृतुगुणान्विदधाति
तमेतमिह पुरुषास्त्रय्या विद्यया वर्णाश्रमाचारानुपथा उच्चावचैः कर्मभिराम्नातैर्योग
वितानैश्च श्रद्धया यजन्तोऽञ्जसा श्रेयः समधिगच्छन्ति
अथ स एष आत्मा लोकानां द्यावापृथिव्योरन्तरेण नभोवलयस्य कालचक्रगतो द्वादश
मासान्भुङ्क्ते राशिसंज्ञान्संवत्सरावयवान्मासः पक्षद्वयं दिवा नक्तं चेति सपादर्क्ष
द्वयमुपदिशन्ति यावता षष्ठमंशं भुञ्जीत स वै ऋतुरित्युपदिश्यते संवत्सरावयवः
अथ च यावतार्धेन नभोवीथ्यां प्रचरति तं कालमयनमाचक्षते
अथ च यावन्नभोमण्डलं सह द्यावापृथिव्योर्मण्डलाभ्यां कार्त्स्न्येन स ह भुञ्जीत तं
कालं संवत्सरं परिवत्सरमिडावत्सरमनुवत्सरं वत्सरमिति भानोर्मान्द्यशैघ्र्यसम
गतिभिः समामनन्ति
एवं चन्द्रमा अर्कगभस्तिभ्य उपरिष्टाल्लक्षयोजनत उपलभ्यमानोऽर्कस्य संवत्सर
भुक्तिं पक्षाभ्यां मासभुक्तिं सपादर्क्षाभ्यां दिनेनैव पक्षभुक्तिमग्रचारी द्रुततरगमनो
भुङ्क्ते
अथ चापूर्यमाणाभिश्च कलाभिरमराणां क्षीयमाणाभिश्च कलाभिः पित्णामहोरात्राणि पूर्व
पक्षापरपक्षाभ्यां वितन्वानः सर्वजीवनिवहप्राणो जीवश्चैकमेकं नक्षत्रं त्रिंशता
मुहूर्तैर्भुङ्क्ते
य एष षोडशकलः पुरुषो भगवान्मनोमयोऽन्नमयोऽमृतमयो देवपितृमनुष्यभूतपशु
पक्षिसरीसृपवीरुधां प्राणाप्यायनशीलत्वात्सर्वमय इति वर्णयन्ति
तत उपरिष्टाद्द्विलक्षयोजनतो नक्षत्राणि मेरुं दक्षिणेनैव कालायन ईश्वरयोजितानि
सहाभिजिताष्टाविंशतिः
तत उपरिष्टादुशना द्विलक्षयोजनत उपलभ्यते पुरतः पश्चात्सहैव वार्कस्य शैघ्र्य
मान्द्यसाम्याभिर्गतिभिरर्कवच्चरति लोकानां नित्यदानुकूल एव प्रायेण वर्षयंश्चारेणानुमीयते स
वृष्टिविष्टम्भग्रहोपशमनः
उशनसा बुधो व्याख्यातस्तत उपरिष्टाद्द्विलक्षयोजनतो बुधः सोमसुत उपलभ्यमानः
प्रायेण शुभकृद्यदार्काद्व्यतिरिच्येत तदातिवाताभ्रप्रायानावृष्ट्यादिभयमाशंसते
अत ऊर्ध्वमङ्गारकोऽपि योजनलक्षद्वितय उपलभ्यमानस्त्रिभिस्त्रिभिः पक्षैरेकैकशो
राशीन्द्वादशानुभुङ्क्ते यदि न वक्रेणाभिवर्तते प्रायेणाशुभग्रहोऽघशंसः
तत उपरिष्टाद्द्विलक्षयोजनान्तरगता भगवान्बृहस्पतिरेकैकस्मिन्राशौ परिवत्सरं
परिवत्सरं चरति यदि न वक्रः स्यात्प्रायेणानुकूलो ब्राह्मणकुलस्य
तत उपरिष्टाद्योजनलक्षद्वयात्प्रतीयमानः शनैश्चर एकैकस्मिन्राशौ
त्रिंशन्मासान्विलम्बमानः सर्वानेवानुपर्येति तावद्भिरनुवत्सरैः प्रायेण हि सर्वेषामशान्तिकरः
तत उत्तरस्मादृषय एकादशलक्षयोजनान्तर उपलभ्यन्ते य एव लोकानां शमनुभावयन्तो
भगवतो विष्णोर्यत्परमं पदं प्रदक्षिणं प्रक्रमन्ति
श्रीशुक उवाच
अथ तस्मात्परतस्त्रयोदशलक्षयोजनान्तरतो यत्तद्विष्णोः परमं पदमभिवदन्ति यत्र
ह महाभागवतो ध्रुव औत्तानपादिरग्निनेन्द्रेण प्रजापतिना कश्यपेन धर्मेण च समकालयुग्भिः
सबहुमानं दक्षिणतः क्रियमाण इदानीमपि कल्पजीविनामाजीव्य उपास्ते तस्येहानुभाव उपवर्णितः
स हि सर्वेषां ज्योतिर्गणानां ग्रहनक्षत्रादीनामनिमिषेणाव्यक्तरंहसा भगवता कालेन
भ्राम्यमाणानां स्थाणुरिवावष्टम्भ ईश्वरेण विहितः शश्वदवभासते
यथा मेढीस्तम्भ आक्रमणपशवः संयोजितास्त्रिभिस्त्रिभिः सवनैर्यथास्थानं मण्डलानि
चरन्त्येवं भगणा ग्रहादय एतस्मिन्नन्तर्बहिर्योगेन कालचक्र आयोजिता ध्रुवमेवावलम्ब्य
वायुनोदीर्यमाणा आकल्पान्तं परिचङ्क्रमन्ति नभसि यथा मेघाः श्येनादयो वायुवशाः कर्मसारथयः
परिवर्तन्ते एवं ज्योतिर्गणाः प्रकृतिपुरुषसंयोगानुगृहीताः कर्मनिर्मितगतयो भुवि न पतन्ति
केचनैतज्ज्योतिरनीकं शिशुमारसंस्थानेन भगवतो वासुदेवस्य योग
धारणायामनुवर्णयन्ति
यस्य पुच्छाग्रेऽवाक्षिरसः कुण्डलीभूतदेहस्य ध्रुव उपकल्पितस्तस्य लाङ्गूले
प्रजापतिरग्निरिन्द्रो धर्म इति पुच्छमूले धाता विधाता च कट्यां सप्तर्षयः तस्य दक्षिणावर्तकुण्डली
भूतशरीरस्य यान्युदगयनानि दक्षिणपार्श्वे तु नक्षत्राण्युपकल्पयन्ति दक्षिणायनानि तु सव्ये यथा
शिशुमारस्य कुण्डलाभोगसन्निवेशस्य पार्श्वयोरुभयोरप्यवयवाः समसङ्ख्या भवन्ति पृष्ठे
त्वजवीथी आकाशगङ्गा चोदरतः
पुनर्वसुपुष्यौ दक्षिणवामयोः श्रोण्योरार्द्राश्लेषे च दक्षिणवामयोः पश्चिमयोः
पादयोरभिजिदुत्तराषाढे दक्षिणवामयोर्नासिकयोर्यथासङ्ख्यं श्रवणपूर्वाषाढे दक्षिण
वामयोर्लोचनयोर्धनिष्ठा मूलं च दक्षिणवामयोः कर्णयोर्मघादीन्यष्ट नक्षत्राणि दक्षिणायनानि
वामपार्श्ववङ्क्रिषु युञ्जीत तथैव मृगशीर्षादीन्युदगयनानि दक्षिणपार्श्ववङ्क्रिषु प्रातिलोम्येन
प्रयुञ्जीत शतभिषाज्येष्ठे स्कन्धयोर्दक्षिणवामयोर्न्यसेत्‌
उत्तराहनावगस्तिरधराहनौ यमो मुखेषु चाङ्गारकः शनैश्चर उपस्थे बृहस्पतिः ककुदि
वक्षस्यादित्यो हृदये नारायणो मनसि चन्द्रो नाभ्यामुशना स्तनयोरश्विनौ बुधः प्राणापानयो
राहुर्गले केतवः सर्वाङ्गेषु रोमसु सर्वे तारागणाः
एतदु हैव भगवतो विष्णोः सर्वदेवतामयं रूपमहरहः सन्ध्यायां प्रयतो वाग्यतो
निरीक्षमाण उपतिष्ठेत नमो ज्योतिर्लोकाय कालायनायानिमिषां पतये महापुरुषायाभिधीमहीति
ग्रहर्क्षतारामयमाधिदैविकं पापापहं मन्त्रकृतां त्रिकालम्‌
नमस्यतः स्मरतो वा त्रिकालं नश्येत तत्कालजमाशु पापम्‌
श्रीशुक उवाच
अधस्तात्सवितुर्योजनायुते स्वर्भानुर्नक्षत्रवच्चरतीत्येके योऽसावमरत्वं ग्रहत्वं
चालभत भगवदनुकम्पया स्वयमसुरापसदः सैंहिकेयो ह्यतदर्हस्तस्य तात जन्म कर्माणि
चोपरिष्टाद्वक्ष्यामः
यददस्तरणेर्मण्डलं प्रतपतस्तद्विस्तरतो योजनायुतमाचक्षते द्वादशसहस्रं
सोमस्य त्रयोदशसहस्रं राहोर्यः पर्वणि तद्व्यवधानकृद्वैरानुबन्धः सूर्या
चन्द्रमसावभिधावति
तन्निशम्योभयत्रापि भगवता रक्षणाय प्रयुक्तं सुदर्शनं नाम भागवतं
दयितमस्त्रं तत्तेजसा दुर्विषहं मुहुः परिवर्तमानमभ्यवस्थितो मुहूर्तमुद्विजमानश्चकित
हृदय आरादेव निवर्तते तदुपरागमिति वदन्ति लोकाः
ततोऽधस्तात्सिद्धचारणविद्याधराणां सदनानि तावन्मात्र एव
ततोऽधस्ताद्यक्षरक्षःपिशाचप्रेतभूतगणानां विहाराजिरमन्तरिक्षं यावद्वायुः प्रवाति
यावन्मेघा उपलभ्यन्ते
ततोऽधस्ताच्छतयोजनान्तर इयं पृथिवी यावद्धंसभासश्येनसुपर्णादयः पतत्त्रि
प्रवरा उत्पतन्तीति
उपवर्णितं भूमेर्यथासन्निवेशावस्थानमवनेरप्यधस्तात्सप्त भूविवरा एकैकशो
योजनायुतान्तरेणायामविस्तारेणोपकॢप्ता अतलं वितलं सुतलं तलातलं महातलं रसातलं पातालमिति
एतेषु हि बिलस्वर्गेषु स्वर्गादप्यधिककामभोगैश्वर्यानन्दभूतिविभूतिभिः सुसमृद्ध
भवनोद्यानाक्रीडविहारेषु दैत्यदानवकाद्रवेया नित्यप्रमुदितानुरक्तकलत्रापत्यबन्धुसुहृद्‌
अनुचरा गृहपतय ईश्वरादप्यप्रतिहतकामा मायाविनोदा निवसन्ति
येषु महाराज मयेन मायाविना विनिर्मिताः पुरो नानामणिप्रवरप्रवेकविरचितविचित्रभवन
प्राकारगोपुरसभाचैत्यचत्वरायतनादिभिर्नागासुरमिथुनपारावतशुकसारिकाकीर्णकृत्रिम
भूमिभिर्विवरेश्वरगृहोत्तमैः समलङ्कृताश्चकासति
उद्यानानि चातितरां मनैन्द्रियानन्दिभिः कुसुमफलस्तबकसुभगकिसलयावनतरुचिर
विटपविटपिनां लताङ्गालिङ्गितानां श्रीभिः समिथुनविविधविहङ्गमजलाशयानाममलजलपूर्णानां
झषकुलोल्लङ्घनक्षुभितनीरनीरजकुमुदकुवलयकह्लारनीलोत्पललोहितशतपत्रादिवनेषु कृत
निकेतनानामेकविहाराकुलमधुरविविधस्वनादिभिरिन्द्रियोत्सवैरमरलोकश्रियमतिशयितानि
यत्र ह वाव न भयमहोरात्रादिभिः कालविभागैरुपलक्ष्यते
यत्र हि महाहिप्रवरशिरोमणयः सर्वं तमः प्रबाधन्ते
न वा एतेषु वसतां दिव्यौषधिरसरसायनान्नपानस्नानादिभिराधयो व्याधयो वलीपलित
जरादयश्च देहवैवर्ण्यदौर्गन्ध्यस्वेदक्लमग्लानिरिति वयोऽवस्थाश्च भवन्ति
न हि तेषां कल्याणानां प्रभवति कुतश्चन मृत्युर्विना भगवत्तेजसश्चक्रापदेशात्‌
यस्मिन्प्रविष्टेऽसुरवधूनां प्रायः पुंसवनानि भयादेव स्रवन्ति पतन्ति च
अथातले मयपुत्रोऽसुरो बलो निवसति येन ह वा इह सृष्टाः षण्णवतिर्मायाः काश्चनाद्यापि
मायाविनो धारयन्ति यस्य च जृम्भमाणस्य मुखतस्त्रयः स्त्रीगणा उदपद्यन्त स्वैरिण्यः कामिन्यः
पुंश्चल्य इति या वै बिलायनं प्रविष्टं पुरुषं रसेन हाटकाख्येन साधयित्वा स्वविलासावलोकनानुराग
स्मितसंलापोपगूहनादिभिः स्वैरं किल रमयन्ति यस्मिन्नुपयुक्ते पुरुष ईश्वरोऽहं सिद्धो
ऽहमित्ययुतमहागजबलमात्मानमभिमन्यमानः कत्थते मदान्ध इव
ततोऽधस्ताद्वितले हरो भगवान्हाटकेश्वरः स्वपार्षदभूतगणावृतः प्रजापति
सर्गोपबृंहणाय भवो भवान्या सह मिथुनीभूत आस्ते यतः प्रवृत्ता सरित्प्रवरा हाटकी नाम
भवयोर्वीर्येण यत्र चित्रभानुर्मातरिश्वना समिध्यमान ओजसा पिबति तन्निष्ठ्यूतं हाटकाख्यं
सुवर्णं भूषणेनासुरेन्द्रावरोधेषु पुरुषाः सह पुरुषीभिर्धारयन्ति
ततोऽधस्तात्सुतले उदारश्रवाः पुण्यश्लोको विरोचनात्मजो बलिर्भगवता महेन्द्रस्य प्रियं
चिकीर्षमाणेनादितेर्लब्धकायो भूत्वा वटुवामनरूपेण पराक्षिप्तलोकत्रयो भगवदनुकम्पयैव
पुनः प्रवेशित इन्द्रादिष्वविद्यमानया सुसमृद्धया श्रियाभिजुष्टः स्वधर्मेणाराधयंस्तमेव
भगवन्तमाराधनीयमपगतसाध्वस आस्तेऽधुनापि
नो एवैतत्साक्षात्कारो भूमिदानस्य यत्तद्भगवत्यशेषजीवनिकायानां जीवभूतात्मभूते
परमात्मनि वासुदेवे तीर्थतमे पात्र उपपन्ने परया श्रद्धया परमादरसमाहितमनसा
सम्प्रतिपादितस्य साक्षादपवर्गद्वारस्य यद्बिलनिलयैश्वर्यम्‌
यस्य ह वाव क्षुतपतनप्रस्खलनादिषु विवशः सकृन्नामाभिगृणन्पुरुषः कर्म
बन्धनमञ्जसा विधुनोति यस्य हैव प्रतिबाधनं मुमुक्षवोऽन्यथैवोपलभन्ते
तद्भक्तानामात्मवतां सर्वेषामात्मन्यात्मद आत्मतयैव
न वै भगवान्नूनममुष्यानुजग्राह यदुत पुनरात्मानुस्मृतिमोषणं मायामय
भोगैश्वर्यमेवातनुतेति
यत्तद्भगवतानधिगतान्योपायेन याच्ञाच्छलेनापहृतस्वशरीरावशेषितलोकत्रयो वरुण
पाशैश्च सम्प्रतिमुक्तो गिरिदर्यां चापविद्ध इति होवाच
नूनं बतायं भगवानर्थेषु न निष्णातो योऽसाविन्द्रो यस्य सचिवो मन्त्राय वृत एकान्ततो
बृहस्पतिस्तमतिहाय स्वयमुपेन्द्रेणात्मानमयाचतात्मनश्चाशिषो नो एव तद्दास्यमतिगम्भीर
वयसः कालस्य मन्वन्तरपरिवृत्तं कियल्लोकत्रयमिदम्‌
यस्यानुदास्यमेवास्मत्पितामहः किल वव्रे न तु स्वपित्र्यं यदुताकुतोभयं पदं
दीयमानं भगवतः परमिति भगवतोपरते खलु स्वपितरि
तस्य महानुभावस्यानुपथममृजितकषायः को वास्मद्विधः परिहीणभगवदनुग्रह
उपजिगमिषतीति
तस्यानुचरितमुपरिष्टाद्विस्तरिष्यते यस्य भगवान्स्वयमखिलजगद्गुरुर्नारायणो द्वारि
गदापाणिरवतिष्ठते निजजनानुकम्पितहृदयो येनाङ्गुष्ठेन पदा दशकन्धरो योजनायुतायुतं दिग्‌
विजय उच्चाटितः
ततोऽधस्तात्तलातले मयो नाम दानवेन्द्रस्त्रिपुराधिपतिर्भगवता पुरारिणा त्रिलोकीशं
चिकीर्षुणा निर्दग्धस्वपुरत्रयस्तत्प्रसादाल्लब्धपदो मायाविनामाचार्यो महादेवेन परिरक्षितो
विगतसुदर्शनभयो महीयते
ततोऽधस्तान्महातले काद्रवेयाणां सर्पाणां नैकशिरसां क्रोधवशो नाम गणः कुहक
तक्षककालियसुषेणादिप्रधाना महाभोगवन्तः पतत्त्रिराजाधिपतेः पुरुष
वाहादनवरतमुद्विजमानाः स्वकलत्रापत्यसुहृत्कुटुम्बसङ्गेन क्वचित्प्रमत्ता विहरन्ति
ततोऽधस्ताद्रसातले दैतेया दानवाः पणयो नाम निवातकवचाः कालेया हिरण्यपुरवासिन इति
विबुधप्रत्यनीका उत्पत्त्या महौजसो महासाहसिनो भगवतः सकललोकानुभावस्य हरेरेव तेजसा
प्रतिहतबलावलेपा बिलेशया इव वसन्ति ये वै सरमयेन्द्रदूत्या वाग्भिर्मन्त्रवर्णाभिरिन्द्राद्बिभ्यति
ततोऽधस्तात्पाताले नागलोकपतयो वासुकिप्रमुखाः शङ्खकुलिकमहाशङ्खश्वेत
धनञ्जयधृतराष्ट्रशङ्खचूडकम्बलाश्वतरदेवदत्तादयो महाभोगिनो महामर्षा निवसन्ति
येषामु ह वै पञ्चसप्तदशशतसहस्रशीर्षाणां फणासु विरचिता महामणयो रोचिष्णवः पाताल
विवरतिमिरनिकरं स्वरोचिषा विधमन्ति
श्रीशुक उवाच
तस्य मूलदेशे त्रिंशद्योजनसहस्रान्तर आस्ते या वै कला भगवतस्तामसी समाख्यातानन्त इति
सात्वतीया द्रष्टृदृश्ययोः सङ्कर्षणमहमित्यभिमानलक्षणं यं सङ्कर्षणमित्याचक्षते
यस्येदं क्षितिमण्डलं भगवतोऽनन्तमूर्तेः सहस्रशिरस एकस्मिन्नेव शीर्षणि ध्रियमाणं
सिद्धार्थ इव लक्ष्यते
यस्य ह वा इदं कालेनोपसञ्जिहीर्षतोऽमर्षविरचितरुचिरभ्रमद्भ्रुवोरन्तरेण साङ्कर्षणो
नाम रुद्र एकादशव्यूहस्त्र्यक्षस्त्रिशिखं शूलमुत्तम्भयन्नुदतिष्ठत्‌
यस्याङ्घ्रिकमलयुगलारुणविशदनखमणिषण्डमण्डलेष्वहिपतयः सह
सात्वतर्षभैरेकान्तभक्तियोगेनावनमन्तः स्ववदनानि परिस्फुरत्कुण्डलप्रभामण्डितगण्ड
स्थलान्यतिमनोहराणि प्रमुदितमनसः खलु विलोकयन्ति
यस्यैव हि नागराजकुमार्य आशिष आशासानाश्चार्वङ्गवलयविलसितविशदविपुलधवल
सुभगरुचिरभुजरजतस्तम्भेष्वगुरुचन्दनकुङ्कुमपङ्कानुलेपेनावलिम्पमानास्तद्‌
अभिमर्शनोन्मथितहृदयमकरध्वजावेशरुचिरललितस्मितास्तदनुरागमदमुदितमद
विघूर्णितारुणकरुणावलोकनयनवदनारविन्दं सव्रीडं किल विलोकयन्ति
स एव भगवाननन्तोऽनन्तगुणार्णव आदिदेव उपसंहृतामर्षरोषवेगो लोकानां स्वस्तय
आस्ते
ध्यायमानः सुरासुरोरगसिद्धगन्धर्वविद्याधरमुनिगणैरनवरतमदमुदितविकृत
विह्वललोचनः सुललितमुखरिकामृतेनाप्यायमानः स्वपार्षदविबुधयूथपतीनपरिम्लानरागनव
तुलसिकामोदमध्वासवेन माद्यन्मधुकरव्रातमधुरगीतश्रियं वैजयन्तीं स्वां वनमालां नील
वासा एककुण्डलो हलककुदि कृतसुभगसुन्दरभुजो भगवान्महेन्द्रो वारणेन्द्र इव काञ्चनीं
कक्षामुदारलीलो बिभर्ति
य एष एवमनुश्रुतो ध्यायमानो मुमुक्षूणामनादिकालकर्मवासनाग्रथितमविद्यामयं
हृदयग्रन्थिं सत्त्वरजस्तमोमयमन्तर्हृदयं गत आशु निर्भिनत्ति
तस्यानुभावान्भगवान्स्वायम्भुवो नारदः सह तुम्बुरुणा सभायां ब्रह्मणः संश्लोकयामास
उत्पत्तिस्थितिलयहेतवोऽस्य कल्पाः
सत्त्वाद्याः प्रकृतिगुणा यदीक्षयासन्‌
यद्रूपं ध्रुवमकृतं यदेकमात्मन्‌
नानाधात्कथमु ह वेद तस्य वर्त्म
मूर्तिं नः पुरुकृपया बभार सत्त्वं
संशुद्धं सदसदिदं विभाति तत्र
यल्लीलां मृगपतिराददेऽनवद्याम्‌
आदातुं स्वजनमनांस्युदारवीर्यः
यन्नाम श्रुतमनुकीर्तयेदकस्माद्‌
आर्तो वा यदि पतितः प्रलम्भनाद्वा
हन्त्यंहः सपदि नृणामशेषमन्यं
कं शेषाद्भगवत आश्रयेन्मुमुक्षुः
मूर्धन्यर्पितमणुवत्सहस्रमूर्ध्नो
भूगोलं सगिरिसरित्समुद्रसत्त्वम्‌
आनन्त्यादनिमितविक्रमस्य भूम्नः
को वीर्याण्यधि गणयेत्सहस्रजिह्वः
एवम्प्रभावो भगवाननन्तो
दुरन्तवीर्योरुगुणानुभावः
मूले रसायाः स्थित आत्मतन्त्रो
यो लीलया क्ष्मां स्थितये बिभर्ति
एता ह्येवेह नृभिरुपगन्तव्या गतयो यथाकर्मविनिर्मिता यथोपदेशमनुवर्णिताः
कामान्कामयमानैः
एतावतीर्हि राजन्पुंसः प्रवृत्तिलक्षणस्य धर्मस्य विपाकगतय उच्चावचा विसदृशा यथा
प्रश्नं व्याचख्ये किमन्यत्कथयाम इति
राजोवाच
महर्ष एतद्वैचित्र्यं लोकस्य कथमिति
ऋषिरुवाच
त्रिगुणत्वात्कर्तुः श्रद्धया कर्मगतयः पृथग्विधाः सर्वा एव सर्वस्य तारतम्येन भवन्ति
अथेदानीं प्रतिषिद्धलक्षणस्याधर्मस्य तथैव कर्तुः श्रद्धाया वैसादृश्यात्कर्मफलं
विसदृशं भवति या ह्यनाद्यविद्यया कृतकामानां तत्परिणामलक्षणाः सृतयः सहस्रशः
प्रवृत्तास्तासां प्राचुर्येणानुवर्णयिष्यामः
राजोवाच
नरका नाम भगवन्किं देशविशेषा अथवा बहिस्त्रिलोक्या आहोस्विदन्तराल इति
ऋषिरुवाच
अन्तराल एव त्रिजगत्यास्तु दिशि दक्षिणस्यामधस्ताद्भूमेरुपरिष्टाच्च
जलाद्यस्यामग्निष्वात्तादयः पितृगणा दिशि स्वानां गोत्राणां परमेण समाधिना सत्या एवाशिष आशासाना
निवसन्ति
यत्र ह वाव भगवान्पितृराजो वैवस्वतः स्वविषयं प्रापितेषु स्वपुरुषैर्जन्तुषु सम्परेतेषु
यथाकर्मावद्यं दोषमेवानुल्लङ्घितभगवच्छासनः सगणो दमं धारयति
तत्र हैके नरकानेकविंशतिं गणयन्ति अथ तांस्ते राजन्नामरूपलक्षणतो
ऽनुक्रमिष्यामस्तामिस्रोऽन्धतामिस्रो रौरवो महारौरवः कुम्भीपाकः कालसूत्रमसिपत्रवनं
सूकरमुखमन्धकूपः कृमिभोजनः सन्दंशस्तप्तसूर्मिर्वज्रकण्टकशाल्मली वैतरणी पूयोदः
प्राणरोधो विशसनं लालाभक्षः सारमेयादनमवीचिरयःपानमिति किञ्च क्षारकर्दमो रक्षोगण
भोजनः शूलप्रोतो दन्दशूकोऽवटनिरोधनः पर्यावर्तनः सूचीमुखमित्यष्टाविंशतिर्नरका विविध
यातनाभूमयः
तत्र यस्तु परवित्तापत्यकलत्राण्यपहरति स हि कालपाशबद्धो यमपुरुषैरति
भयानकैस्तामिस्रे नरके बलान्निपात्यते अनशनानुदपानदण्डताडन
सन्तर्जनादिभिर्यातनाभिर्यात्यमानो जन्तुर्यत्र कश्मलमासादित एकदैव मूर्च्छामुपयाति तामिस्र
प्राये
एवमेवान्धतामिस्रे यस्तु वञ्चयित्वा पुरुषं दारादीनुपयुङ्क्ते यत्र शरीरी निपात्यमानो यातना
स्थो वेदनया नष्टमतिर्नष्टदृष्टिश्च भवति यथा वनस्पतिर्वृश्च्यमानमूलस्तस्मादन्धतामिस्रं
तमुपदिशन्ति
यस्त्विह वा एतदहमिति ममेदमिति भूतद्रोहेण केवलं स्वकुटुम्बमेवानुदिनं
प्रपुष्णाति स तदिह विहाय स्वयमेव तदशुभेन रौरवे निपतति
ये त्विह यथैवामुना विहिंसिता जन्तवः परत्र यमयातनामुपगतं त एव रुरवो भूत्वा
तथा तमेव विहिंसन्ति तस्माद्रौरवमित्याहू रुरुरिति सर्पादतिक्रूरसत्त्वस्यापदेशः
एवमेव महारौरवो यत्र निपतितं पुरुषं क्रव्यादा नाम रुरवस्तं क्रव्येण घातयन्ति यः
केवलं देहम्भरः
यस्त्विह वा उग्रः पशून्पक्षिणो वा प्राणत उपरन्धयति तमपकरुणं पुरुषादैरपि
विगर्हितममुत्र यमानुचराः कुम्भीपाके तप्ततैले उपरन्धयन्ति
यस्त्विह ब्रह्मध्रुक्स कालसूत्रसंज्ञके नरके अयुतयोजनपरिमण्डले ताम्रमये तप्त
खले उपर्यधस्तादग्न्यर्काभ्यामतितप्यमानेऽभिनिवेशितः क्षुत्पिपासाभ्यां च दह्यमानान्तर्बहिः
शरीर आस्ते शेते चेष्टतेऽवतिष्ठति परिधावति च यावन्ति पशुरोमाणि तावद्वर्षसहस्राणि
यस्त्विह वै निजवेदपथादनापद्यपगतः पाखण्डं चोपगतस्तमसिपत्रवनं
प्रवेश्य कशया प्रहरन्ति तत्र हासावितस्ततो धावमान उभयतो धारैस्तालवनासिपत्रैश्छिद्यमान
सर्वाङ्गो हा हतोऽस्मीति परमया वेदनया मूर्च्छितः पदे पदे निपतति स्वधर्महा पाखण्डानुगतं
फलं भुङ्क्ते
यस्त्विह वै राजा राजपुरुषो वा अदण्ड्ये दण्डं प्रणयति ब्राह्मणे वा शरीरदण्डं स
पापीयान्नरकेऽमुत्र सूकरमुखे निपतति तत्रातिबलैर्विनिष्पिष्यमाणावयवो यथैवेहेक्षुखण्ड आर्त
स्वरेण स्वनयन्क्वचिन्मूर्च्छितः कश्मलमुपगतो यथैवेहादृष्टदोषा उपरुद्धाः
यस्त्विह वै भूतानामीश्वरोपकल्पितवृत्तीनामविविक्तपरव्यथानां स्वयं पुरुषोपकल्पित
वृत्तिर्विविक्तपरव्यथो व्यथामाचरति स परत्रान्धकूपे तदभिद्रोहेण निपतति तत्र हासौ
तैर्जन्तुभिः पशुमृगपक्षिसरीसृपैर्मशकयूकामत्कुणमक्षिकादिभिर्ये के चाभिद्रुग्धास्तैः
सर्वतोऽभिद्रुह्यमाणस्तमसि विहतनिद्रानिर्वृतिरलब्धावस्थानः परिक्रामति यथा कुशरीरे जीवः
यस्त्विह वा असंविभज्याश्नाति यत्किञ्चनोपनतमनिर्मितपञ्चयज्ञो वायससंस्तुतः स
परत्र कृमिभोजने नरकाधमे निपतति तत्र शतसहस्रयोजने कृमिकुण्डे कृमिभूतः स्वयं
कृमिभिरेव भक्ष्यमाणः कृमिभोजनो यावत्तदप्रत्ताप्रहूतादोऽनिर्वेशमात्मानं यातयते
यस्त्विह वै स्तेयेन बलाद्वा हिरण्यरत्नादीनि ब्राह्मणस्य वापहरत्यन्यस्य वानापदि
पुरुषस्तममुत्र राजन्यमपुरुषा अयस्मयैरग्निपिण्डैः सन्दंशैस्त्वचि निष्कुषन्ति
यस्त्विह वा अगम्यां स्त्रियमगम्यं वा पुरुषं योषिदभिगच्छति तावमुत्र कशया
ताडयन्तस्तिग्मया सूर्म्या लोहमय्या पुरुषमालिङ्गयन्ति स्त्रियं च पुरुषरूपया सूर्म्या
यस्त्विह वै सर्वाभिगमस्तममुत्र निरये वर्तमानं वज्रकण्टकशाल्मलीमारोप्य
निष्कर्षन्ति
ये त्विह वै राजन्या राजपुरुषा वा अपाखण्डा धर्मसेतून्भिन्दन्ति ते सम्परेत्य वैतरण्यां
निपतन्ति भिन्नमर्यादास्तस्यां निरयपरिखाभूतायां नद्यां यादोगणैरितस्ततो भक्ष्यमाणा
आत्मना न वियुज्यमानाश्चासुभिरुह्यमानाः स्वाघेन कर्मपाकमनुस्मरन्तो विण्मूत्रपूयशोणित
केशनखास्थिमेदोमांसवसावाहिन्यामुपतप्यन्ते
ये त्विह वै वृषलीपतयो नष्टशौचाचारनियमास्त्यक्तलज्जाः पशुचर्यां चरन्ति ते चापि प्रेत्य
पूयविण्मूत्रश्लेष्ममलापूर्णार्णवे निपतन्ति तदेवातिबीभत्सितमश्नन्ति
ये त्विह वै श्वगर्दभपतयो ब्राह्मणादयो मृगया विहारा अतीर्थे च मृगान्निघ्नन्ति तानपि
सम्परेतान्लक्ष्यभूतान्यमपुरुषा इषुभिर्विध्यन्ति
ये त्विह वै दाम्भिका दम्भयज्ञेषु पशून्विशसन्ति तानमुष्मिन्लोके वैशसे नरके
पतितान्निरयपतयो यातयित्वा विशसन्ति
यस्त्विह वै सवर्णां भार्यां द्विजो रेतः पाययति काममोहितस्तं पापकृतममुत्र रेतः
कुल्यायां पातयित्वा रेतः सम्पाययन्ति
ये त्विह वै दस्यवोऽग्निदा गरदा ग्रामान्सार्थान्वा विलुम्पन्ति राजानो राजभटा वा तांश्चापि हि
परेत्य यमदूता वज्रदंष्ट्राः श्वानः सप्तशतानि विंशतिश्च सरभसं खादन्ति
यस्त्विह वा अनृतं वदति साक्ष्ये द्रव्यविनिमये दाने वा कथञ्चित्स वै प्रेत्य नरके
ऽवीचिमत्यधःशिरा निरवकाशे योजनशतोच्छ्रायाद्गिरिमूर्ध्नः सम्पात्यते यत्र जलमिव स्थलमश्म
पृष्ठमवभासते तदवीचिमत्तिलशो विशीर्यमाणशरीरो न म्रियमाणः पुनरारोपितो निपतति
यस्त्विह वै विप्रो राजन्यो वैश्यो वा सोमपीथस्तत्कलत्रं वा सुरां व्रतस्थोऽपि वा पिबति
प्रमादतस्तेषां निरयं नीतानामुरसि पदाक्रम्यास्ये वह्निना द्रवमाणं कार्ष्णायसं निषिञ्चन्ति
अथ च यस्त्विह वा आत्मसम्भावनेन स्वयमधमो जन्मतपोविद्याचारवर्णाश्रमवतो
वरीयसो न बहु मन्येत स मृतक एव मृत्वा क्षारकर्दमे निरयेऽवाक्षिरा निपातितो दुरन्ता यातना ह्यश्नुते
ये त्विह वै पुरुषाः पुरुषमेधेन यजन्ते याश्च स्त्रियो नृपशून्खादन्ति तांश्च ते पशव इव
निहता यमसदने यातयन्तो रक्षोगणाः सौनिका इव स्वधितिनावदायासृक्पिबन्ति नृत्यन्ति च गायन्ति च
हृष्यमाणा यथेह पुरुषादाः
ये त्विह वा अनागसोऽरण्ये ग्रामे वा वैश्रम्भकैरुपसृतानुपविश्रम्भय्य जिजीविषून्शूल
सूत्रादिषूपप्रोतान्क्रीडनकतया यातयन्ति तेऽपि च प्रेत्य यमयातनासु शूलादिषु प्रोतात्मानः क्षुत्तृड्भ्यां
चाभिहताः कङ्कवटादिभिश्चेतस्ततस्तिग्मतुण्डैराहन्यमाना आत्मशमलं स्मरन्ति
ये त्विह वै भूतान्युद्वेजयन्ति नरा उल्बणस्वभावा यथा दन्दशूकास्तेऽपि प्रेत्य नरके
दन्दशूकाख्ये निपतन्ति यत्र नृप दन्दशूकाः पञ्चमुखाः सप्तमुखा उपसृत्य ग्रसन्ति यथा बिलेशयान्‌
ये त्विह वा अन्धावटकुसूलगुहादिषु भूतानि निरुन्धन्ति तथामुत्र तेष्वेवोपवेश्य सगरेण
वह्निना धूमेन निरुन्धन्ति
यस्त्विह वा अतिथीनभ्यागतान्वा गृहपतिरसकृदुपगतमन्युर्दिधक्षुरिव पापेन
चक्षुषा निरीक्षते तस्य चापि निरये पापदृष्टेरक्षिणी वज्रतुण्डा गृध्राः कङ्ककाकवटादयः प्रसह्योरु
बलादुत्पाटयन्ति
यस्त्विह वा आढ्याभिमतिरहङ्कृतिस्तिर्यक्प्रेक्षणः सर्वतोऽभिविशङ्की अर्थव्ययनाशचिन्तया
परिशुष्यमाणहृदयवदनो निर्वृतिमनवगतो ग्रह इवार्थमभिरक्षति स चापि प्रेत्य तद्‌
उत्पादनोत्कर्षणसंरक्षणशमलग्रहः सूचीमुखे नरके निपतति यत्र ह वित्तग्रहं पापपुरुषं
धर्मराजपुरुषा वायका इव सर्वतोऽङ्गेषु सूत्रैः परिवयन्ति
एवंविधा नरका यमालये सन्ति शतशः सहस्रशस्तेषु सर्वेषु च सर्व एवाधर्मवर्तिनो ये
केचिदिहोदिता अनुदिताश्चावनिपते पर्यायेण विशन्ति तथैव धर्मानुवर्तिन इतरत्र इह तु पुनर्भवे त
उभयशेषाभ्यां निविशन्ति
निवृत्तिलक्षणमार्ग आदावेव व्याख्यातः एतावानेवाण्डकोशो यश्चतुर्दशधा पुराणेषु विकल्पित
उपगीयते यत्तद्भगवतो नारायणस्य साक्षान्महापुरुषस्य स्थविष्ठं रूपमात्ममाया
गुणमयमनुवर्णितमादृतः पठति शृणोति श्रावयति स उपगेयं भगवतः परमात्मनोऽग्राह्यमपि
श्रद्धाभक्तिविशुद्धबुद्धिर्वेद
श्रुत्वा स्थूलं तथा सूक्ष्मं रूपं भगवतो यतिः
स्थूले निर्जितमात्मानं शनैः सूक्ष्मं धिया नयेदिति
भूद्वीपवर्षसरिदद्रिनभःसमुद्र
पातालदिङ्नरकभागणलोकसंस्था
गीता मया तव नृपाद्भुतमीश्वरस्य
स्थूलं वपुः सकलजीवनिकायधाम
श्रीपरीक्षिदुवाच
निवृत्तिमार्गः कथित आदौ भगवता यथा
क्रमयोगोपलब्धेन ब्रह्मणा यदसंसृतिः
प्रवृत्तिलक्षणश्चैव त्रैगुण्यविषयो मुने
योऽसावलीनप्रकृतेर्गुणसर्गः पुनः पुनः
अधर्मलक्षणा नाना नरकाश्चानुवर्णिताः
मन्वन्तरश्च व्याख्यात आद्यः स्वायम्भुवो यतः
प्रियव्रतोत्तानपदोर्वंशस्तच्चरितानि च
द्वीपवर्षसमुद्राद्रि नद्युद्यानवनस्पतीन्‌
धरामण्डलसंस्थानं भागलक्षणमानतः
ज्योतिषां विवराणां च यथेदमसृजद्विभुः
अधुनेह महाभाग यथैव नरकान्नरः
नानोग्रयातनान्नेयात्तन्मे व्याख्यातुमर्हसि
श्रीशुक उवाच
न चेदिहैवापचितिं यथांहसः कृतस्य कुर्यान्मनौक्तपाणिभिः
ध्रुवं स वै प्रेत्य नरकानुपैति ये कीर्तिता मे भवतस्तिग्मयातनाः
तस्मात्पुरैवाश्विह पापनिष्कृतौ यतेत मृत्योरविपद्यतात्मना
दोषस्य दृष्ट्वा गुरुलाघवं यथा भिषक्चिकित्सेत रुजां निदानवित्‌
श्रीराजोवाच
दृष्टश्रुताभ्यां यत्पापं जानन्नप्यात्मनोऽहितम्‌
करोति भूयो विवशः प्रायश्चित्तमथो कथम्‌
क्वचिन्निवर्ततेऽभद्रात्क्वचिच्चरति तत्पुनः
प्रायश्चित्तमथोऽपार्थं मन्ये कुञ्जरशौचवत्‌
श्रीबादरायणिरुवाच
कर्मणा कर्मनिर्हारो न ह्यात्यन्तिक इष्यते
अविद्वदधिकारित्वात्प्रायश्चित्तं विमर्शनम्‌
नाश्नतः पथ्यमेवान्नं व्याधयोऽभिभवन्ति हि
एवं नियमकृद्राजन्शनैः क्षेमाय कल्पते
तपसा ब्रह्मचर्येण शमेन च दमेन च
त्यागेन सत्यशौचाभ्यां यमेन नियमेन वा
देहवाग्बुद्धिजं धीरा धर्मज्ञाः श्रद्धयान्विताः
क्षिपन्त्यघं महदपि वेणुगुल्ममिवानलः
केचित्केवलया भक्त्या वासुदेवपरायणाः
अघं धुन्वन्ति कार्त्स्न्येन नीहारमिव भास्करः
न तथा ह्यघवान्राजन्पूयेत तपाअदिभिः
यथा कृष्णार्पितप्राणस्तत्पुरुषनिषेवया
सध्रीचीनो ह्ययं लोके पन्थाः क्षेमोऽकुतोभयः
सुशीलाः साधवो यत्र नारायणपरायणाः
प्रायश्चित्तानि चीर्णानि नारायणपराङ्मुखम्‌
न निष्पुनन्ति राजेन्द्र सुराकुम्भमिवापगाः
सकृन्मनः कृष्णपदारविन्दयोर्निवेशितं तद्गुणरागि यैरिह
न ते यमं पाशभृतश्च तद्भटान्स्वप्नेऽपि पश्यन्ति हि चीर्णनिष्कृताः
अत्र चोदाहरन्तीममितिहासं पुरातनम्‌
दूतानां विष्णुयमयोः संवादस्तं निबोध मे
कान्यकुब्जे द्विजः कश्चिद्दासीपतिरजामिलः
नाम्ना नष्टसदाचारो दास्याः संसर्गदूषितः
बन्द्यक्षैः कैतवैश्चौर्यैर्गर्हितां वृत्तिमास्थितः
बिभ्रत्कुटुम्बमशुचिर्यातयामास देहिनः
एवं निवसतस्तस्य लालयानस्य तत्सुतान्‌
कालोऽत्यगान्महान्राजन्नष्टाशीत्यायुषः समाः
तस्य प्रवयसः पुत्रा दश तेषां तु योऽवमः
बालो नारायणो नाम्ना पित्रोश्च दयितो भृशम्‌
स बद्धहृदयस्तस्मिन्नर्भके कलभाषिणि
निरीक्षमाणस्तल्लीलां मुमुदे जरठो भृशम्‌
भुञ्जानः प्रपिबन्खादन्बालकं स्नेहयन्त्रितः
भोजयन्पाययन्मूढो न वेदागतमन्तकम्‌
स एवं वर्तमानोऽज्ञो मृत्युकाल उपस्थिते
मतिं चकार तनये बाले नारायणाह्वये
स पाशहस्तांस्त्रीन्दृष्ट्वा पुरुषानतिदारुणान्‌
वक्रतुण्डानूर्ध्वरोम्ण आत्मानं नेतुमागतान्‌
दूरे क्रीडनकासक्तं पुत्रं नारायणाह्वयम्‌
प्लावितेन स्वरेणोच्चैराजुहावाकुलेन्द्रियः
निशम्य म्रियमाणस्य मुखतो हरिकीर्तनम्‌
भर्तुर्नाम महाराज पार्षदाः सहसापतन्‌
विकर्षतोऽन्तर्हृदयाद्दासीपतिमजामिलम्‌
यमप्रेष्यान्विष्णुदूता वारयामासुरोजसा
ऊचुर्निषेधितास्तांस्ते वैवस्वतपुरःसराः
के यूयं प्रतिषेद्धारो धर्मराजस्य शासनम्‌
कस्य वा कुत आयाताः कस्मादस्य निषेधथ
किं देवा उपदेवा या यूयं किं सिद्धसत्तमाः
सर्वे पद्मपलाशाक्षाः पीतकौशेयवाससः
किरीटिनः कुण्डलिनो लसत्पुष्करमालिनः
सर्वे च नूत्नवयसः सर्वे चारुचतुर्भुजाः
धनुर्निषङ्गासिगदा शङ्खचक्राम्बुजश्रियः
दिशो वितिमिरालोकाः कुर्वन्तः स्वेन तेजसा
किमर्थं धर्मपालस्य किङ्करान्नो निषेधथ
श्रीशुक उवाच
इत्युक्ते यमदूतैस्ते वासुदेवोक्तकारिणः
तान्प्रत्यूचुः प्रहस्येदं मेघनिर्ह्रादया गिरा
श्रीविष्णुदूता ऊचुः
यूयं वै धर्मराजस्य यदि निर्देशकारिणः
ब्रूत धर्मस्य नस्तत्त्वं यच्चाधर्मस्य लक्षणम्‌
कथं स्विद्ध्रियते दण्डः किं वास्य स्थानमीप्सितम्‌
दण्ड्याः किं कारिणः सर्वे आहो स्वित्कतिचिन्नृणाम्‌
यमदूता ऊचुः
वेदप्रणिहितो धर्मो ह्यधर्मस्तद्विपर्ययः
वेदो नारायणः साक्षात्स्वयम्भूरिति शुश्रुम
येन स्वधाम्न्यमी भावा रजःसत्त्वतमोमयाः
गुणनामक्रियारूपैर्विभाव्यन्ते यथातथम्‌
सूर्योऽग्निः खं मरुद्देवः सोमः सन्ध्याहनी दिशः
कं कुः स्वयं धर्म इति ह्येते दैह्यस्य साक्षिणः
एतैरधर्मो विज्ञातः स्थानं दण्डस्य युज्यते
सर्वे कर्मानुरोधेन दण्डमर्हन्ति कारिणः
सम्भवन्ति हि भद्राणि विपरीतानि चानघाः
कारिणां गुणसङ्गोऽस्ति देहवान्न ह्यकर्मकृत्‌
येन यावान्यथाधर्मो धर्मो वेह समीहितः
स एव तत्फलं भुङ्क्ते तथा तावदमुत्र वै
यथेह देवप्रवरास्त्रैविध्यमुपलभ्यते
भूतेषु गुणवैचित्र्यात्तथान्यत्रानुमीयते
वर्तमानोऽन्ययोः कालो गुणाभिज्ञापको यथा
एवं जन्मान्ययोरेतद्धर्माधर्मनिदर्शनम्‌
मनसैव पुरे देवः पूर्वरूपं विपश्यति
अनुमीमांसतेऽपूर्वं मनसा भगवानजः
यथाज्ञस्तमसा युक्त उपास्ते व्यक्तमेव हि
न वेद पूर्वमपरं नष्टजन्मस्मृतिस्तथा
पञ्चभिः कुरुते स्वार्थान्पञ्च वेदाथ पञ्चभिः
एकस्तु षोडशेन त्रीन्स्वयं सप्तदशोऽश्नुते
तदेतत्षोडशकलं लिङ्गं शक्तित्रयं महत्‌
धत्तेऽनुसंसृतिं पुंसि हर्षशोकभयार्तिदाम्‌
देह्यज्ञोऽजितषड्वर्गो नेच्छन्कर्माणि कार्यते
कोशकार इवात्मानं कर्मणाच्छाद्य मुह्यति
न हि कश्चित्क्षणमपि जातु तिष्ठत्यकर्मकृत्‌
कार्यते ह्यवशः कर्म गुणैः स्वाभाविकैर्बलात्‌
लब्ध्वा निमित्तमव्यक्तं व्यक्ताव्यक्तं भवत्युत
यथायोनि यथाबीजं स्वभावेन बलीयसा
एष प्रकृतिसङ्गेन पुरुषस्य विपर्ययः
आसीत्स एव न चिरादीशसङ्गाद्विलीयते
अयं हि श्रुतसम्पन्नः शीलवृत्तगुणालयः
धृतव्रतो मृदुर्दान्तः सत्यवाङ्मन्त्रविच्छुचिः
गुर्वग्न्यतिथिवृद्धानां शुश्रूषुरनहङ्कृतः
सर्वभूतसुहृत्साधुर्मितवागनसूयकः
एकदासौ वनं यातः पितृसन्देशकृद्द्विजः
आदाय तत आवृत्तः फलपुष्पसमित्कुशान्‌
ददर्श कामिनं कञ्चिच्छूद्रं सह भुजिष्यया
पीत्वा च मधु मैरेयं मदाघूर्णितनेत्रया
मत्तया विश्लथन्नीव्या व्यपेतं निरपत्रपम्‌
क्रीडन्तमनुगायन्तं हसन्तमनयान्तिके
दृष्ट्वा तां कामलिप्तेन बाहुना परिरम्भिताम्‌
जगाम हृच्छयवशं सहसैव विमोहितः
स्तम्भयन्नात्मनात्मानं यावत्सत्त्वं यथाश्रुतम्‌
न शशाक समाधातुं मनो मदनवेपितम्‌
तन्निमित्तस्मरव्याज ग्रहग्रस्तो विचेतनः
तामेव मनसा ध्यायन्स्वधर्माद्विरराम ह
तामेव तोषयामास पित्र्येणार्थेन यावता
ग्राम्यैर्मनोरमैः कामैः प्रसीदेत यथा तथा
विप्रां स्वभार्यामप्रौढां कुले महति लम्भिताम्‌
विससर्जाचिरात्पापः स्वैरिण्यापाङ्गविद्धधीः
यतस्ततश्चोपनिन्ये न्यायतोऽन्यायतो धनम्‌
बभारास्याः कुटुम्बिन्याः कुटुम्बं मन्दधीरयम्‌
यदसौ शास्त्रमुल्लङ्घ्य स्वैरचार्यतिगर्हितः
अवर्तत चिरं कालमघायुरशुचिर्मलात्‌
तत एनं दण्डपाणेः सकाशं कृतकिल्बिषम्‌
नेष्यामोऽकृतनिर्वेशं यत्र दण्डेन शुद्ध्यति
श्रीबादरायणिरुवाच
एवं ते भगवद्दूता यमदूताभिभाषितम्‌
उपधार्याथ तान्राजन्प्रत्याहुर्नयकोविदाः
श्रीविष्णुदूता ऊचुः
अहो कष्टं धर्मदृशामधर्मः स्पृशते सभाम्‌
यत्रादण्ड्येष्वपापेषु दण्डो यैर्ध्रियते वृथा
प्रजानां पितरो ये च शास्तारः साधवः समाः
यदि स्यात्तेषु वैषम्यं कं यान्ति शरणं प्रजाः
यद्यदाचरति श्रेयानितरस्तत्तदीहते
स यत्प्रमाणं कुरुते लोकस्तदनुवर्तते
यस्याङ्के शिर आधाय लोकः स्वपिति निर्वृतः
स्वयं धर्ममधर्मं वा न हि वेद यथा पशुः
स कथं न्यर्पितात्मानं कृतमैत्रमचेतनम्‌
विस्रम्भणीयो भूतानां सघृणो दोग्धुमर्हति
अयं हि कृतनिर्वेशो जन्मकोट्यंहसामपि
यद्व्याजहार विवशो नाम स्वस्त्ययनं हरेः
एतेनैव ह्यघोनोऽस्य कृतं स्यादघनिष्कृतम्‌
यदा नारायणायेति जगाद चतुरक्षरम्‌
स्तेनः सुरापो मित्रध्रुग्ब्रह्महा गुरुतल्पगः
स्त्रीराजपितृगोहन्ता ये च पातकिनोऽपरे
सर्वेषामप्यघवतामिदमेव सुनिष्कृतम्‌
नामव्याहरणं विष्णोर्यतस्तद्विषया मतिः
न निष्कृतैरुदितैर्ब्रह्मवादिभिस्तथा विशुद्ध्यत्यघवान्व्रतादिभिः
यथा हरेर्नामपदैरुदाहृतैस्तदुत्तमश्लोकगुणोपलम्भकम्‌
नैकान्तिकं तद्धि कृतेऽपि निष्कृते मनः पुनर्धावति चेदसत्पथे
तत्कर्मनिर्हारमभीप्सतां हरेर्गुणानुवादः खलु सत्त्वभावनः
अथैनं मापनयत कृताशेषाघनिष्कृतम्‌
यदसौ भगवन्नाम म्रियमाणः समग्रहीत्‌
साङ्केत्यं पारिहास्यं वा स्तोभं हेलनमेव वा
वैकुण्ठनामग्रहणमशेषाघहरं विदुः
पतितः स्खलितो भग्नः सन्दष्टस्तप्त आहतः
हरिरित्यवशेनाह पुमान्नार्हति यातनाः
गुरूणां च लघूनां च गुरूणि च लघूनि च
प्रायश्चित्तानि पापानां ज्ञात्वोक्तानि महर्षिभिः
तैस्तान्यघानि पूयन्ते तपोदानव्रतादिभिः
नाधर्मजं तद्धृदयं तदपीशाङ्घ्रिसेवया
अज्ञानादथवा ज्ञानादुत्तमश्लोकनाम यत्‌
सङ्कीर्तितमघं पुंसो दहेदेधो यथानलः
यथागदं वीर्यतममुपयुक्तं यदृच्छया
अजानतोऽप्यात्मगुणं कुर्यान्मन्त्रोऽप्युदाहृतः
श्रीशुक उवाच
त एवं सुविनिर्णीय धर्मं भागवतं नृप
तं याम्यपाशान्निर्मुच्य विप्रं मृत्योरमूमुचन्‌
इति प्रत्युदिता याम्या दूता यात्वा यमान्तिकम्‌
यमराज्ञे यथा सर्वमाचचक्षुररिन्दम
द्विजः पाशाद्विनिर्मुक्तो गतभीः प्रकृतिं गतः
ववन्दे शिरसा विष्णोः किङ्करान्दर्शनोत्सवः
तं विवक्षुमभिप्रेत्य महापुरुषकिङ्कराः
सहसा पश्यतस्तस्य तत्रान्तर्दधिरेऽनघ
अजामिलोऽप्यथाकर्ण्य दूतानां यमकृष्णयोः
धर्मं भागवतं शुद्धं त्रैवेद्यं च गुणाश्रयम्‌
भक्तिमान्भगवत्याशु माहात्म्यश्रवणाद्धरेः
अनुतापो महानासीत्स्मरतोऽशुभमात्मनः
अहो मे परमं कष्टमभूदविजितात्मनः
येन विप्लावितं ब्रह्म वृषल्यां जायतात्मना
धिङ्मां विगर्हितं सद्भिर्दुष्कृतं कुलकज्जलम्‌
हित्वा बालां सतीं योऽहं सुरापीमसतीमगाम्‌
वृद्धावनाथौ पितरौ नान्यबन्धू तपस्विनौ
अहो मयाधुना त्यक्तावकृतज्ञेन नीचवत्‌
सोऽहं व्यक्तं पतिष्यामि नरके भृशदारुणे
धर्मघ्नाः कामिनो यत्र विन्दन्ति यमयातनाः
किमिदं स्वप्न आहो स्वित्साक्षाद्दृष्टमिहाद्भुतम्‌
क्व याता अद्य ते ये मां व्यकर्षन्पाशपाणयः
अथ ते क्व गताः सिद्धाश्चत्वारश्चारुदर्शनाः
व्यामोचयन्नीयमानं बद्ध्वा पाशैरधो भुवः
अथापि मे दुर्भगस्य विबुधोत्तमदर्शने
भवितव्यं मङ्गलेन येनात्मा मे प्रसीदति
अन्यथा म्रियमाणस्य नाशुचेर्वृषलीपतेः
वैकुण्ठनामग्रहणं जिह्वा वक्तुमिहार्हति
क्व चाहं कितवः पापो ब्रह्मघ्नो निरपत्रपः
क्व च नारायणेत्येतद्भगवन्नाम मङ्गलम्‌
सोऽहं तथा यतिष्यामि यतचित्तेन्द्रियानिलः
यथा न भूय आत्मानमन्धे तमसि मज्जये
विमुच्य तमिमं बन्धमविद्याकामकर्मजम्‌
सर्वभूतसुहृच्छान्तो मैत्रः करुण आत्मवान्‌
मोचये ग्रस्तमात्मानं योषिन्मय्यात्ममायया
विक्रीडितो ययैवाहं क्रीडामृग इवाधमः
ममाहमिति देहादौ हित्वामिथ्यार्थधीर्मतिम्‌
धास्ये मनो भगवति शुद्धं तत्कीर्तनादिभिः
श्रीशुक उवाच
इति जातसुनिर्वेदः क्षणसङ्गेन साधुषु
गङ्गाद्वारमुपेयाय मुक्तसर्वानुबन्धनः
स तस्मिन्देवसदन आसीनो योगमास्थितः
प्रत्याहृतेन्द्रियग्रामो युयोज मन आत्मनि
ततो गुणेभ्य आत्मानं वियुज्यात्मसमाधिना
युयुजे भगवद्धाम्नि ब्रह्मण्यनुभवात्मनि
यर्ह्युपारतधीस्तस्मिन्नद्राक्षीत्पुरुषान्पुरः
उपलभ्योपलब्धान्प्राग्ववन्दे शिरसा द्विजः
हित्वा कलेवरं तीर्थे गङ्गायां दर्शनादनु
सद्यः स्वरूपं जगृहे भगवत्पार्श्ववर्तिनाम्‌
साकं विहायसा विप्रो महापुरुषकिङ्करैः
हैमं विमानमारुह्य ययौ यत्र श्रियः पतिः
एवं स विप्लावितसर्वधर्मा दास्याः पतिः पतितो गर्ह्यकर्मणा
निपात्यमानो निरये हतव्रतः सद्यो विमुक्तो भगवन्नाम गृह्णन्‌
नातः परं कर्मनिबन्धकृन्तनं मुमुक्षतां तीर्थपदानुकीर्तनात्‌
न यत्पुनः कर्मसु सज्जते मनो रजस्तमोभ्यां कलिलं ततोऽन्यथा
य एतं परमं गुह्यमितिहासमघापहम्‌
शृणुयाच्छ्रद्धया युक्तो यश्च भक्त्यानुकीर्तयेत्‌
न वै स नरकं याति नेक्षितो यमकिङ्करैः
यद्यप्यमङ्गलो मर्त्यो विष्णुलोके महीयते
म्रियमाणो हरेर्नाम गृणन्पुत्रोपचारितम्‌
अजामिलोऽप्यगाद्धाम किमुत श्रद्धया गृणन्‌
श्रीराजोवाच
निशम्य देवः स्वभटोपवर्णितं प्रत्याह किं तानपि धर्मराजः
एवं हताज्ञो विहतान्मुरारेर्नैदेशिकैर्यस्य वशे जनोऽयम्‌
यमस्य देवस्य न दण्डभङ्गः कुतश्चनर्षे श्रुतपूर्व आसीत्‌
एतन्मुने वृश्चति लोकसंशयं न हि त्वदन्य इति मे विनिश्चितम्‌
श्रीशुक उवाच
भगवत्पुरुषै राजन्याम्याः प्रतिहतोद्यमाः
पतिं विज्ञापयामासुर्यमं संयमनीपतिम्‌
यमदूता ऊचुः
कति सन्तीह शास्तारो जीवलोकस्य वै प्रभो
त्रैविध्यं कुर्वतः कर्म फलाभिव्यक्तिहेतवः
यदि स्युर्बहवो लोके शास्तारो दण्डधारिणः
कस्य स्यातां न वा कस्य मृत्युश्चामृतमेव वा
किन्तु शास्तृबहुत्वे स्याद्बहूनामिह कर्मिणाम्‌
शास्तृत्वमुपचारो हि यथा मण्डलवर्तिनाम्‌
अतस्त्वमेको भूतानां सेश्वराणामधीश्वरः
शास्ता दण्डधरो न्ऱ्‌^ईणां शुभाशुभविवेचनः
तस्य ते विहितो दण्डो न लोके वर्ततेऽधुना
चतुर्भिरद्भुतैः सिद्धैराज्ञा ते विप्रलम्भिता
नीयमानं तवादेशादस्माभिर्यातनागृहान्‌
व्यामोचयन्पातकिनं छित्त्वा पाशान्प्रसह्य ते
तांस्ते वेदितुमिच्छामो यदि नो मन्यसे क्षमम्‌
नारायणेत्यभिहिते मा भैरित्याययुर्द्रुतम्‌
श्रीबादरायणिरुवाच
इति देवः स आपृष्टः प्रजासंयमनो यमः
प्रीतः स्वदूतान्प्रत्याह स्मरन्पादाम्बुजं हरेः
यम उवाच
परो मदन्यो जगतस्तस्थुषश्च ओतं प्रोतं पटवद्यत्र विश्वम्‌
यदंशतोऽस्य स्थितिजन्मनाशा नस्योतवद्यस्य वशे च लोकः
यो नामभिर्वाचि जनं निजायां बध्नाति तन्त्र्यामिव दामभिर्गाः
यस्मै बलिं त इमे नामकर्म निबन्धबद्धाश्चकिता वहन्ति
अहं महेन्द्रो निरृतिः प्रचेताः सोमोऽग्निरीशः पवनो विरिञ्चिः
आदित्यविश्वे वसवोऽथ साध्या मरुद्गणा रुद्रगणाः ससिद्धाः
अन्ये च ये विश्वसृजोऽमरेशा भृग्वादयोऽस्पृष्टरजस्तमस्काः
यस्येहितं न विदुः स्पृष्टमायाः सत्त्वप्रधाना अपि किं ततोऽन्ये
यं वै न गोभिर्मनसासुभिर्वा हृदा गिरा वासुभृतो विचक्षते
आत्मानमन्तर्हृदि सन्तमात्मनां चक्षुर्यथैवाकृतयस्ततः परम्‌
तस्यात्मतन्त्रस्य हरेरधीशितुः परस्य मायाधिपतेर्महात्मनः
प्रायेण दूता इह वै मनोहराश्चरन्ति तद्रूपगुणस्वभावाः
भूतानि विष्णोः सुरपूजितानि दुर्दर्शलिङ्गानि महाद्भुतानि
रक्षन्ति तद्भक्तिमतः परेभ्यो मत्तश्च मर्त्यानथ सर्वतश्च
धर्मं तु साक्षाद्भगवत्प्रणीतं न वै विदुरृषयो नापि देवाः
न सिद्धमुख्या असुरा मनुष्याः कुतो नु विद्याधरचारणादयः
स्वयम्भूर्नारदः शम्भुः कुमारः कपिलो मनुः
प्रह्लादो जनको भीष्मो बलिर्वैयासकिर्वयम्‌
द्वादशैते विजानीमो धर्मं भागवतं भटाः
गुह्यं विशुद्धं दुर्बोधं यं ज्ञात्वामृतमश्नुते
एतावानेव लोकेऽस्मिन्पुंसां धर्मः परः स्मृतः
भक्तियोगो भगवति तन्नामग्रहणादिभिः
नामोच्चारणमाहात्म्यं हरेः पश्यत पुत्रकाः
अजामिलोऽपि येनैव मृत्युपाशादमुच्यत
एतावतालमघनिर्हरणाय पुंसां
सङ्कीर्तनं भगवतो गुणकर्मनाम्नाम्‌
विक्रुश्य पुत्रमघवान्यदजामिलोऽपि
नारायणेति म्रियमाण इयाय मुक्तिम्‌
प्रायेण वेद तदिदं न महाजनोऽयं
देव्या विमोहितमतिर्बत माययालम्‌
त्रय्यां जडीकृतमतिर्मधुपुष्पितायां
वैतानिके महति कर्मणि युज्यमानः
एवं विमृश्य सुधियो भगवत्यनन्ते
सर्वात्मना विदधते खलु भावयोगम्‌
ते मे न दण्डमर्हन्त्यथ यद्यमीषां
स्यात्पातकं तदपि हन्त्युरुगायवादः
ते देवसिद्धपरिगीतपवित्रगाथा
ये साधवः समदृशो भगवत्प्रपन्नाः
तान्नोपसीदत हरेर्गदयाभिगुप्तान्‌
नैषां वयं न च वयः प्रभवाम दण्डे
तानानयध्वमसतो विमुखान्मुकुन्द
पादारविन्दमकरन्दरसादजस्रम्‌
निष्किञ्चनैः परमहंसकुलैरसङ्गैर्‌
जुष्टाद्गृहे निरयवर्त्मनि बद्धतृष्णान्‌
जिह्वा न वक्ति भगवद्गुणनामधेयं
चेतश्च न स्मरति तच्चरणारविन्दम्‌
कृष्णाय नो नमति यच्छिर एकदापि
तानानयध्वमसतोऽकृतविष्णुकृत्यान्‌
तत्क्षम्यतां स भगवान्पुरुषः पुराणो
नारायणः स्वपुरुषैर्यदसत्कृतं नः
स्वानामहो न विदुषां रचिताञ्जलीनां
क्षान्तिर्गरीयसि नमः पुरुषाय भूम्ने
तस्मात्सङ्कीर्तनं विष्णोर्जगन्मङ्गलमंहसाम्‌
महतामपि कौरव्य विद्ध्यैकान्तिकनिष्कृतम्‌
शृण्वतां गृणतां वीर्याण्युद्दामानि हरेर्मुहुः
यथा सुजातया भक्त्या शुद्ध्येन्नात्मा व्रतादिभिः
कृष्णाङ्घ्रिपद्ममधुलिण्न पुनर्विसृष्ट
मायागुणेषु रमते वृजिनावहेषु
अन्यस्तु कामहत आत्मरजः प्रमार्ष्टुम्‌
ईहेत कर्म यत एव रजः पुनः स्यात्‌
इत्थं स्वभर्तृगदितं भगवन्महित्वं
संस्मृत्य विस्मितधियो यमकिङ्करास्ते
नैवाच्युताश्रयजनं प्रतिशङ्कमाना
द्रष्टुं च बिभ्यति ततः प्रभृति स्म राजन्‌
इतिहासमिमं गुह्यं भगवान्कुम्भसम्भवः
कथयामास मलय आसीनो हरिमर्चयन्‌
श्रीराजोवाच
देवासुरनृणां सर्गो नागानां मृगपक्षिणाम्‌
सामासिकस्त्वया प्रोक्तो यस्तु स्वायम्भुवेऽन्तरे
तस्यैव व्यासमिच्छामि ज्ञातुं ते भगवन्यथा
अनुसर्गं यया शक्त्या ससर्ज भगवान्परः
श्रीसूत उवाच
इति सम्प्रश्नमाकर्ण्य राजर्षेर्बादरायणिः
प्रतिनन्द्य महायोगी जगाद मुनिसत्तमाः
श्रीशुक उवाच
यदा प्रचेतसः पुत्रा दश प्राचीनबर्हिषः
अन्तःसमुद्रादुन्मग्ना ददृशुर्गां द्रुमैर्वृताम्‌
द्रुमेभ्यः क्रुध्यमानास्ते तपोदीपितमन्यवः
मुखतो वायुमग्निं च ससृजुस्तद्दिधक्षया
ताभ्यां निर्दह्यमानांस्तानुपलभ्य कुरूद्वह
राजोवाच महान्सोमो मन्युं प्रशमयन्निव
न द्रुमेभ्यो महाभागा दीनेभ्यो द्रोग्धुमर्हथ
विवर्धयिषवो यूयं प्रजानां पतयः स्मृताः
अहो प्रजापतिपतिर्भगवान्हरिरव्ययः
वनस्पतीनोषधीश्च ससर्जोर्जमिषं विभुः
अन्नं चराणामचरा ह्यपदः पादचारिणाम्‌
अहस्ता हस्तयुक्तानां द्विपदां च चतुष्पदः
यूयं च पित्रान्वादिष्टा देवदेवेन चानघाः
प्रजासर्गाय हि कथं वृक्षान्निर्दग्धुमर्हथ
आतिष्ठत सतां मार्गं कोपं यच्छत दीपितम्‌
पित्रा पितामहेनापि जुष्टं वः प्रपितामहैः
तोकानां पितरौ बन्धू दृशः पक्ष्म स्त्रियाः पतिः
पतिः प्रजानां भिक्षूणां गृह्यज्ञानां बुधः सुहृत्‌
अन्तर्देहेषु भूतानामात्मास्ते हरिरीश्वरः
सर्वं तद्धिष्ण्यमीक्षध्वमेवं वस्तोषितो ह्यसौ
यः समुत्पतितं देह आकाशान्मन्युमुल्बणम्‌
आत्मजिज्ञासया यच्छेत्स गुणानतिवर्तते
अलं दग्धैर्द्रुमैर्दीनैः खिलानां शिवमस्तु वः
वार्क्षी ह्येषा वरा कन्या पत्नीत्वे प्रतिगृह्यताम्‌
इत्यामन्त्र्य वरारोहां कन्यामाप्सरसीं नृप
सोमो राजा ययौ दत्त्वा ते धर्मेणोपयेमिरे
तेभ्यस्तस्यां समभवद्दक्षः प्राचेतसः किल
यस्य प्रजाविसर्गेण लोका आपूरितास्त्रयः
यथा ससर्ज भूतानि दक्षो दुहितृवत्सलः
रेतसा मनसा चैव तन्ममावहितः शृणु
मनसैवासृजत्पूर्वं प्रजापतिरिमाः प्रजाः
देवासुरमनुष्यादीन्नभःस्थलजलौकसः
तमबृंहितमालोक्य प्रजासर्गं प्रजापतिः
विन्ध्यपादानुपव्रज्य सोऽचरद्दुष्करं तपः
तत्राघमर्षणं नाम तीर्थं पापहरं परम्‌
उपस्पृश्यानुसवनं तपसातोषयद्धरिम्‌
अस्तौषीद्धंसगुह्येन भगवन्तमधोक्षजम्‌
तुभ्यं तदभिधास्यामि कस्यातुष्यद्यथा हरिः
श्रीप्रजापतिरुवाच
नमः परायावितथानुभूतये गुणत्रयाभासनिमित्तबन्धवे
अदृष्टधाम्ने गुणतत्त्वबुद्धिभिर्निवृत्तमानाय दधे स्वयम्भुवे
न यस्य सख्यं पुरुषोऽवैति सख्युः सखा वसन्संवसतः पुरेऽस्मिन्‌
गुणो यथा गुणिनो व्यक्तदृष्टेस्तस्मै महेशाय नमस्करोमि
देहोऽसवोऽक्षा मनवो भूतमात्रामात्मानमन्यं च विदुः परं यत्‌
सर्वं पुमान्वेद गुणांश्च तज्ज्ञो न वेद सर्वज्ञमनन्तमीडे
यदोपरामो मनसो नामरूप रूपस्य दृष्टस्मृतिसम्प्रमोषात्‌
य ईयते केवलया स्वसंस्थया हंसाय तस्मै शुचिसद्मने नमः
मनीषिणोऽन्तर्हृदि सन्निवेशितं स्वशक्तिभिर्नवभिश्च त्रिवृद्भिः
वह्निं यथा दारुणि पाञ्चदश्यं मनीषया निष्कर्षन्ति गूढम्‌
स वै ममाशेषविशेषमाया निषेधनिर्वाणसुखानुभूतिः
स सर्वनामा स च विश्वरूपः प्रसीदतामनिरुक्तात्मशक्तिः
यद्यन्निरुक्तं वचसा निरूपितं धियाक्षभिर्वा मनसोत यस्य
मा भूत्स्वरूपं गुणरूपं हि तत्तत्स वै गुणापायविसर्गलक्षणः
यस्मिन्यतो येन च यस्य यस्मै यद्यो यथा कुरुते कार्यते च
परावरेषां परमं प्राक्प्रसिद्धं तद्ब्रह्म तद्धेतुरनन्यदेकम्‌
यच्छक्तयो वदतां वादिनां वै विवादसंवादभुवो भवन्ति
कुर्वन्ति चैषां मुहुरात्ममोहं तस्मै नमोऽनन्तगुणाय भूम्ने
अस्तीति नास्तीति च वस्तुनिष्ठयोरेकस्थयोर्भिन्नविरुद्धधर्मणोः
अवेक्षितं किञ्चन योगसाङ्ख्ययोः समं परं ह्यनुकूलं बृहत्तत्‌
योऽनुग्रहार्थं भजतां पादमूलमनामरूपो भगवाननन्तः
नामानि रूपाणि च जन्मकर्मभिर्भेजे स मह्यं परमः प्रसीदतु
यः प्राकृतैर्ज्ञानपथैर्जनानां यथाशयं देहगतो विभाति
यथानिलः पार्थिवमाश्रितो गुणं स ईश्वरो मे कुरुतां मनोरथम्‌
श्रीशुक उवाच
इति स्तुतः संस्तुवतः स तस्मिन्नघमर्षणे
प्रादुरासीत्कुरुश्रेष्ठ भगवान्भक्तवत्सलः
कृतपादः सुपर्णांसे प्रलम्बाष्टमहाभुजः
चक्रशङ्खासिचर्मेषु धनुःपाशगदाधरः
पीतवासा घनश्यामः प्रसन्नवदनेक्षणः
वनमालानिवीताङ्गो लसच्छ्रीवत्सकौस्तुभः
महाकिरीटकटकः स्फुरन्मकरकुण्डलः
काञ्च्यङ्गुलीयवलय नूपुराङ्गदभूषितः
त्रैलोक्यमोहनं रूपं बिभ्रत्त्रिभुवनेश्वरः
वृतो नारदनन्दाद्यैः पार्षदैः सुरयूथपैः
स्तूयमानोऽनुगायद्भिः सिद्धगन्धर्वचारणैः
रूपं तन्महदाश्चर्यं विचक्ष्यागतसाध्वसः
ननाम दण्डवद्भूमौ प्रहृष्टात्मा प्रजापतिः
न किञ्चनोदीरयितुमशकत्तीव्रया मुदा
आपूरितमनोद्वारैर्ह्रदिन्य इव निर्झरैः
तं तथावनतं भक्तं प्रजाकामं प्रजापतिम्‌
चित्तज्ञः सर्वभूतानामिदमाह जनार्दनः
श्रीभगवानुवाच
प्राचेतस महाभाग संसिद्धस्तपसा भवान्‌
यच्छ्रद्धया मत्परया मयि भावं परं गतः
प्रीतोऽहं ते प्रजानाथ यत्तेऽस्योद्बृंहणं तपः
ममैष कामो भूतानां यद्भूयासुर्विभूतयः
ब्रह्मा भवो भवन्तश्च मनवो विबुधेश्वराः
विभूतयो मम ह्येता भूतानां भूतिहेतवः
तपो मे हृदयं ब्रह्मंस्तनुर्विद्या क्रियाकृतिः
अङ्गानि क्रतवो जाता धर्म आत्मासवः सुराः
अहमेवासमेवाग्रे नान्यत्किञ्चान्तरं बहिः
संज्ञानमात्रमव्यक्तं प्रसुप्तमिव विश्वतः
मय्यनन्तगुणेऽनन्ते गुणतो गुणविग्रहः
यदासीत्तत एवाद्यः स्वयम्भूः समभूदजः
स वै यदा महादेवो मम वीर्योपबृंहितः
मेने खिलमिवात्मानमुद्यतः स्वर्गकर्मणि
अथ मेऽभिहितो देवस्तपोऽतप्यत दारुणम्‌
नव विश्वसृजो युष्मान्येनादावसृजद्विभुः
एषा पञ्चजनस्याङ्ग दुहिता वै प्रजापतेः
असिक्नी नाम पत्नीत्वे प्रजेश प्रतिगृह्यताम्‌
मिथुनव्यवायधर्मस्त्वं प्रजासर्गमिमं पुनः
मिथुनव्यवायधर्मिण्यां भूरिशो भावयिष्यसि
त्वत्तोऽधस्तात्प्रजाः सर्वा मिथुनीभूय मायया
मदीयया भविष्यन्ति हरिष्यन्ति च मे बलिम्‌
श्रीशुक उवाच
इत्युक्त्वा मिषतस्तस्य भगवान्विश्वभावनः
स्वप्नोपलब्धार्थ इव तत्रैवान्तर्दधे हरिः
श्रीशुक उवाच
तस्यां स पाञ्चजन्यां वै विष्णुमायोपबृंहितः
हर्यश्वसंज्ञानयुतं पुत्रानजनयद्विभुः
अपृथग्धर्मशीलास्ते सर्वे दाक्षायणा नृप
पित्रा प्रोक्ताः प्रजासर्गे प्रतीचीं प्रययुर्दिशम्‌
तत्र नारायणसरस्तीर्थं सिन्धुसमुद्रयोः
सङ्गमो यत्र सुमहन्मुनिसिद्धनिषेवितम्‌
तदुपस्पर्शनादेव विनिर्धूतमलाशयाः
धर्मे पारमहंस्ये च प्रोत्पन्नमतयोऽप्युत
तेपिरे तप एवोग्रं पित्रादेशेन यन्त्रिताः
प्रजाविवृद्धये यत्तान्देवर्षिस्तान्ददर्श ह
उवाच चाथ हर्यश्वाः कथं स्रक्ष्यथ वै प्रजाः
अदृष्ट्वान्तं भुवो यूयं बालिशा बत पालकाः
तथैकपुरुषं राष्ट्रं बिलं चादृष्टनिर्गमम्‌
बहुरूपां स्त्रियं चापि पुमांसं पुंश्चलीपतिम्‌
नदीमुभयतो वाहां पञ्चपञ्चाद्भुतं गृहम्‌
क्वचिद्धंसं चित्रकथं क्षौरपव्यं स्वयं भ्रमि
कथं स्वपितुरादेशमविद्वांसो विपश्चितः
अनुरूपमविज्ञाय अहो सर्गं करिष्यथ
श्रीशुक उवाच
तन्निशम्याथ हर्यश्वा औत्पत्तिकमनीषया
वाचः कूटं तु देवर्षेः स्वयं विममृशुर्धिया
भूः क्षेत्रं जीवसंज्ञं यदनादि निजबन्धनम्‌
अदृष्ट्वा तस्य निर्वाणं किमसत्कर्मभिर्भवेत्‌
एक एवेश्वरस्तुर्यो भगवान्स्वाश्रयः परः
तमदृष्ट्वाभवं पुंसः किमसत्कर्मभिर्भवेत्‌
पुमान्नैवैति यद्गत्वा बिलस्वर्गं गतो यथा
प्रत्यग्धामाविद इह किमसत्कर्मभिर्भवेत्‌
नानारूपात्मनो बुद्धिः स्वैरिणीव गुणान्विता
तन्निष्ठामगतस्येह किमसत्कर्मभिर्भवेत्‌
तत्सङ्गभ्रंशितैश्वर्यं संसरन्तं कुभार्यवत्‌
तद्गतीरबुधस्येह किमसत्कर्मभिर्भवेत्‌
सृष्ट्यप्ययकरीं मायां वेलाकूलान्तवेगिताम्‌
मत्तस्य तामविज्ञस्य किमसत्कर्मभिर्भवेत्‌
पञ्चविंशतितत्त्वानां पुरुषोऽद्भुतदर्पणः
अध्यात्ममबुधस्येह किमसत्कर्मभिर्भवेत्‌
ऐश्वरं शास्त्रमुत्सृज्य बन्धमोक्षानुदर्शनम्‌
विविक्तपदमज्ञाय किमसत्कर्मभिर्भवेत्‌
कालचक्रं भ्रमि तीक्ष्णं सर्वं निष्कर्षयज्जगत्‌
स्वतन्त्रमबुधस्येह किमसत्कर्मभिर्भवेत्‌
शास्त्रस्य पितुरादेशं यो न वेद निवर्तकम्‌
कथं तदनुरूपाय गुणविस्रम्भ्युपक्रमेत्‌
इति व्यवसिता राजन्हर्यश्वा एकचेतसः
प्रययुस्तं परिक्रम्य पन्थानमनिवर्तनम्‌
स्वरब्रह्मणि निर्भात हृषीकेशपदाम्बुजे
अखण्डं चित्तमावेश्य लोकाननुचरन्मुनिः
नाशं निशम्य पुत्राणां नारदाच्छीलशालिनाम्‌
अन्वतप्यत कः शोचन्सुप्रजस्त्वं शुचां पदम्‌
स भूयः पाञ्चजन्यायामजेन परिसान्त्वितः
पुत्रानजनयद्दक्षः सवलाश्वान्सहस्रिणः
ते च पित्रा समादिष्टाः प्रजासर्गे धृतव्रताः
नारायणसरो जग्मुर्यत्र सिद्धाः स्वपूर्वजाः
तदुपस्पर्शनादेव विनिर्धूतमलाशयाः
जपन्तो ब्रह्म परमं तेपुस्तत्र महत्तपः
अब्भक्षाः कतिचिन्मासान्कतिचिद्वायुभोजनाः
आराधयन्मन्त्रमिममभ्यस्यन्त इडस्पतिम्‌
ओं नमो नारायणाय पुरुषाय महात्मने
विशुद्धसत्त्वधिष्ण्याय महाहंसाय धीमहि
इति तानपि राजेन्द्र प्रजासर्गधियो मुनिः
उपेत्य नारदः प्राह वाचः कूटानि पूर्ववत्‌
दाक्षायणाः संशृणुत गदतो निगमं मम
अन्विच्छतानुपदवीं भ्रात्ऱ्‌^ईणां भ्रातृवत्सलाः
भ्रात्ऱ्‌^ईणां प्रायणं भ्राता योऽनुतिष्ठति धर्मवित्‌
स पुण्यबन्धुः पुरुषो मरुद्भिः सह मोदते
एतावदुक्त्वा प्रययौ नारदोऽमोघदर्शनः
तेऽपि चान्वगमन्मार्गं भ्रात्ऱ्‌^ईणामेव मारिष
सध्रीचीनं प्रतीचीनं परस्यानुपथं गताः
नाद्यापि ते निवर्तन्ते पश्चिमा यामिनीरिव
एतस्मिन्काल उत्पातान्बहून्पश्यन्प्रजापतिः
पूर्ववन्नारदकृतं पुत्रनाशमुपाशृणोत्‌
चुक्रोध नारदायासौ पुत्रशोकविमूर्च्छितः
देवर्षिमुपलभ्याह रोषाद्विस्फुरिताधरः
श्रीदक्ष उवाच
अहो असाधो साधूनां साधुलिङ्गेन नस्त्वया
असाध्वकार्यर्भकाणां भिक्षोर्मार्गः प्रदर्शितः
ऋणैस्त्रिभिरमुक्तानाममीमांसितकर्मणाम्‌
विघातः श्रेयसः पाप लोकयोरुभयोः कृतः
एवं त्वं निरनुक्रोशो बालानां मतिभिद्धरेः
पार्षदमध्ये चरसि यशोहा निरपत्रपः
ननु भागवता नित्यं भूतानुग्रहकातराः
ऋते त्वां सौहृदघ्नं वै वैरङ्करमवैरिणाम्‌
नेत्थं पुंसां विरागः स्यात्त्वया केवलिना मृषा
मन्यसे यद्युपशमं स्नेहपाशनिकृन्तनम्‌
नानुभूय न जानाति पुमान्विषयतीक्ष्णताम्‌
निर्विद्यते स्वयं तस्मान्न तथा भिन्नधीः परैः
यन्नस्त्वं कर्मसन्धानां साधूनां गृहमेधिनाम्‌
कृतवानसि दुर्मर्षं विप्रियं तव मर्षितम्‌
तन्तुकृन्तन यन्नस्त्वमभद्रमचरः पुनः
तस्माल्लोकेषु ते मूढ न भवेद्भ्रमतः पदम्‌
श्रीशुक उवाच
प्रतिजग्राह तद्बाढं नारदः साधुसम्मतः
एतावान्साधुवादो हि तितिक्षेतेश्वरः स्वयम्‌
श्रीशुक उवाच
ततः प्राचेतसोऽसिक्न्यामनुनीतः स्वयम्भुवा
षष्टिं सञ्जनयामास दुहित्ऱ्‌^ईः पितृवत्सलाः
दश धर्माय कायादाद्द्विषट्त्रिणव चेन्दवे
भूताङ्गिरःकृशाश्वेभ्यो द्वे द्वे तार्क्ष्याय चापराः
नामधेयान्यमूषां त्वं सापत्यानां च मे शृणु
यासां प्रसूतिप्रसवैर्लोका आपूरितास्त्रयः
भानुर्लम्बा ककुद्यामिर्विश्वा साध्या मरुत्वती
वसुर्मुहूर्ता सङ्कल्पा धर्मपत्न्यः सुताञ्शृणु
भानोस्तु देवऋषभ इन्द्रसेनस्ततो नृप
विद्योत आसील्लम्बायास्ततश्च स्तनयित्नवः
ककुदः सङ्कटस्तस्य कीकटस्तनयो यतः
भुवो दुर्गाणि यामेयः स्वर्गो नन्दिस्ततोऽभवत्‌
विश्वेदेवास्तु विश्वाया अप्रजांस्तान्प्रचक्षते
साध्योगणश्च साध्याया अर्थसिद्धिस्तु तत्सुतः
मरुत्वांश्च जयन्तश्च मरुत्वत्या बभूवतुः
जयन्तो वासुदेवांश उपेन्द्र इति यं विदुः
मौहूर्तिका देवगणा मुहूर्तायाश्च जज्ञिरे
ये वै फलं प्रयच्छन्ति भूतानां स्वस्वकालजम्‌
सङ्कल्पायास्तु सङ्कल्पः कामः सङ्कल्पजः स्मृतः
वसवोऽष्टौ वसोः पुत्रास्तेषां नामानि मे शृणु
द्रोणः प्राणो ध्रुवोऽर्कोऽग्निर्दोषो वास्तुर्विभावसुः
द्रोणस्याभिमतेः पत्न्या हर्षशोकभयादयः
प्राणस्योर्जस्वती भार्या सह आयुः पुरोजवः
ध्रुवस्य भार्या धरणिरसूत विविधाः पुरः
अर्कस्य वासना भार्या पुत्रास्तर्षादयः स्मृताः
अग्नेर्भार्या वसोर्धारा पुत्रा द्रविणकादयः
स्कन्दश्च कृत्तिकापुत्रो ये विशाखादयस्ततः
दोषस्य शर्वरीपुत्रः शिशुमारो हरेः कला
वास्तोराङ्गिरसीपुत्रो विश्वकर्माकृतीपतिः
ततो मनुश्चाक्षुषोऽभूद्विश्वे साध्या मनोः सुताः
विभावसोरसूतोषा व्युष्टं रोचिषमातपम्‌
पञ्चयामोऽथ भूतानि येन जाग्रति कर्मसु
सरूपासूत भूतस्य भार्या रुद्रांश्च कोटिशः
रैवतोऽजो भवो भीमो वाम उग्रो वृषाकपिः
अजैकपादहिर्ब्रध्नो बहुरूपो महानिति
रुद्रस्य पार्षदाश्चान्ये घोराः प्रेतविनायकाः
प्रजापतेरङ्गिरसः स्वधा पत्नी पित्ऱ्‌^ईनथ
अथर्वाङ्गिरसं वेदं पुत्रत्वे चाकरोत्सती
कृशाश्वोऽर्चिषि भार्यायां धूमकेतुमजीजनत्‌
धिषणायां वेदशिरो देवलं वयुनं मनुम्‌
तार्क्ष्यस्य विनता कद्रूः पतङ्गी यामिनीति च
पतङ्ग्यसूत पतगान्यामिनी शलभानथ
सुपर्णासूत गरुडं साक्षाद्यज्ञेशवाहनम्‌
सूर्यसूतमनूरुं च कद्रूर्नागाननेकशः
कृत्तिकादीनि नक्षत्राणीन्दोः पत्न्यस्तु भारत
दक्षशापात्सोऽनपत्यस्तासु यक्ष्मग्रहार्दितः
पुनः प्रसाद्य तं सोमः कला लेभे क्षये दिताः
शृणु नामानि लोकानां मात्ऱ्‌^ईणां शङ्कराणि च
अथ कश्यपपत्नीनां यत्प्रसूतमिदं जगत्‌
अदितिर्दितिर्दनुः काष्ठा अरिष्टा सुरसा इला
मुनिः क्रोधवशा ताम्रा सुरभिः सरमा तिमिः
तिमेर्यादोगणा आसन्श्वापदाः सरमासुताः
सुरभेर्महिषा गावो ये चान्ये द्विशफा नृप
ताम्रायाः श्येनगृध्राद्या मुनेरप्सरसां गणाः
दन्दशूकादयः सर्पा राजन्क्रोधवशात्मजाः
इलाया भूरुहाः सर्वे यातुधानाश्च सौरसाः
अरिष्टायास्तु गन्धर्वाः काष्ठाया द्विशफेतराः
सुता दनोरेकषष्टिस्तेषां प्राधानिकाञ्शृणु
द्विमूर्धा शम्बरोऽरिष्टो हयग्रीवो विभावसुः
अयोमुखः शङ्कुशिराः स्वर्भानुः कपिलोऽरुणः
पुलोमा वृषपर्वा च एकचक्रोऽनुतापनः
धूम्रकेशो विरूपाक्षो विप्रचित्तिश्च दुर्जयः
स्वर्भानोः सुप्रभां कन्यामुवाह नमुचिः किल
वृषपर्वणस्तु शर्मिष्ठां ययातिर्नाहुषो बली
वैश्वानरसुता याश्च चतस्रश्चारुदर्शनाः
उपदानवी हयशिरा पुलोमा कालका तथा
उपदानवीं हिरण्याक्षः क्रतुर्हयशिरां नृप
पुलोमां कालकां च द्वे वैश्वानरसुते तु कः
उपयेमेऽथ भगवान्कश्यपो ब्रह्मचोदितः
पौलोमाः कालकेयाश्च दानवा युद्धशालिनः
तयोः षष्टिसहस्राणि यज्ञघ्नांस्ते पितुः पिता
जघान स्वर्गतो राजन्नेक इन्द्रप्रियङ्करः
विप्रचित्तिः सिंहिकायां शतं चैकमजीजनत्‌
राहुज्येष्ठं केतुशतं ग्रहत्वं य उपागताः
अथातः श्रूयतां वंशो योऽदितेरनुपूर्वशः
यत्र नारायणो देवः स्वांशेनावातरद्विभुः
विवस्वानर्यमा पूषा त्वष्टाथ सविता भगः
धाता विधाता वरुणो मित्रः शत्रु उरुक्रमः
विवस्वतः श्राद्धदेवं संज्ञासूयत वै मनुम्‌
मिथुनं च महाभागा यमं देवं यमीं तथा
सैव भूत्वाथ वडवा नासत्यौ सुषुवे भुवि
छाया शनैश्चरं लेभे सावर्णिं च मनुं ततः
कन्यां च तपतीं या वै वव्रे संवरणं पतिम्‌
अर्यम्णो मातृका पत्नी तयोश्चर्षणयः सुताः
यत्र वै मानुषी जातिर्ब्रह्मणा चोपकल्पिता
पूषानपत्यः पिष्टादो भग्नदन्तोऽभवत्पुरा
योऽसौ दक्षाय कुपितं जहास विवृतद्विजः
त्वष्टुर्दैत्यात्मजा भार्या रचना नाम कन्यका
सन्निवेशस्तयोर्जज्ञे विश्वरूपश्च वीर्यवान्‌
तं वव्रिरे सुरगणा स्वस्रीयं द्विषतामपि
विमतेन परित्यक्ता गुरुणाङ्गिरसेन यत्‌
श्रीराजोवाच
कस्य हेतोः परित्यक्ता आचार्येणात्मनः सुराः
एतदाचक्ष्व भगवञ्छिष्याणामक्रमं गुरौ
श्रीबादरायणिरुवाच
इन्द्रस्त्रिभुवनैश्वर्य मदोल्लङ्घितसत्पथः
मरुद्भिर्वसुभी रुद्रैरादित्यैरृभुभिर्नृप
विश्वेदेवैश्च साध्यैश्च नासत्याभ्यां परिश्रितः
सिद्धचारणगन्धर्वैर्मुनिभिर्ब्रह्मवादिभिः
विद्याधराप्सरोभिश्च किन्नरैः पतगोरगैः
निषेव्यमाणो मघवान्स्तूयमानश्च भारत
उपगीयमानो ललितमास्थानाध्यासनाश्रितः
पाण्डुरेणातपत्रेण चन्द्रमण्डलचारुणा
युक्तश्चान्यैः पारमेष्ठ्यैश्चामरव्यजनादिभिः
विराजमानः पौलम्या सहार्धासनया भृशम्‌
स यदा परमाचार्यं देवानामात्मनश्च ह
नाभ्यनन्दत सम्प्राप्तं प्रत्युत्थानासनादिभिः
वाचस्पतिं मुनिवरं सुरासुरनमस्कृतम्‌
नोच्चचालासनादिन्द्रः पश्यन्नपि सभागतम्‌
ततो निर्गत्य सहसा कविराङ्गिरसः प्रभुः
आययौ स्वगृहं तूष्णीं विद्वान्श्रीमदविक्रियाम्‌
तर्ह्येव प्रतिबुध्येन्द्रो गुरुहेलनमात्मनः
गर्हयामास सदसि स्वयमात्मानमात्मना
अहो बत मयासाधु कृतं वै दभ्रबुद्धिना
यन्मयैश्वर्यमत्तेन गुरुः सदसि कात्कृतः
को गृध्येत्पण्डितो लक्ष्मीं त्रिपिष्टपपतेरपि
ययाहमासुरं भावं नीतोऽद्य विबुधेश्वरः
यः पारमेष्ठ्यं धिषणमधितिष्ठन्न कञ्चन
प्रत्युत्तिष्ठेदिति ब्रूयुर्धर्मं ते न परं विदुः
तेषां कुपथदेष्ट्ऱ्‌^ईणां पततां तमसि ह्यधः
ये श्रद्दध्युर्वचस्ते वै मज्जन्त्यश्मप्लवा इव
अथाहममराचार्यमगाधधिषणं द्विजम्‌
प्रसादयिष्ये निशठः शीर्ष्णा तच्चरणं स्पृशन्‌
एवं चिन्तयतस्तस्य मघोनो भगवान्गृहात्‌
बृहस्पतिर्गतोऽदृष्टां गतिमध्यात्ममायया
गुरोर्नाधिगतः संज्ञां परीक्षन्भगवान्स्वराट्‌
ध्यायन्धिया सुरैर्युक्तः शर्म नालभतात्मनः
तच्छ्रुत्वैवासुराः सर्व आश्रित्यौशनसं मतम्‌
देवान्प्रत्युद्यमं चक्रुर्दुर्मदा आततायिनः
तैर्विसृष्टेषुभिस्तीक्ष्णैर्निर्भिन्नाङ्गोरुबाहवः
ब्रह्माणं शरणं जग्मुः सहेन्द्रा नतकन्धराः
तांस्तथाभ्यर्दितान्वीक्ष्य भगवानात्मभूरजः
कृपया परया देव उवाच परिसान्त्वयन्‌
श्रीब्रह्मोवाच
अहो बत सुरश्रेष्ठा ह्यभद्रं वः कृतं महत्‌
ब्रह्मिष्ठं ब्राह्मणं दान्तमैश्वर्यान्नाभ्यनन्दत
तस्यायमनयस्यासीत्परेभ्यो वः पराभवः
प्रक्षीणेभ्यः स्ववैरिभ्यः समृद्धानां च यत्सुराः
मघवन्द्विषतः पश्य प्रक्षीणान्गुर्वतिक्रमात्‌
सम्प्रत्युपचितान्भूयः काव्यमाराध्य भक्तितः
आददीरन्निलयनं ममापि भृगुदेवताः
त्रिपिष्टपं किं गणयन्त्यभेद्य मन्त्रा भृगूणामनुशिक्षितार्थाः
न विप्रगोविन्दगवीश्वराणां भवन्त्यभद्राणि नरेश्वराणाम्‌
तद्विश्वरूपं भजताशु विप्रं तपस्विनं त्वाष्ट्रमथात्मवन्तम्‌
सभाजितोऽर्थान्स विधास्यते वो यदि क्षमिष्यध्वमुतास्य कर्म
श्रीशुक उवाच
त एवमुदिता राजन्ब्रह्मणा विगतज्वराः
ऋषिं त्वाष्ट्रमुपव्रज्य परिष्वज्येदमब्रुवन्‌
श्रीदेवा ऊचुः
वयं तेऽतिथयः प्राप्ता आश्रमं भद्रमस्तु ते
कामः सम्पाद्यतां तात पित्ऱ्‌^ईणां समयोचितः
पुत्राणां हि परो धर्मः पितृशुश्रूषणं सताम्‌
अपि पुत्रवतां ब्रह्मन्किमुत ब्रह्मचारिणाम्‌
आचार्यो ब्रह्मणो मूर्तिः पिता मूर्तिः प्रजापतेः
भ्राता मरुत्पतेर्मूर्तिर्माता साक्षात्क्षितेस्तनुः
दयाया भगिनी मूर्तिर्धर्मस्यात्मातिथिः स्वयम्‌
अग्नेरभ्यागतो मूर्तिः सर्वभूतानि चात्मनः
तस्मात्पित्ऱ्‌^ईणामार्तानामार्तिं परपराभवम्‌
तपसापनयंस्तात सन्देशं कर्तुमर्हसि
वृणीमहे त्वोपाध्यायं ब्रह्मिष्ठं ब्राह्मणं गुरुम्‌
यथाञ्जसा विजेष्यामः सपत्नांस्तव तेजसा
न गर्हयन्ति ह्यर्थेषु यविष्ठाङ्घ्र्यभिवादनम्‌
छन्दोभ्योऽन्यत्र न ब्रह्मन्वयो ज्यैष्ठ्यस्य कारणम्‌
श्रीऋषिरुवाच
अभ्यर्थितः सुरगणैः पौरहित्ये महातपाः
स विश्वरूपस्तानाह प्रसन्नः श्लक्ष्णया गिरा
श्रीविश्वरूप उवाच
विगर्हितं धर्मशीलैर्ब्रह्मवर्चौपव्ययम्‌
कथं नु मद्विधो नाथा लोकेशैरभियाचितम्‌
प्रत्याख्यास्यति तच्छिष्यः स एव स्वार्थ उच्यते
अकिञ्चनानां हि धनं शिलोञ्छनं तेनेह निर्वर्तितसाधुसत्क्रियः
कथं विगर्ह्यं नु करोम्यधीश्वराः पौरोधसं हृष्यति येन दुर्मतिः
तथापि न प्रतिब्रूयां गुरुभिः प्रार्थितं कियत्‌
भवतां प्रार्थितं सर्वं प्राणैरर्थैश्च साधये
श्रीबादरायणिरुवाच
तेभ्य एवं प्रतिश्रुत्य विश्वरूपो महातपाः
पौरहित्यं वृतश्चक्रे परमेण समाधिना
सुरद्विषां श्रियं गुप्तामौशनस्यापि विद्यया
आच्छिद्यादान्महेन्द्राय वैष्णव्या विद्यया विभुः
यया गुप्तः सहस्राक्षो जिग्येऽसुरचमूर्विभुः
तां प्राह स महेन्द्राय विश्वरूप उदारधीः
श्रीराजोवाच
यया गुप्तः सहस्राक्षः सवाहान्रिपुसैनिकान्‌
क्रीडन्निव विनिर्जित्य त्रिलोक्या बुभुजे श्रियम्‌
भगवंस्तन्ममाख्याहि वर्म नारायणात्मकम्‌
यथाततायिनः शत्रून्येन गुप्तोऽजयन्मृधे
श्रीबादरायणिरुवाच
वृतः पुरोहितस्त्वाष्ट्रो महेन्द्रायानुपृच्छते
नारायणाख्यं वर्माह तदिहैकमनाः शृणु
श्रीविश्वरूप उवाच
धौताङ्घ्रिपाणिराचम्य सपवित्र उदङ्मुखः
कृतस्वाङ्गकरन्यासो मन्त्राभ्यां वाग्यतः शुचिः
नारायणपरं वर्म सन्नह्येद्भय आगते
पादयोर्जानुनोरूर्वोरुदरे हृद्यथोरसि
मुखे शिरस्यानुपूर्व्यादोंकारादीनि विन्यसेत्‌
ओं नमो नारायणायेति विपर्ययमथापि वा
करन्यासं ततः कुर्याद्द्वादशाक्षरविद्यया
प्रणवादियकारान्तमङ्गुल्यङ्गुष्ठपर्वसु
न्यसेद्धृदय ओंकारं विकारमनु मूर्धनि
षकारं तु भ्रुवोर्मध्ये णकारं शिखया न्यसेत्‌
वेकारं नेत्रयोर्युञ्ज्यान्नकारं सर्वसन्धिषु
मकारमस्त्रमुद्दिश्य मन्त्रमूर्तिर्भवेद्बुधः
सविसर्गं फडन्तं तत्सर्वदिक्षु विनिर्दिशेत्‌
ओं विष्णवे नम इति
आत्मानं परमं ध्यायेद्ध्येयं षट्शक्तिभिर्युतम्‌
विद्यातेजस्तपोमूर्तिमिमं मन्त्रमुदाहरेत्‌
ओं हरिर्विदध्यान्मम सर्वरक्षां न्यस्ताङ्घ्रिपद्मः पतगेन्द्रपृष्ठे
दरारिचर्मासिगदेषुचाप पाशान्दधानोऽष्टगुणोऽष्टबाहुः
जलेषु मां रक्षतु मत्स्यमूर्तिर्यादोगणेभ्यो वरुणस्य पाशात्‌
स्थलेषु मायावटुवामनोऽव्यात्त्रिविक्रमः खेऽवतु विश्वरूपः
दुर्गेष्वटव्याजिमुखादिषु प्रभुः पायान्नृसिंहोऽसुरयूथपारिः
विमुञ्चतो यस्य महाट्टहासं दिशो विनेदुर्न्यपतंश्च गर्भाः
रक्षत्वसौ माध्वनि यज्ञकल्पः स्वदंष्ट्रयोन्नीतधरो वराहः
रामोऽद्रिकूटेष्वथ विप्रवासे सलक्ष्मणोऽव्याद्भरताग्रजोऽस्मान्‌
मामुग्रधर्मादखिलात्प्रमादान्नारायणः पातु नरश्च हासात्‌
दत्तस्त्वयोगादथ योगनाथः पायाद्गुणेशः कपिलः कर्मबन्धात्‌
सनत्कुमारोऽवतु कामदेवाद्धयशीर्षा मां पथि देवहेलनात्‌
देवर्षिवर्यः पुरुषार्चनान्तरात्कूर्मो हरिर्मां निरयादशेषात्‌
धन्वन्तरिर्भगवान्पात्वपथ्याद्द्वन्द्वाद्भयादृषभो निर्जितात्मा
यज्ञश्च लोकादवताज्जनान्ताद्बलो गणात्क्रोधवशादहीन्द्रः
द्वैपायनो भगवानप्रबोधाद्बुद्धस्तु पाषण्डगणप्रमादात्‌
कल्किः कलेः कालमलात्प्रपातु धर्मावनायोरुकृतावतारः
मां केशवो गदया प्रातरव्याद्गोविन्द आसङ्गवमात्तवेणुः
नारायणः प्राह्ण उदात्तशक्तिर्मध्यन्दिने विष्णुररीन्द्रपाणिः
देवोऽपराह्णे मधुहोग्रधन्वा सायं त्रिधामावतु माधवो माम्‌
दोषे हृषीकेश उतार्धरात्रे निशीथ एकोऽवतु पद्मनाभः
श्रीवत्सधामापररात्र ईशः प्रत्यूष ईशोऽसिधरो जनार्दनः
दामोदरोऽव्यादनुसन्ध्यं प्रभाते विश्वेश्वरो भगवान्कालमूर्तिः
चक्रं युगान्तानलतिग्मनेमि भ्रमत्समन्ताद्भगवत्प्रयुक्तम्‌
दन्दग्धि दन्दग्ध्यरिसैन्यमाशु कक्षं यथा वातसखो हुताशः
गदेऽशनिस्पर्शनविस्फुलिङ्गे निष्पिण्ढि निष्पिण्ढ्यजितप्रियासि
कुष्माण्डवैनायकयक्षरक्षो भूतग्रहांश्चूर्णय चूर्णयारीन्‌
त्वं यातुधानप्रमथप्रेतमातृ पिशाचविप्रग्रहघोरदृष्टीन्‌
दरेन्द्र विद्रावय कृष्णपूरितो भीमस्वनोऽरेर्हृदयानि कम्पयन्‌
त्वं तिग्मधारासिवरारिसैन्यमीशप्रयुक्तो मम छिन्धि छिन्धि
चक्षूंषि चर्मन्छतचन्द्र छादय द्विषामघोनां हर पापचक्षुषाम्‌
यन्नो भयं ग्रहेभ्योऽभूत्केतुभ्यो नृभ्य एव च
सरीसृपेभ्यो दंष्ट्रिभ्यो भूतेभ्योऽंहोभ्य एव च
सर्वाण्येतानि भगवन्नामरूपानुकीर्तनात्‌
प्रयान्तु सङ्क्षयं सद्यो ये नः श्रेयःप्रतीपकाः
गरुडो भगवान्स्तोत्र स्तोभश्छन्दोमयः प्रभुः
रक्षत्वशेषकृच्छ्रेभ्यो विष्वक्सेनः स्वनामभिः
सर्वापद्भ्यो हरेर्नाम रूपयानायुधानि नः
बुद्धीन्द्रियमनःप्राणान्पान्तु पार्षदभूषणाः
यथा हि भगवानेव वस्तुतः सदसच्च यत्‌
सत्येनानेन नः सर्वे यान्तु नाशमुपद्रवाः
यथैकात्म्यानुभावानां विकल्परहितः स्वयम्‌
भूषणायुधलिङ्गाख्या धत्ते शक्तीः स्वमायया
तेनैव सत्यमानेन सर्वज्ञो भगवान्हरिः
पातु सर्वैः स्वरूपैर्नः सदा सर्वत्र सर्वगः
विदिक्षु दिक्षूर्ध्वमधः समन्तादन्तर्बहिर्भगवान्नारसिंहः
प्रहापय लोकभयं स्वनेन स्वतेजसा ग्रस्तसमस्ततेजाः
मघवन्निदमाख्यातं वर्म नारायणात्मकम्‌
विजेष्यसेऽञ्जसा येन दंशितोऽसुरयूथपान्‌
एतद्धारयमाणस्तु यं यं पश्यति चक्षुषा
पदा वा संस्पृशेत्सद्यः साध्वसात्स विमुच्यते
न कुतश्चिद्भयं तस्य विद्यां धारयतो भवेत्‌
राजदस्युग्रहादिभ्यो व्याध्यादिभ्यश्च कर्हिचित्‌
इमां विद्यां पुरा कश्चित्कौशिको धारयन्द्विजः
योगधारणया स्वाङ्गं जहौ स मरुधन्वनि
तस्योपरि विमानेन गन्धर्वपतिरेकदा
ययौ चित्ररथः स्त्रीभिर्वृतो यत्र द्विजक्षयः
गगनान्न्यपतत्सद्यः सविमानो ह्यवाक्षिराः
स वालिखिल्यवचनादस्थीन्यादाय विस्मितः
प्रास्य प्राचीसरस्वत्यां स्नात्वा धाम स्वमन्वगात्‌
श्रीशुक उवाच
य इदं शृणुयात्काले यो धारयति चादृतः
तं नमस्यन्ति भूतानि मुच्यते सर्वतो भयात्‌
एतां विद्यामधिगतो विश्वरूपाच्छतक्रतुः
त्रैलोक्यलक्ष्मीं बुभुजे विनिर्जित्य मृधेऽसुरान्‌
श्रीशुक उवाच
तस्यासन्विश्वरूपस्य शिरांसि त्रीणि भारत
सोमपीथं सुरापीथमन्नादमिति शुश्रुम
स वै बर्हिषि देवेभ्यो भागं प्रत्यक्षमुच्चकैः
अददद्यस्य पितरो देवाः सप्रश्रयं नृप
स एव हि ददौ भागं परोक्षमसुरान्प्रति
यजमानोऽवहद्भागं मातृस्नेहवशानुगः
तद्देवहेलनं तस्य धर्मालीकं सुरेश्वरः
आलक्ष्य तरसा भीतस्तच्छीर्षाण्यच्छिनद्रुषा
सोमपीथं तु यत्तस्य शिर आसीत्कपिञ्जलः
कलविङ्कः सुरापीथमन्नादं यत्स तित्तिरिः
ब्रह्महत्यामञ्जलिना जग्राह यदपीश्वरः
संवत्सरान्ते तदघं भूतानां स विशुद्धये
भूम्यम्बुद्रुमयोषिद्भ्यश्चतुर्धा व्यभजद्धरिः
भूमिस्तुरीयं जग्राह खातपूरवरेण वै
ईरिणं ब्रह्महत्याया रूपं भूमौ प्रदृश्यते
तुर्यं छेदविरोहेण वरेण जगृहुर्द्रुमाः
तेषां निर्यासरूपेण ब्रह्महत्या प्रदृश्यते
शश्वत्कामवरेणांहस्तुरीयं जगृहुः स्त्रियः
रजोरूपेण तास्वंहो मासि मासि प्रदृश्यते
द्रव्यभूयोवरेणापस्तुरीयं जगृहुर्मलम्‌
तासु बुद्बुदफेनाभ्यां दृष्टं तद्धरति क्षिपन्‌
हतपुत्रस्ततस्त्वष्टा जुहावेन्द्राय शत्रवे
इन्द्रशत्रो विवर्धस्व मा चिरं जहि विद्विषम्‌
अथान्वाहार्यपचनादुत्थितो घोरदर्शनः
कृतान्त इव लोकानां युगान्तसमये यथा
विष्वग्विवर्धमानं तमिषुमात्रं दिने दिने
दग्धशैलप्रतीकाशं सन्ध्याभ्रानीकवर्चसम्‌
तप्तताम्रशिखाश्मश्रुं मध्याह्नार्कोग्रलोचनम्‌
देदीप्यमाने त्रिशिखे शूल आरोप्य रोदसी
नृत्यन्तमुन्नदन्तं च चालयन्तं पदा महीम्‌
दरीगम्भीरवक्त्रेण पिबता च नभस्तलम्‌
लिहता जिह्वयर्क्षाणि ग्रसता भुवनत्रयम्‌
महता रौद्रदंष्ट्रेण जृम्भमाणं मुहुर्मुहुः
वित्रस्ता दुद्रुवुर्लोका वीक्ष्य सर्वे दिशो दश
येनावृता इमे लोकास्तपसा त्वाष्ट्रमूर्तिना
स वै वृत्र इति प्रोक्तः पापः परमदारुणः
तं निजघ्नुरभिद्रुत्य सगणा विबुधर्षभाः
स्वैः स्वैर्दिव्यास्त्रशस्त्रौघैः सोऽग्रसत्तानि कृत्स्नशः
ततस्ते विस्मिताः सर्वे विषण्णा ग्रस्ततेजसः
प्रत्यञ्चमादिपुरुषमुपतस्थुः समाहिताः
श्रीदेवा ऊचुः
वाय्वम्बराग्न्यप्क्षितयस्त्रिलोका ब्रह्मादयो ये वयमुद्विजन्तः
हराम यस्मै बलिमन्तकोऽसौ बिभेति यस्मादरणं ततो नः
अविस्मितं तं परिपूर्णकामं स्वेनैव लाभेन समं प्रशान्तम्‌
विनोपसर्पत्यपरं हि बालिशः श्वलाङ्गुलेनातितितर्ति सिन्धुम्‌
यस्योरुशृङ्गे जगतीं स्वनावं मनुर्यथाबध्य ततार दुर्गम्‌
स एव नस्त्वाष्ट्रभयाद्दुरन्तात्त्राताश्रितान्वारिचरोऽपि नूनम्‌
पुरा स्वयम्भूरपि संयमाम्भस्युदीर्णवातोर्मिरवैः कराले
एकोऽरविन्दात्पतितस्ततार तस्माद्भयाद्येन स नोऽस्तु पारः
य एक ईशो निजमायया नः ससर्ज येनानुसृजाम विश्वम्‌
वयं न यस्यापि पुरः समीहतः पश्याम लिङ्गं पृथगीशमानिनः
यो नः सपत्नैर्भृशमर्द्यमानान्देवर्षितिर्यङ्नृषु नित्य एव
कृतावतारस्तनुभिः स्वमायया कृत्वात्मसात्पाति युगे युगे च
तमेव देवं वयमात्मदैवतं परं प्रधानं पुरुषं विश्वमन्यम्‌
व्रजाम सर्वे शरणं शरण्यं स्वानां स नो धास्यति शं महात्मा
श्रीशुक उवाच
इति तेषां महाराज सुराणामुपतिष्ठताम्‌
प्रतीच्यां दिश्यभूदाविः शङ्खचक्रगदाधरः
आत्मतुल्यैः षोडशभिर्विना श्रीवत्सकौस्तुभौ
पर्युपासितमुन्निद्र शरदम्बुरुहेक्षणम्‌
दृष्ट्वा तमवनौ सर्व ईक्षणाह्लादविक्लवाः
दण्डवत्पतिता राजञ्छनैरुत्थाय तुष्टुवुः
श्रीदेवा ऊचुः
नमस्ते यज्ञवीर्याय वयसे उत ते नमः
नमस्ते ह्यस्तचक्राय नमः सुपुरुहूतये
यत्ते गतीनां तिसृणामीशितुः परमं पदम्‌
नार्वाचीनो विसर्गस्य धातर्वेदितुमर्हति
ओं नमस्तेऽस्तु भगवन्नारायण वासुदेवादिपुरुष महापुरुष महानुभाव परममङ्गल
परमकल्याण परमकारुणिक केवल जगदाधार लोकैकनाथ सर्वेश्वर लक्ष्मीनाथ
परमहंसपरिव्राजकैः परमेणात्मयोगसमाधिना
परिभावितपरिस्फुटपारमहंस्यधर्मेणोद्घाटिततमःकपाटद्वारे चित्तेऽपावृत आत्मलोके
स्वयमुपलब्धनिजसुखानुभवो भवान्‌
दुरवबोध इव तवायं विहारयोगो यदशरणोऽशरीर इदमनवेक्षितास्मत्समवाय
आत्मनैवाविक्रियमाणेन सगुणमगुणः सृजसि पासि हरसि
अथ तत्र भवान्किं देवदत्तवदिह गुणविसर्गपतितः पारतन्त्र्येण स्वकृतकुशलाकुशलं
फलमुपाददात्याहोस्विदात्माराम उपशमशीलः समञ्जसदर्शन उदास्त इति ह वाव न विदामः
न हि विरोध उभयं भगवत्यपरिमितगुणगण ईश्वरेऽनवगाह्यमाहात्म्ये
ऽर्वाचीनविकल्पवितर्कविचारप्रमाणाभासकुतर्कशास्त्रकलिलान्तःकरणाश्रयदुरवग्रहवादिनां
विवादानवसर उपरतसमस्तमायामये केवल एवात्ममायामन्तर्धाय को न्वर्थो दुर्घट इव भवति
स्वरूपद्वयाभावात्‌
समविषममतीनां मतमनुसरसि यथा रज्जुखण्डः सर्पादिधियाम्‌
स एव हि पुनः सर्ववस्तुनि वस्तुस्वरूपः सर्वेश्वरः सकलजगत्कारणकारणभूतः
सर्वप्रत्यगात्मत्वात्सर्वगुणाभासोपलक्षित एक एव पर्यवशेषितः
अथ ह वाव तव महिमामृतरससमुद्रविप्रुषा सकृदवलीढया स्वमनसि
निष्यन्दमानानवरतसुखेन विस्मारितदृष्टश्रुतविषयसुखलेशाभासाः परमभागवता एकान्तिनो
भगवति सर्वभूतप्रियसुहृदि सर्वात्मनि नितरां निरन्तरं निर्वृतमनसः कथमु ह वा एते
मधुमथन पुनः स्वार्थकुशला ह्यात्मप्रियसुहृदः साधवस्त्वच्चरणाम्बुजानुसेवां विसृजन्ति न यत्र
पुनरयं संसारपर्यावर्तः
त्रिभुवनात्मभवन त्रिविक्रम त्रिनयन त्रिलोकमनोहरानुभाव तवैव विभूतयो
दितिजदनुजादयश्चापि तेषामुपक्रमसमयोऽयमिति स्वात्ममायया
सुरनरमृगमिश्रितजलचराकृतिभिर्यथापराधं दण्डं दण्डधर दधर्थ एवमेनमपि भगवन्जहि
त्वाष्ट्रमुत यदि मन्यसे
अस्माकं तावकानां तततत नतानां हरे तव
चरणनलिनयुगलध्यानानुबद्धहृदयनिगडानां
स्वलिङ्गविवरणेनात्मसात्कृतानामनुकम्पानुरञ्जितविशदरुचिरशिशिरस्मितावलोकेन
विगलितमधुरमुखरसामृतकलया चान्तस्तापमनघार्हसि शमयितुम्‌
अथ भगवंस्तवास्माभिरखिलजगदुत्पत्तिस्थितिलयनिमित्तायमानदिव्यमायाविनोदस्य
सकलजीवनिकायानामन्तर्हृदयेषु बहिरपि च ब्रह्मप्रत्यगात्मस्वरूपेण प्रधानरूपेण च
यथादेशकालदेहावस्थानविशेषं तदुपादानोपलम्भकतयानुभवतः सर्वप्रत्ययसाक्षिण
आकाशशरीरस्य साक्षात्परब्रह्मणः परमात्मनः कियानिह वार्थविशेषो विज्ञापनीयः
स्याद्विस्फुलिङ्गादिभिरिव हिरण्यरेतसः
अत एव स्वयं तदुपकल्पयास्माकं भगवतः परमगुरोस्तव चरणशतपलाशच्छायां
विविधवृजिनसंसारपरिश्रमोपशमनीमुपसृतानां वयं यत्कामेनोपसादिताः
अथो ईश जहि त्वाष्ट्रं ग्रसन्तं भुवनत्रयम्‌
ग्रस्तानि येन नः कृष्ण तेजांस्यस्त्रायुधानि च
हंसाय दह्रनिलयाय निरीक्षकाय कृष्णाय मृष्टयशसे निरुपक्रमाय
सत्सङ्ग्रहाय भवपान्थनिजाश्रमाप्तावन्ते परीष्टगतये हरये नमस्ते
श्रीशुक उवाच
अथैवमीडितो राजन्सादरं त्रिदशैर्हरिः
स्वमुपस्थानमाकर्ण्य प्राह तानभिनन्दितः
श्रीभगवानुवाच
प्रीतोऽहं वः सुरश्रेष्ठा मदुपस्थानविद्यया
आत्मैश्वर्यस्मृतिः पुंसां भक्तिश्चैव यया मयि
किं दुरापं मयि प्रीते तथापि विबुधर्षभाः
मय्येकान्तमतिर्नान्यन्मत्तो वाञ्छति तत्त्ववित्‌
न वेद कृपणः श्रेय आत्मनो गुणवस्तुदृक्‌
तस्य तानिच्छतो यच्छेद्यदि सोऽपि तथाविधः
स्वयं निःश्रेयसं विद्वान्न वक्त्यज्ञाय कर्म हि
न राति रोगिणोऽपथ्यं वाञ्छतोऽपि भिषक्तमः
मघवन्यात भद्रं वो दध्यञ्चमृषिसत्तमम्‌
विद्याव्रततपःसारं गात्रं याचत मा चिरम्‌
स वा अधिगतो दध्यङ्ङश्विभ्यां ब्रह्म निष्कलम्‌
यद्वा अश्वशिरो नाम तयोरमरतां व्यधात्‌
दध्यङ्ङाथर्वणस्त्वष्ट्रे वर्माभेद्यं मदात्मकम्‌
विश्वरूपाय यत्प्रादात्त्वष्टा यत्त्वमधास्ततः
युष्मभ्यं याचितोऽश्विभ्यां धर्मज्ञोऽङ्गानि दास्यति
ततस्तैरायुधश्रेष्ठो विश्वकर्मविनिर्मितः
येन वृत्रशिरो हर्ता मत्तेजौपबृंहितः
तस्मिन्विनिहते यूयं तेजोऽस्त्रायुधसम्पदः
भूयः प्राप्स्यथ भद्रं वो न हिंसन्ति च मत्परान्‌
श्रीबादरायणिरुवाच
इन्द्रमेवं समादिश्य भगवान्विश्वभावनः
पश्यतामनिमेषाणां अत्रैवान्तर्दधे हरिः
तथाभियाचितो देवैरृषिराथर्वणो महान्‌
मोदमान उवाचेदं प्रहसन्निव भारत
अपि वृन्दारका यूयं न जानीथ शरीरिणाम्‌
संस्थायां यस्त्वभिद्रोहो दुःसहश्चेतनापहः
जिजीविषूणां जीवानामात्मा प्रेष्ठ इहेप्सितः
क उत्सहेत तं दातुं भिक्षमाणाय विष्णवे
श्रीदेवा ऊचुः
किं नु तद्दुस्त्यजं ब्रह्मन्पुंसां भूतानुकम्पिनाम्‌
भवद्विधानां महतां पुण्यश्लोकेड्यकर्मणाम्‌
नूनं स्वार्थपरो लोको न वेद परसङ्कटम्‌
यदि वेद न याचेत नेति नाह यदीश्वरः
श्रीऋषिरुवाच
धर्मं वः श्रोतुकामेन यूयं मे प्रत्युदाहृताः
एष वः प्रियमात्मानं त्यजन्तं सन्त्यजाम्यहम्‌
योऽध्रुवेणात्मना नाथा न धर्मं न यशः पुमान्‌
ईहेत भूतदयया स शोच्यः स्थावरैरपि
एतावानव्ययो धर्मः पुण्यश्लोकैरुपासितः
यो भूतशोकहर्षाभ्यामात्मा शोचति हृष्यति
अहो दैन्यमहो कष्टं पारक्यैः क्षणभङ्गुरैः
यन्नोपकुर्यादस्वार्थैर्मर्त्यः स्वज्ञातिविग्रहैः
श्रीबादरायणिरुवाच
एवं कृतव्यवसितो दध्यङ्ङाथर्वणस्तनुम्‌
परे भगवति ब्रह्मण्यात्मानं सन्नयन्जहौ
यताक्षासुमनोबुद्धिस्तत्त्वदृग्ध्वस्तबन्धनः
आस्थितः परमं योगं न देहं बुबुधे गतम्‌
अथेन्द्रो वज्रमुद्यम्य निर्मितं विश्वकर्मणा
मुनेः शक्तिभिरुत्सिक्तो भगवत्तेजसान्वितः
वृतो देवगणैः सर्वैर्गजेन्द्रोपर्यशोभत
स्तूयमानो मुनिगणैस्त्रैलोक्यं हर्षयन्निव
वृत्रमभ्यद्रवच्छत्रुमसुरानीकयूथपैः
पर्यस्तमोजसा राजन्क्रुद्धो रुद्र इवान्तकम्‌
ततः सुराणामसुरै रणः परमदारुणः
त्रेतामुखे नर्मदायामभवत्प्रथमे युगे
रुद्रैर्वसुभिरादित्यैरश्विभ्यां पितृवह्निभिः
मरुद्भिरृभुभिः साध्यैर्विश्वेदेवैर्मरुत्पतिम्‌
दृष्ट्वा वज्रधरं शक्रं रोचमानं स्वया श्रिया
नामृष्यन्नसुरा राजन्मृधे वृत्रपुरःसराः
नमुचिः शम्बरोऽनर्वा द्विमूर्धा ऋषभोऽसुरः
हयग्रीवः शङ्कुशिरा विप्रचित्तिरयोमुखः
पुलोमा वृषपर्वा च प्रहेतिर्हेतिरुत्कलः
दैतेया दानवा यक्षा रक्षांसि च सहस्रशः
सुमालिमालिप्रमुखाः कार्तस्वरपरिच्छदाः
प्रतिषिध्येन्द्रसेनाग्रं मृत्योरपि दुरासदम्‌
अभ्यर्दयन्नसम्भ्रान्ताः सिंहनादेन दुर्मदाः
गदाभिः परिघैर्बाणैः प्रासमुद्गरतोमरैः
शूलैः परश्वधैः खड्गैः शतघ्नीभिर्भुशुण्डिभिः
सर्वतोऽवाकिरन्शस्त्रैरस्त्रैश्च विबुधर्षभान्‌
न तेऽदृश्यन्त सञ्छन्नाः शरजालैः समन्ततः
पुङ्खानुपुङ्खपतितैर्ज्योतींषीव नभोघनैः
न ते शस्त्रास्त्रवर्षौघा ह्यासेदुः सुरसैनिकान्‌
छिन्नाः सिद्धपथे देवैर्लघुहस्तैः सहस्रधा
अथ क्षीणास्त्रशस्त्रौघा गिरिशृङ्गद्रुमोपलैः
अभ्यवर्षन्सुरबलं चिच्छिदुस्तांश्च पूर्ववत्‌
तानक्षतान्स्वस्तिमतो निशाम्य शस्त्रास्त्रपूगैरथ वृत्रनाथाः
द्रुमैर्दृषद्भिर्विविधाद्रिशृङ्गैरविक्षतांस्तत्रसुरिन्द्रसैनिकान्‌
सर्वे प्रयासा अभवन्विमोघाः कृताः कृता देवगणेषु दैत्यैः
कृष्णानुकूलेषु यथा महत्सु क्षुद्रैः प्रयुक्ता ऊषती रूक्षवाचः
ते स्वप्रयासं वितथं निरीक्ष्य हरावभक्ता हतयुद्धदर्पाः
पलायनायाजिमुखे विसृज्य पतिं मनस्ते दधुरात्तसाराः
वृत्रोऽसुरांस्ताननुगान्मनस्वी प्रधावतः प्रेक्ष्य बभाष एतत्‌
पलायितं प्रेक्ष्य बलं च भग्नं भयेन तीव्रेण विहस्य वीरः
कालोपपन्नां रुचिरां मनस्विनां जगाद वाचं पुरुषप्रवीरः
हे विप्रचित्ते नमुचे पुलोमन्मयानर्वन्छम्बर मे शृणुध्वम्‌
जातस्य मृत्युर्ध्रुव एव सर्वतः प्रतिक्रिया यस्य न चेह कॢप्ता
लोको यशश्चाथ ततो यदि ह्यमुं को नाम मृत्युं न वृणीत युक्तम्‌
द्वौ सम्मताविह मृत्यू दुरापौ यद्ब्रह्मसन्धारणया जितासुः
कलेवरं योगरतो विजह्याद्यदग्रणीर्वीरशयेऽनिवृत्तः
श्रीशुक उवाच
त एवं शंसतो धर्मं वचः पत्युरचेतसः
नैवागृह्णन्त सम्भ्रान्ताः पलायनपरा नृप
विशीर्यमाणां पृतनामासुरीमसुरर्षभः
कालानुकूलैस्त्रिदशैः काल्यमानामनाथवत्‌
दृष्ट्वातप्यत सङ्क्रुद्ध इन्द्रशत्रुरमर्षितः
तान्निवार्यौजसा राजन्निर्भर्त्स्येदमुवाच ह
किं व उच्चरितैर्मातुर्धावद्भिः पृष्ठतो हतैः
न हि भीतवधः श्लाघ्यो न स्वर्ग्यः शूरमानिनाम्‌
यदि वः प्रधने श्रद्धा सारं वा क्षुल्लका हृदि
अग्रे तिष्ठत मात्रं मे न चेद्ग्राम्यसुखे स्पृहा
एवं सुरगणान्क्रुद्धो भीषयन्वपुषा रिपून्‌
व्यनदत्सुमहाप्राणो येन लोका विचेतसः
तेन देवगणाः सर्वे वृत्रविस्फोटनेन वै
निपेतुर्मूर्च्छिता भूमौ यथैवाशनिना हताः
ममर्द पद्भ्यां सुरसैन्यमातुरं निमीलिताक्षं रणरङ्गदुर्मदः
गां कम्पयन्नुद्यतशूल ओजसा नालं वनं यूथपतिर्यथोन्मदः
विलोक्य तं वज्रधरोऽत्यमर्षितः स्वशत्रवेऽभिद्रवते महागदाम्‌
चिक्षेप तामापततीं सुदुःसहां जग्राह वामेन करेण लीलया
स इन्द्रशत्रुः कुपितो भृशं तया महेन्द्रवाहं गदयोरुविक्रमः
जघान कुम्भस्थल उन्नदन्मृधे तत्कर्म सर्वे समपूजयन्नृप
ऐरावतो वृत्रगदाभिमृष्टो विघूर्णितोऽद्रिः कुलिशाहतो यथा
अपासरद्भिन्नमुखः सहेन्द्रो मुञ्चन्नसृक्सप्तधनुर्भृशार्तः
न सन्नवाहाय विषण्णचेतसे प्रायुङ्क्त भूयः स गदां महात्मा
इन्द्रोऽमृतस्यन्दिकराभिमर्श वीतव्यथक्षतवाहोऽवतस्थे
स तं नृपेन्द्राहवकाम्यया रिपुं वज्रायुधं भ्रातृहणं विलोक्य
स्मरंश्च तत्कर्म नृशंसमंहः शोकेन मोहेन हसन्जगाद
श्रीवृत्र उवाच
दिष्ट्या भवान्मे समवस्थितो रिपुर्यो ब्रह्महा गुरुहा भ्रातृहा च
दिष्ट्यानृणोऽद्याहमसत्तम त्वया मच्छूलनिर्भिन्नदृषद्धृदाचिरात्‌
यो नोऽग्रजस्यात्मविदो द्विजातेर्गुरोरपापस्य च दीक्षितस्य
विश्रभ्य खड्गेन शिरांस्यवृश्चत्पशोरिवाकरुणः स्वर्गकामः
श्रीह्रीदयाकीर्तिभिरुज्झितं त्वां स्वकर्मणा पुरुषादैश्च गर्ह्यम्‌
कृच्छ्रेण मच्छूलविभिन्नदेहमस्पृष्टवह्निं समदन्ति गृध्राः
अन्येऽनु ये त्वेह नृशंसमज्ञा यदुद्यतास्त्राः प्रहरन्ति मह्यम्‌
तैर्भूतनाथान्सगणान्निशात त्रिशूलनिर्भिन्नगलैर्यजामि
अथो हरे मे कुलिशेन वीर हर्ता प्रमथ्यैव शिरो यदीह
तत्रानृणो भूतबलिं विधाय मनस्विनां पादरजः प्रपत्स्ये
सुरेश कस्मान्न हिनोषि वज्रं पुरः स्थिते वैरिणि मय्यमोघम्‌
मा संशयिष्ठा न गदेव वज्रः स्यान्निष्फलः कृपणार्थेव याच्ञा
नन्वेष वज्रस्तव शक्र तेजसा हरेर्दधीचेस्तपसा च तेजितः
तेनैव शत्रुं जहि विष्णुयन्त्रितो यतो हरिर्विजयः श्रीर्गुणास्ततः
अहं समाधाय मनो यथाह नः सङ्कर्षणस्तच्चरणारविन्दे
त्वद्वज्ररंहोलुलितग्राम्यपाशो गतिं मुनेर्याम्यपविद्धलोकः
पुंसां किलैकान्तधियां स्वकानां याः सम्पदो दिवि भूमौ रसायाम्‌
न राति यद्द्वेष उद्वेग आधिर्मदः कलिर्व्यसनं सम्प्रयासः
त्रैवर्गिकायासविघातमस्मत्पतिर्विधत्ते पुरुषस्य शक्र
ततोऽनुमेयो भगवत्प्रसादो यो दुर्लभोऽकिञ्चनगोचरोऽन्यैः
अहं हरे तव पादैकमूल दासानुदासो भवितास्मि भूयः
मनः स्मरेतासुपतेर्गुणांस्ते गृणीत वाक्कर्म करोतु कायः
न नाकपृष्ठं न च पारमेष्ठ्यं न सार्वभौमं न रसाधिपत्यम्‌
न योगसिद्धीरपुनर्भवं वा समञ्जस त्वा विरहय्य काङ्क्षे
अजातपक्षा इव मातरं खगाः स्तन्यं यथा वत्सतराः क्षुधार्ताः
प्रियं प्रियेव व्युषितं विषण्णा मनोऽरविन्दाक्ष दिदृक्षते त्वाम्‌
ममोत्तमश्लोकजनेषु सख्यं संसारचक्रे भ्रमतः स्वकर्मभिः
त्वन्माययात्मात्मजदारगेहेष्वासक्तचित्तस्य न नाथ भूयात्‌
श्रीऋषिरुवाच
एवं जिहासुर्नृप देहमाजौ मृत्युं वरं विजयान्मन्यमानः
शूलं प्रगृह्याभ्यपतत्सुरेन्द्रं यथा महापुरुषं कैटभोऽप्सु
ततो युगान्ताग्निकठोरजिह्वमाविध्य शूलं तरसासुरेन्द्रः
क्षिप्त्वा महेन्द्राय विनद्य वीरो हतोऽसि पापेति रुषा जगाद
ख आपतत्तद्विचलद्ग्रहोल्कवन्निरीक्ष्य दुष्प्रेक्ष्यमजातविक्लवः
वज्रेण वज्री शतपर्वणाच्छिनद्भुजं च तस्योरगराजभोगम्‌
छिन्नैकबाहुः परिघेण वृत्रः संरब्ध आसाद्य गृहीतवज्रम्‌
हनौ तताडेन्द्रमथामरेभं वज्रं च हस्तान्न्यपतन्मघोनः
वृत्रस्य कर्मातिमहाद्भुतं तत्सुरासुराश्चारणसिद्धसङ्घाः
अपूजयंस्तत्पुरुहूतसङ्कटं निरीक्ष्य हा हेति विचुक्रुशुर्भृशम्‌
इन्द्रो न वज्रं जगृहे विलज्जितश्च्युतं स्वहस्तादरिसन्निधौ पुनः
तमाह वृत्रो हर आत्तवज्रो जहि स्वशत्रुं न विषादकालः
युयुत्सतां कुत्रचिदाततायिनां जयः सदैकत्र न वै परात्मनाम्‌
विनैकमुत्पत्तिलयस्थितीश्वरं सर्वज्ञमाद्यं पुरुषं सनातनम्‌
लोकाः सपाला यस्येमे श्वसन्ति विवशा वशे
द्विजा इव शिचा बद्धाः स काल इह कारणम्‌
ओजः सहो बलं प्राणममृतं मृत्युमेव च
तमज्ञाय जनो हेतुमात्मानं मन्यते जडम्‌
यथा दारुमयी नारी यथा पत्रमयो मृगः
एवं भूतानि मघवन्नीशतन्त्राणि विद्धि भोः
पुरुषः प्रकृतिर्व्यक्तमात्मा भूतेन्द्रियाशयाः
शक्नुवन्त्यस्य सर्गादौ न विना यदनुग्रहात्‌
अविद्वानेवमात्मानं मन्यतेऽनीशमीश्वरम्‌
भूतैः सृजति भूतानि ग्रसते तानि तैः स्वयम्‌
आयुः श्रीः कीर्तिरैश्वर्यमाशिषः पुरुषस्य याः
भवन्त्येव हि तत्काले यथानिच्छोर्विपर्ययाः
तस्मादकीर्तियशसोर्जयापजययोरपि
समः स्यात्सुखदुःखाभ्यां मृत्युजीवितयोस्तथा
सत्त्वं रजस्तम इति प्रकृतेर्नात्मनो गुणाः
तत्र साक्षिणमात्मानं यो वेद स न बध्यते
पश्य मां निर्जितं शत्रु वृक्णायुधभुजं मृधे
घटमानं यथाशक्ति तव प्राणजिहीर्षया
प्राणग्लहोऽयं समर इष्वक्षो वाहनासनः
अत्र न ज्ञायतेऽमुष्य जयोऽमुष्य पराजयः
श्रीशुक उवाच
इन्द्रो वृत्रवचः श्रुत्वा गतालीकमपूजयत्‌
गृहीतवज्रः प्रहसंस्तमाह गतविस्मयः
इन्द्र उवाच
अहो दानव सिद्धोऽसि यस्य ते मतिरीदृशी
भक्तः सर्वात्मनात्मानं सुहृदं जगदीश्वरम्‌
भवानतार्षीन्मायां वै वैष्णवीं जनमोहिनीम्‌
यद्विहायासुरं भावं महापुरुषतां गतः
खल्विदं महदाश्चर्यं यद्रजःप्रकृतेस्तव
वासुदेवे भगवति सत्त्वात्मनि दृढा मतिः
यस्य भक्तिर्भगवति हरौ निःश्रेयसेश्वरे
विक्रीडतोऽमृताम्भोधौ किं क्षुद्रैः खातकोदकैः
श्रीशुक उवाच
इति ब्रुवाणावन्योन्यं धर्मजिज्ञासया नृप
युयुधाते महावीर्याविन्द्रवृत्रौ युधाम्पती
आविध्य परिघं वृत्रः कार्ष्णायसमरिन्दमः
इन्द्राय प्राहिणोद्घोरं वामहस्तेन मारिष
स तु वृत्रस्य परिघं करं च करभोपमम्‌
चिच्छेद युगपद्देवो वज्रेण शतपर्वणा
दोर्भ्यामुत्कृत्तमूलाभ्यां बभौ रक्तस्रवोऽसुरः
छिन्नपक्षो यथा गोत्रः खाद्भ्रष्टो वज्रिणा हतः
महाप्राणो महावीर्यो महासर्प इव द्विपम्‌
कृत्वाधरां हनुं भूमौ दैत्यो दिव्युत्तरां हनुम्‌
नभोगम्भीरवक्त्रेण लेलिहोल्बणजिह्वया
दंष्ट्राभिः कालकल्पाभिर्ग्रसन्निव जगत्त्रयम्‌
अतिमात्रमहाकाय आक्षिपंस्तरसा गिरीन्‌
गिरिराट्पादचारीव पद्भ्यां निर्जरयन्महीम्‌
जग्रास स समासाद्य वज्रिणं सहवाहनम्‌
वृत्रग्रस्तं तमालोक्य सप्रजापतयः सुराः
हा कष्टमिति निर्विण्णाश्चुक्रुशुः समहर्षयः
निगीर्णोऽप्यसुरेन्द्रेण न ममारोदरं गतः
महापुरुषसन्नद्धो योगमायाबलेन च
भित्त्वा वज्रेण तत्कुक्षिं निष्क्रम्य बलभिद्विभुः
उच्चकर्त शिरः शत्रोर्गिरिशृङ्गमिवौजसा
वज्रस्तु तत्कन्धरमाशुवेगः कृन्तन्समन्तात्परिवर्तमानः
न्यपातयत्तावदहर्गणेन यो ज्योतिषामयने वार्त्रहत्ये
तदा च खे दुन्दुभयो विनेदुर्गन्धर्वसिद्धाः समहर्षिसङ्घाः
वार्त्रघ्नलिङ्गैस्तमभिष्टुवाना मन्त्रैर्मुदा कुसुमैरभ्यवर्षन्‌
वृत्रस्य देहान्निष्क्रान्तमात्मज्योतिररिन्दम
पश्यतां सर्वदेवानामलोकं समपद्यत
श्रीशुक उवाच
वृत्रे हते त्रयो लोका विना शक्रेण भूरिद
सपाला ह्यभवन्सद्यो विज्वरा निर्वृतेन्द्रियाः
देवर्षिपितृभूतानि दैत्या देवानुगाः स्वयम्‌
प्रतिजग्मुः स्वधिष्ण्यानि ब्रह्मेशेन्द्रादयस्ततः
श्रीराजोवाच
इन्द्रस्यानिर्वृतेर्हेतुं श्रोतुमिच्छामि भो मुने
येनासन्सुखिनो देवा हरेर्दुःखं कुतोऽभवत्‌
श्रीशुक उवाच
वृत्रविक्रमसंविग्नाः सर्वे देवाः सहर्षिभिः
तद्वधायार्थयन्निन्द्रं नैच्छद्भीतो बृहद्वधात्‌
इन्द्र उवाच
स्त्रीभूद्रुमजलैरेनो विश्वरूपवधोद्भवम्‌
विभक्तमनुगृह्णद्भिर्वृत्रहत्यां क्व मार्ज्म्यहम्‌
श्रीशुक उवाच
ऋषयस्तदुपाकर्ण्य महेन्द्रमिदमब्रुवन्‌
याजयिष्याम भद्रं ते हयमेधेन मा स्म भैः
हयमेधेन पुरुषं परमात्मानमीश्वरम्‌
इष्ट्वा नारायणं देवं मोक्ष्यसेऽपि जगद्वधात्‌
ब्रह्महा पितृहा गोघ्नो मातृहाचार्यहाघवान्‌
श्वादः पुल्कसको वापि शुद्ध्येरन्यस्य कीर्तनात्‌
तमश्वमेधेन महामखेन श्रद्धान्वितोऽस्माभिरनुष्ठितेन
हत्वापि सब्रह्मचराचरं त्वं न लिप्यसे किं खलनिग्रहेण
श्रीशुक उवाच
एवं सञ्चोदितो विप्रैर्मरुत्वानहनद्रिपुम्‌
ब्रह्महत्या हते तस्मिन्नाससाद वृषाकपिम्‌
तयेन्द्रः स्मासहत्तापं निर्वृतिर्नामुमाविशत्‌
ह्रीमन्तं वाच्यतां प्राप्तं सुखयन्त्यपि नो गुणाः
तां ददर्शानुधावन्तीं चाण्डालीमिव रूपिणीम्‌
जरया वेपमानाङ्गीं यक्ष्मग्रस्तामसृक्पटाम्‌
विकीर्य पलितान्केशांस्तिष्ठ तिष्ठेति भाषिणीम्‌
मीनगन्ध्यसुगन्धेन कुर्वतीं मार्गदूषणम्‌
नभो गतो दिशः सर्वाः सहस्राक्षो विशाम्पते
प्रागुदीचीं दिशं तूर्णं प्रविष्टो नृप मानसम्‌
स आवसत्पुष्करनालतन्तूनलब्धभोगो यदिहाग्निदूतः
वर्षाणि साहस्रमलक्षितोऽन्तः सञ्चिन्तयन्ब्रह्मवधाद्विमोक्षम्‌
तावत्त्रिणाकं नहुषः शशास विद्यातपोयोगबलानुभावः
स सम्पदैश्वर्यमदान्धबुद्धिर्नीतस्तिरश्चां गतिमिन्द्रपत्न्या
ततो गतो ब्रह्मगिरोपहूत ऋतम्भरध्याननिवारिताघः
पापस्तु दिग्देवतया हतौजास्तं नाभ्यभूदवितं विष्णुपत्न्या
तं च ब्रह्मर्षयोऽभ्येत्य हयमेधेन भारत
यथावद्दीक्षयां चक्रुः पुरुषाराधनेन ह
अथेज्यमाने पुरुषे सर्वदेवमयात्मनि
अश्वमेधे महेन्द्रेण वितते ब्रह्मवादिभिः
स वै त्वाष्ट्रवधो भूयानपि पापचयो नृप
नीतस्तेनैव शून्याय नीहार इव भानुना
स वाजिमेधेन यथोदितेन वितायमानेन मरीचिमिश्रैः
इष्ट्वाधियज्ञं पुरुषं पुराणमिन्द्रो महानास विधूतपापः
इदं महाख्यानमशेषपाप्मनां प्रक्षालनं तीर्थपदानुकीर्तनम्‌
भक्त्युच्छ्रयं भक्तजनानुवर्णनं महेन्द्रमोक्षं विजयं मरुत्वतः
पठेयुराख्यानमिदं सदा बुधाः शृण्वन्त्यथो पर्वणि पर्वणीन्द्रियम्‌
धन्यं यशस्यं निखिलाघमोचनं रिपुञ्जयं स्वस्त्ययनं तथायुषम्‌
श्रीपरीक्षिदुवाच
रजस्तमःस्वभावस्य ब्रह्मन्वृत्रस्य पाप्मनः
नारायणे भगवति कथमासीद्दृढा मतिः
देवानां शुद्धसत्त्वानामृषीणां चामलात्मनाम्‌
भक्तिर्मुकुन्दचरणे न प्रायेणोपजायते
रजोभिः समसङ्ख्याताः पार्थिवैरिह जन्तवः
तेषां ये केचनेहन्ते श्रेयो वै मनुजादयः
प्रायो मुमुक्षवस्तेषां केचनैव द्विजोत्तम
मुमुक्षूणां सहस्रेषु कश्चिन्मुच्येत सिध्यति
मुक्तानामपि सिद्धानां नारायणपरायणः
सुदुर्लभः प्रशान्तात्मा कोटिष्वपि महामुने
वृत्रस्तु स कथं पापः सर्वलोकोपतापनः
इत्थं दृढमतिः कृष्ण आसीत्सङ्ग्राम उल्बणे
अत्र नः संशयो भूयाञ्छ्रोतुं कौतूहलं प्रभो
यः पौरुषेण समरे सहस्राक्षमतोषयत्‌
श्रीसूत उवाच
परीक्षितोऽथ सम्प्रश्नं भगवान्बादरायणिः
निशम्य श्रद्दधानस्य प्रतिनन्द्य वचोऽब्रवीत्‌
श्रीशुक उवाच
शृणुष्वावहितो राजन्नितिहासमिमं यथा
श्रुतं द्वैपायनमुखान्नारदाद्देवलादपि
आसीद्राजा सार्वभौमः शूरसेनेषु वै नृप
चित्रकेतुरिति ख्यातो यस्यासीत्कामधुङ्मही
तस्य भार्यासहस्राणां सहस्राणि दशाभवन्‌
सान्तानिकश्चापि नृपो न लेभे तासु सन्ततिम्‌
रूपौदार्यवयोजन्म विद्यैश्वर्यश्रियादिभिः
सम्पन्नस्य गुणैः सर्वैश्चिन्ता बन्ध्यापतेरभूत्‌
न तस्य सम्पदः सर्वा महिष्यो वामलोचनाः
सार्वभौमस्य भूश्चेयमभवन्प्रीतिहेतवः
तस्यैकदा तु भवनमङ्गिरा भगवानृषिः
लोकाननुचरन्नेतानुपागच्छद्यदृच्छया
तं पूजयित्वा विधिवत्प्रत्युत्थानार्हणादिभिः
कृतातिथ्यमुपासीदत्सुखासीनं समाहितः
महर्षिस्तमुपासीनं प्रश्रयावनतं क्षितौ
प्रतिपूज्य महाराज समाभाष्येदमब्रवीत्‌
अङ्गिरा उवाच
अपि तेऽनामयं स्वस्ति प्रकृतीनां तथात्मनः
यथा प्रकृतिभिर्गुप्तः पुमान्राजा च सप्तभिः
आत्मानं प्रकृतिष्वद्धा निधाय श्रेय आप्नुयात्‌
राज्ञा तथा प्रकृतयो नरदेवाहिताधयः
अपि दाराः प्रजामात्या भृत्याः श्रेण्योऽथ मन्त्रिणः
पौरा जानपदा भूपा आत्मजा वशवर्तिनः
यस्यात्मानुवशश्चेत्स्यात्सर्वे तद्वशगा इमे
लोकाः सपाला यच्छन्ति सर्वे बलिमतन्द्रिताः
आत्मनः प्रीयते नात्मा परतः स्वत एव वा
लक्षयेऽलब्धकामं त्वां चिन्तया शबलं मुखम्‌
एवं विकल्पितो राजन्विदुषा मुनिनापि सः
प्रश्रयावनतोऽभ्याह प्रजाकामस्ततो मुनिम्‌
चित्रकेतुरुवाच
भगवन्किं न विदितं तपोज्ञानसमाधिभिः
योगिनां ध्वस्तपापानां बहिरन्तः शरीरिषु
तथापि पृच्छतो ब्रूयां ब्रह्मन्नात्मनि चिन्तितम्‌
भवतो विदुषश्चापि चोदितस्त्वदनुज्ञया
लोकपालैरपि प्रार्थ्याः साम्राज्यैश्वर्यसम्पदः
न नन्दयन्त्यप्रजं मां क्षुत्तृट्काममिवापरे
ततः पाहि महाभाग पूर्वैः सह गतं तमः
यथा तरेम दुष्पारं प्रजया तद्विधेहि नः
श्रीशुक उवाच
इत्यर्थितः स भगवान्कृपालुर्ब्रह्मणः सुतः
श्रपयित्वा चरुं त्वाष्ट्रं त्वष्टारमयजद्विभुः
ज्येष्ठा श्रेष्ठा च या राज्ञो महिषीणां च भारत
नाम्ना कृतद्युतिस्तस्यै यज्ञोच्छिष्टमदाद्द्विजः
अथाह नृपतिं राजन्भवितैकस्तवात्मजः
हर्षशोकप्रदस्तुभ्यमिति ब्रह्मसुतो ययौ
सापि तत्प्राशनादेव चित्रकेतोरधारयत्‌
गर्भं कृतद्युतिर्देवी कृत्तिकाग्नेरिवात्मजम्‌
तस्या अनुदिनं गर्भः शुक्लपक्ष इवोडुपः
ववृधे शूरसेनेश तेजसा शनकैर्नृप
अथ काल उपावृत्ते कुमारः समजायत
जनयन्शूरसेनानां शृण्वतां परमां मुदम्‌
हृष्टो राजा कुमारस्य स्नातः शुचिरलङ्कृतः
वाचयित्वाशिषो विप्रैः कारयामास जातकम्‌
तेभ्यो हिरण्यं रजतं वासांस्याभरणानि च
ग्रामान्हयान्गजान्प्रादाद्धेनूनामर्बुदानि षट्‌
ववर्ष कामानन्येषां पर्जन्य इव देहिनाम्‌
धन्यं यशस्यमायुष्यं कुमारस्य महामनाः
कृच्छ्रलब्धेऽथ राजर्षेस्तनयेऽनुदिनं पितुः
यथा निःस्वस्य कृच्छ्राप्ते धने स्नेहोऽन्ववर्धत
मातुस्त्वतितरां पुत्रे स्नेहो मोहसमुद्भवः
कृतद्युतेः सपत्नीनां प्रजाकामज्वरोऽभवत्‌
चित्रकेतोरतिप्रीतिर्यथा दारे प्रजावति
न तथान्येषु सञ्जज्ञे बालं लालयतोऽन्वहम्‌
ताः पर्यतप्यन्नात्मानं गर्हयन्त्योऽभ्यसूयया
आनपत्येन दुःखेन राज्ञश्चानादरेण च
धिगप्रजां स्त्रियं पापां पत्युश्चागृहसम्मताम्‌
सुप्रजाभिः सपत्नीभिर्दासीमिव तिरस्कृताम्‌
दासीनां को नु सन्तापः स्वामिनः परिचर्यया
अभीक्ष्णं लब्धमानानां दास्या दासीव दुर्भगाः
एवं सन्दह्यमानानां सपत्न्याः पुत्रसम्पदा
राज्ञोऽसम्मतवृत्तीनां विद्वेषो बलवानभूत्‌
विद्वेषनष्टमतयः स्त्रियो दारुणचेतसः
गरं ददुः कुमाराय दुर्मर्षा नृपतिं प्रति
कृतद्युतिरजानन्ती सपत्नीनामघं महत्‌
सुप्त एवेति सञ्चिन्त्य निरीक्ष्य व्यचरद्गृहे
शयानं सुचिरं बालमुपधार्य मनीषिणी
पुत्रमानय मे भद्रे इति धात्रीमचोदयत्‌
सा शयानमुपव्रज्य दृष्ट्वा चोत्तारलोचनम्‌
प्राणेन्द्रियात्मभिस्त्यक्तं हतास्मीत्यपतद्भुवि
तस्यास्तदाकर्ण्य भृशातुरं स्वरं घ्नन्त्याः कराभ्यामुर उच्चकैरपि
प्रविश्य राज्ञी त्वरयात्मजान्तिकं ददर्श बालं सहसा मृतं सुतम्‌
पपात भूमौ परिवृद्धया शुचा मुमोह विभ्रष्टशिरोरुहाम्बरा
ततो नृपान्तःपुरवर्तिनो जना नराश्च नार्यश्च निशम्य रोदनम्‌
आगत्य तुल्यव्यसनाः सुदुःखितास्ताश्च व्यलीकं रुरुदुः कृतागसः
श्रुत्वा मृतं पुत्रमलक्षितान्तकं विनष्टदृष्टिः प्रपतन्स्खलन्पथि
स्नेहानुबन्धैधितया शुचा भृशं विमूर्च्छितोऽनुप्रकृतिर्द्विजैर्वृतः
पपात बालस्य स पादमूले मृतस्य विस्रस्तशिरोरुहाम्बरः
दीर्घं श्वसन्बाष्पकलोपरोधतो निरुद्धकण्ठो न शशाक भाषितुम्‌
पतिं निरीक्ष्योरुशुचार्पितं तदा मृतं च बालं सुतमेकसन्ततिम्‌
जनस्य राज्ञी प्रकृतेश्च हृद्रुजं सती दधाना विललाप चित्रधा
स्तनद्वयं कुङ्कुमपङ्कमण्डितं निषिञ्चती साञ्जनबाष्पबिन्दुभिः
विकीर्य केशान्विगलत्स्रजः सुतं शुशोच चित्रं कुररीव सुस्वरम्‌
अहो विधातस्त्वमतीव बालिशो यस्त्वात्मसृष्ट्यप्रतिरूपमीहसे
परे नु जीवत्यपरस्य या मृतिर्विपर्ययश्चेत्त्वमसि ध्रुवः परः
न हि क्रमश्चेदिह मृत्युजन्मनोः शरीरिणामस्तु तदात्मकर्मभिः
यः स्नेहपाशो निजसर्गवृद्धये स्वयं कृतस्ते तमिमं विवृश्चसि
त्वं तात नार्हसि च मां कृपणामनाथां
त्यक्तुं विचक्ष्व पितरं तव शोकतप्तम्‌
अञ्जस्तरेम भवताप्रजदुस्तरं यद्‌
ध्वान्तं न याह्यकरुणेन यमेन दूरम्‌
उत्तिष्ठ तात त इमे शिशवो वयस्यास्‌
त्वामाह्वयन्ति नृपनन्दन संविहर्तुम्‌
सुप्तश्चिरं ह्यशनया च भवान्परीतो
भुङ्क्ष्व स्तनं पिब शुचो हर नः स्वकानाम्‌
नाहं तनूज ददृशे हतमङ्गला ते
मुग्धस्मितं मुदितवीक्षणमाननाब्जम्‌
किं वा गतोऽस्यपुनरन्वयमन्यलोकं
नीतोऽघृणेन न शृणोमि कला गिरस्ते
श्रीशुक उवाच
विलपन्त्या मृतं पुत्रमिति चित्रविलापनैः
चित्रकेतुर्भृशं तप्तो मुक्तकण्ठो रुरोद ह
तयोर्विलपतोः सर्वे दम्पत्योस्तदनुव्रताः
रुरुदुः स्म नरा नार्यः सर्वमासीदचेतनम्‌
एवं कश्मलमापन्नं नष्टसंज्ञमनायकम्‌
ज्ञात्वाङ्गिरा नाम ऋषिराजगाम सनारदः
श्रीशुक उवाच
ऊचतुर्मृतकोपान्ते पतितं मृतकोपमम्‌
शोकाभिभूतं राजानं बोधयन्तौ सदुक्तिभिः
कोऽयं स्यात्तव राजेन्द्र भवान्यमनुशोचति
त्वं चास्य कतमः सृष्टौ पुरेदानीमतः परम्‌
यथा प्रयान्ति संयान्ति स्रोतोवेगेन बालुकाः
संयुज्यन्ते वियुज्यन्ते तथा कालेन देहिनः
यथा धानासु वै धाना भवन्ति न भवन्ति च
एवं भूतानि भूतेषु चोदितानीशमायया
वयं च त्वं च ये चेमे तुल्यकालाश्चराचराः
जन्ममृत्योर्यथा पश्चात्प्राङ्नैवमधुनापि भोः
भूतैर्भूतानि भूतेशः सृजत्यवति हन्ति च
आत्मसृष्टैरस्वतन्त्रैरनपेक्षोऽपि बालवत्‌
देहेन देहिनो राजन्देहाद्देहोऽभिजायते
बीजादेव यथा बीजं देह्यर्थ इव शाश्वतः
देहदेहिविभागोऽयमविवेककृतः पुरा
जातिव्यक्तिविभागोऽयं यथा वस्तुनि कल्पितः
श्रीशुक उवाच
एवमाश्वासितो राजा चित्रकेतुर्द्विजोक्तिभिः
विमृज्य पाणिना वक्त्रमाधिम्लानमभाषत
श्रीराजोवाच
कौ युवां ज्ञानसम्पन्नौ महिष्ठौ च महीयसाम्‌
अवधूतेन वेषेण गूढाविह समागतौ
चरन्ति ह्यवनौ कामं ब्राह्मणा भगवत्प्रियाः
मादृशां ग्राम्यबुद्धीनां बोधायोन्मत्तलिङ्गिनः
कुमारो नारद ऋभुरङ्गिरा देवलोऽसितः
अपान्तरतमा व्यासो मार्कण्डेयोऽथ गौतमः
वसिष्ठो भगवान्रामः कपिलो बादरायणिः
दुर्वासा याज्ञवल्क्यश्च जातुकर्णस्तथारुणिः
रोमशश्च्यवनो दत्त आसुरिः सपतञ्जलिः
ऋषिर्वेदशिरा धौम्यो मुनिः पञ्चशिखस्तथा
हिरण्यनाभः कौशल्यः श्रुतदेव ऋतध्वजः
एते परे च सिद्धेशाश्चरन्ति ज्ञानहेतवः
तस्माद्युवां ग्राम्यपशोर्मम मूढधियः प्रभू
अन्धे तमसि मग्नस्य ज्ञानदीप उदीर्यताम्‌
श्रीअङ्गिरा उवाच
अहं ते पुत्रकामस्य पुत्रदोऽस्म्यङ्गिरा नृप
एष ब्रह्मसुतः साक्षान्नारदो भगवानृषिः
इत्थं त्वां पुत्रशोकेन मग्नं तमसि दुस्तरे
अतदर्हमनुस्मृत्य महापुरुषगोचरम्‌
अनुग्रहाय भवतः प्राप्तावावामिह प्रभो
ब्रह्मण्यो भगवद्भक्तो नावासादितुमर्हसि
तदैव ते परं ज्ञानं ददामि गृहमागतः
ज्ञात्वान्याभिनिवेशं ते पुत्रमेव ददाम्यहम्‌
अधुना पुत्रिणां तापो भवतैवानुभूयते
एवं दारा गृहा रायो विविधैश्वर्यसम्पदः
शब्दादयश्च विषयाश्चला राज्यविभूतयः
मही राज्यं बलं कोषो भृत्यामात्यसुहृज्जनाः
सर्वेऽपि शूरसेनेमे शोकमोहभयार्तिदाः
गन्धर्वनगरप्रख्याः स्वप्नमायामनोरथाः
दृश्यमाना विनार्थेन न दृश्यन्ते मनोभवाः
कर्मभिर्ध्यायतो नाना कर्माणि मनसोऽभवन्‌
अयं हि देहिनो देहो द्रव्यज्ञानक्रियात्मकः
देहिनो विविधक्लेश सन्तापकृदुदाहृतः
तस्मात्स्वस्थेन मनसा विमृश्य गतिमात्मनः
द्वैते ध्रुवार्थविश्रम्भं त्यजोपशममाविश
श्रीनारद उवाच
एतां मन्त्रोपनिषदं प्रतीच्छ प्रयतो मम
यां धारयन्सप्तरात्राद्द्रष्टा सङ्कर्षणं विभुम्‌
यत्पादमूलमुपसृत्य नरेन्द्र पूर्वे
शर्वादयो भ्रममिमं द्वितयं विसृज्य
सद्यस्तदीयमतुलानधिकं महित्वं
प्रापुर्भवानपि परं न चिरादुपैति
श्रीबादरायणिरुवाच
अथ देवऋषी राजन्सम्परेतं नृपात्मजम्‌
दर्शयित्वेति होवाच ज्ञातीनामनुशोचताम्‌
श्रीनारद उवाच
जीवात्मन्पश्य भद्रं ते मातरं पितरं च ते
सुहृदो बान्धवास्तप्ताः शुचा त्वत्कृतया भृशम्‌
कलेवरं स्वमाविश्य शेषमायुः सुहृद्वृतः
भुङ्क्ष्व भोगान्पितृप्रत्तानधितिष्ठ नृपासनम्‌
जीव उवाच
कस्मिन्जन्मन्यमी मह्यं पितरो मातरोऽभवन्‌
कर्मभिर्भ्राम्यमाणस्य देवतिर्यङ्नृयोनिषु
बन्धुज्ञात्यरिमध्यस्थ मित्रोदासीनविद्विषः
सर्व एव हि सर्वेषां भवन्ति क्रमशो मिथः
यथा वस्तूनि पण्यानि हेमादीनि ततस्ततः
पर्यटन्ति नरेष्वेवं जीवो योनिषु कर्तृषु
नित्यस्यार्थस्य सम्बन्धो ह्यनित्यो दृश्यते नृषु
यावद्यस्य हि सम्बन्धो ममत्वं तावदेव हि
एवं योनिगतो जीवः स नित्यो निरहङ्कृतः
यावद्यत्रोपलभ्येत तावत्स्वत्वं हि तस्य तत्‌
एष नित्योऽव्ययः सूक्ष्म एष सर्वाश्रयः स्वदृक्‌
आत्ममायागुणैर्विश्वमात्मानं सृजते प्रभुः
न ह्यस्यास्ति प्रियः कश्चिन्नाप्रियः स्वः परोऽपि वा
एकः सर्वधियां द्रष्टा कर्त्ऱ्‌^ईणां गुणदोषयोः
नादत्त आत्मा हि गुणं न दोषं न क्रियाफलम्‌
उदासीनवदासीनः परावरदृगीश्वरः
श्रीबादरायणिरुवाच
इत्युदीर्य गतो जीवो ज्ञातयस्तस्य ते तदा
विस्मिता मुमुचुः शोकं छित्त्वात्मस्नेहशृङ्खलाम्‌
निर्हृत्य ज्ञातयो ज्ञातेर्देहं कृत्वोचिताः क्रियाः
तत्यजुर्दुस्त्यजं स्नेहं शोकमोहभयार्तिदम्‌
बालघ्न्यो व्रीडितास्तत्र बालहत्याहतप्रभाः
बालहत्याव्रतं चेरुर्ब्राह्मणैर्यन्निरूपितम्‌
यमुनायां महाराज स्मरन्त्यो द्विजभाषितम्‌
स इत्थं प्रतिबुद्धात्मा चित्रकेतुर्द्विजोक्तिभिः
गृहान्धकूपान्निष्क्रान्तः सरःपङ्कादिव द्विपः
कालिन्द्यां विधिवत्स्नात्वा कृतपुण्यजलक्रियः
मौनेन संयतप्राणो ब्रह्मपुत्राववन्दत
अथ तस्मै प्रपन्नाय भक्ताय प्रयतात्मने
भगवान्नारदः प्रीतो विद्यामेतामुवाच ह
ओं नमस्तुभ्यं भगवते वासुदेवाय धीमहि
प्रद्युम्नायानिरुद्धाय नमः सङ्कर्षणाय च
नमो विज्ञानमात्राय परमानन्दमूर्तये
आत्मारामाय शान्ताय निवृत्तद्वैतदृष्टये
आत्मानन्दानुभूत्यैव न्यस्तशक्त्यूर्मये नमः
हृषीकेशाय महते नमस्तेऽनन्तमूर्तये
वचस्युपरतेऽप्राप्य य एको मनसा सह
अनामरूपश्चिन्मात्रः सोऽव्यान्नः सदसत्परः
यस्मिन्निदं यतश्चेदं तिष्ठत्यप्येति जायते
मृण्मयेष्विव मृज्जातिस्तस्मै ते ब्रह्मणे नमः
यन्न स्पृशन्ति न विदुर्मनोबुद्धीन्द्रियासवः
अन्तर्बहिश्च विततं व्योमवत्तन्नतोऽस्म्यहम्‌
देहेन्द्रियप्राणमनोधियोऽमी यदंशविद्धाः प्रचरन्ति कर्मसु
नैवान्यदा लौहमिवाप्रतप्तं स्थानेषु तद्द्रष्ट्रपदेशमेति
ओं नमो भगवते महापुरुषाय महानुभावाय महाविभूतिपतये
सकलसात्वतपरिवृढनिकरकरकमलकुड्मलोपलालितचरणारविन्दयुगल परमपरमेष्ठिन्नमस्ते
श्रीशुक उवाच
भक्तायैतां प्रपन्नाय विद्यामादिश्य नारदः
ययावङ्गिरसा साकं धाम स्वायम्भुवं प्रभो
चित्रकेतुस्तु तां विद्यां यथा नारदभाषिताम्‌
धारयामास सप्ताहमब्भक्षः सुसमाहितः
ततः स सप्तरात्रान्ते विद्यया धार्यमाणया
विद्याधराधिपत्यं च लेभेऽप्रतिहतं नृप
ततः कतिपयाहोभिर्विद्ययेद्धमनोगतिः
जगाम देवदेवस्य शेषस्य चरणान्तिकम्‌
मृणालगौरं शितिवाससं स्फुरत्किरीटकेयूरकटित्रकङ्कणम्‌
प्रसन्नवक्त्रारुणलोचनं वृतं ददर्श सिद्धेश्वरमण्डलैः प्रभुम्‌
तद्दर्शनध्वस्तसमस्तकिल्बिषः स्वस्थामलान्तःकरणोऽभ्ययान्मुनिः
प्रवृद्धभक्त्या प्रणयाश्रुलोचनः प्रहृष्टरोमानमदादिपुरुषम्‌
स उत्तमश्लोकपदाब्जविष्टरं प्रेमाश्रुलेशैरुपमेहयन्मुहुः
प्रेमोपरुद्धाखिलवर्णनिर्गमो नैवाशकत्तं प्रसमीडितुं चिरम्‌
ततः समाधाय मनो मनीषया बभाष एतत्प्रतिलब्धवागसौ
नियम्य सर्वेन्द्रियबाह्यवर्तनं जगद्गुरुं सात्वतशास्त्रविग्रहम्‌
चित्रकेतुरुवाच
अजित जितः सममतिभिः साधुभिर्भवान्जितात्मभिर्भवता
विजितास्तेऽपि च भजतामकामात्मनां य आत्मदोऽतिकरुणः
तव विभवः खलु भगवन्जगदुदयस्थितिलयादीनि
विश्वसृजस्तेऽंशांशास्तत्र मृषा स्पर्धन्ति पृथगभिमत्या
परमाणुपरममहतोस्त्वमाद्यन्तान्तरवर्ती त्रयविधुरः
आदावन्तेऽपि च सत्त्वानां यद्ध्रुवं तदेवान्तरालेऽपि
क्षित्यादिभिरेष किलावृतः सप्तभिर्दशगुणोत्तरैरण्डकोशः
यत्र पतत्यणुकल्पः सहाण्डकोटिकोटिभिस्तदनन्तः
विषयतृषो नरपशवो य उपासते विभूतीर्न परं त्वाम्‌
तेषामाशिष ईश तदनु विनश्यन्ति यथा राजकुलम्‌
कामधियस्त्वयि रचिता न परम रोहन्ति यथा करम्भबीजानि
ज्ञानात्मन्यगुणमये गुणगणतोऽस्य द्वन्द्वजालानि
जितमजित तदा भवता यदाह भागवतं धर्ममनवद्यम्‌
निष्किञ्चना ये मुनय आत्मारामा यमुपासतेऽपवर्गाय
विषममतिर्न यत्र नृणां त्वमहमिति मम तवेति च यदन्यत्र
विषमधिया रचितो यः स ह्यविशुद्धः क्षयिष्णुरधर्मबहुलः
कः क्षेमो निजपरयोः कियान्वार्थः स्वपरद्रुहा धर्मेण
स्वद्रोहात्तव कोपः परसम्पीडया च तथाधर्मः
न व्यभिचरति तवेक्षा यया ह्यभिहितो भागवतो धर्मः
स्थिरचरसत्त्वकदम्बेष्वपृथग्धियो यमुपासते त्वार्याः
न हि भगवन्नघटितमिदं त्वद्दर्शनान्नृणामखिलपापक्षयः
यन्नाम सकृच्छ्रवणात्पुक्कशोऽपि विमुच्यते संसारात्‌
अथ भगवन्वयमधुना त्वदवलोकपरिमृष्टाशयमलाः
सुरऋषिणा यत्कथितं तावकेन कथमन्यथा भवति
विदितमनन्त समस्तं तव जगदात्मनो जनैरिहाचरितम्‌
विज्ञाप्यं परमगुरोः कियदिव सवितुरिव खद्योतैः
नमस्तुभ्यं भगवते सकलजगत्स्थितिलयोदयेशाय
दुरवसितात्मगतये कुयोगिनां भिदा परमहंसाय
यं वै श्वसन्तमनु विश्वसृजः श्वसन्ति
यं चेकितानमनु चित्तय उच्चकन्ति
भूमण्डलं सर्षपायति यस्य मूर्ध्नि
तस्मै नमो भगवतेऽस्तु सहस्रमूर्ध्ने
श्रीशुक उवाच
संस्तुतो भगवानेवमनन्तस्तमभाषत
विद्याधरपतिं प्रीतश्चित्रकेतुं कुरूद्वह
श्रीभगवानुवाच
यन्नारदाङ्गिरोभ्यां ते व्याहृतं मेऽनुशासनम्‌
संसिद्धोऽसि तया राजन्विद्यया दर्शनाच्च मे
अहं वै सर्वभूतानि भूतात्मा भूतभावनः
शब्दब्रह्म परं ब्रह्म ममोभे शाश्वती तनू
लोके विततमात्मानं लोकं चात्मनि सन्ततम्‌
उभयं च मया व्याप्तं मयि चैवोभयं कृतम्‌
यथा सुषुप्तः पुरुषो विश्वं पश्यति चात्मनि
आत्मानमेकदेशस्थं मन्यते स्वप्न उत्थितः
एवं जागरणादीनि जीवस्थानानि चात्मनः
मायामात्राणि विज्ञाय तद्द्रष्टारं परं स्मरेत्‌
येन प्रसुप्तः पुरुषः स्वापं वेदात्मनस्तदा
सुखं च निर्गुणं ब्रह्म तमात्मानमवेहि माम्‌
उभयं स्मरतः पुंसः प्रस्वापप्रतिबोधयोः
अन्वेति व्यतिरिच्येत तज्ज्ञानं ब्रह्म तत्परम्‌
यदेतद्विस्मृतं पुंसो मद्भावं भिन्नमात्मनः
ततः संसार एतस्य देहाद्देहो मृतेर्मृतिः
लब्ध्वेह मानुषीं योनिं ज्ञानविज्ञानसम्भवाम्‌
आत्मानं यो न बुद्ध्येत न क्वचित्क्षेममाप्नुयात्‌
स्मृत्वेहायां परिक्लेशं ततः फलविपर्ययम्‌
अभयं चाप्यनीहायां सङ्कल्पाद्विरमेत्कविः
सुखाय दुःखमोक्षाय कुर्वाते दम्पती क्रियाः
ततोऽनिवृत्तिरप्राप्तिर्दुःखस्य च सुखस्य च
एवं विपर्ययं बुद्ध्वा नृणां विज्ञाभिमानिनाम्‌
आत्मनश्च गतिं सूक्ष्मां स्थानत्रयविलक्षणाम्‌
दृष्टश्रुताभिर्मात्राभिर्निर्मुक्तः स्वेन तेजसा
ज्ञानविज्ञानसन्तृप्तो मद्भक्तः पुरुषो भवेत्‌
एतावानेव मनुजैर्योगनैपुण्यबुद्धिभिः
स्वार्थः सर्वात्मना ज्ञेयो यत्परात्मैकदर्शनम्‌
त्वमेतच्छ्रद्धया राजन्नप्रमत्तो वचो मम
ज्ञानविज्ञानसम्पन्नो धारयन्नाशु सिध्यसि
श्रीशुक उवाच
आश्वास्य भगवानित्थं चित्रकेतुं जगद्गुरुः
पश्यतस्तस्य विश्वात्मा ततश्चान्तर्दधे हरिः
श्रीशुक उवाच
यतश्चान्तर्हितोऽनन्तस्तस्यै कृत्वा दिशे नमः
विद्याधरश्चित्रकेतुश्चचार गगने चरः
स लक्षं वर्षलक्षाणामव्याहतबलेन्द्रियः
स्तूयमानो महायोगी मुनिभिः सिद्धचारणैः
कुलाचलेन्द्रद्रोणीषु नानासङ्कल्पसिद्धिषु
रेमे विद्याधरस्त्रीभिर्गापयन्हरिमीश्वरम्‌
एकदा स विमानेन विष्णुदत्तेन भास्वता
गिरिशं ददृशे गच्छन्परीतं सिद्धचारणैः
आलिङ्ग्याङ्कीकृतां देवीं बाहुना मुनिसंसदि
उवाच देव्याः शृण्वन्त्या जहासोच्चैस्तदन्तिके
चित्रकेतुरुवाच
एष लोकगुरुः साक्षाद्धर्मं वक्ता शरीरिणाम्‌
आस्ते मुख्यः सभायां वै मिथुनीभूय भार्यया
जटाधरस्तीव्रतपा ब्रह्मवादिसभापतिः
अङ्कीकृत्य स्त्रियं चास्ते गतह्रीः प्राकृतो यथा
प्रायशः प्राकृताश्चापि स्त्रियं रहसि बिभ्रति
अयं महाव्रतधरो बिभर्ति सदसि स्त्रियम्‌
श्रीशुक उवाच
भगवानपि तच्छ्रुत्वा प्रहस्यागाधधीर्नृप
तूष्णीं बभूव सदसि सभ्याश्च तदनुव्रताः
इत्यतद्वीर्यविदुषि ब्रुवाणे बह्वशोभनम्‌
रुषाह देवी धृष्टाय निर्जितात्माभिमानिने
श्रीपार्वत्युवाच
अयं किमधुना लोके शास्ता दण्डधरः प्रभुः
अस्मद्विधानां दुष्टानां निर्लज्जानां च विप्रकृत्‌
न वेद धर्मं किल पद्मयोनिर्न ब्रह्मपुत्रा भृगुनारदाद्याः
न वै कुमारः कपिलो मनुश्च ये नो निषेधन्त्यतिवर्तिनं हरम्‌
एषामनुध्येयपदाब्जयुग्मं जगद्गुरुं मङ्गलमङ्गलं स्वयम्‌
यः क्षत्रबन्धुः परिभूय सूरीन्प्रशास्ति धृष्टस्तदयं हि दण्ड्यः
नायमर्हति वैकुण्ठ पादमूलोपसर्पणम्‌
सम्भावितमतिः स्तब्धः साधुभिः पर्युपासितम्‌
अतः पापीयसीं योनिमासुरीं याहि दुर्मते
यथेह भूयो महतां न कर्ता पुत्र किल्बिषम्‌
श्रीशुक उवाच
एवं शप्तश्चित्रकेतुर्विमानादवरुह्य सः
प्रसादयामास सतीं मूर्ध्ना नम्रेण भारत
चित्रकेतुरुवाच
प्रतिगृह्णामि ते शापमात्मनोऽञ्जलिनाम्बिके
देवैर्मर्त्याय यत्प्रोक्तं पूर्वदिष्टं हि तस्य तत्‌
संसारचक्र एतस्मिञ्जन्तुरज्ञानमोहितः
भ्राम्यन्सुखं च दुःखं च भुङ्क्ते सर्वत्र सर्वदा
नैवात्मा न परश्चापि कर्ता स्यात्सुखदुःखयोः
कर्तारं मन्यतेऽत्राज्ञ आत्मानं परमेव च
गुणप्रवाह एतस्मिन्कः शापः को न्वनुग्रहः
कः स्वर्गो नरकः को वा किं सुखं दुःखमेव वा
एकः सृजति भूतानि भगवानात्ममायया
एषां बन्धं च मोक्षं च सुखं दुःखं च निष्कलः
न तस्य कश्चिद्दयितः प्रतीपो न ज्ञातिबन्धुर्न परो न च स्वः
समस्य सर्वत्र निरञ्जनस्य सुखे न रागः कुत एव रोषः
तथापि तच्छक्तिविसर्ग एषां सुखाय दुःखाय हिताहिताय
बन्धाय मोक्षाय च मृत्युजन्मनोः शरीरिणां संसृतयेऽवकल्पते
अथ प्रसादये न त्वां शापमोक्षाय भामिनि
यन्मन्यसे ह्यसाधूक्तं मम तत्क्षम्यतां सति
श्रीशुक उवाच
इति प्रसाद्य गिरिशौ चित्रकेतुररिन्दम
जगाम स्वविमानेन पश्यतोः स्मयतोस्तयोः
ततस्तु भगवान्रुद्रो रुद्राणीमिदमब्रवीत्‌
देवर्षिदैत्यसिद्धानां पार्षदानां च शृण्वताम्‌
श्रीरुद्र उवाच
दृष्टवत्यसि सुश्रोणि हरेरद्भुतकर्मणः
माहात्म्यं भृत्यभृत्यानां निःस्पृहाणां महात्मनाम्‌
नारायणपराः सर्वे न कुतश्चन बिभ्यति
स्वर्गापवर्गनरकेष्वपि तुल्यार्थदर्शिनः
देहिनां देहसंयोगाद्द्वन्द्वानीश्वरलीलया
सुखं दुःखं मृतिर्जन्म शापोऽनुग्रह एव च
अविवेककृतः पुंसो ह्यर्थभेद इवात्मनि
गुणदोषविकल्पश्च भिदेव स्रजिवत्कृतः
वासुदेवे भगवति भक्तिमुद्वहतां नृणाम्‌
ज्ञानवैराग्यवीर्याणां न हि कश्चिद्व्यपाश्रयः
नाहं विरिञ्चो न कुमारनारदौ न ब्रह्मपुत्रा मुनयः सुरेशाः
विदाम यस्येहितमंशकांशका न तत्स्वरूपं पृथगीशमानिनः
न ह्यस्यास्ति प्रियः कश्चिन्नाप्रियः स्वः परोऽपि वा
आत्मत्वात्सर्वभूतानां सर्वभूतप्रियो हरिः
तस्य चायं महाभागश्चित्रकेतुः प्रियोऽनुगः
सर्वत्र समदृक्षान्तो ह्यहं चैवाच्युतप्रियः
तस्मान्न विस्मयः कार्यः पुरुषेषु महात्मसु
महापुरुषभक्तेषु शान्तेषु समदर्शिषु
श्रीशुक उवाच
इति श्रुत्वा भगवतः शिवस्योमाभिभाषितम्‌
बभूव शान्तधी राजन्देवी विगतविस्मया
इति भागवतो देव्याः प्रतिशप्तुमलन्तमः
मूर्ध्ना स जगृहे शापमेतावत्साधुलक्षणम्‌
जज्ञे त्वष्टुर्दक्षिणाग्नौ दानवीं योनिमाश्रितः
वृत्र इत्यभिविख्यातो ज्ञानविज्ञानसंयुतः
एतत्ते सर्वमाख्यातं यन्मां त्वं परिपृच्छसि
वृत्रस्यासुरजातेश्च कारणं भगवन्मतेः
इतिहासमिमं पुण्यं चित्रकेतोर्महात्मनः
माहात्म्यं विष्णुभक्तानां श्रुत्वा बन्धाद्विमुच्यते
य एतत्प्रातरुत्थाय श्रद्धया वाग्यतः पठेत्‌
इतिहासं हरिं स्मृत्वा स याति परमां गतिम्‌
श्रीशुक उवाच
पृश्निस्तु पत्नी सवितुः सावित्रीं व्याहृतिं त्रयीम्‌
अग्निहोत्रं पशुं सोमं चातुर्मास्यं महामखान्‌
सिद्धिर्भगस्य भार्याङ्ग महिमानं विभुं प्रभुम्‌
आशिषं च वरारोहां कन्यां प्रासूत सुव्रताम्‌
धातुः कुहूः सिनीवाली राका चानुमतिस्तथा
सायं दर्शमथ प्रातः पूर्णमासमनुक्रमात्‌
अग्नीन्पुरीष्यानाधत्त क्रियायां समनन्तरः
चर्षणी वरुणस्यासीद्यस्यां जातो भृगुः पुनः
वाल्मीकिश्च महायोगी वल्मीकादभवत्किल
अगस्त्यश्च वसिष्ठश्च मित्रावरुणयोरृषी
रेतः सिषिचतुः कुम्भे उर्वश्याः सन्निधौ द्रुतम्‌
रेवत्यां मित्र उत्सर्गमरिष्टं पिप्पलं व्यधात्‌
पौलोम्यामिन्द्र आधत्त त्रीन्पुत्रानिति नः श्रुतम्‌
जयन्तमृषभं तात तृतीयं मीढुषं प्रभुः
उरुक्रमस्य देवस्य मायावामनरूपिणः
कीर्तौ पत्न्यां बृहच्छ्लोकस्तस्यासन्सौभगादयः
तत्कर्मगुणवीर्याणि काश्यपस्य महात्मनः
पश्चाद्वक्ष्यामहेऽदित्यां यथैवावततार ह
अथ कश्यपदायादान्दैतेयान्कीर्तयामि ते
यत्र भागवतः श्रीमान्प्रह्रादो बलिरेव च
दितेर्द्वावेव दायादौ दैत्यदानववन्दितौ
हिरण्यकशिपुर्नाम हिरण्याक्षश्च कीर्तितौ
हिरण्यकशिपोर्भार्या कयाधुर्नाम दानवी
जम्भस्य तनया सा तु सुषुवे चतुरः सुतान्‌
संह्रादं प्रागनुह्रादं ह्रादं प्रह्रादमेव च
तत्स्वसा सिंहिका नाम राहुं विप्रचितोऽग्रहीत्‌
शिरोऽहरद्यस्य हरिश्चक्रेण पिबतोऽमृतम्‌
संह्रादस्य कृतिर्भार्या सूत पञ्चजनं ततः
ह्रादस्य धमनिर्भार्या सूत वातापिमिल्वलम्‌
योऽगस्त्याय त्वतिथये पेचे वातापिमिल्वलः
अनुह्रादस्य सूर्यायां बाष्कलो महिषस्तथा
विरोचनस्तु प्राह्रादिर्देव्यां तस्याभवद्बलिः
बाणज्येष्ठं पुत्रशतमशनायां ततोऽभवत्‌
तस्यानुभावं सुश्लोक्यं पश्चादेवाभिधास्यते
बाण आराध्य गिरिशं लेभे तद्गणमुख्यताम्‌
यत्पार्श्वे भगवानास्ते ह्यद्यापि पुरपालकः
मरुतश्च दितेः पुत्राश्चत्वारिंशन्नवाधिकाः
त आसन्नप्रजाः सर्वे नीता इन्द्रेण सात्मताम्‌
श्रीराजोवाच
कथं त आसुरं भावमपोह्यौत्पत्तिकं गुरो
इन्द्रेण प्रापिताः सात्म्यं किं तत्साधु कृतं हि तैः
इमे श्रद्दधते ब्रह्मन्नृषयो हि मया सह
परिज्ञानाय भगवंस्तन्नो व्याख्यातुमर्हसि
श्रीसूत उवाच
तद्विष्णुरातस्य स बादरायणिर्वचो निशम्यादृतमल्पमर्थवत्‌
सभाजयन्सन्निभृतेन चेतसा जगाद सत्रायण सर्वदर्शनः
श्रीशुक उवाच
हतपुत्रा दितिः शक्र पार्ष्णिग्राहेण विष्णुना
मन्युना शोकदीप्तेन ज्वलन्ती पर्यचिन्तयत्‌
कदा नु भ्रातृहन्तारमिन्द्रियाराममुल्बणम्‌
अक्लिन्नहृदयं पापं घातयित्वा शये सुखम्‌
कृमिविड्भस्मसंज्ञासीद्यस्येशाभिहितस्य च
भूतध्रुक्तत्कृते स्वार्थं किं वेद निरयो यतः
आशासानस्य तस्येदं ध्रुवमुन्नद्धचेतसः
मदशोषक इन्द्रस्य भूयाद्येन सुतो हि मे
इति भावेन सा भर्तुराचचारासकृत्प्रियम्‌
शुश्रूषयानुरागेण प्रश्रयेण दमेन च
भक्त्या परमया राजन्मनोज्ञैर्वल्गुभाषितैः
मनो जग्राह भावज्ञा सस्मितापाङ्गवीक्षणैः
एवं स्त्रिया जडीभूतो विद्वानपि मनोज्ञया
बाढमित्याह विवशो न तच्चित्रं हि योषिति
विलोक्यैकान्तभूतानि भूतान्यादौ प्रजापतिः
स्त्रियं चक्रे स्वदेहार्धं यया पुंसां मतिर्हृता
एवं शुश्रूषितस्तात भगवान्कश्यपः स्त्रिया
प्रहस्य परमप्रीतो दितिमाहाभिनन्द्य च
श्रीकश्यप उवाच
वरं वरय वामोरु प्रीतस्तेऽहमनिन्दिते
स्त्रिया भर्तरि सुप्रीते कः काम इह चागमः
पतिरेव हि नारीणां दैवतं परमं स्मृतम्‌
मानसः सर्वभूतानां वासुदेवः श्रियः पतिः
स एव देवतालिङ्गैर्नामरूपविकल्पितैः
इज्यते भगवान्पुम्भिः स्त्रीभिश्च पतिरूपधृक्‌
तस्मात्पतिव्रता नार्यः श्रेयस्कामाः सुमध्यमे
यजन्तेऽनन्यभावेन पतिमात्मानमीश्वरम्‌
सोऽहं त्वयार्चितो भद्रे ईदृग्भावेन भक्तितः
तं ते सम्पादये काममसतीनां सुदुर्लभम्‌
दितिरुवाच
वरदो यदि मे ब्रह्मन्पुत्रमिन्द्रहणं वृणे
अमृत्युं मृतपुत्राहं येन मे घातितौ सुतौ
निशम्य तद्वचो विप्रो विमनाः पर्यतप्यत
अहो अधर्मः सुमहानद्य मे समुपस्थितः
अहो अर्थेन्द्रियारामो योषिन्मय्येह मायया
गृहीतचेताः कृपणः पतिष्ये नरके ध्रुवम्‌
कोऽतिक्रमोऽनुवर्तन्त्याः स्वभावमिह योषितः
धिङ्मां बताबुधं स्वार्थे यदहं त्वजितेन्द्रियः
शरत्पद्मोत्सवं वक्त्रं वचश्च श्रवणामृतम्‌
हृदयं क्षुरधाराभं स्त्रीणां को वेद चेष्टितम्‌
न हि कश्चित्प्रियः स्त्रीणामञ्जसा स्वाशिषात्मनाम्‌
पतिं पुत्रं भ्रातरं वा घ्नन्त्यर्थे घातयन्ति च
प्रतिश्रुतं ददामीति वचस्तन्न मृषा भवेत्‌
वधं नार्हति चेन्द्रोऽपि तत्रेदमुपकल्पते
इति सञ्चिन्त्य भगवान्मारीचः कुरुनन्दन
उवाच किञ्चित्कुपित आत्मानं च विगर्हयन्‌
श्रीकश्यप उवाच
पुत्रस्ते भविता भद्रे इन्द्रहादेवबान्धवः
संवत्सरं व्रतमिदं यद्यञ्जो धारयिष्यसि
दितिरुवाच
धारयिष्ये व्रतं ब्रह्मन्ब्रूहि कार्याणि यानि मे
यानि चेह निषिद्धानि न व्रतं घ्नन्ति यान्युत
श्रीकश्यप उवाच
न हिंस्याद्भूतजातानि न शपेन्नानृतं वदेत्‌
न छिन्द्यान्नखरोमाणि न स्पृशेद्यदमङ्गलम्‌
नाप्सु स्नायान्न कुप्येत न सम्भाषेत दुर्जनैः
न वसीताधौतवासः स्रजं च विधृतां क्वचित्‌
नोच्छिष्टं चण्डिकान्नं च सामिषं वृषलाहृतम्‌
भुञ्जीतोदक्यया दृष्टं पिबेन्नाञ्जलिना त्वपः
नोच्छिष्टास्पृष्टसलिला सन्ध्यायां मुक्तमूर्धजा
अनर्चितासंयतवाक्नासंवीता बहिश्चरेत्‌
नाधौतपादाप्रयता नार्द्रपादा उदक्षिराः
शयीत नापराङ्नान्यैर्न नग्ना न च सन्ध्ययोः
धौतवासा शुचिर्नित्यं सर्वमङ्गलसंयुता
पूजयेत्प्रातराशात्प्राग्गोविप्राञ्श्रियमच्युतम्‌
स्त्रियो वीरवतीश्चार्चेत्स्रग्गन्धबलिमण्डनैः
पतिं चार्च्योपतिष्ठेत ध्यायेत्कोष्ठगतं च तम्‌
सांवत्सरं पुंसवनं व्रतमेतदविप्लुतम्‌
धारयिष्यसि चेत्तुभ्यं शक्रहा भविता सुतः
बाढमित्यभ्युपेत्याथ दिती राजन्महामनाः
कश्यपाद्गर्भमाधत्त व्रतं चाञ्जो दधार सा
मातृष्वसुरभिप्रायमिन्द्र आज्ञाय मानद
शुश्रूषणेनाश्रमस्थां दितिं पर्यचरत्कविः
नित्यं वनात्सुमनसः फलमूलसमित्कुशान्‌
पत्राङ्कुरमृदोऽपश्च काले काल उपाहरत्‌
एवं तस्या व्रतस्थाया व्रतच्छिद्रं हरिर्नृप
प्रेप्सुः पर्यचरज्जिह्मो मृगहेव मृगाकृतिः
नाध्यगच्छद्व्रतच्छिद्रं तत्परोऽथ महीपते
चिन्तां तीव्रां गतः शक्रः केन मे स्याच्छिवं त्विह
एकदा सा तु सन्ध्यायामुच्छिष्टा व्रतकर्शिता
अस्पृष्टवार्यधौताङ्घ्रिः सुष्वाप विधिमोहिता
लब्ध्वा तदन्तरं शक्रो निद्रापहृतचेतसः
दितेः प्रविष्ट उदरं योगेशो योगमायया
चकर्त सप्तधा गर्भं वज्रेण कनकप्रभम्‌
रुदन्तं सप्तधैकैकं मा रोदीरिति तान्पुनः
तमूचुः पाट्यमानास्ते सर्वे प्राञ्जलयो नृप
किं न इन्द्र जिघांससि भ्रातरो मरुतस्तव
मा भैष्ट भ्रातरो मह्यं यूयमित्याह कौशिकः
अनन्यभावान्पार्षदानात्मनो मरुतां गणान्‌
न ममार दितेर्गर्भः श्रीनिवासानुकम्पया
बहुधा कुलिशक्षुण्णो द्रौण्यस्त्रेण यथा भवान्‌
सकृदिष्ट्वादिपुरुषं पुरुषो याति साम्यताम्‌
संवत्सरं किञ्चिदूनं दित्या यद्धरिरर्चितः
सजूरिन्द्रेण पञ्चाशद्देवास्ते मरुतोऽभवन्‌
व्यपोह्य मातृदोषं ते हरिणा सोमपाः कृताः
दितिरुत्थाय ददृशे कुमाराननलप्रभान्‌
इन्द्रेण सहितान्देवी पर्यतुष्यदनिन्दिता
अथेन्द्रमाह ताताहमादित्यानां भयावहम्‌
अपत्यमिच्छन्त्यचरं व्रतमेतत्सुदुष्करम्‌
एकः सङ्कल्पितः पुत्रः सप्त सप्ताभवन्कथम्‌
यदि ते विदितं पुत्र सत्यं कथय मा मृषा
इन्द्र उवाच
अम्ब तेऽहं व्यवसितमुपधार्यागतोऽन्तिकम्‌
लब्धान्तरोऽच्छिदं गर्भमर्थबुद्धिर्न धर्मदृक्‌
कृत्तो मे सप्तधा गर्भ आसन्सप्त कुमारकाः
तेऽपि चैकैकशो वृक्णाः सप्तधा नापि मम्रिरे
ततस्तत्परमाश्चर्यं वीक्ष्य व्यवसितं मया
महापुरुषपूजायाः सिद्धिः काप्यानुषङ्गिणी
आराधनं भगवत ईहमाना निराशिषः
ये तु नेच्छन्त्यपि परं ते स्वार्थकुशलाः स्मृताः
आराध्यात्मप्रदं देवं स्वात्मानं जगदीश्वरम्‌
को वृणीत गुणस्पर्शं बुधः स्यान्नरकेऽपि यत्‌
तदिदं मम दौर्जन्यं बालिशस्य महीयसि
क्षन्तुमर्हसि मातस्त्वं दिष्ट्या गर्भो मृतोत्थितः
श्रीशुक उवाच
इन्द्रस्तयाभ्यनुज्ञातः शुद्धभावेन तुष्टया
मरुद्भिः सह तां नत्वा जगाम त्रिदिवं प्रभुः
एवं ते सर्वमाख्यातं यन्मां त्वं परिपृच्छसि
मङ्गलं मरुतां जन्म किं भूयः कथयामि ते
श्रीराजोवाच
व्रतं पुंसवनं ब्रह्मन्भवता यदुदीरितम्‌
तस्य वेदितुमिच्छामि येन विष्णुः प्रसीदति
श्रीशुक उवाच
शुक्ले मार्गशिरे पक्षे योषिद्भर्तुरनुज्ञया
आरभेत व्रतमिदं सार्वकामिकमादितः
निशम्य मरुतां जन्म ब्राह्मणाननुमन्त्र्य च
स्नात्वा शुक्लदती शुक्ले वसीतालङ्कृताम्बरे
पूजयेत्प्रातराशात्प्राग्भगवन्तं श्रिया सह
अलं ते निरपेक्षाय पूर्णकाम नमोऽस्तु ते
महाविभूतिपतये नमः सकलसिद्धये
यथा त्वं कृपया भूत्या तेजसा महिमौजसा
जुष्ट ईश गुणैः सर्वैस्ततोऽसि भगवान्प्रभुः
विष्णुपत्नि महामाये महापुरुषलक्षणे
प्रीयेथा मे महाभागे लोकमातर्नमोऽस्तु ते
ओं नमो भगवते महापुरुषाय महानुभावाय महाविभूतिपतये सह
महाविभूतिभिर्बलिमुपहरामीति अनेनाहरहर्मन्त्रेण
विष्णोरावाहनार्घ्यपाद्योपस्पर्शनस्नानवासौपवीतविभूषणगन्धपुष्पधूपदीपोपहाराद्युपचारान्सुस्‌
अमाहितोपाहरेत्‌
हविःशेषं च जुहुयादनले द्वादशाहुतीः
ओं नमो भगवते महापुरुषाय महाविभूतिपतये स्वाहेति
श्रियं विष्णुं च वरदावाशिषां प्रभवावुभौ
भक्त्या सम्पूजयेन्नित्यं यदीच्छेत्सर्वसम्पदः
प्रणमेद्दण्डवद्भूमौ भक्तिप्रह्वेण चेतसा
दशवारं जपेन्मन्त्रं ततः स्तोत्रमुदीरयेत्‌
युवां तु विश्वस्य विभू जगतः कारणं परम्‌
इयं हि प्रकृतिः सूक्ष्मा मायाशक्तिर्दुरत्यया
तस्या अधीश्वरः साक्षात्त्वमेव पुरुषः परः
त्वं सर्वयज्ञ इज्येयं क्रियेयं फलभुग्भवान्‌
गुणव्यक्तिरियं देवी व्यञ्जको गुणभुग्भवान्‌
त्वं हि सर्वशरीर्यात्मा श्रीः शरीरेन्द्रियाशयाः
नामरूपे भगवती प्रत्ययस्त्वमपाश्रयः
यथा युवां त्रिलोकस्य वरदौ परमेष्ठिनौ
तथा म उत्तमश्लोक सन्तु सत्या महाशिषः
इत्यभिष्टूय वरदं श्रीनिवासं श्रिया सह
तन्निःसार्योपहरणं दत्त्वाचमनमर्चयेत्‌
ततः स्तुवीत स्तोत्रेण भक्तिप्रह्वेण चेतसा
यज्ञोच्छिष्टमवघ्राय पुनरभ्यर्चयेद्धरिम्‌
पतिं च परया भक्त्या महापुरुषचेतसा
प्रियैस्तैस्तैरुपनमेत्प्रेमशीलः स्वयं पतिः
बिभृयात्सर्वकर्माणि पत्न्या उच्चावचानि च
कृतमेकतरेणापि दम्पत्योरुभयोरपि
पत्न्यां कुर्यादनर्हायां पतिरेतत्समाहितः
विष्णोर्व्रतमिदं बिभ्रन्न विहन्यात्कथञ्चन
विप्रान्स्त्रियो वीरवतीः स्रग्गन्धबलिमण्डनैः
अर्चेदहरहर्भक्त्या देवं नियममास्थिता
उद्वास्य देवं स्वे धाम्नि तन्निवेदितमग्रतः
अद्यादात्मविशुद्ध्यर्थं सर्वकामसमृद्धये
एतेन पूजाविधिना मासान्द्वादश हायनम्‌
नीत्वाथोपरमेत्साध्वी कार्तिके चरमेऽहनि
श्वोभूतेऽप उपस्पृश्य कृष्णमभ्यर्च्य पूर्ववत्‌
पयःशृतेन जुहुयाच्चरुणा सह सर्पिषा
पाकयज्ञविधानेन द्वादशैवाहुतीः पतिः
आशिषः शिरसादाय द्विजैः प्रीतैः समीरिताः
प्रणम्य शिरसा भक्त्या भुञ्जीत तदनुज्ञया
आचार्यमग्रतः कृत्वा वाग्यतः सह बन्धुभिः
दद्यात्पत्न्यै चरोः शेषं सुप्रजास्त्वं सुसौभगम्‌
एतच्चरित्वा विधिवद्व्रतं विभोरभीप्सितार्थं लभते पुमानिह
स्त्री चैतदास्थाय लभेत सौभगं श्रियं प्रजां जीवपतिं यशो गृहम्‌
कन्या च विन्देत समग्रलक्षणं पतिं त्ववीरा हतकिल्बिषां गतिम्‌
मृतप्रजा जीवसुता धनेश्वरी सुदुर्भगा सुभगा रूपमग्र्यम्‌
विन्देद्विरूपा विरुजा विमुच्यते य आमयावीन्द्रियकल्यदेहम्‌
एतत्पठन्नभ्युदये च कर्मण्यनन्ततृप्तिः पितृदेवतानाम्‌
तुष्टाः प्रयच्छन्ति समस्तकामान्होमावसाने हुतभुक्ष्रीहरिश्च
राजन्महन्मरुतां जन्म पुण्यं दितेर्व्रतं चाभिहितं महत्ते
श्रीराजोवाच
समः प्रियः सुहृद्ब्रह्मन्भूतानां भगवान्स्वयम्‌
इन्द्रस्यार्थे कथं दैत्यानवधीद्विषमो यथा
न ह्यस्यार्थः सुरगणैः साक्षान्निःश्रेयसात्मनः
नैवासुरेभ्यो विद्वेषो नोद्वेगश्चागुणस्य हि
इति नः सुमहाभाग नारायणगुणान्प्रति
संशयः सुमहान्जातस्तद्भवांश्छेत्तुमर्हति
श्रीऋषिरुवाच
साधु पृष्टं महाराज हरेश्चरितमद्भुतम्‌
यद्भागवतमाहात्म्यं भगवद्भक्तिवर्धनम्‌
गीयते परमं पुण्यमृषिभिर्नारदादिभिः
नत्वा कृष्णाय मुनये कथयिष्ये हरेः कथाम्‌
निर्गुणोऽपि ह्यजोऽव्यक्तो भगवान्प्रकृतेः परः
स्वमायागुणमाविश्य बाध्यबाधकतां गतः
सत्त्वं रजस्तम इति प्रकृतेर्नात्मनो गुणाः
न तेषां युगपद्राजन्ह्रास उल्लास एव वा
जयकाले तु सत्त्वस्य देवर्षीन्रजसोऽसुरान्‌
तमसो यक्षरक्षांसि तत्कालानुगुणोऽभजत्‌
ज्योतिरादिरिवाभाति सङ्घातान्न विविच्यते
विदन्त्यात्मानमात्मस्थं मथित्वा कवयोऽन्ततः
यदा सिसृक्षुः पुर आत्मनः परो रजः सृजत्येष पृथक्स्वमायया
सत्त्वं विचित्रासु रिरंसुरीश्वरः शयिष्यमाणस्तम ईरयत्यसौ
कालं चरन्तं सृजतीश आश्रयं प्रधानपुम्भ्यां नरदेव सत्यकृत्‌
य एष राजन्नपि काल ईशिता सत्त्वं सुरानीकमिवैधयत्यतः
तत्प्रत्यनीकानसुरान्सुरप्रियो रजस्तमस्कान्प्रमिणोत्युरुश्रवाः
अत्रैवोदाहृतः पूर्वमितिहासः सुरर्षिणा
प्रीत्या महाक्रतौ राजन्पृच्छतेऽजातशत्रवे
दृष्ट्वा महाद्भुतं राजा राजसूये महाक्रतौ
वासुदेवे भगवति सायुज्यं चेदिभूभुजः
तत्रासीनं सुरऋषिं राजा पाण्डुसुतः क्रतौ
पप्रच्छ विस्मितमना मुनीनां शृण्वतामिदम्‌
श्रीयुधिष्ठिर उवाच
अहो अत्यद्भुतं ह्येतद्दुर्लभैकान्तिनामपि
वासुदेवे परे तत्त्वे प्राप्तिश्चैद्यस्य विद्विषः
एतद्वेदितुमिच्छामः सर्व एव वयं मुने
भगवन्निन्दया वेनो द्विजैस्तमसि पातितः
दमघोषसुतः पाप आरभ्य कलभाषणात्‌
सम्प्रत्यमर्षी गोविन्दे दन्तवक्रश्च दुर्मतिः
शपतोरसकृद्विष्णुं यद्ब्रह्म परमव्ययम्‌
श्वित्रो न जातो जिह्वायां नान्धं विविशतुस्तमः
कथं तस्मिन्भगवति दुरवग्राह्यधामनि
पश्यतां सर्वलोकानां लयमीयतुरञ्जसा
एतद्भ्राम्यति मे बुद्धिर्दीपार्चिरिव वायुना
ब्रूह्येतदद्भुततमं भगवान्ह्यत्र कारणम्‌
श्रीबादरायणिरुवाच
राज्ञस्तद्वच आकर्ण्य नारदो भगवानृषिः
तुष्टः प्राह तमाभाष्य शृण्वत्यास्तत्सदः कथाः
श्रीनारद उवाच
निन्दनस्तवसत्कार न्यक्कारार्थं कलेवरम्‌
प्रधानपरयो राजन्नविवेकेन कल्पितम्‌
हिंसा तदभिमानेन दण्डपारुष्ययोर्यथा
वैषम्यमिह भूतानां ममाहमिति पार्थिव
यन्निबद्धोऽभिमानोऽयं तद्वधात्प्राणिनां वधः
तथा न यस्य कैवल्यादभिमानोऽखिलात्मनः
परस्य दमकर्तुर्हि हिंसा केनास्य कल्प्यते
तस्माद्वैरानुबन्धेन निर्वैरेण भयेन वा
स्नेहात्कामेन वा युञ्ज्यात्कथञ्चिन्नेक्षते पृथक्‌
यथा वैरानुबन्धेन मर्त्यस्तन्मयतामियात्‌
न तथा भक्तियोगेन इति मे निश्चिता मतिः
कीटः पेशस्कृता रुद्धः कुड्यायां तमनुस्मरन्‌
संरम्भभययोगेन विन्दते तत्स्वरूपताम्‌
एवं कृष्णे भगवति मायामनुज ईश्वरे
वैरेण पूतपाप्मानस्तमापुरनुचिन्तया
कामाद्द्वेषाद्भयात्स्नेहाद्यथा भक्त्येश्वरे मनः
आवेश्य तदघं हित्वा बहवस्तद्गतिं गताः
गोप्यः कामाद्भयात्कंसो द्वेषाच्चैद्यादयो नृपाः
सम्बन्धाद्वृष्णयः स्नेहाद्यूयं भक्त्या वयं विभो
कतमोऽपि न वेनः स्यात्पञ्चानां पुरुषं प्रति
तस्मात्केनाप्युपायेन मनः कृष्णे निवेशयेत्‌
मातृष्वस्रेयो वश्चैद्यो दन्तवक्रश्च पाण्डव
पार्षदप्रवरौ विष्णोर्विप्रशापात्पदच्युतौ
श्रीयुधिष्ठिर उवाच
कीदृशः कस्य वा शापो हरिदासाभिमर्शनः
अश्रद्धेय इवाभाति हरेरेकान्तिनां भवः
देहेन्द्रियासुहीनानां वैकुण्ठपुरवासिनाम्‌
देहसम्बन्धसम्बद्धमेतदाख्यातुमर्हसि
श्रीनारद उवाच
एकदा ब्रह्मणः पुत्रा विष्णुलोकं यदृच्छया
सनन्दनादयो जग्मुश्चरन्तो भुवनत्रयम्‌
पञ्चषड्ढायनार्भाभाः पूर्वेषामपि पूर्वजाः
दिग्वाससः शिशून्मत्वा द्वाःस्थौ तान्प्रत्यषेधताम्‌
अशपन्कुपिता एवं युवां वासं न चार्हथः
रजस्तमोभ्यां रहिते पादमूले मधुद्विषः
पापिष्ठामासुरीं योनिं बालिशौ यातमाश्वतः
एवं शप्तौ स्वभवनात्पतन्तौ तौ कृपालुभिः
प्रोक्तौ पुनर्जन्मभिर्वां त्रिभिर्लोकाय कल्पताम्‌
जज्ञाते तौ दितेः पुत्रौ दैत्यदानववन्दितौ
हिरण्यकशिपुर्ज्येष्ठो हिरण्याक्षोऽनुजस्ततः
हतो हिरण्यकशिपुर्हरिणा सिंहरूपिणा
हिरण्याक्षो धरोद्धारे बिभ्रता शौकरं वपुः
हिरण्यकशिपुः पुत्रं प्रह्लादं केशवप्रियम्‌
जिघांसुरकरोन्नाना यातना मृत्युहेतवे
तं सर्वभूतात्मभूतं प्रशान्तं समदर्शनम्‌
भगवत्तेजसा स्पृष्टं नाशक्नोद्धन्तुमुद्यमैः
ततस्तौ राक्षसौ जातौ केशिन्यां विश्रवःसुतौ
रावणः कुम्भकर्णश्च सर्वलोकोपतापनौ
तत्रापि राघवो भूत्वा न्यहनच्छापमुक्तये
रामवीर्यं श्रोष्यसि त्वं मार्कण्डेयमुखात्प्रभो
तावत्र क्षत्रियौ जातौ मातृष्वस्रात्मजौ तव
अधुना शापनिर्मुक्तौ कृष्णचक्रहतांहसौ
वैरानुबन्धतीव्रेण ध्यानेनाच्युतसात्मताम्‌
नीतौ पुनर्हरेः पार्श्वं जग्मतुर्विष्णुपार्षदौ
श्रीयुधिष्ठिर उवाच
विद्वेषो दयिते पुत्रे कथमासीन्महात्मनि
ब्रूहि मे भगवन्येन प्रह्लादस्याच्युतात्मता
श्रीनारद उवाच
भ्रातर्येवं विनिहते हरिणा क्रोडमूर्तिना
हिरण्यकशिपू राजन्पर्यतप्यद्रुषा शुचा
आह चेदं रुषा पूर्णः सन्दष्टदशनच्छदः
कोपोज्ज्वलद्भ्यां चक्षुर्भ्यां निरीक्षन्धूम्रमम्बरम्‌
करालदंष्ट्रोग्रदृष्ट्या दुष्प्रेक्ष्यभ्रुकुटीमुखः
शूलमुद्यम्य सदसि दानवानिदमब्रवीत्‌
भो भो दानवदैतेया द्विमूर्धंस्त्र्यक्ष शम्बर
शतबाहो हयग्रीव नमुचे पाक इल्वल
विप्रचित्ते मम वचः पुलोमन्शकुनादयः
शृणुतानन्तरं सर्वे क्रियतामाशु मा चिरम्‌
सपत्नैर्घातितः क्षुद्रैर्भ्राता मे दयितः सुहृत्‌
पार्ष्णिग्राहेण हरिणा समेनाप्युपधावनैः
तस्य त्यक्तस्वभावस्य घृणेर्मायावनौकसः
भजन्तं भजमानस्य बालस्येवास्थिरात्मनः
मच्छूलभिन्नग्रीवस्य भूरिणा रुधिरेण वै
असृक्प्रियं तर्पयिष्ये भ्रातरं मे गतव्यथः
तस्मिन्कूटेऽहिते नष्टे कृत्तमूले वनस्पतौ
विटपा इव शुष्यन्ति विष्णुप्राणा दिवौकसः
तावद्यात भुवं यूयं ब्रह्मक्षत्रसमेधिताम्‌
सूदयध्वं तपोयज्ञ स्वाध्यायव्रतदानिनः
विष्णुर्द्विजक्रियामूलो यज्ञो धर्ममयः पुमान्‌
देवर्षिपितृभूतानां धर्मस्य च परायणम्‌
यत्र यत्र द्विजा गावो वेदा वर्णाश्रमक्रियाः
तं तं जनपदं यात सन्दीपयत वृश्चत
इति ते भर्तृनिर्देशमादाय शिरसादृताः
तथा प्रजानां कदनं विदधुः कदनप्रियाः
पुरग्रामव्रजोद्यान क्षेत्रारामाश्रमाकरान्‌
खेटखर्वटघोषांश्च ददहुः पत्तनानि च
केचित्खनित्रैर्बिभिदुः सेतुप्राकारगोपुरान्‌
आजीव्यांश्चिच्छिदुर्वृक्षान्केचित्परशुपाणयः
प्रादहन्शरणान्येके प्रजानां ज्वलितोल्मुकैः
एवं विप्रकृते लोके दैत्येन्द्रानुचरैर्मुहुः
दिवं देवाः परित्यज्य भुवि चेरुरलक्षिताः
हिरण्यकशिपुर्भ्रातुः सम्परेतस्य दुःखितः
कृत्वा कटोदकादीनि भ्रातृपुत्रानसान्त्वयत्‌
शकुनिं शम्बरं धृष्टिं भूतसन्तापनं वृकम्‌
कालनाभं महानाभं हरिश्मश्रुमथोत्कचम्‌
तन्मातरं रुषाभानुं दितिं च जननीं गिरा
श्लक्ष्णया देशकालज्ञ इदमाह जनेश्वर
श्रीहिरण्यकशिपुरुवाच
अम्बाम्ब हे वधूः पुत्रा वीरं मार्हथ शोचितुम्‌
रिपोरभिमुखे श्लाघ्यः शूराणां वध ईप्सितः
भूतानामिह संवासः प्रपायामिव सुव्रते
दैवेनैकत्र नीतानामुन्नीतानां स्वकर्मभिः
नित्य आत्माव्ययः शुद्धः सर्वगः सर्ववित्परः
धत्तेऽसावात्मनो लिङ्गं मायया विसृजन्गुणान्‌
यथाम्भसा प्रचलता तरवोऽपि चला इव
चक्षुषा भ्राम्यमाणेन दृश्यते चलतीव भूः
एवं गुणैर्भ्राम्यमाणे मनस्यविकलः पुमान्‌
याति तत्साम्यतां भद्रे ह्यलिङ्गो लिङ्गवानिव
एष आत्मविपर्यासो ह्यलिङ्गे लिङ्गभावना
एष प्रियाप्रियैर्योगो वियोगः कर्मसंसृतिः
सम्भवश्च विनाशश्च शोकश्च विविधः स्मृतः
अविवेकश्च चिन्ता च विवेकास्मृतिरेव च
अत्राप्युदाहरन्तीममितिहासं पुरातनम्‌
यमस्य प्रेतबन्धूनां संवादं तं निबोधत
उशीनरेष्वभूद्राजा सुयज्ञ इति विश्रुतः
सपत्नैर्निहतो युद्धे ज्ञातयस्तमुपासत
विशीर्णरत्नकवचं विभ्रष्टाभरणस्रजम्‌
शरनिर्भिन्नहृदयं शयानमसृगाविलम्‌
प्रकीर्णकेशं ध्वस्ताक्षं रभसा दष्टदच्छदम्‌
रजःकुण्ठमुखाम्भोजं छिन्नायुधभुजं मृधे
उशीनरेन्द्रं विधिना तथा कृतं पतिं महिष्यः प्रसमीक्ष्य दुःखिताः
हताः स्म नाथेति करैरुरो भृशं घ्नन्त्यो मुहुस्तत्पदयोरुपापतन्‌
रुदत्य उच्चैर्दयिताङ्घ्रिपङ्कजं सिञ्चन्त्य अस्रैः कुचकुङ्कुमारुणैः
विस्रस्तकेशाभरणाः शुचं नृणां सृजन्त्य आक्रन्दनया विलेपिरे
अहो विधात्राकरुणेन नः प्रभो भवान्प्रणीतो दृगगोचरां दशाम्‌
उशीनराणामसि वृत्तिदः पुरा कृतोऽधुना येन शुचां विवर्धनः
त्वया कृतज्ञेन वयं महीपते कथं विना स्याम सुहृत्तमेन ते
तत्रानुयानं तव वीर पादयोः शुश्रूषतीनां दिश यत्र यास्यसि
एवं विलपतीनां वै परिगृह्य मृतं पतिम्‌
अनिच्छतीनां निर्हारमर्कोऽस्तं सन्न्यवर्तत
तत्र ह प्रेतबन्धूनामाश्रुत्य परिदेवितम्‌
आह तान्बालको भूत्वा यमः स्वयमुपागतः
श्रीयम उवाच
अहो अमीषां वयसाधिकानां विपश्यतां लोकविधिं विमोहः
यत्रागतस्तत्र गतं मनुष्यं स्वयं सधर्मा अपि शोचन्त्यपार्थम्‌
अहो वयं धन्यतमा यदत्र त्यक्ताः पितृभ्यां न विचिन्तयामः
अभक्ष्यमाणा अबला वृकादिभिः स रक्षिता रक्षति यो हि गर्भे
य इच्छयेशः सृजतीदमव्ययो य एव रक्षत्यवलुम्पते च यः
तस्याबलाः क्रीडनमाहुरीशितुश्चराचरं निग्रहसङ्ग्रहे प्रभुः
पथि च्युतं तिष्ठति दिष्टरक्षितं गृहे स्थितं तद्विहतं विनश्यति
जीवत्यनाथोऽपि तदीक्षितो वने गृहेऽभिगुप्तोऽस्य हतो न जीवति
भूतानि तैस्तैर्निजयोनिकर्मभिर्भवन्ति काले न भवन्ति सर्वशः
न तत्र हात्मा प्रकृतावपि स्थितस्तस्या गुणैरन्यतमो हि बध्यते
इदं शरीरं पुरुषस्य मोहजं यथा पृथग्भौतिकमीयते गृहम्‌
यथौदकैः पार्थिवतैजसैर्जनः कालेन जातो विकृतो विनश्यति
यथानलो दारुषु भिन्न ईयते यथानिलो देहगतः पृथक्स्थितः
यथा नभः सर्वगतं न सज्जते तथा पुमान्सर्वगुणाश्रयः परः
सुयज्ञो नन्वयं शेते मूढा यमनुशोचथ
यः श्रोता योऽनुवक्तेह स न दृश्येत कर्हिचित्‌
न श्रोता नानुवक्तायं मुख्योऽप्यत्र महानसुः
यस्त्विहेन्द्रियवानात्मा स चान्यः प्राणदेहयोः
भूतेन्द्रियमनोलिङ्गान्देहानुच्चावचान्विभुः
भजत्युत्सृजति ह्यन्यस्तच्चापि स्वेन तेजसा
यावल्लिङ्गान्वितो ह्यात्मा तावत्कर्मनिबन्धनम्‌
ततो विपर्ययः क्लेशो मायायोगोऽनुवर्तते
वितथाभिनिवेशोऽयं यद्गुणेष्वर्थदृग्वचः
यथा मनोरथः स्वप्नः सर्वमैन्द्रियकं मृषा
अथ नित्यमनित्यं वा नेह शोचन्ति तद्विदः
नान्यथा शक्यते कर्तुं स्वभावः शोचतामिति
लुब्धको विपिने कश्चित्पक्षिणां निर्मितोऽन्तकः
वितत्य जालं विदधे तत्र तत्र प्रलोभयन्‌
कुलिङ्गमिथुनं तत्र विचरत्समदृश्यत
तयोः कुलिङ्गी सहसा लुब्धकेन प्रलोभिता
आसज्जत सिचस्तन्त्र्यां महिष्यः कालयन्त्रिता
कुलिङ्गस्तां तथापन्नां निरीक्ष्य भृशदुःखितः
स्नेहादकल्पः कृपणः कृपणां पर्यदेवयत्‌
अहो अकरुणो देवः स्त्रियाकरुणया विभुः
कृपणं मामनुशोचन्त्या दीनया किं करिष्यति
कामं नयतु मां देवः किमर्धेनात्मनो हि मे
दीनेन जीवता दुःखमनेन विधुरायुषा
कथं त्वजातपक्षांस्तान्मातृहीनान्बिभर्म्यहम्‌
मन्दभाग्याः प्रतीक्षन्ते नीडे मे मातरं प्रजाः
एवं कुलिङ्गं विलपन्तमारात्प्रियावियोगातुरमश्रुकण्ठम्‌
स एव तं शाकुनिकः शरेण विव्याध कालप्रहितो विलीनः
एवं यूयमपश्यन्त्य आत्मापायमबुद्धयः
नैनं प्राप्स्यथ शोचन्त्यः पतिं वर्षशतैरपि
श्रीहिरण्यकशिपुरुवाच
बाल एवं प्रवदति सर्वे विस्मितचेतसः
ज्ञातयो मेनिरे सर्वमनित्यमयथोत्थितम्‌
यम एतदुपाख्याय तत्रैवान्तरधीयत
ज्ञातयो हि सुयज्ञस्य चक्रुर्यत्साम्परायिकम्‌
अतः शोचत मा यूयं परं चात्मानमेव वा
क आत्मा कः परो वात्र स्वीयः पारक्य एव वा
स्वपराभिनिवेशेन विनाज्ञानेन देहिनाम्‌
श्रीनारद उवाच
इति दैत्यपतेर्वाक्यं दितिराकर्ण्य सस्नुषा
पुत्रशोकं क्षणात्त्यक्त्वा तत्त्वे चित्तमधारयत्‌
श्रीनारद उवाच
हिरण्यकशिपू राजन्नजेयमजरामरम्‌
आत्मानमप्रतिद्वन्द्वमेकराजं व्यधित्सत
स तेपे मन्दरद्रोण्यां तपः परमदारुणम्‌
ऊर्ध्वबाहुर्नभोदृष्टिः पादाङ्गुष्ठाश्रितावनिः
जटादीधितिभी रेजे संवर्तार्क इवांशुभिः
तस्मिंस्तपस्तप्यमाने देवाः स्थानानि भेजिरे
तस्य मूर्ध्नः समुद्भूतः सधूमोऽग्निस्तपोमयः
तीर्यगूर्ध्वमधो लोकान्प्रातपद्विष्वगीरितः
चुक्षुभुर्नद्युदन्वन्तः सद्वीपाद्रिश्चचाल भूः
निपेतुः सग्रहास्तारा जज्वलुश्च दिशो दश
तेन तप्ता दिवं त्यक्त्वा ब्रह्मलोकं ययुः सुराः
धात्रे विज्ञापयामासुर्देवदेव जगत्पते
दैत्येन्द्रतपसा तप्ता दिवि स्थातुं न शक्नुमः
तस्य चोपशमं भूमन्विधेहि यदि मन्यसे
लोका न यावन्नङ्क्ष्यन्ति बलिहारास्तवाभिभूः
तस्यायं किल सङ्कल्पश्चरतो दुश्चरं तपः
श्रूयतां किं न विदितस्तवाथापि निवेदितम्‌
सृष्ट्वा चराचरमिदं तपोयोगसमाधिना
अध्यास्ते सर्वधिष्ण्येभ्यः परमेष्ठी निजासनम्‌
तदहं वर्धमानेन तपोयोगसमाधिना
कालात्मनोश्च नित्यत्वात्साधयिष्ये तथात्मनः
अन्यथेदं विधास्येऽहमयथा पूर्वमोजसा
किमन्यैः कालनिर्धूतैः कल्पान्ते वैष्णवादिभिः
इति शुश्रुम निर्बन्धं तपः परममास्थितः
विधत्स्वानन्तरं युक्तं स्वयं त्रिभुवनेश्वर
तवासनं द्विजगवां पारमेष्ठ्यं जगत्पते
भवाय श्रेयसे भूत्यै क्षेमाय विजयाय च
इति विज्ञापितो देवैर्भगवानात्मभूर्नृप
परितो भृगुदक्षाद्यैर्ययौ दैत्येश्वराश्रमम्‌
न ददर्श प्रतिच्छन्नं वल्मीकतृणकीचकैः
पिपीलिकाभिराचीर्णं मेदस्त्वङ्मांसशोणितम्‌
तपन्तं तपसा लोकान्यथाभ्रापिहितं रविम्‌
विलक्ष्य विस्मितः प्राह हसंस्तं हंसवाहनः
श्रीब्रह्मोवाच
उत्तिष्ठोत्तिष्ठ भद्रं ते तपःसिद्धोऽसि काश्यप
वरदोऽहमनुप्राप्तो व्रियतामीप्सितो वरः
अद्राक्षमहमेतं ते हृत्सारं महदद्भुतम्‌
दंशभक्षितदेहस्य प्राणा ह्यस्थिषु शेरते
नैतत्पूर्वर्षयश्चक्रुर्न करिष्यन्ति चापरे
निरम्बुर्धारयेत्प्राणान्को वै दिव्यसमाः शतम्‌
व्यवसायेन तेऽनेन दुष्करेण मनस्विनाम्‌
तपोनिष्ठेन भवताजितोऽहं दितिनन्दन
ततस्त आशिषः सर्वा ददाम्यसुरपुङ्गव
मर्तस्य ते ह्यमर्तस्य दर्शनं नाफलं मम
श्रीनारद उवाच
इत्युक्त्वादिभवो देवो भक्षिताङ्गं पिपीलिकैः
कमण्डलुजलेनौक्षद्दिव्येनामोघराधसा
स तत्कीचकवल्मीकात्सहओजोबलान्वितः
सर्वावयवसम्पन्नो वज्रसंहननो युवा
उत्थितस्तप्तहेमाभो विभावसुरिवैधसः
स निरीक्ष्याम्बरे देवं हंसवाहमुपस्थितम्‌
ननाम शिरसा भूमौ तद्दर्शनमहोत्सवः
उत्थाय प्राञ्जलिः प्रह्व ईक्षमाणो दृशा विभुम्‌
हर्षाश्रुपुलकोद्भेदो गिरा गद्गदयागृणात्‌
श्रीहिरण्यकशिपुरुवाच
कल्पान्ते कालसृष्टेन योऽन्धेन तमसावृतम्‌
अभिव्यनग्जगदिदं स्वयञ्ज्योतिः स्वरोचिषा
आत्मना त्रिवृता चेदं सृजत्यवति लुम्पति
रजःसत्त्वतमोधाम्ने पराय महते नमः
नम आद्याय बीजाय ज्ञानविज्ञानमूर्तये
प्राणेन्द्रियमनोबुद्धि विकारैर्व्यक्तिमीयुषे
त्वमीशिषे जगतस्तस्थुषश्च प्राणेन मुख्येन पतिः प्रजानाम्‌
चित्तस्य चित्तैर्मनैन्द्रियाणां पतिर्महान्भूतगुणाशयेशः
त्वं सप्ततन्तून्वितनोषि तन्वा त्रय्या चतुर्होत्रकविद्यया च
त्वमेक आत्मात्मवतामनादिरनन्तपारः कविरन्तरात्मा
त्वमेव कालोऽनिमिषो जनानामायुर्लवाद्यवयवैः क्षिणोषि
कूटस्थ आत्मा परमेष्ठ्यजो महांस्त्वं जीवलोकस्य च जीव आत्मा
त्वत्तः परं नापरमप्यनेजदेजच्च किञ्चिद्व्यतिरिक्तमस्ति
विद्याः कलास्ते तनवश्च सर्वा हिरण्यगर्भोऽसि बृहत्त्रिपृष्ठः
व्यक्तं विभो स्थूलमिदं शरीरं येनेन्द्रियप्राणमनोगुणांस्त्वम्‌
भुङ्क्षे स्थितो धामनि पारमेष्ठ्ये अव्यक्त आत्मा पुरुषः पुराणः
अनन्ताव्यक्तरूपेण येनेदमखिलं ततम्‌
चिदचिच्छक्तियुक्ताय तस्मै भगवते नमः
यदि दास्यस्यभिमतान्वरान्मे वरदोत्तम
भूतेभ्यस्त्वद्विसृष्टेभ्यो मृत्युर्मा भून्मम प्रभो
नान्तर्बहिर्दिवा नक्तमन्यस्मादपि चायुधैः
न भूमौ नाम्बरे मृत्युर्न नरैर्न मृगैरपि
व्यसुभिर्वासुमद्भिर्वा सुरासुरमहोरगैः
अप्रतिद्वन्द्वतां युद्धे ऐकपत्यं च देहिनाम्‌
सर्वेषां लोकपालानां महिमानं यथात्मनः
तपोयोगप्रभावाणां यन्न रिष्यति कर्हिचित्‌
श्रीनारद उवाच
एवं वृतः शतधृतिर्हिरण्यकशिपोरथ
प्रादात्तत्तपसा प्रीतो वरांस्तस्य सुदुर्लभान्‌
श्रीब्रह्मोवाच
तातेमे दुर्लभाः पुंसां यान्वृणीषे वरान्मम
तथापि वितराम्यङ्ग वरान्यद्यपि दुर्लभान्‌
ततो जगाम भगवानमोघानुग्रहो विभुः
पूजितोऽसुरवर्येण स्तूयमानः प्रजेश्वरैः
एवं लब्धवरो दैत्यो बिभ्रद्धेममयं वपुः
भगवत्यकरोद्द्वेषं भ्रातुर्वधमनुस्मरन्‌
स विजित्य दिशः सर्वा लोकांश्च त्रीन्महासुरः
देवासुरमनुष्येन्द्र गन्धर्वगरुडोरगान्‌
सिद्धचारणविद्याध्रानृषीन्पितृपतीन्मनून्‌
यक्षरक्षःपिशाचेशान्प्रेतभूतपतीनपि
सर्वसत्त्वपतीन्जित्वा वशमानीय विश्वजित्‌
जहार लोकपालानां स्थानानि सह तेजसा
देवोद्यानश्रिया जुष्टमध्यास्ते स्म त्रिपिष्टपम्‌
महेन्द्रभवनं साक्षान्निर्मितं विश्वकर्मणा
त्रैलोक्यलक्ष्म्यायतनमध्युवासाखिलर्द्धिमत्‌
यत्र विद्रुमसोपाना महामारकता भुवः
यत्र स्फाटिककुड्यानि वैदूर्यस्तम्भपङ्क्तयः
यत्र चित्रवितानानि पद्मरागासनानि च
पयःफेननिभाः शय्या मुक्तादामपरिच्छदाः
कूजद्भिर्नूपुरैर्देव्यः शब्दयन्त्य इतस्ततः
रत्नस्थलीषु पश्यन्ति सुदतीः सुन्दरं मुखम्‌
तस्मिन्महेन्द्रभवने महाबलो महामना निर्जितलोक एकराट्‌
रेमेऽभिवन्द्याङ्घ्रियुगः सुरादिभिः प्रतापितैरूर्जितचण्डशासनः
तमङ्ग मत्तं मधुनोरुगन्धिना विवृत्तताम्राक्षमशेषधिष्ण्यपाः
उपासतोपायनपाणिभिर्विना त्रिभिस्तपोयोगबलौजसां पदम्‌
जगुर्महेन्द्रासनमोजसा स्थितं विश्वावसुस्तुम्बुरुरस्मदादयः
गन्धर्वसिद्धा ऋषयोऽस्तुवन्मुहुर्विद्याधराश्चाप्सरसश्च पाण्डव
स एव वर्णाश्रमिभिः क्रतुभिर्भूरिदक्षिणैः
इज्यमानो हविर्भागानग्रहीत्स्वेन तेजसा
अकृष्टपच्या तस्यासीत्सप्तद्वीपवती मही
तथा कामदुघा गावो नानाश्चर्यपदं नभः
रत्नाकराश्च रत्नौघांस्तत्पत्न्यश्चोहुरूर्मिभिः
क्षारसीधुघृतक्षौद्र दधिक्षीरामृतोदकाः
शैला द्रोणीभिराक्रीडं सर्वर्तुषु गुणान्द्रुमाः
दधार लोकपालानामेक एव पृथग्गुणान्‌
स इत्थं निर्जितककुबेकराड्विषयान्प्रियान्‌
यथोपजोषं भुञ्जानो नातृप्यदजितेन्द्रियः
एवमैश्वर्यमत्तस्य दृप्तस्योच्छास्त्रवर्तिनः
कालो महान्व्यतीयाय ब्रह्मशापमुपेयुषः
तस्योग्रदण्डसंविग्नाः सर्वे लोकाः सपालकाः
अन्यत्रालब्धशरणाः शरणं ययुरच्युतम्‌
तस्यै नमोऽस्तु काष्ठायै यत्रात्मा हरिरीश्वरः
यद्गत्वा न निवर्तन्ते शान्ताः सन्न्यासिनोऽमलाः
इति ते संयतात्मानः समाहितधियोऽमलाः
उपतस्थुर्हृषीकेशं विनिद्रा वायुभोजनाः
तेषामाविरभूद्वाणी अरूपा मेघनिःस्वना
सन्नादयन्ती ककुभः साधूनामभयङ्करी
मा भैष्ट विबुधश्रेष्ठाः सर्वेषां भद्रमस्तु वः
मद्दर्शनं हि भूतानां सर्वश्रेयोपपत्तये
ज्ञातमेतस्य दौरात्म्यं दैतेयापसदस्य यत्‌
तस्य शान्तिं करिष्यामि कालं तावत्प्रतीक्षत
यदा देवेषु वेदेषु गोषु विप्रेषु साधुषु
धर्मे मयि च विद्वेषः स वा आशु विनश्यति
निर्वैराय प्रशान्ताय स्वसुताय महात्मने
प्रह्रादाय यदा द्रुह्येद्धनिष्येऽपि वरोर्जितम्‌
श्रीनारद उवाच
इत्युक्ता लोकगुरुणा तं प्रणम्य दिवौकसः
न्यवर्तन्त गतोद्वेगा मेनिरे चासुरं हतम्‌
तस्य दैत्यपतेः पुत्राश्चत्वारः परमाद्भुताः
प्रह्रादोऽभून्महांस्तेषां गुणैर्महदुपासकः
ब्रह्मण्यः शीलसम्पन्नः सत्यसन्धो जितेन्द्रियः
आत्मवत्सर्वभूतानामेकप्रियसुहृत्तमः
दासवत्सन्नतार्याङ्घ्रिः पितृवद्दीनवत्सलः
भ्रातृवत्सदृशे स्निग्धो गुरुष्वीश्वरभावनः
विद्यार्थरूपजन्माढ्यो मानस्तम्भविवर्जितः
नोद्विग्नचित्तो व्यसनेषु निःस्पृहः श्रुतेषु दृष्टेषु गुणेष्ववस्तुदृक्‌
दान्तेन्द्रियप्राणशरीरधीः सदा प्रशान्तकामो रहितासुरोऽसुरः
यस्मिन्महद्गुणा राजन्गृह्यन्ते कविभिर्मुहुः
न तेऽधुना पिधीयन्ते यथा भगवतीश्वरे
यं साधुगाथासदसि रिपवोऽपि सुरा नृप
प्रतिमानं प्रकुर्वन्ति किमुतान्ये भवादृशाः
गुणैरलमसङ्ख्येयैर्माहात्म्यं तस्य सूच्यते
वासुदेवे भगवति यस्य नैसर्गिकी रतिः
न्यस्तक्रीडनको बालो जडवत्तन्मनस्तया
कृष्णग्रहगृहीतात्मा न वेद जगदीदृशम्‌
आसीनः पर्यटन्नश्नन्शयानः प्रपिबन्ब्रुवन्‌
नानुसन्धत्त एतानि गोविन्दपरिरम्भितः
क्वचिद्रुदति वैकुण्ठ चिन्ताशबलचेतनः
क्वचिद्धसति तच्चिन्ता ह्लाद उद्गायति क्वचित्‌
नदति क्वचिदुत्कण्ठो विलज्जो नृत्यति क्वचित्‌
क्वचित्तद्भावनायुक्तस्तन्मयोऽनुचकार ह
क्वचिदुत्पुलकस्तूष्णीमास्ते संस्पर्शनिर्वृतः
अस्पन्दप्रणयानन्द सलिलामीलितेक्षणः
स उत्तमश्लोकपदारविन्दयोर्निषेवयाकिञ्चनसङ्गलब्धया
तन्वन्परां निर्वृतिमात्मनो मुहुर्दुःसङ्गदीनस्य मनः शमं व्यधात्‌
तस्मिन्महाभागवते महाभागे महात्मनि
हिरण्यकशिपू राजन्नकरोदघमात्मजे
श्रीयुधिष्ठिर उवाच
देवर्ष एतदिच्छामो वेदितुं तव सुव्रत
यदात्मजाय शुद्धाय पितादात्साधवे ह्यघम्‌
पुत्रान्विप्रतिकूलान्स्वान्पितरः पुत्रवत्सलाः
उपालभन्ते शिक्षार्थं नैवाघमपरो यथा
किमुतानुवशान्साधूंस्तादृशान्गुरुदेवतान्‌
एतत्कौतूहलं ब्रह्मन्नस्माकं विधम प्रभो
पितुः पुत्राय यद्द्वेषो मरणाय प्रयोजितः
श्रीनारद उवाच
पौरोहित्याय भगवान्वृतः काव्यः किलासुरैः
षण्डामर्कौ सुतौ तस्य दैत्यराजगृहान्तिके
तौ राज्ञा प्रापितं बालं प्रह्लादं नयकोविदम्‌
पाठयामासतुः पाठ्यानन्यांश्चासुरबालकान्‌
यत्तत्र गुरुणा प्रोक्तं शुश्रुवेऽनुपपाठ च
न साधु मनसा मेने स्वपरासद्ग्रहाश्रयम्‌
एकदासुरराट्पुत्रमङ्कमारोप्य पाण्डव
पप्रच्छ कथ्यतां वत्स मन्यते साधु यद्भवान्‌
श्रीप्रह्लाद उवाच
तत्साधु मन्येऽसुरवर्य देहिनां सदा समुद्विग्नधियामसद्ग्रहात्‌
हित्वात्मपातं गृहमन्धकूपं वनं गतो यद्धरिमाश्रयेत
श्रीनारद उवाच
श्रुत्वा पुत्रगिरो दैत्यः परपक्षसमाहिताः
जहास बुद्धिर्बालानां भिद्यते परबुद्धिभिः
सम्यग्विधार्यतां बालो गुरुगेहे द्विजातिभिः
विष्णुपक्षैः प्रतिच्छन्नैर्न भिद्येतास्य धीर्यथा
गृहमानीतमाहूय प्रह्रादं दैत्ययाजकाः
प्रशस्य श्लक्ष्णया वाचा समपृच्छन्त सामभिः
वत्स प्रह्राद भद्रं ते सत्यं कथय मा मृषा
बालानति कुतस्तुभ्यमेष बुद्धिविपर्ययः
बुद्धिभेदः परकृत उताहो ते स्वतोऽभवत्‌
भण्यतां श्रोतुकामानां गुरूणां कुलनन्दन
श्रीप्रह्राद उवाच
परः स्वश्चेत्यसद्ग्राहः पुंसां यन्मायया कृतः
विमोहितधियां दृष्टस्तस्मै भगवते नमः
स यदानुव्रतः पुंसां पशुबुद्धिर्विभिद्यते
अन्य एष तथान्योऽहमिति भेदगतासती
स एष आत्मा स्वपरेत्यबुद्धिभिर्दुरत्ययानुक्रमणो निरूप्यते
मुह्यन्ति यद्वर्त्मनि वेदवादिनो ब्रह्मादयो ह्येष भिनत्ति मे मतिम्‌
यथा भ्राम्यत्ययो ब्रह्मन्स्वयमाकर्षसन्निधौ
तथा मे भिद्यते चेतश्चक्रपाणेर्यदृच्छया
श्रीनारद उवाच
एतावद्ब्राह्मणायोक्त्वा विरराम महामतिः
तं सन्निभर्त्स्य कुपितः सुदीनो राजसेवकः
आनीयतामरे वेत्रमस्माकमयशस्करः
कुलाङ्गारस्य दुर्बुद्धेश्चतुर्थोऽस्योदितो दमः
दैतेयचन्दनवने जातोऽयं कण्टकद्रुमः
यन्मूलोन्मूलपरशोर्विष्णोर्नालायितोऽर्भकः
इति तं विविधोपायैर्भीषयंस्तर्जनादिभिः
प्रह्रादं ग्राहयामास त्रिवर्गस्योपपादनम्‌
तत एनं गुरुर्ज्ञात्वा ज्ञातज्ञेयचतुष्टयम्‌
दैत्येन्द्रं दर्शयामास मातृमृष्टमलङ्कृतम्‌
पादयोः पतितं बालं प्रतिनन्द्याशिषासुरः
परिष्वज्य चिरं दोर्भ्यां परमामाप निर्वृतिम्‌
आरोप्याङ्कमवघ्राय मूर्धन्यश्रुकलाम्बुभिः
आसिञ्चन्विकसद्वक्त्रमिदमाह युधिष्ठिर
हिरण्यकशिपुरुवाच
प्रह्रादानूच्यतां तात स्वधीतं किञ्चिदुत्तमम्‌
कालेनैतावतायुष्मन्यदशिक्षद्गुरोर्भवान्‌
श्रीप्रह्राद उवाच
श्रवणं कीर्तनं विष्णोः स्मरणं पादसेवनम्‌
अर्चनं वन्दनं दास्यं सख्यमात्मनिवेदनम्‌
इति पुंसार्पिता विष्णौ भक्तिश्चेन्नवलक्षणा
क्रियेत भगवत्यद्धा तन्मन्येऽधीतमुत्तमम्‌
निशम्यैतत्सुतवचो हिरण्यकशिपुस्तदा
गुरुपुत्रमुवाचेदं रुषा प्रस्फुरिताधरः
ब्रह्मबन्धो किमेतत्ते विपक्षं श्रयतासता
असारं ग्राहितो बालो मामनादृत्य दुर्मते
सन्ति ह्यसाधवो लोके दुर्मैत्राश्छद्मवेषिणः
तेषामुदेत्यघं काले रोगः पातकिनामिव
श्रीगुरुपुत्र उवाच
न मत्प्रणीतं न परप्रणीतं सुतो वदत्येष तवेन्द्रशत्रो
नैसर्गिकीयं मतिरस्य राजन्नियच्छ मन्युं कददाः स्म मा नः
श्रीनारद उवाच
गुरुणैवं प्रतिप्रोक्तो भूय आहासुरः सुतम्‌
न चेद्गुरुमुखीयं ते कुतोऽभद्रासती मतिः
श्रीप्रह्राद उवाच
मतिर्न कृष्णे परतः स्वतो वा मिथोऽभिपद्येत गृहव्रतानाम्‌
अदान्तगोभिर्विशतां तमिस्रं पुनः पुनश्चर्वितचर्वणानाम्‌
न ते विदुः स्वार्थगतिं हि विष्णुं दुराशया ये बहिरर्थमानिनः
अन्धा यथान्धैरुपनीयमानास्तेऽपीशतन्त्र्यामुरुदाम्नि बद्धाः
नैषां मतिस्तावदुरुक्रमाङ्घ्रिं स्पृशत्यनर्थापगमो यदर्थः
महीयसां पादरजोऽभिषेकं निष्किञ्चनानां न वृणीत यावत्‌
इत्युक्त्वोपरतं पुत्रं हिरण्यकशिपू रुषा
अन्धीकृतात्मा स्वोत्सङ्गान्निरस्यत महीतले
आहामर्षरुषाविष्टः कषायीभूतलोचनः
वध्यतामाश्वयं वध्यो निःसारयत नैरृताः
अयं मे भ्रातृहा सोऽयं हित्वा स्वान्सुहृदोऽधमः
पितृव्यहन्तुः पादौ यो विष्णोर्दासवदर्चति
विष्णोर्वा साध्वसौ किं नु करिष्यत्यसमञ्जसः
सौहृदं दुस्त्यजं पित्रोरहाद्यः पञ्चहायनः
परोऽप्यपत्यं हितकृद्यथौषधं स्वदेहजोऽप्यामयवत्सुतोऽहितः
छिन्द्यात्तदङ्गं यदुतात्मनोऽहितं शेषं सुखं जीवति यद्विवर्जनात्‌
सर्वैरुपायैर्हन्तव्यः सम्भोजशयनासनैः
सुहृल्लिङ्गधरः शत्रुर्मुनेर्दुष्टमिवेन्द्रियम्‌
नैरृतास्ते समादिष्टा भर्त्रा वै शूलपाणयः
तिग्मदंष्ट्रकरालास्यास्ताम्रश्मश्रुशिरोरुहाः
नदन्तो भैरवं नादं छिन्धि भिन्धीति वादिनः
आसीनं चाहनन्शूलैः प्रह्रादं सर्वमर्मसु
परे ब्रह्मण्यनिर्देश्ये भगवत्यखिलात्मनि
युक्तात्मन्यफला आसन्नपुण्यस्येव सत्क्रियाः
प्रयासेऽपहते तस्मिन्दैत्येन्द्रः परिशङ्कितः
चकार तद्वधोपायान्निर्बन्धेन युधिष्ठिर
दिग्गजैर्दन्दशूकेन्द्रैरभिचारावपातनैः
मायाभिः सन्निरोधैश्च गरदानैरभोजनैः
हिमवाय्वग्निसलिलैः पर्वताक्रमणैरपि
न शशाक यदा हन्तुमपापमसुरः सुतम्‌
चिन्तां दीर्घतमां प्राप्तस्तत्कर्तुं नाभ्यपद्यत
एष मे बह्वसाधूक्तो वधोपायाश्च निर्मिताः
तैस्तैर्द्रोहैरसद्धर्मैर्मुक्तः स्वेनैव तेजसा
वर्तमानोऽविदूरे वै बालोऽप्यजडधीरयम्‌
न विस्मरति मेऽनार्यं शुनः शेप इव प्रभुः
अप्रमेयानुभावोऽयमकुतश्चिद्भयोऽमरः
नूनमेतद्विरोधेन मृत्युर्मे भविता न वा
इति तच्चिन्तया किञ्चिन्म्लानश्रियमधोमुखम्‌
शण्डामर्कावौशनसौ विविक्त इति होचतुः
जितं त्वयैकेन जगत्त्रयं भ्रुवोर्विजृम्भणत्रस्तसमस्तधिष्ण्यपम्‌
न तस्य चिन्त्यं तव नाथ चक्ष्वहे न वै शिशूनां गुणदोषयोः पदम्‌
इमं तु पाशैर्वरुणस्य बद्ध्वा निधेहि भीतो न पलायते यथा
बुद्धिश्च पुंसो वयसार्यसेवया यावद्गुरुर्भार्गव आगमिष्यति
तथेति गुरुपुत्रोक्तमनुज्ञायेदमब्रवीत्‌
धर्मो ह्यस्योपदेष्टव्यो राज्ञां यो गृहमेधिनाम्‌
धर्ममर्थं च कामं च नितरां चानुपूर्वशः
प्रह्रादायोचतू राजन्प्रश्रितावनताय च
यथा त्रिवर्गं गुरुभिरात्मने उपशिक्षितम्‌
न साधु मेने तच्छिक्षां द्वन्द्वारामोपवर्णिताम्‌
यदाचार्यः परावृत्तो गृहमेधीयकर्मसु
वयस्यैर्बालकैस्तत्र सोपहूतः कृतक्षणैः
अथ तान्श्लक्ष्णया वाचा प्रत्याहूय महाबुधः
उवाच विद्वांस्तन्निष्ठां कृपया प्रहसन्निव
ते तु तद्गौरवात्सर्वे त्यक्तक्रीडापरिच्छदाः
बाला अदूषितधियो द्वन्द्वारामेरितेहितैः
पर्युपासत राजेन्द्र तन्न्यस्तहृदयेक्षणाः
तानाह करुणो मैत्रो महाभागवतोऽसुरः
श्रीप्रह्राद उवाच
कौमार आचरेत्प्राज्ञो धर्मान्भागवतानिह
दुर्लभं मानुषं जन्म तदप्यध्रुवमर्थदम्‌
यथा हि पुरुषस्येह विष्णोः पादोपसर्पणम्‌
यदेष सर्वभूतानां प्रिय आत्मेश्वरः सुहृत्‌
सुखमैन्द्रियकं दैत्या देहयोगेन देहिनाम्‌
सर्वत्र लभ्यते दैवाद्यथा दुःखमयत्नतः
तत्प्रयासो न कर्तव्यो यत आयुर्व्ययः परम्‌
न तथा विन्दते क्षेमं मुकुन्दचरणाम्बुजम्‌
ततो यतेत कुशलः क्षेमाय भवमाश्रितः
शरीरं पौरुषं यावन्न विपद्येत पुष्कलम्‌
पुंसो वर्षशतं ह्यायुस्तदर्धं चाजितात्मनः
निष्फलं यदसौ रात्र्यां शेतेऽन्धं प्रापितस्तमः
मुग्धस्य बाल्ये कैशोरे क्रीडतो याति विंशतिः
जरया ग्रस्तदेहस्य यात्यकल्पस्य विंशतिः
दुरापूरेण कामेन मोहेन च बलीयसा
शेषं गृहेषु सक्तस्य प्रमत्तस्यापयाति हि
को गृहेषु पुमान्सक्तमात्मानमजितेन्द्रियः
स्नेहपाशैर्दृढैर्बद्धमुत्सहेत विमोचितुम्‌
को न्वर्थतृष्णां विसृजेत्प्राणेभ्योऽपि य ईप्सितः
यं क्रीणात्यसुभिः प्रेष्ठैस्तस्करः सेवको वणिक्‌
कथं प्रियाया अनुकम्पितायाः सङ्गं रहस्यं रुचिरांश्च मन्त्रान्‌
सुहृत्सु तत्स्नेहसितः शिशूनां कलाक्षराणामनुरक्तचित्तः
पुत्रान्स्मरंस्ता दुहित्ऱ्‌^ईर्हृदय्या भ्रात्ऱ्‌^ईन्स्वस्ऱ्‌^ईर्वा पितरौ च दीनौ
गृहान्मनोज्ञोरुपरिच्छदांश्च वृत्तीश्च कुल्याः पशुभृत्यवर्गान्‌
त्यजेत कोशस्कृदिवेहमानः कर्माणि लोभादवितृप्तकामः
औपस्थ्यजैह्वं बहुमन्यमानः कथं विरज्येत दुरन्तमोहः
कुटुम्बपोषाय वियन्निजायुर्न बुध्यतेऽर्थं विहतं प्रमत्तः
सर्वत्र तापत्रयदुःखितात्मा निर्विद्यते न स्वकुटुम्बरामः
वित्तेषु नित्याभिनिविष्टचेता विद्वांश्च दोषं परवित्तहर्तुः
प्रेत्येह वाथाप्यजितेन्द्रियस्तदशान्तकामो हरते कुटुम्बी
विद्वानपीत्थं दनुजाः कुटुम्बं पुष्णन्स्वलोकाय न कल्पते वै
यः स्वीयपारक्यविभिन्नभावस्तमः प्रपद्येत यथा विमूढः
यतो न कश्चित्क्व च कुत्रचिद्वा दीनः स्वमात्मानमलं समर्थः
विमोचितुं कामदृशां विहार क्रीडामृगो यन्निगडो विसर्गः
ततो विदूरात्परिहृत्य दैत्या दैत्येषु सङ्गं विषयात्मकेषु
उपेत नारायणमादिदेवं स मुक्तसङ्गैरिषितोऽपवर्गः
न ह्यच्युतं प्रीणयतो बह्वायासोऽसुरात्मजाः
आत्मत्वात्सर्वभूतानां सिद्धत्वादिह सर्वतः
परावरेषु भूतेषु ब्रह्मान्तस्थावरादिषु
भौतिकेषु विकारेषु भूतेष्वथ महत्सु च
गुणेषु गुणसाम्ये च गुणव्यतिकरे तथा
एक एव परो ह्यात्मा भगवानीश्वरोऽव्ययः
प्रत्यगात्मस्वरूपेण दृश्यरूपेण च स्वयम्‌
व्याप्यव्यापकनिर्देश्यो ह्यनिर्देश्योऽविकल्पितः
केवलानुभवानन्द स्वरूपः परमेश्वरः
माययान्तर्हितैश्वर्य ईयते गुणसर्गया
तस्मात्सर्वेषु भूतेषु दयां कुरुत सौहृदम्‌
भावमासुरमुन्मुच्य यया तुष्यत्यधोक्षजः
तुष्टे च तत्र किमलभ्यमनन्त आद्ये
किं तैर्गुणव्यतिकरादिह ये स्वसिद्धाः
धर्मादयः किमगुणेन च काङ्क्षितेन
सारं जुषां चरणयोरुपगायतां नः
धर्मार्थकाम इति योऽभिहितस्त्रिवर्ग
ईक्षा त्रयी नयदमौ विविधा च वार्ता
मन्ये तदेतदखिलं निगमस्य सत्यं
स्वात्मार्पणं स्वसुहृदः परमस्य पुंसः
ज्ञानं तदेतदमलं दुरवापमाह
नारायणो नरसखः किल नारदाय
एकान्तिनां भगवतस्तदकिञ्चनानां
पादारविन्दरजसाप्लुतदेहिनां स्यात्‌
श्रुतमेतन्मया पूर्वं ज्ञानं विज्ञानसंयुतम्‌
धर्मं भागवतं शुद्धं नारदाद्देवदर्शनात्‌
श्रीदैत्यपुत्रा ऊचुः
प्रह्राद त्वं वयं चापि नर्तेऽन्यं विद्महे गुरुम्‌
एताभ्यां गुरुपुत्राभ्यां बालानामपि हीश्वरौ
बालस्यान्तःपुरस्थस्य महत्सङ्गो दुरन्वयः
छिन्धि नः संशयं सौम्य स्याच्चेद्विस्रम्भकारणम्‌
श्रीनारद उवाच
एवं दैत्यसुतैः पृष्टो महाभागवतोऽसुरः
उवाच तान्स्मयमानः स्मरन्मदनुभाषितम्‌
श्रीप्रह्राद उवाच
पितरि प्रस्थितेऽस्माकं तपसे मन्दराचलम्‌
युद्धोद्यमं परं चक्रुर्विबुधा दानवान्प्रति
पिपीलिकैरहिरिव दिष्ट्या लोकोपतापनः
पापेन पापोऽभक्षीति वदन्तो वासवादयः
तेषामतिबलोद्योगं निशम्यासुरयूथपाः
वध्यमानाः सुरैर्भीता दुद्रुवुः सर्वतो दिशम्‌
कलत्रपुत्रवित्ताप्तान्गृहान्पशुपरिच्छदान्‌
नावेक्ष्यमाणास्त्वरिताः सर्वे प्राणपरीप्सवः
व्यलुम्पन्राजशिबिरममरा जयकाङ्क्षिणः
इन्द्रस्तु राजमहिषीं मातरं मम चाग्रहीत्‌
नीयमानां भयोद्विग्नां रुदतीं कुररीमिव
यदृच्छयागतस्तत्र देवर्षिर्ददृशे पथि
प्राह नैनां सुरपते नेतुमर्हस्यनागसम्‌
मुञ्च मुञ्च महाभाग सतीं परपरिग्रहम्‌
श्रीइन्द्र उवाच
आस्तेऽस्या जठरे वीर्यमविषह्यं सुरद्विषः
आस्यतां यावत्प्रसवं मोक्ष्येऽर्थपदवीं गतः
श्रीनारद उवाच
अयं निष्किल्बिषः साक्षान्महाभागवतो महान्‌
त्वया न प्राप्स्यते संस्थामनन्तानुचरो बली
इत्युक्तस्तां विहायेन्द्रो देवर्षेर्मानयन्वचः
अनन्तप्रियभक्त्यैनां परिक्रम्य दिवं ययौ
ततो मे मातरमृषिः समानीय निजाश्रमे
आश्वास्येहोष्यतां वत्से यावत्ते भर्तुरागमः
तथेत्यवात्सीद्देवर्षेरन्तिके साकुतोभया
यावद्दैत्यपतिर्घोरात्तपसो न न्यवर्तत
ऋषिं पर्यचरत्तत्र भक्त्या परमया सती
अन्तर्वत्नी स्वगर्भस्य क्षेमायेच्छाप्रसूतये
ऋषिः कारुणिकस्तस्याः प्रादादुभयमीश्वरः
धर्मस्य तत्त्वं ज्ञानं च मामप्युद्दिश्य निर्मलम्‌
तत्तु कालस्य दीर्घत्वात्स्त्रीत्वान्मातुस्तिरोदधे
ऋषिणानुगृहीतं मां नाधुनाप्यजहात्स्मृतिः
भवतामपि भूयान्मे यदि श्रद्दधते वचः
वैशारदी धीः श्रद्धातः स्त्रीबालानां च मे यथा
जन्माद्याः षडिमे भावा दृष्टा देहस्य नात्मनः
फलानामिव वृक्षस्य कालेनेश्वरमूर्तिना
आत्मा नित्योऽव्ययः शुद्ध एकः क्षेत्रज्ञ आश्रयः
अविक्रियः स्वदृघेतुर्व्यापकोऽसङ्ग्यनावृतः
एतैर्द्वादशभिर्विद्वानात्मनो लक्षणैः परैः
अहं ममेत्यसद्भावं देहादौ मोहजं त्यजेत्‌
स्वर्णं यथा ग्रावसु हेमकारः क्षेत्रेषु योगैस्तदभिज्ञ आप्नुयात्‌
क्षेत्रेषु देहेषु तथात्मयोगैरध्यात्मविद्ब्रह्मगतिं लभेत
अष्टौ प्रकृतयः प्रोक्तास्त्रय एव हि तद्गुणाः
विकाराः षोडशाचार्यैः पुमानेकः समन्वयात्‌
देहस्तु सर्वसङ्घातो जगत्तस्थुरिति द्विधा
अत्रैव मृग्यः पुरुषो नेति नेतीत्यतत्त्यजन्‌
अन्वयव्यतिरेकेण विवेकेनोशतात्मना
स्वर्गस्थानसमाम्नायैर्विमृशद्भिरसत्वरैः
बुद्धेर्जागरणं स्वप्नः सुषुप्तिरिति वृत्तयः
ता येनैवानुभूयन्ते सोऽध्यक्षः पुरुषः परः
एभिस्त्रिवर्णैः पर्यस्तैर्बुद्धिभेदैः क्रियोद्भवैः
स्वरूपमात्मनो बुध्येद्गन्धैर्वायुमिवान्वयात्‌
एतद्द्वारो हि संसारो गुणकर्मनिबन्धनः
अज्ञानमूलोऽपार्थोऽपि पुंसः स्वप्न इवार्प्यते
तस्माद्भवद्भिः कर्तव्यं कर्मणां त्रिगुणात्मनाम्‌
बीजनिर्हरणं योगः प्रवाहोपरमो धियः
तत्रोपायसहस्राणामयं भगवतोदितः
यदीश्वरे भगवति यथा यैरञ्जसा रतिः
गुरुशुश्रूषया भक्त्या सर्वलब्धार्पणेन च
सङ्गेन साधुभक्तानामीश्वराराधनेन च
श्रद्धया तत्कथायां च कीर्तनैर्गुणकर्मणाम्‌
तत्पादाम्बुरुहध्यानात्तल्लिङ्गेक्षार्हणादिभिः
हरिः सर्वेषु भूतेषु भगवानास्त ईश्वरः
इति भूतानि मनसा कामैस्तैः साधु मानयेत्‌
एवं निर्जितषड्वर्गैः क्रियते भक्तिरीश्वरे
वासुदेवे भगवति यया संलभ्यते रतिः
निशम्य कर्माणि गुणानतुल्यान्वीर्याणि लीलातनुभिः कृतानि
यदातिहर्षोत्पुलकाश्रुगद्गदं प्रोत्कण्ठ उद्गायति रौति नृत्यति
यदा ग्रहग्रस्त इव क्वचिद्धसत्याक्रन्दते ध्यायति वन्दते जनम्‌
मुहुः श्वसन्वक्ति हरे जगत्पते नारायणेत्यात्ममतिर्गतत्रपः
तदा पुमान्मुक्तसमस्तबन्धनस्तद्भावभावानुकृताशयाकृतिः
निर्दग्धबीजानुशयो महीयसा भक्तिप्रयोगेण समेत्यधोक्षजम्‌
अधोक्षजालम्भमिहाशुभात्मनः शरीरिणः संसृतिचक्रशातनम्‌
तद्ब्रह्मनिर्वाणसुखं विदुर्बुधास्ततो भजध्वं हृदये हृदीश्वरम्‌
कोऽतिप्रयासोऽसुरबालका हरेरुपासने स्वे हृदि छिद्रवत्सतः
स्वस्यात्मनः सख्युरशेषदेहिनां सामान्यतः किं विषयोपपादनैः
रायः कलत्रं पशवः सुतादयो गृहा मही कुञ्जरकोशभूतयः
सर्वेऽर्थकामाः क्षणभङ्गुरायुषः कुर्वन्ति मर्त्यस्य कियत्प्रियं चलाः
एवं हि लोकाः क्रतुभिः कृता अमी क्षयिष्णवः सातिशया न निर्मलाः
तस्माददृष्टश्रुतदूषणं परं भक्त्योक्तयेशं भजतात्मलब्धये
यदर्थ इह कर्माणि विद्वन्मान्यसकृन्नरः
करोत्यतो विपर्यासममोघं विन्दते फलम्‌
सुखाय दुःखमोक्षाय सङ्कल्प इह कर्मिणः
सदाप्नोतीहया दुःखमनीहायाः सुखावृतः
कामान्कामयते काम्यैर्यदर्थमिह पूरुषः
स वै देहस्तु पारक्यो भङ्गुरो यात्युपैति च
किमु व्यवहितापत्य दारागारधनादयः
राज्यकोशगजामात्य भृत्याप्ता ममतास्पदाः
किमेतैरात्मनस्तुच्छैः सह देहेन नश्वरैः
अनर्थैरर्थसङ्काशैर्नित्यानन्दरसोदधेः
निरूप्यतामिह स्वार्थः कियान्देहभृतोऽसुराः
निषेकादिष्ववस्थासु क्लिश्यमानस्य कर्मभिः
कर्माण्यारभते देही देहेनात्मानुवर्तिना
कर्मभिस्तनुते देहमुभयं त्वविवेकतः
तस्मादर्थाश्च कामाश्च धर्माश्च यदपाश्रयाः
भजतानीहयात्मानमनीहं हरिमीश्वरम्‌
सर्वेषामपि भूतानां हरिरात्मेश्वरः प्रियः
भूतैर्महद्भिः स्वकृतैः कृतानां जीवसंज्ञितः
देवोऽसुरो मनुष्यो वा यक्षो गन्धर्व एव वा
भजन्मुकुन्दचरणं स्वस्तिमान्स्याद्यथा वयम्‌
नालं द्विजत्वं देवत्वमृषित्वं वासुरात्मजाः
प्रीणनाय मुकुन्दस्य न वृत्तं न बहुज्ञता
न दानं न तपो नेज्या न शौचं न व्रतानि च
प्रीयतेऽमलया भक्त्या हरिरन्यद्विडम्बनम्‌
ततो हरौ भगवति भक्तिं कुरुत दानवाः
आत्मौपम्येन सर्वत्र सर्वभूतात्मनीश्वरे
दैतेया यक्षरक्षांसि स्त्रियः शूद्रा व्रजौकसः
खगा मृगाः पापजीवाः सन्ति ह्यच्युततां गताः
एतावानेव लोकेऽस्मिन्पुंसः स्वार्थः परः स्मृतः
एकान्तभक्तिर्गोविन्दे यत्सर्वत्र तदीक्षणम्‌
श्रीनारद उवाच
अथ दैत्यसुताः सर्वे श्रुत्वा तदनुवर्णितम्‌
जगृहुर्निरवद्यत्वान्नैव गुर्वनुशिक्षितम्‌
अथाचार्यसुतस्तेषां बुद्धिमेकान्तसंस्थिताम्‌
आलक्ष्य भीतस्त्वरितो राज्ञ आवेदयद्यथा
श्रुत्वा तदप्रियं दैत्यो दुःसहं तनयानयम्‌
कोपावेशचलद्गात्रः पुत्रं हन्तुं मनो दधे
क्षिप्त्वा परुषया वाचा प्रह्रादमतदर्हणम्‌
आहेक्षमाणः पापेन तिरश्चीनेन चक्षुषा
प्रश्रयावनतं दान्तं बद्धाञ्जलिमवस्थितम्‌
सर्पः पदाहत इव श्वसन्प्रकृतिदारुणः
श्रीहिरण्यकशिपुरुवाच
हे दुर्विनीत मन्दात्मन्कुलभेदकराधम
स्तब्धं मच्छासनोद्वृत्तं नेष्ये त्वाद्य यमक्षयम्‌
क्रुद्धस्य यस्य कम्पन्ते त्रयो लोकाः सहेश्वराः
तस्य मेऽभीतवन्मूढ शासनं किं बलोऽत्यगाः
श्रीप्रह्राद उवाच
न केवलं मे भवतश्च राजन्स वै बलं बलिनां चापरेषाम्‌
परेऽवरेऽमी स्थिरजङ्गमा ये ब्रह्मादयो येन वशं प्रणीताः
स ईश्वरः काल उरुक्रमोऽसावोजः सहः सत्त्वबलेन्द्रियात्मा
स एव विश्वं परमः स्वशक्तिभिः सृजत्यवत्यत्ति गुणत्रयेशः
जह्यासुरं भावमिमं त्वमात्मनः समं मनो धत्स्व न सन्ति विद्विषः
ऋतेऽजितादात्मन उत्पथे स्थितात्तद्धि ह्यनन्तस्य महत्समर्हणम्‌
दस्यून्पुरा षण्न विजित्य लुम्पतो मन्यन्त एके स्वजिता दिशो दश
जितात्मनो ज्ञस्य समस्य देहिनां साधोः स्वमोहप्रभवाः कुतः परे
श्रीहिरण्यकशिपुरुवाच
व्यक्तं त्वं मर्तुकामोऽसि योऽतिमात्रं विकत्थसे
मुमूर्षूणां हि मन्दात्मन्ननु स्युर्विक्लवा गिरः
यस्त्वया मन्दभाग्योक्तो मदन्यो जगदीश्वरः
क्वासौ यदि स सर्वत्र कस्मात्स्तम्भे न दृश्यते
सोऽहं विकत्थमानस्य शिरः कायाद्धरामि ते
गोपायेत हरिस्त्वाद्य यस्ते शरणमीप्सितम्‌
एवं दुरुक्तैर्मुहुरर्दयन्रुषा सुतं महाभागवतं महासुरः
खड्गं प्रगृह्योत्पतितो वरासनात्स्तम्भं तताडातिबलः स्वमुष्टिना
तदैव तस्मिन्निनदोऽतिभीषणो बभूव येनाण्डकटाहमस्फुटत्‌
यं वै स्वधिष्ण्योपगतं त्वजादयः श्रुत्वा स्वधामात्ययमङ्ग मेनिरे
स विक्रमन्पुत्रवधेप्सुरोजसा निशम्य निर्ह्रादमपूर्वमद्भुतम्‌
अन्तःसभायां न ददर्श तत्पदं वितत्रसुर्येन सुरारियूथपाः
सत्यं विधातुं निजभृत्यभाषितं व्याप्तिं च भूतेष्वखिलेषु चात्मनः
अदृश्यतात्यद्भुतरूपमुद्वहन्स्तम्भे सभायां न मृगं न मानुषम्‌
स सत्त्वमेनं परितो विपश्यन्स्तम्भस्य मध्यादनुनिर्जिहानम्‌
नायं मृगो नापि नरो विचित्रमहो किमेतन्नृमृगेन्द्ररूपम्‌
मीमांसमानस्य समुत्थितोऽग्रतो नृसिंहरूपस्तदलं भयानकम्‌
प्रतप्तचामीकरचण्डलोचनं स्फुरत्सटाकेशरजृम्भिताननम्‌
करालदंष्ट्रं करवालचञ्चल क्षुरान्तजिह्वं भ्रुकुटीमुखोल्बणम्‌
स्तब्धोर्ध्वकर्णं गिरिकन्दराद्भुत व्यात्तास्यनासं हनुभेदभीषणम्‌
दिविस्पृशत्कायमदीर्घपीवर ग्रीवोरुवक्षःस्थलमल्पमध्यमम्‌
चन्द्रांशुगौरैश्छुरितं तनूरुहैर्विष्वग्भुजानीकशतं नखायुधम्‌
दुरासदं सर्वनिजेतरायुध प्रवेकविद्रावितदैत्यदानवम्‌
प्रायेण मेऽयं हरिणोरुमायिना वधः स्मृतोऽनेन समुद्यतेन किम्‌
एवं ब्रुवंस्त्वभ्यपतद्गदायुधो नदन्नृसिंहं प्रति दैत्यकुञ्जरः
अलक्षितोऽग्नौ पतितः पतङ्गमो यथा नृसिंहौजसि सोऽसुरस्तदा
न तद्विचित्रं खलु सत्त्वधामनि स्वतेजसा यो नु पुरापिबत्तमः
ततोऽभिपद्याभ्यहनन्महासुरो रुषा नृसिंहं गदयोरुवेगया
तं विक्रमन्तं सगदं गदाधरो महोरगं तार्क्ष्यसुतो यथाग्रहीत्‌
स तस्य हस्तोत्कलितस्तदासुरो विक्रीडतो यद्वदहिर्गरुत्मतः
असाध्वमन्यन्त हृतौकसोऽमरा घनच्छदा भारत सर्वधिष्ण्यपाः
तं मन्यमानो निजवीर्यशङ्कितं यद्धस्तमुक्तो नृहरिं महासुरः
पुनस्तमासज्जत खड्गचर्मणी प्रगृह्य वेगेन गतश्रमो मृधे
तं श्येनवेगं शतचन्द्रवर्त्मभिश्चरन्तमच्छिद्रमुपर्यधो हरिः
कृत्वाट्टहासं खरमुत्स्वनोल्बणं निमीलिताक्षं जगृहे महाजवः
विष्वक्स्फुरन्तं ग्रहणातुरं हरिर्व्यालो यथाखुं कुलिशाक्षतत्वचम्‌
द्वार्यूरुमापत्य ददार लीलया नखैर्यथाहिं गरुडो महाविषम्‌
संरम्भदुष्प्रेक्ष्यकराललोचनो व्यात्ताननान्तं विलिहन्स्वजिह्वया
असृग्लवाक्तारुणकेशराननो यथान्त्रमाली द्विपहत्यया हरिः
नखाङ्कुरोत्पाटितहृत्सरोरुहं विसृज्य तस्यानुचरानुदायुधान्‌
अहन्समस्तान्नखशस्त्रपाणिभिर्दोर्दण्डयूथोऽनुपथान्सहस्रशः
सटावधूता जलदाः परापतन्ग्रहाश्च तद्दृष्टिविमुष्टरोचिषः
अम्भोधयः श्वासहता विचुक्षुभुर्निर्ह्रादभीता दिगिभा विचुक्रुशुः
द्यौस्तत्सटोत्क्षिप्तविमानसङ्कुला प्रोत्सर्पत क्ष्मा च पदाभिपीडिता
शैलाः समुत्पेतुरमुष्य रंहसा तत्तेजसा खं ककुभो न रेजिरे
ततः सभायामुपविष्टमुत्तमे नृपासने सम्भृततेजसं विभुम्‌
अलक्षितद्वैरथमत्यमर्षणं प्रचण्डवक्त्रं न बभाज कश्चन
निशाम्य लोकत्रयमस्तकज्वरं तमादिदैत्यं हरिणा हतं मृधे
प्रहर्षवेगोत्कलितानना मुहुः प्रसूनवर्षैर्ववृषुः सुरस्त्रियः
तदा विमानावलिभिर्नभस्तलं दिदृक्षतां सङ्कुलमास नाकिनाम्‌
सुरानका दुन्दुभयोऽथ जघ्निरे गन्धर्वमुख्या ननृतुर्जगुः स्त्रियः
तत्रोपव्रज्य विबुधा ब्रह्मेन्द्रगिरिशादयः
ऋषयः पितरः सिद्धा विद्याधरमहोरगाः
मनवः प्रजानां पतयो गन्धर्वाप्सरचारणाः
यक्षाः किम्पुरुषास्तात वेतालाः सहकिन्नराः
ते विष्णुपार्षदाः सर्वे सुनन्दकुमुदादयः
मूर्ध्नि बद्धाञ्जलिपुटा आसीनं तीव्रतेजसम्‌
ईडिरे नरशार्दुलं नातिदूरचराः पृथक्‌
श्रीब्रह्मोवाच
नतोऽस्म्यनन्ताय दुरन्तशक्तये विचित्रवीर्याय पवित्रकर्मणे
विश्वस्य सर्गस्थितिसंयमान्गुणैः स्वलीलया सन्दधतेऽव्ययात्मने
श्रीरुद्र उवाच
कोपकालो युगान्तस्ते हतोऽयमसुरोऽल्पकः
तत्सुतं पाह्युपसृतं भक्तं ते भक्तवत्सल
श्रीइन्द्र उवाच
प्रत्यानीताः परम भवता त्रायता नः स्वभागा
दैत्याक्रान्तं हृदयकमलं तद्गृहं प्रत्यबोधि
कालग्रस्तं कियदिदमहो नाथ शुश्रूषतां ते
मुक्तिस्तेषां न हि बहुमता नारसिंहापरैः किम्‌
श्रीऋषय ऊचुः
त्वं नस्तपः परममात्थ यदात्मतेजो
येनेदमादिपुरुषात्मगतं ससर्क्थ
तद्विप्रलुप्तममुनाद्य शरण्यपाल
रक्षागृहीतवपुषा पुनरन्वमंस्थाः
श्रीपितर ऊचुः
श्राद्धानि नोऽधिबुभुजे प्रसभं तनूजैर्‌
दत्तानि तीर्थसमयेऽप्यपिबत्तिलाम्बु
तस्योदरान्नखविदीर्णवपाद्य आर्च्छत्‌
तस्मै नमो नृहरयेऽखिलधर्मगोप्त्रे
श्रीसिद्धा ऊचुः
यो नो गतिं योगसिद्धामसाधुरहार्षीद्योगतपोबलेन
नाना दर्पं तं नखैर्विददार तस्मै तुभ्यं प्रणताः स्मो नृसिंह
श्रीविद्याधरा ऊचुः
विद्यां पृथग्धारणयानुराद्धां न्यषेधदज्ञो बलवीर्यदृप्तः
स येन सङ्ख्ये पशुवद्धतस्तं मायानृसिंहं प्रणताः स्म नित्यम्‌
श्रीनागा ऊचुः
येन पापेन रत्नानि स्त्रीरत्नानि हृतानि नः
तद्वक्षःपाटनेनासां दत्तानन्द नमोऽस्तु ते
श्रीमनव ऊचुः
मनवो वयं तव निदेशकारिणो दितिजेन देव परिभूतसेतवः
भवता खलः स उपसंहृतः प्रभो करवाम ते किमनुशाधि किङ्करान्‌
श्रीप्रजापतय ऊचुः
प्रजेशा वयं ते परेशाभिसृष्टा न येन प्रजा वै सृजामो निषिद्धाः
स एष त्वया भिन्नवक्षा नु शेते जगन्मङ्गलं सत्त्वमूर्तेऽवतारः
श्रीगन्धर्वा ऊचुः
वयं विभो ते नटनाट्यगायका येनात्मसाद्वीर्यबलौजसा कृताः
स एष नीतो भवता दशामिमां किमुत्पथस्थः कुशलाय कल्पते
श्रीचारणा ऊचुः
हरे तवाङ्घ्रिपङ्कजं भवापवर्गमाश्रिताः
यदेष साधुहृच्छयस्त्वयासुरः समापितः
श्रीयक्षा ऊचुः
वयमनुचरमुख्याः कर्मभिस्ते मनोज्ञैस्‌
त इह दितिसुतेन प्रापिता वाहकत्वम्‌
स तु जनपरितापं तत्कृतं जानता ते
नरहर उपनीतः पञ्चतां पञ्चविंश
श्रीकिम्पुरुषा ऊचुः
वयं किम्पुरुषास्त्वं तु महापुरुष ईश्वरः
अयं कुपुरुषो नष्टो धिक्कृतः साधुभिर्यदा
श्रीवैतालिका ऊचुः
सभासु सत्रेषु तवामलं यशो गीत्वा सपर्यां महतीं लभामहे
यस्तामनैषीद्वशमेष दुर्जनो द्विष्ट्या हतस्ते भगवन्यथामयः
श्रीकिन्नरा ऊचुः
वयमीश किन्नरगणास्तवानुगा दितिजेन विष्टिममुनानुकारिताः
भवता हरे स वृजिनोऽवसादितो नरसिंह नाथ विभवाय नो भव
श्रीविष्णुपार्षदा ऊचुः
अद्यैतद्धरिनररूपमद्भुतं ते दृष्टं नः शरणद सर्वलोकशर्म
सोऽयं ते विधिकर ईश विप्रशप्तस्तस्येदं निधनमनुग्रहाय विद्मः
श्रीनारद उवाच
एवं सुरादयः सर्वे ब्रह्मरुद्रपुरः सराः
नोपैतुमशकन्मन्यु संरम्भं सुदुरासदम्‌
साक्षात्श्रीः प्रेषिता देवैर्दृष्ट्वा तं महदद्भुतम्‌
अदृष्टाश्रुतपूर्वत्वात्सा नोपेयाय शङ्किता
प्रह्रादं प्रेषयामास ब्रह्मावस्थितमन्तिके
तात प्रशमयोपेहि स्वपित्रे कुपितं प्रभुम्‌
तथेति शनकै राजन्महाभागवतोऽर्भकः
उपेत्य भुवि कायेन ननाम विधृताञ्जलिः
स्वपादमूले पतितं तमर्भकं विलोक्य देवः कृपया परिप्लुतः
उत्थाप्य तच्छीर्ष्ण्यदधात्कराम्बुजं कालाहिवित्रस्तधियां कृताभयम्‌
स तत्करस्पर्शधुताखिलाशुभः सपद्यभिव्यक्तपरात्मदर्शनः
तत्पादपद्मं हृदि निर्वृतो दधौ हृष्यत्तनुः क्लिन्नहृदश्रुलोचनः
अस्तौषीद्धरिमेकाग्र मनसा सुसमाहितः
प्रेमगद्गदया वाचा तन्न्यस्तहृदयेक्षणः
श्रीप्रह्राद उवाच
ब्रह्मादयः सुरगणा मुनयोऽथ सिद्धाः
सत्त्वैकतानगतयो वचसां प्रवाहैः
नाराधितुं पुरुगुणैरधुनापि पिप्रुः
किं तोष्टुमर्हति स मे हरिरुग्रजातेः
मन्ये धनाभिजनरूपतपःश्रुतौजस्‌
तेजःप्रभावबलपौरुषबुद्धियोगाः
नाराधनाय हि भवन्ति परस्य पुंसो
भक्त्या तुतोष भगवान्गजयूथपाय
विप्राद्द्विषड्गुणयुतादरविन्दनाभ
पादारविन्दविमुखात्श्वपचं वरिष्ठम्‌
मन्ये तदर्पितमनोवचनेहितार्थ
प्राणं पुनाति स कुलं न तु भूरिमानः
नैवात्मनः प्रभुरयं निजलाभपूर्णो
मानं जनादविदुषः करुणो वृणीते
यद्यज्जनो भगवते विदधीत मानं
तच्चात्मने प्रतिमुखस्य यथा मुखश्रीः
तस्मादहं विगतविक्लव ईश्वरस्य
सर्वात्मना महि गृणामि यथा मनीषम्‌
नीचोऽजया गुणविसर्गमनुप्रविष्टः
पूयेत येन हि पुमाननुवर्णितेन
सर्वे ह्यमी विधिकरास्तव सत्त्वधाम्नो
ब्रह्मादयो वयमिवेश न चोद्विजन्तः
क्षेमाय भूतय उतात्मसुखाय चास्य
विक्रीडितं भगवतो रुचिरावतारैः
तद्यच्छ मन्युमसुरश्च हतस्त्वयाद्य
मोदेत साधुरपि वृश्चिकसर्पहत्या
लोकाश्च निर्वृतिमिताः प्रतियन्ति सर्वे
रूपं नृसिंह विभयाय जनाः स्मरन्ति
नाहं बिभेम्यजित तेऽतिभयानकास्य
जिह्वार्कनेत्रभ्रुकुटीरभसोग्रदंष्ट्रात्‌
आन्त्रस्रजःक्षतजकेशरशङ्कुकर्णान्‌
निर्ह्रादभीतदिगिभादरिभिन्नखाग्रात्‌
त्रस्तोऽस्म्यहं कृपणवत्सल दुःसहोग्र
संसारचक्रकदनाद्ग्रसतां प्रणीतः
बद्धः स्वकर्मभिरुशत्तम तेऽङ्घ्रिमूलं
प्रीतोऽपवर्गशरणं ह्वयसे कदा नु
यस्मात्प्रियाप्रियवियोगसंयोगजन्म
शोकाग्निना सकलयोनिषु दह्यमानः
दुःखौषधं तदपि दुःखमतद्धियाहं
भूमन्भ्रमामि वद मे तव दास्ययोगम्‌
सोऽहं प्रियस्य सुहृदः परदेवताया
लीलाकथास्तव नृसिंह विरिञ्चगीताः
अञ्जस्तितर्म्यनुगृणन्गुणविप्रमुक्तो
दुर्गाणि ते पदयुगालयहंससङ्गः
बालस्य नेह शरणं पितरौ नृसिंह
नार्तस्य चागदमुदन्वति मज्जतो नौः
तप्तस्य तत्प्रतिविधिर्य इहाञ्जसेष्टस्‌
तावद्विभो तनुभृतां त्वदुपेक्षितानाम्‌
यस्मिन्यतो यर्हि येन च यस्य यस्माद्‌
यस्मै यथा यदुत यस्त्वपरः परो वा
भावः करोति विकरोति पृथक्स्वभावः
सञ्चोदितस्तदखिलं भवतः स्वरूपम्‌
माया मनः सृजति कर्ममयं बलीयः
कालेन चोदितगुणानुमतेन पुंसः
छन्दोमयं यदजयार्पितषोडशारं
संसारचक्रमज कोऽतितरेत्त्वदन्यः
स त्वं हि नित्यविजितात्मगुणः स्वधाम्ना
कालो वशीकृतविसृज्यविसर्गशक्तिः
चक्रे विसृष्टमजयेश्वर षोडशारे
निष्पीड्यमानमुपकर्ष विभो प्रपन्नम्‌
दृष्टा मया दिवि विभोऽखिलधिष्ण्यपानाम्‌
आयुः श्रियो विभव इच्छति यान्जनोऽयम्‌
येऽस्मत्पितुः कुपितहासविजृम्भितभ्रू
विस्फूर्जितेन लुलिताः स तु ते निरस्तः
तस्मादमूस्तनुभृतामहमाशिषोऽज्ञ
आयुः श्रियं विभवमैन्द्रियमाविरिञ्च्यात्‌
नेच्छामि ते विलुलितानुरुविक्रमेण
कालात्मनोपनय मां निजभृत्यपार्श्वम्‌
कुत्राशिषः श्रुतिसुखा मृगतृष्णिरूपाः
क्वेदं कलेवरमशेषरुजां विरोहः
निर्विद्यते न तु जनो यदपीति विद्वान्‌
कामानलं मधुलवैः शमयन्दुरापैः
क्वाहं रजःप्रभव ईश तमोऽधिकेऽस्मिन्‌
जातः सुरेतरकुले क्व तवानुकम्पा
न ब्रह्मणो न तु भवस्य न वै रमाया
यन्मेऽर्पितः शिरसि पद्मकरः प्रसादः
नैषा परावरमतिर्भवतो ननु स्याज्‌
जन्तोर्यथात्मसुहृदो जगतस्तथापि
संसेवया सुरतरोरिव ते प्रसादः
सेवानुरूपमुदयो न परावरत्वम्‌
एवं जनं निपतितं प्रभवाहिकूपे
कामाभिकाममनु यः प्रपतन्प्रसङ्गात्‌
कृत्वात्मसात्सुरर्षिणा भगवन्गृहीतः
सोऽहं कथं नु विसृजे तव भृत्यसेवाम्‌
मत्प्राणरक्षणमनन्त पितुर्वधश्च
मन्ये स्वभृत्यऋषिवाक्यमृतं विधातुम्‌
खड्गं प्रगृह्य यदवोचदसद्विधित्सुस्‌
त्वामीश्वरो मदपरोऽवतु कं हरामि
एकस्त्वमेव जगदेतममुष्य यत्त्वम्‌
आद्यन्तयोः पृथगवस्यसि मध्यतश्च
सृष्ट्वा गुणव्यतिकरं निजमाययेदं
नानेव तैरवसितस्तदनुप्रविष्टः
त्वम्वा इदं सदसदीश भवांस्ततोऽन्यो
माया यदात्मपरबुद्धिरियं ह्यपार्था
यद्यस्य जन्म निधनं स्थितिरीक्षणं च
तद्वैतदेव वसुकालवदष्टितर्वोः
न्यस्येदमात्मनि जगद्विलयाम्बुमध्ये
शेषेत्मना निजसुखानुभवो निरीहः
योगेन मीलितदृगात्मनिपीतनिद्रस्‌
तुर्ये स्थितो न तु तमो न गुणांश्च युङ्क्षे
तस्यैव ते वपुरिदं निजकालशक्त्या
सञ्चोदितप्रकृतिधर्मण आत्मगूढम्‌
अम्भस्यनन्तशयनाद्विरमत्समाधेर्‌
नाभेरभूत्स्वकणिकावटवन्महाब्जम्‌
तत्सम्भवः कविरतोऽन्यदपश्यमानस्‌
त्वां बीजमात्मनि ततं स बहिर्विचिन्त्य
नाविन्ददब्दशतमप्सु निमज्जमानो
जातेऽङ्कुरे कथमुहोपलभेत बीजम्‌
स त्वात्मयोनिरतिविस्मित आश्रितोऽब्जं
कालेन तीव्रतपसा परिशुद्धभावः
त्वामात्मनीश भुवि गन्धमिवातिसूक्ष्मं
भूतेन्द्रियाशयमये विततं ददर्श
एवं सहस्रवदनाङ्घ्रिशिरःकरोरु
नासाद्यकर्णनयनाभरणायुधाढ्यम्‌
मायामयं सदुपलक्षितसन्निवेशं
दृष्ट्वा महापुरुषमाप मुदं विरिञ्चः
तस्मै भवान्हयशिरस्तनुवं हि बिभ्रद्‌
वेदद्रुहावतिबलौ मधुकैटभाख्यौ
हत्वानयच्छ्रुतिगणांश्च रजस्तमश्च
सत्त्वं तव प्रियतमां तनुमामनन्ति
इत्थं नृतिर्यगृषिदेवझषावतारैर्‌
लोकान्विभावयसि हंसि जगत्प्रतीपान्‌
धर्मं महापुरुष पासि युगानुवृत्तं
छन्नः कलौ यदभवस्त्रियुगोऽथ स त्वम्‌
नैतन्मनस्तव कथासु विकुण्ठनाथ
सम्प्रीयते दुरितदुष्टमसाधु तीव्रम्‌
कामातुरं हर्षशोकभयैषणार्तं
तस्मिन्कथं तव गतिं विमृशामि दीनः
जिह्वैकतोऽच्युत विकर्षति मावितृप्ता
शिश्नोऽन्यतस्त्वगुदरं श्रवणं कुतश्चित्‌
घ्राणोऽन्यतश्चपलदृक्क्व च कर्मशक्तिर्‌
बह्व्यः सपत्न्य इव गेहपतिं लुनन्ति
एवं स्वकर्मपतितं भववैतरण्याम्‌
अन्योन्यजन्ममरणाशनभीतभीतम्‌
पश्यन्जनं स्वपरविग्रहवैरमैत्रं
हन्तेति पारचर पीपृहि मूढमद्य
को न्वत्र तेऽखिलगुरो भगवन्प्रयास
उत्तारणेऽस्य भवसम्भवलोपहेतोः
मूढेषु वै महदनुग्रह आर्तबन्धो
किं तेन ते प्रियजनाननुसेवतां नः
नैवोद्विजे पर दुरत्ययवैतरण्यास्‌
त्वद्वीर्यगायनमहामृतमग्नचित्तः
शोचे ततो विमुखचेतस इन्द्रियार्थ
मायासुखाय भरमुद्वहतो विमूढान्‌
प्रायेण देव मुनयः स्वविमुक्तिकामा
मौनं चरन्ति विजने न परार्थनिष्ठाः
नैतान्विहाय कृपणान्विमुमुक्ष एको
नान्यं त्वदस्य शरणं भ्रमतोऽनुपश्ये
यन्मैथुनादिगृहमेधिसुखं हि तुच्छं
कण्डूयनेन करयोरिव दुःखदुःखम्‌
तृप्यन्ति नेह कृपणा बहुदुःखभाजः
कण्डूतिवन्मनसिजं विषहेत धीरः
मौनव्रतश्रुततपोऽध्ययनस्वधर्म
व्याख्यारहोजपसमाधय आपवर्ग्याः
प्रायः परं पुरुष ते त्वजितेन्द्रियाणां
वार्ता भवन्त्युत न वात्र तु दाम्भिकानाम्‌
रूपे इमे सदसती तव वेदसृष्टे
बीजाङ्कुराविव न चान्यदरूपकस्य
युक्ताः समक्षमुभयत्र विचक्षन्ते त्वां
योगेन वह्निमिव दारुषु नान्यतः स्यात्‌
त्वं वायुरग्निरवनिर्वियदम्बु मात्राः
प्राणेन्द्रियाणि हृदयं चिदनुग्रहश्च
सर्वं त्वमेव सगुणो विगुणश्च भूमन्‌
नान्यत्त्वदस्त्यपि मनोवचसा निरुक्तम्‌
नैते गुणा न गुणिनो महदादयो ये
सर्वे मनः प्रभृतयः सहदेवमर्त्याः
आद्यन्तवन्त उरुगाय विदन्ति हि त्वाम्‌
एवं विमृश्य सुधियो विरमन्ति शब्दात्‌
तत्तेऽर्हत्तम नमः स्तुतिकर्मपूजाः
कर्म स्मृतिश्चरणयोः श्रवणं कथायाम्‌
संसेवया त्वयि विनेति षडङ्गया किं
भक्तिं जनः परमहंसगतौ लभेत
श्रीनारद उवाच
एतावद्वर्णितगुणो भक्त्या भक्तेन निर्गुणः
प्रह्रादं प्रणतं प्रीतो यतमन्युरभाषत
श्रीभगवानुवाच
प्रह्राद भद्र भद्रं ते प्रीतोऽहं तेऽसुरोत्तम
वरं वृणीष्वाभिमतं कामपूरोऽस्म्यहं नृणाम्‌
मामप्रीणत आयुष्मन्दर्शनं दुर्लभं हि मे
दृष्ट्वा मां न पुनर्जन्तुरात्मानं तप्तुमर्हति
प्रीणन्ति ह्यथ मां धीराः सर्वभावेन साधवः
श्रेयस्कामा महाभाग सर्वासामाशिषां पतिम्‌
श्रीनारद उवाच
एवं प्रलोभ्यमानोऽपि वरैर्लोकप्रलोभनैः
एकान्तित्वाद्भगवति नैच्छत्तानसुरोत्तमः
श्रीनारद उवाच
भक्तियोगस्य तत्सर्वमन्तरायतयार्भकः
मन्यमानो हृषीकेशं स्मयमान उवाच ह
श्रीप्रह्राद उवाच
मा मां प्रलोभयोत्पत्त्या सक्तंकामेषु तैर्वरैः
तत्सङ्गभीतो निर्विण्णो मुमुक्षुस्त्वामुपाश्रितः
भृत्यलक्षणजिज्ञासुर्भक्तं कामेष्वचोदयत्‌
भवान्संसारबीजेषु हृदयग्रन्थिषु प्रभो
नान्यथा तेऽखिलगुरो घटेत करुणात्मनः
यस्त आशिष आशास्ते न स भृत्यः स वै वणिक्‌
आशासानो न वै भृत्यः स्वामिन्याशिष आत्मनः
न स्वामी भृत्यतः स्वाम्यमिच्छन्यो राति चाशिषः
अहं त्वकामस्त्वद्भक्तस्त्वं च स्वाम्यनपाश्रयः
नान्यथेहावयोरर्थो राजसेवकयोरिव
यदि दास्यसि मे कामान्वरांस्त्वं वरदर्षभ
कामानां हृद्यसंरोहं भवतस्तु वृणे वरम्‌
इन्द्रियाणि मनः प्राण आत्मा धर्मो धृतिर्मतिः
ह्रीः श्रीस्तेजः स्मृतिः सत्यं यस्य नश्यन्ति जन्मना
विमुञ्चति यदा कामान्मानवो मनसि स्थितान्‌
तर्ह्येव पुण्डरीकाक्ष भगवत्त्वाय कल्पते
ओं नमो भगवते तुभ्यं पुरुषाय महात्मने
हरयेऽद्भुतसिंहाय ब्रह्मणे परमात्मने
श्रीभगवानुवाच
नैकान्तिनो मे मयि जात्विहाशिष आशासतेऽमुत्र च ये भवद्विधाः
तथापि मन्वन्तरमेतदत्र दैत्येश्वराणामनुभुङ्क्ष्व भोगान्‌
कथा मदीया जुषमाणः प्रियास्त्वमावेश्य मामात्मनि सन्तमेकम्‌
सर्वेषु भूतेष्वधियज्ञमीशं यजस्व योगेन च कर्म हिन्वन्‌
भोगेन पुण्यं कुशलेन पापं कलेवरं कालजवेन हित्वा
कीर्तिं विशुद्धां सुरलोकगीतां विताय मामेष्यसि मुक्तबन्धः
य एतत्कीर्तयेन्मह्यं त्वया गीतमिदं नरः
त्वां च मां च स्मरन्काले कर्मबन्धात्प्रमुच्यते
श्रीप्रह्राद उवाच
वरं वरय एतत्ते वरदेशान्महेश्वर
यदनिन्दत्पिता मे त्वामविद्वांस्तेज ऐश्वरम्‌
विद्धामर्षाशयः साक्षात्सर्वलोकगुरुं प्रभुम्‌
भ्रातृहेति मृषादृष्टिस्त्वद्भक्ते मयि चाघवान्‌
तस्मात्पिता मे पूयेत दुरन्ताद्दुस्तरादघात्‌
पूतस्तेऽपाङ्गसंदृष्टस्तदा कृपणवत्सल
श्रीभगवानुवाच
त्रिःसप्तभिः पिता पूतः पितृभिः सह तेऽनघ
यत्साधोऽस्य कुले जातो भवान्वै कुलपावनः
यत्र यत्र च मद्भक्ताः प्रशान्ताः समदर्शिनः
साधवः समुदाचारास्ते पूयन्तेऽपि कीकटाः
सर्वात्मना न हिंसन्ति भूतग्रामेषु किञ्चन
उच्चावचेषु दैत्येन्द्र मद्भावविगतस्पृहाः
भवन्ति पुरुषा लोके मद्भक्तास्त्वामनुव्रताः
भवान्मे खलु भक्तानां सर्वेषां प्रतिरूपधृक्‌
कुरु त्वं प्रेतकृत्यानि पितुः पूतस्य सर्वशः
मदङ्गस्पर्शनेनाङ्ग लोकान्यास्यति सुप्रजाः
पित्र्यं च स्थानमातिष्ठ यथोक्तं ब्रह्मवादिभिः
मय्यावेश्य मनस्तात कुरु कर्माणि मत्परः
श्रीनारद उवाच
प्रह्रादोऽपि तथा चक्रे पितुर्यत्साम्परायिकम्‌
यथाह भगवान्राजन्नभिषिक्तो द्विजातिभिः
प्रसादसुमुखं दृष्ट्वा ब्रह्मा नरहरिं हरिम्‌
स्तुत्वा वाग्भिः पवित्राभिः प्राह देवादिभिर्वृतः
श्रीब्रह्मोवाच
देवदेवाखिलाध्यक्ष भूतभावन पूर्वज
दिष्ट्या ते निहतः पापो लोकसन्तापनोऽसुरः
योऽसौ लब्धवरो मत्तो न वध्यो मम सृष्टिभिः
तपोयोगबलोन्नद्धः समस्तनिगमानहन्‌
दिष्ट्या तत्तनयः साधुर्महाभागवतोऽर्भकः
त्वया विमोचितो मृत्योर्दिष्ट्या त्वां समितोऽधुना
एतद्वपुस्ते भगवन्ध्यायतः परमात्मनः
सर्वतो गोप्तृ सन्त्रासान्मृत्योरपि जिघांसतः
श्रीभगवानुवाच
मैवं विभोऽसुराणां ते प्रदेयः पद्मसम्भव
वरः क्रूरनिसर्गाणामहीनाममृतं यथा
श्रीनारद उवाच
इत्युक्त्वा भगवान्राजंस्ततश्चान्तर्दधे हरिः
अदृश्यः सर्वभूतानां पूजितः परमेष्ठिना
ततः सम्पूज्य शिरसा ववन्दे परमेष्ठिनम्‌
भवं प्रजापतीन्देवान्प्रह्रादो भगवत्कलाः
ततः काव्यादिभिः सार्धं मुनिभिः कमलासनः
दैत्यानां दानवानां च प्रह्रादमकरोत्पतिम्‌
प्रतिनन्द्य ततो देवाः प्रयुज्य परमाशिषः
स्वधामानि ययू राजन्ब्रह्माद्याः प्रतिपूजिताः
एवं च पार्षदौ विष्णोः पुत्रत्वं प्रापितौ दितेः
हृदि स्थितेन हरिणा वैरभावेन तौ हतौ
पुनश्च विप्रशापेन राक्षसौ तौ बभूवतुः
कुम्भकर्णदशग्रीवौ हतौ तौ रामविक्रमैः
शयानौ युधि निर्भिन्न हृदयौ रामशायकैः
तच्चित्तौ जहतुर्देहं यथा प्राक्तनजन्मनि
ताविहाथ पुनर्जातौ शिशुपालकरूषजौ
हरौ वैरानुबन्धेन पश्यतस्ते समीयतुः
एनः पूर्वकृतं यत्तद्राजानः कृष्णवैरिणः
जहुस्तेऽन्ते तदात्मानः कीटः पेशस्कृतो यथा
यथा यथा भगवतो भक्त्या परमयाभिदा
नृपाश्चैद्यादयः सात्म्यं हरेस्तच्चिन्तया ययुः
आख्यातं सर्वमेतत्ते यन्मां त्वं परिपृष्टवान्‌
दमघोषसुतादीनां हरेः सात्म्यमपि द्विषाम्‌
एषा ब्रह्मण्यदेवस्य कृष्णस्य च महात्मनः
अवतारकथा पुण्या वधो यत्रादिदैत्ययोः
प्रह्रादस्यानुचरितं महाभागवतस्य च
भक्तिर्ज्ञानं विरक्तिश्च याथार्थ्यं चास्य वै हरेः
सर्गस्थित्यप्ययेशस्य गुणकर्मानुवर्णनम्‌
परावरेषां स्थानानां कालेन व्यत्ययो महान्‌
धर्मो भागवतानां च भगवान्येन गम्यते
आख्यानेऽस्मिन्समाम्नातमाध्यात्मिकमशेषतः
य एतत्पुण्यमाख्यानं विष्णोर्वीर्योपबृंहितम्‌
कीर्तयेच्छ्रद्धया श्रुत्वा कर्मपाशैर्विमुच्यते
एतद्य आदिपुरुषस्य मृगेन्द्रलीलां
दैत्येन्द्रयूथपवधं प्रयतः पठेत
दैत्यात्मजस्य च सतां प्रवरस्य पुण्यं
श्रुत्वानुभावमकुतोभयमेति लोकम्‌
यूयं नृलोके बत भूरिभागा लोकं पुनाना मुनयोऽभियन्ति
येषां गृहानावसतीति साक्षाद्गूढं परं ब्रह्म मनुष्यलिङ्गम्‌
स वा अयं ब्रह्म महद्विमृग्य कैवल्यनिर्वाणसुखानुभूतिः
प्रियः सुहृद्वः खलु मातुलेय आत्मार्हणीयो विधिकृद्गुरुश्च
न यस्य साक्षाद्भवपद्मजादिभी रूपं धिया वस्तुतयोपवर्णितम्‌
मौनेन भक्त्योपशमेन पूजितः प्रसीदतामेष स सात्वतां पतिः
स एष भगवान्राजन्व्यतनोद्विहतं यशः
पुरा रुद्रस्य देवस्य मयेनानन्तमायिना
राजोवाच
कस्मिन्कर्मणि देवस्य मयोऽहन्जगदीशितुः
यथा चोपचिता कीर्तिः कृष्णेनानेन कथ्यताम्‌
श्रीनारद उवाच
निर्जिता असुरा देवैर्युध्यनेनोपबृंहितैः
मायिनां परमाचार्यं मयं शरणमाययुः
स निर्माय पुरस्तिस्रो हैमीरौप्यायसीर्विभुः
दुर्लक्ष्यापायसंयोगा दुर्वितर्क्यपरिच्छदाः
ताभिस्तेऽसुरसेनान्यो लोकांस्त्रीन्सेश्वरान्नृप
स्मरन्तो नाशयां चक्रुः पूर्ववैरमलक्षिताः
ततस्ते सेश्वरा लोका उपासाद्येश्वरं नताः
त्राहि नस्तावकान्देव विनष्टांस्त्रिपुरालयैः
अथानुगृह्य भगवान्मा भैष्टेति सुरान्विभुः
शरं धनुषि सन्धाय पुरेष्वस्त्रं व्यमुञ्चत
ततोऽग्निवर्णा इषव उत्पेतुः सूर्यमण्डलात्‌
यथा मयूखसन्दोहा नादृश्यन्त पुरो यतः
तैः स्पृष्टा व्यसवः सर्वे निपेतुः स्म पुरौकसः
तानानीय महायोगी मयः कूपरसेऽक्षिपत्‌
सिद्धामृतरसस्पृष्टा वज्रसारा महौजसः
उत्तस्थुर्मेघदलना वैद्युता इव वह्नयः
विलोक्य भग्नसङ्कल्पं विमनस्कं वृषध्वजम्‌
तदायं भगवान्विष्णुस्तत्रोपायमकल्पयत्‌
वत्सश्चासीत्तदा ब्रह्मा स्वयं विष्णुरयं हि गौः
प्रविश्य त्रिपुरं काले रसकूपामृतं पपौ
तेऽसुरा ह्यपि पश्यन्तो न न्यषेधन्विमोहिताः
तद्विज्ञाय महायोगी रसपालानिदं जगौ
स्मयन्विशोकः शोकार्तान्स्मरन्दैवगतिं च ताम्‌
देवोऽसुरो नरोऽन्यो वा नेश्वरोऽस्तीह कश्चन
आत्मनोऽन्यस्य वा दिष्टं दैवेनापोहितुं द्वयोः
अथासौ शक्तिभिः स्वाभिः शम्भोः प्राधानिकं व्यधात्‌
धर्मज्ञानविरक्त्यृद्धि तपोविद्याक्रियादिभिः
रथं सूतं ध्वजं वाहान्धनुर्वर्मशरादि यत्‌
सन्नद्धो रथमास्थाय शरं धनुरुपाददे
शरं धनुषि सन्धाय मुहूर्तेऽभिजितीश्वरः
ददाह तेन दुर्भेद्या हरोऽथ त्रिपुरो नृप
दिवि दुन्दुभयो नेदुर्विमानशतसङ्कुलाः
देवर्षिपितृसिद्धेशा जयेति कुसुमोत्करैः
अवाकिरन्जगुर्हृष्टा ननृतुश्चाप्सरोगणाः
एवं दग्ध्वा पुरस्तिस्रो भगवान्पुरहा नृप
ब्रह्मादिभिः स्तूयमानः स्वं धाम प्रत्यपद्यत
एवं विधान्यस्य हरेः स्वमायया विडम्बमानस्य नृलोकमात्मनः
वीर्याणि गीतान्यृषिभिर्जगद्गुरोर्लोकं पुनानान्यपरं वदामि किम्‌
श्रीशुक उवाच
श्रुत्वेहितं साधु सभासभाजितं महत्तमाग्रण्य उरुक्रमात्मनः
युधिष्ठिरो दैत्यपतेर्मुदान्वितः पप्रच्छ भूयस्तनयं स्वयम्भुवः
श्रीयुधिष्ठिर उवाच
भगवन्श्रोतुमिच्छामि नृणां धर्मं सनातनम्‌
वर्णाश्रमाचारयुतं यत्पुमान्विन्दते परम्‌
भवान्प्रजापतेः साक्षादात्मजः परमेष्ठिनः
सुतानां सम्मतो ब्रह्मंस्तपोयोगसमाधिभिः
नारायणपरा विप्रा धर्मं गुह्यं परं विदुः
करुणाः साधवः शान्तास्त्वद्विधा न तथापरे
श्रीनारद उवाच
नत्वा भगवतेऽजाय लोकानां धर्मसेतवे
वक्ष्ये सनातनं धर्मं नारायणमुखाच्छ्रुतम्‌
योऽवतीर्यात्मनोऽंशेन दाक्षायण्यां तु धर्मतः
लोकानां स्वस्तयेऽध्यास्ते तपो बदरिकाश्रमे
धर्ममूलं हि भगवान्सर्ववेदमयो हरिः
स्मृतं च तद्विदां राजन्येन चात्मा प्रसीदति
सत्यं दया तपः शौचं तितिक्षेक्षा शमो दमः
अहिंसा ब्रह्मचर्यं च त्यागः स्वाध्याय आर्जवम्‌
सन्तोषः समदृक्सेवा ग्राम्येहोपरमः शनैः
नृणां विपर्ययेहेक्षा मौनमात्मविमर्शनम्‌
अन्नाद्यादेः संविभागो भूतेभ्यश्च यथार्हतः
तेष्वात्मदेवताबुद्धिः सुतरां नृषु पाण्डव
श्रवणं कीर्तनं चास्य स्मरणं महतां गतेः
सेवेज्यावनतिर्दास्यं सख्यमात्मसमर्पणम्‌
नृणामयं परो धर्मः सर्वेषां समुदाहृतः
त्रिंशल्लक्षणवान्राजन्सर्वात्मा येन तुष्यति
संस्कारा यत्राविच्छिन्नाः स द्विजोऽजो जगाद यम्‌
इज्याध्ययनदानानि विहितानि द्विजन्मनाम्‌
जन्मकर्मावदातानां क्रियाश्चाश्रमचोदिताः
विप्रस्याध्ययनादीनि षडन्यस्याप्रतिग्रहः
राज्ञो वृत्तिः प्रजागोप्तुरविप्राद्वा करादिभिः
वैश्यस्तु वार्तावृत्तिः स्यान्नित्यं ब्रह्मकुलानुगः
शूद्रस्य द्विजशुश्रूषा वृत्तिश्च स्वामिनो भवेत्‌
वार्ता विचित्रा शालीन यायावरशिलोञ्छनम्‌
विप्रवृत्तिश्चतुर्धेयं श्रेयसी चोत्तरोत्तरा
जघन्यो नोत्तमां वृत्तिमनापदि भजेन्नरः
ऋते राजन्यमापत्सु सर्वेषामपि सर्वशः
ऋतामृताभ्यां जीवेत मृतेन प्रमृतेन वा
सत्यानृताभ्यामपि वा न श्ववृत्त्या कदाचन
ऋतमुञ्छशिलं प्रोक्तममृतं यदयाचितम्‌
मृतं तु नित्ययाच्ञा स्यात्प्रमृतं कर्षणं स्मृतम्‌
सत्यानृतं च वाणिज्यं श्ववृत्तिर्नीचसेवनम्‌
वर्जयेत्तां सदा विप्रो राजन्यश्च जुगुप्सिताम्‌
सर्ववेदमयो विप्रः सर्वदेवमयो नृपः
शमो दमस्तपः शौचं सन्तोषः क्षान्तिरार्जवम्‌
ज्ञानं दयाच्युतात्मत्वं सत्यं च ब्रह्मलक्षणम्‌
शौर्यं वीर्यं धृतिस्तेजस्त्यागश्चात्मजयः क्षमा
ब्रह्मण्यता प्रसादश्च सत्यं च क्षत्रलक्षणम्‌
देवगुर्वच्युते भक्तिस्त्रिवर्गपरिपोषणम्‌
आस्तिक्यमुद्यमो नित्यं नैपुण्यं वैश्यलक्षणम्‌
शूद्रस्य सन्नतिः शौचं सेवा स्वामिन्यमायया
अमन्त्रयज्ञो ह्यस्तेयं सत्यं गोविप्ररक्षणम्‌
स्त्रीणां च पतिदेवानां तच्छुश्रूषानुकूलता
तद्बन्धुष्वनुवृत्तिश्च नित्यं तद्व्रतधारणम्‌
सम्मार्जनोपलेपाभ्यां गृहमण्डनवर्तनैः
स्वयं च मण्डिता नित्यं परिमृष्टपरिच्छदा
कामैरुच्चावचैः साध्वी प्रश्रयेण दमेन च
वाक्यैः सत्यैः प्रियैः प्रेम्णा काले काले भजेत्पतिम्‌
सन्तुष्टालोलुपा दक्षा धर्मज्ञा प्रियसत्यवाक्‌
अप्रमत्ता शुचिः स्निग्धा पतिं त्वपतितं भजेत्‌
या पतिं हरिभावेन भजेत्श्रीरिव तत्परा
हर्यात्मना हरेर्लोके पत्या श्रीरिव मोदते
वृत्तिः सङ्करजातीनां तत्तत्कुलकृता भवेत्‌
अचौराणामपापानामन्त्यजान्तेवसायिनाम्‌
प्रायः स्वभावविहितो नृणां धर्मो युगे युगे
वेददृग्भिः स्मृतो राजन्प्रेत्य चेह च शर्मकृत्‌
वृत्त्या स्वभावकृतया वर्तमानः स्वकर्मकृत्‌
हित्वा स्वभावजं कर्म शनैर्निर्गुणतामियात्‌
उप्यमानं मुहुः क्षेत्रं स्वयं निर्वीर्यतामियात्‌
न कल्पते पुनः सूत्यै उप्तं बीजं च नश्यति
एवं कामाशयं चित्तं कामानामतिसेवया
विरज्येत यथा राजन्नग्निवत्कामबिन्दुभिः
यस्य यल्लक्षणं प्रोक्तं पुंसो वर्णाभिव्यञ्जकम्‌
यदन्यत्रापि दृश्येत तत्तेनैव विनिर्दिशेत्‌
श्रीनारद उवाच
ब्रह्मचारी गुरुकुले वसन्दान्तो गुरोर्हितम्‌
आचरन्दासवन्नीचो गुरौ सुदृढसौहृदः
सायं प्रातरुपासीत गुर्वग्न्यर्कसुरोत्तमान्‌
सन्ध्ये उभे च यतवाग्जपन्ब्रह्म समाहितः
छन्दांस्यधीयीत गुरोराहूतश्चेत्सुयन्त्रितः
उपक्रमेऽवसाने च चरणौ शिरसा नमेत्‌
मेखलाजिनवासांसि जटादण्डकमण्डलून्‌
बिभृयादुपवीतं च दर्भपाणिर्यथोदितम्‌
सायं प्रातश्चरेद्भैक्ष्यं गुरवे तन्निवेदयेत्‌
भुञ्जीत यद्यनुज्ञातो नो चेदुपवसेत्क्वचित्‌
सुशीलो मितभुग्दक्षः श्रद्दधानो जितेन्द्रियः
यावदर्थं व्यवहरेत्स्त्रीषु स्त्रीनिर्जितेषु च
वर्जयेत्प्रमदागाथामगृहस्थो बृहद्व्रतः
इन्द्रियाणि प्रमाथीनि हरन्त्यपि यतेर्मनः
केशप्रसाधनोन्मर्द स्नपनाभ्यञ्जनादिकम्‌
गुरुस्त्रीभिर्युवतिभिः कारयेन्नात्मनो युवा
नन्वग्निः प्रमदा नाम घृतकुम्भसमः पुमान्‌
सुतामपि रहो जह्यादन्यदा यावदर्थकृत्‌
कल्पयित्वात्मना यावदाभासमिदमीश्वरः
द्वैतं तावन्न विरमेत्ततो ह्यस्य विपर्ययः
एतत्सर्वं गृहस्थस्य समाम्नातं यतेरपि
गुरुवृत्तिर्विकल्पेन गृहस्थस्यर्तुगामिनः
अञ्जनाभ्यञ्जनोन्मर्द स्त्र्यवलेखामिषं मधु
स्रग्गन्धलेपालङ्कारांस्त्यजेयुर्ये बृहद्व्रताः
उषित्वैवं गुरुकुले द्विजोऽधीत्यावबुध्य च
त्रयीं साङ्गोपनिषदं यावदर्थं यथाबलम्‌
दत्त्वा वरमनुज्ञातो गुरोः कामं यदीश्वरः
गृहं वनं वा प्रविशेत्प्रव्रजेत्तत्र वा वसेत्‌
अग्नौ गुरावात्मनि च सर्वभूतेष्वधोक्षजम्‌
भूतैः स्वधामभिः पश्येदप्रविष्टं प्रविष्टवत्‌
एवं विधो ब्रह्मचारी वानप्रस्थो यतिर्गृही
चरन्विदितविज्ञानः परं ब्रह्माधिगच्छति
वानप्रस्थस्य वक्ष्यामि नियमान्मुनिसम्मतान्‌
यानास्थाय मुनिर्गच्छेदृषिलोकमुहाञ्जसा
न कृष्टपच्यमश्नीयादकृष्टं चाप्यकालतः
अग्निपक्वमथामं वा अर्कपक्वमुताहरेत्‌
वन्यैश्चरुपुरोडाशान्निर्वपेत्कालचोदितान्‌
लब्धे नवे नवेऽन्नाद्ये पुराणं च परित्यजेत्‌
अग्न्यर्थमेव शरणमुटजं वाद्रिकन्दरम्‌
श्रयेत हिमवाय्वग्नि वर्षार्कातपषाट्स्वयम्‌
केशरोमनखश्मश्रु मलानि जटिलो दधत्‌
कमण्डल्वजिने दण्ड वल्कलाग्निपरिच्छदान्‌
चरेद्वने द्वादशाब्दानष्टौ वा चतुरो मुनिः
द्वावेकं वा यथा बुद्धिर्न विपद्येत कृच्छ्रतः
यदाकल्पः स्वक्रियायां व्याधिभिर्जरयाथवा
आन्वीक्षिक्यां वा विद्यायां कुर्यादनशनादिकम्‌
आत्मन्यग्नीन्समारोप्य सन्न्यस्याहं ममात्मताम्‌
कारणेषु न्यसेत्सम्यक्सङ्घातं तु यथार्हतः
खे खानि वायौ निश्वासांस्तेजःसूष्माणमात्मवान्‌
अप्स्वसृक्ष्लेष्मपूयानि क्षितौ शेषं यथोद्भवम्‌
वाचमग्नौ सवक्तव्यामिन्द्रे शिल्पं करावपि
पदानि गत्या वयसि रत्योपस्थं प्रजापतौ
मृत्यौ पायुं विसर्गं च यथास्थानं विनिर्दिशेत्‌
दिक्षु श्रोत्रं सनादेन स्पर्शेनाध्यात्मनि त्वचम्‌
रूपाणि चक्षुषा राजन्ज्योतिष्यभिनिवेशयेत्‌
अप्सु प्रचेतसा जिह्वां घ्रेयैर्घ्राणं क्षितौ न्यसेत्‌
मनो मनोरथैश्चन्द्रे बुद्धिं बोध्यैः कवौ परे
कर्माण्यध्यात्मना रुद्रे यदहं ममताक्रिया
सत्त्वेन चित्तं क्षेत्रज्ञे गुणैर्वैकारिकं परे
अप्सु क्षितिमपो ज्योतिष्यदो वायौ नभस्यमुम्‌
कूटस्थे तच्च महति तदव्यक्तेऽक्षरे च तत्‌
इत्यक्षरतयात्मानं चिन्मात्रमवशेषितम्‌
ज्ञात्वाद्वयोऽथ विरमेद्दग्धयोनिरिवानलः
श्रीनारद उवाच
कल्पस्त्वेवं परिव्रज्य देहमात्रावशेषितः
ग्रामैकरात्रविधिना निरपेक्षश्चरेन्महीम्‌
बिभृयाद्यद्यसौ वासः कौपीनाच्छादनं परम्‌
त्यक्तं न लिङ्गाद्दण्डादेरन्यत्किञ्चिदनापदि
एक एव चरेद्भिक्षुरात्मारामोऽनपाश्रयः
सर्वभूतसुहृच्छान्तो नारायणपरायणः
पश्येदात्मन्यदो विश्वं परे सदसतोऽव्यये
आत्मानं च परं ब्रह्म सर्वत्र सदसन्मये
सुप्तिप्रबोधयोः सन्धावात्मनो गतिमात्मदृक्‌
पश्यन्बन्धं च मोक्षं च मायामात्रं न वस्तुतः
नाभिनन्देद्ध्रुवं मृत्युमध्रुवं वास्य जीवितम्‌
कालं परं प्रतीक्षेत भूतानां प्रभवाप्ययम्‌
नासच्छास्त्रेषु सज्जेत नोपजीवेत जीविकाम्‌
वादवादांस्त्यजेत्तर्कान्पक्षं कंच न संश्रयेत्‌
न शिष्याननुबध्नीत ग्रन्थान्नैवाभ्यसेद्बहून्‌
न व्याख्यामुपयुञ्जीत नारम्भानारभेत्क्वचित्‌
न यतेराश्रमः प्रायो धर्महेतुर्महात्मनः
शान्तस्य समचित्तस्य बिभृयादुत वा त्यजेत्‌
अव्यक्तलिङ्गो व्यक्तार्थो मनीष्युन्मत्तबालवत्‌
कविर्मूकवदात्मानं स दृष्ट्या दर्शयेन्नृणाम्‌
अत्राप्युदाहरन्तीममितिहासं पुरातनम्‌
प्रह्रादस्य च संवादं मुनेराजगरस्य च
तं शयानं धरोपस्थे कावेर्यां सह्यसानुनि
रजस्वलैस्तनूदेशैर्निगूढामलतेजसम्‌
ददर्श लोकान्विचरन्लोकतत्त्वविवित्सया
वृतोऽमात्यैः कतिपयैः प्रह्रादो भगवत्प्रियः
कर्मणाकृतिभिर्वाचा लिङ्गैर्वर्णाश्रमादिभिः
न विदन्ति जना यं वै सोऽसाविति न वेति च
तं नत्वाभ्यर्च्य विधिवत्पादयोः शिरसा स्पृशन्‌
विवित्सुरिदमप्राक्षीन्महाभागवतोऽसुरः
बिभर्षि कायं पीवानं सोद्यमो भोगवान्यथा
वित्तं चैवोद्यमवतां भोगो वित्तवतामिह
भोगिनां खलु देहोऽयं पीवा भवति नान्यथा
न ते शयानस्य निरुद्यमस्य ब्रह्मन्नु हार्थो यत एव भोगः
अभोगिनोऽयं तव विप्र देहः पीवा यतस्तद्वद नः क्षमं चेत्‌
कविः कल्पो निपुणदृक्चित्रप्रियकथः समः
लोकस्य कुर्वतः कर्म शेषे तद्वीक्षितापि वा
श्रीनारद उवाच
स इत्थं दैत्यपतिना परिपृष्टो महामुनिः
स्मयमानस्तमभ्याह तद्वागमृतयन्त्रितः
श्रीब्राह्मण उवाच
वेदेदमसुरश्रेष्ठ भवान्नन्वार्यसम्मतः
ईहोपरमयोर्न्ऱ्‌^ईणां पदान्यध्यात्मचक्षुषा
यस्य नारायणो देवो भगवान्हृद्गतः सदा
भक्त्या केवलयाज्ञानं धुनोति ध्वान्तमर्कवत्‌
तथापि ब्रूमहे प्रश्नांस्तव राजन्यथाश्रुतम्‌
सम्भाषणीयो हि भवानात्मनः शुद्धिमिच्छता
तृष्णया भववाहिन्या योग्यैः कामैरपूर्यया
कर्माणि कार्यमाणोऽहं नानायोनिषु योजितः
यदृच्छया लोकमिमं प्रापितः कर्मभिर्भ्रमन्‌
स्वर्गापवर्गयोर्द्वारं तिरश्चां पुनरस्य च
तत्रापि दम्पतीनां च सुखायान्यापनुत्तये
कर्माणि कुर्वतां दृष्ट्वा निवृत्तोऽस्मि विपर्ययम्‌
सुखमस्यात्मनो रूपं सर्वेहोपरतिस्तनुः
मनःसंस्पर्शजान्दृष्ट्वा भोगान्स्वप्स्यामि संविशन्‌
इत्येतदात्मनः स्वार्थं सन्तं विस्मृत्य वै पुमान्‌
विचित्रामसति द्वैते घोरामाप्नोति संसृतिम्‌
जलं तदुद्भवैश्छन्नं हित्वाज्ञो जलकाम्यया
मृगतृष्णामुपाधावेत्तथान्यत्रार्थदृक्स्वतः
देहादिभिर्दैवतन्त्रैरात्मनः सुखमीहतः
दुःखात्ययं चानीशस्य क्रिया मोघाः कृताः कृताः
आध्यात्मिकादिभिर्दुःखैरविमुक्तस्य कर्हिचित्‌
मर्त्यस्य कृच्छ्रोपनतैरर्थैः कामैः क्रियेत किम्‌
पश्यामि धनिनां क्लेशं लुब्धानामजितात्मनाम्‌
भयादलब्धनिद्राणां सर्वतोऽभिविशङ्किनाम्‌
राजतश्चौरतः शत्रोः स्वजनात्पशुपक्षितः
अर्थिभ्यः कालतः स्वस्मान्नित्यं प्राणार्थवद्भयम्‌
शोकमोहभयक्रोध रागक्लैब्यश्रमादयः
यन्मूलाः स्युर्नृणां जह्यात्स्पृहां प्राणार्थयोर्बुधः
मधुकारमहासर्पौ लोकेऽस्मिन्नो गुरूत्तमौ
वैराग्यं परितोषं च प्राप्ता यच्छिक्षया वयम्‌
विरागः सर्वकामेभ्यः शिक्षितो मे मधुव्रतात्‌
कृच्छ्राप्तं मधुवद्वित्तं हत्वाप्यन्यो हरेत्पतिम्‌
अनीहः परितुष्टात्मा यदृच्छोपनतादहम्‌
नो चेच्छये बह्वहानि महाहिरिव सत्त्ववान्‌
क्वचिदल्पं क्वचिद्भूरि भुञ्जेऽन्नं स्वाद्वस्वादु वा
क्वचिद्भूरि गुणोपेतं गुणहीनमुत क्वचित्‌
श्रद्धयोपहृतं क्वापि कदाचिन्मानवर्जितम्‌
भुञ्जे भुक्त्वाथ कस्मिंश्चिद्दिवा नक्तं यदृच्छया
क्षौमं दुकूलमजिनं चीरं वल्कलमेव वा
वसेऽन्यदपि सम्प्राप्तं दिष्टभुक्तुष्टधीरहम्‌
क्वचिच्छये धरोपस्थे तृणपर्णाश्मभस्मसु
क्वचित्प्रासादपर्यङ्के कशिपौ वा परेच्छया
क्वचित्स्नातोऽनुलिप्ताङ्गः सुवासाः स्रग्व्यलङ्कृतः
रथेभाश्वैश्चरे क्वापि दिग्वासा ग्रहवद्विभो
नाहं निन्दे न च स्तौमि स्वभावविषमं जनम्‌
एतेषां श्रेय आशासे उतैकात्म्यं महात्मनि
विकल्पं जुहुयाच्चित्तौ तां मनस्यर्थविभ्रमे
मनो वैकारिके हुत्वा तं मायायां जुहोत्यनु
आत्मानुभूतौ तां मायां जुहुयात्सत्यदृङ्मुनिः
ततो निरीहो विरमेत्स्वानुभूत्यात्मनि स्थितः
स्वात्मवृत्तं मयेत्थं ते सुगुप्तमपि वर्णितम्‌
व्यपेतं लोकशास्त्राभ्यां भवान्हि भगवत्परः
श्रीनारद उवाच
धर्मं पारमहंस्यं वै मुनेः श्रुत्वासुरेश्वरः
पूजयित्वा ततः प्रीत आमन्त्र्य प्रययौ गृहम्‌
श्रीयुधिष्ठिर उवाच
गृहस्थ एतां पदवीं विधिना येन चाञ्जसा
यायाद्देवऋषे ब्रूहि मादृशो गृहमूढधीः
श्रीनारद उवाच
गृहेष्ववस्थितो राजन्क्रियाः कुर्वन्यथोचिताः
वासुदेवार्पणं साक्षादुपासीत महामुनीन्‌
शृण्वन्भगवतोऽभीक्ष्णमवतारकथामृतम्‌
श्रद्दधानो यथाकालमुपशान्तजनावृतः
सत्सङ्गाच्छनकैः सङ्गमात्मजायात्मजादिषु
विमुञ्चेन्मुच्यमानेषु स्वयं स्वप्नवदुत्थितः
यावदर्थमुपासीनो देहे गेहे च पण्डितः
विरक्तो रक्तवत्तत्र नृलोके नरतां न्यसेत्‌
ज्ञातयः पितरौ पुत्रा भ्रातरः सुहृदोऽपरे
यद्वदन्ति यदिच्छन्ति चानुमोदेत निर्ममः
दिव्यं भौमं चान्तरीक्षं वित्तमच्युतनिर्मितम्‌
तत्सर्वमुपयुञ्जान एतत्कुर्यात्स्वतो बुधः
यावद्भ्रियेत जठरं तावत्स्वत्वं हि देहिनाम्‌
अधिकं योऽभिमन्येत स स्तेनो दण्डमर्हति
मृगोष्ट्रखरमर्काखु सरीसृप्खगमक्षिकाः
आत्मनः पुत्रवत्पश्येत्तैरेषामन्तरं कियत्‌
त्रिवर्गं नातिकृच्छ्रेण भजेत गृहमेध्यपि
यथादेशं यथाकालं यावद्दैवोपपादितम्‌
आश्वाघान्तेऽवसायिभ्यः कामान्संविभजेद्यथा
अप्येकामात्मनो दारां नृणां स्वत्वग्रहो यतः
जह्याद्यदर्थे स्वान्प्राणान्हन्याद्वा पितरं गुरुम्‌
तस्यां स्वत्वं स्त्रियां जह्याद्यस्तेन ह्यजितो जितः
कृमिविड्भस्मनिष्ठान्तं क्वेदं तुच्छं कलेवरम्‌
क्व तदीयरतिर्भार्या क्वायमात्मा नभश्छदिः
सिद्धैर्यज्ञावशिष्टार्थैः कल्पयेद्वृत्तिमात्मनः
शेषे स्वत्वं त्यजन्प्राज्ञः पदवीं महतामियात्‌
देवानृषीन्नृभूतानि पित्ऱ्‌^ईनात्मानमन्वहम्‌
स्ववृत्त्यागतवित्तेन यजेत पुरुषं पृथक्‌
यर्ह्यात्मनोऽधिकाराद्याः सर्वाः स्युर्यज्ञसम्पदः
वैतानिकेन विधिना अग्निहोत्रादिना यजेत्‌
न ह्यग्निमुखतोऽयं वै भगवान्सर्वयज्ञभुक्‌
इज्येत हविषा राजन्यथा विप्रमुखे हुतैः
तस्माद्ब्राह्मणदेवेषु मर्त्यादिषु यथार्हतः
तैस्तैः कामैर्यजस्वैनं क्षेत्रज्ञं ब्राह्मणाननु
कुर्यादपरपक्षीयं मासि प्रौष्ठपदे द्विजः
श्राद्धं पित्रोर्यथावित्तं तद्बन्धूनां च वित्तवान्‌
अयने विषुवे कुर्याद्व्यतीपाते दिनक्षये
चन्द्रादित्योपरागे च द्वादश्यां श्रवणेषु च
तृतीयायां शुक्लपक्षे नवम्यामथ कार्तिके
चतसृष्वप्यष्टकासु हेमन्ते शिशिरे तथा
माघे च सितसप्तम्यां मघाराकासमागमे
राकया चानुमत्या च मासर्क्षाणि युतान्यपि
द्वादश्यामनुराधा स्याच्छ्रवणस्तिस्र उत्तराः
तिसृष्वेकादशी वासु जन्मर्क्षश्रोणयोगयुक्‌
त एते श्रेयसः काला न्ऱ्‌^ईणां श्रेयोविवर्धनाः
कुर्यात्सर्वात्मनैतेषु श्रेयोऽमोघं तदायुषः
एषु स्नानं जपो होमो व्रतं देवद्विजार्चनम्‌
पितृदेवनृभूतेभ्यो यद्दत्तं तद्ध्यनश्वरम्‌
संस्कारकालो जायाया अपत्यस्यात्मनस्तथा
प्रेतसंस्था मृताहश्च कर्मण्यभ्युदये नृप
अथ देशान्प्रवक्ष्यामि धर्मादिश्रेयाअवहान्‌
स वै पुण्यतमो देशः सत्पात्रं यत्र लभ्यते
बिम्बं भगवतो यत्र सर्वमेतच्चराचरम्‌
यत्र ह ब्राह्मणकुलं तपोविद्यादयान्वितम्‌
यत्र यत्र हरेरर्चा स देशः श्रेयसां पदम्‌
यत्र गङ्गादयो नद्यः पुराणेषु च विश्रुताः
सरांसि पुष्करादीनि क्षेत्राण्यर्हाश्रितान्युत
कुरुक्षेत्रं गयशिरः प्रयागः पुलहाश्रमः
नैमिषं फाल्गुनं सेतुः प्रभासोऽथ कुशस्थली
वाराणसी मधुपुरी पम्पा बिन्दुसरस्तथा
नारायणाश्रमो नन्दा सीतारामाश्रमादयः
सर्वे कुलाचला राजन्महेन्द्रमलयादयः
एते पुण्यतमा देशा हरेरर्चाश्रिताश्च ये
एतान्देशान्निषेवेत श्रेयस्कामो ह्यभीक्ष्णशः
धर्मो ह्यत्रेहितः पुंसां सहस्राधिफलोदयः
पात्रं त्वत्र निरुक्तं वै कविभिः पात्रवित्तमैः
हरिरेवैक उर्वीश यन्मयं वै चराचरम्‌
देवर्ष्यर्हत्सु वै सत्सु तत्र ब्रह्मात्मजादिषु
राजन्यदग्रपूजायां मतः पात्रतयाच्युतः
जीवराशिभिराकीर्ण अण्डकोशाङ्घ्रिपो महान्‌
तन्मूलत्वादच्युतेज्या सर्वजीवात्मतर्पणम्‌
पुराण्यनेन सृष्टानि नृतिर्यगृषिदेवताः
शेते जीवेन रूपेण पुरेषु पुरुषो ह्यसौ
तेष्वेव भगवान्राजंस्तारतम्येन वर्तते
तस्मात्पात्रं हि पुरुषो यावानात्मा यथेयते
दृष्ट्वा तेषां मिथो नृणामवज्ञानात्मतां नृप
त्रेतादिषु हरेरर्चा क्रियायै कविभिः कृता
ततोऽर्चायां हरिं केचित्संश्रद्धाय सपर्यया
उपासत उपास्तापि नार्थदा पुरुषद्विषाम्‌
पुरुषेष्वपि राजेन्द्र सुपात्रं ब्राह्मणं विदुः
तपसा विद्यया तुष्ट्या धत्ते वेदं हरेस्तनुम्‌
नन्वस्य ब्राह्मणा राजन्कृष्णस्य जगदात्मनः
पुनन्तः पादरजसा त्रिलोकीं दैवतं महत्‌
श्रीनारद उवाच
कर्मनिष्ठा द्विजाः केचित्तपोनिष्ठा नृपापरे
स्वाध्यायेऽन्ये प्रवचने केचन ज्ञानयोगयोः
ज्ञाननिष्ठाय देयानि कव्यान्यानन्त्यमिच्छता
दैवे च तदभावे स्यादितरेभ्यो यथार्हतः
द्वौ दैवे पितृकार्ये त्रीनेकैकमुभयत्र वा
भोजयेत्सुसमृद्धोऽपि श्राद्धे कुर्यान्न विस्तरम्‌
देशकालोचितश्रद्धा द्रव्यपात्रार्हणानि च
सम्यग्भवन्ति नैतानि विस्तरात्स्वजनार्पणात्‌
देशे काले च सम्प्राप्ते मुन्यन्नं हरिदैवतम्‌
श्रद्धया विधिवत्पात्रे न्यस्तं कामधुगक्षयम्‌
देवर्षिपितृभूतेभ्य आत्मने स्वजनाय च
अन्नं संविभजन्पश्येत्सर्वं तत्पुरुषात्मकम्‌
न दद्यादामिषं श्राद्धे न चाद्याद्धर्मतत्त्ववित्‌
मुन्यन्नैः स्यात्परा प्रीतिर्यथा न पशुहिंसया
नैतादृशः परो धर्मो नृणां सद्धर्ममिच्छताम्‌
न्यासो दण्डस्य भूतेषु मनोवाक्कायजस्य यः
एके कर्ममयान्यज्ञान्ज्ञानिनो यज्ञवित्तमाः
आत्मसंयमनेऽनीहा जुह्वति ज्ञानदीपिते
द्रव्ययज्ञैर्यक्ष्यमाणं दृष्ट्वा भूतानि बिभ्यति
एष माकरुणो हन्यादतज्ज्ञो ह्यसुतृप्ध्रुवम्‌
तस्माद्दैवोपपन्नेन मुन्यन्नेनापि धर्मवित्‌
सन्तुष्टोऽहरहः कुर्यान्नित्यनैमित्तिकीः क्रियाः
विधर्मः परधर्मश्च आभास उपमा छलः
अधर्मशाखाः पञ्चेमा धर्मज्ञोऽधर्मवत्त्यजेत्‌
धर्मबाधो विधर्मः स्यात्परधर्मोऽन्यचोदितः
उपधर्मस्तु पाखण्डो दम्भो वा शब्दभिच्छलः
यस्त्विच्छया कृतः पुम्भिराभासो ह्याश्रमात्पृथक्‌
स्वभावविहितो धर्मः कस्य नेष्टः प्रशान्तये
धर्मार्थमपि नेहेत यात्रार्थं वाधनो धनम्‌
अनीहानीहमानस्य महाहेरिव वृत्तिदा
सन्तुष्टस्य निरीहस्य स्वात्मारामस्य यत्सुखम्‌
कुतस्तत्कामलोभेन धावतोऽर्थेहया दिशः
सदा सन्तुष्टमनसः सर्वाः शिवमया दिशः
शर्कराकण्टकादिभ्यो यथोपानत्पदः शिवम्‌
सन्तुष्टः केन वा राजन्न वर्तेतापि वारिणा
औपस्थ्यजैह्व्यकार्पण्याद्गृहपालायते जनः
असन्तुष्टस्य विप्रस्य तेजो विद्या तपो यशः
स्रवन्तीन्द्रियलौल्येन ज्ञानं चैवावकीर्यते
कामस्यान्तं हि क्षुत्तृड्भ्यां क्रोधस्यैतत्फलोदयात्‌
जनो याति न लोभस्य जित्वा भुक्त्वा दिशो भुवः
पण्डिता बहवो राजन्बहुज्ञाः संशयच्छिदः
सदसस्पतयोऽप्येके असन्तोषात्पतन्त्यधः
असङ्कल्पाज्जयेत्कामं क्रोधं कामविवर्जनात्‌
अर्थानर्थेक्षया लोभं भयं तत्त्वावमर्शनात्‌
आन्वीक्षिक्या शोकमोहौ दम्भं महदुपासया
योगान्तरायान्मौनेन हिंसां कामाद्यनीहया
कृपया भूतजं दुःखं दैवं जह्यात्समाधिना
आत्मजं योगवीर्येण निद्रां सत्त्वनिषेवया
रजस्तमश्च सत्त्वेन सत्त्वं चोपशमेन च
एतत्सर्वं गुरौ भक्त्या पुरुषो ह्यञ्जसा जयेत्‌
यस्य साक्षाद्भगवति ज्ञानदीपप्रदे गुरौ
मर्त्यासद्धीः श्रुतं तस्य सर्वं कुञ्जरशौचवत्‌
एष वै भगवान्साक्षात्प्रधानपुरुषेश्वरः
योगेश्वरैर्विमृग्याङ्घ्रिर्लोको यं मन्यते नरम्‌
षड्वर्गसंयमैकान्ताः सर्वा नियमचोदनाः
तदन्ता यदि नो योगानावहेयुः श्रमावहाः
यथा वार्तादयो ह्यर्था योगस्यार्थं न बिभ्रति
अनर्थाय भवेयुः स्म पूर्तमिष्टं तथासतः
यश्चित्तविजये यत्तः स्यान्निःसङ्गोऽपरिग्रहः
एको विविक्तशरणो भिक्षुर्भैक्ष्यमिताशनः
देशे शुचौ समे राजन्संस्थाप्यासनमात्मनः
स्थिरं सुखं समं तस्मिन्नासीतर्ज्वङ्ग ओमिति
प्राणापानौ सन्निरुन्ध्यात्पूरकुम्भकरेचकैः
यावन्मनस्त्यजेत्कामान्स्वनासाग्रनिरीक्षणः
यतो यतो निःसरति मनः कामहतं भ्रमत्‌
ततस्तत उपाहृत्य हृदि रुन्ध्याच्छनैर्बुधः
एवमभ्यस्यतश्चित्तं कालेनाल्पीयसा यतेः
अनिशं तस्य निर्वाणं यात्यनिन्धनवह्निवत्‌
कामादिभिरनाविद्धं प्रशान्ताखिलवृत्ति यत्‌
चित्तं ब्रह्मसुखस्पृष्टं नैवोत्तिष्ठेत कर्हिचित्‌
यः प्रव्रज्य गृहात्पूर्वं त्रिवर्गावपनात्पुनः
यदि सेवेत तान्भिक्षुः स वै वान्ताश्यपत्रपः
यैः स्वदेहः स्मृतोऽनात्मा मर्त्यो विट्कृमिभस्मवत्‌
त एनमात्मसात्कृत्वा श्लाघयन्ति ह्यसत्तमाः
गृहस्थस्य क्रियात्यागो व्रतत्यागो वटोरपि
तपस्विनो ग्रामसेवा भिक्षोरिन्द्रियलोलता
आश्रमापसदा ह्येते खल्वाश्रमविडम्बनाः
देवमायाविमूढांस्तानुपेक्षेतानुकम्पया
आत्मानं चेद्विजानीयात्परं ज्ञानधुताशयः
किमिच्छन्कस्य वा हेतोर्देहं पुष्णाति लम्पटः
आहुः शरीरं रथमिन्द्रियाणि हयानभीषून्मन इन्द्रियेशम्‌
वर्त्मानि मात्रा धिषणां च सूतं सत्त्वं बृहद्बन्धुरमीशसृष्टम्‌
अक्षं दशप्राणमधर्मधर्मौ चक्रेऽभिमानं रथिनं च जीवम्‌
धनुर्हि तस्य प्रणवं पठन्ति शरं तु जीवं परमेव लक्ष्यम्‌
रागो द्वेषश्च लोभश्च शोकमोहौ भयं मदः
मानोऽवमानोऽसूया च माया हिंसा च मत्सरः
रजः प्रमादः क्षुन्निद्रा शत्रवस्त्वेवमादयः
रजस्तमःप्रकृतयः सत्त्वप्रकृतयः क्वचित्‌
यावन्नृकायरथमात्मवशोपकल्पं
धत्ते गरिष्ठचरणार्चनया निशातम्‌
ज्ञानासिमच्युतबलो दधदस्तशत्रुः
स्वानन्दतुष्ट उपशान्त इदं विजह्यात्‌
नोचेत्प्रमत्तमसदिन्द्रियवाजिसूता
नीत्वोत्पथं विषयदस्युषु निक्षिपन्ति
ते दस्यवः सहयसूतममुं तमोऽन्धे
संसारकूप उरुमृत्युभये क्षिपन्ति
प्रवृत्तं च निवृत्तं च द्विविधं कर्म वैदिकम्‌
आवर्तते प्रवृत्तेन निवृत्तेनाश्नुतेऽमृतम्‌
हिंस्रं द्रव्यमयं काम्यमग्निहोत्राद्यशान्तिदम्‌
दर्शश्च पूर्णमासश्च चातुर्मास्यं पशुः सुतः
एतदिष्टं प्रवृत्ताख्यं हुतं प्रहुतमेव च
पूर्तं सुरालयाराम कूपाजीव्यादिलक्षणम्‌
द्रव्यसूक्ष्मविपाकश्च धूमो रात्रिरपक्षयः
अयनं दक्षिणं सोमो दर्श ओषधिवीरुधः
अन्नं रेत इति क्ष्मेश पितृयानं पुनर्भवः
एकैकश्येनानुपूर्वं भूत्वा भूत्वेह जायते
निषेकादिश्मशानान्तैः संस्कारैः संस्कृतो द्विजः
इन्द्रियेषु क्रियायज्ञान्ज्ञानदीपेषु जुह्वति
इन्द्रियाणि मनस्यूर्मौ वाचि वैकारिकं मनः
वाचं वर्णसमाम्नाये तमोंकारे स्वरे न्यसेत्‌
ओंकारं बिन्दौ नादे तं तं तु प्राणे महत्यमुम्‌
अग्निः सूर्यो दिवा प्राह्णः शुक्लो राकोत्तरं स्वराट्‌
विश्वोऽथ तैजसः प्राज्ञस्तुर्य आत्मा समन्वयात्‌
देवयानमिदं प्राहुर्भूत्वा भूत्वानुपूर्वशः
आत्मयाज्युपशान्तात्मा ह्यात्मस्थो न निवर्तते
य एते पितृदेवानामयने वेदनिर्मिते
शास्त्रेण चक्षुषा वेद जनस्थोऽपि न मुह्यति
आदावन्ते जनानां सद्बहिरन्तः परावरम्‌
ज्ञानं ज्ञेयं वचो वाच्यं तमो ज्योतिस्त्वयं स्वयम्‌
आबाधितोऽपि ह्याभासो यथा वस्तुतया स्मृतः
दुर्घटत्वादैन्द्रियकं तद्वदर्थविकल्पितम्‌
क्षित्यादीनामिहार्थानां छाया न कतमापि हि
न सङ्घातो विकारोऽपि न पृथङ्नान्वितो मृषा
धातवोऽवयवित्वाच्च तन्मात्रावयवैर्विना
न स्युर्ह्यसत्यवयविन्यसन्नवयवोऽन्ततः
स्यात्सादृश्यभ्रमस्तावद्विकल्पे सति वस्तुनः
जाग्रत्स्वापौ यथा स्वप्ने तथा विधिनिषेधता
भावाद्वैतं क्रियाद्वैतं द्रव्याद्वैतं तथात्मनः
वर्तयन्स्वानुभूत्येह त्रीन्स्वप्नान्धुनुते मुनिः
कार्यकारणवस्त्वैक्य दर्शनं पटतन्तुवत्‌
अवस्तुत्वाद्विकल्पस्य भावाद्वैतं तदुच्यते
यद्ब्रह्मणि परे साक्षात्सर्वकर्मसमर्पणम्‌
मनोवाक्तनुभिः पार्थ क्रियाद्वैतं तदुच्यते
आत्मजायासुतादीनामन्येषां सर्वदेहिनाम्‌
यत्स्वार्थकामयोरैक्यं द्रव्याद्वैतं तदुच्यते
यद्यस्य वानिषिद्धं स्याद्येन यत्र यतो नृप
स तेनेहेत कार्याणि नरो नान्यैरनापदि
एतैरन्यैश्च वेदोक्तैर्वर्तमानः स्वकर्मभिः
गृहेऽप्यस्य गतिं यायाद्राजंस्तद्भक्तिभाङ्नरः
यथा हि यूयं नृपदेव दुस्त्यजादापद्गणादुत्तरतात्मनः प्रभोः
यत्पादपङ्केरुहसेवया भवानहारषीन्निर्जितदिग्गजः क्रतून्‌
अहं पुराभवं कश्चिद्गन्धर्व उपबर्हणः
नाम्नातीते महाकल्पे गन्धर्वाणां सुसम्मतः
रूपपेशलमाधुर्य सौगन्ध्यप्रियदर्शनः
स्त्रीणां प्रियतमो नित्यं मत्तः स्वपुरलम्पटः
एकदा देवसत्रे तु गन्धर्वाप्सरसां गणाः
उपहूता विश्वसृग्भिर्हरिगाथोपगायने
अहं च गायंस्तद्विद्वान्स्त्रीभिः परिवृतो गतः
ज्ञात्वा विश्वसृजस्तन्मे हेलनं शेपुरोजसा
याहि त्वं शूद्रतामाशु नष्टश्रीः कृतहेलनः
तावद्दास्यामहं जज्ञे तत्रापि ब्रह्मवादिनाम्‌
शुश्रूषयानुषङ्गेण प्राप्तोऽहं ब्रह्मपुत्रताम्‌
धर्मस्ते गृहमेधीयो वर्णितः पापनाशनः
गृहस्थो येन पदवीमञ्जसा न्यासिनामियात्‌
यूयं नृलोके बत भूरिभागा लोकं पुनाना मुनयोऽभियन्ति
येषां गृहानावसतीति साक्षाद्गूढं परं ब्रह्म मनुष्यलिङ्गम्‌
स वा अयं ब्रह्म महद्विमृग्य कैवल्यनिर्वाणसुखानुभूतिः
प्रियः सुहृद्वः खलु मातुलेय आत्मार्हणीयो विधिकृद्गुरुश्च
न यस्य साक्षाद्भवपद्मजादिभी रूपं धिया वस्तुतयोपवर्णितम्‌
मौनेन भक्त्योपशमेन पूजितः प्रसीदतामेष स सात्वतां पतिः
श्रीशुक उवाच
इति देवर्षिणा प्रोक्तं निशम्य भरतर्षभः
पूजयामास सुप्रीतः कृष्णं च प्रेमविह्वलः
कृष्णपार्थावुपामन्त्र्य पूजितः प्रययौ मुनिः
श्रुत्वा कृष्णं परं ब्रह्म पार्थः परमविस्मितः
इति दाक्षायिणीनां ते पृथग्वंशा प्रकीर्तिताः
देवासुरमनुष्याद्या लोका यत्र चराचराः
श्रीराजोवाच
स्वायम्भुवस्येह गुरो वंशोऽयं विस्तराच्छ्रुतः
यत्र विश्वसृजां सर्गो मनूनन्यान्वदस्व नः
मन्वन्तरे हरेर्जन्म कर्माणि च महीयसः
गृणन्ति कवयो ब्रह्मंस्तानि नो वद शृण्वताम्‌
यद्यस्मिन्नन्तरे ब्रह्मन्भगवान्विश्वभावनः
कृतवान्कुरुते कर्ता ह्यतीतेऽनागतेऽद्य वा
श्रीऋषिरुवाच
मनवोऽस्मिन्व्यतीताः षट्कल्पे स्वायम्भुवादयः
आद्यस्ते कथितो यत्र देवादीनां च सम्भवः
आकूत्यां देवहूत्यां च दुहित्रोस्तस्य वै मनोः
धर्मज्ञानोपदेशार्थं भगवान्पुत्रतां गतः
कृतं पुरा भगवतः कपिलस्यानुवर्णितम्‌
आख्यास्ये भगवान्यज्ञो यच्चकार कुरूद्वह
विरक्तः कामभोगेषु शतरूपापतिः प्रभुः
विसृज्य राज्यं तपसे सभार्यो वनमाविशत्‌
सुनन्दायां वर्षशतं पदैकेन भुवं स्पृशन्‌
तप्यमानस्तपो घोरमिदमन्वाह भारत
श्रीमनुरुवाच
येन चेतयते विश्वं विश्वं चेतयते न यम्‌
यो जागर्ति शयानेऽस्मिन्नायं तं वेद वेद सः
आत्मावास्यमिदं विश्वं यत्किञ्चिज्जगत्यां जगत्‌
तेन त्यक्तेन भुञ्जीथा मा गृधः कस्य स्विद्धनम्‌
यं पश्यति न पश्यन्तं चक्षुर्यस्य न रिष्यति
तं भूतनिलयं देवं सुपर्णमुपधावत
न यस्याद्यन्तौ मध्यं च स्वः परो नान्तरं बहिः
विश्वस्यामूनि यद्यस्माद्विश्वं च तदृतं महत्‌
स विश्वकायः पुरुहूतईशः सत्यः स्वयंज्योतिरजः पुराणः
धत्तेऽस्य जन्माद्यजयात्मशक्त्या तां विद्ययोदस्य निरीह आस्ते
अथाग्रे ऋषयः कर्माणीहन्तेऽकर्महेतवे
ईहमानो हि पुरुषः प्रायोऽनीहां प्रपद्यते
ईहते भगवानीशो न हि तत्र विसज्जते
आत्मलाभेन पूर्णार्थो नावसीदन्ति येऽनु तम्‌
तमीहमानं निरहङ्कृतं बुधं निराशिषं पूर्णमनन्यचोदितम्‌
न्ऱ्‌^ईन्शिक्षयन्तं निजवर्त्मसंस्थितं प्रभुं प्रपद्येऽखिलधर्मभावनम्‌
श्रीशुक उवाच
इति मन्त्रोपनिषदं व्याहरन्तं समाहितम्‌
दृष्ट्वासुरा यातुधाना जग्धुमभ्यद्रवन्क्षुधा
तांस्तथावसितान्वीक्ष्य यज्ञः सर्वगतो हरिः
यामैः परिवृतो देवैर्हत्वाशासत्त्रिविष्टपम्‌
स्वारोचिषो द्वितीयस्तु मनुरग्नेः सुतोऽभवत्‌
द्युमत्सुषेणरोचिष्मत्प्रमुखास्तस्य चात्मजाः
तत्रेन्द्रो रोचनस्त्वासीद्देवाश्च तुषितादयः
ऊर्जस्तम्भादयः सप्त ऋषयो ब्रह्मवादिनः
ऋषेस्तु वेदशिरसस्तुषिता नाम पत्न्यभूत्‌
तस्यां जज्ञे ततो देवो विभुरित्यभिविश्रुतः
अष्टाशीतिसहस्राणि मुनयो ये धृतव्रताः
अन्वशिक्षन्व्रतं तस्य कौमारब्रह्मचारिणः
तृतीय उत्तमो नाम प्रियव्रतसुतो मनुः
पवनः सृञ्जयो यज्ञ होत्राद्यास्तत्सुता नृप
वसिष्ठतनयाः सप्त ऋषयः प्रमदादयः
सत्या वेदश्रुता भद्रा देवा इन्द्रस्तु सत्यजित्‌
धर्मस्य सूनृतायां तु भगवान्पुरुषोत्तमः
सत्यसेन इति ख्यातो जातः सत्यव्रतैः सह
सोऽनृतव्रतदुःशीलानसतो यक्षराक्षसान्‌
भूतद्रुहो भूतगणांश्चावधीत्सत्यजित्सखः
चतुर्थ उत्तमभ्राता मनुर्नाम्ना च तामसः
पृथुः ख्यातिर्नरः केतुरित्याद्या दश तत्सुताः
सत्यका हरयो वीरा देवास्त्रिशिख ईश्वरः
ज्योतिर्धामादयः सप्त ऋषयस्तामसेऽन्तरे
देवा वैधृतयो नाम विधृतेस्तनया नृप
नष्टाः कालेन यैर्वेदा विधृताः स्वेन तेजसा
तत्रापि जज्ञे भगवान्हरिण्यां हरिमेधसः
हरिरित्याहृतो येन गजेन्द्रो मोचितो ग्रहात्‌
श्रीराजोवाच
बादरायण एतत्ते श्रोतुमिच्छामहे वयम्‌
हरिर्यथा गजपतिं ग्राहग्रस्तममूमुचत्‌
तत्कथासु महत्पुण्यं धन्यं स्वस्त्ययनं शुभम्‌
यत्र यत्रोत्तमश्लोको भगवान्गीयते हरिः
श्रीसूत उवाच
परीक्षितैवं स तु बादरायणिः प्रायोपविष्टेन कथासु चोदितः
उवाच विप्राः प्रतिनन्द्य पार्थिवं मुदा मुनीनां सदसि स्म शृण्वताम्‌
श्रीशुक उवाच
आसीद्गिरिवरो राजंस्त्रिकूट इति विश्रुतः
क्षीरोदेनावृतः श्रीमान्योजनायुतमुच्छ्रितः
तावता विस्तृतः पर्यक्त्रिभिः शृङ्गैः पयोनिधिम्‌
दिशः खं रोचयन्नास्ते रौप्यायसहिरण्मयैः
अन्यैश्च ककुभः सर्वा रत्नधातुविचित्रितैः
नानाद्रुमलतागुल्मैर्निर्घोषैर्निर्झराम्भसाम्‌
स चावनिज्यमानाङ्घ्रिः समन्तात्पयऊर्मिभिः
करोति श्यामलां भूमिं हरिन्मरकताश्मभिः
सिद्धचारणगन्धर्वैर्विद्याधरमहोरगैः
किन्नरैरप्सरोभिश्च क्रीडद्भिर्जुष्टकन्दरः
यत्र सङ्गीतसन्नादैर्नदद्गुहममर्षया
अभिगर्जन्ति हरयः श्लाघिनः परशङ्कया
नानारण्यपशुव्रात सङ्कुलद्रोण्यलङ्कृतः
चित्रद्रुमसुरोद्यान कलकण्ठविहङ्गमः
सरित्सरोभिरच्छोदैः पुलिनैर्मणिवालुकैः
देवस्त्रीमज्जनामोद सौरभाम्ब्वनिलैर्युतः
तस्य द्रोण्यां भगवतो वरुणस्य महात्मनः
उद्यानमृतुमन्नाम आक्रीडं सुरयोषिताम्‌
सर्वतोऽलङ्कृतं दिव्यैर्नित्यपुष्पफलद्रुमैः
मन्दारैः पारिजातैश्च पाटलाशोकचम्पकैः
चूतैः पियालैः पनसैराम्रैराम्रातकैरपि
क्रमुकैर्नारिकेलैश्च खर्जूरैर्बीजपूरकैः
मधुकैः शालतालैश्च तमालैरसनार्जुनैः
अरिष्टोडुम्बरप्लक्षैर्वटैः किंशुकचन्दनैः
पिचुमर्दैः कोविदारैः सरलैः सुरदारुभिः
द्राक्षेक्षुरम्भाजम्बुभिर्बदर्यक्षाभयामलैः
बिल्वैः कपित्थैर्जम्बीरैर्वृतो भल्लातकादिभिः
तस्मिन्सरः सुविपुलं लसत्काञ्चनपङ्कजम्‌
कुमुदोत्पलकह्लार शतपत्रश्रियोर्जितम्‌
मत्तषट्पदनिर्घुष्टं शकुन्तैश्च कलस्वनैः
हंसकारण्डवाकीर्णं चक्राह्वैः सारसैरपि
जलकुक्कुटकोयष्टि दात्यूहकुलकूजितम्‌
मत्स्यकच्छपसञ्चार चलत्पद्मरजःपयः
कदम्बवेतसनल नीपवञ्जुलकैर्वृतम्‌
कुन्दैः कुरुबकाशोकैः शिरीषैः कूटजेङ्गुदैः
कुब्जकैः स्वर्णयूथीभिर्नागपुन्नागजातिभिः
मल्लिकाशतपत्रैश्च माधवीजालकादिभिः
शोभितं तीरजैश्चान्यैर्नित्यर्तुभिरलं द्रुमैः
तत्रैकदा तद्गिरिकाननाश्रयः करेणुभिर्वारणयूथपश्चरन्‌
सकण्टकं कीचकवेणुवेत्रवद्विशालगुल्मं प्ररुजन्वनस्पतीन्‌
यद्गन्धमात्राद्धरयो गजेन्द्रा व्याघ्रादयो व्यालमृगाः सखड्गाः
महोरगाश्चापि भयाद्द्रवन्ति सगौरकृष्णाः सरभाश्चमर्यः
वृका वराहा महिषर्क्षशल्या गोपुच्छशालावृकमर्कटाश्च
अन्यत्र क्षुद्रा हरिणाः शशादयश्चरन्त्यभीता यदनुग्रहेण
स घर्मतप्तः करिभिः करेणुभिर्वृतो मदच्युत्करभैरनुद्रुतः
गिरिं गरिम्णा परितः प्रकम्पयन्निषेव्यमाणोऽलिकुलैर्मदाशनैः
सरोऽनिलं पङ्कजरेणुरूषितं जिघ्रन्विदूरान्मदविह्वलेक्षणः
वृतः स्वयूथेन तृषार्दितेन तत्सरोवराभ्यासमथागमद्द्रुतम्‌
विगाह्य तस्मिन्नमृताम्बु निर्मलं हेमारविन्दोत्पलरेणुरूषितम्‌
पपौ निकामं निजपुष्करोद्धृतमात्मानमद्भिः स्नपयन्गतक्लमः
स पुष्करेणोद्धृतशीकराम्बुभिर्निपाययन्संस्नपयन्यथा गृही
घृणी करेणुः करभांश्च दुर्मदो नाचष्ट कृच्छ्रं कृपणोऽजमायया
तं तत्र कश्चिन्नृप दैवचोदितो ग्राहो बलीयांश्चरणे रुषाग्रहीत्‌
यदृच्छयैवं व्यसनं गतो गजो यथाबलं सोऽतिबलो विचक्रमे
तथातुरं यूथपतिं करेणवो विकृष्यमाणं तरसा बलीयसा
विचुक्रुशुर्दीनधियोऽपरे गजाः पार्ष्णिग्रहास्तारयितुं न चाशकन्‌
नियुध्यतोरेवमिभेन्द्रनक्रयोर्विकर्षतोरन्तरतो बहिर्मिथः
समाः सहस्रं व्यगमन्महीपते सप्राणयोश्चित्रममंसतामराः
ततो गजेन्द्रस्य मनोबलौजसां कालेन दीर्घेण महानभूद्व्ययः
विकृष्यमाणस्य जलेऽवसीदतो विपर्ययोऽभूत्सकलं जलौकसः
इत्थं गजेन्द्रः स यदाप सङ्कटं प्राणस्य देही विवशो यदृच्छया
अपारयन्नात्मविमोक्षणे चिरं दध्याविमां बुद्धिमथाभ्यपद्यत
न मामिमे ज्ञातय आतुरं गजाः कुतः करिण्यः प्रभवन्ति मोचितुम्‌
ग्राहेण पाशेन विधातुरावृतोऽप्यहं च तं यामि परं परायणम्‌
यः कश्चनेशो बलिनोऽन्तकोरगात्प्रचण्डवेगादभिधावतो भृशम्‌
भीतं प्रपन्नं परिपाति यद्भयान्मृत्युः प्रधावत्यरणं तमीमहि
श्रीबादरायणिरुवाच
एवं व्यवसितो बुद्ध्या समाधाय मनो हृदि
जजाप परमं जाप्यं प्राग्जन्मन्यनुशिक्षितम्‌
श्रीगजेन्द्र उवाच
ओं नमो भगवते तस्मै यत एतच्चिदात्मकम्‌
पुरुषायादिबीजाय परेशायाभिधीमहि
यस्मिन्निदं यतश्चेदं येनेदं य इदं स्वयम्‌
योऽस्मात्परस्माच्च परस्तं प्रपद्ये स्वयम्भुवम्‌
यः स्वात्मनीदं निजमाययार्पितं क्वचिद्विभातं क्व च तत्तिरोहितम्‌
अविद्धदृक्साक्ष्युभयं तदीक्षते स आत्ममूलोऽवतु मां परात्परः
कालेन पञ्चत्वमितेषु कृत्स्नशो लोकेषु पालेषु च सर्वहेतुषु
तमस्तदासीद्गहनं गभीरं यस्तस्य पारेऽभिविराजते विभुः
न यस्य देवा ऋषयः पदं विदुर्जन्तुः पुनः कोऽर्हति गन्तुमीरितुम्‌
यथा नटस्याकृतिभिर्विचेष्टतो दुरत्ययानुक्रमणः स मावतु
दिदृक्षवो यस्य पदं सुमङ्गलं विमुक्तसङ्गा मुनयः सुसाधवः
चरन्त्यलोकव्रतमव्रणं वने भूतात्मभूताः सुहृदः स मे गतिः
न विद्यते यस्य च जन्म कर्म वा न नामरूपे गुणदोष एव वा
तथापि लोकाप्ययसम्भवाय यः स्वमायया तान्यनुकालमृच्छति
तस्मै नमः परेशाय ब्रह्मणेऽनन्तशक्तये
अरूपायोरुरूपाय नम आश्चर्यकर्मणे
नम आत्मप्रदीपाय साक्षिणे परमात्मने
नमो गिरां विदूराय मनसश्चेतसामपि
सत्त्वेन प्रतिलभ्याय नैष्कर्म्येण विपश्चिता
नमः कैवल्यनाथाय निर्वाणसुखसंविदे
नमः शान्ताय घोराय मूढाय गुणधर्मिणे
निर्विशेषाय साम्याय नमो ज्ञानघनाय च
क्षेत्रज्ञाय नमस्तुभ्यं सर्वाध्यक्षाय साक्षिणे
पुरुषायात्ममूलाय मूलप्रकृतये नमः
सर्वेन्द्रियगुणद्रष्ट्रे सर्वप्रत्ययहेतवे
असता च्छाययोक्ताय सदाभासाय ते नमः
नमो नमस्तेऽखिलकारणाय निष्कारणायाद्भुतकारणाय
सर्वागमाम्नायमहार्णवाय नमोऽपवर्गाय परायणाय
गुणारणिच्छन्नचिदुष्मपाय तत्क्षोभविस्फूर्जितमानसाय
नैष्कर्म्यभावेन विवर्जितागम स्वयंप्रकाशाय नमस्करोमि
मादृक्प्रपन्नपशुपाशविमोक्षणाय मुक्ताय भूरिकरुणाय नमोऽलयाय
स्वांशेन सर्वतनुभृन्मनसि प्रतीत प्रत्यग्दृशे भगवते बृहते नमस्ते
आत्मात्मजाप्तगृहवित्तजनेषु सक्तैर्दुष्प्रापणाय गुणसङ्गविवर्जिताय
मुक्तात्मभिः स्वहृदये परिभाविताय ज्ञानात्मने भगवते नम ईश्वराय
यं धर्मकामार्थविमुक्तिकामा भजन्त इष्टां गतिमाप्नुवन्ति
किं चाशिषो रात्यपि देहमव्ययं करोतु मेऽदभ्रदयो विमोक्षणम्‌
एकान्तिनो यस्य न कञ्चनार्थं वाञ्छन्ति ये वै भगवत्प्रपन्नाः
अत्यद्भुतं तच्चरितं सुमङ्गलं गायन्त आनन्दसमुद्रमग्नाः
तमक्षरं ब्रह्म परं परेशमव्यक्तमाध्यात्मिकयोगगम्यम्‌
अतीन्द्रियं सूक्ष्ममिवातिदूरमनन्तमाद्यं परिपूर्णमीडे
यस्य ब्रह्मादयो देवा वेदा लोकाश्चराचराः
नामरूपविभेदेन फल्ग्व्या च कलया कृताः
यथार्चिषोऽग्नेः सवितुर्गभस्तयो निर्यान्ति संयान्त्यसकृत्स्वरोचिषः
तथा यतोऽयं गुणसम्प्रवाहो बुद्धिर्मनः खानि शरीरसर्गाः
स वै न देवासुरमर्त्यतिर्यङ्न स्त्री न षण्ढो न पुमान्न जन्तुः
नायं गुणः कर्म न सन्न चासन्निषेधशेषो जयतादशेषः
जिजीविषे नाहमिहामुया किमन्तर्बहिश्चावृतयेभयोन्या
इच्छामि कालेन न यस्य विप्लवस्तस्यात्मलोकावरणस्य मोक्षम्‌
सोऽहं विश्वसृजं विश्वमविश्वं विश्ववेदसम्‌
विश्वात्मानमजं ब्रह्म प्रणतोऽस्मि परं पदम्‌
योगरन्धितकर्माणो हृदि योगविभाविते
योगिनो यं प्रपश्यन्ति योगेशं तं नतोऽस्म्यहम्‌
नमो नमस्तुभ्यमसह्यवेग शक्तित्रयायाखिलधीगुणाय
प्रपन्नपालाय दुरन्तशक्तये कदिन्द्रियाणामनवाप्यवर्त्मने
नायं वेद स्वमात्मानं यच्छक्त्याहंधिया हतम्‌
तं दुरत्ययमाहात्म्यं भगवन्तमितोऽस्म्यहम्‌
श्रीशुक उवाच
एवं गजेन्द्रमुपवर्णितनिर्विशेषं
ब्रह्मादयो विविधलिङ्गभिदाभिमानाः
नैते यदोपससृपुर्निखिलात्मकत्वात्‌
तत्राखिलामरमयो हरिराविरासीत्‌
तं तद्वदार्तमुपलभ्य जगन्निवासः
स्तोत्रं निशम्य दिविजैः सह संस्तुवद्भिः
छन्दोमयेन गरुडेन समुह्यमानश्‌
चक्रायुधोऽभ्यगमदाशु यतो गजेन्द्रः
सोऽन्तःसरस्युरुबलेन गृहीत आर्तो
दृष्ट्वा गरुत्मति हरिं ख उपात्तचक्रम्‌
उत्क्षिप्य साम्बुजकरं गिरमाह कृच्छ्रान्‌
नारायणाखिलगुरो भगवन्नमस्ते
तं वीक्ष्य पीडितमजः सहसावतीर्य
सग्राहमाशु सरसः कृपयोज्जहार
ग्राहाद्विपाटितमुखादरिणा गजेन्द्रं
सम्पश्यतां हरिरमूमुचदुच्छ्रियाणाम्‌
श्रीशुक उवाच
तदा देवर्षिगन्धर्वा ब्रह्मेशानपुरोगमाः
मुमुचुः कुसुमासारं शंसन्तः कर्म तद्धरेः
नेदुर्दुन्दुभयो दिव्या गन्धर्वा ननृतुर्जगुः
ऋषयश्चारणाः सिद्धास्तुष्टुवुः पुरुषोत्तमम्‌
योऽसौ ग्राहः स वै सद्यः परमाश्चर्यरूपधृक्‌
मुक्तो देवलशापेन हूहूर्गन्धर्वसत्तमः
प्रणम्य शिरसाधीशमुत्तमश्लोकमव्ययम्‌
अगायत यशोधाम कीर्तन्यगुणसत्कथम्‌
सोऽनुकम्पित ईशेन परिक्रम्य प्रणम्य तम्‌
लोकस्य पश्यतो लोकं स्वमगान्मुक्तकिल्बिषः
गजेन्द्रो भगवत्स्पर्शाद्विमुक्तोऽज्ञानबन्धनात्‌
प्राप्तो भगवतो रूपं पीतवासाश्चतुर्भुजः
स वै पूर्वमभूद्राजा पाण्ड्यो द्रविडसत्तमः
इन्द्रद्युम्न इति ख्यातो विष्णुव्रतपरायणः
स एकदाराधनकाल आत्मवान्गृहीतमौनव्रत ईश्वरं हरिम्‌
जटाधरस्तापस आप्लुतोऽच्युतं समर्चयामास कुलाचलाश्रमः
यदृच्छया तत्र महायशा मुनिः समागमच्छिष्यगणैः परिश्रितः
तं वीक्ष्य तूष्णीमकृतार्हणादिकं रहस्युपासीनमृषिश्चुकोप ह
तस्मा इमं शापमदादसाधुरयं दुरात्माकृतबुद्धिरद्य
विप्रावमन्ता विशतां तमिस्रं यथा गजः स्तब्धमतिः स एव
श्रीशुक उवाच
एवं शप्त्वा गतोऽगस्त्यो भगवान्नृप सानुगः
इन्द्रद्युम्नोऽपि राजर्षिर्दिष्टं तदुपधारयन्‌
आपन्नः कौञ्जरीं योनिमात्मस्मृतिविनाशिनीम्‌
हर्यर्चनानुभावेन यद्गजत्वेऽप्यनुस्मृतिः
एवं विमोक्ष्य गजयूथपमब्जनाभस्‌
तेनापि पार्षदगतिं गमितेन युक्तः
गन्धर्वसिद्धविबुधैरुपगीयमान
कर्माद्भुतं स्वभवनं गरुडासनोऽगात्‌
एतन्महाराज तवेरितो मया कृष्णानुभावो गजराजमोक्षणम्‌
स्वर्ग्यं यशस्यं कलिकल्मषापहं दुःस्वप्ननाशं कुरुवर्य शृण्वताम्‌
यथानुकीर्तयन्त्येतच्छ्रेयस्कामा द्विजातयः
शुचयः प्रातरुत्थाय दुःस्वप्नाद्युपशान्तये
इदमाह हरिः प्रीतो गजेन्द्रं कुरुसत्तम
शृण्वतां सर्वभूतानां सर्वभूतमयो विभुः
श्रीभगवानुवाच
ये मां त्वां च सरश्चेदं गिरिकन्दरकाननम्‌
वेत्रकीचकवेणूनां गुल्मानि सुरपादपान्‌
शृङ्गाणीमानि धिष्ण्यानि ब्रह्मणो मे शिवस्य च
क्षीरोदं मे प्रियं धाम श्वेतद्वीपं च भास्वरम्‌
श्रीवत्सं कौस्तुभं मालां गदां कौमोदकीं मम
सुदर्शनं पाञ्चजन्यं सुपर्णं पतगेश्वरम्‌
शेषं च मत्कलां सूक्ष्मां श्रियं देवीं मदाश्रयाम्‌
ब्रह्माणं नारदमृषिं भवं प्रह्रादमेव च
मत्स्यकूर्मवराहाद्यैरवतारैः कृतानि मे
कर्माण्यनन्तपुण्यानि सूर्यं सोमं हुताशनम्‌
प्रणवं सत्यमव्यक्तं गोविप्रान्धर्ममव्ययम्‌
दाक्षायणीर्धर्मपत्नीः सोमकश्यपयोरपि
गङ्गां सरस्वतीं नन्दां कालिन्दीं सितवारणम्‌
ध्रुवं ब्रह्मऋषीन्सप्त पुण्यश्लोकांश्च मानवान्‌
उत्थायापररात्रान्ते प्रयताः सुसमाहिताः
स्मरन्ति मम रूपाणि मुच्यन्ते तेऽंहसोऽखिलात्‌
ये मां स्तुवन्त्यनेनाङ्ग प्रतिबुध्य निशात्यये
तेषां प्राणात्यये चाहं ददामि विपुलां गतिम्‌
श्रीशुक उवाच
इत्यादिश्य हृषीकेशः प्राध्माय जलजोत्तमम्‌
हर्षयन्विबुधानीकमारुरोह खगाधिपम्‌
श्रीशुक उवाच
राजन्नुदितमेतत्ते हरेः कर्माघनाशनम्‌
गजेन्द्रमोक्षणं पुण्यं रैवतं त्वन्तरं शृणु
पञ्चमो रैवतो नाम मनुस्तामससोदरः
बलिविन्ध्यादयस्तस्य सुता हार्जुनपूर्वकाः
विभुरिन्द्रः सुरगणा राजन्भूतरयादयः
हिरण्यरोमा वेदशिरा ऊर्ध्वबाह्वादयो द्विजाः
पत्नी विकुण्ठा शुभ्रस्य वैकुण्ठैः सुरसत्तमैः
तयोः स्वकलया जज्ञे वैकुण्ठो भगवान्स्वयम्‌
वैकुण्ठः कल्पितो येन लोको लोकनमस्कृतः
रमया प्रार्थ्यमानेन देव्या तत्प्रियकाम्यया
तस्यानुभावः कथितो गुणाश्च परमोदयाः
भौमान्रेणून्स विममे यो विष्णोर्वर्णयेद्गुणान्‌
षष्ठश्च चक्षुषः पुत्रश्चाक्षुषो नाम वै मनुः
पूरुपूरुषसुद्युम्न प्रमुखाश्चाक्षुषात्मजाः
इन्द्रो मन्त्रद्रुमस्तत्र देवा आप्यादयो गणाः
मुनयस्तत्र वै राजन्हविष्मद्वीरकादयः
तत्रापि देवसम्भूत्यां वैराजस्याभवत्सुतः
अजितो नाम भगवानंशेन जगतः पतिः
पयोधिं येन निर्मथ्य सुराणां साधिता सुधा
भ्रममाणोऽम्भसि धृतः कूर्मरूपेण मन्दरः
श्रीराजोवाच
यथा भगवता ब्रह्मन्मथितः क्षीरसागरः
यदर्थं वा यतश्चाद्रिं दधाराम्बुचरात्मना
यथामृतं सुरैः प्राप्तं किं चान्यदभवत्ततः
एतद्भगवतः कर्म वदस्व परमाद्भुतम्‌
त्वया सङ्कथ्यमानेन महिम्ना सात्वतां पतेः
नातितृप्यति मे चित्तं सुचिरं तापतापितम्‌
श्रीसूत उवाच
सम्पृष्टो भगवानेवं द्वैपायनसुतो द्विजाः
अभिनन्द्य हरेर्वीर्यमभ्याचष्टुं प्रचक्रमे
श्रीशुक उवाच
यदा युद्धेऽसुरैर्देवा बध्यमानाः शितायुधैः
गतासवो निपतिता नोत्तिष्ठेरन्स्म भूरिशः
यदा दुर्वासः शापेन सेन्द्रा लोकास्त्रयो नृप
निःश्रीकाश्चाभवंस्तत्र नेशुरिज्यादयः क्रियाः
निशाम्यैतत्सुरगणा महेन्द्रवरुणादयः
नाध्यगच्छन्स्वयं मन्त्रैर्मन्त्रयन्तो विनिश्चितम्‌
ततो ब्रह्मसभां जग्मुर्मेरोर्मूर्धनि सर्वशः
सर्वं विज्ञापयां चक्रुः प्रणताः परमेष्ठिने
स विलोक्येन्द्रवाय्वादीन्निःसत्त्वान्विगतप्रभान्‌
लोकानमङ्गलप्रायानसुरानयथा विभुः
समाहितेन मनसा संस्मरन्पुरुषं परम्‌
उवाचोत्फुल्लवदनो देवान्स भगवान्परः
अहं भवो यूयमथोऽसुरादयो मनुष्यतिर्यग्द्रुमघर्मजातयः
यस्यावतारांशकलाविसर्जिता व्रजाम सर्वे शरणं तमव्ययम्‌
न यस्य वध्यो न च रक्षणीयो नोपेक्षणीयादरणीयपक्षः
तथापि सर्गस्थितिसंयमार्थं धत्ते रजःसत्त्वतमांसि काले
अयं च तस्य स्थितिपालनक्षणः सत्त्वं जुषाणस्य भवाय देहिनाम्‌
तस्माद्व्रजामः शरणं जगद्गुरुं स्वानां स नो धास्यति शं सुरप्रियः
श्रीशुक उवाच
इत्याभाष्य सुरान्वेधाः सह देवैररिन्दम
अजितस्य पदं साक्षाज्जगाम तमसः परम्‌
तत्रादृष्टस्वरूपाय श्रुतपूर्वाय वै प्रभुः
स्तुतिमब्रूत दैवीभिर्गीर्भिस्त्ववहितेन्द्रियः
श्रीब्रह्मोवाच
अविक्रियं सत्यमनन्तमाद्यं गुहाशयं निष्कलमप्रतर्क्यम्‌
मनोऽग्रयानं वचसानिरुक्तं नमामहे देववरं वरेण्यम्‌
विपश्चितं प्राणमनोधियात्मनामर्थेन्द्रियाभासमनिद्रमव्रणम्‌
छायातपौ यत्र न गृध्रपक्षौ तमक्षरं खं त्रियुगं व्रजामहे
अजस्य चक्रं त्वजयेर्यमाणं मनोमयं पञ्चदशारमाशु
त्रिनाभि विद्युच्चलमष्टनेमि यदक्षमाहुस्तमृतं प्रपद्ये
य एकवर्णं तमसः परं तदलोकमव्यक्तमनन्तपारम्‌
आसां चकारोपसुपर्णमेनमुपासते योगरथेन धीराः
न यस्य कश्चातितितर्ति मायां यया जनो मुह्यति वेद नार्थम्‌
तं निर्जितात्मात्मगुणं परेशं नमाम भूतेषु समं चरन्तम्‌
इमे वयं यत्प्रिययैव तन्वा सत्त्वेन सृष्टा बहिरन्तराविः
गतिं न सूक्ष्मामृषयश्च विद्महे कुतोऽसुराद्या इतरप्रधानाः
पादौ महीयं स्वकृतैव यस्य चतुर्विधो यत्र हि भूतसर्गः
स वै महापूरुष आत्मतन्त्रः प्रसीदतां ब्रह्म महाविभूतिः
अम्भस्तु यद्रेत उदारवीर्यं सिध्यन्ति जीवन्त्युत वर्धमानाः
लोका यतोऽथाखिललोकपालाः प्रसीदतां नः स महाविभूतिः
सोमं मनो यस्य समामनन्ति दिवौकसां यो बलमन्ध आयुः
ईशो नगानां प्रजनः प्रजानां प्रसीदतां नः स महाविभूतिः
अग्निर्मुखं यस्य तु जातवेदा जातः क्रियाकाण्डनिमित्तजन्मा
अन्तःसमुद्रेऽनुपचन्स्वधातून्प्रसीदतां नः स महाविभूतिः
यच्चक्षुरासीत्तरणिर्देवयानं त्रयीमयो ब्रह्मण एष धिष्ण्यम्‌
द्वारं च मुक्तेरमृतं च मृत्युः प्रसीदतां नः स महाविभूतिः
प्राणादभूद्यस्य चराचराणां प्राणः सहो बलमोजश्च वायुः
अन्वास्म सम्राजमिवानुगा वयं प्रसीदतां नः स महाविभूतिः
श्रोत्राद्दिशो यस्य हृदश्च खानि प्रजज्ञिरे खं पुरुषस्य नाभ्याः
प्राणेन्द्रियात्मासुशरीरकेतः प्रसीदतां नः स महाविभूतिः
बलान्महेन्द्रस्त्रिदशाः प्रसादान्मन्योर्गिरीशो धिषणाद्विरिञ्चः
खेभ्यस्तु छन्दांस्यृषयो मेढ्रतः कः प्रसीदतां नः स महाविभूतिः
श्रीर्वक्षसः पितरश्छाययासन्धर्मः स्तनादितरः पृष्ठतोऽभूत्‌
द्यौर्यस्य शीर्ष्णोऽप्सरसो विहारात्प्रसीदतां नः स महाविभूतिः
विप्रो मुखाद्ब्रह्म च यस्य गुह्यं राजन्य आसीद्भुजयोर्बलं च
ऊर्वोर्विडोजोऽङ्घ्रिरवेदशूद्रौ प्रसीदतां नः स महाविभूतिः
लोभोऽधरात्प्रीतिरुपर्यभूद्द्युतिर्नस्तः पशव्यः स्पर्शेन कामः
भ्रुवोर्यमः पक्ष्मभवस्तु कालः प्रसीदतां नः स महाविभूतिः
द्रव्यं वयः कर्म गुणान्विशेषं यद्योगमायाविहितान्वदन्ति
यद्दुर्विभाव्यं प्रबुधापबाधं प्रसीदतां नः स महाविभूतिः
नमोऽस्तु तस्मा उपशान्तशक्तये स्वाराज्यलाभप्रतिपूरितात्मने
गुणेषु मायारचितेषु वृत्तिभिर्न सज्जमानाय नभस्वदूतये
स त्वं नो दर्शयात्मानमस्मत्करणगोचरम्‌
प्रपन्नानां दिदृक्षूणां सस्मितं ते मुखाम्बुजम्‌
तैस्तैः स्वेच्छाभूतै रूपैः काले काले स्वयं विभो
कर्म दुर्विषहं यन्नो भगवांस्तत्करोति हि
क्लेशभूर्यल्पसाराणि कर्माणि विफलानि वा
देहिनां विषयार्तानां न तथैवार्पितं त्वयि
नावमः कर्मकल्पोऽपि विफलायेश्वरार्पितः
कल्पते पुरुषस्यैव स ह्यात्मा दयितो हितः
यथा हि स्कन्धशाखानां तरोर्मूलावसेचनम्‌
एवमाराधनं विष्णोः सर्वेषामात्मनश्च हि
नमस्तुभ्यमनन्ताय दुर्वितर्क्यात्मकर्मणे
निर्गुणाय गुणेशाय सत्त्वस्थाय च साम्प्रतम्‌
श्रीशुक उवाच
एवं स्तुतः सुरगणैर्भगवान्हरिरीश्वरः
तेषामाविरभूद्राजन्सहस्रार्कोदयद्युतिः
तेनैव सहसा सर्वे देवाः प्रतिहतेक्षणाः
नापश्यन्खं दिशः क्षौणीमात्मानं च कुतो विभुम्‌
विरिञ्चो भगवान्दृष्ट्वा सह शर्वेण तां तनुम्‌
स्वच्छां मरकतश्यामां कञ्जगर्भारुणेक्षणाम्‌
तप्तहेमावदातेन लसत्कौशेयवाससा
प्रसन्नचारुसर्वाङ्गीं सुमुखीं सुन्दरभ्रुवम्‌
महामणिकिरीटेन केयूराभ्यां च भूषिताम्‌
कर्णाभरणनिर्भात कपोलश्रीमुखाम्बुजाम्‌
काञ्चीकलापवलय हारनूपुरशोभिताम्‌
कौस्तुभाभरणां लक्ष्मीं बिभ्रतीं वनमालिनीम्‌
सुदर्शनादिभिः स्वास्त्रैर्मूर्तिमद्भिरुपासिताम्‌
तुष्टाव देवप्रवरः सशर्वः पुरुषं परम्‌
सर्वामरगणैः साकं सर्वाङ्गैरवनिं गतैः
श्रीब्रह्मोवाच
अजातजन्मस्थितिसंयमाया गुणाय निर्वाणसुखार्णवाय
अणोरणिम्नेऽपरिगण्यधाम्ने महानुभावाय नमो नमस्ते
रूपं तवैतत्पुरुषर्षभेज्यं श्रेयोऽर्थिभिर्वैदिकतान्त्रिकेण
योगेन धातः सह नस्त्रिलोकान्पश्याम्यमुष्मिन्नु ह विश्वमूर्तौ
त्वय्यग्र आसीत्त्वयि मध्य आसीत्त्वय्यन्त आसीदिदमात्मतन्त्रे
त्वमादिरन्तो जगतोऽस्य मध्यं घटस्य मृत्स्नेव परः परस्मात्‌
त्वं माययात्माश्रयया स्वयेदं निर्माय विश्वं तदनुप्रविष्टः
पश्यन्ति युक्ता मनसा मनीषिणो गुणव्यवायेऽप्यगुणं विपश्चितः
यथाग्निमेधस्यमृतं च गोषु भुव्यन्नमम्बूद्यमने च वृत्तिम्‌
योगैर्मनुष्या अधियन्ति हि त्वां गुणेषु बुद्ध्या कवयो वदन्ति
तं त्वां वयं नाथ समुज्जिहानं सरोजनाभातिचिरेप्सितार्थम्‌
दृष्ट्वा गता निर्वृतमद्य सर्वे गजा दवार्ता इव गाङ्गमम्भः
स त्वं विधत्स्वाखिललोकपाला वयं यदर्थास्तव पादमूलम्‌
समागतास्ते बहिरन्तरात्मन्किं वान्यविज्ञाप्यमशेषसाक्षिणः
अहं गिरित्रश्च सुरादयो ये दक्षादयोऽग्नेरिव केतवस्ते
किं वा विदामेश पृथग्विभाता विधत्स्व शं नो द्विजदेवमन्त्रम्‌
श्रीशुक उवाच
एवं विरिञ्चादिभिरीडितस्तद्विज्ञाय तेषां हृदयं यथैव
जगाद जीमूतगभीरया गिरा बद्धाञ्जलीन्संवृतसर्वकारकान्‌
एक एवेश्वरस्तस्मिन्सुरकार्ये सुरेश्वरः
विहर्तुकामस्तानाह समुद्रोन्मथनादिभिः
श्रीभगवानुवाच
हन्त ब्रह्मन्नहो शम्भो हे देवा मम भाषितम्‌
शृणुतावहिताः सर्वे श्रेयो वः स्याद्यथा सुराः
यात दानवदैतेयैस्तावत्सन्धिर्विधीयताम्‌
कालेनानुगृहीतैस्तैर्यावद्वो भव आत्मनः
अरयोऽपि हि सन्धेयाः सति कार्यार्थगौरवे
अहिमूषिकवद्देवा ह्यर्थस्य पदवीं गतैः
अमृतोत्पादने यत्नः क्रियतामविलम्बितम्‌
यस्य पीतस्य वै जन्तुर्मृत्युग्रस्तोऽमरो भवेत्‌
क्षिप्त्वा क्षीरोदधौ सर्वा वीरुत्तृणलतौषधीः
मन्थानं मन्दरं कृत्वा नेत्रं कृत्वा तु वासुकिम्‌
सहायेन मया देवा निर्मन्थध्वमतन्द्रिताः
क्लेशभाजो भविष्यन्ति दैत्या यूयं फलग्रहाः
यूयं तदनुमोदध्वं यदिच्छन्त्यसुराः सुराः
न संरम्भेण सिध्यन्ति सर्वार्थाः सान्त्वया यथा
न भेतव्यं कालकूटाद्विषाज्जलधिसम्भवात्‌
लोभः कार्यो न वो जातु रोषः कामस्तु वस्तुषु
श्रीशुक उवाच
इति देवान्समादिश्य भगवान्पुरुषोत्तमः
तेषामन्तर्दधे राजन्स्वच्छन्दगतिरीश्वरः
अथ तस्मै भगवते नमस्कृत्य पितामहः
भवश्च जग्मतुः स्वं स्वं धामोपेयुर्बलिं सुराः
दृष्ट्वारीनप्यसंयत्तान्जातक्षोभान्स्वनायकान्‌
न्यषेधद्दैत्यराट्श्लोक्यः सन्धिविग्रहकालवित्‌
ते वैरोचनिमासीनं गुप्तं चासुरयूथपैः
श्रिया परमया जुष्टं जिताशेषमुपागमन्‌
महेन्द्रः श्लक्ष्णया वाचा सान्त्वयित्वा महामतिः
अभ्यभाषत तत्सर्वं शिक्षितं पुरुषोत्तमात्‌
तत्त्वरोचत दैत्यस्य तत्रान्ये येऽसुराधिपाः
शम्बरोऽरिष्टनेमिश्च ये च त्रिपुरवासिनः
ततो देवासुराः कृत्वा संविदं कृतसौहृदाः
उद्यमं परमं चक्रुरमृतार्थे परन्तप
ततस्ते मन्दरगिरिमोजसोत्पाट्य दुर्मदाः
नदन्त उदधिं निन्युः शक्ताः परिघबाहवः
दूरभारोद्वहश्रान्ताः शक्रवैरोचनादयः
अपारयन्तस्तं वोढुं विवशा विजहुः पथि
निपतन्स गिरिस्तत्र बहूनमरदानवान्‌
चूर्णयामास महता भारेण कनकाचलः
तांस्तथा भग्नमनसो भग्नबाहूरुकन्धरान्‌
विज्ञाय भगवांस्तत्र बभूव गरुडध्वजः
गिरिपातविनिष्पिष्टान्विलोक्यामरदानवान्‌
ईक्षया जीवयामास निर्जरान्निर्व्रणान्यथा
गिरिं चारोप्य गरुडे हस्तेनैकेन लीलया
आरुह्य प्रययावब्धिं सुरासुरगणैर्वृतः
अवरोप्य गिरिं स्कन्धात्सुपर्णः पततां वरः
ययौ जलान्त उत्सृज्य हरिणा स विसर्जितः
श्रीशुक उवाच
ते नागराजमामन्त्र्य फलभागेन वासुकिम्‌
परिवीय गिरौ तस्मिन्नेत्रमब्धिं मुदान्विताः
आरेभिरे सुरा यत्ता अमृतार्थे कुरूद्वह
हरिः पुरस्ताज्जगृहे पूर्वं देवास्ततोऽभवन्‌
तन्नैच्छन्दैत्यपतयो महापुरुषचेष्टितम्‌
न गृह्णीमो वयं पुच्छमहेरङ्गममङ्गलम्‌
स्वाध्यायश्रुतसम्पन्नाः प्रख्याता जन्मकर्मभिः
इति तूष्णीं स्थितान्दैत्यान्विलोक्य पुरुषोत्तमः
स्मयमानो विसृज्याग्रं पुच्छं जग्राह सामरः
कृतस्थानविभागास्त एवं कश्यपनन्दनाः
ममन्थुः परमं यत्ता अमृतार्थं पयोनिधिम्‌
मथ्यमानेऽर्णवे सोऽद्रिरनाधारो ह्यपोऽविशत्‌
ध्रियमाणोऽपि बलिभिर्गौरवात्पाण्डुनन्दन
ते सुनिर्विण्णमनसः परिम्लानमुखश्रियः
आसन्स्वपौरुषे नष्टे दैवेनातिबलीयसा
विलोक्य विघ्नेशविधिं तदेश्वरो दुरन्तवीर्योऽवितथाभिसन्धिः
कृत्वा वपुः कच्छपमद्भुतं महत्प्रविश्य तोयं गिरिमुज्जहार
तमुत्थितं वीक्ष्य कुलाचलं पुनः समुद्यता निर्मथितुं सुरासुराः
दधार पृष्ठेन स लक्षयोजन प्रस्तारिणा द्वीप इवापरो महान्‌
सुरासुरेन्द्रैर्भुजवीर्यवेपितं परिभ्रमन्तं गिरिमङ्ग पृष्ठतः
बिभ्रत्तदावर्तनमादिकच्छपो मेनेऽङ्गकण्डूयनमप्रमेयः
तथासुरानाविशदासुरेण रूपेण तेषां बलवीर्यमीरयन्‌
उद्दीपयन्देवगणांश्च विष्णुर्दैवेन नागेन्द्रमबोधरूपः
उपर्यगेन्द्रं गिरिराडिवान्य आक्रम्य हस्तेन सहस्रबाहुः
तस्थौ दिवि ब्रह्मभवेन्द्रमुख्यैरभिष्टुवद्भिः सुमनोऽभिवृष्टः
उपर्यधश्चात्मनि गोत्रनेत्रयोः परेण ते प्राविशता समेधिताः
ममन्थुरब्धिं तरसा मदोत्कटा महाद्रिणा क्षोभितनक्रचक्रम्‌
अहीन्द्रसाहस्रकठोरदृङ्मुख श्वासाग्निधूमाहतवर्चसोऽसुराः
पौलोमकालेयबलील्वलादयो दवाग्निदग्धाः सरला इवाभवन्‌
देवांश्च तच्छ्वासशिखाहतप्रभान्धूम्राम्बरस्रग्वरकञ्चुकाननान्‌
समभ्यवर्षन्भगवद्वशा घना ववुः समुद्रोर्म्युपगूढवायवः
मथ्यमानात्तथा सिन्धोर्देवासुरवरूथपैः
यदा सुधा न जायेत निर्ममन्थाजितः स्वयम्‌
मेघश्यामः कनकपरिधिः कर्णविद्योतविद्युन्‌
मूर्ध्नि भ्राजद्विलुलितकचः स्रग्धरो रक्तनेत्रः
जैत्रैर्दोर्भिर्जगदभयदैर्दन्दशूकं गृहीत्वा
मथ्नन्मथ्ना प्रतिगिरिरिवाशोभताथो धृताद्रिः
निर्मथ्यमानादुदधेरभूद्विषं महोल्बणं हालहलाह्वमग्रतः
सम्भ्रान्तमीनोन्मकराहिकच्छपात्तिमिद्विपग्राहतिमिङ्गिलाकुलात्‌
तदुग्रवेगं दिशि दिश्युपर्यधो विसर्पदुत्सर्पदसह्यमप्रति
भीताः प्रजा दुद्रुवुरङ्ग सेश्वरा अरक्ष्यमाणाः शरणं सदाशिवम्‌
विलोक्य तं देववरं त्रिलोक्या भवाय देव्याभिमतं मुनीनाम्‌
आसीनमद्रावपवर्गहेतोस्तपो जुषाणं स्तुतिभिः प्रणेमुः
श्रीप्रजापतय ऊचुः
देवदेव महादेव भूतात्मन्भूतभावन
त्राहि नः शरणापन्नांस्त्रैलोक्यदहनाद्विषात्‌
त्वमेकः सर्वजगत ईश्वरो बन्धमोक्षयोः
तं त्वामर्चन्ति कुशलाः प्रपन्नार्तिहरं गुरुम्‌
गुणमय्या स्वशक्त्यास्य सर्गस्थित्यप्ययान्विभो
धत्से यदा स्वदृग्भूमन्ब्रह्मविष्णुशिवाभिधाम्‌
त्वं ब्रह्म परमं गुह्यं सदसद्भावभावनम्‌
नानाशक्तिभिराभातस्त्वमात्मा जगदीश्वरः
त्वं शब्दयोनिर्जगदादिरात्मा प्राणेन्द्रियद्रव्यगुणः स्वभावः
कालः क्रतुः सत्यमृतं च धर्मस्त्वय्यक्षरं यत्त्रिवृदामनन्ति
अग्निर्मुखं तेऽखिलदेवतात्मा क्षितिं विदुर्लोकभवाङ्घ्रिपङ्कजम्‌
कालं गतिं तेऽखिलदेवतात्मनो दिशश्च कर्णौ रसनं जलेशम्‌
नाभिर्नभस्ते श्वसनं नभस्वान्सूर्यश्च चक्षूंषि जलं स्म रेतः
परावरात्माश्रयणं तवात्मा सोमो मनो द्यौर्भगवन्शिरस्ते
कुक्षिः समुद्रा गिरयोऽस्थिसङ्घा रोमाणि सर्वौषधिवीरुधस्ते
छन्दांसि साक्षात्तव सप्त धातवस्त्रयीमयात्मन्हृदयं सर्वधर्मः
मुखानि पञ्चोपनिषदस्तवेश यैस्त्रिंशदष्टोत्तरमन्त्रवर्गः
यत्तच्छिवाख्यं परमात्मतत्त्वं देव स्वयंज्योतिरवस्थितिस्ते
छाया त्वधर्मोर्मिषु यैर्विसर्गो नेत्रत्रयं सत्त्वरजस्तमांसि
साङ्ख्यात्मनः शास्त्रकृतस्तवेक्षा छन्दोमयो देव ऋषिः पुराणः
न ते गिरित्राखिललोकपाल विरिञ्चवैकुण्ठसुरेन्द्रगम्यम्‌
ज्योतिः परं यत्र रजस्तमश्च सत्त्वं न यद्ब्रह्म निरस्तभेदम्‌
कामाध्वरत्रिपुरकालगराद्यनेक
भूतद्रुहः क्षपयतः स्तुतये न तत्ते
यस्त्वन्तकाल इदमात्मकृतं स्वनेत्र
वह्निस्फुलिङ्गशिखया भसितं न वेद
ये त्वात्मरामगुरुभिर्हृदि चिन्तिताङ्घ्रि
द्वन्द्वं चरन्तमुमया तपसाभितप्तम्‌
कत्थन्त उग्रपरुषं निरतं श्मशाने
ते नूनमूतिमविदंस्तव हातलज्जाः
तत्तस्य ते सदसतोः परतः परस्य
नाञ्जः स्वरूपगमने प्रभवन्ति भूम्नः
ब्रह्मादयः किमुत संस्तवने वयं तु
तत्सर्गसर्गविषया अपि शक्तिमात्रम्‌
एतत्परं प्रपश्यामो न परं ते महेश्वर
मृडनाय हि लोकस्य व्यक्तिस्तेऽव्यक्तकर्मणः
श्रीशुक उवाच
तद्वीक्ष्य व्यसनं तासां कृपया भृशपीडितः
सर्वभूतसुहृद्देव इदमाह सतीं प्रियाम्‌
श्रीशिव उवाच
अहो बत भवान्येतत्प्रजानां पश्य वैशसम्‌
क्षीरोदमथनोद्भूतात्कालकूटादुपस्थितम्‌
आसां प्राणपरीप्सूनां विधेयमभयं हि मे
एतावान्हि प्रभोरर्थो यद्दीनपरिपालनम्‌
प्राणैः स्वैः प्राणिनः पान्ति साधवः क्षणभङ्गुरैः
बद्धवैरेषु भूतेषु मोहितेष्वात्ममायया
पुंसः कृपयतो भद्रे सर्वात्मा प्रीयते हरिः
प्रीते हरौ भगवति प्रीयेऽहं सचराचरः
तस्मादिदं गरं भुञ्जे प्रजानां स्वस्तिरस्तु मे
श्रीशुक उवाच
एवमामन्त्र्य भगवान्भवानीं विश्वभावनः
तद्विषं जग्धुमारेभे प्रभावज्ञान्वमोदत
ततः करतलीकृत्य व्यापि हालाहलं विषम्‌
अभक्षयन्महादेवः कृपया भूतभावनः
तस्यापि दर्शयामास स्ववीर्यं जलकल्मषः
यच्चकार गले नीलं तच्च साधोर्विभूषणम्‌
तप्यन्ते लोकतापेन साधवः प्रायशो जनाः
परमाराधनं तद्धि पुरुषस्याखिलात्मनः
निशम्य कर्म तच्छम्भोर्देवदेवस्य मीढुषः
प्रजा दाक्षायणी ब्रह्मा वैकुण्ठश्च शशंसिरे
प्रस्कन्नं पिबतः पाणेर्यत्किञ्चिज्जगृहुः स्म तत्‌
वृश्चिकाहिविषौषध्यो दन्दशूकाश्च येऽपरे
श्रीशुक उवाच
पीते गरे वृषाङ्केण प्रीतास्तेऽमरदानवाः
ममन्थुस्तरसा सिन्धुं हविर्धानी ततोऽभवत्‌
तामग्निहोत्रीमृषयो जगृहुर्ब्रह्मवादिनः
यज्ञस्य देवयानस्य मेध्याय हविषे नृप
तत उच्चैःश्रवा नाम हयोऽभूच्चन्द्रपाण्डुरः
तस्मिन्बलिः स्पृहां चक्रे नेन्द्र ईश्वरशिक्षया
तत ऐरावतो नाम वारणेन्द्रो विनिर्गतः
दन्तैश्चतुर्भिः श्वेताद्रेर्हरन्भगवतो महिम्‌
ऐरावणादयस्त्वष्टौ दिग्गजा अभवंस्ततः
अभ्रमुप्रभृतयोऽष्टौ च करिण्यस्त्वभवन्नृप
कौस्तुभाख्यमभूद्रत्नं पद्मरागो महोदधेः
तस्मिन्मणौ स्पृहां चक्रे वक्षोऽलङ्करणे हरिः
ततोऽभवत्पारिजातः सुरलोकविभूषणम्‌
पूरयत्यर्थिनो योऽर्थैः शश्वद्भुवि यथा भवान्‌
ततश्चाप्सरसो जाता निष्ककण्ठ्यः सुवाससः
रमण्यः स्वर्गिणां वल्गु गतिलीलावलोकनैः
ततश्चाविरभूत्साक्षाच्छ्री रमा भगवत्परा
रञ्जयन्ती दिशः कान्त्या विद्युत्सौदामनी यथा
तस्यां चक्रुः स्पृहां सर्वे ससुरासुरमानवाः
रूपौदार्यवयोवर्ण महिमाक्षिप्तचेतसः
तस्या आसनमानिन्ये महेन्द्रो महदद्भुतम्‌
मूर्तिमत्यः सरिच्छ्रेष्ठा हेमकुम्भैर्जलं शुचि
आभिषेचनिका भूमिराहरत्सकलौषधीः
गावः पञ्च पवित्राणि वसन्तो मधुमाधवौ
ऋषयः कल्पयां चक्रुराभिषेकं यथाविधि
जगुर्भद्राणि गन्धर्वा नट्यश्च ननृतुर्जगुः
मेघा मृदङ्गपणव मुरजानकगोमुखान्‌
व्यनादयन्शङ्खवेणु वीणास्तुमुलनिःस्वनान्‌
ततोऽभिषिषिचुर्देवीं श्रियं पद्मकरां सतीम्‌
दिगिभाः पूर्णकलशैः सूक्तवाक्यैर्द्विजेरितैः
समुद्रः पीतकौशेय वाससी समुपाहरत्‌
वरुणः स्रजं वैजयन्तीं मधुना मत्तषट्पदाम्‌
भूषणानि विचित्राणि विश्वकर्मा प्रजापतिः
हारं सरस्वती पद्ममजो नागाश्च कुण्डले
ततः कृतस्वस्त्ययनोत्पलस्रजं नदद्द्विरेफां परिगृह्य पाणिना
चचाल वक्त्रं सुकपोलकुण्डलं सव्रीडहासं दधती सुशोभनम्‌
स्तनद्वयं चातिकृशोदरी समं निरन्तरं चन्दनकुङ्कुमोक्षितम्‌
ततस्ततो नूपुरवल्गु शिञ्जितैर्विसर्पती हेमलतेव सा बभौ
विलोकयन्ती निरवद्यमात्मनः पदं ध्रुवं चाव्यभिचारिसद्गुणम्‌
गन्धर्वसिद्धासुरयक्षचारण त्रैपिष्टपेयादिषु नान्वविन्दत
नूनं तपो यस्य न मन्युनिर्जयो ज्ञानं क्वचित्तच्च न सङ्गवर्जितम्‌
कश्चिन्महांस्तस्य न कामनिर्जयः स ईश्वरः किं परतो व्यपाश्रयः
धर्मः क्वचित्तत्र न भूतसौहृदं त्यागः क्वचित्तत्र न मुक्तिकारणम्‌
वीर्यं न पुंसोऽस्त्यजवेगनिष्कृतं न हि द्वितीयो गुणसङ्गवर्जितः
क्वचिच्चिरायुर्न हि शीलमङ्गलं क्वचित्तदप्यस्ति न वेद्यमायुषः
यत्रोभयं कुत्र च सोऽप्यमङ्गलः सुमङ्गलः कश्च न काङ्क्षते हि माम्‌
एवं विमृश्याव्यभिचारिसद्गुणैर्वरं निजैकाश्रयतयागुणाश्रयम्‌
वव्रे वरं सर्वगुणैरपेक्षितं रमा मुकुन्दं निरपेक्षमीप्सितम्‌
तस्यांसदेश उशतीं नवकञ्जमालां
माद्यन्मधुव्रतवरूथगिरोपघुष्टाम्‌
तस्थौ निधाय निकटे तदुरः स्वधाम
सव्रीडहासविकसन्नयनेन याता
तस्याः श्रियस्त्रिजगतो जनको जनन्या
वक्षो निवासमकरोत्परमं विभूतेः
श्रीः स्वाः प्रजाः सकरुणेन निरीक्षणेन
यत्र स्थितैधयत साधिपतींस्त्रिलोकान्‌
शङ्खतूर्यमृदङ्गानां वादित्राणां पृथुः स्वनः
देवानुगानां सस्त्रीणां नृत्यतां गायतामभूत्‌
ब्रह्मरुद्राङ्गिरोमुख्याः सर्वे विश्वसृजो विभुम्‌
ईडिरेऽवितथैर्मन्त्रैस्तल्लिङ्गैः पुष्पवर्षिणः
श्रियावलोकिता देवाः सप्रजापतयः प्रजाः
शीलादिगुणसम्पन्ना लेभिरे निर्वृतिं पराम्‌
निःसत्त्वा लोलुपा राजन्निरुद्योगा गतत्रपाः
यदा चोपेक्षिता लक्ष्म्या बभूवुर्दैत्यदानवाः
अथासीद्वारुणी देवी कन्या कमललोचना
असुरा जगृहुस्तां वै हरेरनुमतेन ते
अथोदधेर्मथ्यमानात्काश्यपैरमृतार्थिभिः
उदतिष्ठन्महाराज पुरुषः परमाद्भुतः
दीर्घपीवरदोर्दण्डः कम्बुग्रीवोऽरुणेक्षणः
श्यामलस्तरुणः स्रग्वी सर्वाभरणभूषितः
पीतवासा महोरस्कः सुमृष्टमणिकुण्डलः
स्निग्धकुञ्चितकेशान्त सुभगः सिंहविक्रमः
अमृतापूर्णकलसं बिभ्रद्वलयभूषितः
स वै भगवतः साक्षाद्विष्णोरंशांशसम्भवः
धन्वन्तरिरिति ख्यात आयुर्वेददृगिज्यभाक्‌
तमालोक्यासुराः सर्वे कलसं चामृताभृतम्‌
लिप्सन्तः सर्ववस्तूनि कलसं तरसाहरन्‌
नीयमानेऽसुरैस्तस्मिन्कलसेऽमृतभाजने
विषण्णमनसो देवा हरिं शरणमाययुः
इति तद्दैन्यमालोक्य भगवान्भृत्यकामकृत्‌
मा खिद्यत मिथोऽर्थं वः साधयिष्ये स्वमायया
मिथः कलिरभूत्तेषां तदर्थे तर्षचेतसाम्‌
अहं पूर्वमहं पूर्वं न त्वं न त्वमिति प्रभो
देवाः स्वं भागमर्हन्ति ये तुल्यायासहेतवः
सत्रयाग इवैतस्मिन्नेष धर्मः सनातनः
इति स्वान्प्रत्यषेधन्वै दैतेया जातमत्सराः
दुर्बलाः प्रबलान्राजन्गृहीतकलसान्मुहुः
एतस्मिन्नन्तरे विष्णुः सर्वोपायविदीश्वरः
योषिद्रूपमनिर्देश्यं दधारपरमाद्भुतम्‌
प्रेक्षणीयोत्पलश्यामं सर्वावयवसुन्दरम्‌
समानकर्णाभरणं सुकपोलोन्नसाननम्‌
नवयौवननिर्वृत्त स्तनभारकृशोदरम्‌
मुखामोदानुरक्तालि झङ्कारोद्विग्नलोचनम्‌
बिभ्रत्सुकेशभारेण मालामुत्फुल्लमल्लिकाम्‌
सुग्रीवकण्ठाभरणं सुभुजाङ्गदभूषितम्‌
विरजाम्बरसंवीत नितम्बद्वीपशोभया
काञ्च्या प्रविलसद्वल्गु चलच्चरणनूपुरम्‌
सव्रीडस्मितविक्षिप्त भ्रूविलासावलोकनैः
दैत्ययूथपचेतःसु काममुद्दीपयन्मुहुः
श्रीशुक उवाच
तेऽन्योन्यतोऽसुराः पात्रं हरन्तस्त्यक्तसौहृदाः
क्षिपन्तो दस्युधर्माण आयान्तीं ददृशुः स्त्रियम्‌
अहो रूपमहो धाम अहो अस्या नवं वयः
इति ते तामभिद्रुत्य पप्रच्छुर्जातहृच्छयाः
का त्वं कञ्जपलाशाक्षि कुतो वा किं चिकीर्षसि
कस्यासि वद वामोरु मथ्नतीव मनांसि नः
न वयं त्वामरैर्दैत्यैः सिद्धगन्धर्वचारणैः
नास्पृष्टपूर्वां जानीमो लोकेशैश्च कुतो नृभिः
नूनं त्वं विधिना सुभ्रूः प्रेषितासि शरीरिणाम्‌
सर्वेन्द्रियमनःप्रीतिं विधातुं सघृणेन किम्‌
सा त्वं नः स्पर्धमानानामेकवस्तुनि मानिनि
ज्ञातीनां बद्धवैराणां शं विधत्स्व सुमध्यमे
वयं कश्यपदायादा भ्रातरः कृतपौरुषाः
विभजस्व यथान्यायं नैव भेदो यथा भवेत्‌
इत्युपामन्त्रितो दैत्यैर्मायायोषिद्वपुर्हरिः
प्रहस्य रुचिरापाङ्गैर्निरीक्षन्निदमब्रवीत्‌
श्रीभगवानुवाच
कथं कश्यपदायादाः पुंश्चल्यां मयि सङ्गताः
विश्वासं पण्डितो जातु कामिनीषु न याति हि
सालावृकाणां स्त्रीणां च स्वैरिणीनां सुरद्विषः
सख्यान्याहुरनित्यानि नूत्नं नूत्नं विचिन्वताम्‌
श्रीशुक उवाच
इति ते क्ष्वेलितैस्तस्या आश्वस्तमनसोऽसुराः
जहसुर्भावगम्भीरं ददुश्चामृतभाजनम्‌
ततो गृहीत्वामृतभाजनं हरिर्बभाष ईषत्स्मितशोभया गिरा
यद्यभ्युपेतं क्व च साध्वसाधु वा कृतं मया वो विभजे सुधामिमाम्‌
इत्यभिव्याहृतं तस्या आकर्ण्यासुरपुङ्गवाः
अप्रमाणविदस्तस्यास्तत्तथेत्यन्वमंसत
अथोपोष्य कृतस्नाना हुत्वा च हविषानलम्‌
दत्त्वा गोविप्रभूतेभ्यः कृतस्वस्त्ययना द्विजैः
यथोपजोषं वासांसि परिधायाहतानि ते
कुशेषु प्राविशन्सर्वे प्रागग्रेष्वभिभूषिताः
प्राङ्मुखेषूपविष्टेषु सुरेषु दितिजेषु च
धूपामोदितशालायांजुष्टायां माल्यदीपकैः
तस्यां नरेन्द्र करभोरुरुशद्दुकूल श्रोणीतटालसगतिर्मदविह्वलाक्षी
सा कूजती कनकनूपुरशिञ्जितेन कुम्भस्तनी कलसपाणिरथाविवेश
तां श्रीसखीं कनककुण्डलचारुकर्ण नासाकपोलवदनां परदेवताख्याम्‌
संवीक्ष्य सम्मुमुहुरुत्स्मितवीक्षणेन देवासुरा विगलितस्तनपट्टिकान्ताम्‌
असुराणां सुधादानं सर्पाणामिव दुर्नयम्‌
मत्वा जातिनृशंसानां न तां व्यभजदच्युतः
कल्पयित्वा पृथक्पङ्क्तीरुभयेषां जगत्पतिः
तांश्चोपवेशयामास स्वेषु स्वेषु च पङ्क्तिषु
दैत्यान्गृहीतकलसो वञ्चयन्नुपसञ्चरैः
दूरस्थान्पाययामासजरामृत्युहरां सुधाम्‌
ते पालयन्तः समयमसुराः स्वकृतं नृप
तूष्णीमासन्कृतस्नेहाः स्त्रीविवादजुगुप्सया
तस्यां कृतातिप्रणयाः प्रणयापायकातराः
बहुमानेन चाबद्धा नोचुः किञ्चन विप्रियम्‌
देवलिङ्गप्रतिच्छन्नः स्वर्भानुर्देवसंसदि
प्रविष्टः सोममपिबच्चन्द्रार्काभ्यां च सूचितः
चक्रेण क्षुरधारेण जहार पिबतः शिरः
हरिस्तस्य कबन्धस्तु सुधयाप्लावितोऽपतत्‌
शिरस्त्वमरतां नीतमजो ग्रहमचीकॢपत्‌
यस्तु पर्वणि चन्द्रार्कावभिधावति वैरधीः
पीतप्रायेऽमृते देवैर्भगवान्लोकभावनः
पश्यतामसुरेन्द्राणां स्वं रूपं जगृहे हरिः
एवं सुरासुरगणाः समदेशकाल
हेत्वर्थकर्ममतयोऽपि फले विकल्पाः
तत्रामृतं सुरगणाः फलमञ्जसापुर्‌
यत्पादपङ्कजरजःश्रयणान्न दैत्याः
यद्युज्यतेऽसुवसुकर्ममनोवचोभिर्‌
देहात्मजादिषु नृभिस्तदसत्पृथक्त्वात्‌
तैरेव सद्भवति यत्क्रियतेऽपृथक्त्वात्‌
सर्वस्य तद्भवति मूलनिषेचनं यत्‌
श्रीशुक उवाच
इति दानवदैतेया नाविन्दन्नमृतं नृप
युक्ताः कर्मणि यत्ताश्च वासुदेवपराङ्मुखाः
साधयित्वामृतं राजन्पाययित्वा स्वकान्सुरान्‌
पश्यतां सर्वभूतानां ययौ गरुडवाहनः
सपत्नानां परामृद्धिं दृष्ट्वा ते दितिनन्दनाः
अमृष्यमाणा उत्पेतुर्देवान्प्रत्युद्यतायुधाः
ततः सुरगणाः सर्वे सुधया पीतयैधिताः
प्रतिसंयुयुधुः शस्त्रैर्नारायणपदाश्रयाः
तत्र दैवासुरो नाम रणः परमदारुणः
रोधस्युदन्वतो राजंस्तुमुलो रोमहर्षणः
तत्रान्योन्यं सपत्नास्ते संरब्धमनसो रणे
समासाद्यासिभिर्बाणैर्निजघ्नुर्विविधायुधैः
शङ्खतूर्यमृदङ्गानां भेरीडमरिणां महान्‌
हस्त्यश्वरथपत्तीनां नदतां निस्वनोऽभवत्‌
रथिनो रथिभिस्तत्र पत्तिभिः सह पत्तयः
हया हयैरिभाश्चेभैः समसज्जन्त संयुगे
उष्ट्रैः केचिदिभैः केचिदपरे युयुधुः खरैः
केचिद्गौरमुखैरृक्षैर्द्वीपिभिर्हरिभिर्भटाः
गृध्रैः कङ्कैर्बकैरन्ये श्येनभासैस्तिमिङ्गिलैः
शरभैर्महिषैः खड्गैर्गोवृषैर्गवयारुणैः
शिवाभिराखुभिः केचित्कृकलासैः शशैर्नरैः
बस्तैरेके कृष्णसारैर्हंसैरन्ये च सूकरैः
अन्ये जलस्थलखगैः सत्त्वैर्विकृतविग्रहैः
सेनयोरुभयो राजन्विविशुस्तेऽग्रतोऽग्रतः
चित्रध्वजपटै राजन्नातपत्रैः सितामलैः
महाधनैर्वज्रदण्डैर्व्यजनैर्बार्हचामरैः
वातोद्धूतोत्तरोष्णीषैरर्चिर्भिर्वर्मभूषणैः
स्फुरद्भिर्विशदैः शस्त्रैः सुतरां सूर्यरश्मिभिः
देवदानववीराणां ध्वजिन्यौ पाण्डुनन्दन
रेजतुर्वीरमालाभिर्यादसामिव सागरौ
वैरोचनो बलिः सङ्ख्ये सोऽसुराणां चमूपतिः
यानं वैहायसं नाम कामगं मयनिर्मितम्‌
सर्वसाङ्ग्रामिकोपेतं सर्वाश्चर्यमयं प्रभो
अप्रतर्क्यमनिर्देश्यं दृश्यमानमदर्शनम्‌
आस्थितस्तद्विमानाग्र्यं सर्वानीकाधिपैर्वृतः
बालव्यजनछत्राग्र्यै रेजे चन्द्र इवोदये
तस्यासन्सर्वतो यानैर्यूथानां पतयोऽसुराः
नमुचिः शम्बरो बाणो विप्रचित्तिरयोमुखः
द्विमूर्धा कालनाभोऽथ प्रहेतिर्हेतिरिल्वलः
शकुनिर्भूतसन्तापो वज्रदंष्ट्रो विरोचनः
हयग्रीवः शङ्कुशिराः कपिलो मेघदुन्दुभिः
तारकश्चक्रदृक्षुम्भो निशुम्भो जम्भ उत्कलः
अरिष्टोऽरिष्टनेमिश्च मयश्च त्रिपुराधिपः
अन्ये पौलोमकालेया निवातकवचादयः
अलब्धभागाः सोमस्य केवलं क्लेशभागिनः
सर्व एते रणमुखे बहुशो निर्जितामराः
सिंहनादान्विमुञ्चन्तः शङ्खान्दध्मुर्महारवान्‌
दृष्ट्वा सपत्नानुत्सिक्तान्बलभित्कुपितो भृशम्‌
ऐरावतं दिक्करिणमारूढः शुशुभे स्वराट्‌
यथा स्रवत्प्रस्रवणमुदयाद्रिमहर्पतिः
तस्यासन्सर्वतो देवा नानावाहध्वजायुधाः
लोकपालाः सहगणैर्वाय्वग्निवरुणादयः
तेऽन्योन्यमभिसंसृत्य क्षिपन्तो मर्मभिर्मिथः
आह्वयन्तो विशन्तोऽग्रे युयुधुर्द्वन्द्वयोधिनः
युयोध बलिरिन्द्रेण तारकेण गुहोऽस्यत
वरुणो हेतिनायुध्यन्मित्रो राजन्प्रहेतिना
यमस्तु कालनाभेन विश्वकर्मा मयेन वै
शम्बरो युयुधे त्वष्ट्रा सवित्रा तु विरोचनः
अपराजितेन नमुचिरश्विनौ वृषपर्वणा
सूर्यो बलिसुतैर्देवो बाणज्येष्ठैः शतेन च
राहुणा च तथा सोमः पुलोम्ना युयुधेऽनिलः
निशुम्भशुम्भयोर्देवी भद्रकाली तरस्विनी
वृषाकपिस्तु जम्भेन महिषेण विभावसुः
इल्वलः सह वातापिर्ब्रह्मपुत्रैररिन्दम
कामदेवेन दुर्मर्ष उत्कलो मातृभिः सह
बृहस्पतिश्चोशनसा नरकेण शनैश्चरः
मरुतो निवातकवचैः कालेयैर्वसवोऽमराः
विश्वेदेवास्तु पौलोमै रुद्राः क्रोधवशैः सह
त एवमाजावसुराः सुरेन्द्रा द्वन्द्वेन संहत्य च युध्यमानाः
अन्योन्यमासाद्य निजघ्नुरोजसा जिगीषवस्तीक्ष्णशरासितोमरैः
भुशुण्डिभिश्चक्रगदर्ष्टिपट्टिशैः शक्त्युल्मुकैः प्रासपरश्वधैरपि
निस्त्रिंशभल्लैः परिघैः समुद्गरैः सभिन्दिपालैश्च शिरांसि चिच्छिदुः
गजास्तुरङ्गाः सरथाः पदातयः सारोहवाहा विविधा विखण्डिताः
निकृत्तबाहूरुशिरोधराङ्घ्रयश्छिन्नध्वजेष्वासतनुत्रभूषणाः
तेषां पदाघातरथाङ्गचूर्णितादायोधनादुल्बण उत्थितस्तदा
रेणुर्दिशः खं द्युमणिं च छादयन्न्यवर्ततासृक्स्रुतिभिः परिप्लुतात्‌
शिरोभिरुद्धूतकिरीटकुण्डलैः संरम्भदृग्भिः परिदष्टदच्छदैः
महाभुजैः साभरणैः सहायुधैः सा प्रास्तृता भूः करभोरुभिर्बभौ
कबन्धास्तत्र चोत्पेतुः पतितस्वशिरोऽक्षिभिः
उद्यतायुधदोर्दण्डैराधावन्तो भटान्मृधे
बलिर्महेन्द्रं दशभिस्त्रिभिरैरावतं शरैः
चतुर्भिश्चतुरो वाहानेकेनारोहमार्च्छयत्‌
स तानापततः शक्रस्तावद्भिः शीघ्रविक्रमः
चिच्छेद निशितैर्भल्लैरसम्प्राप्तान्हसन्निव
तस्य कर्मोत्तमं वीक्ष्य दुर्मर्षः शक्तिमाददे
तां ज्वलन्तीं महोल्काभां हस्तस्थामच्छिनद्धरिः
ततः शूलं ततः प्रासं ततस्तोमरमृष्टयः
यद्यच्छस्त्रं समादद्यात्सर्वं तदच्छिनद्विभुः
ससर्जाथासुरीं मायामन्तर्धानगतोऽसुरः
ततः प्रादुरभूच्छैलः सुरानीकोपरि प्रभो
ततो निपेतुस्तरवो दह्यमाना दवाग्निना
शिलाः सटङ्कशिखराश्चूर्णयन्त्यो द्विषद्बलम्‌
महोरगाः समुत्पेतुर्दन्दशूकाः सवृश्चिकाः
सिंहव्याघ्रवराहाश्च मर्दयन्तो महागजाः
यातुधान्यश्च शतशः शूलहस्ता विवाससः
छिन्धि भिन्धीति वादिन्यस्तथा रक्षोगणाः प्रभो
ततो महाघना व्योम्नि गम्भीरपरुषस्वनाः
अङ्गारान्मुमुचुर्वातैराहताः स्तनयित्नवः
सृष्टो दैत्येन सुमहान्वह्निः श्वसनसारथिः
सांवर्तक इवात्युग्रो विबुधध्वजिनीमधाक्‌
ततः समुद्र उद्वेलः सर्वतः प्रत्यदृश्यत
प्रचण्डवातैरुद्धूत तरङ्गावर्तभीषणः
एवं दैत्यैर्महामायैरलक्ष्यगतिभी रणे
सृज्यमानासु मायासु विषेदुः सुरसैनिकाः
न तत्प्रतिविधिं यत्र विदुरिन्द्रादयो नृप
ध्यातः प्रादुरभूत्तत्र भगवान्विश्वभावनः
ततः सुपर्णांसकृताङ्घ्रिपल्लवः पिशङ्गवासा नवकञ्जलोचनः
अदृश्यताष्टायुधबाहुरुल्लसच्छ्रीकौस्तुभानर्घ्यकिरीटकुण्डलः
तस्मिन्प्रविष्टेऽसुरकूटकर्मजा माया विनेशुर्महिना महीयसः
स्वप्नो यथा हि प्रतिबोध आगते हरिस्मृतिः सर्वविपद्विमोक्षणम्‌
दृष्ट्वा मृधे गरुडवाहमिभारिवाह आविध्य शूलमहिनोदथ कालनेमिः
तल्लीलया गरुडमूर्ध्नि पतद्गृहीत्वा तेनाहनन्नृप सवाहमरिं त्र्यधीशः
माली सुमाल्यतिबलौ युधि पेततुर्यच्चक्रेण कृत्तशिरसावथ माल्यवांस्तम्‌
आहत्य तिग्मगदयाहनदण्डजेन्द्रं तावच्छिरोऽच्छिनदरेर्नदतोऽरिणाद्यः
श्रीशुक उवाच
अथो सुराः प्रत्युपलब्धचेतसः परस्य पुंसः परयानुकम्पया
जघ्नुर्भृशं शक्रसमीरणादयस्तांस्तान्रणे यैरभिसंहताः पुरा
वैरोचनाय संरब्धो भगवान्पाकशासनः
उदयच्छद्यदा वज्रं प्रजा हा हेति चुक्रुशुः
वज्रपाणिस्तमाहेदं तिरस्कृत्य पुरःस्थितम्‌
मनस्विनं सुसम्पन्नं विचरन्तं महामृधे
नटवन्मूढ मायाभिर्मायेशान्नो जिगीषसि
जित्वा बालान्निबद्धाक्षान्नटो हरति तद्धनम्‌
आरुरुक्षन्ति मायाभिरुत्सिसृप्सन्ति ये दिवम्‌
तान्दस्यून्विधुनोम्यज्ञान्पूर्वस्माच्च पदादधः
सोऽहं दुर्मायिनस्तेऽद्य वज्रेण शतपर्वणा
शिरो हरिष्ये मन्दात्मन्घटस्व ज्ञातिभिः सह
श्रीबलिरुवाच
सङ्ग्रामे वर्तमानानां कालचोदितकर्मणाम्‌
कीर्तिर्जयोऽजयो मृत्युः सर्वेषां स्युरनुक्रमात्‌
तदिदं कालरशनं जगत्पश्यन्ति सूरयः
न हृष्यन्ति न शोचन्ति तत्र यूयमपण्डिताः
न वयं मन्यमानानामात्मानं तत्र साधनम्‌
गिरो वः साधुशोच्यानां गृह्णीमो मर्मताडनाः
श्रीशुक उवाच
इत्याक्षिप्य विभुं वीरो नाराचैर्वीरमर्दनः
आकर्णपूर्णैरहनदाक्षेपैराह तं पुनः
एवं निराकृतो देवो वैरिणा तथ्यवादिना
नामृष्यत्तदधिक्षेपं तोत्राहत इव द्विपः
प्राहरत्कुलिशं तस्मा अमोघं परमर्दनः
सयानो न्यपतद्भूमौ छिन्नपक्ष इवाचलः
सखायं पतितं दृष्ट्वा जम्भो बलिसखः सुहृत्‌
अभ्ययात्सौहृदं सख्युर्हतस्यापि समाचरन्‌
स सिंहवाह आसाद्य गदामुद्यम्य रंहसा
जत्रावताडयच्छक्रं गजं च सुमहाबलः
गदाप्रहारव्यथितो भृशं विह्वलितो गजः
जानुभ्यां धरणीं स्पृष्ट्वा कश्मलं परमं ययौ
ततो रथो मातलिना हरिभिर्दशशतैर्वृतः
आनीतो द्विपमुत्सृज्य रथमारुरुहे विभुः
तस्य तत्पूजयन्कर्म यन्तुर्दानवसत्तमः
शूलेन ज्वलता तं तु स्मयमानोऽहनन्मृधे
सेहे रुजं सुदुर्मर्षां सत्त्वमालम्ब्य मातलिः
इन्द्रो जम्भस्य सङ्क्रुद्धो वज्रेणापाहरच्छिरः
जम्भं श्रुत्वा हतं तस्य ज्ञातयो नारदादृषेः
नमुचिश्च बलः पाकस्तत्रापेतुस्त्वरान्विताः
वचोभिः परुषैरिन्द्रमर्दयन्तोऽस्य मर्मसु
शरैरवाकिरन्मेघा धाराभिरिव पर्वतम्‌
हरीन्दशशतान्याजौ हर्यश्वस्य बलः शरैः
तावद्भिरर्दयामास युगपल्लघुहस्तवान्‌
शताभ्यां मातलिं पाको रथं सावयवं पृथक्‌
सकृत्सन्धानमोक्षेण तदद्भुतमभूद्रणे
नमुचिः पञ्चदशभिः स्वर्णपुङ्खैर्महेषुभिः
आहत्य व्यनदत्सङ्ख्ये सतोय इव तोयदः
सर्वतः शरकूटेन शक्रं सरथसारथिम्‌
छादयामासुरसुराः प्रावृट्सूर्यमिवाम्बुदाः
अलक्षयन्तस्तमतीव विह्वला विचुक्रुशुर्देवगणाः सहानुगाः
अनायकाः शत्रुबलेन निर्जिता वणिक्पथा भिन्ननवो यथार्णवे
ततस्तुराषाडिषुबद्धपञ्जराद्विनिर्गतः साश्वरथध्वजाग्रणीः
बभौ दिशः खं पृथिवीं च रोचयन्स्वतेजसा सूर्य इव क्षपात्यये
निरीक्ष्य पृतनां देवः परैरभ्यर्दितां रणे
उदयच्छद्रिपुं हन्तुं वज्रं वज्रधरो रुषा
स तेनैवाष्टधारेण शिरसी बलपाकयोः
ज्ञातीनां पश्यतां राजन्जहार जनयन्भयम्‌
नमुचिस्तद्वधं दृष्ट्वा शोकामर्षरुषान्वितः
जिघांसुरिन्द्रं नृपते चकार परमोद्यमम्‌
अश्मसारमयं शूलं घण्टावद्धेमभूषणम्‌
प्रगृह्याभ्यद्रवत्क्रुद्धो हतोऽसीति वितर्जयन्‌
प्राहिणोद्देवराजाय निनदन्मृगराडिव
तदापतद्गगनतले महाजवं विचिच्छिदे हरिरिषुभिः सहस्रधा
तमाहनन्नृप कुलिशेन कन्धरे रुषान्वितस्त्रिदशपतिः शिरो हरन्‌
न तस्य हि त्वचमपि वज्र ऊर्जितो बिभेद यः सुरपतिनौजसेरितः
तदद्भुतं परमतिवीर्यवृत्रभित्तिरस्कृतो नमुचिशिरोधरत्वचा
तस्मादिन्द्रोऽबिभेच्छत्रोर्वज्रः प्रतिहतो यतः
किमिदं दैवयोगेन भूतं लोकविमोहनम्‌
येन मे पूर्वमद्रीणां पक्षच्छेदः प्रजात्यये
कृतो निविशतां भारैः पतत्त्रैः पततां भुवि
तपःसारमयं त्वाष्ट्रं वृत्रो येन विपाटितः
अन्ये चापि बलोपेताः सर्वास्त्रैरक्षतत्वचः
सोऽयं प्रतिहतो वज्रो मया मुक्तोऽसुरेऽल्पके
नाहं तदाददे दण्डं ब्रह्मतेजोऽप्यकारणम्‌
इति शक्रं विषीदन्तमाह वागशरीरिणी
नायं शुष्कैरथो नार्द्रैर्वधमर्हति दानवः
मयास्मै यद्वरो दत्तो मृत्युर्नैवार्द्रशुष्कयोः
अतोऽन्यश्चिन्तनीयस्ते उपायो मघवन्रिपोः
तां दैवीं गिरमाकर्ण्य मघवान्सुसमाहितः
ध्यायन्फेनमथापश्यदुपायमुभयात्मकम्‌
न शुष्केण न चार्द्रेण जहार नमुचेः शिरः
तं तुष्टुवुर्मुनिगणा माल्यैश्चावाकिरन्विभुम्‌
गन्धर्वमुख्यौ जगतुर्विश्वावसुपरावसू
देवदुन्दुभयो नेदुर्नर्तक्यो ननृतुर्मुदा
अन्येऽप्येवं प्रतिद्वन्द्वान्वाय्वग्निवरुणादयः
सूदयामासुरसुरान्मृगान्केसरिणो यथा
ब्रह्मणा प्रेषितो देवान्देवर्षिर्नारदो नृप
वारयामास विबुधान्दृष्ट्वा दानवसङ्क्षयम्‌
श्रीनारद उवाच
भवद्भिरमृतं प्राप्तं नारायणभुजाश्रयैः
श्रिया समेधिताः सर्व उपारमत विग्रहात्‌
श्रीशुक उवाच
संयम्य मन्युसंरम्भं मानयन्तो मुनेर्वचः
उपगीयमानानुचरैर्ययुः सर्वे त्रिविष्टपम्‌
येऽवशिष्टा रणे तस्मिन्नारदानुमतेन ते
बलिं विपन्नमादाय अस्तं गिरिमुपागमन्‌
तत्राविनष्टावयवान्विद्यमानशिरोधरान्‌
उशना जीवयामास संजीवन्या स्वविद्यया
बलिश्चोशनसा स्पृष्टः प्रत्यापन्नेन्द्रियस्मृतिः
पराजितोऽपि नाखिद्यल्लोकतत्त्वविचक्षणः
श्रीबादरायणिरुवाच
वृषध्वजो निशम्येदं योषिद्रूपेण दानवान्‌
मोहयित्वा सुरगणान्हरिः सोममपाययत्‌
वृषमारुह्य गिरिशः सर्वभूतगणैर्वृतः
सह देव्या ययौ द्रष्टुं यत्रास्ते मधुसूदनः
सभाजितो भगवता सादरं सोमया भवः
सूपविष्ट उवाचेदं प्रतिपूज्य स्मयन्हरिम्‌
श्रीमहादेव उवाच
देवदेव जगद्व्यापिन्जगदीश जगन्मय
सर्वेषामपि भावानां त्वमात्मा हेतुरीश्वरः
आद्यन्तावस्य यन्मध्यमिदमन्यदहं बहिः
यतोऽव्ययस्य नैतानि तत्सत्यं ब्रह्म चिद्भवान्‌
तवैव चरणाम्भोजं श्रेयस्कामा निराशिषः
विसृज्योभयतः सङ्गं मुनयः समुपासते
त्वं ब्रह्म पूर्णममृतं विगुणं विशोकम्‌
आनन्दमात्रमविकारमनन्यदन्यत्‌
विश्वस्य हेतुरुदयस्थितिसंयमानाम्‌
आत्मेश्वरश्च तदपेक्षतयानपेक्षः
एकस्त्वमेव सदसद्द्वयमद्वयं च
स्वर्णं कृताकृतमिवेह न वस्तुभेदः
अज्ञानतस्त्वयि जनैर्विहितो विकल्पो
यस्माद्गुणव्यतिकरो निरुपाधिकस्य
त्वां ब्रह्म केचिदवयन्त्युत धर्ममेके
एके परं सदसतोः पुरुषं परेशम्‌
अन्येऽवयन्ति नवशक्तियुतं परं त्वां
केचिन्महापुरुषमव्ययमात्मतन्त्रम्‌
नाहं परायुरृषयो न मरीचिमुख्या
जानन्ति यद्विरचितं खलु सत्त्वसर्गाः
यन्मायया मुषितचेतस ईश दैत्य
मर्त्यादयः किमुत शश्वदभद्रवृत्ताः
स त्वं समीहितमदः स्थितिजन्मनाशं
भूतेहितं च जगतो भवबन्धमोक्षौ
वायुर्यथा विशति खं च चराचराख्यं
सर्वं तदात्मकतयावगमोऽवरुन्त्से
अवतारा मया दृष्टा रममाणस्य ते गुणैः
सोऽहं तद्द्रष्टुमिच्छामि यत्ते योषिद्वपुर्धृतम्‌
येन सम्मोहिता दैत्याः पायिताश्चामृतं सुराः
तद्दिदृक्षव आयाताः परं कौतूहलं हि नः
श्रीशुक उवाच
एवमभ्यर्थितो विष्णुर्भगवान्शूलपाणिना
प्रहस्य भावगम्भीरं गिरिशं प्रत्यभाषत
श्रीभगवानुवाच
कौतूहलाय दैत्यानां योषिद्वेषो मया धृतः
पश्यता सुरकार्याणि गते पीयूषभाजने
तत्तेऽहं दर्शयिष्यामि दिदृक्षोः सुरसत्तम
कामिनां बहु मन्तव्यं सङ्कल्पप्रभवोदयम्‌
श्रीशुक उवाच
इति ब्रुवाणो भगवांस्तत्रैवान्तरधीयत
सर्वतश्चारयंश्चक्षुर्भव आस्ते सहोमया
ततो ददर्शोपवने वरस्त्रियं विचित्रपुष्पारुणपल्लवद्रुमे
विक्रीडतीं कन्दुकलीलया लसद्दुकूलपर्यस्तनितम्बमेखलाम्‌
आवर्तनोद्वर्तनकम्पितस्तन प्रकृष्टहारोरुभरैः पदे पदे
प्रभज्यमानामिव मध्यतश्चलत्पदप्रवालं नयतीं ततस्ततः
दिक्षु भ्रमत्कन्दुकचापलैर्भृशं प्रोद्विग्नतारायतलोललोचनाम्‌
स्वकर्णविभ्राजितकुण्डलोल्लसत्कपोलनीलालकमण्डिताननाम्‌
श्लथद्दुकूलं कबरीं च विच्युतां सन्नह्यतीं वामकरेण वल्गुना
विनिघ्नतीमन्यकरेण कन्दुकं विमोहयन्तीं जगदात्ममायया
तां वीक्ष्य देव इति कन्दुकलीलयेषद्व्रीडास्फुटस्मितविसृष्टकटाक्षमुष्टः
स्त्रीप्रेक्षणप्रतिसमीक्षणविह्वलात्मा नात्मानमन्तिक उमां स्वगणांश्च वेद
तस्याः कराग्रात्स तु कन्दुको यदा गतो विदूरं तमनुव्रजत्स्त्रियाः
वासः ससूत्रं लघु मारुतोऽहरद्भवस्य देवस्य किलानुपश्यतः
एवं तां रुचिरापाङ्गीं दर्शनीयां मनोरमाम्‌
दृष्ट्वा तस्यां मनश्चक्रे विषज्जन्त्यां भवः किल
तयापहृतविज्ञानस्तत्कृतस्मरविह्वलः
भवान्या अपि पश्यन्त्या गतह्रीस्तत्पदं ययौ
सा तमायान्तमालोक्य विवस्त्रा व्रीडिता भृशम्‌
निलीयमाना वृक्षेषु हसन्ती नान्वतिष्ठत
तामन्वगच्छद्भगवान्भवः प्रमुषितेन्द्रियः
कामस्य च वशं नीतः करेणुमिव यूथपः
सोऽनुव्रज्यातिवेगेन गृहीत्वानिच्छतीं स्त्रियम्‌
केशबन्ध उपानीय बाहुभ्यां परिषस्वजे
सोपगूढा भगवता करिणा करिणी यथा
इतस्ततः प्रसर्पन्ती विप्रकीर्णशिरोरुहा
आत्मानं मोचयित्वाङ्ग सुरर्षभभुजान्तरात्‌
प्राद्रवत्सा पृथुश्रोणी माया देवविनिर्मिता
तस्यासौ पदवीं रुद्रो विष्णोरद्भुतकर्मणः
प्रत्यपद्यत कामेन वैरिणेव विनिर्जितः
तस्यानुधावतो रेतश्चस्कन्दामोघरेतसः
शुष्मिणो यूथपस्येव वासितामनुधावतः
यत्र यत्रापतन्मह्यां रेतस्तस्य महात्मनः
तानि रूप्यस्य हेम्नश्च क्षेत्राण्यासन्महीपते
सरित्सरःसु शैलेषु वनेषूपवनेषु च
यत्र क्व चासन्नृषयस्तत्र सन्निहितो हरः
स्कन्ने रेतसि सोऽपश्यदात्मानं देवमायया
जडीकृतं नृपश्रेष्ठ सन्न्यवर्तत कश्मलात्‌
अथावगतमाहात्म्य आत्मनो जगदात्मनः
अपरिज्ञेयवीर्यस्य न मेने तदु हाद्भुतम्‌
तमविक्लवमव्रीडमालक्ष्य मधुसूदनः
उवाच परमप्रीतो बिभ्रत्स्वां पौरुषीं तनुम्‌
श्रीभगवानुवाच
दिष्ट्या त्वं विबुधश्रेष्ठ स्वां निष्ठामात्मना स्थितः
यन्मे स्त्रीरूपया स्वैरं मोहितोऽप्यङ्ग मायया
को नु मेऽतितरेन्मायां विषक्तस्त्वदृते पुमान्‌
तांस्तान्विसृजतीं भावान्दुस्तरामकृतात्मभिः
सेयं गुणमयी माया न त्वामभिभविष्यति
मया समेता कालेन कालरूपेण भागशः
श्रीशुक उवाच
एवं भगवता राजन्श्रीवत्साङ्केन सत्कृतः
आमन्त्र्य तं परिक्रम्य सगणः स्वालयं ययौ
आत्मांशभूतां तां मायां भवानीं भगवान्भवः
सम्मतामृषिमुख्यानां प्रीत्याचष्टाथ भारत
अयि व्यपश्यस्त्वमजस्य मायां परस्य पुंसः परदेवतायाः
अहं कलानामृषभोऽपि मुह्ये ययावशोऽन्ये किमुतास्वतन्त्राः
यं मामपृच्छस्त्वमुपेत्य योगात्समासहस्रान्त उपारतं वै
स एष साक्षात्पुरुषः पुराणो न यत्र कालो विशते न वेदः
श्रीशुक उवाच
इति तेऽभिहितस्तात विक्रमः शार्ङ्गधन्वनः
सिन्धोर्निर्मथने येन धृतः पृष्ठे महाचलः
एतन्मुहुः कीर्तयतोऽनुशृण्वतो न रिष्यते जातु समुद्यमः क्वचित्‌
यदुत्तमश्लोकगुणानुवर्णनं समस्तसंसारपरिश्रमापहम्‌
असदविषयमङ्घ्रिं भावगम्यं प्रपन्नान्‌
अमृतममरवर्यानाशयत्सिन्धुमथ्यम्‌
कपटयुवतिवेषो मोहयन्यः सुरारींस्‌
तमहमुपसृतानां कामपूरं नतोऽस्मि
श्रीशुक उवाच
मनुर्विवस्वतः पुत्रः श्राद्धदेव इति श्रुतः
सप्तमो वर्तमानो यस्तदपत्यानि मे शृणु
इक्ष्वाकुर्नभगश्चैव धृष्टः शर्यातिरेव च
नरिष्यन्तोऽथ नाभागः सप्तमो दिष्ट उच्यते
तरूषश्च पृषध्रश्च दशमो वसुमान्स्मृतः
मनोर्वैवस्वतस्यैते दशपुत्राः परन्तप
आदित्या वसवो रुद्रा विश्वेदेवा मरुद्गणाः
अश्विनावृभवो राजन्निन्द्रस्तेषां पुरन्दरः
कश्यपोऽत्रिर्वसिष्ठश्च विश्वामित्रोऽथ गौतमः
जमदग्निर्भरद्वाज इति सप्तर्षयः स्मृताः
अत्रापि भगवज्जन्म कश्यपाददितेरभूत्‌
आदित्यानामवरजो विष्णुर्वामनरूपधृक्‌
सङ्क्षेपतो मयोक्तानि सप्तमन्वन्तराणि ते
भविष्याण्यथ वक्ष्यामि विष्णोः शक्त्यान्वितानि च
विवस्वतश्च द्वे जाये विश्वकर्मसुते उभे
संज्ञा छाया च राजेन्द्र ये प्रागभिहिते तव
तृतीयां वडवामेके तासां संज्ञासुतास्त्रयः
यमो यमी श्राद्धदेवश्छायायाश्च सुतान्छृणु
सावर्णिस्तपती कन्या भार्या संवरणस्य या
शनैश्चरस्तृतीयोऽभूदश्विनौ वडवात्मजौ
अष्टमेऽन्तर आयाते सावर्णिर्भविता मनुः
निर्मोकविरजस्काद्याः सावर्णितनया नृप
तत्र देवाः सुतपसो विरजा अमृतप्रभाः
तेषां विरोचनसुतो बलिरिन्द्रो भविष्यति
दत्त्वेमां याचमानाय विष्णवे यः पदत्रयम्‌
राद्धमिन्द्रपदं हित्वा ततः सिद्धिमवाप्स्यति
योऽसौ भगवता बद्धः प्रीतेन सुतले पुनः
निवेशितोऽधिके स्वर्गादधुनास्ते स्वराडिव
गालवो दीप्तिमान्रामो द्रोणपुत्रः कृपस्तथा
ऋष्यशृङ्गः पितास्माकं भगवान्बादरायणः
इमे सप्तर्षयस्तत्र भविष्यन्ति स्वयोगतः
इदानीमासते राजन्स्वे स्व आश्रममण्डले
देवगुह्यात्सरस्वत्यां सार्वभौम इति प्रभुः
स्थानं पुरन्दराद्धृत्वा बलये दास्यतीश्वरः
नवमो दक्षसावर्णिर्मनुर्वरुणसम्भवः
भूतकेतुर्दीप्तकेतुरित्याद्यास्तत्सुता नृप
पारामरीचिगर्भाद्या देवा इन्द्रोऽद्भुतः स्मृतः
द्युतिमत्प्रमुखास्तत्र भविष्यन्त्यृषयस्ततः
आयुष्मतोऽम्बुधारायामृषभो भगवत्कला
भविता येन संराद्धां त्रिलोकीं भोक्ष्यतेऽद्भुतः
दशमो ब्रह्मसावर्णिरुपश्लोकसुतो मनुः
तत्सुता भूरिषेणाद्या हविष्मत्प्रमुखा द्विजाः
हविष्मान्सुकृतः सत्यो जयो मूर्तिस्तदा द्विजाः
सुवासनविरुद्धाद्या देवाः शम्भुः सुरेश्वरः
विष्वक्सेनो विषूच्यां तु शम्भोः सख्यं करिष्यति
जातः स्वांशेन भगवान्गृहे विश्वसृजो विभुः
मनुर्वै धर्मसावर्णिरेकादशम आत्मवान्‌
अनागतास्तत्सुताश्च सत्यधर्मादयो दश
विहङ्गमाः कामगमा निर्वाणरुचयः सुराः
इन्द्रश्च वैधृतस्तेषामृषयश्चारुणादयः
आर्यकस्य सुतस्तत्र धर्मसेतुरिति स्मृतः
वैधृतायां हरेरंशस्त्रिलोकीं धारयिष्यति
भविता रुद्रसावर्णी राजन्द्वादशमो मनुः
देववानुपदेवश्च देवश्रेष्ठादयः सुताः
ऋतधामा च तत्रेन्द्रो देवाश्च हरितादयः
ऋषयश्च तपोमूर्तिस्तपस्व्याग्नीध्रकादयः
स्वधामाख्यो हरेरंशः साधयिष्यति तन्मनोः
अन्तरं सत्यसहसः सुनृतायाः सुतो विभुः
मनुस्त्रयोदशो भाव्यो देवसावर्णिरात्मवान्‌
चित्रसेनविचित्राद्या देवसावर्णिदेहजाः
देवाः सुकर्मसुत्राम संज्ञा इन्द्रो दिवस्पतिः
निर्मोकतत्त्वदर्शाद्या भविष्यन्त्यृषयस्तदा
देवहोत्रस्य तनय उपहर्ता दिवस्पतेः
योगेश्वरो हरेरंशो बृहत्यां सम्भविष्यति
मनुर्वा इन्द्रसावर्णिश्चतुर्दशम एष्यति
उरुगम्भीरबुधाद्या इन्द्रसावर्णिवीर्यजाः
पवित्राश्चाक्षुषा देवाः शुचिरिन्द्रो भविष्यति
अग्निर्बाहुः शुचिः शुद्धो मागधाद्यास्तपस्विनः
सत्रायणस्य तनयो बृहद्भानुस्तदा हरिः
वितानायां महाराज क्रियातन्तून्वितायिता
राजंश्चतुर्दशैतानि त्रिकालानुगतानि ते
प्रोक्तान्येभिर्मितः कल्पो युगसाहस्रपर्ययः
श्रीराजोवाच
मन्वन्तरेषु भगवन्यथा मन्वादयस्त्विमे
यस्मिन्कर्मणि ये येन नियुक्तास्तद्वदस्व मे
श्रीऋषिरुवाच
मनवो मनुपुत्राश्च मुनयश्च महीपते
इन्द्राः सुरगणाश्चैव सर्वे पुरुषशासनाः
यज्ञादयो याः कथिताः पौरुष्यस्तनवो नृप
मन्वादयो जगद्यात्रां नयन्त्याभिः प्रचोदिताः
चतुर्युगान्ते कालेन ग्रस्तान्छ्रुतिगणान्यथा
तपसा ऋषयोऽपश्यन्यतो धर्मः सनातनः
ततो धर्मं चतुष्पादं मनवो हरिणोदिताः
युक्ताः सञ्चारयन्त्यद्धा स्वे स्वे काले महीं नृप
पालयन्ति प्रजापाला यावदन्तं विभागशः
यज्ञभागभुजो देवा ये च तत्रान्विताश्च तैः
इन्द्रो भगवता दत्तां त्रैलोक्यश्रियमूर्जिताम्‌
भुञ्जानः पाति लोकांस्त्रीन्कामं लोके प्रवर्षति
ज्ञानं चानुयुगं ब्रूते हरिः सिद्धस्वरूपधृक्‌
ऋषिरूपधरः कर्म योगं योगेशरूपधृक्‌
सर्गं प्रजेशरूपेण दस्यून्हन्यात्स्वराड्वपुः
कालरूपेण सर्वेषामभावाय पृथग्गुणः
स्तूयमानो जनैरेभिर्मायया नामरूपया
विमोहितात्मभिर्नाना दर्शनैर्न च दृश्यते
एतत्कल्पविकल्पस्य प्रमाणं परिकीर्तितम्‌
यत्र मन्वन्तराण्याहुश्चतुर्दश पुराविदः
श्रीराजोवाच
बलेः पदत्रयं भूमेः कस्माद्धरिरयाचत
भूतेश्वरः कृपणवल्लब्धार्थोऽपि बबन्ध तम्‌
एतद्वेदितुमिच्छामो महत्कौतूहलं हि नः
यज्ञेश्वरस्य पूर्णस्य बन्धनं चाप्यनागसः
श्रीशुक उवाच
पराजितश्रीरसुभिश्च हापितो हीन्द्रेण राजन्भृगुभिः स जीवितः
सर्वात्मना तानभजद्भृगून्बलिः शिष्यो महात्मार्थनिवेदनेन
तं ब्राह्मणा भृगवः प्रीयमाणा अयाजयन्विश्वजिता त्रिणाकम्‌
जिगीषमाणं विधिनाभिषिच्य महाभिषेकेण महानुभावाः
ततो रथः काञ्चनपट्टनद्धो हयाश्च हर्यश्वतुरङ्गवर्णाः
ध्वजश्च सिंहेन विराजमानो हुताशनादास हविर्भिरिष्टात्‌
धनुश्च दिव्यं पुरटोपनद्धं तूणावरिक्तौ कवचं च दिव्यम्‌
पितामहस्तस्य ददौ च मालामम्लानपुष्पां जलजं च शुक्रः
एवं स विप्रार्जितयोधनार्थस्तैः कल्पितस्वस्त्ययनोऽथ विप्रान्‌
प्रदक्षिणीकृत्य कृतप्रणामः प्रह्रादमामन्त्र्य नमश्चकार
अथारुह्य रथं दिव्यं भृगुदत्तं महारथः
सुस्रग्धरोऽथ सन्नह्य धन्वी खड्गी धृतेषुधिः
हेमाङ्गदलसद्बाहुः स्फुरन्मकरकुण्डलः
रराज रथमारूढो धिष्ण्यस्थ इव हव्यवाट्‌
तुल्यैश्वर्यबलश्रीभिः स्वयूथैर्दैत्ययूथपैः
पिबद्भिरिव खं दृग्भिर्दहद्भिः परिधीनिव
वृतो विकर्षन्महतीमासुरीं ध्वजिनीं विभुः
ययाविन्द्रपुरीं स्वृद्धां कम्पयन्निव रोदसी
रम्यामुपवनोद्यानैः श्रीमद्भिर्नन्दनादिभिः
कूजद्विहङ्गमिथुनैर्गायन्मत्तमधुव्रतैः
प्रवालफलपुष्पोरु भारशाखामरद्रुमैः
हंससारसचक्राह्व कारण्डवकुलाकुलाः
नलिन्यो यत्र क्रीडन्ति प्रमदाः सुरसेविताः
आकाशगङ्गया देव्या वृतां परिखभूतया
प्राकारेणाग्निवर्णेन साट्टालेनोन्नतेन च
रुक्मपट्टकपाटैश्च द्वारैः स्फटिकगोपुरैः
जुष्टां विभक्तप्रपथां विश्वकर्मविनिर्मिताम्‌
सभाचत्वररथ्याढ्यां विमानैर्न्यर्बुदैर्युताम्‌
शृङ्गाटकैर्मणिमयैर्वज्रविद्रुमवेदिभिः
यत्र नित्यवयोरूपाः श्यामा विरजवाससः
भ्राजन्ते रूपवन्नार्यो ह्यर्चिर्भिरिव वह्नयः
सुरस्त्रीकेशविभ्रष्ट नवसौगन्धिकस्रजाम्‌
यत्रामोदमुपादाय मार्ग आवाति मारुतः
हेमजालाक्षनिर्गच्छद्धूमेनागुरुगन्धिना
पाण्डुरेण प्रतिच्छन्न मार्गे यान्ति सुरप्रियाः
मुक्तावितानैर्मणिहेमकेतुभिर्नानापताकावलभीभिरावृताम्‌
शिखण्डिपारावतभृङ्गनादितां वैमानिकस्त्रीकलगीतमङ्गलाम्‌
मृदङ्गशङ्खानकदुन्दुभिस्वनैः सतालवीणामुरजेष्टवेणुभिः
नृत्यैः सवाद्यैरुपदेवगीतकैर्मनोरमां स्वप्रभया जितप्रभाम्‌
यां न व्रजन्त्यधर्मिष्ठाः खला भूतद्रुहः शठाः
मानिनः कामिनो लुब्धा एभिर्हीना व्रजन्ति यत्‌
तां देवधानीं स वरूथिनीपतिर्बहिः समन्ताद्रुरुधे पृतन्यया
आचार्यदत्तं जलजं महास्वनं दध्मौ प्रयुञ्जन्भयमिन्द्रयोषिताम्‌
मघवांस्तमभिप्रेत्य बलेः परममुद्यमम्‌
सर्वदेवगणोपेतो गुरुमेतदुवाच ह
भगवन्नुद्यमो भूयान्बलेर्नः पूर्ववैरिणः
अविषह्यमिमं मन्ये केनासीत्तेजसोर्जितः
नैनं कश्चित्कुतो वापि प्रतिव्योढुमधीश्वरः
पिबन्निव मुखेनेदं लिहन्निव दिशो दश
दहन्निव दिशो दृग्भिः संवर्ताग्निरिवोत्थितः
ब्रूहि कारणमेतस्य दुर्धर्षत्वस्य मद्रिपोः
ओजः सहो बलं तेजो यत एतत्समुद्यमः
श्रीगुरुरुवाच
जानामि मघवन्छत्रोरुन्नतेरस्य कारणम्‌
शिष्यायोपभृतं तेजो भृगुभिर्ब्रह्मवादिभिः
ओजस्विनं बलिं जेतुं न समर्थोऽस्ति कश्चन
भवद्विधो भवान्वापि वर्जयित्वेश्वरं हरिम्‌
विजेष्यति न कोऽप्येनं ब्रह्मतेजःसमेधितम्‌
नास्य शक्तः पुरः स्थातुं कृतान्तस्य यथा जनाः
तस्मान्निलयमुत्सृज्य यूयं सर्वे त्रिविष्टपम्‌
यात कालं प्रतीक्षन्तो यतः शत्रोर्विपर्ययः
एष विप्रबलोदर्कः सम्प्रत्यूर्जितविक्रमः
तेषामेवापमानेन सानुबन्धो विनङ्क्ष्यति
एवं सुमन्त्रितार्थास्ते गुरुणार्थानुदर्शिना
हित्वा त्रिविष्टपं जग्मुर्गीर्वाणाः कामरूपिणः
देवेष्वथ निलीनेषु बलिर्वैरोचनः पुरीम्‌
देवधानीमधिष्ठाय वशं निन्ये जगत्त्रयम्‌
तं विश्वजयिनं शिष्यं भृगवः शिष्यवत्सलाः
शतेन हयमेधानामनुव्रतमयाजयन्‌
ततस्तदनुभावेन भुवनत्रयविश्रुताम्‌
कीर्तिं दिक्षुवितन्वानः स रेज उडुराडिव
बुभुजे च श्रियं स्वृद्धां द्विजदेवोपलम्भिताम्‌
कृतकृत्यमिवात्मानं मन्यमानो महामनाः
श्रीशुक उवाच
एवं पुत्रेषु नष्टेषु देवमातादितिस्तदा
हृते त्रिविष्टपे दैत्यैः पर्यतप्यदनाथवत्‌
एकदा कश्यपस्तस्या आश्रमं भगवानगात्‌
निरुत्सवं निरानन्दं समाधेर्विरतश्चिरात्‌
स पत्नीं दीनवदनां कृतासनपरिग्रहः
सभाजितो यथान्यायमिदमाह कुरूद्वह
अप्यभद्रं न विप्राणां भद्रे लोकेऽधुनागतम्‌
न धर्मस्य न लोकस्य मृत्योश्छन्दानुवर्तिनः
अपि वाकुशलं किञ्चिद्गृहेषु गृहमेधिनि
धर्मस्यार्थस्य कामस्य यत्र योगो ह्ययोगिनाम्‌
अपि वातिथयोऽभ्येत्य कुटुम्बासक्तया त्वया
गृहादपूजिता याताः प्रत्युत्थानेन वा क्वचित्‌
गृहेषु येष्वतिथयो नार्चिताः सलिलैरपि
यदि निर्यान्ति ते नूनं फेरुराजगृहोपमाः
अप्यग्नयस्तु वेलायां न हुता हविषा सति
त्वयोद्विग्नधिया भद्रे प्रोषिते मयि कर्हिचित्‌
यत्पूजया कामदुघान्याति लोकान्गृहान्वितः
ब्राह्मणोऽग्निश्च वै विष्णोः सर्वदेवात्मनो मुखम्‌
अपि सर्वे कुशलिनस्तव पुत्रा मनस्विनि
लक्षयेऽस्वस्थमात्मानं भवत्या लक्षणैरहम्‌
श्रीअदितिरुवाच
भद्रं द्विजगवां ब्रह्मन्धर्मस्यास्य जनस्य च
त्रिवर्गस्य परं क्षेत्रं गृहमेधिन्गृहा इमे
अग्नयोऽतिथयो भृत्या भिक्षवो ये च लिप्सवः
सर्वं भगवतो ब्रह्मन्ननुध्यानान्न रिष्यति
को नु मे भगवन्कामो न सम्पद्येत मानसः
यस्या भवान्प्रजाध्यक्ष एवं धर्मान्प्रभाषते
तवैव मारीच मनःशरीरजाः प्रजा इमाः सत्त्वरजस्तमोजुषः
समो भवांस्तास्वसुरादिषु प्रभो तथापि भक्तं भजते महेश्वरः
तस्मादीश भजन्त्या मे श्रेयश्चिन्तय सुव्रत
हृतश्रियो हृतस्थानान्सपत्नैः पाहि नः प्रभो
परैर्विवासिता साहं मग्ना व्यसनसागरे
ऐश्वर्यं श्रीर्यशः स्थानं हृतानि प्रबलैर्मम
यथा तानि पुनः साधो प्रपद्येरन्ममात्मजाः
तथा विधेहि कल्याणं धिया कल्याणकृत्तम
श्रीशुक उवाच
एवमभ्यर्थितोऽदित्या कस्तामाह स्मयन्निव
अहो मायाबलं विष्णोः स्नेहबद्धमिदं जगत्‌
क्व देहो भौतिकोऽनात्मा क्व चात्मा प्रकृतेः परः
कस्य के पतिपुत्राद्या मोह एव हि कारणम्‌
उपतिष्ठस्व पुरुषं भगवन्तं जनार्दनम्‌
सर्वभूतगुहावासं वासुदेवं जगद्गुरुम्‌
स विधास्यति ते कामान्हरिर्दीनानुकम्पनः
अमोघा भगवद्भक्तिर्नेतरेति मतिर्मम
श्रीअदितिरुवाच
केनाहं विधिना ब्रह्मन्नुपस्थास्ये जगत्पतिम्‌
यथा मे सत्यसङ्कल्पो विदध्यात्स मनोरथम्‌
आदिश त्वं द्विजश्रेष्ठ विधिं तदुपधावनम्‌
आशु तुष्यति मे देवः सीदन्त्याः सह पुत्रकैः
श्रीकश्यप उवाच
एतन्मे भगवान्पृष्टः प्रजाकामस्य पद्मजः
यदाह ते प्रवक्ष्यामि व्रतं केशवतोषणम्‌
फाल्गुनस्यामले पक्षे द्वादशाहं पयोव्रतम्‌
अर्चयेदरविन्दाक्षं भक्त्या परमयान्वितः
सिनीवाल्यां मृदालिप्य स्नायात्क्रोडविदीर्णया
यदि लभ्येत वै स्रोतस्येतं मन्त्रमुदीरयेत्‌
त्वं देव्यादिवराहेण रसायाः स्थानमिच्छता
उद्धृतासि नमस्तुभ्यं पाप्मानं मे प्रणाशय
निर्वर्तितात्मनियमो देवमर्चेत्समाहितः
अर्चायां स्थण्डिले सूर्ये जले वह्नौ गुरावपि
नमस्तुभ्यं भगवते पुरुषाय महीयसे
सर्वभूतनिवासाय वासुदेवाय साक्षिणे
नमोऽव्यक्ताय सूक्ष्माय प्रधानपुरुषाय च
चतुर्विंशद्गुणज्ञाय गुणसङ्ख्यानहेतवे
नमो द्विशीर्ष्णे त्रिपदे चतुःशृङ्गाय तन्तवे
सप्तहस्ताय यज्ञाय त्रयीविद्यात्मने नमः
नमः शिवाय रुद्राय नमः शक्तिधराय च
सर्वविद्याधिपतये भूतानां पतये नमः
नमो हिरण्यगर्भाय प्राणाय जगदात्मने
योगैश्वर्यशरीराय नमस्ते योगहेतवे
नमस्त आदिदेवाय साक्षिभूताय ते नमः
नारायणाय ऋषये नराय हरये नमः
नमो मरकतश्याम वपुषेऽधिगतश्रिये
केशवाय नमस्तुभ्यं नमस्ते पीतवाससे
त्वं सर्ववरदः पुंसां वरेण्य वरदर्षभ
अतस्ते श्रेयसे धीराः पादरेणुमुपासते
अन्ववर्तन्त यं देवाः श्रीश्च तत्पादपद्मयोः
स्पृहयन्त इवामोदं भगवान्मे प्रसीदताम्‌
एतैर्मन्त्रैर्हृषीकेशमावाहनपुरस्कृतम्‌
अर्चयेच्छ्रद्धया युक्तः पाद्योपस्पर्शनादिभिः
अर्चित्वा गन्धमाल्याद्यैः पयसा स्नपयेद्विभुम्‌
वस्त्रोपवीताभरण पाद्योपस्पर्शनैस्ततः
गन्धधूपादिभिश्चार्चेद्द्वादशाक्षरविद्यया
शृतं पयसि नैवेद्यं शाल्यन्नं विभवे सति
ससर्पिः सगुडं दत्त्वा जुहुयान्मूलविद्यया
निवेदितं तद्भक्ताय दद्याद्भुञ्जीत वा स्वयम्‌
दत्त्वाचमनमर्चित्वा ताम्बूलं च निवेदयेत्‌
जपेदष्टोत्तरशतं स्तुवीत स्तुतिभिः प्रभुम्‌
कृत्वा प्रदक्षिणं भूमौ प्रणमेद्दण्डवन्मुदा
कृत्वा शिरसि तच्छेषां देवमुद्वासयेत्ततः
द्व्यवरान्भोजयेद्विप्रान्पायसेन यथोचितम्‌
भुञ्जीत तैरनुज्ञातः सेष्टः शेषं सभाजितैः
ब्रह्मचार्यथ तद्रात्र्यां श्वो भूते प्रथमेऽहनि
स्नातः शुचिर्यथोक्तेन विधिना सुसमाहितः
पयसा स्नापयित्वार्चेद्यावद्व्रतसमापनम्‌
पयोभक्षो व्रतमिदं चरेद्विष्ण्वर्चनादृतः
पूर्ववज्जुहुयादग्निं ब्राह्मणांश्चापि भोजयेत्‌
एवं त्वहरहः कुर्याद्द्वादशाहं पयोव्रतम्‌
हरेराराधनं होममर्हणं द्विजतर्पणम्‌
प्रतिपद्दिनमारभ्य यावच्छुक्लत्रयोदशीम्‌
ब्रह्मचर्यमधःस्वप्नं स्नानं त्रिषवणं चरेत्‌
वर्जयेदसदालापं भोगानुच्चावचांस्तथा
अहिंस्रः सर्वभूतानां वासुदेवपरायणः
त्रयोदश्यामथो विष्णोः स्नपनं पञ्चकैर्विभोः
कारयेच्छास्त्रदृष्टेन विधिना विधिकोविदैः
पूजां च महतीं कुर्याद्वित्तशाठ्यविवर्जितः
चरुं निरूप्य पयसि शिपिविष्टाय विष्णवे
सूक्तेन तेन पुरुषं यजेत सुसमाहितः
नैवेद्यं चातिगुणवद्दद्यात्पुरुषतुष्टिदम्‌
आचार्यं ज्ञानसम्पन्नं वस्त्राभरणधेनुभिः
तोषयेदृत्विजश्चैव तद्विद्ध्याराधनं हरेः
भोजयेत्तान्गुणवता सदन्नेन शुचिस्मिते
अन्यांश्च ब्राह्मणान्छक्त्या ये च तत्र समागताः
दक्षिणां गुरवे दद्यादृत्विग्भ्यश्च यथार्हतः
अन्नाद्येनाश्वपाकांश्च प्रीणयेत्समुपागतान्‌
भुक्तवत्सु च सर्वेषु दीनान्धकृपणादिषु
विष्णोस्तत्प्रीणनं विद्वान्भुञ्जीत सह बन्धुभिः
नृत्यवादित्रगीतैश्च स्तुतिभिः स्वस्तिवाचकैः
कारयेत्तत्कथाभिश्च पूजां भगवतोऽन्वहम्‌
एतत्पयोव्रतं नाम पुरुषाराधनं परम्‌
पितामहेनाभिहितं मया ते समुदाहृतम्‌
त्वं चानेन महाभागे सम्यक्चीर्णेन केशवम्‌
आत्मना शुद्धभावेन नियतात्मा भजाव्ययम्‌
अयं वै सर्वयज्ञाख्यः सर्वव्रतमिति स्मृतम्‌
तपःसारमिदं भद्रे दानं चेश्वरतर्पणम्‌
त एव नियमाः साक्षात्त एव च यमोत्तमाः
तपो दानं व्रतं यज्ञो येन तुष्यत्यधोक्षजः
तस्मादेतद्व्रतं भद्रे प्रयता श्रद्धयाचर
भगवान्परितुष्टस्ते वरानाशु विधास्यति
श्रीशुक उवाच
इत्युक्ता सादिती राजन्स्वभर्त्रा कश्यपेन वै
अन्वतिष्ठद्व्रतमिदं द्वादशाहमतन्द्रिता
चिन्तयन्त्येकया बुद्ध्या महापुरुषमीश्वरम्‌
प्रगृह्येन्द्रियदुष्टाश्वान्मनसा बुद्धिसारथिः
मनश्चैकाग्रया बुद्ध्या भगवत्यखिलात्मनि
वासुदेवे समाधाय चचार ह पयोव्रतम्‌
तस्याः प्रादुरभूत्तात भगवानादिपुरुषः
पीतवासाश्चतुर्बाहुः शङ्खचक्रगदाधरः
तं नेत्रगोचरं वीक्ष्य सहसोत्थाय सादरम्‌
ननाम भुवि कायेन दण्डवत्प्रीतिविह्वला
सोत्थाय बद्धाञ्जलिरीडितुं स्थिता नोत्सेह आनन्दजलाकुलेक्षणा
बभूव तूष्णीं पुलकाकुलाकृतिस्तद्दर्शनात्युत्सवगात्रवेपथुः
प्रीत्या शनैर्गद्गदया गिरा हरिं तुष्टाव सा देव्यदितिः कुरूद्वह
उद्वीक्षती सा पिबतीव चक्षुषा रमापतिं यज्ञपतिं जगत्पतिम्‌
श्रीअदितिरुवाच
यज्ञेश यज्ञपुरुषाच्युत तीर्थपाद
तीर्थश्रवः श्रवणमङ्गलनामधेय
आपन्नलोकवृजिनोपशमोदयाद्य
शं नः कृधीश भगवन्नसि दीननाथः
विश्वाय विश्वभवनस्थितिसंयमाय
स्वैरं गृहीतपुरुशक्तिगुणाय भूम्ने
स्वस्थाय शश्वदुपबृंहितपूर्णबोध
व्यापादितात्मतमसे हरये नमस्ते
आयुः परं वपुरभीष्टमतुल्यलक्ष्मीर्‌
द्योभूरसाः सकलयोगगुणास्त्रिवर्गः
ज्ञानं च केवलमनन्त भवन्ति तुष्टात्‌
त्वत्तो नृणां किमु सपत्नजयादिराशीः
श्रीशुक उवाच
अदित्यैवं स्तुतो राजन्भगवान्पुष्करेक्षणः
क्षेत्रज्ञः सर्वभूतानामिति होवाच भारत
श्रीभगवानुवाच
देवमातर्भवत्या मे विज्ञातं चिरकाङ्क्षितम्‌
यत्सपत्नैर्हृतश्रीणां च्यावितानां स्वधामतः
तान्विनिर्जित्य समरे दुर्मदानसुरर्षभान्‌
प्रतिलब्धजयश्रीभिः पुत्रैरिच्छस्युपासितुम्‌
इन्द्रज्येष्ठैः स्वतनयैर्हतानां युधि विद्विषाम्‌
स्त्रियो रुदन्तीरासाद्य द्रष्टुमिच्छसि दुःखिताः
आत्मजान्सुसमृद्धांस्त्वं प्रत्याहृतयशःश्रियः
नाकपृष्ठमधिष्ठाय क्रीडतो द्रष्टुमिच्छसि
प्रायोऽधुना तेऽसुरयूथनाथा अपारणीया इति देवि मे मतिः
यत्तेऽनुकूलेश्वरविप्रगुप्ता न विक्रमस्तत्र सुखं ददाति
अथाप्युपायो मम देवि चिन्त्यः सन्तोषितस्य व्रतचर्यया ते
ममार्चनं नार्हति गन्तुमन्यथा श्रद्धानुरूपं फलहेतुकत्वात्‌
त्वयार्चितश्चाहमपत्यगुप्तये पयोव्रतेनानुगुणं समीडितः
स्वांशेन पुत्रत्वमुपेत्य ते सुतान्गोप्तास्मि मारीचतपस्यधिष्ठितः
उपधाव पतिं भद्रे प्रजापतिमकल्मषम्‌
मां च भावयती पत्यावेवं रूपमवस्थितम्‌
नैतत्परस्मा आख्येयं पृष्टयापि कथञ्चन
सर्वं सम्पद्यते देवि देवगुह्यं सुसंवृतम्‌
श्रीशुक उवाच
एतावदुक्त्वा भगवांस्तत्रैवान्तरधीयत
अदितिर्दुर्लभं लब्ध्वा हरेर्जन्मात्मनि प्रभोः
उपाधावत्पतिं भक्त्या परया कृतकृत्यवत्‌
स वै समाधियोगेन कश्यपस्तदबुध्यत
प्रविष्टमात्मनि हरेरंशं ह्यवितथेक्षणः
सोऽदित्यां वीर्यमाधत्त तपसा चिरसम्भृतम्‌
अमाहितमना राजन्दारुण्यग्निं यथानिलः
अदितेर्धिष्ठितं गर्भं भगवन्तं सनातनम्‌
हिरण्यगर्भो विज्ञाय समीडे गुह्यनामभिः
श्रीब्रह्मोवाच
जयोरुगाय भगवन्नुरुक्रम नमोऽस्तु ते
नमो ब्रह्मण्यदेवाय त्रिगुणाय नमो नमः
नमस्ते पृश्निगर्भाय वेदगर्भाय वेधसे
त्रिनाभाय त्रिपृष्ठाय शिपिविष्टाय विष्णवे
त्वमादिरन्तो भुवनस्य मध्यमनन्तशक्तिं पुरुषं यमाहुः
कालो भवानाक्षिपतीश विश्वं स्रोतो यथान्तः पतितं गभीरम्‌
त्वं वै प्रजानां स्थिरजङ्गमानां प्रजापतीनामसि सम्भविष्णुः
दिवौकसां देव दिवश्च्युतानां परायणं नौरिव मज्जतोऽप्सु
श्रीशुक उवाच
इत्थं विरिञ्चस्तुतकर्मवीर्यः प्रादुर्बभूवामृतभूरदित्याम्‌
चतुर्भुजः शङ्खगदाब्जचक्रः पिशङ्गवासा नलिनायतेक्षणः
श्यामावदातो झषराजकुण्डल त्विषोल्लसच्छ्रीवदनाम्बुजः पुमान्‌
श्रीवत्सवक्षा बलयाङ्गदोल्लसत्किरीटकाञ्चीगुणचारुनूपुरः
मधुव्रातव्रतविघुष्टया स्वया विराजितः श्रीवनमालया हरिः
प्रजापतेर्वेश्मतमः स्वरोचिषा विनाशयन्कण्ठनिविष्टकौस्तुभः
दिशः प्रसेदुः सलिलाशयास्तदा प्रजाः प्रहृष्टा ऋतवो गुणान्विताः
द्यौरन्तरीक्षं क्षितिरग्निजिह्वा गावो द्विजाः सञ्जहृषुर्नगाश्च
श्रोणायां श्रवणद्वादश्यां मुहूर्तेऽभिजिति प्रभुः
सर्वे नक्षत्रताराद्याश्चक्रुस्तज्जन्म दक्षिणम्‌
द्वादश्यां सवितातिष्ठन्मध्यन्दिनगतो नृप
विजयानाम सा प्रोक्ता यस्यां जन्म विदुर्हरेः
शङ्खदुन्दुभयो नेदुर्मृदङ्गपणवानकाः
चित्रवादित्रतूर्याणां निर्घोषस्तुमुलोऽभवत्‌
प्रीताश्चाप्सरसोऽनृत्यन्गन्धर्वप्रवरा जगुः
तुष्टुवुर्मुनयो देवा मनवः पितरोऽग्नयः
सिद्धविद्याधरगणाः सकिम्पुरुषकिन्नराः
चारणा यक्षरक्षांसि सुपर्णा भुजगोत्तमाः
गायन्तोऽतिप्रशंसन्तो नृत्यन्तो विबुधानुगाः
अदित्या आश्रमपदं कुसुमैः समवाकिरन्‌
दृष्ट्वादितिस्तं निजगर्भसम्भवं परं पुमांसं मुदमाप विस्मिता
गृहीतदेहं निजयोगमायया प्रजापतिश्चाह जयेति विस्मितः
यत्तद्वपुर्भाति विभूषणायुधैरव्यक्तचिद्व्यक्तमधारयद्धरिः
बभूव तेनैव स वामनो वटुः सम्पश्यतोर्दिव्यगतिर्यथा नटः
तं वटुं वामनं दृष्ट्वा मोदमाना महर्षयः
कर्माणि कारयामासुः पुरस्कृत्य प्रजापतिम्‌
तस्योपनीयमानस्य सावित्रीं सविताब्रवीत्‌
बृहस्पतिर्ब्रह्मसूत्रं मेखलां कश्यपोऽददात्‌
ददौ कृष्णाजिनं भूमिर्दण्डं सोमो वनस्पतिः
कौपीनाच्छादनं माता द्यौश्छत्रं जगतः पतेः
कमण्डलुं वेदगर्भः कुशान्सप्तर्षयो ददुः
अक्षमालां महाराज सरस्वत्यव्ययात्मनः
तस्मा इत्युपनीताय यक्षराट्पात्रिकामदात्‌
भिक्षां भगवती साक्षादुमादादम्बिका सती
स ब्रह्मवर्चसेनैवं सभां सम्भावितो वटुः
ब्रह्मर्षिगणसञ्जुष्टामत्यरोचत मारिषः
समिद्धमाहितं वह्निं कृत्वा परिसमूहनम्‌
परिस्तीर्य समभ्यर्च्य समिद्भिरजुहोद्द्विजः
श्रुत्वाश्वमेधैर्यजमानमूर्जितं बलिं भृगूणामुपकल्पितैस्ततः
जगाम तत्राखिलसारसम्भृतो भारेण गां सन्नमयन्पदे पदे
तं नर्मदायास्तट उत्तरे बलेर्य ऋत्विजस्ते भृगुकच्छसंज्ञके
प्रवर्तयन्तो भृगवः क्रतूत्तमं व्यचक्षतारादुदितं यथा रविम्‌
ते ऋत्विजो यजमानः सदस्या हतत्विषो वामनतेजसा नृप
सूर्यः किलायात्युत वा विभावसुः सनत्कुमारोऽथ दिदृक्षया क्रतोः
इत्थं सशिष्येषु भृगुष्वनेकधा वितर्क्यमाणो भगवान्स वामनः
छत्रं सदण्डं सजलं कमण्डलुं विवेश बिभ्रद्धयमेधवाटम्‌
मौञ्ज्या मेखलया वीतमुपवीताजिनोत्तरम्‌
जटिलं वामनं विप्रं मायामाणवकं हरिम्‌
प्रविष्टं वीक्ष्य भृगवः सशिष्यास्ते सहाग्निभिः
प्रत्यगृह्णन्समुत्थाय सङ्क्षिप्तास्तस्य तेजसा
यजमानः प्रमुदितो दर्शनीयं मनोरमम्‌
रूपानुरूपावयवं तस्मा आसनमाहरत्‌
स्वागतेनाभिनन्द्याथ पादौ भगवतो बलिः
अवनिज्यार्चयामास मुक्तसङ्गमनोरमम्‌
तत्पादशौचं जनकल्मषापहं स धर्मविन्मूर्ध्न्यदधात्सुमङ्गलम्‌
यद्देवदेवो गिरिशश्चन्द्रमौलिर्दधार मूर्ध्ना परया च भक्त्या
श्रीबलिरुवाच
स्वागतं ते नमस्तुभ्यं ब्रह्मन्किं करवाम ते
ब्रह्मर्षीणां तपः साक्षान्मन्ये त्वार्य वपुर्धरम्‌
अद्य नः पितरस्तृप्ता अद्य नः पावितं कुलम्‌
अद्य स्विष्टः क्रतुरयं यद्भवानागतो गृहान्‌
अद्याग्नयो मे सुहुता यथाविधि द्विजात्मज त्वच्चरणावनेजनैः
हतांहसो वार्भिरियं च भूरहो तथा पुनीता तनुभिः पदैस्तव
यद्यद्वटो वाञ्छसि तत्प्रतीच्छ मे त्वामर्थिनं विप्रसुतानुतर्कये
गां काञ्चनं गुणवद्धाम मृष्टं तथान्नपेयमुत वा विप्रकन्याम्‌
ग्रामान्समृद्धांस्तुरगान्गजान्वा रथांस्तथार्हत्तम सम्प्रतीच्छ
श्रीशुक उवाच
इति वैरोचनेर्वाक्यं धर्मयुक्तं स सूनृतम्‌
निशम्य भगवान्प्रीतः प्रतिनन्द्येदमब्रवीत्‌
श्रीभगवानुवाच
वचस्तवैतज्जनदेव सूनृतं कुलोचितं धर्मयुतं यशस्करम्‌
यस्य प्रमाणं भृगवः साम्पराये पितामहः कुलवृद्धः प्रशान्तः
न ह्येतस्मिन्कुले कश्चिन्निःसत्त्वः कृपणः पुमान्‌
प्रत्याख्याता प्रतिश्रुत्य यो वादाता द्विजातये
न सन्ति तीर्थे युधि चार्थिनार्थिताः पराङ्मुखा ये त्वमनस्विनो नृप
युष्मत्कुले यद्यशसामलेन प्रह्राद उद्भाति यथोडुपः खे
यतो जातो हिरण्याक्षश्चरन्नेक इमां महीम्‌
प्रतिवीरं दिग्विजये नाविन्दत गदायुधः
यं विनिर्जित्य कृच्छ्रेण विष्णुः क्ष्मोद्धार आगतम्‌
आत्मानं जयिनं मेने तद्वीर्यं भूर्यनुस्मरन्‌
निशम्य तद्वधं भ्राता हिरण्यकशिपुः पुरा
हन्तुं भ्रातृहणं क्रुद्धो जगाम निलयं हरेः
तमायान्तं समालोक्य शूलपाणिं कृतान्तवत्‌
चिन्तयामास कालज्ञो विष्णुर्मायाविनां वरः
यतो यतोऽहं तत्रासौ मृत्युः प्राणभृतामिव
अतोऽहमस्य हृदयं प्रवेक्ष्यामि पराग्दृशः
एवं स निश्चित्य रिपोः शरीरमाधावतो निर्विविशेऽसुरेन्द्र
श्वासानिलान्तर्हितसूक्ष्मदेहस्तत्प्राणरन्ध्रेण विविग्नचेताः
स तन्निकेतं परिमृश्य शून्यमपश्यमानः कुपितो ननाद
क्ष्मां द्यां दिशः खं विवरान्समुद्रान्विष्णुं विचिन्वन्न ददर्श वीरः
अपश्यन्निति होवाच मयान्विष्टमिदं जगत्‌
भ्रातृहा मे गतो नूनं यतो नावर्तते पुमान्‌
वैरानुबन्ध एतावानामृत्योरिह देहिनाम्‌
अज्ञानप्रभवो मन्युरहंमानोपबृंहितः
पिता प्रह्रादपुत्रस्ते तद्विद्वान्द्विजवत्सलः
स्वमायुर्द्विजलिङ्गेभ्यो देवेभ्योऽदात्स याचितः
भवानाचरितान्धर्मानास्थितो गृहमेधिभिः
ब्राह्मणैः पूर्वजैः शूरैरन्यैश्चोद्दामकीर्तिभिः
तस्मात्त्वत्तो महीमीषद्वृणेऽहं वरदर्षभात्‌
पदानि त्रीणि दैत्येन्द्र सम्मितानि पदा मम
नान्यत्ते कामये राजन्वदान्याज्जगदीश्वरात्‌
नैनः प्राप्नोति वै विद्वान्यावदर्थप्रतिग्रहः
श्रीबलिरुवाच
अहो ब्राह्मणदायाद वाचस्ते वृद्धसम्मताः
त्वं बालो बालिशमतिः स्वार्थं प्रत्यबुधो यथा
मां वचोभिः समाराध्य लोकानामेकमीश्वरम्‌
पदत्रयं वृणीते योऽबुद्धिमान्द्वीपदाशुषम्‌
न पुमान्मामुपव्रज्य भूयो याचितुमर्हति
तस्माद्वृत्तिकरीं भूमिं वटो कामं प्रतीच्छ मे
श्रीभगवानुवाच
यावन्तो विषयाः प्रेष्ठास्त्रिलोक्यामजितेन्द्रियम्‌
न शक्नुवन्ति ते सर्वे प्रतिपूरयितुं नृप
त्रिभिः क्रमैरसन्तुष्टो द्वीपेनापि न पूर्यते
नववर्षसमेतेन सप्तद्वीपवरेच्छया
सप्तद्वीपाधिपतयो नृपा वैण्यगयादयः
अर्थैः कामैर्गता नान्तं तृष्णाया इति नः श्रुतम्‌
यदृच्छयोपपन्नेन सन्तुष्टो वर्तते सुखम्‌
नासन्तुष्टस्त्रिभिर्लोकैरजितात्मोपसादितैः
पुंसोऽयं संसृतेर्हेतुरसन्तोषोऽर्थकामयोः
यदृच्छयोपपन्नेन सन्तोषो मुक्तये स्मृतः
यदृच्छालाभतुष्टस्य तेजो विप्रस्य वर्धते
तत्प्रशाम्यत्यसन्तोषादम्भसेवाशुशुक्षणिः
तस्मात्त्रीणि पदान्येव वृणे त्वद्वरदर्षभात्‌
एतावतैव सिद्धोऽहं वित्तं यावत्प्रयोजनम्‌
श्रीशुक उवाच
इत्युक्तः स हसन्नाह वाञ्छातः प्रतिगृह्यताम्‌
वामनाय महीं दातुं जग्राह जलभाजनम्‌
विष्णवे क्ष्मां प्रदास्यन्तमुशना असुरेश्वरम्‌
जानंश्चिकीर्षितं विष्णोः शिष्यं प्राह विदां वरः
श्रीशुक्र उवाच
एष वैरोचने साक्षाद्भगवान्विष्णुरव्ययः
कश्यपाददितेर्जातो देवानां कार्यसाधकः
प्रतिश्रुतं त्वयैतस्मै यदनर्थमजानता
न साधु मन्ये दैत्यानां महानुपगतोऽनयः
एष ते स्थानमैश्वर्यं श्रियं तेजो यशः श्रुतम्‌
दास्यत्याच्छिद्य शक्राय मायामाणवको हरिः
त्रिभिः क्रमैरिमाल्लोकान्विश्वकायः क्रमिष्यति
सर्वस्वं विष्णवे दत्त्वा मूढ वर्तिष्यसे कथम्‌
क्रमतो गां पदैकेन द्वितीयेन दिवं विभोः
खं च कायेन महता तार्तीयस्य कुतो गतिः
निष्ठां ते नरके मन्ये ह्यप्रदातुः प्रतिश्रुतम्‌
प्रतिश्रुतस्य योऽनीशः प्रतिपादयितुं भवान्‌
न तद्दानं प्रशंसन्ति येन वृत्तिर्विपद्यते
दानं यज्ञस्तपः कर्म लोके वृत्तिमतो यतः
धर्माय यशसेऽर्थाय कामाय स्वजनाय च
पञ्चधा विभजन्वित्तमिहामुत्र च मोदते
अत्रापि बह्वृचैर्गीतं शृणु मेऽसुरसत्तम
सत्यमोमिति यत्प्रोक्तं यन्नेत्याहानृतं हि तत्‌
सत्यं पुष्पफलं विद्यादात्मवृक्षस्य गीयते
वृक्षेऽजीवति तन्न स्यादनृतं मूलमात्मनः
तद्यथा वृक्ष उन्मूलः शुष्यत्युद्वर्ततेऽचिरात्‌
एवं नष्टानृतः सद्य आत्मा शुष्येन्न संशयः
पराग्रिक्तमपूर्णं वा अक्षरं यत्तदोमिति
यत्किञ्चिदोमिति ब्रूयात्तेन रिच्येत वै पुमान्‌
भिक्षवे सर्वमों कुर्वन्नालं कामेन चात्मने
अथैतत्पूर्णमभ्यात्मं यच्च नेत्यनृतं वचः
सर्वं नेत्यनृतं ब्रूयात्स दुष्कीर्तिः श्वसन्मृतः
स्त्रीषु नर्मविवाहे च वृत्त्यर्थे प्राणसङ्कटे
गोब्राह्मणार्थे हिंसायां नानृतं स्याज्जुगुप्सितम्‌
श्रीशुक उवाच
बलिरेवं गृहपतिः कुलाचार्येण भाषितः
तूष्णीं भूत्वा क्षणं राजन्नुवाचावहितो गुरुम्‌
श्रीबलिरुवाच
सत्यं भगवता प्रोक्तं धर्मोऽयं गृहमेधिनाम्‌
अर्थं कामं यशो वृत्तिं यो न बाधेत कर्हिचित्‌
स चाहं वित्तलोभेन प्रत्याचक्षे कथं द्विजम्‌
प्रतिश्रुत्य ददामीति प्राह्रादिः कितवो यथा
न ह्यसत्यात्परोऽधर्म इति होवाच भूरियम्‌
सर्वं सोढुमलं मन्ये ऋतेऽलीकपरं नरम्‌
नाहं बिभेमि निरयान्नाधन्यादसुखार्णवात्‌
न स्थानच्यवनान्मृत्योर्यथा विप्रप्रलम्भनात्‌
यद्यद्धास्यति लोकेऽस्मिन्सम्परेतं धनादिकम्‌
तस्य त्यागे निमित्तं किं विप्रस्तुष्येन्न तेन चेत्‌
श्रेयः कुर्वन्ति भूतानां साधवो दुस्त्यजासुभिः
दध्यङ्शिबिप्रभृतयः को विकल्पो धरादिषु
यैरियं बुभुजे ब्रह्मन्दैत्येन्द्रैरनिवर्तिभिः
तेषां कालोऽग्रसील्लोकान्न यशोऽधिगतं भुवि
सुलभा युधि विप्रर्षे ह्यनिवृत्तास्तनुत्यजः
न तथा तीर्थ आयाते श्रद्धया ये धनत्यजः
मनस्विनः कारुणिकस्य शोभनं यदर्थिकामोपनयेन दुर्गतिः
कुतः पुनर्ब्रह्मविदां भवादृशां ततो वटोरस्य ददामि वाञ्छितम्‌
यजन्ति यज्ञं क्रतुभिर्यमादृता भवन्त आम्नायविधानकोविदाः
स एव विष्णुर्वरदोऽस्तु वा परो दास्याम्यमुष्मै क्षितिमीप्सितां मुने
यद्यप्यसावधर्मेण मां बध्नीयादनागसम्‌
तथाप्येनं न हिंसिष्ये भीतं ब्रह्मतनुं रिपुम्‌
एष वा उत्तमश्लोको न जिहासति यद्यशः
हत्वा मैनां हरेद्युद्धे शयीत निहतो मया
श्रीशुक उवाच
एवमश्रद्धितं शिष्यमनादेशकरं गुरुः
शशाप दैवप्रहितः सत्यसन्धं मनस्विनम्‌
दृढं पण्डितमान्यज्ञः स्तब्धोऽस्यस्मदुपेक्षया
मच्छासनातिगो यस्त्वमचिराद्भ्रश्यसे श्रियः
एवं शप्तः स्वगुरुणा सत्यान्न चलितो महान्‌
वामनाय ददावेनामर्चित्वोदकपूर्वकम्‌
विन्ध्यावलिस्तदागत्य पत्नी जालकमालिनी
आनिन्ये कलशं हैममवनेजन्यपां भृतम्‌
यजमानः स्वयं तस्य श्रीमत्पादयुगं मुदा
अवनिज्यावहन्मूर्ध्नि तदपो विश्वपावनीः
तदासुरेन्द्रं दिवि देवतागणा गन्धर्वविद्याधरसिद्धचारणाः
तत्कर्म सर्वेऽपि गृणन्त आर्जवं प्रसूनवर्षैर्ववृषुर्मुदान्विताः
नेदुर्मुहुर्दुन्दुभयः सहस्रशो गन्धर्वकिम्पूरुषकिन्नरा जगुः
मनस्विनानेन कृतं सुदुष्करं विद्वानदाद्यद्रिपवे जगत्त्रयम्‌
तद्वामनं रूपमवर्धताद्भुतं हरेरनन्तस्य गुणत्रयात्मकम्‌
भूः खं दिशो द्यौर्विवराः पयोधयस्तिर्यङ्नृदेवा ऋषयो यदासत
काये बलिस्तस्य महाविभूतेः सहर्त्विगाचार्यसदस्य एतत्‌
ददर्श विश्वं त्रिगुणं गुणात्मके भूतेन्द्रियार्थाशयजीवयुक्तम्‌
रसामचष्टाङ्घ्रितलेऽथ पादयोर्महीं महीध्रान्पुरुषस्य जङ्घयोः
पतत्त्रिणो जानुनि विश्वमूर्तेरूर्वोर्गणं मारुतमिन्द्रसेनः
सन्ध्यां विभोर्वाससि गुह्य ऐक्षत्प्रजापतीन्जघने आत्ममुख्यान्‌
नाभ्यां नभः कुक्षिषु सप्तसिन्धूनुरुक्रमस्योरसि चर्क्षमालाम्‌
हृद्यङ्ग धर्मं स्तनयोर्मुरारेरृतं च सत्यं च मनस्यथेन्दुम्‌
श्रियं च वक्षस्यरविन्दहस्तां कण्ठे च सामानि समस्तरेफान्‌
इन्द्रप्रधानानमरान्भुजेषु तत्कर्णयोः ककुभो द्यौश्च मूर्ध्नि
केशेषु मेघान्छ्वसनं नासिकायामक्ष्णोश्च सूर्यं वदने च वह्निम्‌
वाण्यां च छन्दांसि रसे जलेशं भ्रुवोर्निषेधं च विधिं च पक्ष्मसु
अहश्च रात्रिं च परस्य पुंसो मन्युं ललाटेऽधर एव लोभम्‌
स्पर्शे च कामं नृप रेतसाम्भः पृष्ठे त्वधर्मं क्रमणेषु यज्ञम्‌
छायासु मृत्युं हसिते च मायां तनूरुहेष्वोषधिजातयश्च
नदीश्च नाडीषु शिला नखेषु बुद्धावजं देवगणानृषींश्च
प्राणेषु गात्रे स्थिरजङ्गमानि सर्वाणि भूतानि ददर्श वीरः
सर्वात्मनीदं भुवनं निरीक्ष्य सर्वेऽसुराः कश्मलमापुरङ्ग
सुदर्शनं चक्रमसह्यतेजो धनुश्च शार्ङ्गं स्तनयित्नुघोषम्‌
पर्जन्यघोषो जलजः पाञ्चजन्यः कौमोदकी विष्णुगदा तरस्विनी
विद्याधरोऽसिः शतचन्द्रयुक्तस्तूणोत्तमावक्षयसायकौ च
सुनन्दमुख्या उपतस्थुरीशं पार्षदमुख्याः सहलोकपालाः
स्फुरत्किरीटाङ्गदमीनकुण्डलः श्रीवत्सरत्नोत्तममेखलाम्बरैः
मधुव्रतस्रग्वनमालयावृतो रराज राजन्भगवानुरुक्रमः
क्षितिं पदैकेन बलेर्विचक्रमे नभः शरीरेण दिशश्च बाहुभिः
पदं द्वितीयं क्रमतस्त्रिविष्टपं न वै तृतीयाय तदीयमण्वपि
उरुक्रमस्याङ्घ्रिरुपर्युपर्यथो महर्जनाभ्यां तपसः परं गतः
श्रीशुक उवाच
सत्यं समीक्ष्याब्जभवो नखेन्दुभिर्हतस्वधामद्युतिरावृतोऽभ्यगात्‌
मरीचिमिश्रा ऋषयो बृहद्व्रताः सनन्दनाद्या नरदेव योगिनः
वेदोपवेदा नियमा यमान्वितास्तर्केतिहासाङ्गपुराणसंहिताः
ये चापरे योगसमीरदीपित ज्ञानाग्निना रन्धितकर्मकल्मषाः
ववन्दिरे यत्स्मरणानुभावतः स्वायम्भुवं धाम गता अकर्मकम्‌
अथाङ्घ्रये प्रोन्नमिताय विष्णोरुपाहरत्पद्मभवोऽर्हणोदकम्‌
समर्च्य भक्त्याभ्यगृणाच्छुचिश्रवा यन्नाभिपङ्केरुहसम्भवः स्वयम्‌
धातुः कमण्डलुजलं तदुरुक्रमस्य पादावनेजनपवित्रतया नरेन्द्र
स्वर्धुन्यभून्नभसि सा पतती निमार्ष्टि लोकत्रयं भगवतो विशदेव कीर्तिः
ब्रह्मादयो लोकनाथाः स्वनाथाय समादृताः
सानुगा बलिमाजह्रुः सङ्क्षिप्तात्मविभूतये
तोयैः समर्हणैः स्रग्भिर्दिव्यगन्धानुलेपनैः
धूपैर्दीपैः सुरभिभिर्लाजाक्षतफलाङ्कुरैः
स्तवनैर्जयशब्दैश्च तद्वीर्यमहिमाङ्कितैः
नृत्यवादित्रगीतैश्च शङ्खदुन्दुभिनिःस्वनैः
जाम्बवानृक्षराजस्तु भेरीशब्दैर्मनोजवः
विजयं दिक्षु सर्वासु महोत्सवमघोषयत्‌
महीं सर्वां हृतां दृष्ट्वा त्रिपदव्याजयाच्ञया
ऊचुः स्वभर्तुरसुरा दीक्षितस्यात्यमर्षिताः
न वायं ब्रह्मबन्धुर्विष्णुर्मायाविनां वरः
द्विजरूपप्रतिच्छन्नो देवकार्यं चिकीर्षति
अनेन याचमानेन शत्रुणा वटुरूपिणा
सर्वस्वं नो हृतं भर्तुर्न्यस्तदण्डस्य बर्हिषि
सत्यव्रतस्य सततं दीक्षितस्य विशेषतः
नानृतं भाषितुं शक्यं ब्रह्मण्यस्य दयावतः
तस्मादस्य वधो धर्मो भर्तुः शुश्रूषणं च नः
इत्यायुधानि जगृहुर्बलेरनुचरासुराः
ते सर्वे वामनं हन्तुं शूलपट्टिशपाणयः
अनिच्छन्तो बले राजन्प्राद्रवन्जातमन्यवः
तानभिद्रवतो दृष्ट्वा दितिजानीकपान्नृप
प्रहस्यानुचरा विष्णोः प्रत्यषेधन्नुदायुधाः
नन्दः सुनन्दोऽथ जयो विजयः प्रबलो बलः
कुमुदः कुमुदाक्षश्च विष्वक्सेनः पतत्त्रिराट्‌
जयन्तः श्रुतदेवश्च पुष्पदन्तोऽथ सात्वतः
सर्वे नागायुतप्राणाश्चमूं ते जघ्नुरासुरीम्‌
हन्यमानान्स्वकान्दृष्ट्वा पुरुषानुचरैर्बलिः
वारयामास संरब्धान्काव्यशापमनुस्मरन्‌
हे विप्रचित्ते हे राहो हे नेमे श्रूयतां वचः
मा युध्यत निवर्तध्वं न नः कालोऽयमर्थकृत्‌
यः प्रभुः सर्वभूतानां सुखदुःखोपपत्तये
तं नातिवर्तितुं दैत्याः पौरुषैरीश्वरः पुमान्‌
यो नो भवाय प्रागासीदभवाय दिवौकसाम्‌
स एव भगवानद्य वर्तते तद्विपर्ययम्‌
बलेन सचिवैर्बुद्ध्या दुर्गैर्मन्त्रौषधादिभिः
सामादिभिरुपायैश्च कालं नात्येति वै जनः
भवद्भिर्निर्जिता ह्येते बहुशोऽनुचरा हरेः
दैवेनर्द्धैस्त एवाद्य युधि जित्वा नदन्ति नः
एतान्वयं विजेष्यामो यदि दैवं प्रसीदति
तस्मात्कालं प्रतीक्षध्वं यो नोऽर्थत्वाय कल्पते
श्रीशुक उवाच
पत्युर्निगदितं श्रुत्वा दैत्यदानवयूथपाः
रसां निर्विविशू राजन्विष्णुपार्षद ताडिताः
अथ तार्क्ष्यसुतो ज्ञात्वा विराट्प्रभुचिकीर्षितम्‌
बबन्ध वारुणैः पाशैर्बलिं सूत्येऽहनि क्रतौ
हाहाकारो महानासीद्रोदस्योः सर्वतो दिशम्‌
निगृह्यमाणेऽसुरपतौ विष्णुना प्रभविष्णुना
तं बद्धं वारुणैः पाशैर्भगवानाह वामनः
नष्टश्रियं स्थिरप्रज्ञमुदारयशसं नृप
पदानि त्रीणि दत्तानि भूमेर्मह्यं त्वयासुर
द्वाभ्यां क्रान्ता मही सर्वा तृतीयमुपकल्पय
यावत्तपत्यसौ गोभिर्यावदिन्दुः सहोडुभिः
यावद्वर्षति पर्जन्यस्तावती भूरियं तव
पदैकेन मयाक्रान्तो भूर्लोकः खं दिशस्तनोः
स्वर्लोकस्ते द्वितीयेन पश्यतस्ते स्वमात्मना
प्रतिश्रुतमदातुस्ते निरये वास इष्यते
विश त्वं निरयं तस्माद्गुरुणा चानुमोदितः
वृथा मनोरथस्तस्य दूरः स्वर्गः पतत्यधः
प्रतिश्रुतस्यादानेन योऽर्थिनं विप्रलम्भते
विप्रलब्धो ददामीति त्वयाहं चाढ्यमानिना
तद्व्यलीकफलं भुङ्क्ष्व निरयं कतिचित्समाः
श्रीशुक उवाच
एवं विप्रकृतो राजन्बलिर्भगवतासुरः
भिद्यमानोऽप्यभिन्नात्मा प्रत्याहाविक्लवं वचः
श्रीबलिरुवाच
यद्युत्तमश्लोक भवान्ममेरितं वचो व्यलीकं सुरवर्य मन्यते
करोम्यृतं तन्न भवेत्प्रलम्भनं पदं तृतीयं कुरु शीर्ष्णि मे निजम्‌
बिभेमि नाहं निरयात्पदच्युतो न पाशबन्धाद्व्यसनाद्दुरत्ययात्‌
नैवार्थकृच्छ्राद्भवतो विनिग्रहादसाधुवादाद्भृशमुद्विजे यथा
पुंसां श्लाघ्यतमं मन्ये दण्डमर्हत्तमार्पितम्‌
यं न माता पिता भ्राता सुहृदश्चादिशन्ति हि
त्वं नूनमसुराणां नः परोक्षः परमो गुरुः
यो नोऽनेकमदान्धानां विभ्रंशं चक्षुरादिशत्‌
यस्मिन्वैरानुबन्धेन व्यूढेन विबुधेतराः
बहवो लेभिरे सिद्धिं यामु हैकान्तयोगिनः
तेनाहं निगृहीतोऽस्मि भवता भूरिकर्मणा
बद्धश्च वारुणैः पाशैर्नातिव्रीडे न च व्यथे
पितामहो मे भवदीयसम्मतः प्रह्राद आविष्कृतसाधुवादः
भवद्विपक्षेण विचित्रवैशसं सम्प्रापितस्त्वं परमः स्वपित्रा
किमात्मनानेन जहाति योऽन्ततः किं रिक्थहारैः स्वजनाख्यदस्युभिः
किं जायया संसृतिहेतुभूतया मर्त्यस्य गेहैः किमिहायुषो व्ययः
इत्थं स निश्चित्य पितामहो महानगाधबोधो भवतः पादपद्मम्‌
ध्रुवं प्रपेदे ह्यकुतोभयं जनाद्भीतः स्वपक्षक्षपणस्य सत्तम
अथाहमप्यात्मरिपोस्तवान्तिकं दैवेन नीतः प्रसभं त्याजितश्रीः
इदं कृतान्तान्तिकवर्ति जीवितं ययाध्रुवं स्तब्धमतिर्न बुध्यते
श्रीशुक उवाच
तस्येत्थं भाषमाणस्य प्रह्रादो भगवत्प्रियः
आजगाम कुरुश्रेष्ठ राकापतिरिवोत्थितः
तमिन्द्रसेनः स्वपितामहं श्रिया विराजमानं नलिनायतेक्षणम्‌
प्रांशुं पिशङ्गाम्बरमञ्जनत्विषं प्रलम्बबाहुं शुभगर्षभमैक्षत
तस्मै बलिर्वारुणपाशयन्त्रितः समर्हणं नोपजहार पूर्ववत्‌
ननाम मूर्ध्नाश्रुविलोललोचनः सव्रीडनीचीनमुखो बभूव ह
स तत्र हासीनमुदीक्ष्य सत्पतिं हरिं सुनन्दाद्यनुगैरुपासितम्‌
उपेत्य भूमौ शिरसा महामना ननाम मूर्ध्ना पुलकाश्रुविक्लवः
श्रीप्रह्राद उवाच
त्वयैव दत्तं पदमैन्द्रमूर्जितं हृतं तदेवाद्य तथैव शोभनम्‌
मन्ये महानस्य कृतो ह्यनुग्रहो विभ्रंशितो यच्छ्रिय आत्ममोहनात्‌
यया हि विद्वानपि मुह्यते यतस्तत्को विचष्टे गतिमात्मनो यथा
तस्मै नमस्ते जगदीश्वराय वै नारायणायाखिललोकसाक्षिणे
श्रीशुक उवाच
तस्यानुशृण्वतो राजन्प्रह्रादस्य कृताञ्जलेः
हिरण्यगर्भो भगवानुवाच मधुसूदनम्‌
बद्धं वीक्ष्य पतिं साध्वी तत्पत्नी भयविह्वला
प्राञ्जलिः प्रणतोपेन्द्रं बभाषेऽवाङ्मुखी नृप
श्रीविन्ध्यावलिरुवाच
क्रीडार्थमात्मन इदं त्रिजगत्कृतं ते स्वाम्यं तु तत्र कुधियोऽपर ईश कुर्युः
कर्तुः प्रभोस्तव किमस्यत आवहन्ति त्यक्तह्रियस्त्वदवरोपितकर्तृवादाः
श्रीब्रह्मोवाच
भूतभावन भूतेश देवदेव जगन्मय
मुञ्चैनं हृतसर्वस्वं नायमर्हति निग्रहम्‌
कृत्स्ना तेऽनेन दत्ता भूर्लोकाः कर्मार्जिताश्च ये
निवेदितं च सर्वस्वमात्माविक्लवया धिया
यत्पादयोरशठधीः सलिलं प्रदाय
दूर्वाङ्कुरैरपि विधाय सतीं सपर्याम्‌
अप्युत्तमां गतिमसौ भजते त्रिलोकीं
दाश्वानविक्लवमनाः कथमार्तिमृच्छेत्‌
श्रीभगवानुवाच
ब्रह्मन्यमनुगृह्णामि तद्विशो विधुनोम्यहम्‌
यन्मदः पुरुषः स्तब्धो लोकं मां चावमन्यते
यदा कदाचिज्जीवात्मा संसरन्निजकर्मभिः
नानायोनिष्वनीशोऽयं पौरुषीं गतिमाव्रजेत्‌
जन्मकर्मवयोरूप विद्यैश्वर्यधनादिभिः
यद्यस्य न भवेत्स्तम्भस्तत्रायं मदनुग्रहः
मानस्तम्भनिमित्तानां जन्मादीनां समन्ततः
सर्वश्रेयःप्रतीपानां हन्त मुह्येन्न मत्परः
एष दानवदैत्यानामग्रनीः कीर्तिवर्धनः
अजैषीदजयां मायां सीदन्नपि न मुह्यति
क्षीणरिक्थश्च्युतः स्थानात्क्षिप्तो बद्धश्च शत्रुभिः
ज्ञातिभिश्च परित्यक्तो यातनामनुयापितः
गुरुणा भर्त्सितः शप्तो जहौ सत्यं न सुव्रतः
छलैरुक्तो मया धर्मो नायं त्यजति सत्यवाक्‌
एष मे प्रापितः स्थानं दुष्प्रापममरैरपि
सावर्णेरन्तरस्यायं भवितेन्द्रो मदाश्रयः
तावत्सुतलमध्यास्तां विश्वकर्मविनिर्मितम्‌
यदाधयो व्याधयश्च क्लमस्तन्द्रा पराभवः
नोपसर्गा निवसतां सम्भवन्ति ममेक्षया
इन्द्रसेन महाराज याहि भो भद्रमस्तु ते
सुतलं स्वर्गिभिः प्रार्थ्यं ज्ञातिभिः परिवारितः
न त्वामभिभविष्यन्ति लोकेशाः किमुतापरे
त्वच्छासनातिगान्दैत्यांश्चक्रं मे सूदयिष्यति
रक्षिष्ये सर्वतोऽहं त्वां सानुगं सपरिच्छदम्‌
सदा सन्निहितं वीर तत्र मां द्रक्ष्यते भवान्‌
तत्र दानवदैत्यानां सङ्गात्ते भाव आसुरः
दृष्ट्वा मदनुभावं वै सद्यः कुण्ठो विनङ्क्ष्यति
श्रीशुक उवाच
इत्युक्तवन्तं पुरुषं पुरातनं महानुभावोऽखिलसाधुसम्मतः
बद्धाञ्जलिर्बाष्पकलाकुलेक्षणो भक्त्युत्कलो गद्गदया गिराब्रवीत्‌
श्रीबलिरुवाच
अहो प्रणामाय कृतः समुद्यमः प्रपन्नभक्तार्थविधौ समाहितः
यल्लोकपालैस्त्वदनुग्रहोऽमरैरलब्धपूर्वोऽपसदेऽसुरेऽर्पितः
श्रीशुक उवाच
इत्युक्त्वा हरिमानत्य ब्रह्माणं सभवं ततः
विवेश सुतलं प्रीतो बलिर्मुक्तः सहासुरैः
एवमिन्द्राय भगवान्प्रत्यानीय त्रिविष्टपम्‌
पूरयित्वादितेः काममशासत्सकलं जगत्‌
लब्धप्रसादं निर्मुक्तं पौत्रं वंशधरं बलिम्‌
निशाम्य भक्तिप्रवणः प्रह्राद इदमब्रवीत्‌
श्रीप्रह्राद उवाच
नेमं विरिञ्चो लभते प्रसादं न श्रीर्न शर्वः किमुतापरेऽन्ये
यन्नोऽसुराणामसि दुर्गपालो विश्वाभिवन्द्यैरभिवन्दिताङ्घ्रिः
यत्पादपद्ममकरन्दनिषेवणेन
ब्रह्मादयः शरणदाश्नुवते विभूतीः
कस्माद्वयं कुसृतयः खलयोनयस्ते
दाक्षिण्यदृष्टिपदवीं भवतः प्रणीताः
चित्रं तवेहितमहोऽमितयोगमाया
लीलाविसृष्टभुवनस्य विशारदस्य
सर्वात्मनः समदृशोऽविषमः स्वभावो
भक्तप्रियो यदसि कल्पतरुस्वभावः
श्रीभगवानुवाच
वत्स प्रह्राद भद्रं ते प्रयाहि सुतलालयम्‌
मोदमानः स्वपौत्रेण ज्ञातीनां सुखमावह
नित्यं द्रष्टासि मां तत्र गदापाणिमवस्थितम्‌
मद्दर्शनमहाह्लाद ध्वस्तकर्मनिबन्धनः
श्रीशुक उवाच
आज्ञां भगवतो राजन्प्रह्रादो बलिना सह
बाढमित्यमलप्रज्ञो मूर्ध्न्याधाय कृताञ्जलिः
परिक्रम्यादिपुरुषं सर्वासुरचमूपतिः
प्रणतस्तदनुज्ञातः प्रविवेश महाबिलम्‌
अथाहोशनसं राजन्हरिर्नारायणोऽन्तिके
आसीनमृत्विजां मध्ये सदसि ब्रह्मवादिनाम्‌
ब्रह्मन्सन्तनु शिष्यस्य कर्मच्छिद्रं वितन्वतः
यत्तत्कर्मसु वैषम्यं ब्रह्मदृष्टं समं भवेत्‌
श्रीशुक्र उवाच
कुतस्तत्कर्मवैषम्यं यस्य कर्मेश्वरो भवान्‌
यज्ञेशो यज्ञपुरुषः सर्वभावेन पूजितः
मन्त्रतस्तन्त्रतश्छिद्रं देशकालार्हवस्तुतः
सर्वं करोति निश्छिद्रमनुसङ्कीर्तनं तव
तथापि वदतो भूमन्करिष्याम्यनुशासनम्‌
एतच्छ्रेयः परं पुंसां यत्तवाज्ञानुपालनम्‌
श्रीशुक उवाच
प्रतिनन्द्य हरेराज्ञामुशना भगवानिति
यज्ञच्छिद्रं समाधत्त बलेर्विप्रर्षिभिः सह
एवं बलेर्महीं राजन्भिक्षित्वा वामनो हरिः
ददौ भ्रात्रे महेन्द्राय त्रिदिवं यत्परैर्हृतम्‌
प्रजापतिपतिर्ब्रह्मा देवर्षिपितृभूमिपैः
दक्षभृग्वङ्गिरोमुख्यैः कुमारेण भवेन च
कश्यपस्यादितेः प्रीत्यै सर्वभूतभवाय च
लोकानां लोकपालानामकरोद्वामनं पतिम्‌
वेदानां सर्वदेवानां धर्मस्य यशसः श्रियः
मङ्गलानां व्रतानां च कल्पं स्वर्गापवर्गयोः
उपेन्द्रं कल्पयां चक्रे पतिं सर्वविभूतये
तदा सर्वाणि भूतानि भृशं मुमुदिरे नृप
ततस्त्विन्द्रः पुरस्कृत्य देवयानेन वामनम्‌
लोकपालैर्दिवं निन्ये ब्रह्मणा चानुमोदितः
प्राप्य त्रिभुवनं चेन्द्र उपेन्द्रभुजपालितः
श्रिया परमया जुष्टो मुमुदे गतसाध्वसः
ब्रह्मा शर्वः कुमारश्च भृग्वाद्या मुनयो नृप
पितरः सर्वभूतानि सिद्धा वैमानिकाश्च ये
सुमहत्कर्म तद्विष्णोर्गायन्तः परमद्भुतम्‌
धिष्ण्यानि स्वानि ते जग्मुरदितिं च शशंसिरे
सर्वमेतन्मयाख्यातं भवतः कुलनन्दन
उरुक्रमस्य चरितं श्रोत्ऱ्‌^ईणामघमोचनम्‌
पारं महिम्न उरुविक्रमतो गृणानो
यः पार्थिवानि विममे स रजांसि मर्त्यः
किं जायमान उत जात उपैति मर्त्य
इत्याह मन्त्रदृगृषिः पुरुषस्य यस्य
य इदं देवदेवस्य हरेरद्भुतकर्मणः
अवतारानुचरितं शृण्वन्याति परां गतिम्‌
क्रियमाणे कर्मणीदं दैवे पित्र्येऽथ मानुषे
यत्र यत्रानुकीर्त्येत तत्तेषां सुकृतं विदुः
श्रीराजोवाच
भगवन्छ्रोतुमिच्छामि हरेरद्भुतकर्मणः
अवतारकथामाद्यां मायामत्स्यविडम्बनम्‌
यदर्थमदधाद्रूपं मात्स्यं लोकजुगुप्सितम्‌
तमःप्रकृतिदुर्मर्षं कर्मग्रस्त इवेश्वरः
एतन्नो भगवन्सर्वं यथावद्वक्तुमर्हसि
उत्तमश्लोकचरितं सर्वलोकसुखावहम्‌
श्रीसूत उवाच
इत्युक्तो विष्णुरातेन भगवान्बादरायणिः
उवाच चरितं विष्णोर्मत्स्यरूपेण यत्कृतम्‌
श्रीशुक उवाच
गोविप्रसुरसाधूनां छन्दसामपि चेश्वरः
रक्षामिच्छंस्तनूर्धत्ते धर्मस्यार्थस्य चैव हि
उच्चावचेषु भूतेषु चरन्वायुरिवेश्वरः
नोच्चावचत्वं भजते निर्गुणत्वाद्धियो गुणैः
आसीदतीतकल्पान्ते ब्राह्मो नैमित्तिको लयः
समुद्रोपप्लुतास्तत्र लोका भूरादयो नृप
कालेनागतनिद्रस्य धातुः शिशयिषोर्बली
मुखतो निःसृतान्वेदान्हयग्रीवोऽन्तिकेऽहरत्‌
ज्ञात्वा तद्दानवेन्द्रस्य हयग्रीवस्य चेष्टितम्‌
दधार शफरीरूपं भगवान्हरिरीश्वरः
तत्र राजऋषिः कश्चिन्नाम्ना सत्यव्रतो महान्‌
नारायणपरोऽतपत्तपः स सलिलाशनः
योऽसावस्मिन्महाकल्पे तनयः स विवस्वतः
श्राद्धदेव इति ख्यातो मनुत्वे हरिणार्पितः
एकदा कृतमालायां कुर्वतो जलतर्पणम्‌
तस्याञ्जल्युदके काचिच्छफर्येकाभ्यपद्यत
सत्यव्रतोऽञ्जलिगतां सह तोयेन भारत
उत्ससर्ज नदीतोये शफरीं द्रविडेश्वरः
तमाह सातिकरुणं महाकारुणिकं नृपम्‌
यादोभ्यो ज्ञातिघातिभ्यो दीनां मां दीनवत्सल
कथं विसृजसे राजन्भीतामस्मिन्सरिज्जले
तमात्मनोऽनुग्रहार्थं प्रीत्या मत्स्यवपुर्धरम्‌
अजानन्रक्षणार्थाय शफर्याः स मनो दधे
तस्या दीनतरं वाक्यमाश्रुत्य स महीपतिः
कलशाप्सु निधायैनां दयालुर्निन्य आश्रमम्‌
सा तु तत्रैकरात्रेण वर्धमाना कमण्डलौ
अलब्ध्वात्मावकाशं वा इदमाह महीपतिम्‌
नाहं कमण्डलावस्मिन्कृच्छ्रं वस्तुमिहोत्सहे
कल्पयौकः सुविपुलं यत्राहं निवसे सुखम्‌
स एनां तत आदाय न्यधादौदञ्चनोदके
तत्र क्षिप्ता मुहूर्तेन हस्तत्रयमवर्धत
न म एतदलं राजन्सुखं वस्तुमुदञ्चनम्‌
पृथु देहि पदं मह्यं यत्त्वाहं शरणं गता
तत आदाय सा राज्ञा क्षिप्ता राजन्सरोवरे
तदावृत्यात्मना सोऽयं महामीनोऽन्ववर्धत
नैतन्मे स्वस्तये राजन्नुदकं सलिलौकसः
निधेहि रक्षायोगेन ह्रदे मामविदासिनि
इत्युक्तः सोऽनयन्मत्स्यं तत्र तत्राविदासिनि
जलाशयेऽसम्मितं तं समुद्रे प्राक्षिपज्झषम्‌
क्षिप्यमाणस्तमाहेदमिह मां मकरादयः
अदन्त्यतिबला वीर मां नेहोत्स्रष्टुमर्हसि
एवं विमोहितस्तेन वदता वल्गुभारतीम्‌
तमाह को भवानस्मान्मत्स्यरूपेण मोहयन्‌
नैवं वीर्यो जलचरो दृष्टोऽस्माभिः श्रुतोऽपि वा
यो भवान्योजनशतमह्नाभिव्यानशे सरः
नूनं त्वं भगवान्साक्षाद्धरिर्नारायणोऽव्ययः
अनुग्रहाय भूतानां धत्से रूपं जलौकसाम्‌
नमस्ते पुरुषश्रेष्ठ स्थित्युत्पत्त्यप्ययेश्वर
भक्तानां नः प्रपन्नानां मुख्यो ह्यात्मगतिर्विभो
सर्वे लीलावतारास्ते भूतानां भूतिहेतवः
ज्ञातुमिच्छाम्यदो रूपं यदर्थं भवता धृतम्‌
न तेऽरविन्दाक्ष पदोपसर्पणं मृषा भवेत्सर्वसुहृत्प्रियात्मनः
यथेतरेषां पृथगात्मनां सतामदीदृशो यद्वपुरद्भुतं हि नः
श्रीशुक उवाच
इति ब्रुवाणं नृपतिं जगत्पतिः सत्यव्रतं मत्स्यवपुर्युगक्षये
विहर्तुकामः प्रलयार्णवेऽब्रवीच्चिकीर्षुरेकान्तजनप्रियः प्रियम्‌
श्रीभगवानुवाच
सप्तमे ह्यद्यतनादूर्ध्वमहन्येतदरिन्दम
निमङ्क्ष्यत्यप्ययाम्भोधौ त्रैलोक्यं भूर्भुवादिकम्‌
त्रिलोक्यां लीयमानायां संवर्ताम्भसि वै तदा
उपस्थास्यति नौः काचिद्विशाला त्वां मयेरिता
त्वं तावदोषधीः सर्वा बीजान्युच्चावचानि च
सप्तर्षिभिः परिवृतः सर्वसत्त्वोपबृंहितः
आरुह्य बृहतीं नावं विचरिष्यस्यविक्लवः
एकार्णवे निरालोके ऋषीणामेव वर्चसा
दोधूयमानां तां नावं समीरेण बलीयसा
उपस्थितस्य मे शृङ्गे निबध्नीहि महाहिना
अहं त्वामृषिभिः सार्धं सहनावमुदन्वति
विकर्षन्विचरिष्यामि यावद्ब्राह्मी निशा प्रभो
मदीयं महिमानं च परं ब्रह्मेति शब्दितम्‌
वेत्स्यस्यनुगृहीतं मे सम्प्रश्नैर्विवृतं हृदि
इत्थमादिश्य राजानं हरिरन्तरधीयत
सोऽन्ववैक्षत तं कालं यं हृषीकेश आदिशत्‌
आस्तीर्य दर्भान्प्राक्कूलान्राजर्षिः प्रागुदङ्मुखः
निषसाद हरेः पादौ चिन्तयन्मत्स्यरूपिणः
ततः समुद्र उद्वेलः सर्वतः प्लावयन्महीम्‌
वर्धमानो महामेघैर्वर्षद्भिः समदृश्यत
ध्यायन्भगवदादेशं ददृशे नावमागताम्‌
तामारुरोह विप्रेन्द्रैरादायौषधिवीरुधः
तमूचुर्मुनयः प्रीता राजन्ध्यायस्व केशवम्‌
स वै नः सङ्कटादस्मादविता शं विधास्यति
सोऽनुध्यातस्ततो राज्ञा प्रादुरासीन्महार्णवे
एकशृङ्गधरो मत्स्यो हैमो नियुतयोजनः
निबध्य नावं तच्छृङ्गे यथोक्तो हरिणा पुरा
वरत्रेणाहिना तुष्टस्तुष्टाव मधुसूदनम्‌
श्रीराजोवाच
अनाद्यविद्योपहतात्मसंविदस्तन्मूलसंसारपरिश्रमातुराः
यदृच्छयोपसृता यमाप्नुयुर्विमुक्तिदो नः परमो गुरुर्भवान्‌
जनोऽबुधोऽयं निजकर्मबन्धनः सुखेच्छया कर्म समीहतेऽसुखम्‌
यत्सेवया तां विधुनोत्यसन्मतिं ग्रन्थिं स भिन्द्याद्धृदयं स नो गुरुः
यत्सेवयाग्नेरिव रुद्ररोदनं पुमान्विजह्यान्मलमात्मनस्तमः
भजेत वर्णं निजमेष सोऽव्ययो भूयात्स ईशः परमो गुरोर्गुरुः
न यत्प्रसादायुतभागलेशमन्ये च देवा गुरवो जनाः स्वयम्‌
कर्तुं समेताः प्रभवन्ति पुंसस्तमीश्वरं त्वां शरणं प्रपद्ये
अचक्षुरन्धस्य यथाग्रणीः कृतस्तथा जनस्याविदुषोऽबुधो गुरुः
त्वमर्कदृक्सर्वदृशां समीक्षणो वृतो गुरुर्नः स्वगतिं बुभुत्सताम्‌
जनो जनस्यादिशतेऽसतीं गतिं यया प्रपद्येत दुरत्ययं तमः
त्वं त्वव्ययं ज्ञानममोघमञ्जसा प्रपद्यते येन जनो निजं पदम्‌
त्वं सर्वलोकस्य सुहृत्प्रियेश्वरो ह्यात्मा गुरुर्ज्ञानमभीष्टसिद्धिः
तथापि लोको न भवन्तमन्धधीर्जानाति सन्तं हृदि बद्धकामः
तं त्वामहं देववरं वरेण्यं प्रपद्य ईशं प्रतिबोधनाय
छिन्ध्यर्थदीपैर्भगवन्वचोभिर्ग्रन्थीन्हृदय्यान्विवृणु स्वमोकः
श्रीशुक उवाच
इत्युक्तवन्तं नृपतिं भगवानादिपूरुषः
मत्स्यरूपी महाम्भोधौ विहरंस्तत्त्वमब्रवीत्‌
पुराणसंहितां दिव्यां साङ्ख्ययोगक्रियावतीम्‌
सत्यव्रतस्य राजर्षेरात्मगुह्यमशेषतः
अश्रौषीदृषिभिः साकमात्मतत्त्वमसंशयम्‌
नाव्यासीनो भगवता प्रोक्तं ब्रह्म सनातनम्‌
अतीतप्रलयापाय उत्थिताय स वेधसे
हत्वासुरं हयग्रीवं वेदान्प्रत्याहरद्धरिः
स तु सत्यव्रतो राजा ज्ञानविज्ञानसंयुतः
विष्णोः प्रसादात्कल्पेऽस्मिन्नासीद्वैवस्वतो मनुः
सत्यव्रतस्य राजर्षेर्मायामत्स्यस्य शार्ङ्गिणः
संवादं महदाख्यानं श्रुत्वा मुच्येत किल्बिषात्‌
अवतारं हरेर्योऽयं कीर्तयेदन्वहं नरः
सङ्कल्पास्तस्य सिध्यन्ति स याति परमां गतिम्‌
प्रलयपयसि धातुः सुप्तशक्तेर्मुखेभ्यः
श्रुतिगणमपनीतं प्रत्युपादत्त हत्वा
दितिजमकथयद्यो ब्रह्म सत्यव्रतानां
तमहमखिलहेतुं जिह्ममीनं नतोऽस्मि
श्रीराजोवाच
मन्वन्तराणि सर्वाणि त्वयोक्तानि श्रुतानि मे
वीर्याण्यनन्तवीर्यस्य हरेस्तत्र कृतानि च
योऽसौ सत्यव्रतो नाम राजर्षिर्द्रविडेश्वरः
ज्ञानं योऽतीतकल्पान्ते लेभे पुरुषसेवया
स वै विवस्वतः पुत्रो मनुरासीदिति श्रुतम्‌
त्वत्तस्तस्य सुताः प्रोक्ता इक्ष्वाकुप्रमुखा नृपाः
तेषां वंशं पृथग्ब्रह्मन्वंशानुचरितानि च
कीर्तयस्व महाभाग नित्यं शुश्रूषतां हि नः
ये भूता ये भविष्याश्च भवन्त्यद्यतनाश्च ये
तेषां नः पुण्यकीर्तीनां सर्वेषां वद विक्रमान्‌
श्रीसूत उवाच
एवं परीक्षिता राज्ञा सदसि ब्रह्मवादिनाम्‌
पृष्टः प्रोवाच भगवाञ्छुकः परमधर्मवित्‌
श्रीशुक उवाच
श्रूयतां मानवो वंशः प्राचुर्येण परन्तप
न शक्यते विस्तरतो वक्तुं वर्षशतैरपि
परावरेषां भूतानामात्मा यः पुरुषः परः
स एवासीदिदं विश्वं कल्पान्तेऽन्यन्न किञ्चन
तस्य नाभेः समभवत्पद्मकोषो हिरण्मयः
तस्मिन्जज्ञे महाराज स्वयम्भूश्चतुराननः
मरीचिर्मनसस्तस्य जज्ञे तस्यापि कश्यपः
दाक्षायण्यां ततोऽदित्यां विवस्वानभवत्सुतः
ततो मनुः श्राद्धदेवः संज्ञायामास भारत
श्रद्धायां जनयामास दश पुत्रान्स आत्मवान्‌
इक्ष्वाकुनृगशर्याति दिष्टधृष्टकरूषकान्‌
नरिष्यन्तं पृषध्रं च नभगं च कविं विभुः
अप्रजस्य मनोः पूर्वं वसिष्ठो भगवान्किल
मित्रावरुणयोरिष्टिं प्रजार्थमकरोद्विभुः
तत्र श्रद्धा मनोः पत्नी होतारं समयाचत
दुहित्रर्थमुपागम्य प्रणिपत्य पयोव्रता
प्रेषितोऽध्वर्युणा होता व्यचरत्तत्समाहितः
गृहीते हविषि वाचा वषट्कारं गृणन्द्विजः
होतुस्तद्व्यभिचारेण कन्येला नाम साभवत्‌
तां विलोक्य मनुः प्राह नातितुष्टमना गुरुम्‌
भगवन्किमिदं जातं कर्म वो ब्रह्मवादिनाम्‌
विपर्ययमहो कष्टं मैवं स्याद्ब्रह्मविक्रिया
यूयं ब्रह्मविदो युक्तास्तपसा दग्धकिल्बिषाः
कुतः सङ्कल्पवैषम्यमनृतं विबुधेष्विव
निशम्य तद्वचस्तस्य भगवान्प्रपितामहः
होतुर्व्यतिक्रमं ज्ञात्वा बभाषे रविनन्दनम्‌
एतत्सङ्कल्पवैषम्यं होतुस्ते व्यभिचारतः
तथापि साधयिष्ये ते सुप्रजास्त्वं स्वतेजसा
एवं व्यवसितो राजन्भगवान्स महायशाः
अस्तौषीदादिपुरुषमिलायाः पुंस्त्वकाम्यया
तस्मै कामवरं तुष्टो भगवान्हरिरीश्वरः
ददाविलाभवत्तेन सुद्युम्नः पुरुषर्षभः
स एकदा महाराज विचरन्मृगयां वने
वृतः कतिपयामात्यैरश्वमारुह्य सैन्धवम्‌
प्रगृह्य रुचिरं चापं शरांश्च परमाद्भुतान्‌
दंशितोऽनुमृगं वीरो जगाम दिशमुत्तराम्‌
सुकुमारवनं मेरोरधस्तात्प्रविवेश ह
यत्रास्ते भगवान्छर्वो रममाणः सहोमया
तस्मिन्प्रविष्ट एवासौ सुद्युम्नः परवीरहा
अपश्यत्स्त्रियमात्मानमश्वं च वडवां नृप
तथा तदनुगाः सर्वे आत्मलिङ्गविपर्ययम्‌
दृष्ट्वा विमनसोऽभूवन्वीक्षमाणाः परस्परम्‌
श्रीराजोवाच
कथमेवं गुणो देशः केन वा भगवन्कृतः
प्रश्नमेनं समाचक्ष्व परं कौतूहलं हि नः
श्रीशुक उवाच
एकदा गिरिशं द्रष्टुमृषयस्तत्र सुव्रताः
दिशो वितिमिराभासाः कुर्वन्तः समुपागमन्‌
तान्विलोक्याम्बिका देवी विवासा व्रीडिता भृशम्‌
भर्तुरङ्कात्समुत्थाय नीवीमाश्वथ पर्यधात्‌
ऋषयोऽपि तयोर्वीक्ष्य प्रसङ्गं रममाणयोः
निवृत्ताः प्रययुस्तस्मान्नरनारायणाश्रमम्‌
तदिदं भगवानाह प्रियायाः प्रियकाम्यया
स्थानं यः प्रविशेदेतत्स वै योषिद्भवेदिति
तत ऊर्ध्वं वनं तद्वै पुरुषा वर्जयन्ति हि
सा चानुचरसंयुक्ता विचचार वनाद्वनम्‌
अथ तामाश्रमाभ्याशे चरन्तीं प्रमदोत्तमाम्‌
स्त्रीभिः परिवृतां वीक्ष्य चकमे भगवान्बुधः
सापि तं चकमे सुभ्रूः सोमराजसुतं पतिम्‌
स तस्यां जनयामास पुरूरवसमात्मजम्‌
एवं स्त्रीत्वमनुप्राप्तः सुद्युम्नो मानवो नृपः
सस्मार स कुलाचार्यं वसिष्ठमिति शुश्रुम
स तस्य तां दशां दृष्ट्वा कृपया भृशपीडितः
सुद्युम्नस्याशयन्पुंस्त्वमुपाधावत शङ्करम्‌
तुष्टस्तस्मै स भगवानृषये प्रियमावहन्‌
स्वां च वाचमृतां कुर्वन्निदमाह विशाम्पते
मासं पुमान्स भविता मासं स्त्री तव गोत्रजः
इत्थं व्यवस्थया कामं सुद्युम्नोऽवतु मेदिनीम्‌
आचार्यानुग्रहात्कामं लब्ध्वा पुंस्त्वं व्यवस्थया
पालयामास जगतीं नाभ्यनन्दन्स्म तं प्रजाः
तस्योत्कलो गयो राजन्विमलश्च त्रयः सुताः
दक्षिणापथराजानो बभूवुर्धर्मवत्सलाः
ततः परिणते काले प्रतिष्ठानपतिः प्रभुः
पुरूरवस उत्सृज्य गां पुत्राय गतो वनम्‌
श्रीशुक उवाच
एवं गतेऽथ सुद्युम्ने मनुर्वैवस्वतः सुते
पुत्रकामस्तपस्तेपे यमुनायां शतं समाः
ततोऽयजन्मनुर्देवमपत्यार्थं हरिं प्रभुम्‌
इक्ष्वाकुपूर्वजान्पुत्रान्लेभे स्वसदृशान्दश
पृषध्रस्तु मनोः पुत्रो गोपालो गुरुणा कृतः
पालयामास गा यत्तो रात्र्यां वीरासनव्रतः
एकदा प्राविशद्गोष्ठं शार्दूलो निशि वर्षति
शयाना गाव उत्थाय भीतास्ता बभ्रमुर्व्रजे
एकां जग्राह बलवान्सा चुक्रोश भयातुरा
तस्यास्तु क्रन्दितं श्रुत्वा पृषध्रोऽनुससार ह
खड्गमादाय तरसा प्रलीनोडुगणे निशि
अजानन्नच्छिनोद्बभ्रोः शिरः शार्दूलशङ्कया
व्याघ्रोऽपि वृक्णश्रवणो निस्त्रिंशाग्राहतस्ततः
निश्चक्राम भृशं भीतो रक्तं पथि समुत्सृजन्‌
मन्यमानो हतं व्याघ्रं पृषध्रः परवीरहा
अद्राक्षीत्स्वहतां बभ्रुं व्युष्टायां निशि दुःखितः
तं शशाप कुलाचार्यः कृतागसमकामतः
न क्षत्रबन्धुः शूद्रस्त्वं कर्मणा भवितामुना
एवं शप्तस्तु गुरुणा प्रत्यगृह्णात्कृताञ्जलिः
अधारयद्व्रतं वीर ऊर्ध्वरेता मुनिप्रियम्‌
वासुदेवे भगवति सर्वात्मनि परेऽमले
एकान्तित्वं गतो भक्त्या सर्वभूतसुहृत्समः
विमुक्तसङ्गः शान्तात्मा संयताक्षोऽपरिग्रहः
यदृच्छयोपपन्नेन कल्पयन्वृत्तिमात्मनः
आत्मन्यात्मानमाधाय ज्ञानतृप्तः समाहितः
विचचार महीमेतां जडान्धबधिराकृतिः
एवं वृत्तो वनं गत्वा दृष्ट्वा दावाग्निमुत्थितम्‌
तेनोपयुक्तकरणो ब्रह्म प्राप परं मुनिः
कविः कनीयान्विषयेषु निःस्पृहो विसृज्य राज्यं सह बन्धुभिर्वनम्‌
निवेश्य चित्ते पुरुषं स्वरोचिषं विवेश कैशोरवयाः परं गतः
करूषान्मानवादासन्कारूषाः क्षत्रजातयः
उत्तरापथगोप्तारो ब्रह्मण्या धर्मवत्सलाः
धृष्टाद्धार्ष्टमभूत्क्षत्रं ब्रह्मभूयं गतं क्षितौ
नृगस्य वंशः सुमतिर्भूतज्योतिस्ततो वसुः
वसोः प्रतीकस्तत्पुत्र ओघवानोघवत्पिता
कन्या चौघवती नाम सुदर्शन उवाह ताम्‌
चित्रसेनो नरिष्यन्तादृक्षस्तस्य सुतोऽभवत्‌
तस्य मीढ्वांस्ततः पूर्ण इन्द्रसेनस्तु तत्सुतः
वीतिहोत्रस्त्विन्द्रसेनात्तस्य सत्यश्रवा अभूत्‌
उरुश्रवाः सुतस्तस्य देवदत्तस्ततोऽभवत्‌
ततोऽग्निवेश्यो भगवानग्निः स्वयमभूत्सुतः
कानीन इति विख्यातो जातूकर्ण्यो महानृषिः
ततो ब्रह्मकुलं जातमाग्निवेश्यायनं नृप
नरिष्यन्तान्वयः प्रोक्तो दिष्टवंशमतः शृणु
नाभागो दिष्टपुत्रोऽन्यः कर्मणा वैश्यतां गतः
भलन्दनः सुतस्तस्य वत्सप्रीतिर्भलन्दनात्‌
वत्सप्रीतेः सुतः प्रांशुस्तत्सुतं प्रमतिं विदुः
खनित्रः प्रमतेस्तस्माच्चाक्षुषोऽथ विविंशतिः
विविंशतेः सुतो रम्भः खनीनेत्रोऽस्य धार्मिकः
करन्धमो महाराज तस्यासीदात्मजो नृप
तस्यावीक्षित्सुतो यस्य मरुत्तश्चक्रवर्त्यभूत्‌
संवर्तोऽयाजयद्यं वै महायोग्यङ्गिरःसुतः
मरुत्तस्य यथा यज्ञो न तथान्योऽस्ति कश्चन
सर्वं हिरण्मयं त्वासीद्यत्किञ्चिच्चास्य शोभनम्‌
अमाद्यदिन्द्रः सोमेन दक्षिणाभिर्द्विजातयः
मरुतः परिवेष्टारो विश्वेदेवाः सभासदः
मरुत्तस्य दमः पुत्रस्तस्यासीद्राज्यवर्धनः
सुधृतिस्तत्सुतो जज्ञे सौधृतेयो नरः सुतः
तत्सुतः केवलस्तस्माद्धुन्धुमान्वेगवांस्ततः
बुधस्तस्याभवद्यस्य तृणबिन्दुर्महीपतिः
तं भेजेऽलम्बुषा देवी भजनीयगुणालयम्‌
वराप्सरा यतः पुत्राः कन्या चेलविलाभवत्‌
यस्यामुत्पादयामास विश्रवा धनदं सुतम्‌
प्रादाय विद्यां परमामृषिर्योगेश्वरः पितुः
विशालः शून्यबन्धुश्च धूम्रकेतुश्च तत्सुताः
विशालो वंशकृद्राजा वैशालीं निर्ममे पुरीम्‌
हेमचन्द्रः सुतस्तस्य धूम्राक्षस्तस्य चात्मजः
तत्पुत्रात्संयमादासीत्कृशाश्वः सहदेवजः
कृशाश्वात्सोमदत्तोऽभूद्योऽश्वमेधैरिडस्पतिम्‌
इष्ट्वा पुरुषमापाग्र्यां गतिं योगेश्वराश्रिताम्‌
सौमदत्तिस्तु सुमतिस्तत्पुत्रो जनमेजयः
एते वैशालभूपालास्तृणबिन्दोर्यशोधराः
श्रीशुक उवाच
शर्यातिर्मानवो राजा ब्रह्मिष्ठः सम्बभूव ह
यो वा अङ्गिरसां सत्रे द्वितीयमहरूचिवान्‌
सुकन्या नाम तस्यासीत्कन्या कमललोचना
तया सार्धं वनगतो ह्यगमच्च्यवनाश्रमम्‌
सा सखीभिः परिवृता विचिन्वन्त्यङ्घ्रिपान्वने
वल्मीकरन्ध्रे ददृशे खद्योते इव ज्योतिषी
ते दैवचोदिता बाला ज्योतिषी कण्टकेन वै
अविध्यन्मुग्धभावेन सुस्रावासृक्ततो बहिः
शकृन्मूत्रनिरोधोऽभूत्सैनिकानां च तत्क्षणात्‌
राजर्षिस्तमुपालक्ष्य पुरुषान्विस्मितोऽब्रवीत्‌
अप्यभद्रं न युष्माभिर्भार्गवस्य विचेष्टितम्‌
व्यक्तं केनापि नस्तस्य कृतमाश्रमदूषणम्‌
सुकन्या प्राह पितरं भीता किञ्चित्कृतं मया
द्वे ज्योतिषी अजानन्त्या निर्भिन्ने कण्टकेन वै
दुहितुस्तद्वचः श्रुत्वा शर्यातिर्जातसाध्वसः
मुनिं प्रसादयामास वल्मीकान्तर्हितं शनैः
तदभिप्रायमाज्ञाय प्रादाद्दुहितरं मुनेः
कृच्छ्रान्मुक्तस्तमामन्त्र्य पुरं प्रायात्समाहितः
सुकन्या च्यवनं प्राप्य पतिं परमकोपनम्‌
प्रीणयामास चित्तज्ञा अप्रमत्तानुवृत्तिभिः
कस्यचित्त्वथ कालस्य नासत्यावाश्रमागतौ
तौ पूजयित्वा प्रोवाच वयो मे दत्तमीश्वरौ
ग्रहं ग्रहीष्ये सोमस्य यज्ञे वामप्यसोमपोः
क्रियतां मे वयोरूपं प्रमदानां यदीप्सितम्‌
बाढमित्यूचतुर्विप्रमभिनन्द्य भिषक्तमौ
निमज्जतां भवानस्मिन्ह्रदे सिद्धविनिर्मिते
इत्युक्तो जरया ग्रस्त देहो धमनिसन्ततः
ह्रदं प्रवेशितोऽश्विभ्यां वलीपलितविग्रहः
पुरुषास्त्रय उत्तस्थुरपीव्या वनिताप्रियाः
पद्मस्रजः कुण्डलिनस्तुल्यरूपाः सुवाससः
तान्निरीक्ष्य वरारोहा सरूपान्सूर्यवर्चसः
अजानती पतिं साध्वी अश्विनौ शरणं ययौ
दर्शयित्वा पतिं तस्यै पातिव्रत्येन तोषितौ
ऋषिमामन्त्र्य ययतुर्विमानेन त्रिविष्टपम्‌
यक्ष्यमाणोऽथ शर्यातिश्च्यवनस्याश्रमं गतः
ददर्श दुहितुः पार्श्वे पुरुषं सूर्यवर्चसम्‌
राजा दुहितरं प्राह कृतपादाभिवन्दनाम्‌
आशिषश्चाप्रयुञ्जानो नातिप्रीतिमना इव
चिकीर्षितं ते किमिदं पतिस्त्वया प्रलम्भितो लोकनमस्कृतो मुनिः
यत्त्वं जराग्रस्तमसत्यसम्मतं विहाय जारं भजसेऽमुमध्वगम्‌
कथं मतिस्तेऽवगतान्यथा सतां कुलप्रसूते कुलदूषणं त्विदम्‌
बिभर्षि जारं यदपत्रपा कुलं पितुश्च भर्तुश्च नयस्यधस्तमः
एवं ब्रुवाणं पितरं स्मयमाना शुचिस्मिता
उवाच तात जामाता तवैष भृगुनन्दनः
शशंस पित्रे तत्सर्वं वयोरूपाभिलम्भनम्‌
विस्मितः परमप्रीतस्तनयां परिषस्वजे
सोमेन याजयन्वीरं ग्रहं सोमस्य चाग्रहीत्‌
असोमपोरप्यश्विनोश्च्यवनः स्वेन तेजसा
हन्तुं तमाददे वज्रं सद्यो मन्युरमर्षितः
सवज्रं स्तम्भयामास भुजमिन्द्रस्य भार्गवः
अन्वजानंस्ततः सर्वे ग्रहं सोमस्य चाश्विनोः
भिषजाविति यत्पूर्वं सोमाहुत्या बहिष्कृतौ
उत्तानबर्हिरानर्तो भूरिषेण इति त्रयः
शर्यातेरभवन्पुत्रा आनर्ताद्रेवतोऽभवत्‌
सोऽन्तःसमुद्रे नगरीं विनिर्माय कुशस्थलीम्‌
आस्थितोऽभुङ्क्त विषयानानर्तादीनरिन्दम
तस्य पुत्रशतं जज्ञे ककुद्मिज्येष्ठमुत्तमम्‌
ककुद्मी रेवतीं कन्यां स्वामादाय विभुं गतः
पुत्र्या वरं परिप्रष्टुं ब्रह्मलोकमपावृतम्‌
आवर्तमाने गान्धर्वे स्थितोऽलब्धक्षणः क्षणम्‌
तदन्त आद्यमानम्य स्वाभिप्रायं न्यवेदयत्‌
तच्छ्रुत्वा भगवान्ब्रह्मा प्रहस्य तमुवाच ह
अहो राजन्निरुद्धास्ते कालेन हृदि ये कृताः
तत्पुत्रपौत्रनप्त्णां गोत्राणि च न शृण्महे
कालोऽभियातस्त्रिणव चतुर्युगविकल्पितः
तद्गच्छ देवदेवांशो बलदेवो महाबलः
कन्यारत्नमिदं राजन्नररत्नाय देहि भोः
भुवो भारावताराय भगवान्भूतभावनः
अवतीर्णो निजांशेन पुण्यश्रवणकीर्तनः
इत्यादिष्टोऽभिवन्द्याजं नृपः स्वपुरमागतः
त्यक्तं पुण्यजनत्रासाद्भ्रातृभिर्दिक्ष्ववस्थितैः
सुतां दत्त्वानवद्याङ्गीं बलाय बलशालिने
बदर्याख्यं गतो राजा तप्तुं नारायणाश्रमम्‌
श्रीशुक उवाच
नाभागो नभगापत्यं यं ततं भ्रातरः कविम्‌
यविष्ठं व्यभजन्दायं ब्रह्मचारिणमागतम्‌
भ्रातरोऽभाङ्क्त किं मह्यं भजाम पितरं तव
त्वां ममार्यास्तताभाङ्क्षुर्मा पुत्रक तदादृथाः
इमे अङ्गिरसः सत्रमासतेऽद्य सुमेधसः
षष्ठं षष्ठमुपेत्याहः कवे मुह्यन्ति कर्मणि
तांस्त्वं शंसय सूक्ते द्वे वैश्वदेवे महात्मनः
ते स्वर्यन्तो धनं सत्र परिशेषितमात्मनः
दास्यन्ति तेऽथ तानर्च्छ तथा स कृतवान्यथा
तस्मै दत्त्वा ययुः स्वर्गं ते सत्रपरिशेषणम्‌
तं कश्चित्स्वीकरिष्यन्तं पुरुषः कृष्णदर्शनः
उवाचोत्तरतोऽभ्येत्य ममेदं वास्तुकं वसु
ममेदमृषिभिर्दत्तमिति तर्हि स्म मानवः
स्यान्नौ ते पितरि प्रश्नः पृष्टवान्पितरं यथा
यज्ञवास्तुगतं सर्वमुच्छिष्टमृषयः क्वचित्‌
चक्रुर्हि भागं रुद्राय स देवः सर्वमर्हति
नाभागस्तं प्रणम्याह तवेश किल वास्तुकम्‌
इत्याह मे पिता ब्रह्मञ्छिरसा त्वां प्रसादये
यत्ते पितावदद्धर्मं त्वं च सत्यं प्रभाषसे
ददामि ते मन्त्रदृशो ज्ञानं ब्रह्म सनातनम्‌
गृहाण द्रविणं दत्तं मत्सत्रपरिशेषितम्‌
इत्युक्त्वान्तर्हितो रुद्रो भगवान्धर्मवत्सलः
य एतत्संस्मरेत्प्रातः सायं च सुसमाहितः
कविर्भवति मन्त्रज्ञो गतिं चैव तथात्मनः
नाभागादम्बरीषोऽभून्महाभागवतः कृती
नास्पृशद्ब्रह्मशापोऽपि यं न प्रतिहतः क्वचित्‌
श्रीराजोवाच
भगवन्छ्रोतुमिच्छामि राजर्षेस्तस्य धीमतः
न प्राभूद्यत्र निर्मुक्तो ब्रह्मदण्डो दुरत्ययः
श्रीशुक उवाच
अम्बरीषो महाभागः सप्तद्वीपवतीं महीम्‌
अव्ययां च श्रियं लब्ध्वा विभवं चातुलं भुवि
मेनेऽतिदुर्लभं पुंसां सर्वं तत्स्वप्नसंस्तुतम्‌
विद्वान्विभवनिर्वाणं तमो विशति यत्पुमान्‌
वासुदेवे भगवति तद्भक्तेषु च साधुषु
प्राप्तो भावं परं विश्वं येनेदं लोष्ट्रवत्स्मृतम्‌
स वै मनः कृष्णपदारविन्दयोर्वचांसि वैकुण्ठगुणानुवर्णने
करौ हरेर्मन्दिरमार्जनादिषु श्रुतिं चकाराच्युतसत्कथोदये
मुकुन्दलिङ्गालयदर्शने दृशौ तद्भृत्यगात्रस्पर्शेऽङ्गसङ्गमम्‌
घ्राणं च तत्पादसरोजसौरभे श्रीमत्तुलस्या रसनां तदर्पिते
पादौ हरेः क्षेत्रपदानुसर्पणे शिरो हृषीकेशपदाभिवन्दने
कामं च दास्ये न तु कामकाम्यया यथोत्तमश्लोकजनाश्रया रतिः
एवं सदा कर्मकलापमात्मनः परेऽधियज्ञे भगवत्यधोक्षजे
सर्वात्मभावं विदधन्महीमिमां तन्निष्ठविप्राभिहितः शशास ह
ईजेऽश्वमेधैरधियज्ञमीश्वरं महाविभूत्योपचिताङ्गदक्षिणैः
ततैर्वसिष्ठासितगौतमादिभिर्धन्वन्यभिस्रोतमसौ सरस्वतीम्‌
यस्य क्रतुषु गीर्वाणैः सदस्या ऋत्विजो जनाः
तुल्यरूपाश्चानिमिषा व्यदृश्यन्त सुवाससः
स्वर्गो न प्रार्थितो यस्य मनुजैरमरप्रियः
शृण्वद्भिरुपगायद्भिरुत्तमश्लोकचेष्टितम्‌
संवर्धयन्ति यत्कामाः स्वाराज्यपरिभाविताः
दुर्लभा नापि सिद्धानां मुकुन्दं हृदि पश्यतः
स इत्थं भक्तियोगेन तपोयुक्तेन पार्थिवः
स्वधर्मेण हरिं प्रीणन्सर्वान्कामान्शनैर्जहौ
गृहेषु दारेषु सुतेषु बन्धुषु द्विपोत्तमस्यन्दनवाजिवस्तुषु
अक्षय्यरत्नाभरणाम्बरादिष्वनन्तकोशेष्वकरोदसन्मतिम्‌
तस्मा अदाद्धरिश्चक्रं प्रत्यनीकभयावहम्‌
एकान्तभक्तिभावेन प्रीतो भक्ताभिरक्षणम्‌
आरिराधयिषुः कृष्णं महिष्या तुल्यशीलया
युक्तः सांवत्सरं वीरो दधार द्वादशीव्रतम्‌
व्रतान्ते कार्तिके मासि त्रिरात्रं समुपोषितः
स्नातः कदाचित्कालिन्द्यां हरिं मधुवनेऽर्चयत्‌
महाभिषेकविधिना सर्वोपस्करसम्पदा
अभिषिच्याम्बराकल्पैर्गन्धमाल्यार्हणादिभिः
तद्गतान्तरभावेन पूजयामास केशवम्‌
ब्राह्मणांश्च महाभागान्सिद्धार्थानपि भक्तितः
गवां रुक्मविषाणीनां रूप्याङ्घ्रीणां सुवाससाम्‌
पयःशीलवयोरूप वत्सोपस्करसम्पदाम्‌
प्राहिणोत्साधुविप्रेभ्यो गृहेषु न्यर्बुदानि षट्‌
भोजयित्वा द्विजानग्रे स्वाद्वन्नं गुणवत्तमम्‌
लब्धकामैरनुज्ञातः पारणायोपचक्रमे
तस्य तर्ह्यतिथिः साक्षाद्दुर्वासा भगवानभूत्‌
तमानर्चातिथिं भूपः प्रत्युत्थानासनार्हणैः
ययाचेऽभ्यवहाराय पादमूलमुपागतः
प्रतिनन्द्य स तां याच्ञां कर्तुमावश्यकं गतः
निममज्ज बृहद्ध्यायन्कालिन्दीसलिले शुभे
मुहूर्तार्धावशिष्टायां द्वादश्यां पारणं प्रति
चिन्तयामास धर्मज्ञो द्विजैस्तद्धर्मसङ्कटे
ब्राह्मणातिक्रमे दोषो द्वादश्यां यदपारणे
यत्कृत्वा साधु मे भूयादधर्मो वा न मां स्पृशेत्‌
अम्भसा केवलेनाथ करिष्ये व्रतपारणम्‌
आहुरब्भक्षणं विप्रा ह्यशितं नाशितं च तत्‌
इत्यपः प्राश्य राजर्षिश्चिन्तयन्मनसाच्युतम्‌
प्रत्यचष्ट कुरुश्रेष्ठ द्विजागमनमेव सः
दुर्वासा यमुनाकूलात्कृतावश्यक आगतः
राज्ञाभिनन्दितस्तस्य बुबुधे चेष्टितं धिया
मन्युना प्रचलद्गात्रो भ्रुकुटीकुटिलाननः
बुभुक्षितश्च सुतरां कृताञ्जलिमभाषत
अहो अस्य नृशंसस्य श्रियोन्मत्तस्य पश्यत
धर्मव्यतिक्रमं विष्णोरभक्तस्येशमानिनः
यो मामतिथिमायातमातिथ्येन निमन्त्र्य च
अदत्त्वा भुक्तवांस्तस्य सद्यस्ते दर्शये फलम्‌
एवं ब्रुवाण उत्कृत्य जटां रोषप्रदीपितः
तया स निर्ममे तस्मै कृत्यां कालानलोपमाम्‌
तामापतन्तीं ज्वलतीमसिहस्तां पदा भुवम्‌
वेपयन्तीं समुद्वीक्ष्य न चचाल पदान्नृपः
प्राग्दिष्टं भृत्यरक्षायां पुरुषेण महात्मना
ददाह कृत्यां तां चक्रं क्रुद्धाहिमिव पावकः
तदभिद्रवदुद्वीक्ष्य स्वप्रयासं च निष्फलम्‌
दुर्वासा दुद्रुवे भीतो दिक्षु प्राणपरीप्सया
तमन्वधावद्भगवद्रथाङ्गं दावाग्निरुद्धूतशिखो यथाहिम्‌
तथानुषक्तं मुनिरीक्षमाणो गुहां विविक्षुः प्रससार मेरोः
दिशो नभः क्ष्मां विवरान्समुद्रान्लोकान्सपालांस्त्रिदिवं गतः सः
यतो यतो धावति तत्र तत्र सुदर्शनं दुष्प्रसहं ददर्श
अलब्धनाथः स सदा कुतश्चित्सन्त्रस्तचित्तोऽरणमेषमाणः
देवं विरिञ्चं समगाद्विधातस्त्राह्यात्मयोनेऽजिततेजसो माम्‌
श्रीब्रह्मोवाच
स्थानं मदीयं सहविश्वमेतत्क्रीडावसाने द्विपरार्धसंज्ञे
भ्रूभङ्गमात्रेण हि सन्दिधक्षोः कालात्मनो यस्य तिरोभविष्यति
अहं भवो दक्षभृगुप्रधानाः प्रजेशभूतेशसुरेशमुख्याः
सर्वे वयं यन्नियमं प्रपन्ना मूर्ध्न्यार्पितं लोकहितं वहामः
प्रत्याख्यातो विरिञ्चेन विष्णुचक्रोपतापितः
दुर्वासाः शरणं यातः शर्वं कैलासवासिनम्‌
श्रीशङ्कर उवाच
वयं न तात प्रभवाम भूम्नि यस्मिन्परेऽन्येऽप्यजजीवकोशाः
भवन्ति काले न भवन्ति हीदृशाः सहस्रशो यत्र वयं भ्रमामः
अहं सनत्कुमारश्च नारदो भगवानजः
कपिलोऽपान्तरतमो देवलो धर्म आसुरिः
मरीचिप्रमुखाश्चान्ये सिद्धेशाः पारदर्शनाः
विदाम न वयं सर्वे यन्मायां माययावृताः
तस्य विश्वेश्वरस्येदं शस्त्रं दुर्विषहं हि नः
तमेवं शरणं याहि हरिस्ते शं विधास्यति
ततो निराशो दुर्वासाः पदं भगवतो ययौ
वैकुण्ठाख्यं यदध्यास्ते श्रीनिवासः श्रिया सह
सन्दह्यमानोऽजितशस्त्रवह्निना तत्पादमूले पतितः सवेपथुः
आहाच्युतानन्त सदीप्सित प्रभो कृतागसं मावहि विश्वभावन
अजानता ते परमानुभावं कृतं मयाघं भवतः प्रियाणाम्‌
विधेहि तस्यापचितिं विधातर्मुच्येत यन्नाम्न्युदिते नारकोऽपि
श्रीभगवानुवाच
अहं भक्तपराधीनो ह्यस्वतन्त्र इव द्विज
साधुभिर्ग्रस्तहृदयो भक्तैर्भक्तजनप्रियः
नाहमात्मानमाशासे मद्भक्तैः साधुभिर्विना
श्रियं चात्यन्तिकीं ब्रह्मन्येषां गतिरहं परा
ये दारागारपुत्राप्त प्राणान्वित्तमिमं परम्‌
हित्वा मां शरणं याताः कथं तांस्त्यक्तुमुत्सहे
मयि निर्बद्धहृदयाः साधवः समदर्शनाः
वशे कुर्वन्ति मां भक्त्या सत्स्त्रियः सत्पतिं यथा
मत्सेवया प्रतीतं ते सालोक्यादिचतुष्टयम्‌
नेच्छन्ति सेवया पूर्णाः कुतोऽन्यत्कालविप्लुतम्‌
साधवो हृदयं मह्यं साधूनां हृदयं त्वहम्‌
मदन्यत्ते न जानन्ति नाहं तेभ्यो मनागपि
उपायं कथयिष्यामि तव विप्र शृणुष्व तत्‌
अयं ह्यात्माभिचारस्ते यतस्तं याहि मा चिरम्‌
साधुषु प्रहितं तेजः प्रहर्तुः कुरुतेऽशिवम्‌
तपो विद्या च विप्राणां निःश्रेयसकरे उभे
ते एव दुर्विनीतस्य कल्पेते कर्तुरन्यथा
ब्रह्मंस्तद्गच्छ भद्रं ते नाभागतनयं नृपम्‌
क्षमापय महाभागं ततः शान्तिर्भविष्यति
श्रीशुक उवाच
एवं भगवतादिष्टो दुर्वासाश्चक्रतापितः
अम्बरीषमुपावृत्य तत्पादौ दुःखितोऽग्रहीत्‌
तस्य सोद्यममावीक्ष्य पादस्पर्शविलज्जितः
अस्तावीत्तद्धरेरस्त्रं कृपया पीडितो भृशम्‌
अम्बरीष उवाच
त्वमग्निर्भगवान्सूर्यस्त्वं सोमो ज्योतिषां पतिः
त्वमापस्त्वं क्षितिर्व्योम वायुर्मात्रेन्द्रियाणि च
सुदर्शन नमस्तुभ्यं सहस्राराच्युतप्रिय
सर्वास्त्रघातिन्विप्राय स्वस्ति भूया इडस्पते
त्वं धर्मस्त्वमृतं सत्यं त्वं यज्ञोऽखिलयज्ञभुक्‌
त्वं लोकपालः सर्वात्मा त्वं तेजः पौरुषं परम्‌
नमः सुनाभाखिलधर्मसेतवे ह्यधर्मशीलासुरधूमकेतवे
त्रैलोक्यगोपाय विशुद्धवर्चसे मनोजवायाद्भुतकर्मणे गृणे
त्वत्तेजसा धर्ममयेन संहृतं तमः प्रकाशश्च दृशो महात्मनाम्‌
दुरत्ययस्ते महिमा गिरां पते त्वद्रूपमेतत्सदसत्परावरम्‌
यदा विसृष्टस्त्वमनञ्जनेन वै बलं प्रविष्टोऽजित दैत्यदानवम्‌
बाहूदरोर्वङ्घ्रिशिरोधराणि वृश्चन्नजस्रं प्रधने विराजसे
स त्वं जगत्त्राण खलप्रहाणये निरूपितः सर्वसहो गदाभृता
विप्रस्य चास्मत्कुलदैवहेतवे विधेहि भद्रं तदनुग्रहो हि नः
यद्यस्ति दत्तमिष्टं वा स्वधर्मो वा स्वनुष्ठितः
कुलं नो विप्रदैवं चेद्द्विजो भवतु विज्वरः
यदि नो भगवान्प्रीत एकः सर्वगुणाश्रयः
सर्वभूतात्मभावेन द्विजो भवतु विज्वरः
श्रीशुक उवाच
इति संस्तुवतो राज्ञो विष्णुचक्रं सुदर्शनम्‌
अशाम्यत्सर्वतो विप्रं प्रदहद्राजयाच्ञया
स मुक्तोऽस्त्राग्नितापेन दुर्वासाः स्वस्तिमांस्ततः
प्रशशंस तमुर्वीशं युञ्जानः परमाशिषः
दुर्वासा उवाच
अहो अनन्तदासानां महत्त्वं दृष्टमद्य मे
कृतागसोऽपि यद्राजन्मङ्गलानि समीहसे
दुष्करः को नु साधूनां दुस्त्यजो वा महात्मनाम्‌
यैः सङ्गृहीतो भगवान्सात्वतामृषभो हरिः
यन्नामश्रुतिमात्रेण पुमान्भवति निर्मलः
तस्य तीर्थपदः किं वा दासानामवशिष्यते
राजन्ननुगृहीतोऽहं त्वयातिकरुणात्मना
मदघं पृष्ठतः कृत्वा प्राणा यन्मेऽभिरक्षिताः
राजा तमकृताहारः प्रत्यागमनकाङ्क्षया
चरणावुपसङ्गृह्य प्रसाद्य समभोजयत्‌
सोऽशित्वादृतमानीतमातिथ्यं सार्वकामिकम्‌
तृप्तात्मा नृपतिं प्राह भुज्यतामिति सादरम्‌
प्रीतोऽस्म्यनुगृहीतोऽस्मि तव भागवतस्य वै
दर्शनस्पर्शनालापैरातिथ्येनात्ममेधसा
कर्मावदातमेतत्ते गायन्ति स्वःस्त्रियो मुहुः
कीर्तिं परमपुण्यां च कीर्तयिष्यति भूरियम्‌
श्रीशुक उवाच
एवं सङ्कीर्त्य राजानं दुर्वासाः परितोषितः
ययौ विहायसामन्त्र्य ब्रह्मलोकमहैतुकम्‌
संवत्सरोऽत्यगात्तावद्यावता नागतो गतः
मुनिस्तद्दर्शनाकाङ्क्षो राजाब्भक्षो बभूव ह
गतेऽथ दुर्वाससि सोऽम्बरीषो द्विजोपयोगातिपवित्रमाहरत्‌
ऋषेर्विमोक्षं व्यसनं च वीक्ष्य मेने स्ववीर्यं च परानुभावम्‌
एवं विधानेकगुणः स राजा परात्मनि ब्रह्मणि वासुदेवे
क्रियाकलापैः समुवाह भक्तिं ययाविरिञ्च्यान्निरयांश्चकार
श्रीशुक उवाच
अथाम्बरीषस्तनयेषु राज्यं समानशीलेषु विसृज्य धीरः
वनं विवेशात्मनि वासुदेवे मनो दधद्ध्वस्तगुणप्रवाहः
इत्येतत्पुण्यमाख्यानमम्बरीषस्य भूपते
सङ्कीर्तयन्ननुध्यायन्भक्तो भगवतो भवेत्‌
अम्बरीषस्य चरितं ये शृण्वन्ति महात्मनः
मुक्तिं प्रयान्ति ते सर्वे भक्त्या विष्णोः प्रसादतः
श्रीशुक उवाच
विरूपः केतुमान्छम्भुरम्बरीषसुतास्त्रयः
विरूपात्पृषदश्वोऽभूत्तत्पुत्रस्तु रथीतरः
रथीतरस्याप्रजस्य भार्यायां तन्तवेऽर्थितः
अङ्गिरा जनयामास ब्रह्मवर्चस्विनः सुतान्‌
एते क्षेत्रप्रसूता वै पुनस्त्वाङ्गिरसाः स्मृताः
रथीतराणां प्रवराः क्षेत्रोपेता द्विजातयः
क्षुवतस्तु मनोर्जज्ञे इक्ष्वाकुर्घ्राणतः सुतः
तस्य पुत्रशतज्येष्ठा विकुक्षिनिमिदण्डकाः
तेषां पुरस्तादभवन्नार्यावर्ते नृपा नृप
पञ्चविंशतिः पश्चाच्च त्रयो मध्येऽपरेऽन्यतः
स एकदाष्टकाश्राद्धे इक्ष्वाकुः सुतमादिशत्‌
मांसमानीयतां मेध्यं विकुक्षे गच्छ मा चिरम्‌
तथेति स वनं गत्वा मृगान्हत्वा क्रियार्हणान्‌
श्रान्तो बुभुक्षितो वीरः शशं चाददपस्मृतिः
शेषं निवेदयामास पित्रे तेन च तद्गुरुः
चोदितः प्रोक्षणायाह दुष्टमेतदकर्मकम्‌
ज्ञात्वा पुत्रस्य तत्कर्म गुरुणाभिहितं नृपः
देशान्निःसारयामास सुतं त्यक्तविधिं रुषा
स तु विप्रेण संवादं ज्ञापकेन समाचरन्‌
त्यक्त्वा कलेवरं योगी स तेनावाप यत्परम्‌
पितर्युपरतेऽभ्येत्य विकुक्षिः पृथिवीमिमाम्‌
शासदीजे हरिं यज्ञैः शशाद इति विश्रुतः
पुरञ्जयस्तस्य सुत इन्द्रवाह इतीरितः
ककुत्स्थ इति चाप्युक्तः शृणु नामानि कर्मभिः
कृतान्त आसीत्समरो देवानां सह दानवैः
पार्ष्णिग्राहो वृतो वीरो देवैर्दैत्यपराजितैः
वचनाद्देवदेवस्य विष्णोर्विश्वात्मनः प्रभोः
वाहनत्वे वृतस्तस्य बभूवेन्द्रो महावृषः
स सन्नद्धो धनुर्दिव्यमादाय विशिखान्छितान्‌
स्तूयमानस्तमारुह्य युयुत्सुः ककुदि स्थितः
तेजसाप्यायितो विष्णोः पुरुषस्य महात्मनः
प्रतीच्यां दिशि दैत्यानां न्यरुणत्त्रिदशैः पुरम्‌
तैस्तस्य चाभूत्प्रधनं तुमुलं लोमहर्षणम्‌
यमाय भल्लैरनयद्दैत्यानभिययुर्मृधे
तस्येषुपाताभिमुखं युगान्ताग्निमिवोल्बणम्‌
विसृज्य दुद्रुवुर्दैत्या हन्यमानाः स्वमालयम्‌
जित्वा परं धनं सर्वं सस्त्रीकं वज्रपाणये
प्रत्ययच्छत्स राजर्षिरिति नामभिराहृतः
पुरञ्जयस्य पुत्रोऽभूदनेनास्तत्सुतः पृथुः
विश्वगन्धिस्ततश्चन्द्रो युवनाश्वस्तु तत्सुतः
श्रावस्तस्तत्सुतो येन श्रावस्ती निर्ममे पुरी
बृहदश्वस्तु श्रावस्तिस्ततः कुवलयाश्वकः
यः प्रियार्थमुतङ्कस्य धुन्धुनामासुरं बली
सुतानामेकविंशत्या सहस्रैरहनद्वृतः
धुन्धुमार इति ख्यातस्तत्सुतास्ते च जज्वलुः
धुन्धोर्मुखाग्निना सर्वे त्रय एवावशेषिताः
दृढाश्वः कपिलाश्वश्च भद्राश्व इति भारत
दृढाश्वपुत्रो हर्यश्वो निकुम्भस्तत्सुतः स्मृतः
बहुलाश्वो निकुम्भस्य कृशाश्वोऽथास्य सेनजित्‌
युवनाश्वोऽभवत्तस्य सोऽनपत्यो वनं गतः
भार्याशतेन निर्विण्ण ऋषयोऽस्य कृपालवः
इष्टिं स्म वर्तयां चक्रुरैन्द्रीं ते सुसमाहिताः
राजा तद्यज्ञसदनं प्रविष्टो निशि तर्षितः
दृष्ट्वा शयानान्विप्रांस्तान्पपौ मन्त्रजलं स्वयम्‌
उत्थितास्ते निशम्याथ व्युदकं कलशं प्रभो
पप्रच्छुः कस्य कर्मेदं पीतं पुंसवनं जलम्‌
राज्ञा पीतं विदित्वा वै ईश्वरप्रहितेन ते
ईश्वराय नमश्चक्रुरहो दैवबलं बलम्‌
ततः काल उपावृत्ते कुक्षिं निर्भिद्य दक्षिणम्‌
युवनाश्वस्य तनयश्चक्रवर्ती जजान ह
कं धास्यति कुमारोऽयं स्तन्ये रोरूयते भृशम्‌
मां धाता वत्स मा रोदीरितीन्द्रो देशिनीमदात्‌
न ममार पिता तस्य विप्रदेवप्रसादतः
युवनाश्वोऽथ तत्रैव तपसा सिद्धिमन्वगात्‌
त्रसद्दस्युरितीन्द्रोऽङ्ग विदधे नाम यस्य वै
यस्मात्त्रसन्ति ह्युद्विग्ना दस्यवो रावणादयः
यौवनाश्वोऽथ मान्धाता चक्रवर्त्यवनीं प्रभुः
सप्तद्वीपवतीमेकः शशासाच्युततेजसा
ईजे च यज्ञं क्रतुभिरात्मविद्भूरिदक्षिणैः
सर्वदेवमयं देवं सर्वात्मकमतीन्द्रियम्‌
द्रव्यं मन्त्रो विधिर्यज्ञो यजमानस्तथर्त्विजः
धर्मो देशश्च कालश्च सर्वमेतद्यदात्मकम्‌
यावत्सूर्य उदेति स्म यावच्च प्रतितिष्ठति
तत्सर्वं यौवनाश्वस्य मान्धातुः क्षेत्रमुच्यते
शशबिन्दोर्दुहितरि बिन्दुमत्यामधान्नृपः
पुरुकुत्समम्बरीषं मुचुकुन्दं च योगिनम्‌
तेषां स्वसारः पञ्चाशत्सौभरिं वव्रिरे पतिम्‌
यमुनान्तर्जले मग्नस्तप्यमानः परं तपः
निर्वृतिं मीनराजस्य दृष्ट्वा मैथुनधर्मिणः
जातस्पृहो नृपं विप्रः कन्यामेकामयाचत
सोऽप्याह गृह्यतां ब्रह्मन्कामं कन्या स्वयंवरे
स विचिन्त्याप्रियं स्त्रीणां जरठोऽहमसन्मतः
वलीपलित एजत्क इत्यहं प्रत्युदाहृतः
साधयिष्ये तथात्मानं सुरस्त्रीणामभीप्सितम्‌
किं पुनर्मनुजेन्द्राणामिति व्यवसितः प्रभुः
मुनिः प्रवेशितः क्षत्रा कन्यान्तःपुरमृद्धिमत्‌
वृतः स राजकन्याभिरेकं पञ्चाशता वरः
तासां कलिरभूद्भूयांस्तदर्थेऽपोह्य सौहृदम्‌
ममानुरूपो नायं व इति तद्गतचेतसाम्‌
स बह्वृचस्ताभिरपारणीय तपःश्रियानर्घ्यपरिच्छदेषु
गृहेषु नानोपवनामलाम्भः सरःसु सौगन्धिककाननेषु
महार्हशय्यासनवस्त्रभूषण स्नानानुलेपाभ्यवहारमाल्यकैः
स्वलङ्कृतस्त्रीपुरुषेषु नित्यदा रेमेऽनुगायद्द्विजभृङ्गवन्दिषु
यद्गार्हस्थ्यं तु संवीक्ष्य सप्तद्वीपवतीपतिः
विस्मितः स्तम्भमजहात्सार्वभौमश्रियान्वितम्‌
एवं गृहेष्वभिरतो विषयान्विविधैः सुखैः
सेवमानो न चातुष्यदाज्यस्तोकैरिवानलः
स कदाचिदुपासीन आत्मापह्नवमात्मनः
ददर्श बह्वृचाचार्यो मीनसङ्गसमुत्थितम्‌
अहो इमं पश्यत मे विनाशं तपस्विनः सच्चरितव्रतस्य
अन्तर्जले वारिचरप्रसङ्गात्प्रच्यावितं ब्रह्म चिरं धृतं यत्‌
सङ्गं त्यजेत मिथुनव्रतीनां मुमुक्षुः
सर्वात्मना न विसृजेद्बहिरिन्द्रियाणि
एकश्चरन्रहसि चित्तमनन्त ईशे
युञ्जीत तद्व्रतिषु साधुषु चेत्प्रसङ्गः
एकस्तपस्व्यहमथाम्भसि मत्स्यसङ्गात्‌
पञ्चाशदासमुत पञ्चसहस्रसर्गः
नान्तं व्रजाम्युभयकृत्यमनोरथानां
मायागुणैर्हृतमतिर्विषयेऽर्थभावः
एवं वसन्गृहे कालं विरक्तो न्यासमास्थितः
वनं जगामानुययुस्तत्पत्न्यः पतिदेवताः
तत्र तप्त्वा तपस्तीक्ष्णमात्मदर्शनमात्मवान्‌
सहैवाग्निभिरात्मानं युयोज परमात्मनि
ताः स्वपत्युर्महाराज निरीक्ष्याध्यात्मिकीं गतिम्‌
अन्वीयुस्तत्प्रभावेण अग्निं शान्तमिवार्चिषः
श्रीशुक उवाच
मान्धातुः पुत्रप्रवरो योऽम्बरीषः प्रकीर्तितः
पितामहेन प्रवृतो यौवनाश्वस्तु तत्सुतः
हारीतस्तस्य पुत्रोऽभून्मान्धातृप्रवरा इमे
नर्मदा भ्रातृभिर्दत्ता पुरुकुत्साय योरगैः
तया रसातलं नीतो भुजगेन्द्रप्रयुक्तया
गन्धर्वानवधीत्तत्र वध्यान्वै विष्णुशक्तिधृक्‌
नागाल्लब्धवरः सर्पादभयं स्मरतामिदम्‌
त्रसद्दस्युः पौरुकुत्सो योऽनरण्यस्य देहकृत्‌
हर्यश्वस्तत्सुतस्तस्मात्प्रारुणोऽथ त्रिबन्धनः
तस्य सत्यव्रतः पुत्रस्त्रिशङ्कुरिति विश्रुतः
प्राप्तश्चाण्डालतां शापाद्गुरोः कौशिकतेजसा
सशरीरो गतः स्वर्गमद्यापि दिवि दृश्यते
पातितोऽवाक्षिरा देवैस्तेनैव स्तम्भितो बलात्‌
त्रैशङ्कवो हरिश्चन्द्रो विश्वामित्रवसिष्ठयोः
यन्निमित्तमभूद्युद्धं पक्षिणोर्बहुवार्षिकम्‌
सोऽनपत्यो विषण्णात्मा नारदस्योपदेशतः
वरुणं शरणं यातः पुत्रो मे जायतां प्रभो
यदि वीरो महाराज तेनैव त्वां यजे इति
तथेति वरुणेनास्य पुत्रो जातस्तु रोहितः
जातः सुतो ह्यनेनाङ्ग मां यजस्वेति सोऽब्रवीत्‌
यदा पशुर्निर्दशः स्यादथ मेध्यो भवेदिति
निर्दशे च स आगत्य यजस्वेत्याह सोऽब्रवीत्‌
दन्ताः पशोर्यज्जायेरन्नथ मेध्यो भवेदिति
दन्ता जाता यजस्वेति स प्रत्याहाथ सोऽब्रवीत्‌
यदा पतन्त्यस्य दन्ता अथ मेध्यो भवेदिति
पशोर्निपतिता दन्ता यजस्वेत्याह सोऽब्रवीत्‌
यदा पशोः पुनर्दन्ता जायन्तेऽथ पशुः शुचिः
पुनर्जाता यजस्वेति स प्रत्याहाथ सोऽब्रवीत्‌
सान्नाहिको यदा राजन्राजन्योऽथ पशुः शुचिः
इति पुत्रानुरागेण स्नेहयन्त्रितचेतसा
कालं वञ्चयता तं तमुक्तो देवस्तमैक्षत
रोहितस्तदभिज्ञाय पितुः कर्म चिकीर्षितम्‌
प्राणप्रेप्सुर्धनुष्पाणिररण्यं प्रत्यपद्यत
पितरं वरुणग्रस्तं श्रुत्वा जातमहोदरम्‌
रोहितो ग्राममेयाय तमिन्द्रः प्रत्यषेधत
भूमेः पर्यटनं पुण्यं तीर्थक्षेत्रनिषेवणैः
रोहितायादिशच्छक्रः सोऽप्यरण्येऽवसत्समाम्‌
एवं द्वितीये तृतीये चतुर्थे पञ्चमे तथा
अभ्येत्याभ्येत्य स्थविरो विप्रो भूत्वाह वृत्रहा
षष्ठं संवत्सरं तत्र चरित्वा रोहितः पुरीम्‌
उपव्रजन्नजीगर्तादक्रीणान्मध्यमं सुतम्‌
शुनःशेफं पशुं पित्रे प्रदाय समवन्दत
ततः पुरुषमेधेन हरिश्चन्द्रो महायशाः
मुक्तोदरोऽयजद्देवान्वरुणादीन्महत्कथः
विश्वामित्रोऽभवत्तस्मिन्होता चाध्वर्युरात्मवान्‌
जमदग्निरभूद्ब्रह्मा वसिष्ठोऽयास्यः सामगः
तस्मै तुष्टो ददाविन्द्रः शातकौम्भमयं रथम्‌
शुनःशेफस्य माहात्म्यमुपरिष्टात्प्रचक्ष्यते
सत्यं सारं धृतिं दृष्ट्वा सभार्यस्य च भूपतेः
विश्वामित्रो भृशं प्रीतो ददावविहतां गतिम्‌
मनः पृथिव्यां तामद्भिस्तेजसापोऽनिलेन तत्‌
खे वायुं धारयंस्तच्च भूतादौ तं महात्मनि
तस्मिन्ज्ञानकलां ध्यात्वा तयाज्ञानं विनिर्दहन्‌
हित्वा तां स्वेन भावेन निर्वाणसुखसंविदा
अनिर्देश्याप्रतर्क्येण तस्थौ विध्वस्तबन्धनः
श्रीशुक उवाच
हरितो रोहितसुतश्चम्पस्तस्माद्विनिर्मिता
चम्पापुरी सुदेवोऽतो विजयो यस्य चात्मजः
भरुकस्तत्सुतस्तस्माद्वृकस्तस्यापि बाहुकः
सोऽरिभिर्हृतभू राजा सभार्यो वनमाविशत्‌
वृद्धं तं पञ्चतां प्राप्तं महिष्यनुमरिष्यती
और्वेण जानतात्मानं प्रजावन्तं निवारिता
आज्ञायास्यै सपत्नीभिर्गरो दत्तोऽन्धसा सह
सह तेनैव सञ्जातः सगराख्यो महायशाः
सगरश्चक्रवर्त्यासीत्सागरो यत्सुतैः कृतः
यस्तालजङ्घान्यवनाञ्छकान्हैहयबर्बरान्‌
नावधीद्गुरुवाक्येन चक्रे विकृतवेषिणः
मुण्डान्छ्मश्रुधरान्कांश्चिन्मुक्तकेशार्धमुण्डितान्‌
अनन्तर्वाससः कांश्चिदबहिर्वाससोऽपरान्‌
सोऽश्वमेधैरयजत सर्ववेदसुरात्मकम्‌
और्वोपदिष्टयोगेन हरिमात्मानमीश्वरम्‌
तस्योत्सृष्टं पशुं यज्ञे जहाराश्वं पुरन्दरः
सुमत्यास्तनया दृप्ताः पितुरादेशकारिणः
हयमन्वेषमाणास्ते समन्तान्न्यखनन्महीम्‌
प्रागुदीच्यां दिशि हयं ददृशुः कपिलान्तिके
एष वाजिहरश्चौर आस्ते मीलितलोचनः
हन्यतां हन्यतां पाप इति षष्टिसहस्रिणः
उदायुधा अभिययुरुन्मिमेष तदा मुनिः
स्वशरीराग्निना तावन्महेन्द्रहृतचेतसः
महद्व्यतिक्रमहता भस्मसादभवन्क्षणात्‌
न साधुवादो मुनिकोपभर्जिता नृपेन्द्रपुत्रा इति सत्त्वधामनि
कथं तमो रोषमयं विभाव्यते जगत्पवित्रात्मनि खे रजो भुवः
यस्येरिता साङ्ख्यमयी दृढेह नौर्यया मुमुक्षुस्तरते दुरत्ययम्‌
भवार्णवं मृत्युपथं विपश्चितः परात्मभूतस्य कथं पृथङ्मतिः
योऽसमञ्जस इत्युक्तः स केशिन्या नृपात्मजः
तस्य पुत्रोऽंशुमान्नाम पितामहहिते रतः
असमञ्जस आत्मानं दर्शयन्नसमञ्जसम्‌
जातिस्मरः पुरा सङ्गाद्योगी योगाद्विचालितः
आचरन्गर्हितं लोके ज्ञातीनां कर्म विप्रियम्‌
सरय्वां क्रीडतो बालान्प्रास्यदुद्वेजयन्जनम्‌
एवं वृत्तः परित्यक्तः पित्रा स्नेहमपोह्य वै
योगैश्वर्येण बालांस्तान्दर्शयित्वा ततो ययौ
अयोध्यावासिनः सर्वे बालकान्पुनरागतान्‌
दृष्ट्वा विसिस्मिरे राजन्राजा चाप्यन्वतप्यत
अंशुमांश्चोदितो राज्ञा तुरगान्वेषणे ययौ
पितृव्यखातानुपथं भस्मान्ति ददृशे हयम्‌
तत्रासीनं मुनिं वीक्ष्य कपिलाख्यमधोक्षजम्‌
अस्तौत्समाहितमनाः प्राञ्जलिः प्रणतो महान्‌
अंशुमानुवाच
न पश्यति त्वां परमात्मनोऽजनो न बुध्यतेऽद्यापि समाधियुक्तिभिः
कुतोऽपरे तस्य मनःशरीरधी विसर्गसृष्टा वयमप्रकाशाः
ये देहभाजस्त्रिगुणप्रधाना गुणान्विपश्यन्त्युत वा तमश्च
यन्मायया मोहितचेतसस्त्वां विदुः स्वसंस्थं न बहिःप्रकाशाः
तं त्वां अहं ज्ञानघनं स्वभाव प्रध्वस्तमायागुणभेदमोहैः
सनन्दनाद्यैर्मुनिभिर्विभाव्यं कथं विमूढः परिभावयामि
प्रशान्त मायागुणकर्मलिङ्गमनामरूपं सदसद्विमुक्तम्‌
ज्ञानोपदेशाय गृहीतदेहं नमामहे त्वां पुरुषं पुराणम्‌
त्वन्मायारचिते लोके वस्तुबुद्ध्या गृहादिषु
भ्रमन्ति कामलोभेर्ष्या मोहविभ्रान्तचेतसः
अद्य नः सर्वभूतात्मन्कामकर्मेन्द्रियाशयः
मोहपाशो दृढश्छिन्नो भगवंस्तव दर्शनात्‌
श्रीशुक उवाच
इत्थं गीतानुभावस्तं भगवान्कपिलो मुनिः
अंशुमन्तमुवाचेदमनुग्राह्य धिया नृप
श्रीभगवानुवाच
अश्वोऽयं नीयतां वत्स पितामहपशुस्तव
इमे च पितरो दग्धा गङ्गाम्भोऽर्हन्ति नेतरत्‌
तं परिक्रम्य शिरसा प्रसाद्य हयमानयत्‌
सगरस्तेन पशुना यज्ञशेषं समापयत्‌
राज्यमंशुमते न्यस्य निःस्पृहो मुक्तबन्धनः
और्वोपदिष्टमार्गेण लेभे गतिमनुत्तमाम्‌
श्रीशुक उवाच
अंशुमांश्च तपस्तेपे गङ्गानयनकाम्यया
कालं महान्तं नाशक्नोत्ततः कालेन संस्थितः
दिलीपस्तत्सुतस्तद्वदशक्तः कालमेयिवान्‌
भगीरथस्तस्य सुतस्तेपे स सुमहत्तपः
दर्शयामास तं देवी प्रसन्ना वरदास्मि ते
इत्युक्तः स्वमभिप्रायं शशंसावनतो नृपः
कोऽपि धारयिता वेगं पतन्त्या मे महीतले
अन्यथा भूतलं भित्त्वा नृप यास्ये रसातलम्‌
किं चाहं न भुवं यास्ये नरा मय्यामृजन्त्यघम्‌
मृजामि तदघं क्वाहं राजंस्तत्र विचिन्त्यताम्‌
श्रीभगीरथ उवाच
साधवो न्यासिनः शान्ता ब्रह्मिष्ठा लोकपावनाः
हरन्त्यघं तेऽङ्गसङ्गात्तेष्वास्ते ह्यघभिद्धरिः
धारयिष्यति ते वेगं रुद्रस्त्वात्मा शरीरिणाम्‌
यस्मिन्नोतमिदं प्रोतं विश्वं शाटीव तन्तुषु
इत्युक्त्वा स नृपो देवं तपसातोषयच्छिवम्‌
कालेनाल्पीयसा राजंस्तस्येशश्चाश्वतुष्यत
तथेति राज्ञाभिहितं सर्वलोकहितः शिवः
दधारावहितो गङ्गां पादपूतजलां हरेः
भगीरथः स राजर्षिर्निन्ये भुवनपावनीम्‌
यत्र स्वपित्णां देहा भस्मीभूताः स्म शेरते
रथेन वायुवेगेन प्रयान्तमनुधावती
देशान्पुनन्ती निर्दग्धानासिञ्चत्सगरात्मजान्‌
यज्जलस्पर्शमात्रेण ब्रह्मदण्डहता अपि
सगरात्मजा दिवं जग्मुः केवलं देहभस्मभिः
भस्मीभूताङ्गसङ्गेन स्वर्याताः सगरात्मजाः
किं पुनः श्रद्धया देवीं सेवन्ते ये धृतव्रताः
न ह्येतत्परमाश्चर्यं स्वर्धुन्या यदिहोदितम्‌
अनन्तचरणाम्भोज प्रसूताया भवच्छिदः
सन्निवेश्य मनो यस्मिञ्छ्रद्धया मुनयोऽमलाः
त्रैगुण्यं दुस्त्यजं हित्वा सद्यो यातास्तदात्मताम्‌
श्रुतो भगीरथाज्जज्ञे तस्य नाभोऽपरोऽभवत्‌
सिन्धुद्वीपस्ततस्तस्मादयुतायुस्ततोऽभवत्‌
ऋतूपर्णो नलसखो योऽश्वविद्यामयान्नलात्‌
दत्त्वाक्षहृदयं चास्मै सर्वकामस्तु तत्सुतम्‌
ततः सुदासस्तत्पुत्रो दमयन्तीपतिर्नृपः
आहुर्मित्रसहं यं वै कल्माषाङ्घ्रिमुत क्वचित्‌
वसिष्ठशापाद्रक्षोऽभूदनपत्यः स्वकर्मणा
श्रीराजोवाच
किं निमित्तो गुरोः शापः सौदासस्य महात्मनः
एतद्वेदितुमिच्छामः कथ्यतां न रहो यदि
श्रीशुक उवाच
सौदासो मृगयां किञ्चिच्चरन्रक्षो जघान ह
मुमोच भ्रातरं सोऽथ गतः प्रतिचिकीर्षया
सञ्चिन्तयन्नघं राज्ञः सूदरूपधरो गृहे
गुरवे भोक्तुकामाय पक्त्वा निन्ये नरामिषम्‌
परिवेक्ष्यमाणं भगवान्विलोक्याभक्ष्यमञ्जसा
राजानमशपत्क्रुद्धो रक्षो ह्येवं भविष्यसि
रक्षःकृतं तद्विदित्वा चक्रे द्वादशवार्षिकम्‌
सोऽप्यपोऽञ्जलिमादाय गुरुं शप्तुं समुद्यतः
वारितो मदयन्त्यापो रुशतीः पादयोर्जहौ
दिशः खमवनीं सर्वं पश्यन्जीवमयं नृपः
राक्षसं भावमापन्नः पादे कल्माषतां गतः
व्यवायकाले ददृशे वनौकोदम्पती द्विजौ
क्षुधार्तो जगृहे विप्रं तत्पत्न्याहाकृतार्थवत्‌
न भवान्राक्षसः साक्षादिक्ष्वाकूणां महारथः
मदयन्त्याः पतिर्वीर नाधर्मं कर्तुमर्हसि
देहि मेऽपत्यकामाया अकृतार्थं पतिं द्विजम्‌
देहोऽयं मानुषो राजन्पुरुषस्याखिलार्थदः
तस्मादस्य वधो वीर सर्वार्थवध उच्यते
एष हि ब्राह्मणो विद्वांस्तपःशीलगुणान्वितः
आरिराधयिषुर्ब्रह्म महापुरुषसंज्ञितम्‌
सर्वभूतात्मभावेन भूतेष्वन्तर्हितं गुणैः
सोऽयं ब्रह्मर्षिवर्यस्ते राजर्षिप्रवराद्विभो
कथमर्हति धर्मज्ञ वधं पितुरिवात्मजः
तस्य साधोरपापस्य भ्रूणस्य ब्रह्मवादिनः
कथं वधं यथा बभ्रोर्मन्यते सन्मतो भवान्‌
यद्ययं क्रियते भक्ष्यस्तर्हि मां खाद पूर्वतः
न जीविष्ये विना येन क्षणं च मृतकं यथा
एवं करुणभाषिण्या विलपन्त्या अनाथवत्‌
व्याघ्रः पशुमिवाखादत्सौदासः शापमोहितः
ब्राह्मणी वीक्ष्य दिधिषुं पुरुषादेन भक्षितम्‌
शोचन्त्यात्मानमुर्वीशमशपत्कुपिता सती
यस्मान्मे भक्षितः पाप कामार्तायाः पतिस्त्वया
तवापि मृत्युराधानादकृतप्रज्ञ दर्शितः
एवं मित्रसहं शप्त्वा पतिलोकपरायणा
तदस्थीनि समिद्धेऽग्नौ प्रास्य भर्तुर्गतिं गता
विशापो द्वादशाब्दान्ते मैथुनाय समुद्यतः
विज्ञाप्य ब्राह्मणीशापं महिष्या स निवारितः
अत ऊर्ध्वं स तत्याज स्त्रीसुखं कर्मणाप्रजाः
वसिष्ठस्तदनुज्ञातो मदयन्त्यां प्रजामधात्‌
सा वै सप्त समा गर्भमबिभ्रन्न व्यजायत
जघ्नेऽश्मनोदरं तस्याः सोऽश्मकस्तेन कथ्यते
अश्मकाद्बालिको जज्ञे यः स्त्रीभिः परिरक्षितः
नारीकवच इत्युक्तो निःक्षत्रे मूलकोऽभवत्‌
ततो दशरथस्तस्मात्पुत्र ऐडविडिस्ततः
राजा विश्वसहो यस्य खट्वाङ्गश्चक्रवर्त्यभूत्‌
यो देवैरर्थितो दैत्यानवधीद्युधि दुर्जयः
मुहूर्तमायुर्ज्ञात्वैत्य स्वपुरं सन्दधे मनः
न मे ब्रह्मकुलात्प्राणाः कुलदैवान्न चात्मजाः
न श्रियो न मही राज्यं न दाराश्चातिवल्लभाः
न बाल्येऽपि मतिर्मह्यमधर्मे रमते क्वचित्‌
नापश्यमुत्तमश्लोकादन्यत्किञ्चन वस्त्वहम्‌
देवैः कामवरो दत्तो मह्यं त्रिभुवनेश्वरैः
न वृणे तमहं कामं भूतभावनभावनः
ये विक्षिप्तेन्द्रियधियो देवास्ते स्वहृदि स्थितम्‌
न विन्दन्ति प्रियं शश्वदात्मानं किमुतापरे
अथेशमायारचितेषु सङ्गं गुणेषु गन्धर्वपुरोपमेषु
रूढं प्रकृत्यात्मनि विश्वकर्तुर्भावेन हित्वा तमहं प्रपद्ये
इति व्यवसितो बुद्ध्या नारायणगृहीतया
हित्वान्यभावमज्ञानं ततः स्वं भावमास्थितः
यत्तद्ब्रह्म परं सूक्ष्ममशून्यं शून्यकल्पितम्‌
भगवान्वासुदेवेति यं गृणन्ति हि सात्वताः
श्रीशुक उवाच
खट्वाङ्गाद्दीर्घबाहुश्च रघुस्तस्मात्पृथुश्रवाः
अजस्ततो महाराजस्तस्माद्दशरथोऽभवत्‌
तस्यापि भगवानेष साक्षाद्ब्रह्ममयो हरिः
अंशांशेन चतुर्धागात्पुत्रत्वं प्रार्थितः सुरैः
रामलक्ष्मणभरत शत्रुघ्ना इति संज्ञया
तस्यानुचरितं राजन्नृषिभिस्तत्त्वदर्शिभिः
श्रुतं हि वर्णितं भूरि त्वया सीतापतेर्मुहुः
गुर्वर्थे त्यक्तराज्यो व्यचरदनुवनं पद्मपद्भ्यां प्रियायाः
पाणिस्पर्शाक्षमाभ्यां मृजितपथरुजो यो हरीन्द्रानुजाभ्याम्‌
वैरूप्याच्छूर्पणख्याः प्रियविरहरुषारोपितभ्रूविजृम्भ
त्रस्ताब्धिर्बद्धसेतुः खलदवदहनः कोसलेन्द्रोऽवतान्नः
विश्वामित्राध्वरे येन मारीचाद्या निशाचराः
पश्यतो लक्ष्मणस्यैव हता नैरृतपुङ्गवाः
यो लोकवीरसमितौ धनुरैशमुग्रं
सीतास्वयंवरगृहे त्रिशतोपनीतम्‌
आदाय बालगजलील इवेक्षुयष्टिं
सज्ज्यीकृतं नृप विकृष्य बभञ्ज मध्ये
जित्वानुरूपगुणशीलवयोऽङ्गरूपां
सीताभिधां श्रियमुरस्यभिलब्धमानाम्‌
मार्गे व्रजन्भृगुपतेर्व्यनयत्प्ररूढं
दर्पं महीमकृत यस्त्रिरराजबीजाम्‌
यः सत्यपाशपरिवीतपितुर्निदेशं
स्त्रैणस्य चापि शिरसा जगृहे सभार्यः
राज्यं श्रियं प्रणयिनः सुहृदो निवासं
त्यक्त्वा ययौ वनमसूनिव मुक्तसङ्गः
रक्षःस्वसुर्व्यकृत रूपमशुद्धबुद्धेस्‌
तस्याः खरत्रिशिरदूषणमुख्यबन्धून्‌
जघ्ने चतुर्दशसहस्रमपारणीय
कोदण्डपाणिरटमान उवास कृच्छ्रम्‌
सीताकथाश्रवणदीपितहृच्छयेन
सृष्टं विलोक्य नृपते दशकन्धरेण
जघ्नेऽद्भुतैणवपुषाश्रमतोऽपकृष्टो
मारीचमाशु विशिखेन यथा कमुग्रः
रक्षोऽधमेन वृकवद्विपिनेऽसमक्षं
वैदेहराजदुहितर्यपयापितायाम्‌
भ्रात्रा वने कृपणवत्प्रियया वियुक्तः
स्त्रीसङ्गिनां गतिमिति प्रथयंश्चचार
दग्ध्वात्मकृत्यहतकृत्यमहन्कबन्धं
सख्यं विधाय कपिभिर्दयितागतिं तैः
बुद्ध्वाथ वालिनि हते प्लवगेन्द्रसैन्यैर्‌
वेलामगात्स मनुजोऽजभवार्चिताङ्घ्रिः
यद्रोषविभ्रमविवृत्तकटाक्षपात
सम्भ्रान्तनक्रमकरो भयगीर्णघोषः
सिन्धुः शिरस्यर्हणं परिगृह्य रूपी
पादारविन्दमुपगम्य बभाष एतत्‌
न त्वां वयं जडधियो नु विदाम भूमन्‌
कूटस्थमादिपुरुषं जगतामधीशम्‌
यत्सत्त्वतः सुरगणा रजसः प्रजेशा
मन्योश्च भूतपतयः स भवान्गुणेशः
कामं प्रयाहि जहि विश्रवसोऽवमेहं
त्रैलोक्यरावणमवाप्नुहि वीर पत्नीम्‌
बध्नीहि सेतुमिह ते यशसो वितत्यै
गायन्ति दिग्विजयिनो यमुपेत्य भूपाः
बद्ध्वोदधौ रघुपतिर्विविधाद्रिकूटैः
सेतुं कपीन्द्रकरकम्पितभूरुहाङ्गैः
सुग्रीवनीलहनुमत्प्रमुखैरनीकैर्‌
लङ्कां विभीषणदृशाविशदग्रदग्धाम्‌
सा वानरेन्द्रबलरुद्धविहारकोष्ठ
श्रीद्वारगोपुरसदोवलभीविटङ्का
निर्भज्यमानधिषणध्वजहेमकुम्भ
शृङ्गाटका गजकुलैर्ह्रदिनीव घूर्णा
रक्षःपतिस्तदवलोक्य निकुम्भकुम्भ
धूम्राक्षदुर्मुखसुरान्तकनरान्तकादीन्‌
पुत्रं प्रहस्तमतिकायविकम्पनादीन्‌
सर्वानुगान्समहिनोदथ कुम्भकर्णम्‌
तां यातुधानपृतनामसिशूलचाप
प्रासर्ष्टिशक्तिशरतोमरखड्गदुर्गाम्‌
सुग्रीवलक्ष्मणमरुत्सुतगन्धमाद
नीलाङ्गदर्क्षपनसादिभिरन्वितोऽगात्‌
तेऽनीकपा रघुपतेरभिपत्य सर्वे
द्वन्द्वं वरूथमिभपत्तिरथाश्वयोधैः
जघ्नुर्द्रुमैर्गिरिगदेषुभिरङ्गदाद्याः
सीताभिमर्षहतमङ्गलरावणेशान्‌
रक्षःपतिः स्वबलनष्टिमवेक्ष्य रुष्ट
आरुह्य यानकमथाभिससार रामम्‌
स्वःस्यन्दने द्युमति मातलिनोपनीते
विभ्राजमानमहनन्निशितैः क्षुरप्रैः
रामस्तमाह पुरुषादपुरीष यन्नः
कान्तासमक्षमसतापहृता श्ववत्ते
त्यक्तत्रपस्य फलमद्य जुगुप्सितस्य
यच्छामि काल इव कर्तुरलङ्घ्यवीर्यः
एवं क्षिपन्धनुषि सन्धितमुत्ससर्ज
बाणं स वज्रमिव तद्धृदयं बिभेद
सोऽसृग्वमन्दशमुखैर्न्यपतद्विमानाद्‌
धाहेति जल्पति जने सुकृतीव रिक्तः
ततो निष्क्रम्य लङ्काया यातुधान्यः सहस्रशः
मन्दोदर्या समं तत्र प्ररुदन्त्य उपाद्रवन्‌
स्वान्स्वान्बन्धून्परिष्वज्य लक्ष्मणेषुभिरर्दितान्‌
रुरुदुः सुस्वरं दीना घ्नन्त्य आत्मानमात्मना
हा हताः स्म वयं नाथ लोकरावण रावण
कं यायाच्छरणं लङ्का त्वद्विहीना परार्दिता
न वै वेद महाभाग भवान्कामवशं गतः
तेजोऽनुभावं सीताया येन नीतो दशामिमाम्‌
कृतैषा विधवा लङ्का वयं च कुलनन्दन
देहः कृतोऽन्नं गृध्राणामात्मा नरकहेतवे
श्रीशुक उवाच
स्वानां विभीषणश्चक्रे कोसलेन्द्रानुमोदितः
पितृमेधविधानेन यदुक्तं साम्परायिकम्‌
ततो ददर्श भगवानशोकवनिकाश्रमे
क्षामां स्वविरहव्याधिं शिंशपामूलमाश्रिताम्‌
रामः प्रियतमां भार्यां दीनां वीक्ष्यान्वकम्पत
आत्मसन्दर्शनाह्लाद विकसन्मुखपङ्कजाम्‌
आरोप्यारुरुहे यानं भ्रातृभ्यां हनुमद्युतः
विभीषणाय भगवान्दत्त्वा रक्षोगणेशताम्‌
लङ्कामायुश्च कल्पान्तं ययौ चीर्णव्रतः पुरीम्‌
अवकीर्यमाणः सुकुसुमैर्लोकपालार्पितैः पथि
उपगीयमानचरितः शतधृत्यादिभिर्मुदा
गोमूत्रयावकं श्रुत्वा भ्रातरं वल्कलाम्बरम्‌
महाकारुणिकोऽतप्यज्जटिलं स्थण्डिलेशयम्‌
भरतः प्राप्तमाकर्ण्य पौरामात्यपुरोहितैः
पादुके शिरसि न्यस्य रामं प्रत्युद्यतोऽग्रजम्‌
नन्दिग्रामात्स्वशिबिराद्गीतवादित्रनिःस्वनैः
ब्रह्मघोषेण च मुहुः पठद्भिर्ब्रह्मवादिभिः
स्वर्णकक्षपताकाभिर्हैमैश्चित्रध्वजै रथैः
सदश्वै रुक्मसन्नाहैर्भटैः पुरटवर्मभिः
श्रेणीभिर्वारमुख्याभिर्भृत्यैश्चैव पदानुगैः
पारमेष्ठ्यान्युपादाय पण्यान्युच्चावचानि च
पादयोर्न्यपतत्प्रेम्णा प्रक्लिन्नहृदयेक्षणः
पादुके न्यस्य पुरतः प्राञ्जलिर्बाष्पलोचनः
तमाश्लिष्य चिरं दोर्भ्यां स्नापयन्नेत्रजैर्जलैः
रामो लक्ष्मणसीताभ्यां विप्रेभ्यो येऽर्हसत्तमाः
तेभ्यः स्वयं नमश्चक्रे प्रजाभिश्च नमस्कृतः
धुन्वन्त उत्तरासङ्गान्पतिं वीक्ष्य चिरागतम्‌
उत्तराः कोसला माल्यैः किरन्तो ननृतुर्मुदा
पादुके भरतोऽगृह्णाच्चामरव्यजनोत्तमे
विभीषणः ससुग्रीवः श्वेतच्छत्रं मरुत्सुतः
धनुर्निषङ्गान्छत्रुघ्नः सीता तीर्थकमण्डलुम्‌
अबिभ्रदङ्गदः खड्गं हैमं चर्मर्क्षराण्नृप
पुष्पकस्थो नुतः स्त्रीभिः स्तूयमानश्च वन्दिभिः
विरेजे भगवान्राजन्ग्रहैश्चन्द्र इवोदितः
भ्रात्राभिनन्दितः सोऽथ सोत्सवां प्राविशत्पुरीम्‌
प्रविश्य राजभवनं गुरुपत्नीः स्वमातरम्‌
गुरून्वयस्यावरजान्पूजितः प्रत्यपूजयत्‌
वैदेही लक्ष्मणश्चैव यथावत्समुपेयतुः
पुत्रान्स्वमातरस्तास्तु प्राणांस्तन्व इवोत्थिताः
आरोप्याङ्केऽभिषिञ्चन्त्यो बाष्पौघैर्विजहुः शुचः
जटा निर्मुच्य विधिवत्कुलवृद्धैः समं गुरुः
अभ्यषिञ्चद्यथैवेन्द्रं चतुःसिन्धुजलादिभिः
एवं कृतशिरःस्नानः सुवासाः स्रग्व्यलङ्कृतः
स्वलङ्कृतैः सुवासोभिर्भ्रातृभिर्भार्यया बभौ
अग्रहीदासनं भ्रात्रा प्रणिपत्य प्रसादितः
प्रजाः स्वधर्मनिरता वर्णाश्रमगुणान्विताः
जुगोप पितृवद्रामो मेनिरे पितरं च तम्‌
त्रेतायां वर्तमानायां कालः कृतसमोऽभवत्‌
रामे राजनि धर्मज्ञे सर्वभूतसुखावहे
वनानि नद्यो गिरयो वर्षाणि द्वीपसिन्धवः
सर्वे कामदुघा आसन्प्रजानां भरतर्षभ
नाधिव्याधिजराग्लानि दुःखशोकभयक्लमाः
मृत्युश्चानिच्छतां नासीद्रामे राजन्यधोक्षजे
एकपत्नीव्रतधरो राजर्षिचरितः शुचिः
स्वधर्मं गृहमेधीयं शिक्षयन्स्वयमाचरत्‌
प्रेम्णानुवृत्त्या शीलेन प्रश्रयावनता सती
भिया ह्रिया च भावज्ञा भर्तुः सीताहरन्मनः
श्रीशुक उवाच
भगवानात्मनात्मानं राम उत्तमकल्पकैः
सर्वदेवमयं देवमीजेऽथाचार्यवान्मखैः
होत्रेऽददाद्दिशं प्राचीं ब्रह्मणे दक्षिणां प्रभुः
अध्वर्यवे प्रतीचीं वा उत्तरां सामगाय सः
आचार्याय ददौ शेषां यावती भूस्तदन्तरा
अन्यमान इदं कृत्स्नं ब्राह्मणोऽर्हति निःस्पृहः
इत्ययं तदलङ्कार वासोभ्यामवशेषितः
तथा राज्ञ्यपि वैदेही सौमङ्गल्यावशेषिता
ते तु ब्राह्मणदेवस्य वात्सल्यं वीक्ष्य संस्तुतम्‌
प्रीताः क्लिन्नधियस्तस्मै प्रत्यर्प्येदं बभाषिरे
अप्रत्तं नस्त्वया किं नु भगवन्भुवनेश्वर
यन्नोऽन्तर्हृदयं विश्य तमो हंसि स्वरोचिषा
नमो ब्रह्मण्यदेवाय रामायाकुण्ठमेधसे
उत्तमश्लोकधुर्याय न्यस्तदण्डार्पिताङ्घ्रये
कदाचिल्लोकजिज्ञासुर्गूढो रात्र्यामलक्षितः
चरन्वाचोऽशृणोद्रामो भार्यामुद्दिश्य कस्यचित्‌
नाहं बिभर्मि त्वां दुष्टामसतीं परवेश्मगाम्‌
स्त्रैणो हि बिभृयात्सीतां रामो नाहं भजे पुनः
इति लोकाद्बहुमुखाद्दुराराध्यादसंविदः
पत्या भीतेन सा त्यक्ता प्राप्ता प्राचेतसाश्रमम्‌
अन्तर्वत्न्यागते काले यमौ सा सुषुवे सुतौ
कुशो लव इति ख्यातौ तयोश्चक्रे क्रिया मुनिः
अङ्गदश्चित्रकेतुश्च लक्ष्मणस्यात्मजौ स्मृतौ
तक्षः पुष्कल इत्यास्तां भरतस्य महीपते
सुबाहुः श्रुतसेनश्च शत्रुघ्नस्य बभूवतुः
गन्धर्वान्कोटिशो जघ्ने भरतो विजये दिशाम्‌
तदीयं धनमानीय सर्वं राज्ञे न्यवेदयत्‌
शत्रुघ्नश्च मधोः पुत्रं लवणं नाम राक्षसम्‌
हत्वा मधुवने चक्रे मथुरां नाम वै पुरीम्‌
मुनौ निक्षिप्य तनयौ सीता भर्त्रा विवासिता
ध्यायन्ती रामचरणौ विवरं प्रविवेश ह
तच्छ्रुत्वा भगवान्रामो रुन्धन्नपि धिया शुचः
स्मरंस्तस्या गुणांस्तांस्तान्नाशक्नोद्रोद्धुमीश्वरः
स्त्रीपुंप्रसङ्ग एतादृक्सर्वत्र त्रासमावहः
अपीश्वराणां किमुत ग्राम्यस्य गृहचेतसः
तत ऊर्ध्वं ब्रह्मचर्यं धार्यन्नजुहोत्प्रभुः
त्रयोदशाब्दसाहस्रमग्निहोत्रमखण्डितम्‌
स्मरतां हृदि विन्यस्य विद्धं दण्डककण्टकैः
स्वपादपल्लवं राम आत्मज्योतिरगात्ततः
नेदं यशो रघुपतेः सुरयाच्ञयात्त
लीलातनोरधिकसाम्यविमुक्तधाम्नः
रक्षोवधो जलधिबन्धनमस्त्रपूगैः
किं तस्य शत्रुहनने कपयः सहायाः
यस्यामलं नृपसदःसु यशोऽधुनापि
गायन्त्यघघ्नमृषयो दिगिभेन्द्रपट्टम्‌
तं नाकपालवसुपालकिरीटजुष्ट
पादाम्बुजं रघुपतिं शरणं प्रपद्ये
स यैः स्पृष्टोऽभिदृष्टो वा संविष्टोऽनुगतोऽपि वा
कोसलास्ते ययुः स्थानं यत्र गच्छन्ति योगिनः
पुरुषो रामचरितं श्रवणैरुपधारयन्‌
आनृशंस्यपरो राजन्कर्मबन्धैर्विमुच्यते
श्रीराजोवाच
कथं स भगवान्रामो भ्रात्न्वा स्वयमात्मनः
तस्मिन्वा तेऽन्ववर्तन्त प्रजाः पौराश्च ईश्वरे
श्रीबादरायणिरुवाच
अथादिशद्दिग्विजये भ्रात्‌ंस्त्रिभुवनेश्वरः
आत्मानं दर्शयन्स्वानां पुरीमैक्षत सानुगः
आसिक्तमार्गां गन्धोदैः करिणां मदशीकरैः
स्वामिनं प्राप्तमालोक्य मत्तां वा सुतरामिव
प्रासादगोपुरसभा चैत्यदेवगृहादिषु
विन्यस्तहेमकलशैः पताकाभिश्च मण्डिताम्‌
पूगैः सवृन्तै रम्भाभिः पट्टिकाभिः सुवाससाम्‌
आदर्शैरंशुकैः स्रग्भिः कृतकौतुकतोरणाम्‌
तमुपेयुस्तत्र तत्र पौरा अर्हणपाणयः
आशिषो युयुजुर्देव पाहीमां प्राक्त्वयोद्धृताम्‌
ततः प्रजा वीक्ष्य पतिं चिरागतं दिदृक्षयोत्सृष्टगृहाः स्त्रियो नराः
आरुह्य हर्म्याण्यरविन्दलोचनमतृप्तनेत्राः कुसुमैरवाकिरन्‌
अथ प्रविष्टः स्वगृहं जुष्टं स्वैः पूर्वराजभिः
अनन्ताखिलकोषाढ्यमनर्घ्योरुपरिच्छदम्‌
विद्रुमोदुम्बरद्वारैर्वैदूर्यस्तम्भपङ्क्तिभिः
स्थलैर्मारकतैः स्वच्छैर्भ्राजत्स्फटिकभित्तिभिः
चित्रस्रग्भिः पट्टिकाभिर्वासोमणिगणांशुकैः
मुक्ताफलैश्चिदुल्लासैः कान्तकामोपपत्तिभिः
धूपदीपैः सुरभिभिर्मण्डितं पुष्पमण्डनैः
स्त्रीपुम्भिः सुरसङ्काशैर्जुष्टं भूषणभूषणैः
तस्मिन्स भगवान्रामः स्निग्धया प्रिययेष्टया
रेमे स्वारामधीराणामृषभः सीतया किल
बुभुजे च यथाकालं कामान्धर्ममपीडयन्‌
वर्षपूगान्बहून्न्ऱ्‌^ईणामभिध्याताङ्घ्रिपल्लवः
श्रीशुक उवाच
कुशस्य चातिथिस्तस्मान्निषधस्तत्सुतो नभः
पुण्डरीकोऽथ तत्पुत्रः क्षेमधन्वाभवत्ततः
देवानीकस्ततोऽनीहः पारियात्रोऽथ तत्सुतः
ततो बलस्थलस्तस्माद्वज्रनाभोऽर्कसम्भवः
सगणस्तत्सुतस्तस्माद्विधृतिश्चाभवत्सुतः
ततो हिरण्यनाभोऽभूद्योगाचार्यस्तु जैमिनेः
शिष्यः कौशल्य आध्यात्मं याज्ञवल्क्योऽध्यगाद्यतः
योगं महोदयमृषिर्हृदयग्रन्थिभेदकम्‌
पुष्पो हिरण्यनाभस्य ध्रुवसन्धिस्ततोऽभवत्‌
सुदर्शनोऽथाग्निवर्णः शीघ्रस्तस्य मरुः सुतः
सोऽसावास्ते योगसिद्धः कलापग्राममास्थितः
कलेरन्ते सूर्यवंशं नष्टं भावयिता पुनः
तस्मात्प्रसुश्रुतस्तस्य सन्धिस्तस्याप्यमर्षणः
महस्वांस्तत्सुतस्तस्माद्विश्वबाहुरजायत
ततः प्रसेनजित्तस्मात्तक्षको भविता पुनः
ततो बृहद्बलो यस्तु पित्रा ते समरे हतः
एते हीक्ष्वाकुभूपाला अतीताः शृण्वनागतान्‌
बृहद्बलस्य भविता पुत्रो नाम्ना बृहद्रणः
ऊरुक्रियः सुतस्तस्य वत्सवृद्धो भविष्यति
प्रतिव्योमस्ततो भानुर्दिवाको वाहिनीपतिः
सहदेवस्ततो वीरो बृहदश्वोऽथ भानुमान्‌
प्रतीकाश्वो भानुमतः सुप्रतीकोऽथ तत्सुतः
भविता मरुदेवोऽथ सुनक्षत्रोऽथ पुष्करः
तस्यान्तरिक्षस्तत्पुत्रः सुतपास्तदमित्रजित्‌
बृहद्राजस्तु तस्यापि बर्हिस्तस्मात्कृतञ्जयः
रणञ्जयस्तस्य सुतः सञ्जयो भविता ततः
तस्माच्छाक्योऽथ शुद्धोदो लाङ्गलस्तत्सुतः स्मृतः
ततः प्रसेनजित्तस्मात्क्षुद्रको भविता ततः
रणको भविता तस्मात्सुरथस्तनयस्ततः
सुमित्रो नाम निष्ठान्त एते बार्हद्बलान्वयाः
इक्ष्वाकूणामयं वंशः सुमित्रान्तो भविष्यति
यतस्तं प्राप्य राजानं संस्थां प्राप्स्यति वै कलौ
श्रीशुक उवाच
निमिरिक्ष्वाकुतनयो वसिष्ठमवृतर्त्विजम्‌
आरभ्य सत्रं सोऽप्याह शक्रेण प्राग्वृतोऽस्मि भोः
तं निर्वर्त्यागमिष्यामि तावन्मां प्रतिपालय
तूष्णीमासीद्गृहपतिः सोऽपीन्द्रस्याकरोन्मखम्‌
निमित्तश्चलमिदं विद्वान्सत्रमारभतामात्मवान्‌
ऋत्विग्भिरपरैस्तावन्नागमद्यावता गुरुः
शिष्यव्यतिक्रमं वीक्ष्य तं निर्वर्त्यागतो गुरुः
अशपत्पतताद्देहो निमेः पण्डितमानिनः
निमिः प्रतिददौ शापं गुरवेऽधर्मवर्तिने
तवापि पतताद्देहो लोभाद्धर्ममजानतः
इत्युत्ससर्ज स्वं देहं निमिरध्यात्मकोविदः
मित्रावरुणयोर्जज्ञे उर्वश्यां प्रपितामहः
गन्धवस्तुषु तद्देहं निधाय मुनिसत्तमाः
समाप्ते सत्रयागे च देवानूचुः समागतान्‌
राज्ञो जीवतु देहोऽयं प्रसन्नाः प्रभवो यदि
तथेत्युक्ते निमिः प्राह मा भून्मे देहबन्धनम्‌
यस्य योगं न वाञ्छन्ति वियोगभयकातराः
भजन्ति चरणाम्भोजं मुनयो हरिमेधसः
देहं नावरुरुत्सेऽहं दुःखशोकभयावहम्‌
सर्वत्रास्य यतो मृत्युर्मत्स्यानामुदके यथा
देवा ऊचुः
विदेह उष्यतां कामं लोचनेषु शरीरिणाम्‌
उन्मेषणनिमेषाभ्यां लक्षितोऽध्यात्मसंस्थितः
अराजकभयं न्णां मन्यमाना महर्षयः
देहं ममन्थुः स्म निमेः कुमारः समजायत
जन्मना जनकः सोऽभूद्वैदेहस्तु विदेहजः
मिथिलो मथनाज्जातो मिथिला येन निर्मिता
तस्मादुदावसुस्तस्य पुत्रोऽभून्नन्दिवर्धनः
ततः सुकेतुस्तस्यापि देवरातो महीपते
तस्माद्बृहद्रथस्तस्य महावीर्यः सुधृत्पिता
सुधृतेर्धृष्टकेतुर्वै हर्यश्वोऽथ मरुस्ततः
मरोः प्रतीपकस्तस्माज्जातः कृतरथो यतः
देवमीढस्तस्य पुत्रो विश्रुतोऽथ महाधृतिः
कृतिरातस्ततस्तस्मान्महारोमा च तत्सुतः
स्वर्णरोमा सुतस्तस्य ह्रस्वरोमा व्यजायत
ततः शीरध्वजो जज्ञे यज्ञार्थं कर्षतो महीम्‌
सीता शीराग्रतो जाता तस्मात्शीरध्वजः स्मृतः
कुशध्वजस्तस्य पुत्रस्ततो धर्मध्वजो नृपः
धर्मध्वजस्य द्वौ पुत्रौ कृतध्वजमितध्वजौ
कृतध्वजात्केशिध्वजः खाण्डिक्यस्तु मितध्वजात्‌
कृतध्वजसुतो राजन्नात्मविद्याविशारदः
खाण्डिक्यः कर्मतत्त्वज्ञो भीतः केशिध्वजाद्द्रुतः
भानुमांस्तस्य पुत्रोऽभूच्छतद्युम्नस्तु तत्सुतः
शुचिस्तु तनयस्तस्मात्सनद्वाजः सुतोऽभवत्‌
ऊर्जकेतुः सनद्वाजादजोऽथ पुरुजित्सुतः
अरिष्टनेमिस्तस्यापि श्रुतायुस्तत्सुपार्श्वकः
ततश्चित्ररथो यस्य क्षेमाधिर्मिथिलाधिपः
तस्मात्समरथस्तस्य सुतः सत्यरथस्ततः
आसीदुपगुरुस्तस्मादुपगुप्तोऽग्निसम्भवः
वस्वनन्तोऽथ तत्पुत्रो युयुधो यत्सुभाषणः
श्रुतस्ततो जयस्तस्माद्विजयोऽस्मादृतः सुतः
शुनकस्तत्सुतो जज्ञे वीतहव्यो धृतिस्ततः
बहुलाश्वो धृतेस्तस्य कृतिरस्य महावशी
एते वै मैथिला राजन्नात्मविद्याविशारदाः
योगेश्वरप्रसादेन द्वन्द्वैर्मुक्ता गृहेष्वपि
श्रीशुक उवाच
अथातः श्रूयतां राजन्वंशः सोमस्य पावनः
यस्मिन्नैलादयो भूपाः कीर्त्यन्ते पुण्यकीर्तयः
सहस्रशिरसः पुंसो नाभिह्रदसरोरुहात्‌
जातस्यासीत्सुतो धातुरत्रिः पितृसमो गुणैः
तस्य दृग्भ्योऽभवत्पुत्रः सोमोऽमृतमयः किल
विप्रौषध्युडुगणानां ब्रह्मणा कल्पितः पतिः
सोऽयजद्राजसूयेन विजित्य भुवनत्रयम्‌
पत्नीं बृहस्पतेर्दर्पात्तारां नामाहरद्बलात्‌
यदा स देवगुरुणा याचितोऽभीक्ष्णशो मदात्‌
नात्यजत्तत्कृते जज्ञे सुरदानवविग्रहः
शुक्रो बृहस्पतेर्द्वेषादग्रहीत्सासुरोडुपम्‌
हरो गुरुसुतं स्नेहात्सर्वभूतगणावृतः
सर्वदेवगणोपेतो महेन्द्रो गुरुमन्वयात्‌
सुरासुरविनाशोऽभूत्समरस्तारकामयः
निवेदितोऽथाङ्गिरसा सोमं निर्भर्त्स्य विश्वकृत्‌
तारां स्वभर्त्रे प्रायच्छदन्तर्वत्नीमवैत्पतिः
त्यज त्यजाशु दुष्प्रज्ञे मत्क्षेत्रादाहितं परैः
नाहं त्वां भस्मसात्कुर्यां स्त्रियं सान्तानिकेऽसति
तत्याज व्रीडिता तारा कुमारं कनकप्रभम्‌
स्पृहामाङ्गिरसश्चक्रे कुमारे सोम एव च
ममायं न तवेत्युच्चैस्तस्मिन्विवदमानयोः
पप्रच्छुरृषयो देवा नैवोचे व्रीडिता तु सा
कुमारो मातरं प्राह कुपितोऽलीकलज्जया
किं न वचस्यसद्वृत्ते आत्मावद्यं वदाशु मे
ब्रह्मा तां रह आहूय समप्राक्षीच्च सान्त्वयन्‌
सोमस्येत्याह शनकैः सोमस्तं तावदग्रहीत्‌
तस्यात्मयोनिरकृत बुध इत्यभिधां नृप
बुद्ध्या गम्भीरया येन पुत्रेणापोडुराण्मुदम्‌
ततः पुरूरवा जज्ञे इलायां य उदाहृतः
तस्य रूपगुणौदार्य शीलद्रविणविक्रमान्‌
श्रुत्वोर्वशीन्द्रभवने गीयमानान्सुरर्षिणा
तदन्तिकमुपेयाय देवी स्मरशरार्दिता
मित्रावरुणयोः शापादापन्ना नरलोकताम्‌
निशम्य पुरुषश्रेष्ठं कन्दर्पमिव रूपिणम्‌
धृतिं विष्टभ्य ललना उपतस्थे तदन्तिके
स तां विलोक्य नृपतिर्हर्षेणोत्फुल्ललोचनः
उवाच श्लक्ष्णया वाचा देवीं हृष्टतनूरुहः
श्रीराजोवाच
स्वागतं ते वरारोहे आस्यतां करवाम किम्‌
संरमस्व मया साकं रतिर्नौ शाश्वतीः समाः
उर्वश्युवाच
कस्यास्त्वयि न सज्जेत मनो दृष्टिश्च सुन्दर
यदङ्गान्तरमासाद्य च्यवते ह रिरंसया
एतावुरणकौ राजन्न्यासौ रक्षस्व मानद
संरंस्ये भवता साकं श्लाघ्यः स्त्रीणां वरः स्मृतः
घृतं मे वीर भक्ष्यं स्यान्नेक्षे त्वान्यत्र मैथुनात्‌
विवाससं तत्तथेति प्रतिपेदे महामनाः
अहो रूपमहो भावो नरलोकविमोहनम्‌
को न सेवेत मनुजो देवीं त्वां स्वयमागताम्‌
तया स पुरुषश्रेष्ठो रमयन्त्या यथार्हतः
रेमे सुरविहारेषु कामं चैत्ररथादिषु
रममाणस्तया देव्या पद्मकिञ्जल्कगन्धया
तन्मुखामोदमुषितो मुमुदेऽहर्गणान्बहून्‌
अपश्यन्नुर्वशीमिन्द्रो गन्धर्वान्समचोदयत्‌
उर्वशीरहितं मह्यमास्थानं नातिशोभते
ते उपेत्य महारात्रे तमसि प्रत्युपस्थिते
उर्वश्या उरणौ जह्रुर्न्यस्तौ राजनि जायया
निशम्याक्रन्दितं देवी पुत्रयोर्नीयमानयोः
हतास्म्यहं कुनाथेन नपुंसा वीरमानिना
यद्विश्रम्भादहं नष्टा हृतापत्या च दस्युभिः
यः शेते निशि सन्त्रस्तो यथा नारी दिवा पुमान्‌
इति वाक्सायकैर्बिद्धः प्रतोत्त्रैरिव कुञ्जरः
निशि निस्त्रिंशमादाय विवस्त्रोऽभ्यद्रवद्रुषा
ते विसृज्योरणौ तत्र व्यद्योतन्त स्म विद्युतः
आदाय मेषावायान्तं नग्नमैक्षत सा पतिम्‌
ऐलोऽपि शयने जायामपश्यन्विमना इव
तच्चित्तो विह्वलः शोचन्बभ्रामोन्मत्तवन्महीम्‌
स तां वीक्ष्य कुरुक्षेत्रे सरस्वत्यां च तत्सखीः
पञ्च प्रहृष्टवदनः प्राह सूक्तं पुरूरवाः
अहो जाये तिष्ठ तिष्ठ घोरे न त्यक्तुमर्हसि
मां त्वमद्याप्यनिर्वृत्य वचांसि कृणवावहै
सुदेहोऽयं पतत्यत्र देवि दूरं हृतस्त्वया
खादन्त्येनं वृका गृध्रास्त्वत्प्रसादस्य नास्पदम्‌
उर्वश्युवाच
मा मृथाः पुरुषोऽसि त्वं मा स्म त्वाद्युर्वृका इमे
क्वापि सख्यं न वै स्त्रीणां वृकाणां हृदयं यथा
स्त्रियो ह्यकरुणाः क्रूरा दुर्मर्षाः प्रियसाहसाः
घ्नन्त्यल्पार्थेऽपि विश्रब्धं पतिं भ्रातरमप्युत
विधायालीकविश्रम्भमज्ञेषु त्यक्तसौहृदाः
नवं नवमभीप्सन्त्यः पुंश्चल्यः स्वैरवृत्तयः
संवत्सरान्ते हि भवानेकरात्रं मयेश्वरः
रंस्यत्यपत्यानि च ते भविष्यन्त्यपराणि भोः
अन्तर्वत्नीमुपालक्ष्य देवीं स प्रययौ पुरीम्‌
पुनस्तत्र गतोऽब्दान्ते उर्वशीं वीरमातरम्‌
उपलभ्य मुदा युक्तः समुवास तया निशाम्‌
अथैनमुर्वशी प्राह कृपणं विरहातुरम्‌
गन्धर्वानुपधावेमांस्तुभ्यं दास्यन्ति मामिति
तस्य संस्तुवतस्तुष्टा अग्निस्थालीं ददुर्नृप
उर्वशीं मन्यमानस्तां सोऽबुध्यत चरन्वने
स्थालीं न्यस्य वने गत्वा गृहानाध्यायतो निशि
त्रेतायां सम्प्रवृत्तायां मनसि त्रय्यवर्तत
स्थालीस्थानं गतोऽश्वत्थं शमीगर्भं विलक्ष्य सः
तेन द्वे अरणी कृत्वा उर्वशीलोककाम्यया
उर्वशीं मन्त्रतो ध्यायन्नधरारणिमुत्तराम्‌
आत्मानमुभयोर्मध्ये यत्तत्प्रजननं प्रभुः
तस्य निर्मन्थनाज्जातो जातवेदा विभावसुः
त्रय्या स विद्यया राज्ञा पुत्रत्वे कल्पितस्त्रिवृत्‌
तेनायजत यज्ञेशं भगवन्तमधोक्षजम्‌
उर्वशीलोकमन्विच्छन्सर्वदेवमयं हरिम्‌
एक एव पुरा वेदः प्रणवः सर्ववाङ्मयः
देवो नारायणो नान्य एकोऽग्निर्वर्ण एव च
पुरूरवस एवासीत्त्रयी त्रेतामुखे नृप
अग्निना प्रजया राजा लोकं गान्धर्वमेयिवान्‌
श्रीबादरायणिरुवाच
ऐलस्य चोर्वशीगर्भात्षडासन्नात्मजा नृप
आयुः श्रुतायुः सत्यायू रयोऽथ विजयो जयः
श्रुतायोर्वसुमान्पुत्रः सत्यायोश्च श्रुतञ्जयः
रयस्य सुत एकश्च जयस्य तनयोऽमितः
भीमस्तु विजयस्याथ काञ्चनो होत्रकस्ततः
तस्य जह्नुः सुतो गङ्गां गण्डूषीकृत्य योऽपिबत्‌
जह्नोस्तु पुरुस्तस्याथ बलाकश्चात्मजोऽजकः
ततः कुशः कुशस्यापि कुशाम्बुस्तनयो वसुः
कुशनाभश्च चत्वारो गाधिरासीत्कुशाम्बुजः
तस्य सत्यवतीं कन्यामृचीकोऽयाचत द्विजः
वरं विसदृशं मत्वा गाधिर्भार्गवमब्रवीत्‌
एकतः श्यामकर्णानां हयानां चन्द्रवर्चसाम्‌
सहस्रं दीयतां शुल्कं कन्यायाः कुशिका वयम्‌
इत्युक्तस्तन्मतं ज्ञात्वा गतः स वरुणान्तिकम्‌
आनीय दत्त्वा तानश्वानुपयेमे वराननाम्‌
स ऋषिः प्रार्थितः पत्न्या श्वश्र्वा चापत्यकाम्यया
श्रपयित्वोभयैर्मन्त्रैश्चरुं स्नातुं गतो मुनिः
तावत्सत्यवती मात्रा स्वचरुं याचिता सती
श्रेष्ठं मत्वा तयायच्छन्मात्रे मातुरदत्स्वयम्‌
तद्विदित्वा मुनिः प्राह पत्नीं कष्टमकारषीः
घोरो दण्डधरः पुत्रो भ्राता ते ब्रह्मवित्तमः
प्रसादितः सत्यवत्या मैवं भूरिति भार्गवः
अथ तर्हि भवेत्पौत्रोजमदग्निस्ततोऽभवत्‌
सा चाभूत्सुमहत्पुण्या कौशिकी लोकपावनी
रेणोः सुतां रेणुकां वै जमदग्निरुवाह याम्‌
तस्यां वै भार्गवऋषेः सुता वसुमदादयः
यवीयान्जज्ञ एतेषां राम इत्यभिविश्रुतः
यमाहुर्वासुदेवांशं हैहयानां कुलान्तकम्‌
त्रिःसप्तकृत्वो य इमां चक्रे निःक्षत्रियां महीम्‌
दृप्तं क्षत्रं भुवो भारमब्रह्मण्यमनीनशत्‌
रजस्तमोवृतमहन्फल्गुन्यपि कृतेऽंहसि
श्रीराजोवाच
किं तदंहो भगवतो राजन्यैरजितात्मभिः
कृतं येन कुलं नष्टं क्षत्रियाणामभीक्ष्णशः
श्रीबादरायणिरुवाच
हैहयानामधिपतिरर्जुनः क्षत्रियर्षभः
दत्तं नारायणांशांशमाराध्य परिकर्मभिः
बाहून्दशशतं लेभे दुर्धर्षत्वमरातिषु
अव्याहतेन्द्रियौजः श्री तेजोवीर्ययशोबलम्‌
योगेश्वरत्वमैश्वर्यं गुणा यत्राणिमादयः
चचाराव्याहतगतिर्लोकेषु पवनो यथा
स्त्रीरत्नैरावृतः क्रीडन्रेवाम्भसि मदोत्कटः
वैजयन्तीं स्रजं बिभ्रद्रुरोध सरितं भुजैः
विप्लावितं स्वशिबिरं प्रतिस्रोतःसरिज्जलैः
नामृष्यत्तस्य तद्वीर्यं वीरमानी दशाननः
गृहीतो लीलया स्त्रीणां समक्षं कृतकिल्बिषः
माहिष्मत्यां सन्निरुद्धो मुक्तो येन कपिर्यथा
स एकदा तु मृगयां विचरन्विजने वने
यदृच्छयाश्रमपदं जमदग्नेरुपाविशत्‌
तस्मै स नरदेवाय मुनिरर्हणमाहरत्‌
ससैन्यामात्यवाहाय हविष्मत्या तपोधनः
स वै रत्नं तु तद्दृष्ट्वा आत्मैश्वर्यातिशायनम्‌
तन्नाद्रियताग्निहोत्र्यां साभिलाषः सहैहयः
हविर्धानीमृषेर्दर्पान्नरान्हर्तुमचोदयत्‌
ते च माहिष्मतीं निन्युः सवत्सां क्रन्दतीं बलात्‌
अथ राजनि निर्याते राम आश्रम आगतः
श्रुत्वा तत्तस्य दौरात्म्यं चुक्रोधाहिरिवाहतः
घोरमादाय परशुं सतूणं वर्म कार्मुकम्‌
अन्वधावत दुर्मर्षो मृगेन्द्र इव यूथपम्‌
तमापतन्तं भृगुवर्यमोजसा धनुर्धरं बाणपरश्वधायुधम्‌
ऐणेयचर्माम्बरमर्कधामभिर्युतं जटाभिर्ददृशे पुरीं विशन्‌
अचोदयद्धस्तिरथाश्वपत्तिभिर्गदासिबाणर्ष्टिशतघ्निशक्तिभिः
अक्षौहिणीः सप्तदशातिभीषणास्ता राम एको भगवानसूदयत्‌
यतो यतोऽसौ प्रहरत्परश्वधो मनोऽनिलौजाः परचक्रसूदनः
ततश्ततस्छिन्नभुजोरुकन्धरा निपेतुरुर्व्यां हतसूतवाहनाः
दृष्ट्वा स्वसैन्यं रुधिरौघकर्दमे रणाजिरे रामकुठारसायकैः
विवृक्णवर्मध्वजचापविग्रहं निपातितं हैहय आपतद्रुषा
अथार्जुनः पञ्चशतेषु बाहुभिर्धनुःषु बाणान्युगपत्स सन्दधे
रामाय रामोऽस्त्रभृतां समग्रणीस्तान्येकधन्वेषुभिराच्छिनत्समम्‌
पुनः स्वहस्तैरचलान्मृधेऽङ्घ्रिपानुत्क्षिप्य वेगादभिधावतो युधि
भुजान्कुठारेण कठोरनेमिना चिच्छेद रामः प्रसभं त्वहेरिव
कृत्तबाहोः शिरस्तस्य गिरेः शृङ्गमिवाहरत्‌
हते पितरि तत्पुत्रा अयुतं दुद्रुवुर्भयात्‌
अग्निहोत्रीमुपावर्त्य सवत्सां परवीरहा
समुपेत्याश्रमं पित्रे परिक्लिष्टां समर्पयत्‌
स्वकर्म तत्कृतं रामः पित्रे भ्रातृभ्य एव च
वर्णयामास तच्छ्रुत्वाजमदग्निरभाषत
राम राम महाबाहो भवान्पापमकारषीत्‌
अवधीन्नरदेवं यत्सर्वदेवमयं वृथा
वयं हि ब्राह्मणास्तात क्षमयार्हणतां गताः
यया लोकगुरुर्देवः पारमेष्ठ्यमगात्पदम्‌
क्षमया रोचते लक्ष्मीर्ब्राह्मी सौरी यथा प्रभा
क्षमिणामाशु भगवांस्तुष्यते हरिरीश्वरः
राज्ञो मूर्धाभिषिक्तस्य वधो ब्रह्मवधाद्गुरुः
तीर्थसंसेवया चांहो जह्यङ्गाच्युतचेतनः
श्रीशुक उवाच
पित्रोपशिक्षितो रामस्तथेति कुरुनन्दन
संवत्सरं तीर्थयात्रां चरित्वाश्रममाव्रजत्‌
कदाचिद्रेणुका याता गङ्गायां पद्ममालिनम्‌
गन्धर्वराजं क्रीडन्तमप्सरोभिरपश्यत
विलोकयन्ती क्रीडन्तमुदकार्थं नदीं गता
होमवेलां न सस्मार किञ्चिच्चित्ररथस्पृहा
कालात्ययं तं विलोक्य मुनेः शापविशङ्किता
आगत्य कलशं तस्थौ पुरोधाय कृताञ्जलिः
व्यभिचारं मुनिर्ज्ञात्वा पत्न्याः प्रकुपितोऽब्रवीत्‌
घ्नतैनां पुत्रकाः पापामित्युक्तास्ते न चक्रिरे
रामः सञ्चोदितः पित्रा भ्रात्न्मात्रा सहावधीत्‌
प्रभावज्ञो मुनेः सम्यक्समाधेस्तपसश्च सः
वरेण च्छन्दयामास प्रीतः सत्यवतीसुतः
वव्रे हतानां रामोऽपि जीवितं चास्मृतिं वधे
उत्तस्थुस्ते कुशलिनो निद्रापाय इवाञ्जसा
पितुर्विद्वांस्तपोवीर्यं रामश्चक्रे सुहृद्वधम्‌
येऽर्जुनस्य सुता राजन्स्मरन्तः स्वपितुर्वधम्‌
रामवीर्यपराभूता लेभिरे शर्म न क्वचित्‌
एकदाश्रमतो रामे सभ्रातरि वनं गते
वैरं सिषाधयिषवो लब्धच्छिद्रा उपागमन्‌
दृष्ट्वाग्न्यागार आसीनमावेशितधियं मुनिम्‌
भगवत्युत्तमश्लोके जघ्नुस्ते पापनिश्चयाः
याच्यमानाः कृपणया राममात्रातिदारुणाः
प्रसह्य शिर उत्कृत्य निन्युस्ते क्षत्रबन्धवः
रेणुका दुःखशोकार्ता निघ्नन्त्यात्मानमात्मना
राम रामेति तातेति विचुक्रोशोच्चकैः सती
तदुपश्रुत्य दूरस्था हा रामेत्यार्तवत्स्वनम्‌
त्वरयाश्रममासाद्य ददृशुः पितरं हतम्‌
ते दुःखरोषामर्षार्ति शोकवेगविमोहिताः
हा तात साधो धर्मिष्ठ त्यक्त्वास्मान्स्वर्गतो भवान्‌
विलप्यैवं पितुर्देहं निधाय भ्रातृषु स्वयम्‌
प्रगृह्य परशुं रामः क्षत्रान्ताय मनो दधे
गत्वा माहिष्मतीं रामो ब्रह्मघ्नविहतश्रियम्‌
तेषां स शीर्षभी राजन्मध्ये चक्रे महागिरिम्‌
तद्रक्तेन नदीं घोरामब्रह्मण्यभयावहाम्‌
हेतुं कृत्वा पितृवधं क्षत्रेऽमङ्गलकारिणि
त्रिःसप्तकृत्वः पृथिवीं कृत्वा निःक्षत्रियां प्रभुः
समन्तपञ्चके चक्रे शोणितोदान्ह्रदान्नव
पितुः कायेन सन्धाय शिर आदाय बर्हिषि
सर्वदेवमयं देवमात्मानमयजन्मखैः
ददौ प्राचीं दिशं होत्रे ब्रह्मणे दक्षिणां दिशम्‌
अध्वर्यवे प्रतीचीं वै उद्गात्रे उत्तरां दिशम्‌
अन्येभ्योऽवान्तरदिशः कश्यपाय च मध्यतः
आर्यावर्तमुपद्रष्ट्रे सदस्येभ्यस्ततः परम्‌
ततश्चावभृथस्नान विधूताशेषकिल्बिषः
सरस्वत्यां महानद्यां रेजे व्यब्भ्र इवांशुमान्‌
स्वदेहं जमदग्निस्तु लब्ध्वा संज्ञानलक्षणम्‌
ऋषीणां मण्डले सोऽभूत्सप्तमो रामपूजितः
जामदग्न्योऽपि भगवान्रामः कमललोचनः
आगामिन्यन्तरे राजन्वर्तयिष्यति वै बृहत्‌
आस्तेऽद्यापि महेन्द्राद्रौ न्यस्तदण्डः प्रशान्तधीः
उपगीयमानचरितः सिद्धगन्धर्वचारणैः
एवं भृगुषु विश्वात्मा भगवान्हरिरीश्वरः
अवतीर्य परं भारं भुवोऽहन्बहुशो नृपान्‌
गाधेरभून्महातेजाः समिद्ध इव पावकः
तपसा क्षात्रमुत्सृज्य यो लेभे ब्रह्मवर्चसम्‌
विश्वामित्रस्य चैवासन्पुत्रा एकशतं नृप
मध्यमस्तु मधुच्छन्दा मधुच्छन्दस एव ते
पुत्रं कृत्वा शुनःशेफं देवरातं च भार्गवम्‌
आजीगर्तं सुतानाह ज्येष्ठ एष प्रकल्प्यताम्‌
यो वै हरिश्चन्द्रमखे विक्रीतः पुरुषः पशुः
स्तुत्वा देवान्प्रजेशादीन्मुमुचे पाशबन्धनात्‌
यो रातो देवयजने देवैर्गाधिषु तापसः
देवरात इति ख्यातः शुनःशेफस्तु भार्गवः
ये मधुच्छन्दसो ज्येष्ठाः कुशलं मेनिरे न तत्‌
अशपत्तान्मुनिः क्रुद्धो म्लेच्छा भवत दुर्जनाः
स होवाच मधुच्छन्दाः सार्धं पञ्चाशता ततः
यन्नो भवान्सञ्जानीते तस्मिंस्तिष्ठामहे वयम्‌
ज्येष्ठं मन्त्रदृशं चक्रुस्त्वामन्वञ्चो वयं स्म हि
विश्वामित्रः सुतानाह वीरवन्तो भविष्यथ
ये मानं मेऽनुगृह्णन्तो वीरवन्तमकर्त माम्‌
एष वः कुशिका वीरो देवरातस्तमन्वित
अन्ये चाष्टकहारीत जयक्रतुमदादयः
एवं कौशिकगोत्रं तु विश्वामित्रैः पृथग्विधम्‌
प्रवरान्तरमापन्नं तद्धि चैवं प्रकल्पितम्‌
श्रीबादरायणिरुवाच
यः पुरूरवसः पुत्र आयुस्तस्याभवन्सुताः
नहुषः क्षत्रवृद्धश्च रजी राभश्च वीर्यवान्‌
अनेना इति राजेन्द्र शृणु क्षत्रवृधोऽन्वयम्‌
क्षत्रवृद्धसुतस्यासन्सुहोत्रस्यात्मजास्त्रयः
काश्यः कुशो गृत्समद इति गृत्समदादभूत्‌
शुनकः शौनको यस्य बह्वृचप्रवरो मुनिः
काश्यस्य काशिस्तत्पुत्रो राष्ट्रो दीर्घतमःपिता
धन्वन्तरिर्दीर्घतमस आयुर्वेदप्रवर्तकः
यज्ञभुग्वासुदेवांशः स्मृतमात्रार्तिनाशनः
तत्पुत्रः केतुमानस्य जज्ञे भीमरथस्ततः
दिवोदासो द्युमांस्तस्मात्प्रतर्दन इति स्मृतः
स एव शत्रुजिद्वत्स ऋतध्वज इतीरितः
तथा कुवलयाश्वेति प्रोक्तोऽलर्कादयस्ततः
षष्टिं वर्षसहस्राणि षष्टिं वर्षशतानि च
नालर्कादपरो राजन्बुभुजे मेदिनीं युवा
अलर्कात्सन्ततिस्तस्मात्सुनीथोऽथ निकेतनः
धर्मकेतुः सुतस्तस्मात्सत्यकेतुरजायत
धृष्टकेतुस्ततस्तस्मात्सुकुमारः क्षितीश्वरः
वीतिहोत्रोऽस्य भर्गोऽतो भार्गभूमिरभून्नृप
इतीमे काशयो भूपाः क्षत्रवृद्धान्वयायिनः
राभस्य रभसः पुत्रो गम्भीरश्चाक्रियस्ततः
तद्गोत्रं ब्रह्मविज्जज्ञे शृणु वंशमनेनसः
शुद्धस्ततः शुचिस्तस्माच्चित्रकृद्धर्मसारथिः
ततः शान्तरजो जज्ञे कृतकृत्यः स आत्मवान्‌
रजेः पञ्चशतान्यासन्पुत्राणाममितौजसाम्‌
देवैरभ्यर्थितो दैत्यान्हत्वेन्द्रायाददाद्दिवम्‌
इन्द्रस्तस्मै पुनर्दत्त्वा गृहीत्वा चरणौ रजेः
आत्मानमर्पयामास प्रह्रादाद्यरिशङ्कितः
पितर्युपरते पुत्रा याचमानाय नो ददुः
त्रिविष्टपं महेन्द्राय यज्ञभागान्समाददुः
गुरुणा हूयमानेऽग्नौ बलभित्तनयान्रजेः
अवधीद्भ्रंशितान्मार्गान्न कश्चिदवशेषितः
कुशात्प्रतिः क्षात्रवृद्धात्सञ्जयस्तत्सुतो जयः
ततः कृतः कृतस्यापि जज्ञे हर्यबलो नृपः
सहदेवस्ततो हीनो जयसेनस्तु तत्सुतः
सङ्कृतिस्तस्य च जयः क्षत्रधर्मा महारथः
क्षत्रवृद्धान्वया भूपा इमे शृण्वथ नाहुषान्‌
श्रीशुक उवाच
यतिर्ययातिः संयातिरायतिर्वियतिः कृतिः
षडिमे नहुषस्यासन्निन्द्रियाणीव देहिनः
राज्यं नैच्छद्यतिः पित्रा दत्तं तत्परिणामवित्‌
यत्र प्रविष्टः पुरुष आत्मानं नावबुध्यते
पितरि भ्रंशिते स्थानादिन्द्राण्या धर्षणाद्द्विजैः
प्रापितेऽजगरत्वं वै ययातिरभवन्नृपः
चतसृष्वादिशद्दिक्षु भ्रात्न्भ्राता यवीयसः
कृतदारो जुगोपोर्वीं काव्यस्य वृषपर्वणः
श्रीराजोवाच
ब्रह्मर्षिर्भगवान्काव्यः क्षत्रबन्धुश्च नाहुषः
राजन्यविप्रयोः कस्माद्विवाहः प्रतिलोमकः
श्रीशुक उवाच
एकदा दानवेन्द्रस्य शर्मिष्ठा नाम कन्यका
सखीसहस्रसंयुक्ता गुरुपुत्र्या च भामिनी
देवयान्या पुरोद्याने पुष्पितद्रुमसङ्कुले
व्यचरत्कलगीतालि नलिनीपुलिनेऽबला
ता जलाशयमासाद्य कन्याः कमललोचनाः
तीरे न्यस्य दुकूलानि विजह्रुः सिञ्चतीर्मिथः
वीक्ष्य व्रजन्तं गिरिशं सह देव्या वृषस्थितम्‌
सहसोत्तीर्य वासांसि पर्यधुर्व्रीडिताः स्त्रियः
शर्मिष्ठाजानती वासो गुरुपुत्र्याः समव्ययत्‌
स्वीयं मत्वा प्रकुपिता देवयानीदमब्रवीत्‌
अहो निरीक्ष्यतामस्या दास्याः कर्म ह्यसाम्प्रतम्‌
अस्मद्धार्यं धृतवती शुनीव हविरध्वरे
यैरिदं तपसा सृष्टं मुखं पुंसः परस्य ये
धार्यते यैरिह ज्योतिः शिवः पन्थाः प्रदर्शितः
यान्वन्दन्त्युपतिष्ठन्ते लोकनाथाः सुरेश्वराः
भगवानपि विश्वात्मा पावनः श्रीनिकेतनः
वयं तत्रापि भृगवः शिष्योऽस्या नः पितासुरः
अस्मद्धार्यं धृतवती शूद्रो वेदमिवासती
एवं क्षिपन्तीं शर्मिष्ठा गुरुपुत्रीमभाषत
रुषा श्वसन्त्युरङ्गीव धर्षिता दष्टदच्छदा
आत्मवृत्तमविज्ञाय कत्थसे बहु भिक्षुकि
किं न प्रतीक्षसेऽस्माकं गृहान्बलिभुजो यथा
एवंविधैः सुपरुषैः क्षिप्त्वाचार्यसुतां सतीम्‌
शर्मिष्ठा प्राक्षिपत्कूपे वासश्चादाय मन्युना
तस्यां गतायां स्वगृहं ययातिर्मृगयां चरन्‌
प्राप्तो यदृच्छया कूपे जलार्थी तां ददर्श ह
दत्त्वा स्वमुत्तरं वासस्तस्यै राजा विवाससे
गृहीत्वा पाणिना पाणिमुज्जहार दयापरः
तं वीरमाहौशनसी प्रेमनिर्भरया गिरा
राजंस्त्वया गृहीतो मे पाणिः परपुरञ्जय
हस्तग्राहोऽपरो मा भूद्गृहीतायास्त्वया हि मे
एष ईशकृतो वीर सम्बन्धो नौ न पौरुषः
यदिदं कूपमग्नाया भवतो दर्शनं मम
न ब्राह्मणो मे भविता हस्तग्राहो महाभुज
कचस्य बार्हस्पत्यस्य शापाद्यमशपं पुरा
ययातिरनभिप्रेतं दैवोपहृतमात्मनः
मनस्तु तद्गतं बुद्ध्वा प्रतिजग्राह तद्वचः
गते राजनि सा धीरे तत्र स्म रुदती पितुः
न्यवेदयत्ततः सर्वमुक्तं शर्मिष्ठया कृतम्‌
दुर्मना भगवान्काव्यः पौरोहित्यं विगर्हयन्‌
स्तुवन्वृत्तिं च कापोतीं दुहित्रा स ययौ पुरात्‌
वृषपर्वा तमाज्ञाय प्रत्यनीकविवक्षितम्‌
गुरुं प्रसादयन्मूर्ध्ना पादयोः पतितः पथि
क्षणार्धमन्युर्भगवान्शिष्यं व्याचष्ट भार्गवः
कामोऽस्याः क्रियतां राजन्नैनां त्यक्तुमिहोत्सहे
तथेत्यवस्थिते प्राह देवयानी मनोगतम्‌
पित्रा दत्ता यतो यास्ये सानुगा यातु मामनु
पित्रा दत्ता देवयान्यै शर्मिष्ठा सानुगा तदा
स्वानां तत्सङ्कटं वीक्ष्य तदर्थस्य च गौरवम्‌
देवयानीं पर्यचरत्स्त्रीसहस्रेण दासवत्‌
नाहुषाय सुतां दत्त्वा सह शर्मिष्ठयोशना
तमाह राजन्छर्मिष्ठामाधास्तल्पे न कर्हिचित्‌
विलोक्यौशनसीं राजञ्छर्मिष्ठा सुप्रजां क्वचित्‌
तमेव वव्रे रहसि सख्याः पतिमृतौ सती
राजपुत्र्यार्थितोऽपत्ये धर्मं चावेक्ष्य धर्मवित्‌
स्मरन्छुक्रवचः काले दिष्टमेवाभ्यपद्यत
यदुं च तुर्वसुं चैव देवयानी व्यजायत
द्रुह्युं चानुं च पूरुं च शर्मिष्ठा वार्षपर्वणी
गर्भसम्भवमासुर्या भर्तुर्विज्ञाय मानिनी
देवयानी पितुर्गेहं ययौ क्रोधविमूर्छिता
प्रियामनुगतः कामी वचोभिरुपमन्त्रयन्‌
न प्रसादयितुं शेके पादसंवाहनादिभिः
शुक्रस्तमाह कुपितः स्त्रीकामानृतपूरुष
त्वां जरा विशतां मन्द विरूपकरणी नृणाम्‌
श्रीययातिरुवाच
अतृप्तोऽस्म्यद्य कामानां ब्रह्मन्दुहितरि स्म ते
व्यत्यस्यतां यथाकामं वयसा योऽभिधास्यति
इति लब्धव्यवस्थानः पुत्रं ज्येष्ठमवोचत
यदो तात प्रतीच्छेमां जरां देहि निजं वयः
मातामहकृतां वत्स न तृप्तो विषयेष्वहम्‌
वयसा भवदीयेन रंस्ये कतिपयाः समाः
श्रीयदुरुवाच
नोत्सहे जरसा स्थातुमन्तरा प्राप्तया तव
अविदित्वा सुखं ग्राम्यं वैतृष्ण्यं नैति पूरुषः
तुर्वसुश्चोदितः पित्रा द्रुह्युश्चानुश्च भारत
प्रत्याचख्युरधर्मज्ञा ह्यनित्ये नित्यबुद्धयः
अपृच्छत्तनयं पूरुं वयसोनं गुणाधिकम्‌
न त्वमग्रजवद्वत्स मां प्रत्याख्यातुमर्हसि
श्रीपूरुरुवाच
को नु लोके मनुष्येन्द्र पितुरात्मकृतः पुमान्‌
प्रतिकर्तुं क्षमो यस्य प्रसादाद्विन्दते परम्‌
उत्तमश्चिन्तितं कुर्यात्प्रोक्तकारी तु मध्यमः
अधमोऽश्रद्धया कुर्यादकर्तोच्चरितं पितुः
इति प्रमुदितः पूरुः प्रत्यगृह्णाज्जरां पितुः
सोऽपि तद्वयसा कामान्यथावज्जुजुषे नृप
सप्तद्वीपपतिः संयक्पितृवत्पालयन्प्रजाः
यथोपजोषं विषयाञ्जुजुषेऽव्याहतेन्द्रियः
देवयान्यप्यनुदिनं मनोवाग्देहवस्तुभिः
प्रेयसः परमां प्रीतिमुवाह प्रेयसी रहः
अयजद्यज्ञपुरुषं क्रतुभिर्भूरिदक्षिणैः
सर्वदेवमयं देवं सर्ववेदमयं हरिम्‌
यस्मिन्निदं विरचितं व्योम्नीव जलदावलिः
नानेव भाति नाभाति स्वप्नमायामनोरथः
तमेव हृदि विन्यस्य वासुदेवं गुहाशयम्‌
नारायणमणीयांसं निराशीरयजत्प्रभुम्‌
एवं वर्षसहस्राणि मनःषष्ठैर्मनःसुखम्‌
विदधानोऽपि नातृप्यत्सार्वभौमः कदिन्द्रियैः
श्रीशुक उवाच
स इत्थमाचरन्कामान्स्त्रैणोऽपह्नवमात्मनः
बुद्ध्वा प्रियायै निर्विण्णो गाथामेतामगायत
शृणु भार्गव्यमूं गाथां मद्विधाचरितां भुवि
धीरा यस्यानुशोचन्ति वने ग्रामनिवासिनः
बस्त एको वने कश्चिद्विचिन्वन्प्रियमात्मनः
ददर्श कूपे पतितां स्वकर्मवशगामजाम्‌
तस्या उद्धरणोपायं बस्तः कामी विचिन्तयन्‌
व्यधत्त तीर्थमुद्धृत्य विषाणाग्रेण रोधसी
सोत्तीर्य कूपात्सुश्रोणी तमेव चकमे किल
तया वृतं समुद्वीक्ष्य बह्व्योऽजाः कान्तकामिनीः
पीवानं श्मश्रुलं प्रेष्ठं मीढ्वांसं याभकोविदम्‌
स एकोऽजवृषस्तासां बह्वीनां रतिवर्धनः
रेमे कामग्रहग्रस्त आत्मानं नावबुध्यत
तमेव प्रेष्ठतमया रममाणमजान्यया
विलोक्य कूपसंविग्ना नामृष्यद्बस्तकर्म तत्‌
तं दुर्हृदं सुहृद्रूपं कामिनं क्षणसौहृदम्‌
इन्द्रियाराममुत्सृज्य स्वामिनं दुःखिता ययौ
सोऽपि चानुगतः स्त्रैणः कृपणस्तां प्रसादितुम्‌
कुर्वन्निडविडाकारं नाशक्नोत्पथि सन्धितुम्‌
तस्य तत्र द्विजः कश्चिदजास्वाम्यच्छिनद्रुषा
लम्बन्तं वृषणं भूयः सन्दधेऽर्थाय योगवित्‌
सम्बद्धवृषणः सोऽपि ह्यजया कूपलब्धया
कालं बहुतिथं भद्रे कामैर्नाद्यापि तुष्यति
तथाहं कृपणः सुभ्रु भवत्याः प्रेमयन्त्रितः
आत्मानं नाभिजानामि मोहितस्तव मायया
यत्पृथिव्यां व्रीहियवं हिरण्यं पशवः स्त्रियः
न दुह्यन्ति मनःप्रीतिं पुंसः कामहतस्य ते
न जातु कामः कामानामुपभोगेन शांयति
हविषा कृष्णवर्त्मेव भूय एवाभिवर्धते
यदा न कुरुते भावं सर्वभूतेष्वमङ्गलम्‌
समदृष्टेस्तदा पुंसः सर्वाः सुखमया दिशः
या दुस्त्यजा दुर्मतिभिर्जीर्यतो या न जीर्यते
तां तृष्णां दुःखनिवहां शर्मकामो द्रुतं त्यजेत्‌
मात्रा स्वस्रा दुहित्रा वा नाविविक्तासनो भवेत्‌
बलवानिन्द्रियग्रामो विद्वांसमपि कर्षति
पूर्णं वर्षसहस्रं मे विषयान्सेवतोऽसकृत्‌
तथापि चानुसवनं तृष्णा तेषूपजायते
तस्मादेतामहं त्यक्त्वा ब्रह्मण्यध्याय मानसम्‌
निर्द्वन्द्वो निरहङ्कारश्चरिष्यामि मृगैः सह
दृष्टं श्रुतमसद्बुद्ध्वा नानुध्यायेन्न सन्दिशेत्‌
संसृतिं चात्मनाशं च तत्र विद्वान्स आत्मदृक्‌
इत्युक्त्वा नाहुषो जायां तदीयं पूरवे वयः
दत्त्वा स्वजरसं तस्मादाददे विगतस्पृहः
दिशि दक्षिणपूर्वस्यां द्रुह्युं दक्षिणतो यदुम्‌
प्रतीच्यां तुर्वसुं चक्र उदीच्यामनुमीश्वरम्‌
भूमण्डलस्य सर्वस्य पूरुमर्हत्तमं विशाम्‌
अभिषिच्याग्रजांस्तस्य वशे स्थाप्य वनं ययौ
आसेवितं वर्षपूगान्षड्वर्गं विषयेषु सः
क्षणेन मुमुचे नीडं जातपक्ष इव द्विजः
स तत्र निर्मुक्तसमस्तसङ्ग आत्मानुभूत्या विधुतत्रिलिङ्गः
परेऽमले ब्रह्मणि वासुदेवे लेभे गतिं भागवतीं प्रतीतः
श्रुत्वा गाथां देवयानी मेने प्रस्तोभमात्मनः
स्त्रीपुंसोः स्नेहवैक्लव्यात्परिहासमिवेरितम्‌
सा सन्निवासं सुहृदां प्रपायामिव गच्छताम्‌
विज्ञायेश्वरतन्त्राणां मायाविरचितं प्रभोः
सर्वत्र सङ्गमुत्सृज्य स्वप्नौपम्येन भार्गवी
कृष्णे मनः समावेश्य व्यधुनोल्लिङ्गमात्मनः
नमस्तुभ्यं भगवते वासुदेवाय वेधसे
सर्वभूताधिवासाय शान्ताय बृहते नमः
श्रीबादरायणिरुवाच
पूरोर्वंशं प्रवक्ष्यामि यत्र जातोऽसि भारत
यत्र राजर्षयो वंश्या ब्रह्मवंश्याश्च जज्ञिरे
जनमेजयो ह्यभूत्पूरोः प्रचिन्वांस्तत्सुतस्ततः
प्रवीरोऽथ मनुस्युर्वै तस्माच्चारुपदोऽभवत्‌
तस्य सुद्युरभूत्पुत्रस्तस्माद्बहुगवस्ततः
संयातिस्तस्याहंयाती रौद्राश्वस्तत्सुतः स्मृतः
ऋतेयुस्तस्य कक्षेयुः स्थण्डिलेयुः कृतेयुकः
जलेयुः सन्नतेयुश्च धर्मसत्यव्रतेयवः
दशैतेऽप्सरसः पुत्रा वनेयुश्चावमः स्मृतः
घृताच्यामिन्द्रियाणीव मुख्यस्य जगदात्मनः
ऋतेयो रन्तिनावोऽभूत्त्रयस्तस्यात्मजा नृप
सुमतिर्ध्रुवोऽप्रतिरथः कण्वोऽप्रतिरथात्मजः
तस्य मेधातिथिस्तस्मात्प्रस्कन्नाद्या द्विजातयः
पुत्रोऽभूत्सुमते रेभिर्दुष्मन्तस्तत्सुतो मतः
दुष्मन्तो मृगयां यातः कण्वाश्रमपदं गतः
तत्रासीनां स्वप्रभया मण्डयन्तीं रमामिव
विलोक्य सद्यो मुमुहे देवमायामिव स्त्रियम्‌
बभाषे तां वरारोहां भटैः कतिपयैर्वृतः
तद्दर्शनप्रमुदितः सन्निवृत्तपरिश्रमः
पप्रच्छ कामसन्तप्तः प्रहसञ्श्लक्ष्णया गिरा
का त्वं कमलपत्राक्षि कस्यासि हृदयङ्गमे
किं स्विच्चिकीर्षितं तत्र भवत्या निर्जने वने
व्यक्तं राजन्यतनयां वेद्म्यहं त्वां सुमध्यमे
न हि चेतः पौरवाणामधर्मे रमते क्वचित्‌
श्रीशकुन्तलोवाच
विश्वामित्रात्मजैवाहं त्यक्ता मेनकया वने
वेदैतद्भगवान्कण्वो वीर किं करवाम ते
आस्यतां ह्यरविन्दाक्ष गृह्यतामर्हणं च नः
भुज्यतां सन्ति नीवारा उष्यतां यदि रोचते
श्रीदुष्मन्त उवाच
उपपन्नमिदं सुभ्रु जातायाः कुशिकान्वये
स्वयं हि वृणुते राज्ञां कन्यकाः सदृशं वरम्‌
ओमित्युक्ते यथाधर्ममुपयेमे शकुन्तलाम्‌
गान्धर्वविधिना राजा देशकालविधानवित्‌
अमोघवीर्यो राजर्षिर्महिष्यां वीर्यमादधे
श्वोभूते स्वपुरं यातः कालेनासूत सा सुतम्‌
कण्वः कुमारस्य वने चक्रे समुचिताः क्रियाः
बद्ध्वा मृगेन्द्रं तरसा क्रीडति स्म स बालकः
तं दुरत्ययविक्रान्तमादाय प्रमदोत्तमा
हरेरंशांशसम्भूतं भर्तुरन्तिकमागमत्‌
यदा न जगृहे राजा भार्यापुत्रावनिन्दितौ
शृण्वतां सर्वभूतानां खे वागाहाशरीरिणी
माता भस्त्रा पितुः पुत्रो येन जातः स एव सः
भरस्व पुत्रं दुष्मन्त मावमंस्थाः शकुन्तलाम्‌
रेतोधाः पुत्रो नयति नरदेव यमक्षयात्‌
त्वं चास्य धाता गर्भस्य सत्यमाह शकुन्तला
पितर्युपरते सोऽपि चक्रवर्ती महायशाः
महिमा गीयते तस्य हरेरंशभुवो भुवि
चक्रं दक्षिणहस्तेऽस्य पद्मकोशोऽस्य पादयोः
ईजे महाभिषेकेण सोऽभिषिक्तोऽधिराड्विभुः
पञ्चपञ्चाशता मेध्यैर्गङ्गायामनु वाजिभिः
मामतेयं पुरोधाय यमुनामनु च प्रभुः
अष्टसप्ततिमेध्याश्वान्बबन्ध प्रददद्वसु
भरतस्य हि दौष्मन्तेरग्निः साचीगुणे चितः
सहस्रं बद्वशो यस्मिन्ब्राह्मणा गा विभेजिरे
त्रयस्त्रिंशच्छतं ह्यश्वान्बद्ध्वा विस्मापयन्नृपान्‌
दौष्मन्तिरत्यगान्मायां देवानां गुरुमाययौ
मृगान्छुक्लदतः कृष्णान्हिरण्येन परीवृतान्‌
अदात्कर्मणि मष्णारे नियुतानि चतुर्दश
भरतस्य महत्कर्म न पूर्वे नापरे नृपाः
नैवापुर्नैव प्राप्स्यन्ति बाहुभ्यां त्रिदिवं यथा
किरातहूणान्यवनान्पौण्ड्रान्कङ्कान्खशान्छकान्‌
अब्रह्मण्यनृपांश्चाहन्म्लेच्छान्दिग्विजयेऽखिलान्‌
जित्वा पुरासुरा देवान्ये रसौकांसि भेजिरे
देवस्त्रियो रसां नीताः प्राणिभिः पुनराहरत्‌
सर्वान्कामान्दुदुहतुः प्रजानां तस्य रोदसी
समास्त्रिणवसाहस्रीर्दिक्षु चक्रमवर्तयत्‌
स संराड्लोकपालाख्यमैश्वर्यमधिराट्श्रियम्‌
चक्रं चास्खलितं प्राणान्मृषेत्युपरराम ह
तस्यासन्नृप वैदर्भ्यः पत्न्यस्तिस्रः सुसम्मताः
जघ्नुस्त्यागभयात्पुत्रान्नानुरूपा इतीरिते
तस्यैवं वितथे वंशे तदर्थं यजतः सुतम्‌
मरुत्स्तोमेन मरुतो भरद्वाजमुपाददुः
अन्तर्वत्न्यां भ्रातृपत्न्यां मैथुनाय बृहस्पतिः
प्रवृत्तो वारितो गर्भं शप्त्वा वीर्यमुपासृजत्‌
तं त्यक्तुकामां ममतां भर्तुस्त्यागविशङ्किताम्‌
नामनिर्वाचनं तस्य श्लोकमेनं सुरा जगुः
मूढे भर द्वाजमिमं भर द्वाजं बृहस्पते
यातौ यदुक्त्वा पितरौ भरद्वाजस्ततस्त्वयम्‌
चोद्यमाना सुरैरेवं मत्वा वितथमात्मजम्‌
व्यसृजन्मरुतोऽबिभ्रन्दत्तोऽयं वितथेऽन्वये
श्रीशुक उवाच
वितथस्य सुतान्मन्योर्बृहत्क्षत्रो जयस्ततः
महावीर्यो नरो गर्गः सङ्कृतिस्तु नरात्मजः
गुरुश्च रन्तिदेवश्च सङ्कृतेः पाण्डुनन्दन
रन्तिदेवस्य महिमा इहामुत्र च गीयते
वियद्वित्तस्य ददतो लब्धं लब्धं बुभुक्षतः
निष्किञ्चनस्य धीरस्य सकुटुम्बस्य सीदतः
व्यतीयुरष्टचत्वारिंशदहान्यपिबतः किल
घृतपायससंयावं तोयं प्रातरुपस्थितम्‌
कृच्छ्रप्राप्तकुटुम्बस्य क्षुत्तृड्भ्यां जातवेपथोः
अतिथिर्ब्राह्मणः काले भोक्तुकामस्य चागमत्‌
तस्मै संव्यभजत्सोऽन्नमादृत्य श्रद्धयान्वितः
हरिं सर्वत्र सम्पश्यन्स भुक्त्वा प्रययौ द्विजः
अथान्यो भोक्ष्यमाणस्य विभक्तस्य महीपतेः
विभक्तं व्यभजत्तस्मै वृषलाय हरिं स्मरन्‌
याते शूद्रे तमन्योऽगादतिथिः श्वभिरावृतः
राजन्मे दीयतामन्नं सगणाय बुभुक्षते
स आदृत्यावशिष्टं यद्बहुमानपुरस्कृतम्‌
तच्च दत्त्वा नमश्चक्रे श्वभ्यः श्वपतये विभुः
पानीयमात्रमुच्छेषं तच्चैकपरितर्पणम्‌
पास्यतः पुल्कसोऽभ्यागादपो देह्यशुभाय मे
तस्य तां करुणां वाचं निशम्य विपुलश्रमाम्‌
कृपया भृशसन्तप्त इदमाहामृतं वचः
न कामयेऽहं गतिमीश्वरात्परामष्टर्द्धियुक्तामपुनर्भवं वा
आर्तिं प्रपद्येऽखिलदेहभाजामन्तःस्थितो येन भवन्त्यदुःखाः
क्षुत्तृट्श्रमो गात्रपरिभ्रमश्च दैन्यं क्लमः शोकविषादमोहाः
सर्वे निवृत्ताः कृपणस्य जन्तोर्जिजीविषोर्जीवजलार्पणान्मे
इति प्रभाष्य पानीयं म्रियमाणः पिपासया
पुल्कसायाददाद्धीरो निसर्गकरुणो नृपः
तस्य त्रिभुवनाधीशाः फलदाः फलमिच्छताम्‌
आत्मानं दर्शयां चक्रुर्माया विष्णुविनिर्मिताः
स वै तेभ्यो नमस्कृत्य निःसङ्गो विगतस्पृहः
वासुदेवे भगवति भक्त्या चक्रे मनः परम्‌
ईश्वरालम्बनं चित्तं कुर्वतोऽनन्यराधसः
माया गुणमयी राजन्स्वप्नवत्प्रत्यलीयत
तत्प्रसङ्गानुभावेन रन्तिदेवानुवर्तिनः
अभवन्योगिनः सर्वे नारायणपरायणाः
गर्गाच्छिनिस्ततो गार्ग्यः क्षत्राद्ब्रह्म ह्यवर्तत
दुरितक्षयो महावीर्यात्तस्य त्रय्यारुणिः कविः
पुष्करारुणिरित्यत्र ये ब्राह्मणगतिं गताः
बृहत्क्षत्रस्य पुत्रोऽभूद्धस्ती यद्धस्तिनापुरम्‌
अजमीढो द्विमीढश्च पुरुमीढश्च हस्तिनः
अजमीढस्य वंश्याः स्युः प्रियमेधादयो द्विजाः
अजमीढाद्बृहदिषुस्तस्य पुत्रो बृहद्धनुः
बृहत्कायस्ततस्तस्य पुत्र आसीज्जयद्रथः
तत्सुतो विशदस्तस्य स्येनजित्समजायत
रुचिराश्वो दृढहनुः काश्यो वत्सश्च तत्सुताः
रुचिराश्वसुतः पारः पृथुसेनस्तदात्मजः
पारस्य तनयो नीपस्तस्य पुत्रशतं त्वभूत्‌
स कृत्व्यां शुककन्यायां ब्रह्मदत्तमजीजनत्‌
योगी स गवि भार्यायां विष्वक्सेनमधात्सुतम्‌
जैगीषव्योपदेशेन योगतन्त्रं चकार ह
उदक्सेनस्ततस्तस्माद्भल्लाटो बार्हदीषवाः
यवीनरो द्विमीढस्य कृतिमांस्तत्सुतः स्मृतः
नाम्ना सत्यधृतिस्तस्य दृढनेमिः सुपार्श्वकृत्‌
सुपार्श्वात्सुमतिस्तस्य पुत्रः सन्नतिमांस्ततः
कृती हिरण्यनाभाद्यो योगं प्राप्य जगौ स्म षट्‌
संहिताः प्राच्यसाम्नां वै नीपो ह्युद्ग्रायुधस्ततः
तस्य क्षेम्यः सुवीरोऽथ सुवीरस्य रिपुञ्जयः
ततो बहुरथो नाम पुरुमीढोऽप्रजोऽभवत्‌
नलिन्यामजमीढस्य नीलः शान्तिस्तु तत्सुतः
शान्तेः सुशान्तिस्तत्पुत्रः पुरुजोऽर्कस्ततोऽभवत्‌
भर्म्याश्वस्तनयस्तस्य पञ्चासन्मुद्गलादयः
यवीनरो बृहद्विश्वः काम्पिल्लः सञ्जयः सुताः
भर्म्याश्वः प्राह पुत्रा मे पञ्चानां रक्षणाय हि
विषयाणामलमिमे इति पञ्चालसंज्ञिताः
मुद्गलाद्ब्रह्मनिर्वृत्तं गोत्रं मौद्गल्यसंज्ञितम्‌
मिथुनं मुद्गलाद्भार्म्याद्दिवोदासः पुमानभूत्‌
अहल्या कन्यका यस्यां शतानन्दस्तु गौतमात्‌
तस्य सत्यधृतिः पुत्रो धनुर्वेदविशारदः
शरद्वांस्तत्सुतो यस्मादुर्वशीदर्शनात्किल
शरस्तम्बेऽपतद्रेतो मिथुनं तदभूच्छुभम्‌
तद्दृष्ट्वा कृपयागृह्णाच्छान्तनुर्मृगयां चरन्‌
कृपः कुमारः कन्या च द्रोणपत्न्यभवत्कृपी
श्रीशुक उवाच
मित्रायुश्च दिवोदासाच्च्यवनस्तत्सुतो नृप
सुदासः सहदेवोऽथ सोमको जन्तुजन्मकृत्‌
तस्य पुत्रशतं तेषां यवीयान्पृषतः सुतः
स तस्माद्द्रुपदो जज्ञे सर्वसम्पत्समन्वितः
द्रुपदाद्द्रौपदी तस्य धृष्टद्युम्नादयः सुताः
धृष्टद्युम्नाद्धृष्टकेतुर्भार्म्याः पाञ्चालका इमे
योऽजमीढसुतो ह्यन्य ऋक्षः संवरणस्ततः
तपत्यां सूर्यकन्यायां कुरुक्षेत्रपतिः कुरुः
परीक्षिः सुधनुर्जह्नुर्निषधश्च कुरोः सुताः
सुहोत्रोऽभूत्सुधनुषश्च्यवनोऽथ ततः कृती
वसुस्तस्योपरिचरो बृहद्रथमुखास्ततः
कुशाम्बमत्स्यप्रत्यग्र चेदिपाद्याश्च चेदिपाः
बृहद्रथात्कुशाग्रोऽभूदृषभस्तस्य तत्सुतः
जज्ञे सत्यहितोऽपत्यं पुष्पवांस्तत्सुतो जहुः
अन्यस्यामपि भार्यायां शकले द्वे बृहद्रथात्‌
ये मात्रा बहिरुत्सृष्टे जरया चाभिसन्धिते
जीव जीवेति क्रीडन्त्या जरासन्धोऽभवत्सुतः
ततश्च सहदेवोऽभूत्सोमापिर्यच्छ्रुतश्रवाः
परीक्षिरनपत्योऽभूत्सुरथो नाम जाह्नवः
ततो विदूरथस्तस्मात्सार्वभौमस्ततोऽभवत्‌
जयसेनस्तत्तनयो राधिकोऽतोऽयुताय्वभूत्‌
ततश्चाक्रोधनस्तस्माद्देवातिथिरमुष्य च
ऋक्षस्तस्य दिलीपोऽभूत्प्रतीपस्तस्य चात्मजः
देवापिः शान्तनुस्तस्य बाह्लीक इति चात्मजाः
पितृराज्यं परित्यज्य देवापिस्तु वनं गतः
अभवच्छान्तनू राजा प्राङ्महाभिषसंज्ञितः
यं यं कराभ्यां स्पृशति जीर्णं यौवनमेति सः
शान्तिमाप्नोति चैवाग्र्यां कर्मणा तेन शान्तनुः
समा द्वादश तद्राज्ये न ववर्ष यदा विभुः
शान्तनुर्ब्राह्मणैरुक्तः परिवेत्तायमग्रभुक्‌
राज्यं देह्यग्रजायाशु पुरराष्ट्रविवृद्धये
एवमुक्तो द्विजैर्ज्येष्ठं छन्दयामास सोऽब्रवीत्‌
तन्मन्त्रिप्रहितैर्विप्रैर्वेदाद्विभ्रंशितो गिरा
वेदवादातिवादान्वै तदा देवो ववर्ष ह
देवापिर्योगमास्थाय कलापग्राममाश्रितः
सोमवंशे कलौ नष्टे कृतादौ स्थापयिष्यति
बाह्लीकात्सोमदत्तोऽभूद्भूरिर्भूरिश्रवास्ततः
शलश्च शान्तनोरासीद्गङ्गायां भीष्म आत्मवान्‌
सर्वधर्मविदां श्रेष्ठो महाभागवतः कविः
वीरयूथाग्रणीर्येन रामोऽपि युधि तोषितः
शान्तनोर्दासकन्यायां जज्ञे चित्राङ्गदः सुतः
विचित्रवीर्यश्चावरजो नाम्ना चित्राङ्गदो हतः
यस्यां पराशरात्साक्षादवतीर्णो हरेः कला
वेदगुप्तो मुनिः कृष्णो यतोऽहमिदमध्यगाम्‌
हित्वा स्वशिष्यान्पैलादीन्भगवान्बादरायणः
मह्यं पुत्राय शान्ताय परं गुह्यमिदं जगौ
विचित्रवीर्योऽथोवाह काशीराजसुते बलात्‌
स्वयंवरादुपानीते अम्बिकाम्बालिके उभे
तयोरासक्तहृदयो गृहीतो यक्ष्मणा मृतः
क्षेत्रेऽप्रजस्य वै भ्रातुर्मात्रोक्तो बादरायणः
धृतराष्ट्रं च पाण्डुं च विदुरं चाप्यजीजनत्‌
गान्धार्यां धृतराष्ट्रस्य जज्ञे पुत्रशतं नृप
तत्र दुर्योधनो ज्येष्ठो दुःशला चापि कन्यका
शापान्मैथुनरुद्धस्य पाण्डोः कुन्त्यां महारथाः
जाता धर्मानिलेन्द्रेभ्यो युधिष्ठिरमुखास्त्रयः
नकुलः सहदेवश्च माद्र्यां नासत्यदस्रयोः
द्रौपद्यां पञ्च पञ्चभ्यः पुत्रास्ते पितरोऽभवन्‌
युधिष्ठिरात्प्रतिविन्ध्यः श्रुतसेनो वृकोदरात्‌
अर्जुनाच्छ्रुतकीर्तिस्तु शतानीकस्तु नाकुलिः
सहदेवसुतो राजन्छ्रुतकर्मा तथापरे
युधिष्ठिरात्तु पौरव्यां देवकोऽथ घटोत्कचः
भीमसेनाद्धिडिम्बायां काल्यां सर्वगतस्ततः
सहदेवात्सुहोत्रं तु विजयासूत पार्वती
करेणुमत्यां नकुलो नरमित्रं तथार्जुनः
इरावन्तमुलुप्यां वै सुतायां बभ्रुवाहनम्‌
मणिपुरपतेः सोऽपि तत्पुत्रः पुत्रिकासुतः
तव तातः सुभद्रायामभिमन्युरजायत
सर्वातिरथजिद्वीर उत्तरायां ततो भवान्‌
परिक्षीणेषु कुरुषु द्रौणेर्ब्रह्मास्त्रतेजसा
त्वं च कृष्णानुभावेन सजीवो मोचितोऽन्तकात्‌
तवेमे तनयास्तात जनमेजयपूर्वकाः
श्रुतसेनो भीमसेन उग्रसेनश्च वीर्यवान्‌
जनमेजयस्त्वां विदित्वा तक्षकान्निधनं गतम्‌
सर्पान्वै सर्पयागाग्नौ स होष्यति रुषान्वितः
कालषेयं पुरोधाय तुरं तुरगमेधषाट्‌
समन्तात्पृथिवीं सर्वां जित्वा यक्ष्यति चाध्वरैः
तस्य पुत्रः शतानीको याज्ञवल्क्यात्त्रयीं पठन्‌
अस्त्रज्ञानं क्रियाज्ञानं शौनकात्परमेष्यति
सहस्रानीकस्तत्पुत्रस्ततश्चैवाश्वमेधजः
असीमकृष्णस्तस्यापि नेमिचक्रस्तु तत्सुतः
गजाह्वये हृते नद्या कौशाम्ब्यां साधु वत्स्यति
उक्तस्ततश्चित्ररथस्तस्माच्छुचिरथः सुतः
तस्माच्च वृष्टिमांस्तस्य सुषेणोऽथ महीपतिः
सुनीथस्तस्य भविता नृचक्षुर्यत्सुखीनलः
परिप्लवः सुतस्तस्मान्मेधावी सुनयात्मजः
नृपञ्जयस्ततो दूर्वस्तिमिस्तस्माज्जनिष्यति
तिमेर्बृहद्रथस्तस्माच्छतानीकः सुदासजः
शतानीकाद्दुर्दमनस्तस्यापत्यं महीनरः
दण्डपाणिर्निमिस्तस्य क्षेमको भविता यतः
ब्रह्मक्षत्रस्य वै योनिर्वंशो देवर्षिसत्कृतः
क्षेमकं प्राप्य राजानं संस्थां प्राप्स्यति वै कलौ
अथ मागधराजानो भाविनो ये वदामि ते
भविता सहदेवस्य मार्जारिर्यच्छ्रुतश्रवाः
ततो युतायुस्तस्यापि निरमित्रोऽथ तत्सुतः
सुनक्षत्रः सुनक्षत्राद्बृहत्सेनोऽथ कर्मजित्‌
ततः सुतञ्जयाद्विप्रः शुचिस्तस्य भविष्यति
क्षेमोऽथ सुव्रतस्तस्माद्धर्मसूत्रः समस्ततः
द्युमत्सेनोऽथ सुमतिः सुबलो जनिता ततः
सुनीथः सत्यजिदथ विश्वजिद्यद्रिपुञ्जयः
बार्हद्रथाश्च भूपाला भाव्याः साहस्रवत्सरम्‌
श्रीशुक उवाच
अनोः सभानरश्चक्षुः परेष्णुश्च त्रयः सुताः
सभानरात्कालनरः सृञ्जयस्तत्सुतस्ततः
जनमेजयस्तस्य पुत्रो महाशालो महामनाः
उशीनरस्तितिक्षुश्च महामनस आत्मजौ
शिबिर्वरः कृमिर्दक्षश्चत्वारोशीनरात्मजाः
वृषादर्भः सुधीरश्च मद्रः केकय आत्मवान्‌
शिबेश्चत्वार एवासंस्तितिक्षोश्च रुषद्रथः
ततो होमोऽथ सुतपा बलिः सुतपसोऽभवत्‌
अङ्गवङ्गकलिङ्गाद्याः सुह्मपुण्ड्रौड्रसंज्ञिताः
जज्ञिरे दीर्घतमसो बलेः क्षेत्रे महीक्षितः
चक्रुः स्वनाम्ना विषयान्षडिमान्प्राच्यकांश्च ते
खलपानोऽङ्गतो जज्ञे तस्माद्दिविरथस्ततः
सुतो धर्मरथो यस्य जज्ञे चित्ररथोऽप्रजाः
रोमपाद इति ख्यातस्तस्मै दशरथः सखा
शान्तां स्वकन्यां प्रायच्छदृष्यशृङ्ग उवाह याम्‌
देवेऽवर्षति यं रामा आनिन्युर्हरिणीसुतम्‌
नाट्यसङ्गीतवादित्रैर्विभ्रमालिङ्गनार्हणैः
स तु राज्ञोऽनपत्यस्य निरूप्येष्टिं मरुत्वते
प्रजामदाद्दशरथो येन लेभेऽप्रजाः प्रजाः
चतुरङ्गो रोमपादात्पृथुलाक्षस्तु तत्सुतः
बृहद्रथो बृहत्कर्मा बृहद्भानुश्च तत्सुताः
आद्याद्बृहन्मनास्तस्माज्जयद्रथ उदाहृतः
विजयस्तस्य सम्भूत्यां ततो धृतिरजायत
ततो धृतव्रतस्तस्य सत्कर्माधिरथस्ततः
योऽसौ गङ्गातटे क्रीडन्मञ्जूषान्तर्गतं शिशुम्‌
कुन्त्यापविद्धं कानीनमनपत्योऽकरोत्सुतम्‌
वृषसेनः सुतस्तस्य कर्णस्य जगतीपते
द्रुह्योश्च तनयो बभ्रुः सेतुस्तस्यात्मजस्ततः
आरब्धस्तस्य गान्धारस्तस्य धर्मस्ततो धृतः
धृतस्य दुर्मदस्तस्मात्प्रचेताः प्राचेतसः शतम्‌
म्लेच्छाधिपतयोऽभूवन्नुदीचीं दिशमाश्रिताः
तुर्वसोश्च सुतो वह्निर्वह्नेर्भर्गोऽथ भानुमान्‌
त्रिभानुस्तत्सुतोऽस्यापि करन्धम उदारधीः
मरुतस्तत्सुतोऽपुत्रः पुत्रं पौरवमन्वभूत्‌
दुष्मन्तः स पुनर्भेजे स्ववंशं राज्यकामुकः
ययातेर्ज्येष्ठपुत्रस्य यदोर्वंशं नरर्षभ
वर्णयामि महापुण्यं सर्वपापहरं नृणाम्‌
यदोर्वंशं नरः श्रुत्वा सर्वपापैः प्रमुच्यते
यत्रावतीर्णो भगवान्परमात्मा नराकृतिः
यदोः सहस्रजित्क्रोष्टा नलो रिपुरिति श्रुताः
चत्वारः सूनवस्तत्र शतजित्प्रथमात्मजः
महाहयो रेणुहयो हैहयश्चेति तत्सुताः
धर्मस्तु हैहयसुतो नेत्रः कुन्तेः पिता ततः
सोहञ्जिरभवत्कुन्तेर्महिष्मान्भद्रसेनकः
दुर्मदो भद्रसेनस्य धनकः कृतवीर्यसूः
कृताग्निः कृतवर्मा च कृतौजा धनकात्मजाः
अर्जुनः कृतवीर्यस्य सप्तद्वीपेश्वरोऽभवत्‌
दत्तात्रेयाद्धरेरंशात्प्राप्तयोगमहागुणः
न नूनं कार्तवीर्यस्य गतिं यास्यन्ति पार्थिवाः
यज्ञदानतपोयोगैः श्रुतवीर्यदयादिभिः
पञ्चाशीति सहस्राणि ह्यव्याहतबलः समाः
अनष्टवित्तस्मरणो बुभुजेऽक्षय्यषड्वसु
तस्य पुत्रसहस्रेषु पञ्चैवोर्वरिता मृधे
जयध्वजः शूरसेनो वृषभो मधुरूर्जितः
जयध्वजात्तालजङ्घस्तस्य पुत्रशतं त्वभूत्‌
क्षत्रं यत्तालजङ्घाख्यमौर्वतेजोपसंहृतम्‌
तेषां ज्येष्ठो वीतिहोत्रो वृष्णिः पुत्रो मधोः स्मृतः
तस्य पुत्रशतं त्वासीद्वृष्णिज्येष्ठं यतः कुलम्‌
माधवा वृष्णयो राजन्यादवाश्चेति संज्ञिताः
यदुपुत्रस्य च क्रोष्टोः पुत्रो वृजिनवांस्ततः
स्वाहितोऽतो विषद्गुर्वै तस्य चित्ररथस्ततः
शशबिन्दुर्महायोगी महाभागो महानभूत्‌
चतुर्दशमहारत्नश्चक्रवर्त्यपराजितः
तस्य पत्नीसहस्राणां दशानां सुमहायशाः
दशलक्षसहस्राणि पुत्राणां तास्वजीजनत्‌
तेषां तु षट्प्रधानानां पृथुश्रवस आत्मजः
धर्मो नामोशना तस्य हयमेधशतस्य याट्‌
तत्सुतो रुचकस्तस्य पञ्चासन्नात्मजाः शृणु
पुरुजिद्रुक्मरुक्मेषु पृथुज्यामघसंज्ञिताः
ज्यामघस्त्वप्रजोऽप्यन्यां भार्यां शैब्यापतिर्भयात्‌
नाविन्दच्छत्रुभवनाद्भोज्यां कन्यामहारषीत्‌
रथस्थां तां निरीक्ष्याह शैब्या पतिममर्षिता
केयं कुहक मत्स्थानं रथमारोपितेति वै
स्नुषा तवेत्यभिहिते स्मयन्ती पतिमब्रवीत्‌
अहं बन्ध्यासपत्नी च स्नुषा मे युज्यते कथम्‌
जनयिष्यसि यं राज्ञि तस्येयमुपयुज्यते
अन्वमोदन्त तद्विश्वे देवाः पितर एव च
शैब्या गर्भमधात्काले कुमारं सुषुवे शुभम्‌
स विदर्भ इति प्रोक्त उपयेमे स्नुषां सतीम्‌
श्रीशुक उवाच
तस्यां विदर्भोऽजनयत्पुत्रौ नाम्ना कुशक्रथौ
तृतीयं रोमपादं च विदर्भकुलनन्दनम्‌
रोमपादसुतो बभ्रुर्बभ्रोः कृतिरजायत
उशिकस्तत्सुतस्तस्माच्चेदिश्चैद्यादयो नृपाः
क्रथस्य कुन्तिः पुत्रोऽभूद्वृष्णिस्तस्याथ निर्वृतिः
ततो दशार्हो नाम्नाभूत्तस्य व्योमः सुतस्ततः
जीमूतो विकृतिस्तस्य यस्य भीमरथः सुतः
ततो नवरथः पुत्रो जातो दशरथस्ततः
करम्भिः शकुनेः पुत्रो देवरातस्तदात्मजः
देवक्षत्रस्ततस्तस्य मधुः कुरुवशादनुः
पुरुहोत्रस्त्वनोः पुत्रस्तस्यायुः सात्वतस्ततः
भजमानो भजिर्दिव्यो वृष्णिर्देवावृधोऽन्धकः
सात्वतस्य सुताः सप्त महाभोजश्च मारिष
भजमानस्य निम्लोचिः किङ्कणो धृष्टिरेव च
एकस्यामात्मजाः पत्न्यामन्यस्यां च त्रयः सुताः
शताजिच्च सहस्राजिदयुताजिदिति प्रभो
बभ्रुर्देवावृधसुतस्तयोः श्लोकौ पठन्त्यमू
यथैव शृणुमो दूरात्सम्पश्यामस्तथान्तिकात्‌
बभ्रुः श्रेष्ठो मनुष्याणां देवैर्देवावृधः समः
पुरुषाः पञ्चषष्टिश्च षट्सहस्राणि चाष्ट च
येऽमृतत्वमनुप्राप्ता बभ्रोर्देवावृधादपि
महाभोजोऽतिधर्मात्मा भोजा आसंस्तदन्वये
वृष्णेः सुमित्रः पुत्रोऽभूद्युधाजिच्च परन्तप
शिनिस्तस्यानमित्रश्च निघ्नोऽभूदनमित्रतः
सत्राजितः प्रसेनश्च निघ्नस्याथासतुः सुतौ
अनमित्रसुतो योऽन्यः शिनिस्तस्य च सत्यकः
युयुधानः सात्यकिर्वै जयस्तस्य कुणिस्ततः
युगन्धरोऽनमित्रस्य वृष्णिः पुत्रोऽपरस्ततः
श्वफल्कश्चित्ररथश्च गान्दिन्यां च श्वफल्कतः
अक्रूरप्रमुखा आसन्पुत्रा द्वादश विश्रुताः
आसङ्गः सारमेयश्च मृदुरो मृदुविद्गिरिः
धर्मवृद्धः सुकर्मा च क्षेत्रोपेक्षोऽरिमर्दनः
शत्रुघ्नो गन्धमादश्च प्रतिबाहुश्च द्वादश
तेषां स्वसा सुचाराख्या द्वावक्रूरसुतावपि
देववानुपदेवश्च तथा चित्ररथात्मजाः
पृथुर्विदूरथाद्याश्च बहवो वृष्णिनन्दनाः
कुकुरो भजमानश्च शुचिः कम्बलबर्हिषः
कुकुरस्य सुतो वह्निर्विलोमा तनयस्ततः
कपोतरोमा तस्यानुः सखा यस्य च तुम्बुरुः
अन्धकाद्दुन्दुभिस्तस्मादविद्योतः पुनर्वसुः
तस्याहुकश्चाहुकी च कन्या चैवाहुकात्मजौ
देवकश्चोग्रसेनश्च चत्वारो देवकात्मजाः
देववानुपदेवश्च सुदेवो देववर्धनः
तेषां स्वसारः सप्तासन्धृतदेवादयो नृप
शान्तिदेवोपदेवा च श्रीदेवा देवरक्षिता
सहदेवा देवकी च वसुदेव उवाह ताः
कंसः सुनामा न्यग्रोधः कङ्कः शङ्कुः सुहूस्तथा
राष्ट्रपालोऽथ धृष्टिश्च तुष्टिमानौग्रसेनयः
कंसा कंसवती कङ्का शूरभू राष्ट्रपालिका
उग्रसेनदुहितरो वसुदेवानुजस्त्रियः
शूरो विदूरथादासीद्भजमानस्तु तत्सुतः
शिनिस्तस्मात्स्वयं भोजो हृदिकस्तत्सुतो मतः
देवमीढः शतधनुः कृतवर्मेति तत्सुताः
देवमीढस्य शूरस्य मारिषा नाम पत्न्यभूत्‌
तस्यां स जनयामास दश पुत्रानकल्मषान्‌
वसुदेवं देवभागं देवश्रवसमानकम्‌
सृञ्जयं श्यामकं कङ्कं शमीकं वत्सकं वृकम्‌
देवदुन्दुभयो नेदुरानका यस्य जन्मनि
वसुदेवं हरेः स्थानं वदन्त्यानकदुन्दुभिम्‌
पृथा च श्रुतदेवा च श्रुतकीर्तिः श्रुतश्रवाः
राजाधिदेवी चैतेषां भगिन्यः पञ्च कन्यकाः
कुन्तेः सख्युः पिता शूरो ह्यपुत्रस्य पृथामदात्‌
साप दुर्वाससो विद्यां देवहूतीं प्रतोषितात्‌
तस्या वीर्यपरीक्षार्थमाजुहाव रविं शुचिः
तदैवोपागतं देवं वीक्ष्य विस्मितमानसा
प्रत्ययार्थं प्रयुक्ता मे याहि देव क्षमस्व मे
अमोघं देवसन्दर्शमादधे त्वयि चात्मजम्‌
योनिर्यथा न दुष्येत कर्ताहं ते सुमध्यमे
इति तस्यां स आधाय गर्भं सूर्यो दिवं गतः
सद्यः कुमारः सञ्जज्ञे द्वितीय इव भास्करः
तं सात्यजन्नदीतोये कृच्छ्राल्लोकस्य बिभ्यती
प्रपितामहस्तामुवाह पाण्डुर्वै सत्यविक्रमः
श्रुतदेवां तु कारूषो वृद्धशर्मा समग्रहीत्‌
यस्यामभूद्दन्तवक्र ऋषिशप्तो दितेः सुतः
कैकेयो धृष्टकेतुश्च श्रुतकीर्तिमविन्दत
सन्तर्दनादयस्तस्यां पञ्चासन्कैकयाः सुताः
राजाधिदेव्यामावन्त्यौ जयसेनोऽजनिष्ट ह
दमघोषश्चेदिराजः श्रुतश्रवसमग्रहीत्‌
शिशुपालः सुतस्तस्याः कथितस्तस्य सम्भवः
देवभागस्य कंसायां चित्रकेतुबृहद्बलौ
कंसवत्यां देवश्रवसः सुवीर इषुमांस्तथा
बकः कङ्कात्तु कङ्कायां सत्यजित्पुरुजित्तथा
सृञ्जयो राष्ट्रपाल्यां च वृषदुर्मर्षणादिकान्‌
हरिकेशहिरण्याक्षौ शूरभूम्यां च श्यामकः
मिश्रकेश्यामप्सरसि वृकादीन्वत्सकस्तथा
तक्षपुष्करशालादीन्दुर्वाक्ष्यां वृक आदधे
सुमित्रार्जुनपालादीन्समीकात्तु सुदामनी
आनकः कर्णिकायां वै ऋतधामाजयावपि
पौरवी रोहिणी भद्रा मदिरा रोचना इला
देवकीप्रमुखाश्चासन्पत्न्य आनकदुन्दुभेः
बलं गदं सारणं च दुर्मदं विपुलं ध्रुवम्‌
वसुदेवस्तु रोहिण्यां कृतादीनुदपादयत्‌
सुभद्रो भद्रबाहुश्च दुर्मदो भद्र एव च
पौरव्यास्तनया ह्येते भूताद्या द्वादशाभवन्‌
नन्दोपनन्दकृतक शूराद्या मदिरात्मजाः
कौशल्या केशिनं त्वेकमसूत कुलनन्दनम्‌
रोचनायामतो जाता हस्तहेमाङ्गदादयः
इलायामुरुवल्कादीन्यदुमुख्यानजीजनत्‌
विपृष्ठो धृतदेवायामेक आनकदुन्दुभेः
शान्तिदेवात्मजा राजन्प्रशमप्रसितादयः
राजन्यकल्पवर्षाद्या उपदेवासुता दश
वसुहंससुवंशाद्याः श्रीदेवायास्तु षट्सुताः
देवरक्षितया लब्धा नव चात्र गदादयः
वसुदेवः सुतानष्टावादधे सहदेवया
प्रवरश्रुतमुख्यांश्च साक्षाद्धर्मो वसूनिव
वसुदेवस्तु देवक्यामष्ट पुत्रानजीजनत्‌
कीर्तिमन्तं सुषेणं च भद्रसेनमुदारधीः
ऋजुं सम्मर्दनं भद्रं सङ्कर्षणमहीश्वरम्‌
अष्टमस्तु तयोरासीत्स्वयमेव हरिः किल
सुभद्रा च महाभागा तव राजन्पितामही
यदा यदा हि धर्मस्य क्षयो वृद्धिश्च पाप्मनः
तदा तु भगवानीश आत्मानं सृजते हरिः
न ह्यस्य जन्मनो हेतुः कर्मणो वा महीपते
आत्ममायां विनेशस्य परस्य द्रष्टुरात्मनः
यन्मायाचेष्टितं पुंसः स्थित्युत्पत्त्यप्ययाय हि
अनुग्रहस्तन्निवृत्तेरात्मलाभाय चेष्यते
अक्षौहिणीनां पतिभिरसुरैर्नृपलाञ्छनैः
भुव आक्रम्यमाणाया अभाराय कृतोद्यमः
कर्माण्यपरिमेयाणि मनसापि सुरेश्वरैः
सहसङ्कर्षणश्चक्रे भगवान्मधुसूदनः
कलौ जनिष्यमाणानां दुःखशोकतमोनुदम्‌
अनुग्रहाय भक्तानां सुपुण्यं व्यतनोद्यशः
यस्मिन्सत्कर्णपीयुषे यशस्तीर्थवरे सकृत्‌
श्रोत्राञ्जलिरुपस्पृश्य धुनुते कर्मवासनाम्‌
भोजवृष्ण्यन्धकमधु शूरसेनदशार्हकैः
श्लाघनीयेहितः शश्वत्कुरुसृञ्जयपाण्डुभिः
स्निग्धस्मितेक्षितोदारैर्वाक्यैर्विक्रमलीलया
नृलोकं रमयामास मूर्त्या सर्वाङ्गरम्यया
यस्याननं मकरकुण्डलचारुकर्ण भ्राजत्कपोलसुभगं सविलासहासम्‌
नित्योत्सवं न ततृपुर्दृशिभिः पिबन्त्यो नार्यो नराश्च मुदिताः कुपिता निमेश्च
जातो गतः पितृगृहाद्व्रजमेधितार्थो हत्वा रिपून्सुतशतानि कृतोरुदारः
उत्पाद्य तेषु पुरुषः क्रतुभिः समीजे आत्मानमात्मनिगमं प्रथयन्जनेषु
पृथ्व्याः स वै गुरुभरं क्षपयन्कुरूणामन्तःसमुत्थकलिना युधि भूपचम्वः
दृष्ट्या विधूय विजये जयमुद्विघोष्य प्रोच्योद्धवाय च परं समगात्स्वधाम
श्रीराजोवाच
कथितो वंशविस्तारो भवता सोमसूर्ययोः
राज्ञां चोभयवंश्यानां चरितं परमाद्भुतम्‌
यदोश्च धर्मशीलस्य नितरां मुनिसत्तम
तत्रांशेनावतीर्णस्य विष्णोर्वीर्याणि शंस नः
अवतीर्य यदोर्वंशे भगवान्भूतभावनः
कृतवान्यानि विश्वात्मा तानि नो वद विस्तरात्‌
निवृत्ततर्षैरुपगीयमानाद्भवौषधाच्छ्रोत्रमनोऽभिरामात्‌
क उत्तमश्लोकगुणानुवादात्पुमान्विरज्येत विना पशुघ्नात्‌
पितामहा मे समरेऽमरञ्जयैर्देवव्रताद्यातिरथैस्तिमिङ्गिलैः
दुरत्ययं कौरवसैन्यसागरं कृत्वातरन्वत्सपदं स्म यत्प्लवाः
द्रौण्यस्त्रविप्लुष्टमिदं मदङ्गं सन्तानबीजं कुरुपाण्डवानाम्‌
जुगोप कुक्षिं गत आत्तचक्रो मातुश्च मे यः शरणं गतायाः
वीर्याणि तस्याखिलदेहभाजामन्तर्बहिः पूरुषकालरूपैः
प्रयच्छतो मृत्युमुतामृतं च मायामनुष्यस्य वदस्व विद्वन्‌
रोहिण्यास्तनयः प्रोक्तो रामः सङ्कर्षणस्त्वया
देवक्या गर्भसम्बन्धः कुतो देहान्तरं विना
कस्मान्मुकुन्दो भगवान्पितुर्गेहाद्व्रजं गतः
क्व वासं ज्ञातिभिः सार्धं कृतवान्सात्वतां पतिः
व्रजे वसन्किमकरोन्मधुपुर्यां च केशवः
भ्रातरं चावधीत्कंसं मातुरद्धातदर्हणम्‌
देहं मानुषमाश्रित्य कति वर्षाणि वृष्णिभिः
यदुपुर्यां सहावात्सीत्पत्न्यः कत्यभवन्प्रभोः
एतदन्यच्च सर्वं मे मुने कृष्णविचेष्टितम्‌
वक्तुमर्हसि सर्वज्ञ श्रद्दधानाय विस्तृतम्‌
नैषातिदुःसहा क्षुन्मां त्यक्तोदमपि बाधते
पिबन्तं त्वन्मुखाम्भोज च्युतं हरिकथामृतम्‌
सूत उवाच
एवं निशम्य भृगुनन्दन साधुवादं
वैयासकिः स भगवानथ विष्णुरातम्‌
प्रत्यर्च्य कृष्णचरितं कलिकल्मषघ्नं
व्याहर्तुमारभत भागवतप्रधानः
श्रीशुक उवाच
सम्यग्व्यवसिता बुद्धिस्तव राजर्षिसत्तम
वासुदेवकथायां ते यज्जाता नैष्ठिकी रतिः
वासुदेवकथाप्रश्नः पुरुषांस्त्रीन्पुनाति हि
वक्तारं प्रच्छकं श्रोत्ऱ्‌^ईंस्तत्पादसलिलं यथा
भूमिर्दृप्तनृपव्याज दैत्यानीकशतायुतैः
आक्रान्ता भूरिभारेण ब्रह्माणं शरणं ययौ
गौर्भूत्वाश्रुमुखी खिन्ना क्रन्दन्ती करुणं विभोः
उपस्थितान्तिके तस्मै व्यसनं समवोचत
ब्रह्मा तदुपधार्याथ सह देवैस्तया सह
जगाम सत्रिनयनस्तीरं क्षीरपयोनिधेः
तत्र गत्वा जगन्नाथं देवदेवं वृषाकपिम्‌
पुरुषं पुरुषसूक्तेन उपतस्थे समाहितः
गिरं समाधौ गगने समीरितां निशम्य वेधास्त्रिदशानुवाच ह
गां पौरुषीं मे शृणुतामराः पुनर्विधीयतामाशु तथैव मा चिरम्‌
पुरैव पुंसावधृतो धराज्वरो भवद्भिरंशैर्यदुषूपजन्यताम्‌
स यावदुर्व्या भरमीश्वरेश्वरः स्वकालशक्त्या क्षपयंश्चरेद्भुवि
वसुदेवगृहे साक्षाद्भगवान्पुरुषः परः
जनिष्यते तत्प्रियार्थं सम्भवन्तु सुरस्त्रियः
वासुदेवकलानन्तः सहस्रवदनः स्वराट्‌
अग्रतो भविता देवो हरेः प्रियचिकीर्षया
विष्णोर्माया भगवती यया सम्मोहितं जगत्‌
आदिष्टा प्रभुणांशेन कार्यार्थे सम्भविष्यति
श्रीशुक उवाच
इत्यादिश्यामरगणान्प्रजापतिपतिर्विभुः
आश्वास्य च महीं गीर्भिः स्वधाम परमं ययौ
शूरसेनो यदुपतिर्मथुरामावसन्पुरीम्‌
माथुराञ्छूरसेनांश्च विषयान्बुभुजे पुरा
राजधानी ततः साभूत्सर्वयादवभूभुजाम्‌
मथुरा भगवान्यत्र नित्यं सन्निहितो हरिः
तस्यां तु कर्हिचिच्छौरिर्वसुदेवः कृतोद्वहः
देवक्या सूर्यया सार्धं प्रयाणे रथमारुहत्‌
उग्रसेनसुतः कंसः स्वसुः प्रियचिकीर्षया
रश्मीन्हयानां जग्राह रौक्मै रथशतैर्वृतः
चतुःशतं पारिबर्हं गजानां हेममालिनाम्‌
अश्वानामयुतं सार्धं रथानां च त्रिषट्शतम्‌
दासीनां सुकुमारीणां द्वे शते समलङ्कृते
दुहित्रे देवकः प्रादाद्याने दुहितृवत्सलः
शङ्खतूर्यमृदङ्गाश्च नेदुर्दुन्दुभयः समम्‌
प्रयाणप्रक्रमे तात वरवध्वोः सुमङ्गलम्‌
पथि प्रग्रहिणं कंसमाभाष्याहाशरीरवाक्‌
अस्यास्त्वामष्टमो गर्भो हन्ता यां वहसेऽबुध
इत्युक्तः स खलः पापो भोजानां कुलपांसनः
भगिनीं हन्तुमारब्धं खड्गपाणिः कचेऽग्रहीत्‌
तं जुगुप्सितकर्माणं नृशंसं निरपत्रपम्‌
वसुदेवो महाभाग उवाच परिसान्त्वयन्‌
श्रीवसुदेव उवाच
श्लाघनीयगुणः शूरैर्भवान्भोजयशस्करः
स कथं भगिनीं हन्यात्स्त्रियमुद्वाहपर्वणि
मृत्युर्जन्मवतां वीर देहेन सह जायते
अद्य वाब्दशतान्ते वा मृत्युर्वै प्राणिनां ध्रुवः
देहे पञ्चत्वमापन्ने देही कर्मानुगोऽवशः
देहान्तरमनुप्राप्य प्राक्तनं त्यजते वपुः
व्रजंस्तिष्ठन्पदैकेन यथैवैकेन गच्छति
यथा तृणजलौकैवं देही कर्मगतिं गतः
स्वप्ने यथा पश्यति देहमीदृशं मनोरथेनाभिनिविष्टचेतनः
दृष्टश्रुताभ्यां मनसानुचिन्तयन्प्रपद्यते तत्किमपि ह्यपस्मृतिः
यतो यतो धावति दैवचोदितं मनो विकारात्मकमाप पञ्चसु
गुणेषु मायारोचितेषु देह्यसौ प्रपद्यमानः सह तेन जायते
ज्योतिर्यथैवोदकपार्थिवेष्वदः
समीरवेगानुगतं विभाव्यते
एवं स्वमायारचितेष्वसौ पुमान्‌
गुणेषु रागानुगतो विमुह्यति
तस्मान्न कस्यचिद्द्रोहमाचरेत्स तथाविधः
आत्मनः क्षेममन्विच्छन्द्रोग्धुर्वै परतो भयम्‌
एषा तवानुजा बाला कृपणा पुत्रिकोपमा
हन्तुं नार्हसि कल्याणीमिमां त्वं दीनवत्सलः
श्रीशुक उवाच
एवं स सामभिर्भेदैर्बोध्यमानोऽपि दारुणः
न न्यवर्तत कौरव्य पुरुषादाननुव्रतः
निर्बन्धं तस्य तं ज्ञात्वा विचिन्त्यानकदुन्दुभिः
प्राप्तं कालं प्रतिव्योढुमिदं तत्रान्वपद्यत
मृत्युर्बुद्धिमतापोह्यो यावद्बुद्धिबलोदयम्‌
यद्यसौ न निवर्तेत नापराधोऽस्ति देहिनः
प्रदाय मृत्यवे पुत्रान्मोचये कृपणामिमाम्‌
सुता मे यदि जायेरन्मृत्युर्वा न म्रियेत चेत्‌
विपर्ययो वा किं न स्याद्गतिर्धातुर्दुरत्यया
उपस्थितो निवर्तेत निवृत्तः पुनरापतेत्‌
अग्नेर्यथा दारुवियोगयोगयोरदृष्टतोऽन्यन्न निमित्तमस्ति
एवं हि जन्तोरपि दुर्विभाव्यः शरीरसंयोगवियोगहेतुः
एवं विमृश्य तं पापं यावदात्मनिदर्शनम्‌
पूजयामास वै शौरिर्बहुमानपुरःसरम्‌
प्रसन्नवदनाम्भोजो नृशंसं निरपत्रपम्‌
मनसा दूयमानेन विहसन्निदमब्रवीत्‌
श्रीवसुदेव उवाच
न ह्यस्यास्ते भयं सौम्य यद्वै साहाशरीरवाक्‌
पुत्रान्समर्पयिष्येऽस्या यतस्ते भयमुत्थितम्‌
श्रीशुक उवाच
स्वसुर्वधान्निववृते कंसस्तद्वाक्यसारवित्‌
वसुदेवोऽपि तं प्रीतः प्रशस्य प्राविशद्गृहम्‌
अथ काल उपावृत्ते देवकी सर्वदेवता
पुत्रान्प्रसुषुवे चाष्टौ कन्यां चैवानुवत्सरम्‌
कीर्तिमन्तं प्रथमजं कंसायानकदुन्दुभिः
अर्पयामास कृच्छ्रेण सोऽनृतादतिविह्वलः
किं दुःसहं नु साधूनां विदुषां किमपेक्षितम्‌
किमकार्यं कदर्याणां दुस्त्यजं किं धृतात्मनाम्‌
दृष्ट्वा समत्वं तच्छौरेः सत्ये चैव व्यवस्थितिम्‌
कंसस्तुष्टमना राजन्प्रहसन्निदमब्रवीत्‌
प्रतियातु कुमारोऽयं न ह्यस्मादस्ति मे भयम्‌
अष्टमाद्युवयोर्गर्भान्मृत्युर्मे विहितः किल
तथेति सुतमादाय ययावानकदुन्दुभिः
नाभ्यनन्दत तद्वाक्यमसतोऽविजितात्मनः
नन्दाद्या ये व्रजे गोपा याश्चामीषां च योषितः
वृष्णयो वसुदेवाद्या देवक्याद्या यदुस्त्रियः
सर्वे वै देवताप्राया उभयोरपि भारत
ज्ञातयो बन्धुसुहृदो ये च कंसमनुव्रताः
एतत्कंसाय भगवाञ्छशंसाभ्येत्य नारदः
भूमेर्भारायमाणानां दैत्यानां च वधोद्यमम्‌
ऋषेर्विनिर्गमे कंसो यदून्मत्वा सुरानिति
देवक्या गर्भसम्भूतं विष्णुं च स्ववधं प्रति
देवकीं वसुदेवं च निगृह्य निगडैर्गृहे
जातं जातमहन्पुत्रं तयोरजनशङ्कया
मातरं पितरं भ्रात्ऱ्‌^ईन्सर्वांश्च सुहृदस्तथा
घ्नन्ति ह्यसुतृपो लुब्धा राजानः प्रायशो भुवि
आत्मानमिह सञ्जातं जानन्प्राग्विष्णुना हतम्‌
महासुरं कालनेमिं यदुभिः स व्यरुध्यत
उग्रसेनं च पितरं यदुभोजान्धकाधिपम्‌
स्वयं निगृह्य बुभुजे शूरसेनान्महाबलः
श्रीशुक उवाच
प्रलम्बबकचाणूर तृणावर्तमहाशनैः
मुष्टिकारिष्टद्विविद पूतनाकेशीधेनुकैः
अन्यैश्चासुरभूपालैर्बाणभौमादिभिर्युतः
यदूनां कदनं चक्रे बली मागधसंश्रयः
ते पीडिता निविविशुः कुरुपञ्चालकेकयान्‌
शाल्वान्विदर्भान्निषधान्विदेहान्कोशलानपि
एके तमनुरुन्धाना ज्ञातयः पर्युपासते
हतेषु षट्सु बालेषु देवक्या औग्रसेनिना
सप्तमो वैष्णवं धाम यमनन्तं प्रचक्षते
गर्भो बभूव देवक्या हर्षशोकविवर्धनः
भगवानपि विश्वात्मा विदित्वा कंसजं भयम्‌
यदूनां निजनाथानां योगमायां समादिशत्‌
गच्छ देवि व्रजं भद्रे गोपगोभिरलङ्कृतम्‌
रोहिणी वसुदेवस्य भार्यास्ते नन्दगोकुले
अन्याश्च कंससंविग्ना विवरेषु वसन्ति हि
देवक्या जठरे गर्भं शेषाख्यं धाम मामकम्‌
तत्सन्निकृष्य रोहिण्या उदरे सन्निवेशय
अथाहमंशभागेन देवक्याः पुत्रतां शुभे
प्राप्स्यामि त्वं यशोदायां नन्दपत्न्यां भविष्यसि
अर्चिष्यन्ति मनुष्यास्त्वां सर्वकामवरेश्वरीम्‌
धूपोपहारबलिभिः सर्वकामवरप्रदाम्‌
नामधेयानि कुर्वन्ति स्थानानि च नरा भुवि
दुर्गेति भद्रकालीति विजया वैष्णवीति च
कुमुदा चण्डिका कृष्णा माधवी कन्यकेति च
माया नारायणीशानी शारदेत्यम्बिकेति च
गर्भसङ्कर्षणात्तं वै प्राहुः सङ्कर्षणं भुवि
रामेति लोकरमणाद्बलभद्रं बलोच्छ्रयात्‌
सन्दिष्टैवं भगवता तथेत्योमिति तद्वचः
प्रतिगृह्य परिक्रम्य गां गता तत्तथाकरोत्‌
गर्भे प्रणीते देवक्या रोहिणीं योगनिद्रया
अहो विस्रंसितो गर्भ इति पौरा विचुक्रुशुः
भगवानपि विश्वात्मा भक्तानामभयङ्करः
आविवेशांशभागेन मन आनकदुन्दुभेः
स बिभ्रत्पौरुषं धाम भ्राजमानो यथा रविः
दुरासदोऽतिदुर्धर्षो भूतानां सम्बभूव ह
ततो जगन्मङ्गलमच्युतांशं समाहितं शूरसुतेन देवी
दधार सर्वात्मकमात्मभूतं काष्ठा यथानन्दकरं मनस्तः
सा देवकी सर्वजगन्निवास निवासभूता नितरां न रेजे
भोजेन्द्रगेहेऽग्निशिखेव रुद्धा सरस्वती ज्ञानखले यथा सती
तां वीक्ष्य कंसः प्रभयाजितान्तरां
विरोचयन्तीं भवनं शुचिस्मिताम्‌
आहैष मे प्राणहरो हरिर्गुहां
ध्रुवं श्रितो यन्न पुरेयमीदृशी
किमद्य तस्मिन्करणीयमाशु मे यदर्थतन्त्रो न विहन्ति विक्रमम्‌
स्त्रियाः स्वसुर्गुरुमत्या वधोऽयं यशः श्रियं हन्त्यनुकालमायुः
स एष जीवन्खलु सम्परेतो वर्तेत योऽत्यन्तनृशंसितेन
देहे मृते तं मनुजाः शपन्ति गन्ता तमोऽन्धं तनुमानिनो ध्रुवम्‌
इति घोरतमाद्भावात्सन्निवृत्तः स्वयं प्रभुः
आस्ते प्रतीक्षंस्तज्जन्म हरेर्वैरानुबन्धकृत्‌
आसीनः संविशंस्तिष्ठन्भुञ्जानः पर्यटन्महीम्‌
चिन्तयानो हृषीकेशमपश्यत्तन्मयं जगत्‌
ब्रह्मा भवश्च तत्रैत्य मुनिभिर्नारदादिभिः
देवैः सानुचरैः साकं गीर्भिर्वृषणमैडयन्‌
सत्यव्रतं सत्यपरं त्रिसत्यं
सत्यस्य योनिं निहितं च सत्ये
सत्यस्य सत्यमृतसत्यनेत्रं
सत्यात्मकं त्वां शरणं प्रपन्नाः
एकायनोऽसौ द्विफलस्त्रिमूलश्चतूरसः पञ्चविधः षडात्मा
सप्तत्वगष्टविटपो नवाक्षो दशच्छदी द्विखगो ह्यादिवृक्षः
त्वमेक एवास्य सतः प्रसूतिस्त्वं सन्निधानं त्वमनुग्रहश्च
त्वन्मायया संवृतचेतसस्त्वां पश्यन्ति नाना न विपश्चितो ये
बिभर्षि रूपाण्यवबोध आत्मा क्षेमाय लोकस्य चराचरस्य
सत्त्वोपपन्नानि सुखावहानि सतामभद्राणि मुहुः खलानाम्‌
त्वय्यम्बुजाक्षाखिलसत्त्वधाम्नि समाधिनावेशितचेतसैके
त्वत्पादपोतेन महत्कृतेन कुर्वन्ति गोवत्सपदं भवाब्धिम्‌
स्वयं समुत्तीर्य सुदुस्तरं द्युमन्‌
भवार्णवं भीममदभ्रसौहृदाः
भवत्पदाम्भोरुहनावमत्र ते
निधाय याताः सदनुग्रहो भवान्‌
येऽन्येऽरविन्दाक्ष विमुक्तमानिनस्‌
त्वय्यस्तभावादविशुद्धबुद्धयः
आरुह्य कृच्छ्रेण परं पदं ततः
पतन्त्यधोऽनादृतयुष्मदङ्घ्रयः
तथा न ते माधव तावकाः क्वचिद्भ्रश्यन्ति मार्गात्त्वयि बद्धसौहृदाः
त्वयाभिगुप्ता विचरन्ति निर्भया विनायकानीकपमूर्धसु प्रभो
सत्त्वं विशुद्धं श्रयते भवान्स्थितौ
शरीरिणां श्रेयौपायनं वपुः
वेदक्रियायोगतपःसमाधिभिस्‌
तवार्हणं येन जनः समीहते
सत्त्वं न चेद्धातरिदं निजं भवेद्‌
विज्ञानमज्ञानभिदापमार्जनम्‌
गुणप्रकाशैरनुमीयते भवान्‌
प्रकाशते यस्य च येन वा गुणः
न नामरूपे गुणजन्मकर्मभिर्निरूपितव्ये तव तस्य साक्षिणः
मनोवचोभ्यामनुमेयवर्त्मनो देव क्रियायां प्रतियन्त्यथापि हि
शृण्वन्गृणन्संस्मरयंश्च चिन्तयन्‌
नामानि रूपाणि च मङ्गलानि ते
क्रियासु यस्त्वच्चरणारविन्दयोर्‌
आविष्टचेता न भवाय कल्पते
दिष्ट्या हरेऽस्या भवतः पदो भुवो
भारोऽपनीतस्तव जन्मनेशितुः
दिष्ट्याङ्कितां त्वत्पदकैः सुशोभनैर्‌
द्रक्ष्याम गां द्यां च तवानुकम्पिताम्‌
न तेऽभवस्येश भवस्य कारणं विना विनोदं बत तर्कयामहे
भवो निरोधः स्थितिरप्यविद्यया कृता यतस्त्वय्यभयाश्रयात्मनि
मत्स्याश्वकच्छपनृसिंहवराहहंस
राजन्यविप्रविबुधेषु कृतावतारः
त्वं पासि नस्त्रिभुवनं च यथाधुनेश
भारं भुवो हर यदूत्तम वन्दनं ते
दिष्ट्याम्ब ते कुक्षिगतः परः पुमान्‌
अंशेन साक्षाद्भगवान्भवाय नः
माभूद्भयं भोजपतेर्मुमूर्षोर्‌
गोप्ता यदूनां भविता तवात्मजः
श्रीशुक उवाच
इत्यभिष्टूय पुरुषं यद्रूपमनिदं यथा
ब्रह्मेशानौ पुरोधाय देवाः प्रतिययुर्दिवम्‌
श्रीशुक उवाच
अथ सर्वगुणोपेतः कालः परमशोभनः
यर्ह्येवाजनजन्मर्क्षं शान्तर्क्षग्रहतारकम्‌
दिशः प्रसेदुर्गगनं निर्मलोडुगणोदयम्‌
मही मङ्गलभूयिष्ठ पुरग्रामव्रजाकरा
नद्यः प्रसन्नसलिला ह्रदा जलरुहश्रियः
द्विजालिकुलसन्नाद स्तवका वनराजयः
ववौ वायुः सुखस्पर्शः पुण्यगन्धवहः शुचिः
अग्नयश्च द्विजातीनां शान्तास्तत्र समिन्धत
मनांस्यासन्प्रसन्नानि साधूनामसुरद्रुहाम्‌
जायमानेऽजने तस्मिन्नेदुर्दुन्दुभयः समम्‌
जगुः किन्नरगन्धर्वास्तुष्टुवुः सिद्धचारणाः
विद्याधर्यश्च ननृतुरप्सरोभिः समं मुदा
मुमुचुर्मुनयो देवाः सुमनांसि मुदान्विताः
मन्दं मन्दं जलधरा जगर्जुरनुसागरम्‌
निशीथे तमौद्भूते जायमाने जनार्दने
देवक्यां देवरूपिण्यां विष्णुः सर्वगुहाशयः
आविरासीद्यथा प्राच्यां दिशीन्दुरिव पुष्कलः
तमद्भुतं बालकमम्बुजेक्षणं चतुर्भुजं शङ्खगदाद्युदायुधम्‌
श्रीवत्सलक्ष्मं गलशोभिकौस्तुभं पीताम्बरं सान्द्रपयोदसौभगम्‌
महार्हवैदूर्यकिरीटकुण्डल त्विषा परिष्वक्तसहस्रकुन्तलम्‌
उद्दामकाञ्च्यङ्गदकङ्कणादिभिर्विरोचमानं वसुदेव ऐक्षत
स विस्मयोत्फुल्लविलोचनो हरिं सुतं विलोक्यानकदुन्दुभिस्तदा
कृष्णावतारोत्सवसम्भ्रमोऽस्पृशन्मुदा द्विजेभ्योऽयुतमाप्लुतो गवाम्‌
अथैनमस्तौदवधार्य पूरुषं परं नताङ्गः कृतधीः कृताञ्जलिः
स्वरोचिषा भारत सूतिकागृहं विरोचयन्तं गतभीः प्रभाववित्‌
श्रीवसुदेव उवाच
विदितोऽसि भवान्साक्षात्पुरुषः प्रकृतेः परः
केवलानुभवानन्द स्वरूपः सर्वबुद्धिदृक्‌
स एव स्वप्रकृत्येदं सृष्ट्वाग्रे त्रिगुणात्मकम्‌
तदनु त्वं ह्यप्रविष्टः प्रविष्ट इव भाव्यसे
यथेमेऽविकृता भावास्तथा ते विकृतैः सह
नानावीर्याः पृथग्भूता विराजं जनयन्ति हि
सन्निपत्य समुत्पाद्य दृश्यन्तेऽनुगता इव
प्रागेव विद्यमानत्वान्न तेषामिह सम्भवः
एवं भवान्बुद्ध्यनुमेयलक्षणैर्ग्राह्यैर्गुणैः सन्नपि तद्गुणाग्रहः
अनावृतत्वाद्बहिरन्तरं न ते सर्वस्य सर्वात्मन आत्मवस्तुनः
य आत्मनो दृश्यगुणेषु सन्निति व्यवस्यते स्वव्यतिरेकतोऽबुधः
विनानुवादं न च तन्मनीषितं सम्यग्यतस्त्यक्तमुपाददत्पुमान्‌
त्वत्तोऽस्य जन्मस्थितिसंयमान्विभो
वदन्त्यनीहादगुणादविक्रियात्‌
त्वयीश्वरे ब्रह्मणि नो विरुध्यते
त्वदाश्रयत्वादुपचर्यते गुणैः
स त्वं त्रिलोकस्थितये स्वमायया
बिभर्षि शुक्लं खलु वर्णमात्मनः
सर्गाय रक्तं रजसोपबृंहितं
कृष्णं च वर्णं तमसा जनात्यये
त्वमस्य लोकस्य विभो रिरक्षिषुर्गृहेऽवतीर्णोऽसि ममाखिलेश्वर
राजन्यसंज्ञासुरकोटियूथपैर्निर्व्यूह्यमाना निहनिष्यसे चमूः
अयं त्वसभ्यस्तव जन्म नौ गृहे
श्रुत्वाग्रजांस्ते न्यवधीत्सुरेश्वर
स तेऽवतारं पुरुषैः समर्पितं
श्रुत्वाधुनैवाभिसरत्युदायुधः
श्रीशुक उवाच
अथैनमात्मजं वीक्ष्य महापुरुषलक्षणम्‌
देवकी तमुपाधावत्कंसाद्भीता सुविस्मिता
श्रीदेवक्युवाच
रूपं यत्तत्प्राहुरव्यक्तमाद्यं
ब्रह्म ज्योतिर्निर्गुणं निर्विकारम्‌
सत्तामात्रं निर्विशेषं निरीहं
स त्वं साक्षाद्विष्णुरध्यात्मदीपः
नष्टे लोके द्विपरार्धावसाने महाभूतेष्वादिभूतं गतेषु
व्यक्तेऽव्यक्तं कालवेगेन याते भवानेकः शिष्यतेऽशेषसंज्ञः
योऽयं कालस्तस्य तेऽव्यक्तबन्धो
चेष्टामाहुश्चेष्टते येन विश्वम्‌
निमेषादिर्वत्सरान्तो महीयांस्‌
तं त्वेशानं क्षेमधाम प्रपद्ये
मर्त्यो मृत्युव्यालभीतः पलायन्लोकान्सर्वान्निर्भयं नाध्यगच्छत्‌
त्वत्पादाब्जं प्राप्य यदृच्छयाद्य सुस्थः शेते मृत्युरस्मादपैति
स त्वं घोरादुग्रसेनात्मजान्नस्त्राहि त्रस्तान्भृत्यवित्रासहासि
रूपं चेदं पौरुषं ध्यानधिष्ण्यं मा प्रत्यक्षं मांसदृशां कृषीष्ठाः
जन्म ते मय्यसौ पापो मा विद्यान्मधुसूदन
समुद्विजे भवद्धेतोः कंसादहमधीरधीः
उपसंहर विश्वात्मन्नदो रूपमलौकिकम्‌
शङ्खचक्रगदापद्म श्रिया जुष्टं चतुर्भुजम्‌
विश्वं यदेतत्स्वतनौ निशान्ते यथावकाशं पुरुषः परो भवान्‌
बिभर्ति सोऽयं मम गर्भगोऽभूदहो नृलोकस्य विडम्बनं हि तत्‌
श्रीभगवानुवाच
त्वमेव पूर्वसर्गेऽभूः पृश्निः स्वायम्भुवे सति
तदायं सुतपा नाम प्रजापतिरकल्मषः
युवां वै ब्रह्मणादिष्टौ प्रजासर्गे यदा ततः
सन्नियम्येन्द्रियग्रामं तेपाथे परमं तपः
वर्षवातातपहिम घर्मकालगुणाननु
सहमानौ श्वासरोध विनिर्धूतमनोमलौ
शीर्णपर्णानिलाहारावुपशान्तेन चेतसा
मत्तः कामानभीप्सन्तौ मदाराधनमीहतुः
एवं वां तप्यतोस्तीव्रं तपः परमदुष्करम्‌
दिव्यवर्षसहस्राणि द्वादशेयुर्मदात्मनोः
तदा वां परितुष्टोऽहममुना वपुषानघे
तपसा श्रद्धया नित्यं भक्त्या च हृदि भावितः
प्रादुरासं वरदराड्युवयोः कामदित्सया
व्रियतां वर इत्युक्ते मादृशो वां वृतः सुतः
अजुष्टग्राम्यविषयावनपत्यौ च दम्पती
न वव्राथेऽपवर्गं मे मोहितौ देवमायया
गते मयि युवां लब्ध्वा वरं मत्सदृशं सुतम्‌
ग्राम्यान्भोगानभुञ्जाथां युवां प्राप्तमनोरथौ
अदृष्ट्वान्यतमं लोके शीलौदार्यगुणैः समम्‌
अहं सुतो वामभवं पृश्निगर्भ इति श्रुतः
तयोर्वां पुनरेवाहमदित्यामास कश्यपात्‌
उपेन्द्र इति विख्यातो वामनत्वाच्च वामनः
तृतीयेऽस्मिन्भवेऽहं वै तेनैव वपुषाथ वाम्‌
जातो भूयस्तयोरेव सत्यं मे व्याहृतं सति
एतद्वां दर्शितं रूपं प्राग्जन्मस्मरणाय मे
नान्यथा मद्भवं ज्ञानं मर्त्यलिङ्गेन जायते
युवां मां पुत्रभावेन ब्रह्मभावेन चासकृत्‌
चिन्तयन्तौ कृतस्नेहौ यास्येथे मद्गतिं पराम्‌
श्रीशुक उवाच
इत्युक्त्वासीद्धरिस्तूष्णीं भगवानात्ममायया
पित्रोः सम्पश्यतोः सद्यो बभूव प्राकृतः शिशुः
ततश्च शौरिर्भगवत्प्रचोदितः
सुतं समादाय स सूतिकागृहात्‌
यदा बहिर्गन्तुमियेष तर्ह्यजा
या योगमायाजनि नन्दजायया
तया हृतप्रत्ययसर्ववृत्तिषु द्वाःस्थेषु पौरेष्वपि शायितेष्वथ
द्वारश्च सर्वाः पिहिता दुरत्यया बृहत्कपाटायसकीलशृङ्खलैः
ताः कृष्णवाहे वसुदेव आगते स्वयं व्यवर्यन्त यथा तमो रवेः
ववर्ष पर्जन्य उपांशुगर्जितः शेषोऽन्वगाद्वारि निवारयन्फणैः
मघोनि वर्षत्यसकृद्यमानुजा गम्भीरतोयौघजवोर्मिफेनिला
भयानकावर्तशताकुला नदी मार्गं ददौ सिन्धुरिव श्रियः पतेः
नन्दव्रजं शौरिरुपेत्य तत्र तान्‌
गोपान्प्रसुप्तानुपलभ्य निद्रया
सुतं यशोदाशयने निधाय तत्‌
सुतामुपादाय पुनर्गृहानगात्‌
देवक्याः शयने न्यस्य वसुदेवोऽथ दारिकाम्‌
प्रतिमुच्य पदोर्लोहमास्ते पूर्ववदावृतः
यशोदा नन्दपत्नी च जातं परमबुध्यत
न तल्लिङ्गं परिश्रान्ता निद्रयापगतस्मृतिः
श्रीशुक उवाच
बहिरन्तःपुरद्वारः सर्वाः पूर्ववदावृताः
ततो बालध्वनिं श्रुत्वा गृहपालाः समुत्थिताः
ते तु तूर्णमुपव्रज्य देवक्या गर्भजन्म तत्‌
आचख्युर्भोजराजाय यदुद्विग्नः प्रतीक्षते
स तल्पात्तूर्णमुत्थाय कालोऽयमिति विह्वलः
सूतीगृहमगात्तूर्णं प्रस्खलन्मुक्तमूर्धजः
तमाह भ्रातरं देवी कृपणा करुणं सती
स्नुषेयं तव कल्याण स्त्रियं मा हन्तुमर्हसि
बहवो हिंसिता भ्रातः शिशवः पावकोपमाः
त्वया दैवनिसृष्टेन पुत्रिकैका प्रदीयताम्‌
नन्वहं ते ह्यवरजा दीना हतसुता प्रभो
दातुमर्हसि मन्दाया अङ्गेमां चरमां प्रजाम्‌
श्रीशुक उवाच
उपगुह्यात्मजामेवं रुदत्या दीनदीनवत्‌
याचितस्तां विनिर्भर्त्स्य हस्तादाचिच्छिदे खलः
तां गृहीत्वा चरणयोर्जातमात्रां स्वसुः सुताम्‌
अपोथयच्छिलापृष्ठे स्वार्थोन्मूलितसौहृदः
सा तद्धस्तात्समुत्पत्य सद्यो देव्यम्बरं गता
अदृश्यतानुजा विष्णोः सायुधाष्टमहाभुजा
दिव्यस्रगम्बरालेप रत्नाभरणभूषिता
धनुःशूलेषुचर्मासि शङ्खचक्रगदाधरा
सिद्धचारणगन्धर्वैरप्सरःकिन्नरोरगैः
उपाहृतोरुबलिभिः स्तूयमानेदमब्रवीत्‌
किं मया हतया मन्द जातः खलु तवान्तकृत्‌
यत्र क्व वा पूर्वशत्रुर्मा हिंसीः कृपणान्वृथा
इति प्रभाष्य तं देवी माया भगवती भुवि
बहुनामनिकेतेषु बहुनामा बभूव ह
तयाभिहितमाकर्ण्य कंसः परमविस्मितः
देवकीं वसुदेवं च विमुच्य प्रश्रितोऽब्रवीत्‌
अहो भगिन्यहो भाम मया वां बत पाप्मना
पुरुषाद इवापत्यं बहवो हिंसिताः सुताः
स त्वहं त्यक्तकारुण्यस्त्यक्तज्ञातिसुहृत्खलः
कान्लोकान्वै गमिष्यामि ब्रह्महेव मृतः श्वसन्‌
दैवमप्यनृतं वक्ति न मर्त्या एव केवलम्‌
यद्विश्रम्भादहं पापः स्वसुर्निहतवाञ्छिशून्‌
मा शोचतं महाभागावात्मजान्स्वकृतं भुजः
जान्तवो न सदैकत्र दैवाधीनास्तदासते
भुवि भौमानि भूतानि यथा यान्त्यपयान्ति च
नायमात्मा तथैतेषु विपर्येति यथैव भूः
यथानेवंविदो भेदो यत आत्मविपर्ययः
देहयोगवियोगौ च संसृतिर्न निवर्तते
तस्माद्भद्रे स्वतनयान्मया व्यापादितानपि
मानुशोच यतः सर्वः स्वकृतं विन्दतेऽवशः
यावद्धतोऽस्मि हन्तास्मी त्यात्मानं मन्यतेऽस्वदृक्‌
तावत्तदभिमान्यज्ञो बाध्यबाधकतामियात्‌
क्षमध्वं मम दौरात्म्यं साधवो दीनवत्सलाः
इत्युक्त्वाश्रुमुखः पादौ श्यालः स्वस्रोरथाग्रहीत्‌
मोचयामास निगडाद्विश्रब्धः कन्यकागिरा
देवकीं वसुदेवं च दर्शयन्नात्मसौहृदम्‌
भ्रातुः समनुतप्तस्य क्षान्तरोषा च देवकी
व्यसृजद्वसुदेवश्च प्रहस्य तमुवाच ह
एवमेतन्महाभाग यथा वदसि देहिनाम्‌
अज्ञानप्रभवाहंधीः स्वपरेति भिदा यतः
शोकहर्षभयद्वेष लोभमोहमदान्विताः
मिथो घ्नन्तं न पश्यन्ति भावैर्भावं पृथग्दृशः
श्रीशुक उवाच
कंस एवं प्रसन्नाभ्यां विशुद्धं प्रतिभाषितः
देवकीवसुदेवाभ्यामनुज्ञातोऽविशद्गृहम्‌
तस्यां रात्र्यां व्यतीतायां कंस आहूय मन्त्रिणः
तेभ्य आचष्ट तत्सर्वं यदुक्तं योगनिद्रया
आकर्ण्य भर्तुर्गदितं तमूचुर्देवशत्रवः
देवान्प्रति कृतामर्षा दैतेया नातिकोविदाः
एवं चेत्तर्हि भोजेन्द्र पुरग्रामव्रजादिषु
अनिर्दशान्निर्दशांश्च हनिष्यामोऽद्य वै शिशून्‌
किमुद्यमैः करिष्यन्ति देवाः समरभीरवः
नित्यमुद्विग्नमनसो ज्याघोषैर्धनुषस्तव
अस्यतस्ते शरव्रातैर्हन्यमानाः समन्ततः
जिजीविषव उत्सृज्य पलायनपरा ययुः
केचित्प्राञ्जलयो दीना न्यस्तशस्त्रा दिवौकसः
मुक्तकच्छशिखाः केचिद्भीताः स्म इति वादिनः
न त्वं विस्मृतशस्त्रास्त्रान्विरथान्भयसंवृतान्‌
हंस्यन्यासक्तविमुखान्भग्नचापानयुध्यतः
किं क्षेमशूरैर्विबुधैरसंयुगविकत्थनैः
रहोजुषा किं हरिणा शम्भुना वा वनौकसा
किमिन्द्रेणाल्पवीर्येण ब्रह्मणा वा तपस्यता
तथापि देवाः सापत्न्यान्नोपेक्ष्या इति मन्महे
ततस्तन्मूलखनने नियुङ्क्ष्वास्माननुव्रतान्‌
यथामयोऽङ्गे समुपेक्षितो नृभिर्न शक्यते रूढपदश्चिकित्सितुम्‌
यथेन्द्रियग्राम उपेक्षितस्तथा रिपुर्महान्बद्धबलो न चाल्यते
मूलं हि विष्णुर्देवानां यत्र धर्मः सनातनः
तस्य च ब्रह्मगोविप्रास्तपो यज्ञाः सदक्षिणाः
तस्मात्सर्वात्मना राजन्ब्राह्मणान्ब्रह्मवादिनः
तपस्विनो यज्ञशीलान्गाश्च हन्मो हविर्दुघाः
विप्रा गावश्च वेदाश्च तपः सत्यं दमः शमः
श्रद्धा दया तितिक्षा च क्रतवश्च हरेस्तनूः
स हि सर्वसुराध्यक्षो ह्यसुरद्विड्गुहाशयः
तन्मूला देवताः सर्वाः सेश्वराः सचतुर्मुखाः
अयं वै तद्वधोपायो यदृषीणां विहिंसनम्‌
श्रीशुक उवाच
एवं दुर्मन्त्रिभिः कंसः सह सम्मन्त्र्य दुर्मतिः
ब्रह्महिंसां हितं मेने कालपाशावृतोऽसुरः
सन्दिश्य साधुलोकस्य कदने कदनप्रियान्‌
कामरूपधरान्दिक्षु दानवान्गृहमाविशत्‌
ते वै रजःप्रकृतयस्तमसा मूढचेतसः
सतां विद्वेषमाचेरुरारादागतमृत्यवः
आयुः श्रियं यशो धर्मं लोकानाशिष एव च
हन्ति श्रेयांसि सर्वाणि पुंसो महदतिक्रमः
श्रीशुक उवाच
नन्दस्त्वात्मज उत्पन्ने जाताह्लादो महामनाः
आहूय विप्रान्वेदज्ञान्स्नातः शुचिरलङ्कृतः
वाचयित्वा स्वस्त्ययनं जातकर्मात्मजस्य वै
कारयामास विधिवत्पितृदेवार्चनं तथा
धेनूनां नियुते प्रादाद्विप्रेभ्यः समलङ्कृते
तिलाद्रीन्सप्त रत्नौघ शातकौम्भाम्बरावृतान्‌
कालेन स्नानशौचाभ्यां संस्कारैस्तपसेज्यया
शुध्यन्ति दानैः सन्तुष्ट्या द्रव्याण्यात्मात्मविद्यया
सौमङ्गल्यगिरो विप्राः सूतमागधवन्दिनः
गायकाश्च जगुर्नेदुर्भेर्यो दुन्दुभयो मुहुः
व्रजः सम्मृष्टसंसिक्त द्वाराजिरगृहान्तरः
चित्रध्वजपताकास्रक्‌ चैलपल्लवतोरणैः
गावो वृषा वत्सतरा हरिद्रातैलरूषिताः
विचित्रधातुबर्हस्रग्‌ वस्त्रकाञ्चनमालिनः
महार्हवस्त्राभरण कञ्चुकोष्णीषभूषिताः
गोपाः समाययू राजन्नानोपायनपाणयः
गोप्यश्चाकर्ण्य मुदिता यशोदायाः सुतोद्भवम्‌
आत्मानं भूषयां चक्रुर्वस्त्राकल्पाञ्जनादिभिः
नवकुङ्कुमकिञ्जल्क मुखपङ्कजभूतयः
बलिभिस्त्वरितं जग्मुः पृथुश्रोण्यश्चलत्कुचाः
गोप्यः सुमृष्टमणिकुण्डलनिष्ककण्ठ्यश्‌
चित्राम्बराः पथि शिखाच्युतमाल्यवर्षाः
नन्दालयं सवलया व्रजतीर्विरेजुर्‌
व्यालोलकुण्डलपयोधरहारशोभाः
ता आशिषः प्रयुञ्जानाश्चिरं पाहीति बालके
हरिद्राचूर्णतैलाद्भिः सिञ्चन्त्योऽजनमुज्जगुः
अवाद्यन्त विचित्राणि वादित्राणि महोत्सवे
कृष्णे विश्वेश्वरेऽनन्ते नन्दस्य व्रजमागते
गोपाः परस्परं हृष्टा दधिक्षीरघृताम्बुभिः
आसिञ्चन्तो विलिम्पन्तो नवनीतैश्च चिक्षिपुः
नन्दो महामनास्तेभ्यो वासोऽलङ्कारगोधनम्‌
सूतमागधवन्दिभ्यो येऽन्ये विद्योपजीविनः
तैस्तैः कामैरदीनात्मा यथोचितमपूजयत्‌
विष्णोराराधनार्थाय स्वपुत्रस्योदयाय च
रोहिणी च महाभागा नन्दगोपाभिनन्दिता
व्यचरद्दिव्यवासस्रक्‌ कण्ठाभरणभूषिता
तत आरभ्य नन्दस्य व्रजः सर्वसमृद्धिमान्‌
हरेर्निवासात्मगुणै रमाक्रीडमभून्नृप
गोपान्गोकुलरक्षायां निरूप्य मथुरां गतः
नन्दः कंसस्य वार्षिक्यं करं दातुं कुरूद्वह
वसुदेव उपश्रुत्य भ्रातरं नन्दमागतम्‌
ज्ञात्वा दत्तकरं राज्ञे ययौ तदवमोचनम्‌
तं दृष्ट्वा सहसोत्थाय देहः प्राणमिवागतम्‌
प्रीतः प्रियतमं दोर्भ्यां सस्वजे प्रेमविह्वलः
पूजितः सुखमासीनः पृष्ट्वानामयमादृतः
प्रसक्तधीः स्वात्मजयोरिदमाह विशाम्पते
दिष्ट्या भ्रातः प्रवयस इदानीमप्रजस्य ते
प्रजाशाया निवृत्तस्य प्रजा यत्समपद्यत
दिष्ट्या संसारचक्रेऽस्मिन्वर्तमानः पुनर्भवः
उपलब्धो भवानद्य दुर्लभं प्रियदर्शनम्‌
नैकत्र प्रियसंवासः सुहृदां चित्रकर्मणाम्‌
ओघेन व्यूह्यमानानां प्लवानां स्रोतसो यथा
कच्चित्पशव्यं निरुजं भूर्यम्बुतृणवीरुधम्‌
बृहद्वनं तदधुना यत्रास्से त्वं सुहृद्वृतः
भ्रातर्मम सुतः कच्चिन्मात्रा सह भवद्व्रजे
तातं भवन्तं मन्वानो भवद्भ्यामुपलालितः
पुंसस्त्रिवर्गो विहितः सुहृदो ह्यनुभावितः
न तेषु क्लिश्यमानेषु त्रिवर्गोऽर्थाय कल्पते
श्रीनन्द उवाच
अहो ते देवकीपुत्राः कंसेन बहवो हताः
एकावशिष्टावरजा कन्या सापि दिवं गता
नूनं ह्यदृष्टनिष्ठोऽयमदृष्टपरमो जनः
अदृष्टमात्मनस्तत्त्वं यो वेद न स मुह्यति
श्रीवसुदेव उवाच
करो वै वार्षिको दत्तो राज्ञे दृष्टा वयं च वः
नेह स्थेयं बहुतिथं सन्त्युत्पाताश्च गोकुले
श्रीशुक उवाच
इति नन्दादयो गोपाः प्रोक्तास्ते शौरिणा ययुः
अनोभिरनडुद्युक्तैस्तमनुज्ञाप्य गोकुलम्‌
श्रीशुक उवाच
नन्दः पथि वचः शौरेर्न मृषेति विचिन्तयन्‌
हरिं जगाम शरणमुत्पातागमशङ्कितः
कंसेन प्रहिता घोरा पूतना बालघातिनी
शिशूंश्चचार निघ्नन्ती पुरग्रामव्रजादिषु
न यत्र श्रवणादीनि रक्षोघ्नानि स्वकर्मसु
कुर्वन्ति सात्वतां भर्तुर्यातुधान्यश्च तत्र हि
सा खेचर्येकदोत्पत्य पूतना नन्दगोकुलम्‌
योषित्वा माययात्मानं प्राविशत्कामचारिणी
तां केशबन्धव्यतिषक्तमल्लिकां
बृहन्नितम्बस्तनकृच्छ्रमध्यमाम्‌
सुवाससं कल्पितकर्णभूषण
त्विषोल्लसत्कुन्तलमण्डिताननाम्‌
वल्गुस्मितापाङ्गविसर्गवीक्षितैर्‌
मनो हरन्तीं वनितां व्रजौकसाम्‌
अमंसताम्भोजकरेण रूपिणीं
गोप्यः श्रियं द्रष्टुमिवागतां पतिम्‌
बालग्रहस्तत्र विचिन्वती शिशून्यदृच्छया नन्दगृहेऽसदन्तकम्‌
बालं प्रतिच्छन्ननिजोरुतेजसं ददर्श तल्पेऽग्निमिवाहितं भसि
विबुध्य तां बालकमारिकाग्रहं चराचरात्मा स निमीलितेक्षणः
अनन्तमारोपयदङ्कमन्तकं यथोरगं सुप्तमबुद्धिरज्जुधीः
तां तीक्ष्णचित्तामतिवामचेष्टितां वीक्ष्यान्तरा कोषपरिच्छदासिवत्‌
वरस्त्रियं तत्प्रभया च धर्षिते निरीक्ष्यमाणे जननी ह्यतिष्ठताम्‌
तस्मिन्स्तनं दुर्जरवीर्यमुल्बणं
घोराङ्कमादाय शिशोर्ददावथ
गाढं कराभ्यां भगवान्प्रपीड्य तत्‌
प्राणैः समं रोषसमन्वितोऽपिबत्‌
सा मुञ्च मुञ्चालमिति प्रभाषिणी निष्पीड्यमानाखिलजीवमर्मणि
विवृत्य नेत्रे चरणौ भुजौ मुहुः प्रस्विन्नगात्रा क्षिपती रुरोद ह
तस्याः स्वनेनातिगभीररंहसा साद्रिर्मही द्यौश्च चचाल सग्रहा
रसा दिशश्च प्रतिनेदिरे जनाः पेतुः क्षितौ वज्रनिपातशङ्कया
निशाचरीत्थं व्यथितस्तना व्यसुर्‌
व्यादाय केशांश्चरणौ भुजावपि
प्रसार्य गोष्ठे निजरूपमास्थिता
वज्राहतो वृत्र इवापतन्नृप
पतमानोऽपि तद्देहस्त्रिगव्यूत्यन्तरद्रुमान्‌
चूर्णयामास राजेन्द्र महदासीत्तदद्भुतम्‌
ईषामात्रोग्रदंष्ट्रास्यं गिरिकन्दरनासिकम्‌
गण्डशैलस्तनं रौद्रं प्रकीर्णारुणमूर्धजम्‌
अन्धकूपगभीराक्षं पुलिनारोहभीषणम्‌
बद्धसेतुभुजोर्वङ्घ्रि शून्यतोयह्रदोदरम्‌
सन्तत्रसुः स्म तद्वीक्ष्य गोपा गोप्यः कलेवरम्‌
पूर्वं तु तन्निःस्वनित भिन्नहृत्कर्णमस्तकाः
बालं च तस्या उरसि क्रीडन्तमकुतोभयम्‌
गोप्यस्तूर्णं समभ्येत्य जगृहुर्जातसम्भ्रमाः
यशोदारोहिणीभ्यां ताः समं बालस्य सर्वतः
रक्षां विदधिरे सम्यग्गोपुच्छभ्रमणादिभिः
गोमूत्रेण स्नापयित्वा पुनर्गोरजसार्भकम्‌
रक्षां चक्रुश्च शकृता द्वादशाङ्गेषु नामभिः
गोप्यः संस्पृष्टसलिला अङ्गेषु करयोः पृथक्‌
न्यस्यात्मन्यथ बालस्य बीजन्यासमकुर्वत
अव्यादजोऽङ्घ्रि मणिमांस्तव जान्वथोरू
यज्ञोऽच्युतः कटितटं जठरं हयास्यः
हृत्केशवस्त्वदुर ईश इनस्तु कण्ठं
विष्णुर्भुजं मुखमुरुक्रम ईश्वरः कम्‌
चक्र्यग्रतः सहगदो हरिरस्तु पश्चात्‌
त्वत्पार्श्वयोर्धनुरसी मधुहाजनश्च
कोणेषु शङ्ख उरुगाय उपर्युपेन्द्रस्‌
तार्क्ष्यः क्षितौ हलधरः पुरुषः समन्तात्‌
इन्द्रियाणि हृषीकेशः प्राणान्नारायणोऽवतु
श्वेतद्वीपपतिश्चित्तं मनो योगेश्वरोऽवतु
पृश्निगर्भस्तु ते बुद्धिमात्मानं भगवान्परः
क्रीडन्तं पातु गोविन्दः शयानं पातु माधवः
व्रजन्तमव्याद्वैकुण्ठ आसीनं त्वां श्रियः पतिः
भुञ्जानं यज्ञभुक्पातु सर्वग्रहभयङ्करः
डाकिन्यो यातुधान्यश्च कुष्माण्डा येऽर्भकग्रहाः
भूतप्रेतपिशाचाश्च यक्षरक्षोविनायकाः
कोटरा रेवती ज्येष्ठा पूतना मातृकादयः
उन्मादा ये ह्यपस्मारा देहप्राणेन्द्रियद्रुहः
स्वप्नदृष्टा महोत्पाता वृद्धा बालग्रहाश्च ये
सर्वे नश्यन्तु ते विष्णोर्नामग्रहणभीरवः
श्रीशुक उवाच
इति प्रणयबद्धाभिर्गोपीभिः कृतरक्षणम्‌
पाययित्वा स्तनं माता सन्न्यवेशयदात्मजम्‌
तावन्नन्दादयो गोपा मथुराया व्रजं गताः
विलोक्य पूतनादेहं बभूवुरतिविस्मिताः
नूनं बतर्षिः सञ्जातो योगेशो वा समास सः
स एव दृष्टो ह्युत्पातो यदाहानकदुन्दुभिः
कलेवरं परशुभिश्छित्त्वा तत्ते व्रजौकसः
दूरे क्षिप्त्वावयवशो न्यदहन्काष्ठवेष्टितम्‌
दह्यमानस्य देहस्य धूमश्चागुरुसौरभः
उत्थितः कृष्णनिर्भुक्त सपद्याहतपाप्मनः
पूतना लोकबालघ्नी राक्षसी रुधिराशना
जिघांसयापि हरये स्तनं दत्त्वाप सद्गतिम्‌
किं पुनः श्रद्धया भक्त्या कृष्णाय परमात्मने
यच्छन्प्रियतमं किं नु रक्तास्तन्मातरो यथा
पद्भ्यां भक्तहृदिस्थाभ्यां वन्द्याभ्यां लोकवन्दितैः
अङ्गं यस्याः समाक्रम्य भगवानपि तत्स्तनम्‌
यातुधान्यपि सा स्वर्गमवाप जननीगतिम्‌
कृष्णभुक्तस्तनक्षीराः किमु गावोऽनुमातरः
पयांसि यासामपिबत्पुत्रस्नेहस्नुतान्यलम्‌
भगवान्देवकीपुत्रः कैवल्याद्यखिलप्रदः
तासामविरतं कृष्णे कुर्वतीनां सुतेक्षणम्‌
न पुनः कल्पते राजन्संसारोऽज्ञानसम्भवः
कटधूमस्य सौरभ्यमवघ्राय व्रजौकसः
किमिदं कुत एवेति वदन्तो व्रजमाययुः
ते तत्र वर्णितं गोपैः पूतनागमनादिकम्‌
श्रुत्वा तन्निधनं स्वस्ति शिशोश्चासन्सुविस्मिताः
नन्दः स्वपुत्रमादाय प्रेत्यागतमुदारधीः
मूर्ध्न्युपाघ्राय परमां मुदं लेभे कुरूद्वह
य एतत्पूतनामोक्षं कृष्णस्यार्भकमद्भुतम्‌
शृणुयाच्छ्रद्धया मर्त्यो गोविन्दे लभते रतिम्‌
श्रीराजोवाच
येन येनावतारेण भगवान्हरिरीश्वरः
करोति कर्णरम्याणि मनोज्ञानि च नः प्रभो
यच्छृण्वतोऽपैत्यरतिर्वितृष्णा सत्त्वं च शुद्ध्यत्यचिरेण पुंसः
भक्तिर्हरौ तत्पुरुषे च सख्यं तदेव हारं वद मन्यसे चेत्‌
अथान्यदपि कृष्णस्य तोकाचरितमद्भुतम्‌
मानुषं लोकमासाद्य तज्जातिमनुरुन्धतः
श्रीशुक उवाच
कदाचिदौत्थानिककौतुकाप्लवे जन्मर्क्षयोगे समवेतयोषिताम्‌
वादित्रगीतद्विजमन्त्रवाचकैश्चकार सूनोरभिषेचनं सती
नन्दस्य पत्नी कृतमज्जनादिकं विप्रैः कृतस्वस्त्ययनं सुपूजितैः
अन्नाद्यवासःस्रगभीष्टधेनुभिः सञ्जातनिद्राक्षमशीशयच्छनैः
औत्थानिकौत्सुक्यमना मनस्विनी समागतान्पूजयती व्रजौकसः
नैवाशृणोद्वै रुदितं सुतस्य सा रुदन्स्तनार्थी चरणावुदक्षिपत्‌
अधःशयानस्य शिशोरनोऽल्पक प्रवालमृद्वङ्घ्रिहतं व्यवर्तत
विध्वस्तनानारसकुप्यभाजनं व्यत्यस्तचक्राक्षविभिन्नकूबरम्‌
दृष्ट्वा यशोदाप्रमुखा व्रजस्त्रिय
औत्थानिके कर्मणि याः समागताः
नन्दादयश्चाद्भुतदर्शनाकुलाः
कथं स्वयं वै शकटं विपर्यगात्‌
ऊचुरव्यवसितमतीन्गोपान्गोपीश्च बालकाः
रुदतानेन पादेन क्षिप्तमेतन्न संशयः
न ते श्रद्दधिरे गोपा बालभाषितमित्युत
अप्रमेयं बलं तस्य बालकस्य न ते विदुः
रुदन्तं सुतमादाय यशोदा ग्रहशङ्किता
कृतस्वस्त्ययनं विप्रैः सूक्तैः स्तनमपाययत्‌
पूर्ववत्स्थापितं गोपैर्बलिभिः सपरिच्छदम्‌
विप्रा हुत्वार्चयां चक्रुर्दध्यक्षतकुशाम्बुभिः
येऽसूयानृतदम्भेर्षा हिंसामानविवर्जिताः
न तेषां सत्यशीलानामाशिषो विफलाः कृताः
इति बालकमादाय सामर्ग्यजुरुपाकृतैः
जलैः पवित्रौषधिभिरभिषिच्य द्विजोत्तमैः
वाचयित्वा स्वस्त्ययनं नन्दगोपः समाहितः
हुत्वा चाग्निं द्विजातिभ्यः प्रादादन्नं महागुणम्‌
गावः सर्वगुणोपेता वासःस्रग्रुक्ममालिनीः
आत्मजाभ्युदयार्थाय प्रादात्ते चान्वयुञ्जत
विप्रा मन्त्रविदो युक्तास्तैर्याः प्रोक्तास्तथाशिषः
ता निष्फला भविष्यन्ति न कदाचिदपि स्फुटम्‌
एकदारोहमारूढं लालयन्ती सुतं सती
गरिमाणं शिशोर्वोढुं न सेहे गिरिकूटवत्‌
भूमौ निधाय तं गोपी विस्मिता भारपीडिता
महापुरुषमादध्यौ जगतामास कर्मसु
दैत्यो नाम्ना तृणावर्तः कंसभृत्यः प्रणोदितः
चक्रवातस्वरूपेण जहारासीनमर्भकम्‌
गोकुलं सर्वमावृण्वन्मुष्णंश्चक्षूंषि रेणुभिः
ईरयन्सुमहाघोर शब्देन प्रदिशो दिशः
मुहूर्तमभवद्गोष्ठं रजसा तमसावृतम्‌
सुतं यशोदा नापश्यत्तस्मिन्न्यस्तवती यतः
नापश्यत्कश्चनात्मानं परं चापि विमोहितः
तृणावर्तनिसृष्टाभिः शर्कराभिरुपद्रुतः
इति खरपवनचक्रपांशुवर्षे सुतपदवीमबलाविलक्ष्य माता
अतिकरुणमनुस्मरन्त्यशोचद्भुवि पतिता मृतवत्सका यथा गौः
रुदितमनुनिशम्य तत्र गोप्यो भृशमनुतप्तधियोऽश्रुपूर्णमुख्यः
रुरुदुरनुपलभ्य नन्दसूनुं पवन उपारतपांशुवर्षवेगे
तृणावर्तः शान्तरयो वात्यारूपधरो हरन्‌
कृष्णं नभोगतो गन्तुं नाशक्नोद्भूरिभारभृत्‌
तमश्मानं मन्यमान आत्मनो गुरुमत्तया
गले गृहीत उत्स्रष्टुं नाशक्नोदद्भुतार्भकम्‌
गलग्रहणनिश्चेष्टो दैत्यो निर्गतलोचनः
अव्यक्तरावो न्यपतत्सहबालो व्यसुर्व्रजे
तमन्तरिक्षात्पतितं शिलायां विशीर्णसर्वावयवं करालम्‌
पुरं यथा रुद्रशरेण विद्धं स्त्रियो रुदत्यो ददृशुः समेताः
प्रादाय मात्रे प्रतिहृत्य विस्मिताः कृष्णं च तस्योरसि लम्बमानम्‌
तं स्वस्तिमन्तं पुरुषादनीतं विहायसा मृत्युमुखात्प्रमुक्तम्‌
गोप्यश्च गोपाः किल नन्दमुख्या लब्ध्वा पुनः प्रापुरतीव मोदम्‌
अहो बतात्यद्भुतमेष रक्षसा बालो निवृत्तिं गमितोऽभ्यगात्पुनः
हिंस्रः स्वपापेन विहिंसितः खलः साधुः समत्वेन भयाद्विमुच्यते
किं नस्तपश्चीर्णमधोक्षजार्चनं
पूर्तेष्टदत्तमुत भूतसौहृदम्‌
यत्सम्परेतः पुनरेव बालको
दिष्ट्या स्वबन्धून्प्रणयन्नुपस्थितः
दृष्ट्वाद्भुतानि बहुशो नन्दगोपो बृहद्वने
वसुदेववचो भूयो मानयामास विस्मितः
एकदार्भकमादाय स्वाङ्कमारोप्य भामिनी
प्रस्नुतं पाययामास स्तनं स्नेहपरिप्लुता
पीतप्रायस्य जननी सुतस्य रुचिरस्मितम्‌
मुखं लालयती राजञ्जृम्भतो ददृशे इदम्‌
खं रोदसी ज्योतिरनीकमाशाः सूर्येन्दुवह्निश्वसनाम्बुधींश्च
द्वीपान्नगांस्तद्दुहित्ऱ्‌^ईर्वनानि भूतानि यानि स्थिरजङ्गमानि
सा वीक्ष्य विश्वं सहसा राजन्सञ्जातवेपथुः
सम्मील्य मृगशावाक्षी नेत्रे आसीत्सुविस्मिता
श्रीशुक उवाच
गर्गः पुरोहितो राजन्यदूनां सुमहातपाः
व्रजं जगाम नन्दस्य वसुदेवप्रचोदितः
तं दृष्ट्वा परमप्रीतः प्रत्युत्थाय कृताञ्जलिः
आनर्चाधोक्षजधिया प्रणिपातपुरःसरम्‌
सूपविष्टं कृतातिथ्यं गिरा सूनृतया मुनिम्‌
नन्दयित्वाब्रवीद्ब्रह्मन्पूर्णस्य करवाम किम्‌
महद्विचलनं न्ऱ्‌^ईणां गृहिणां दीनचेतसाम्‌
निःश्रेयसाय भगवन्कल्पते नान्यथा क्वचित्‌
ज्योतिषामयनं साक्षाद्यत्तज्ज्ञानमतीन्द्रियम्‌
प्रणीतं भवता येन पुमान्वेद परावरम्‌
त्वं हि ब्रह्मविदां श्रेष्ठः संस्कारान्कर्तुमर्हसि
बालयोरनयोर्न्ऱ्‌^ईणां जन्मना ब्राह्मणो गुरुः
श्रीगर्ग उवाच
यदूनामहमाचार्यः ख्यातश्च भुवि सर्वदा
सुतं मया संस्कृतं ते मन्यते देवकीसुतम्‌
कंसः पापमतिः सख्यं तव चानकदुन्दुभेः
देवक्या अष्टमो गर्भो न स्त्री भवितुमर्हति
इति सञ्चिन्तयञ्छ्रुत्वा देवक्या दारिकावचः
अपि हन्ता गताशङ्कस्तर्हि तन्नोऽनयो भवेत्‌
श्रीनन्द उवाच
अलक्षितोऽस्मिन्रहसि मामकैरपि गोव्रजे
कुरु द्विजातिसंस्कारं स्वस्तिवाचनपूर्वकम्‌
श्रीशुक उवाच
एवं सम्प्रार्थितो विप्रः स्वचिकीर्षितमेव तत्‌
चकार नामकरणं गूढो रहसि बालयोः
श्रीगर्ग उवाच
अयं हि रोहिणीपुत्रो रमयन्सुहृदो गुणैः
आख्यास्यते राम इति बलाधिक्याद्बलं विदुः
यदूनामपृथग्भावात्सङ्कर्षणमुशन्त्यपि
आसन्वर्णास्त्रयो ह्यस्य गृह्णतोऽनुयुगं तनूः
शुक्लो रक्तस्तथा पीत इदानीं कृष्णतां गतः
प्रागयं वसुदेवस्य क्वचिज्जातस्तवात्मजः
वासुदेव इति श्रीमानभिज्ञाः सम्प्रचक्षते
बहूनि सन्ति नामानि रूपाणि च सुतस्य ते
गुणकर्मानुरूपाणि तान्यहं वेद नो जनाः
एष वः श्रेय आधास्यद्गोपगोकुलनन्दनः
अनेन सर्वदुर्गाणि यूयमञ्जस्तरिष्यथ
पुरानेन व्रजपते साधवो दस्युपीडिताः
अराजके रक्ष्यमाणा जिग्युर्दस्यून्समेधिताः
य एतस्मिन्महाभागाः प्रीतिं कुर्वन्ति मानवाः
नारयोऽभिभवन्त्येतान्विष्णुपक्षानिवासुराः
तस्मान्नन्दात्मजोऽयं ते नारायणसमो गुणैः
श्रिया कीर्त्यानुभावेन गोपायस्व समाहितः
श्रीशुक उवाच
इत्यात्मानं समादिश्य गर्गे च स्वगृहं गते
नन्दः प्रमुदितो मेने आत्मानं पूर्णमाशिषाम्‌
कालेन व्रजताल्पेन गोकुले रामकेशवौ
जानुभ्यां सह पाणिभ्यां रिङ्गमाणौ विजह्रतुः
तावङ्घ्रियुग्ममनुकृष्य सरीसृपन्तौ
घोषप्रघोषरुचिरं व्रजकर्दमेषु
तन्नादहृष्टमनसावनुसृत्य लोकं
मुग्धप्रभीतवदुपेयतुरन्ति मात्रोः
तन्मातरौ निजसुतौ घृणया स्नुवन्त्यौ
पङ्काङ्गरागरुचिरावुपगृह्य दोर्भ्याम्‌
दत्त्वा स्तनं प्रपिबतोः स्म मुखं निरीक्ष्य
मुग्धस्मिताल्पदशनं ययतुः प्रमोदम्‌
यर्ह्यङ्गनादर्शनीयकुमारलीलाव्‌
अन्तर्व्रजे तदबलाः प्रगृहीतपुच्छैः
वत्सैरितस्तत उभावनुकृष्यमाणौ
प्रेक्षन्त्य उज्झितगृहा जहृषुर्हसन्त्यः
शृङ्ग्यग्निदंष्ट्र्यसिजलद्विजकण्टकेभ्यः
क्रीडापरावतिचलौ स्वसुतौ निषेद्धुम्‌
गृह्याणि कर्तुमपि यत्र न तज्जनन्यौ
शेकात आपतुरलं मनसोऽनवस्थाम्‌
कालेनाल्पेन राजर्षे रामः कृष्णश्च गोकुले
अघृष्टजानुभिः पद्भिर्विचक्रमतुरञ्जसा
ततस्तु भगवान्कृष्णो वयस्यैर्व्रजबालकैः
सहरामो व्रजस्त्रीणां चिक्रीडे जनयन्मुदम्‌
कृष्णस्य गोप्यो रुचिरं वीक्ष्य कौमारचापलम्‌
शृण्वन्त्याः किल तन्मातुरिति होचुः समागताः
वत्सान्मुञ्चन्क्वचिदसमये क्रोशसञ्जातहासः
स्तेयं स्वाद्वत्त्यथ दधिपयः कल्पितैः स्तेययोगैः
मर्कान्भोक्ष्यन्विभजति स चेन्नात्ति भाण्डं भिन्नत्ति
द्रव्यालाभे सगृहकुपितो यात्युपक्रोश्य तोकान्‌
हस्ताग्राह्ये रचयति विधिं पीठकोलूखलाद्यैश्‌
छिद्रं ह्यन्तर्निहितवयुनः शिक्यभाण्डेषु तद्वित्‌
ध्वान्तागारे धृतमणिगणं स्वाङ्गमर्थप्रदीपं
काले गोप्यो यर्हि गृहकृत्येषु सुव्यग्रचित्ताः
एवं धार्ष्ट्यान्युशति कुरुते मेहनादीनि वास्तौ
स्तेयोपायैर्विरचितकृतिः सुप्रतीको यथास्ते
इत्थं स्त्रीभिः सभयनयनश्रीमुखालोकिनीभिर्‌
व्याख्यातार्था प्रहसितमुखी न ह्युपालब्धुमैच्छत्‌
एकदा क्रीडमानास्ते रामाद्या गोपदारकाः
कृष्णो मृदं भक्षितवानिति मात्रे न्यवेदयन्‌
सा गृहीत्वा करे कृष्णमुपालभ्य हितैषिणी
यशोदा भयसम्भ्रान्त प्रेक्षणाक्षमभाषत
कस्मान्मृदमदान्तात्मन्भवान्भक्षितवान्रहः
वदन्ति तावका ह्येते कुमारास्तेऽग्रजोऽप्ययम्‌
नाहं भक्षितवानम्ब सर्वे मिथ्याभिशंसिनः
यदि सत्यगिरस्तर्हि समक्षं पश्य मे मुखम्‌
यद्येवं तर्हि व्यादेही त्युक्तः स भगवान्हरिः
व्यादत्ताव्याहतैश्वर्यः क्रीडामनुजबालकः
सा तत्र ददृशे विश्वं जगत्स्थास्नु च खं दिशः
साद्रिद्वीपाब्धिभूगोलं सवाय्वग्नीन्दुतारकम्‌
ज्योतिश्चक्रं जलं तेजो नभस्वान्वियदेव च
वैकारिकाणीन्द्रियाणि मनो मात्रा गुणास्त्रयः
एतद्विचित्रं सहजीवकाल स्वभावकर्माशयलिङ्गभेदम्‌
सूनोस्तनौ वीक्ष्य विदारितास्ये व्रजं सहात्मानमवाप शङ्काम्‌
किं स्वप्न एतदुत देवमाया किं वा मदीयो बत बुद्धिमोहः
अथो अमुष्यैव ममार्भकस्य यः कश्चनौत्पत्तिक आत्मयोगः
अथो यथावन्न वितर्कगोचरं चेतोमनःकर्मवचोभिरञ्जसा
यदाश्रयं येन यतः प्रतीयते सुदुर्विभाव्यं प्रणतास्मि तत्पदम्‌
अहं ममासौ पतिरेष मे सुतो व्रजेश्वरस्याखिलवित्तपा सती
गोप्यश्च गोपाः सहगोधनाश्च मे यन्माययेत्थं कुमतिः स मे गतिः
इत्थं विदिततत्त्वायां गोपिकायां स ईश्वरः
वैष्णवीं व्यतनोन्मायां पुत्रस्नेहमयीं विभुः
सद्यो नष्टस्मृतिर्गोपी सारोप्यारोहमात्मजम्‌
प्रवृद्धस्नेहकलिल हृदयासीद्यथा पुरा
त्रय्या चोपनिषद्भिश्च साङ्ख्ययोगैश्च सात्वतैः
उपगीयमानमाहात्म्यं हरिं सामन्यतात्मजम्‌
श्रीराजोवाच
नन्दः किमकरोद्ब्रह्मन्श्रेय एवं महोदयम्‌
यशोदा च महाभागा पपौ यस्याः स्तनं हरिः
पितरौ नान्वविन्देतां कृष्णोदारार्भकेहितम्‌
गायन्त्यद्यापि कवयो यल्लोकशमलापहम्‌
श्रीशुक उवाच
द्रोणो वसूनां प्रवरो धरया भार्यया सह
करिष्यमाण आदेशान्ब्रह्मणस्तमुवाच ह
जातयोर्नौ महादेवे भुवि विश्वेश्वरे हरौ
भक्तिः स्यात्परमा लोके ययाञ्जो दुर्गतिं तरेत्‌
अस्त्वित्युक्तः स भगवान्व्रजे द्रोणो महायशाः
जज्ञे नन्द इति ख्यातो यशोदा सा धराभवत्‌
ततो भक्तिर्भगवति पुत्रीभूते जनार्दने
दम्पत्योर्नितरामासीद्गोपगोपीषु भारत
कृष्णो ब्रह्मण आदेशं सत्यं कर्तुं व्रजे विभुः
सहरामो वसंश्चक्रे तेषां प्रीतिं स्वलीलया
श्रीशुक उवाच
एकदा गृहदासीषु यशोदा नन्दगेहिनी
कर्मान्तरनियुक्तासु निर्ममन्थ स्वयं दधि
यानि यानीह गीतानि तद्बालचरितानि च
दधिनिर्मन्थने काले स्मरन्ती तान्यगायत
क्षौमं वासः पृथुकटितटे बिभ्रती सूत्रनद्धं
पुत्रस्नेहस्नुतकुचयुगं जातकम्पं च सुभ्रूः
रज्ज्वाकर्षश्रमभुजचलत्कङ्कणौ कुण्डले च
स्विन्नं वक्त्रं कबरविगलन्मालती निर्ममन्थ
तां स्तन्यकाम आसाद्य मथ्नन्तीं जननीं हरिः
गृहीत्वा दधिमन्थानं न्यषेधत्प्रीतिमावहन्‌
तमङ्कमारूढमपाययत्स्तनं स्नेहस्नुतं सस्मितमीक्षती मुखम्‌
अतृप्तमुत्सृज्य जवेन सा ययावुत्सिच्यमाने पयसि त्वधिश्रिते
सञ्जातकोपः स्फुरितारुणाधरं सन्दश्य दद्भिर्दधिमन्थभाजनम्‌
भित्त्वा मृषाश्रुर्दृषदश्मना रहो जघास हैयङ्गवमन्तरं गतः
उत्तार्य गोपी सुशृतं पयः पुनः प्रविश्य संदृश्य च दध्यमत्रकम्‌
भग्नं विलोक्य स्वसुतस्य कर्म तज्जहास तं चापि न तत्र पश्यती
उलूखलाङ्घ्रेरुपरि व्यवस्थितं मर्काय कामं ददतं शिचि स्थितम्‌
हैयङ्गवं चौर्यविशङ्कितेक्षणं निरीक्ष्य पश्चात्सुतमागमच्छनैः
तामात्तयष्टिं प्रसमीक्ष्य सत्वरस्‌
ततोऽवरुह्यापससार भीतवत्‌
गोप्यन्वधावन्न यमाप योगिनां
क्षमं प्रवेष्टुं तपसेरितं मनः
अन्वञ्चमाना जननी बृहच्चलच्‌ छ्रोणीभराक्रान्तगतिः सुमध्यमा
जवेन विस्रंसितकेशबन्धन च्युतप्रसूनानुगतिः परामृशत्‌
कृतागसं तं प्ररुदन्तमक्षिणी कषन्तमञ्जन्मषिणी स्वपाणिना
उद्वीक्षमाणं भयविह्वलेक्षणं हस्ते गृहीत्वा भिषयन्त्यवागुरत्‌
त्यक्त्वा यष्टिं सुतं भीतं विज्ञायार्भकवत्सला
इयेष किल तं बद्धुं दाम्नातद्वीर्यकोविदा
न चान्तर्न बहिर्यस्य न पूर्वं नापि चापरम्‌
पूर्वापरं बहिश्चान्तर्जगतो यो जगच्च यः
तं मत्वात्मजमव्यक्तं मर्त्यलिङ्गमधोक्षजम्‌
गोपिकोलूखले दाम्ना बबन्ध प्राकृतं यथा
तद्दाम बध्यमानस्य स्वार्भकस्य कृतागसः
द्व्यङ्गुलोनमभूत्तेन सन्दधेऽन्यच्च गोपिका
यदासीत्तदपि न्यूनं तेनान्यदपि सन्दधे
तदपि द्व्यङ्गुलं न्यूनं यद्यदादत्त बन्धनम्‌
एवं स्वगेहदामानि यशोदा सन्दधत्यपि
गोपीनां सुस्मयन्तीनां स्मयन्ती विस्मिताभवत्‌
स्वमातुः स्विन्नगात्राया विस्रस्तकबरस्रजः
दृष्ट्वा परिश्रमं कृष्णः कृपयासीत्स्वबन्धने
एवं सन्दर्शिता ह्यङ्ग हरिणा भृत्यवश्यता
स्ववशेनापि कृष्णेन यस्येदं सेश्वरं वशे
नेमं विरिञ्चो न भवो न श्रीरप्यङ्गसंश्रया
प्रसादं लेभिरे गोपी यत्तत्प्राप विमुक्तिदात्‌
नायं सुखापो भगवान्देहिनां गोपिकासुतः
ज्ञानिनां चात्मभूतानां यथा भक्तिमतामिह
कृष्णस्तु गृहकृत्येषु व्यग्रायां मातरि प्रभुः
अद्राक्षीदर्जुनौ पूर्वं गुह्यकौ धनदात्मजौ
पुरा नारदशापेन वृक्षतां प्रापितौ मदात्‌
नलकूवरमणिग्रीवाविति ख्यातौ श्रियान्वितौ
श्रीराजोवाच
कथ्यतां भगवन्नेतत्तयोः शापस्य कारणम्‌
यत्तद्विगर्हितं कर्म येन वा देवर्षेस्तमः
श्रीशुक उवाच
रुद्रस्यानुचरौ भूत्वा सुदृप्तौ धनदात्मजौ
कैलासोपवने रम्ये मन्दाकिन्यां मदोत्कटौ
वारुणीं मदिरां पीत्वा मदाघूर्णितलोचनौ
स्त्रीजनैरनुगायद्भिश्चेरतुः पुष्पिते वने
अन्तः प्रविश्य गङ्गायामम्भोजवनराजिनि
चिक्रीडतुर्युवतिभिर्गजाविव करेणुभिः
यदृच्छया च देवर्षिर्भगवांस्तत्र कौरव
अपश्यन्नारदो देवौ क्षीबाणौ समबुध्यत
तं दृष्ट्वा व्रीडिता देव्यो विवस्त्राः शापशङ्किताः
वासांसि पर्यधुः शीघ्रं विवस्त्रौ नैव गुह्यकौ
तौ दृष्ट्वा मदिरामत्तौ श्रीमदान्धौ सुरात्मजौ
तयोरनुग्रहार्थाय शापं दास्यन्निदं जगौ
श्रीनारद उवाच
न ह्यन्यो जुषतो जोष्यान्बुद्धिभ्रंशो रजोगुणः
श्रीमदादाभिजात्यादिर्यत्र स्त्री द्यूतमासवः
हन्यन्ते पशवो यत्र निर्दयैरजितात्मभिः
मन्यमानैरिमं देहमजरामृत्यु नश्वरम्‌
देवसंज्ञितमप्यन्ते कृमिविड्भस्मसंज्ञितम्‌
भूतध्रुक्तत्कृते स्वार्थं किं वेद निरयो यतः
देहः किमन्नदातुः स्वं निषेक्तुर्मातुरेव च
मातुः पितुर्वा बलिनः क्रेतुरग्नेः शुनोऽपि वा
एवं साधारणं देहमव्यक्तप्रभवाप्ययम्‌
को विद्वानात्मसात्कृत्वा हन्ति जन्तूनृतेऽसतः
असतः श्रीमदान्धस्य दारिद्र्यं परमञ्जनम्‌
आत्मौपम्येन भूतानि दरिद्रः परमीक्षते
यथा कण्टकविद्धाङ्गो जन्तोर्नेच्छति तां व्यथाम्‌
जीवसाम्यं गतो लिङ्गैर्न तथाविद्धकण्टकः
दरिद्रो निरहंस्तम्भो मुक्तः सर्वमदैरिह
कृच्छ्रं यदृच्छयाप्नोति तद्धि तस्य परं तपः
नित्यं क्षुत्क्षामदेहस्य दरिद्रस्यान्नकाङ्क्षिणः
इन्द्रियाण्यनुशुष्यन्ति हिंसापि विनिवर्तते
दरिद्रस्यैव युज्यन्ते साधवः समदर्शिनः
सद्भिः क्षिणोति तं तर्षं तत आराद्विशुद्ध्यति
साधूनां समचित्तानां मुकुन्दचरणैषिणाम्‌
उपेक्ष्यैः किं धनस्तम्भैरसद्भिरसदाश्रयैः
तदहं मत्तयोर्माध्व्या वारुण्या श्रीमदान्धयोः
तमोमदं हरिष्यामि स्त्रैणयोरजितात्मनोः
यदिमौ लोकपालस्य पुत्रौ भूत्वा तमःप्लुतौ
न विवाससमात्मानं विजानीतः सुदुर्मदौ
अतोऽर्हतः स्थावरतां स्यातां नैवं यथा पुनः
स्मृतिः स्यान्मत्प्रसादेन तत्रापि मदनुग्रहात्‌
वासुदेवस्य सान्निध्यं लब्ध्वा दिव्यशरच्छते
वृत्ते स्वर्लोकतां भूयो लब्धभक्ती भविष्यतः
श्रीशुक उवाच
एवमुक्त्वा स देवर्षिर्गतो नारायणाश्रमम्‌
नलकूवरमणिग्रीवावासतुर्यमलार्जुनौ
ऋषेर्भागवतमुख्यस्य सत्यं कर्तुं वचो हरिः
जगाम शनकैस्तत्र यत्रास्तां यमलार्जुनौ
देवर्षिर्मे प्रियतमो यदिमौ धनदात्मजौ
तत्तथा साधयिष्यामि यद्गीतं तन्महात्मना
इत्यन्तरेणार्जुनयोः कृष्णस्तु यमयोर्ययौ
आत्मनिर्वेशमात्रेण तिर्यग्गतमुलूखलम्‌
बालेन निष्कर्षयतान्वगुलूखलं तद्‌
दामोदरेण तरसोत्कलिताङ्घ्रिबन्धौ
निष्पेततुः परमविक्रमितातिवेप
स्कन्धप्रवालविटपौ कृतचण्डशब्दौ
तत्र श्रिया परमया ककुभः स्फुरन्तौ
सिद्धावुपेत्य कुजयोरिव जातवेदाः
कृष्णं प्रणम्य शिरसाखिललोकनाथं
बद्धाञ्जली विरजसाविदमूचतुः स्म
कृष्ण कृष्ण महायोगिंस्त्वमाद्यः पुरुषः परः
व्यक्ताव्यक्तमिदं विश्वं रूपं ते ब्राह्मणा विदुः
त्वमेकः सर्वभूतानां देहास्वात्मेन्द्रियेश्वरः
त्वमेव कालो भगवान्विष्णुरव्यय ईश्वरः
त्वं महान्प्रकृतिः सूक्ष्मा रजःसत्त्वतमोमयी
त्वमेव पुरुषोऽध्यक्षः सर्वक्षेत्रविकारवित्‌
गृह्यमाणैस्त्वमग्राह्यो विकारैः प्राकृतैर्गुणैः
को न्विहार्हति विज्ञातुं प्राक्सिद्धं गुणसंवृतः
तस्मै तुभ्यं भगवते वासुदेवाय वेधसे
आत्मद्योतगुणैश्छन्न महिम्ने ब्रह्मणे नमः
यस्यावतारा ज्ञायन्ते शरीरेष्वशरीरिणः
तैस्तैरतुल्यातिशयैर्वीर्यैर्देहिष्वसङ्गतैः
स भवान्सर्वलोकस्य भवाय विभवाय च
अवतीर्णोऽंशभागेन साम्प्रतं पतिराशिषाम्‌
नमः परमकल्याण नमः परममङ्गल
वासुदेवाय शान्ताय यदूनां पतये नमः
अनुजानीहि नौ भूमंस्तवानुचरकिङ्करौ
दर्शनं नौ भगवत ऋषेरासीदनुग्रहात्‌
वाणी गुणानुकथने श्रवणौ कथायां
हस्तौ च कर्मसु मनस्तव पादयोर्नः
स्मृत्यां शिरस्तव निवासजगत्प्रणामे
दृष्टिः सतां दर्शनेऽस्तु भवत्तनूनाम्‌
श्रीशुक उवाच
इत्थं सङ्कीर्तितस्ताभ्यां भगवान्गोकुलेश्वरः
दाम्ना चोलूखले बद्धः प्रहसन्नाह गुह्यकौ
श्रीभगवानुवाच
ज्ञातं मम पुरैवैतदृषिणा करुणात्मना
यच्छ्रीमदान्धयोर्वाग्भिर्विभ्रंशोऽनुग्रहः कृतः
साधूनां समचित्तानां सुतरां मत्कृतात्मनाम्‌
दर्शनान्नो भवेद्बन्धः पुंसोऽक्ष्णोः सवितुर्यथा
तद्गच्छतं मत्परमौ नलकूवर सादनम्‌
सञ्जातो मयि भावो वामीप्सितः परमोऽभवः
श्रीशुक उवाच
इत्युक्तौ तौ परिक्रम्य प्रणम्य च पुनः पुनः
बद्धोलूखलमामन्त्र्य जग्मतुर्दिशमुत्तराम्‌
श्रीशुक उवाच
गोपा नन्दादयः श्रुत्वा द्रुमयोः पततो रवम्‌
तत्राजग्मुः कुरुश्रेष्ठ निर्घातभयशङ्किताः
भूम्यां निपतितौ तत्र ददृशुर्यमलार्जुनौ
बभ्रमुस्तदविज्ञाय लक्ष्यं पतनकारणम्‌
उलूखलं विकर्षन्तं दाम्ना बद्धं च बालकम्‌
कस्येदं कुत आश्चर्यमुत्पात इति कातराः
बाला ऊचुरनेनेति तिर्यग्गतमुलूखलम्‌
विकर्षता मध्यगेन पुरुषावप्यचक्ष्महि
न ते तदुक्तं जगृहुर्न घटेतेति तस्य तत्‌
बालस्योत्पाटनं तर्वोः केचित्सन्दिग्धचेतसः
उलूखलं विकर्षन्तं दाम्ना बद्धं स्वमात्मजम्‌
विलोक्य नन्दः प्रहसद्‌ वदनो विमुमोच ह
गोपीभिः स्तोभितोऽनृत्यद्भगवान्बालवत्क्वचित्‌
उद्गायति क्वचिन्मुग्धस्तद्वशो दारुयन्त्रवत्‌
बिभर्ति क्वचिदाज्ञप्तः पीठकोन्मानपादुकम्‌
बाहुक्षेपं च कुरुते स्वानां च प्रीतिमावहन्‌
दर्शयंस्तद्विदां लोक आत्मनो भृत्यवश्यताम्‌
व्रजस्योवाह वै हर्षं भगवान्बालचेष्टितैः
क्रीणीहि भोः फलानीति श्रुत्वा सत्वरमच्युतः
फलार्थी धान्यमादाय ययौ सर्वफलप्रदः
फलविक्रयिणी तस्य च्युतधान्यकरद्वयम्‌
फलैरपूरयद्रत्नैः फलभाण्डमपूरि च
सरित्तीरगतं कृष्णं भग्नार्जुनमथाह्वयत्‌
रामं च रोहिणी देवी क्रीडन्तं बालकैर्भृशम्‌
नोपेयातां यदाहूतौ क्रीडासङ्गेन पुत्रकौ
यशोदां प्रेषयामास रोहिणी पुत्रवत्सलाम्‌
क्रीडन्तं सा सुतं बालैरतिवेलं सहाग्रजम्‌
यशोदाजोहवीत्कृष्णं पुत्रस्नेहस्नुतस्तनी
कृष्ण कृष्णारविन्दाक्ष तात एहि स्तनं पिब
अलं विहारैः क्षुत्क्षान्तः क्रीडाश्रान्तोऽसि पुत्रक
हे रामागच्छ ताताशु सानुजः कुलनन्दन
प्रातरेव कृताहारस्तद्भवान्भोक्तुमर्हति
प्रतीक्षते त्वां दाशार्ह भोक्ष्यमाणो व्रजाधिपः
एह्यावयोः प्रियं धेहि स्वगृहान्यात बालकाः
धूलिधूसरिताङ्गस्त्वं पुत्र मज्जनमावह
जन्मर्क्षं तेऽद्य भवति विप्रेभ्यो देहि गाः शुचिः
पश्य पश्य वयस्यांस्ते मातृमृष्टान्स्वलङ्कृतान्‌
त्वं च स्नातः कृताहारो विहरस्व स्वलङ्कृतः
इत्थं यशोदा तमशेषशेखरं मत्वा सुतं स्नेहनिबद्धधीर्नृप
हस्ते गृहीत्वा सहराममच्युतं नीत्वा स्ववाटं कृतवत्यथोदयम्‌
श्रीशुक उवाच
गोपवृद्धा महोत्पाताननुभूय बृहद्वने
नन्दादयः समागम्य व्रजकार्यममन्त्रयन्‌
तत्रोपानन्दनामाह गोपो ज्ञानवयोऽधिकः
देशकालार्थतत्त्वज्ञः प्रियकृद्रामकृष्णयोः
उत्थातव्यमितोऽस्माभिर्गोकुलस्य हितैषिभिः
आयान्त्यत्र महोत्पाता बालानां नाशहेतवः
मुक्तः कथञ्चिद्राक्षस्या बालघ्न्या बालको ह्यसौ
हरेरनुग्रहान्नूनमनश्चोपरि नापतत्‌
चक्रवातेन नीतोऽयं दैत्येन विपदं वियत्‌
शिलायां पतितस्तत्र परित्रातः सुरेश्वरैः
यन्न म्रियेत द्रुमयोरन्तरं प्राप्य बालकः
असावन्यतमो वापि तदप्यच्युतरक्षणम्‌
यावदौत्पातिकोऽरिष्टो व्रजं नाभिभवेदितः
तावद्बालानुपादाय यास्यामोऽन्यत्र सानुगाः
वनं वृन्दावनं नाम पशव्यं नवकाननम्‌
गोपगोपीगवां सेव्यं पुण्याद्रितृणवीरुधम्‌
तत्तत्राद्यैव यास्यामः शकटान्युङ्क्त मा चिरम्‌
गोधनान्यग्रतो यान्तु भवतां यदि रोचते
तच्छ्रुत्वैकधियो गोपाः साधु साध्विति वादिनः
व्रजान्स्वान्स्वान्समायुज्य ययू रूढपरिच्छदाः
वृद्धान्बालान्स्त्रियो राजन्सर्वोपकरणानि च
अनःस्वारोप्य गोपाला यत्ता आत्तशरासनाः
गोधनानि पुरस्कृत्य शृङ्गाण्यापूर्य सर्वतः
तूर्यघोषेण महता ययुः सहपुरोहिताः
गोप्यो रूढरथा नूत्न कुचकुङ्कुमकान्तयः
कृष्णलीला जगुः प्रीत्या निष्ककण्ठ्यः सुवाससः
तथा यशोदारोहिण्यावेकं शकटमास्थिते
रेजतुः कृष्णरामाभ्यां तत्कथाश्रवणोत्सुके
वृन्दावनं सम्प्रविश्य सर्वकालसुखावहम्‌
तत्र चक्रुर्व्रजावासं शकटैरर्धचन्द्रवत्‌
वृन्दावनं गोवर्धनं यमुनापुलिनानि च
वीक्ष्यासीदुत्तमा प्रीती राममाधवयोर्नृप
एवं व्रजौकसां प्रीतिं यच्छन्तौ बालचेष्टितैः
कलवाक्यैः स्वकालेन वत्सपालौ बभूवतुः
अविदूरे व्रजभुवः सह गोपालदारकैः
चारयामासतुर्वत्सान्नानाक्रीडापरिच्छदौ
क्वचिद्वादयतो वेणुं क्षेपणैः क्षिपतः क्वचित्‌
क्वचित्पादैः किङ्किणीभिः क्वचित्कृत्रिमगोवृषैः
वृषायमाणौ नर्दन्तौ युयुधाते परस्परम्‌
अनुकृत्य रुतैर्जन्तूंश्चेरतुः प्राकृतौ यथा
कदाचिद्यमुनातीरे वत्सांश्चारयतोः स्वकैः
वयस्यैः कृष्णबलयोर्जिघांसुर्दैत्य आगमत्‌
तं वत्सरूपिणं वीक्ष्य वत्सयूथगतं हरिः
दर्शयन्बलदेवाय शनैर्मुग्ध इवासदत्‌
गृहीत्वापरपादाभ्यां सहलाङ्गूलमच्युतः
भ्रामयित्वा कपित्थाग्रे प्राहिणोद्गतजीवितम्‌
स कपित्थैर्महाकायः पात्यमानैः पपात ह
तं वीक्ष्य विस्मिता बालाः शशंसुः साधु साध्विति
देवाश्च परिसन्तुष्टा बभूवुः पुष्पवर्षिणः
तौ वत्सपालकौ भूत्वा सर्वलोकैकपालकौ
सप्रातराशौ गोवत्सांश्चारयन्तौ विचेरतुः
स्वं स्वं वत्सकुलं सर्वे पाययिष्यन्त एकदा
गत्वा जलाशयाभ्याशं पाययित्वा पपुर्जलम्‌
ते तत्र ददृशुर्बाला महासत्त्वमवस्थितम्‌
तत्रसुर्वज्रनिर्भिन्नं गिरेः शृङ्गमिव च्युतम्‌
स वै बको नाम महानसुरो बकरूपधृक्‌
आगत्य सहसा कृष्णं तीक्ष्णतुण्डोऽग्रसद्बली
कृष्णं महाबकग्रस्तं दृष्ट्वा रामादयोऽर्भकाः
बभूवुरिन्द्रियाणीव विना प्राणं विचेतसः
तं तालुमूलं प्रदहन्तमग्निवद्गोपालसूनुं पितरं जगद्गुरोः
चच्छर्द सद्योऽतिरुषाक्षतं बकस्तुण्डेन हन्तुं पुनरभ्यपद्यत
तमापतन्तं स निगृह्य तुण्डयोर्दोर्भ्यां बकं कंससखं सतां पतिः
पश्यत्सु बालेषु ददार लीलया मुदावहो वीरणवद्दिवौकसाम्‌
तदा बकारिं सुरलोकवासिनः समाकिरन्नन्दनमल्लिकादिभिः
समीडिरे चानकशङ्खसंस्तवैस्तद्वीक्ष्य गोपालसुता विसिस्मिरे
मुक्तं बकास्यादुपलभ्य बालका रामादयः प्राणमिवेन्द्रियो गणः
स्थानागतं तं परिरभ्य निर्वृताः प्रणीय वत्सान्व्रजमेत्य तज्जगुः
श्रुत्वा तद्विस्मिता गोपा गोप्यश्चातिप्रियादृताः
प्रेत्यागतमिवोत्सुक्यादैक्षन्त तृषितेक्षणाः
अहो बतास्य बालस्य बहवो मृत्यवोऽभवन्‌
अप्यासीद्विप्रियं तेषां कृतं पूर्वं यतो भयम्‌
अथाप्यभिभवन्त्येनं नैव ते घोरदर्शनाः
जिघांसयैनमासाद्य नश्यन्त्यग्नौ पतङ्गवत्‌
अहो ब्रह्मविदां वाचो नासत्याः सन्ति कर्हिचित्‌
गर्गो यदाह भगवानन्वभावि तथैव तत्‌
इति नन्दादयो गोपाः कृष्णरामकथां मुदा
कुर्वन्तो रममाणाश्च नाविन्दन्भववेदनाम्‌
एवं विहारैः कौमारैः कौमारं जहतुर्व्रजे
निलायनैः सेतुबन्धैर्मर्कटोत्प्लवनादिभिः
श्रीशुक उवाच
क्वचिद्वनाशाय मनो दधद्व्रजात्प्रातः समुत्थाय वयस्यवत्सपान्‌
प्रबोधयञ्छृङ्गरवेण चारुणा विनिर्गतो वत्सपुरःसरो हरिः
तेनैव साकं पृथुकाः सहस्रशः स्निग्धाः सुशिग्वेत्रविषाणवेणवः
स्वान्स्वान्सहस्रोपरिसङ्ख्ययान्वितान्वत्सान्पुरस्कृत्य विनिर्ययुर्मुदा
कृष्णवत्सैरसङ्ख्यातैर्यूथीकृत्य स्ववत्सकान्‌
चारयन्तोऽर्भलीलाभिर्विजह्रुस्तत्र तत्र ह
फलप्रबालस्तवक सुमनःपिच्छधातुभिः
काचगुञ्जामणिस्वर्ण भूषिता अप्यभूषयन्‌
मुष्णन्तोऽन्योन्यशिक्यादीन्ज्ञातानाराच्च चिक्षिपुः
तत्रत्याश्च पुनर्दूराद्धसन्तश्च पुनर्ददुः
यदि दूरं गतः कृष्णो वनशोभेक्षणाय तम्‌
अहं पूर्वमहं पूर्वमिति संस्पृश्य रेमिरे
केचिद्वेणून्वादयन्तो ध्मान्तः शृङ्गाणि केचन
केचिद्भृङ्गैः प्रगायन्तः कूजन्तः कोकिलैः परे
विच्छायाभिः प्रधावन्तो गच्छन्तः साधुहंसकैः
बकैरुपविशन्तश्च नृत्यन्तश्च कलापिभिः
विकर्षन्तः कीशबालानारोहन्तश्च तैर्द्रुमान्‌
विकुर्वन्तश्च तैः साकं प्लवन्तश्च पलाशिषु
साकं भेकैर्विलङ्घन्तः सरितः स्रवसम्प्लुताः
विहसन्तः प्रतिच्छायाः शपन्तश्च प्रतिस्वनान्‌
इत्थं सतां ब्रह्मसुखानुभूत्या दास्यं गतानां परदैवतेन
मायाश्रितानां नरदारकेण साकं विजह्रुः कृतपुण्यपुञ्जाः
यत्पादपांसुर्बहुजन्मकृच्छ्रतो
धृतात्मभिर्योगिभिरप्यलभ्यः
स एव यद्दृग्विषयः स्वयं स्थितः
किं वर्ण्यते दिष्टमतो व्रजौकसाम्‌
अथाघनामाभ्यपतन्महासुरस्तेषां सुखक्रीडनवीक्षणाक्षमः
नित्यं यदन्तर्निजजीवितेप्सुभिः पीतामृतैरप्यमरैः प्रतीक्ष्यते
दृष्ट्वार्भकान्कृष्णमुखानघासुरः
कंसानुशिष्टः स बकीबकानुजः
अयं तु मे सोदरनाशकृत्तयोर्‌
द्वयोर्ममैनं सबलं हनिष्ये
एते यदा मत्सुहृदोस्तिलापः कृतास्तदा नष्टसमा व्रजौकसः
प्राणे गते वर्ष्मसु का नु चिन्ता प्रजासवः प्राणभृतो हि ये ते
इति व्यवस्याजगरं बृहद्वपुः स योजनायाममहाद्रिपीवरम्‌
धृत्वाद्भुतं व्यात्तगुहाननं तदा पथि व्यशेत ग्रसनाशया खलः
धराधरोष्ठो जलदोत्तरोष्ठो दर्याननान्तो गिरिशृङ्गदंष्ट्रः
ध्वान्तान्तरास्यो वितताध्वजिह्वः परुषानिलश्वासदवेक्षणोष्णः
दृष्ट्वा तं तादृशं सर्वे मत्वा वृन्दावनश्रियम्‌
व्यात्ताजगरतुण्डेन ह्युत्प्रेक्षन्ते स्म लीलया
अहो मित्राणि गदत सत्त्वकूटं पुरः स्थितम्‌
अस्मत्सङ्ग्रसनव्यात्त व्यालतुण्डायते न वा
सत्यमर्ककरारक्तमुत्तराहनुवद्घनम्‌
अधराहनुवद्रोधस्तत्प्रतिच्छाययारुणम्‌
प्रतिस्पर्धेते सृक्कभ्यां सव्यासव्ये नगोदरे
तुङ्गशृङ्गालयोऽप्येतास्तद्दंष्ट्राभिश्च पश्यत
आस्तृतायाममार्गोऽयं रसनां प्रतिगर्जति
एषां अन्तर्गतं ध्वान्तमेतदप्यन्तराननम्‌
दावोष्णखरवातोऽयं श्वासवद्भाति पश्यत
तद्दग्धसत्त्वदुर्गन्धोऽप्यन्तरामिषगन्धवत्‌
अस्मान्किमत्र ग्रसिता निविष्टानयं तथा चेद्बकवद्विनङ्क्ष्यति
क्षणादनेनेति बकार्युशन्मुखं वीक्ष्योद्धसन्तः करताडनैर्ययुः
इत्थं मिथोऽतथ्यमतज्ज्ञभाषितं
श्रुत्वा विचिन्त्येत्यमृषा मृषायते
रक्षो विदित्वाखिलभूतहृत्स्थितः
स्वानां निरोद्धुं भगवान्मनो दधे
तावत्प्रविष्टास्त्वसुरोदरान्तरं परं न गीर्णाः शिशवः सवत्साः
प्रतीक्षमाणेन बकारिवेशनं हतस्वकान्तस्मरणेन रक्षसा
तान्वीक्ष्य कृष्णः सकलाभयप्रदो
ह्यनन्यनाथान्स्वकरादवच्युतान्‌
दीनांश्च मृत्योर्जठराग्निघासान्‌
घृणार्दितो दिष्टकृतेन विस्मितः
कृत्यं किमत्रास्य खलस्य जीवनं
न वा अमीषां च सतां विहिंसनम्‌
द्वयं कथं स्यादिति संविचिन्त्य
ज्ञात्वाविशत्तुण्डमशेषदृग्घरिः
तदा घनच्छदा देवा भयाद्धाहेति चुक्रुशुः
जहृषुर्ये च कंसाद्याः कौणपास्त्वघबान्धवाः
तच्छ्रुत्वा भगवान्कृष्णस्त्वव्ययः सार्भवत्सकम्‌
चूर्णीचिकीर्षोरात्मानं तरसा ववृधे गले
ततोऽतिकायस्य निरुद्धमार्गिणो ह्युद्गीर्णदृष्टेर्भ्रमतस्त्वितस्ततः
पूर्णोऽन्तरङ्गे पवनो निरुद्धो मूर्धन्विनिर्भिद्य विनिर्गतो बहिः
तेनैव सर्वेषु बहिर्गतेषु प्राणेषु वत्सान्सुहृदः परेतान्‌
दृष्ट्या स्वयोत्थाप्य तदन्वितः पुनर्वक्त्रान्मुकुन्दो भगवान्विनिर्ययौ
पीनाहिभोगोत्थितमद्भुतं महज्ज्योतिः स्वधाम्ना ज्वलयद्दिशो दश
प्रतीक्ष्य खेऽवस्थितमीशनिर्गमं विवेश तस्मिन्मिषतां दिवौकसाम्‌
ततोऽतिहृष्टाः स्वकृतोऽकृतार्हणं
पुष्पैः सुगा अप्सरसश्च नर्तनैः
गीतैः सुरा वाद्यधराश्च वाद्यकैः
स्तवैश्च विप्रा जयनिःस्वनैर्गणाः
तदद्भुतस्तोत्रसुवाद्यगीतिका जयादिनैकोत्सवमङ्गलस्वनान्‌
श्रुत्वा स्वधाम्नोऽन्त्यज आगतोऽचिराद्दृष्ट्वा महीशस्य जगाम विस्मयम्‌
राजन्नाजगरं चर्म शुष्कं वृन्दावनेऽद्भुतम्‌
व्रजौकसां बहुतिथं बभूवाक्रीडगह्वरम्‌
एतत्कौमारजं कर्म हरेरात्माहिमोक्षणम्‌
मृत्योः पौगण्डके बाला दृष्ट्वोचुर्विस्मिता व्रजे
नैतद्विचित्रं मनुजार्भमायिनः परावराणां परमस्य वेधसः
अघोऽपि यत्स्पर्शनधौतपातकः प्रापात्मसाम्यं त्वसतां सुदुर्लभम्‌
सकृद्यदङ्गप्रतिमान्तराहिता मनोमयी भागवतीं ददौ गतिम्‌
स एव नित्यात्मसुखानुभूत्यभि व्युदस्तमायोऽन्तर्गतो हि किं पुनः
श्रीसूत उवाच
इत्थं द्विजा यादवदेवदत्तः श्रुत्वा स्वरातुश्चरितं विचित्रम्‌
पप्रच्छ भूयोऽपि तदेव पुण्यं वैयासकिं यन्निगृहीतचेताः
श्रीराजोवाच
ब्रह्मन्कालान्तरकृतं तत्कालीनं कथं भवेत्‌
यत्कौमारे हरिकृतं जगुः पौगण्डकेऽर्भकाः
तद्ब्रूहि मे महायोगिन्परं कौतूहलं गुरो
नूनमेतद्धरेरेव माया भवति नान्यथा
वयं धन्यतमा लोके गुरोऽपि क्षत्रबन्धवः
वयं पिबामो मुहुस्त्वत्तः पुण्यं कृष्णकथामृतम्‌
श्रीसूत उवाच
इत्थं स्म पृष्टः स तु बादरायणिस्‌
तत्स्मारितानन्तहृताखिलेन्द्रियः
कृच्छ्रात्पुनर्लब्धबहिर्दृशिः शनैः
प्रत्याह तं भागवतोत्तमोत्तम
श्रीशुक उवाच
साधु पृष्टं महाभाग त्वया भागवतोत्तम
यन्नूतनयसीशस्य शृण्वन्नपि कथां मुहुः
सतामयं सारभृतां निसर्गो यदर्थवाणीश्रुतिचेतसामपि
प्रतिक्षणं नव्यवदच्युतस्य यत्स्त्रिया विटानामिव साधु वार्ता
शृणुष्वावहितो राजन्नपि गुह्यं वदामि ते
ब्रूयुः स्निग्धस्य शिष्यस्य गुरवो गुह्यमप्युत
तथाघवदनान्मृत्यो रक्षित्वा वत्सपालकान्‌
सरित्पुलिनमानीय भगवानिदमब्रवीत्‌
अहोऽतिरम्यं पुलिनं वयस्याः स्वकेलिसम्पन्मृदुलाच्छबालुकम्‌
स्फुटत्सरोगन्धहृतालिपत्रिक ध्वनिप्रतिध्वानलसद्द्रुमाकुलम्‌
अत्र भोक्तव्यमस्माभिर्दिवारूढं क्षुधार्दिताः
वत्साः समीपेऽपः पीत्वा चरन्तु शनकैस्तृणम्‌
तथेति पाययित्वार्भा वत्सानारुध्य शाद्वले
मुक्त्वा शिक्यानि बुभुजुः समं भगवता मुदा
कृष्णस्य विष्वक्पुरुराजिमण्डलैर्‌
अभ्याननाः फुल्लदृशो व्रजार्भकाः
सहोपविष्टा विपिने विरेजुश्‌
छदा यथाम्भोरुहकर्णिकायाः
केचित्पुष्पैर्दलैः केचित्पल्लवैरङ्कुरैः फलैः
शिग्भिस्त्वग्भिर्दृषद्भिश्च बुभुजुः कृतभाजनाः
सर्वे मिथो दर्शयन्तः स्वस्वभोज्यरुचिं पृथक्‌
हसन्तो हासयन्तश्चा भ्यवजह्रुः सहेश्वराः
बिभ्रद्वेणुं जठरपटयोः शृङ्गवेत्रे च कक्षे
वामे पाणौ मसृणकवलं तत्फलान्यङ्गुलीषु
तिष्ठन्मध्ये स्वपरिसुहृदो हासयन्नर्मभिः स्वैः
स्वर्गे लोके मिषति बुभुजे यज्ञभुग्बालकेलिः
भारतैवं वत्सपेषु भुञ्जानेष्वच्युतात्मसु
वत्सास्त्वन्तर्वने दूरं विविशुस्तृणलोभिताः
तान्दृष्ट्वा भयसन्त्रस्तानूचे कृष्णोऽस्य भीभयम्‌
मित्राण्याशान्मा विरमते हानेष्ये वत्सकानहम्‌
इत्युक्त्वाद्रिदरीकुञ्ज गह्वरेष्वात्मवत्सकान्‌
विचिन्वन्भगवान्कृष्णः सपाणिकवलो ययौ
अम्भोजन्मजनिस्तदन्तरगतो मायार्भकस्येशितुर्‌
द्रष्टुं मञ्जु महित्वमन्यदपि तद्वत्सानितो वत्सपान्‌
नीत्वान्यत्र कुरूद्वहान्तरदधात्खेऽवस्थितो यः पुरा
दृष्ट्वाघासुरमोक्षणं प्रभवतः प्राप्तः परं विस्मयम्‌
ततो वत्सानदृष्ट्वैत्य पुलिनेऽपि च वत्सपान्‌
उभावपि वने कृष्णो विचिकाय समन्ततः
क्वाप्यदृष्ट्वान्तर्विपिने वत्सान्पालांश्च विश्ववित्‌
सर्वं विधिकृतं कृष्णः सहसावजगाम ह
ततः कृष्णो मुदं कर्तुं तन्मात्ऱ्‌^ईणां च कस्य च
उभयायितमात्मानं चक्रे विश्वकृदीश्वरः
यावद्वत्सपवत्सकाल्पकवपुर्यावत्कराङ्घ्र्यादिकं
यावद्यष्टिविषाणवेणुदलशिग्यावद्विभूषाम्बरम्‌
यावच्छीलगुणाभिधाकृतिवयो यावद्विहारादिकं
सर्वं विष्णुमयं गिरोऽङ्गवदजः सर्वस्वरूपो बभौ
स्वयमात्मात्मगोवत्सान्प्रतिवार्यात्मवत्सपैः
क्रीडन्नात्मविहारैश्च सर्वात्मा प्राविशद्व्रजम्‌
तत्तद्वत्सान्पृथङ्नीत्वा तत्तद्गोष्ठे निवेश्य सः
तत्तदात्माभवद्राजंस्तत्तत्सद्म प्रविष्टवान्‌
तन्मातरो वेणुरवत्वरोत्थिता उत्थाप्य दोर्भिः परिरभ्य निर्भरम्‌
स्नेहस्नुतस्तन्यपयःसुधासवं मत्वा परं ब्रह्म सुतानपाययन्‌
ततो नृपोन्मर्दनमज्जलेपना लङ्काररक्षातिलकाशनादिभिः
संलालितः स्वाचरितैः प्रहर्षयन्सायं गतो यामयमेन माधवः
गावस्ततो गोष्ठमुपेत्य सत्वरं हुङ्कारघोषैः परिहूतसङ्गतान्‌
स्वकान्स्वकान्वत्सतरानपाययन्मुहुर्लिहन्त्यः स्रवदौधसं पयः
गोगोपीनां मातृतास्मिन्नासीत्स्नेहर्धिकां विना
पुरोवदास्वपि हरेस्तोकता मायया विना
व्रजौकसां स्वतोकेषु स्नेहवल्ल्याब्दमन्वहम्‌
शनैर्निःसीम ववृधे यथा कृष्णे त्वपूर्ववत्‌
इत्थमात्मात्मनात्मानं वत्सपालमिषेण सः
पालयन्वत्सपो वर्षं चिक्रीडे वनगोष्ठयोः
एकदा चारयन्वत्सान्सरामो वनमाविशत्‌
पञ्चषासु त्रियामासु हायनापूरणीष्वजः
ततो विदूराच्चरतो गावो वत्सानुपव्रजम्‌
गोवर्धनाद्रिशिरसि चरन्त्यो ददृशुस्तृणम्‌
दृष्ट्वाथ तत्स्नेहवशोऽस्मृतात्मा स गोव्रजोऽत्यात्मपदुर्गमार्गः
द्विपात्ककुद्ग्रीव उदास्यपुच्छोऽगाद्धुङ्कृतैरास्रुपया जवेन
समेत्य गावोऽधो वत्सान्वत्सवत्योऽप्यपाययन्‌
गिलन्त्य इव चाङ्गानि लिहन्त्यः स्वौधसं पयः
गोपास्तद्रोधनायास मौघ्यलज्जोरुमन्युना
दुर्गाध्वकृच्छ्रतोऽभ्येत्य गोवत्सैर्ददृशुः सुतान्‌
तदीक्षणोत्प्रेमरसाप्लुताशया जातानुरागा गतमन्यवोऽर्भकान्‌
उदुह्य दोर्भिः परिरभ्य मूर्धनि घ्राणैरवापुः परमां मुदं ते
ततः प्रवयसो गोपास्तोकाश्लेषसुनिर्वृताः
कृच्छ्राच्छनैरपगतास्तदनुस्मृत्युदश्रवः
व्रजस्य रामः प्रेमर्धेर्वीक्ष्यौत्कण्ठ्यमनुक्षणम्‌
मुक्तस्तनेष्वपत्येष्वप्यहेतुविदचिन्तयत्‌
किमेतदद्भुतमिव वासुदेवेऽखिलात्मनि
व्रजस्य सात्मनस्तोकेष्वपूर्वं प्रेम वर्धते
केयं वा कुत आयाता दैवी वा नार्युतासुरी
प्रायो मायास्तु मे भर्तुर्नान्या मेऽपि विमोहिनी
इति सञ्चिन्त्य दाशार्हो वत्सान्सवयसानपि
सर्वानाचष्ट वैकुण्ठं चक्षुषा वयुनेन सः
नैते सुरेशा ऋषयो न चैते त्वमेव भासीश भिदाश्रयेऽपि
सर्वं पृथक्त्वं निगमात्कथं वदेत्युक्तेन वृत्तं प्रभुणा बलोऽवैत्‌
तावदेत्यात्मभूरात्म मानेन त्रुट्यनेहसा
पुरोवदाब्दं क्रीडन्तं ददृशे सकलं हरिम्‌
यावन्तो गोकुले बालाः सवत्साः सर्व एव हि
मायाशये शयाना मे नाद्यापि पुनरुत्थिताः
इत एतेऽत्र कुत्रत्या मन्मायामोहितेतरे
तावन्त एव तत्राब्दं क्रीडन्तो विष्णुना समम्‌
एवमेतेषु भेदेषु चिरं ध्यात्वा स आत्मभूः
सत्याः के कतरे नेति ज्ञातुं नेष्टे कथञ्चन
एवं सम्मोहयन्विष्णुं विमोहं विश्वमोहनम्‌
स्वयैव माययाजोऽपि स्वयमेव विमोहितः
तम्यां तमोवन्नैहारं खद्योतार्चिरिवाहनि
महतीतरमायैश्यं निहन्त्यात्मनि युञ्जतः
तावत्सर्वे वत्सपालाः पश्यतोऽजस्य तत्क्षणात्‌
व्यदृश्यन्त घनश्यामाः पीतकौशेयवाससः
चतुर्भुजाः शङ्खचक्र गदाराजीवपाणयः
किरीटिनः कुण्डलिनो हारिणो वनमालिनः
श्रीवत्साङ्गददोरत्न कम्बुकङ्कणपाणयः
नूपुरैः कटकैर्भाताः कटिसूत्राङ्गुलीयकैः
आङ्घ्रिमस्तकमापूर्णास्तुलसीनवदामभिः
कोमलैः सर्वगात्रेषु भूरिपुण्यवदर्पितैः
चन्द्रिकाविशदस्मेरैः सारुणापाङ्गवीक्षितैः
स्वकार्थानामिव रजः सत्त्वाभ्यां स्रष्टृपालकाः
आत्मादिस्तम्बपर्यन्तैर्मूर्तिमद्भिश्चराचरैः
नृत्यगीताद्यनेकार्हैः पृथक्पृथगुपासिताः
अणिमाद्यैर्महिमभिरजाद्याभिर्विभूतिभिः
चतुर्विंशतिभिस्तत्त्वैः परीता महदादिभिः
कालस्वभावसंस्कार कामकर्मगुणादिभिः
स्वमहिध्वस्तमहिभिर्मूर्तिमद्भिरुपासिताः
सत्यज्ञानानन्तानन्द मात्रैकरसमूर्तयः
अस्पृष्टभूरिमाहात्म्या अपि ह्युपनिषद्दृशाम्‌
एवं सकृद्ददर्शाजः परब्रह्मात्मनोऽखिलान्‌
यस्य भासा सर्वमिदं विभाति सचराचरम्‌
ततोऽतिकुतुकोद्वृत्य स्तिमितैकादशेन्द्रियः
तद्धाम्नाभूदजस्तूष्णीं पूर्देव्यन्तीव पुत्रिका
इतीरेशेऽतर्क्ये निजमहिमनि स्वप्रमितिके
परत्राजातोऽतन्निरसनमुखब्रह्मकमितौ
अनीशेऽपि द्रष्टुं किमिदमिति वा मुह्यति सति
चच्छादाजो ज्ञात्वा सपदि परमोऽजाजवनिकाम्‌
ततोऽर्वाक्प्रतिलब्धाक्षः कः परेतवदुत्थितः
कृच्छ्रादुन्मील्य वै दृष्टीराचष्टेदं सहात्मना
सपद्येवाभितः पश्यन्दिशोऽपश्यत्पुरःस्थितम्‌
वृन्दावनं जनाजीव्य द्रुमाकीर्णं समाप्रियम्‌
यत्र नैसर्गदुर्वैराः सहासन्नृमृगादयः
मित्राणीवाजितावास द्रुतरुट्तर्षकादिकम्‌
तत्रोद्वहत्पशुपवंशशिशुत्वनाट्यं
ब्रह्माद्वयं परमनन्तमगाधबोधम्‌
वत्सान्सखीनिव पुरा परितो विचिन्वद्‌
एकं सपाणिकवलं परमेष्ठ्यचष्ट
दृष्ट्वा त्वरेण निजधोरणतोऽवतीर्य
पृथ्व्यां वपुः कनकदण्डमिवाभिपात्य
स्पृष्ट्वा चतुर्मुकुटकोटिभिरङ्घ्रियुग्मं
नत्वा मुदश्रुसुजलैरकृताभिषेकम्‌
उत्थायोत्थाय कृष्णस्य चिरस्य पादयोः पतन्‌
आस्ते महित्वं प्राग्दृष्टं स्मृत्वा स्मृत्वा पुनः पुनः
शनैरथोत्थाय विमृज्य लोचने मुकुन्दमुद्वीक्ष्य विनम्रकन्धरः
कृताञ्जलिः प्रश्रयवान्समाहितः सवेपथुर्गद्गदयैलतेलया
श्रीब्रह्मोवाच
नौमीड्य तेऽभ्रवपुषे तडिदम्बराय
गुञ्जावतंसपरिपिच्छलसन्मुखाय
वन्यस्रजे कवलवेत्रविषाणवेणु
लक्ष्मश्रिये मृदुपदे पशुपाङ्गजाय
अस्यापि देव वपुषो मदनुग्रहस्य स्वेच्छामयस्य न तु भूतमयस्य कोऽपि
नेशे महि त्ववसितुं मनसान्तरेण साक्षात्तवैव किमुतात्मसुखानुभूतेः
ज्ञाने प्रयासमुदपास्य नमन्त एव
जीवन्ति सन्मुखरितां भवदीयवार्ताम्‌
स्थाने स्थिताः श्रुतिगतां तनुवाङ्मनोभिर्‌
ये प्रायशोऽजित जितोऽप्यसि तैस्त्रिलोक्याम्‌
श्रेयःसृतिं भक्तिमुदस्य ते विभो
क्लिश्यन्ति ये केवलबोधलब्धये
तेषामसौ क्लेशल एव शिष्यते
नान्यद्यथा स्थूलतुषावघातिनाम्‌
पुरेह भूमन्बहवोऽपि योगिनस्त्वदर्पितेहा निजकर्मलब्धया
विबुध्य भक्त्यैव कथोपनीतया प्रपेदिरेऽञ्जोऽच्युत ते गतिं पराम्‌
तथापि भूमन्महिमागुणस्य ते विबोद्धुमर्हत्यमलान्तरात्मभिः
अविक्रियात्स्वानुभवादरूपतो ह्यनन्यबोध्यात्मतया न चान्यथा
गुणात्मनस्तेऽपि गुणान्विमातुं हितावतीऋनस्य क ईशिरेऽस्य
कालेन यैर्वा विमिताः सुकल्पैर्भूपांशवः खे मिहिका द्युभासः
तत्तेऽनुकम्पां सुसमीक्षमाणो भुञ्जान एवात्मकृतं विपाकम्‌
हृद्वाग्वपुर्भिर्विदधन्नमस्ते जीवेत यो मुक्तिपदे स दायभाक्‌
पश्येश मेऽनार्यमनन्त आद्ये परात्मनि त्वय्यपि मायिमायिनि
मायां वितत्येक्षितुमात्मवैभवं ह्यहं कियानैच्छमिवार्चिरग्नौ
अतः क्षमस्वाच्युत मे रजोभुवो ह्यजानतस्त्वत्पृथगीशमानिनः
अजावलेपान्धतमोऽन्धचक्षुष एषोऽनुकम्प्यो मयि नाथवानिति
क्वाहं तमोमहदहंखचराग्निवार्भू
संवेष्टिताण्डघटसप्तवितस्तिकायः
क्वेदृग्विधाविगणिताण्डपराणुचर्या
वाताध्वरोमविवरस्य च ते महित्वम्‌
उत्क्षेपणं गर्भगतस्य पादयोः किं कल्पते मातुरधोक्षजागसे
किमस्तिनास्तिव्यपदेशभूषितं तवास्ति कुक्षेः कियदप्यनन्तः
जगत्त्रयान्तोदधिसम्प्लवोदे नारायणस्योदरनाभिनालात्‌
विनिर्गतोऽजस्त्विति वाङ्न वै मृषा किन्त्वीश्वर त्वन्न विनिर्गतोऽस्मि
नारायणस्त्वं न हि सर्वदेहिनामात्मास्यधीशाखिललोकसाक्षी
नारायणोऽङ्गं नरभूजलायनात्तच्चापि सत्यं न तवैव माया
तच्चेज्जलस्थं तव सज्जगद्वपुः
किं मे न दृष्टं भगवंस्तदैव
किं वा सुदृष्टं हृदि मे तदैव
किं नो सपद्येव पुनर्व्यदर्शि
अत्रैव मायाधमनावतारे ह्यस्य प्रपञ्चस्य बहिः स्फुटस्य
कृत्स्नस्य चान्तर्जठरे जनन्या मायात्वमेव प्रकटीकृतं ते
यस्य कुक्षाविदं सर्वं सात्मं भाति यथा तथा
तत्त्वय्यपीह तत्सर्वं किमिदं मायया विना
अद्यैव त्वदृतेऽस्य किं मम न ते मायात्वमादर्शितम्‌
एकोऽसि प्रथमं ततो व्रजसुहृद्वत्साः समस्ता अपि
तावन्तोऽसि चतुर्भुजास्तदखिलैः साकं मयोपासितास्‌
तावन्त्येव जगन्त्यभूस्तदमितं ब्रह्माद्वयं शिष्यते
अजानतां त्वत्पदवीमनात्मन्यात्मात्मना भासि वितत्य मायाम्‌
सृष्टाविवाहं जगतो विधान इव त्वमेषोऽन्त इव त्रिनेत्रः
सुरेष्वृषिष्वीश तथैव नृष्वपि तिर्यक्षु यादःस्वपि तेऽजनस्य
जन्मासतां दुर्मदनिग्रहाय प्रभो विधातः सदनुग्रहाय च
को वेत्ति भूमन्भगवन्परात्मन्योगेश्वरोतीर्भवतस्त्रिलोक्याम्‌
क्व वा कथं वा कति वा कदेति विस्तारयन्क्रीडसि योगमायाम्‌
तस्मादिदं जगदशेषमसत्स्वरूपं
स्वप्नाभमस्तधिषणं पुरुदुःखदुःखम्‌
त्वय्येव नित्यसुखबोधतनावनन्ते
मायात उद्यदपि यत्सदिवावभाति
एकस्त्वमात्मा पुरुषः पुराणः सत्यः स्वयंज्योतिरनन्त आद्यः
नित्योऽक्षरोऽजस्रसुखो निरञ्जनः पूर्णाद्वयो मुक्त उपाधितोऽमृतः
एवंविधं त्वां सकलात्मनामपि स्वात्मानमात्मात्मतया विचक्षते
गुर्वर्कलब्धोपनिषत्सुचक्षुषा ये ते तरन्तीव भवानृताम्बुधिम्‌
आत्मानमेवात्मतयाविजानतां तेनैव जातं निखिलं प्रपञ्चितम्‌
ज्ञानेन भूयोऽपि च तत्प्रलीयते रज्ज्वामहेर्भोगभवाभवौ यथा
अज्ञानसंज्ञौ भवबन्धमोक्षौ द्वौ नाम नान्यौ स्त ऋतज्ञभावात्‌
अजस्रचित्यात्मनि केवले परे विचार्यमाणे तरणाविवाहनी
त्वामात्मानं परं मत्वा परमात्मानमेव च
आत्मा पुनर्बहिर्मृग्य अहोऽज्ञजनताज्ञता
अन्तर्भवेऽनन्त भवन्तमेव ह्यतत्त्यजन्तो मृगयन्ति सन्तः
असन्तमप्यन्त्यहिमन्तरेण सन्तं गुणं तं किमु यन्ति सन्तः
अथापि ते देव पदाम्बुजद्वय प्रसादलेशानुगृहीत एव हि
जानाति तत्त्वं भगवन्महिम्नो न चान्य एकोऽपि चिरं विचिन्वन्‌
तदस्तु मे नाथ स भूरिभागो भवेऽत्र वान्यत्र तु वा तिरश्चाम्‌
येनाहमेकोऽपि भवज्जनानां भूत्वा निषेवे तव पादपल्लवम्‌
अहोऽतिधन्या व्रजगोरमण्यः स्तन्यामृतं पीतमतीव ते मुदा
यासां विभो वत्सतरात्मजात्मना यत्तृप्तयेऽद्यापि न चालमध्वराः
अहो भाग्यमहो भाग्यं नन्दगोपव्रजौकसाम्‌
यन्मित्रं परमानन्दं पूर्णं ब्रह्म सनातनम्‌
एषां तु भाग्यमहिमाच्युत तावदास्ताम्‌
एकादशैव हि वयं बत भूरिभागाः
एतद्धृषीकचषकैरसकृत्पिबामः
शर्वादयोऽङ्घ्र्युदजमध्वमृतासवं ते
तद्भूरिभाग्यमिह जन्म किमप्यटव्यां
यद्गोकुलेऽपि कतमाङ्घ्रिरजोऽभिषेकम्‌
यज्जीवितं तु निखिलं भगवान्मुकुन्दस्‌
त्वद्यापि यत्पदरजः श्रुतिमृग्यमेव
एषां घोषनिवासिनामुत भवान्किं देव रातेति नश्‌
चेतो विश्वफलात्फलं त्वदपरं कुत्राप्ययन्मुह्यति
सद्वेषादिव पूतनापि सकुला त्वामेव देवापिता
यद्धामार्थसुहृत्प्रियात्मतनयप्राणाशयास्त्वत्कृते
तावद्रागादयः स्तेनास्तावत्कारागृहं गृहम्‌
तावन्मोहोऽङ्घ्रिनिगडो यावत्कृष्ण न ते जनाः
प्रपञ्चं निष्प्रपञ्चोऽपि विडम्बयसि भूतले
प्रपन्नजनतानन्द सन्दोहं प्रथितुं प्रभो
जानन्त एव जानन्तु किं बहूक्त्या न मे प्रभो
मनसो वपुषो वाचो वैभवं तव गोचरः
अनुजानीहि मां कृष्ण सर्वं त्वं वेत्सि सर्वदृक्‌
त्वमेव जगतां नाथो जगदेतत्तवार्पितम्‌
श्रीकृष्ण वृष्णिकुलपुष्करजोषदायिन्‌
क्ष्मानिर्जरद्विजपशूदधिवृद्धिकारिन्‌
उद्धर्मशार्वरहर क्षितिराक्षसध्रुग्‌
आकल्पमार्कमर्हन्भगवन्नमस्ते
श्रीशुक उवाच
इत्यभिष्टूय भूमानं त्रिः परिक्रम्य पादयोः
नत्वाभीष्टं जगद्धाता स्वधाम प्रत्यपद्यत
ततोऽनुज्ञाप्य भगवान्स्वभुवं प्रागवस्थितान्‌
वत्सान्पुलिनमानिन्ये यथापूर्वसखं स्वकम्‌
एकस्मिन्नपि यातेऽब्दे प्राणेशं चान्तरात्मनः
कृष्णमायाहता राजन्क्षणार्धं मेनिरेऽर्भकाः
किं किं न विस्मरन्तीह मायामोहितचेतसः
यन्मोहितं जगत्सर्वमभीक्ष्णं विस्मृतात्मकम्‌
ऊचुश्च सुहृदः कृष्णं स्वागतं तेऽतिरंहसा
नैकोऽप्यभोजि कवल एहीतः साधु भुज्यताम्‌
ततो हसन्हृषीकेशोऽभ्यवहृत्य सहार्भकैः
दर्शयंश्चर्माजगरं न्यवर्तत वनाद्व्रजम्‌
बर्हप्रसूनवनधातुविचित्रिताङ्गः
प्रोद्दामवेणुदलशृङ्गरवोत्सवाढ्यः
वत्सान्गृणन्ननुगगीतपवित्रकीर्तिर्‌
गोपीदृगुत्सवदृशिः प्रविवेश गोष्ठम्‌
अद्यानेन महाव्यालो यशोदानन्दसूनुना
हतोऽविता वयं चास्मादिति बाला व्रजे जगुः
श्रीराजोवाच
ब्रह्मन्परोद्भवे कृष्णे इयान्प्रेमा कथं भवेत्‌
योऽभूतपूर्वस्तोकेषु स्वोद्भवेष्वपि कथ्यताम्‌
श्रीशुक उवाच
सर्वेषामपि भूतानां नृप स्वात्मैव वल्लभः
इतरेऽपत्यवित्ताद्यास्तद्वल्लभतयैव हि
तद्राजेन्द्र यथा स्नेहः स्वस्वकात्मनि देहिनाम्‌
न तथा ममतालम्बि पुत्रवित्तगृहादिषु
देहात्मवादिनां पुंसामपि राजन्यसत्तम
यथा देहः प्रियतमस्तथा न ह्यनु ये च तम्‌
देहोऽपि ममताभाक्चेत्तर्ह्यसौ नात्मवत्प्रियः
यज्जीर्यत्यपि देहेऽस्मिन्जीविताशा बलीयसी
तस्मात्प्रियतमः स्वात्मा सर्वेषामपि देहिनाम्‌
तदर्थमेव सकलं जगदेतच्चराचरम्‌
कृष्णमेनमवेहि त्वमात्मानमखिलात्मनाम्‌
जगद्धिताय सोऽप्यत्र देहीवाभाति मायया
वस्तुतो जानतामत्र कृष्णं स्थास्नु चरिष्णु च
भगवद्रूपमखिलं नान्यद्वस्त्विह किञ्चन
सर्वेषामपि वस्तूनां भावार्थो भवति स्थितः
तस्यापि भगवान्कृष्णः किमतद्वस्तु रूप्यताम्‌
समाश्रिता ये पदपल्लवप्लवं महत्पदं पुण्ययशो मुरारेः
भवाम्बुधिर्वत्सपदं परं पदं पदं पदं यद्विपदां न तेषाम्‌
एतत्ते सर्वमाख्यातं यत्पृष्टोऽहमिह त्वया
तत्कौमारे हरिकृतं पौगण्डे परिकीर्तितम्‌
एतत्सुहृद्भिश्चरितं मुरारेरघार्दनं शाद्वलजेमनं च
व्यक्तेतरद्रूपमजोर्वभिष्टवं शृण्वन्गृणन्नेति नरोऽखिलार्थान्‌
एवं विहारैः कौमारैः कौमारं जहतुर्व्रजे
निलायनैः सेतुबन्धैर्मर्कटोत्प्लवनादिभिः
श्रीशुक उवाच
ततश्च पौगण्डवयःश्रीतौ व्रजे
बभूवतुस्तौ पशुपालसम्मतौ
गाश्चारयन्तौ सखिभिः समं पदैर्‌
वृन्दावनं पुण्यमतीव चक्रतुः
तन्माधवो वेणुमुदीरयन्वृतो गोपैर्गृणद्भिः स्वयशो बलान्वितः
पशून्पुरस्कृत्य पशव्यमाविशद्विहर्तुकामः कुसुमाकरं वनम्‌
तन्मञ्जुघोषालिमृगद्विजाकुलं महन्मनःप्रख्यपयःसरस्वता
वातेन जुष्टं शतपत्रगन्धिना निरीक्ष्य रन्तुं भगवान्मनो दधे
स तत्र तत्रारुणपल्लवश्रिया फलप्रसूनोरुभरेण पादयोः
स्पृशच्छिखान्वीक्ष्य वनस्पतीन्मुदा स्मयन्निवाहाग्रजमादिपूरुषः
श्रीभगवानुवाच
अहो अमी देववरामरार्चितं पादाम्बुजं ते सुमनःफलार्हणम्‌
नमन्त्युपादाय शिखाभिरात्मनस्तमोऽपहत्यै तरुजन्म यत्कृतम्‌
एतेऽलिनस्तव यशोऽखिललोकतीर्थं
गायन्त आदिपुरुषानुपथं भजन्ते
प्रायो अमी मुनिगणा भवदीयमुख्या
गूढं वनेऽपि न जहत्यनघात्मदैवम्‌
नृत्यन्त्यमी शिखिन ईड्य मुदा हरिण्यः
कुर्वन्ति गोप्य इव ते प्रियमीक्षणेन
सूक्तैश्च कोकिलगणा गृहमागताय
धन्या वनौकस इयान्हि सतां निसर्गः
धन्येयमद्य धरणी तृणवीरुधस्त्वत्‌
पादस्पृशो द्रुमलताः करजाभिमृष्टाः
नद्योऽद्रयः खगमृगाः सदयावलोकैर्‌
गोप्योऽन्तरेण भुजयोरपि यत्स्पृहा श्रीः
श्रीशुक उवाच
एवं वृन्दावनं श्रीमत्कृष्णः प्रीतमनाः पशून्‌
रेमे सञ्चारयन्नद्रेः सरिद्रोधःसु सानुगः
क्वचिद्गायति गायत्सु मदान्धालिष्वनुव्रतैः
उपगीयमानचरितः पथि सङ्कर्षणान्वितः
अनुजल्पति जल्पन्तं कलवाक्यैः शुकं क्वचित्‌
क्वचित्सवल्गु कूजन्तमनुकूजति कोकिलम्‌
क्वचिच्च कालहंसानामनुकूजति कूजितम्‌
अभिनृत्यति नृत्यन्तं बर्हिणं हासयन्क्वचित्‌
मेघगम्भीरया वाचा नामभिर्दूरगान्पशून्‌
क्वचिदाह्वयति प्रीत्या गोगोपालमनोज्ञया
चकोरक्रौञ्चचक्राह्व भारद्वाजांश्च बर्हिणः
अनुरौति स्म सत्त्वानां भीतवद्व्याघ्रसिंहयोः
क्वचित्क्रीडापरिश्रान्तं गोपोत्सङ्गोपबर्हणम्‌
स्वयं विश्रमयत्यार्यं पादसंवाहनादिभिः
नृत्यतो गायतः क्वापि वल्गतो युध्यतो मिथः
गृहीतहस्तौ गोपालान्हसन्तौ प्रशशंसतुः
क्वचित्पल्लवतल्पेषु नियुद्धश्रमकर्शितः
वृक्षमूलाश्रयः शेते गोपोत्सङ्गोपबर्हणः
पादसंवाहनं चक्रुः केचित्तस्य महात्मनः
अपरे हतपाप्मानो व्यजनैः समवीजयन्‌
अन्ये तदनुरूपाणि मनोज्ञानि महात्मनः
गायन्ति स्म महाराज स्नेहक्लिन्नधियः शनैः
एवं निगूढात्मगतिः स्वमायया गोपात्मजत्वं चरितैर्विडम्बयन्‌
रेमे रमालालितपादपल्लवो ग्राम्यैः समं ग्राम्यवदीशचेष्टितः
श्रीदामा नाम गोपालो रामकेशवयोः सखा
सुबलस्तोककृष्णाद्या गोपाः प्रेम्णेदमब्रुवन्‌
राम राम महाबाहो कृष्ण दुष्टनिबर्हण
इतोऽविदूरे सुमहद्वनं तालालिसङ्कुलम्‌
फलानि तत्र भूरीणि पतन्ति पतितानि च
सन्ति किन्त्ववरुद्धानि धेनुकेन दुरात्मना
सोऽतिवीर्योऽसुरो राम हे कृष्ण खररूपधृक्‌
आत्मतुल्यबलैरन्यैर्ज्ञातिभिर्बहुभिर्वृतः
तस्मात्कृतनराहाराद्भीतैर्नृभिरमित्रहन्‌
न सेव्यते पशुगणैः पक्षिसङ्घैर्विवर्जितम्‌
विद्यन्तेऽभुक्तपूर्वाणि फलानि सुरभीणि च
एष वै सुरभिर्गन्धो विषूचीनोऽवगृह्यते
प्रयच्छ तानि नः कृष्ण गन्धलोभितचेतसाम्‌
वाञ्छास्ति महती राम गम्यतां यदि रोचते
एवं सुहृद्वचः श्रुत्वा सुहृत्प्रियचिकीर्षया
प्रहस्य जग्मतुर्गोपैर्वृतौ तालवनं प्रभू
बलः प्रविश्य बाहुभ्यां तालान्सम्परिकम्पयन्‌
फलानि पातयामास मतङ्गज इवौजसा
फलानां पततां शब्दं निशम्यासुररासभः
अभ्यधावत्क्षितितलं सनगं परिकम्पयन्‌
समेत्य तरसा प्रत्यग्द्वाभ्यां पद्भ्यां बलं बली
निहत्योरसि काशब्दं मुञ्चन्पर्यसरत्खलः
पुनरासाद्य संरब्ध उपक्रोष्टा पराक्स्थितः
चरणावपरौ राजन्बलाय प्राक्षिपद्रुषा
स तं गृहीत्वा प्रपदोर्भ्रामयित्वैकपाणिना
चिक्षेप तृणराजाग्रे भ्रामणत्यक्तजीवितम्‌
तेनाहतो महातालो वेपमानो बृहच्छिराः
पार्श्वस्थं कम्पयन्भग्नः स चान्यं सोऽपि चापरम्‌
बलस्य लीलयोत्सृष्ट खरदेहहताहताः
तालाश्चकम्पिरे सर्वे महावातेरिता इव
नैतच्चित्रं भगवति ह्यनन्ते जगदीश्वरे
ओतप्रोतमिदं यस्मिंस्तन्तुष्वङ्ग यथा पटः
ततः कृष्णं च रामं च ज्ञातयो धेनुकस्य ये
क्रोष्टारोऽभ्यद्रवन्सर्वे संरब्धा हतबान्धवाः
तांस्तानापततः कृष्णो रामश्च नृप लीलया
गृहीतपश्चाच्चरणान्प्राहिणोत्तृणराजसु
फलप्रकरसङ्कीर्णं दैत्यदेहैर्गतासुभिः
रराज भूः सतालाग्रैर्घनैरिव नभस्तलम्‌
तयोस्तत्सुमहत्कर्म निशम्य विबुधादयः
मुमुचुः पुष्पवर्षाणि चक्रुर्वाद्यानि तुष्टुवुः
अथ तालफलान्यादन्मनुष्या गतसाध्वसाः
तृणं च पशवश्चेरुर्हतधेनुककानने
कृष्णः कमलपत्राक्षः पुण्यश्रवणकीर्तनः
स्तूयमानोऽनुगैर्गोपैः साग्रजो व्रजमाव्रजत्‌
तं गोरजश्छुरितकुन्तलबद्धबर्ह
वन्यप्रसूनरुचिरेक्षणचारुहासम्‌
वेणुम्क्वणन्तमनुगैरुपगीतकीर्तिं
गोप्यो दिदृक्षितदृशोऽभ्यगमन्समेताः
पीत्वा मुकुन्दमुखसारघमक्षिभृङ्गैस्‌
तापं जहुर्विरहजं व्रजयोषितोऽह्नि
तत्सत्कृतिं समधिगम्य विवेश गोष्ठं
सव्रीडहासविनयं यदपाङ्गमोक्षम्‌
तयोर्यशोदारोहिण्यौ पुत्रयोः पुत्रवत्सले
यथाकामं यथाकालं व्यधत्तां परमाशिषः
गताध्वानश्रमौ तत्र मज्जनोन्मर्दनादिभिः
नीवीं वसित्वा रुचिरां दिव्यस्रग्गन्धमण्डितौ
जनन्युपहृतं प्राश्य स्वाद्यन्नमुपलालितौ
संविश्य वरशय्यायां सुखं सुषुपतुर्व्रजे
एवं स भगवान्कृष्णो वृन्दावनचरः क्वचित्‌
ययौ राममृते राजन्कालिन्दीं सखिभिर्वृतः
अथ गावश्च गोपाश्च निदाघातपपीडिताः
दुष्टं जलं पपुस्तस्यास्तृष्णार्ता विषदूषितम्‌
विषाम्भस्तदुपस्पृश्य दैवोपहतचेतसः
निपेतुर्व्यसवः सर्वे सलिलान्ते कुरूद्वह
वीक्ष्य तान्वै तथाभूतान्कृष्णो योगेश्वरेश्वरः
ईक्षयामृतवर्षिण्या स्वनाथान्समजीवयत्‌
ते सम्प्रतीतस्मृतयः समुत्थाय जलान्तिकात्‌
आसन्सुविस्मिताः सर्वे वीक्षमाणाः परस्परम्‌
अन्वमंसत तद्राजन्गोविन्दानुग्रहेक्षितम्‌
पीत्वा विषं परेतस्य पुनरुत्थानमात्मनः
श्रीशुक उवाच
विलोक्य दूषितां कृष्णां कृष्णः कृष्णाहिना विभुः
तस्या विशुद्धिमन्विच्छन्सर्पं तमुदवासयत्‌
श्रीराजोवाच
कथमन्तर्जलेऽगाधे न्यगृह्णाद्भगवानहिम्‌
स वै बहुयुगावासं यथासीद्विप्र कथ्यताम्‌
ब्रह्मन्भगवतस्तस्य भूम्नः स्वच्छन्दवर्तिनः
गोपालोदारचरितं कस्तृप्येतामृतं जुषन्‌
श्रीशुक उवाच
कालिन्द्यां कालियस्यासीध्रदः कश्चिद्विषाग्निना
श्रप्यमाणपया यस्मिन्पतन्त्युपरिगाः खगाः
विप्रुष्मता विषदोर्मि मारुतेनाभिमर्शिताः
म्रियन्ते तीरगा यस्य प्राणिनः स्थिरजङ्गमाः
तं चण्डवेगविषवीर्यमवेक्ष्य तेन
दुष्टां नदीं च खलसंयमनावतारः
कृष्णः कदम्बमधिरुह्य ततोऽतितुङ्गम्‌
आस्फोट्य गाढरशनो न्यपतद्विषोदे
सर्पह्रदः पुरुषसारनिपातवेग
सङ्क्षोभितोरगविषोच्छ्वसिताम्बुराशिः
पर्यक्प्लुतो विषकषायबिभीषणोर्मिर्‌
धावन्धनुःशतमनन्तबलस्य किं तत्‌
तस्य ह्रदे विहरतो भुजदण्डघूर्ण
वार्घोषमङ्ग वरवारणविक्रमस्य
आश्रुत्य तत्स्वसदनाभिभवं निरीक्ष्य
चक्षुःश्रवाः समसरत्तदमृष्यमाणः
तं प्रेक्षणीयसुकुमारघनावदातं
श्रीवत्सपीतवसनं स्मितसुन्दरास्यम्‌
क्रीडन्तमप्रतिभयं कमलोदराङ्घ्रिं
सन्दश्य मर्मसु रुषा भुजया चछाद
तं नागभोगपरिवीतमदृष्टचेष्टम्‌
आलोक्य तत्प्रियसखाः पशुपा भृशार्ताः
कृष्णेऽर्पितात्मसुहृदर्थकलत्रकामा
दुःखानुशोकभयमूढधियो निपेतुः
गावो वृषा वत्सतर्यः क्रन्दमानाः सुदुःखिताः
कृष्णे न्यस्तेक्षणा भीता रुदन्त्य इव तस्थिरे
अथ व्रजे महोत्पातास्त्रिविधा ह्यतिदारुणाः
उत्पेतुर्भुवि दिव्यात्मन्यासन्नभयशंसिनः
तानालक्ष्य भयोद्विग्ना गोपा नन्दपुरोगमाः
विना रामेण गाः कृष्णं ज्ञात्वा चारयितुं गतम्‌
तैर्दुर्निमित्तैर्निधनं मत्वा प्राप्तमतद्विदः
तत्प्राणास्तन्मनस्कास्ते दुःखशोकभयातुराः
आबालवृद्धवनिताः सर्वेऽङ्ग पशुवृत्तयः
निर्जग्मुर्गोकुलाद्दीनाः कृष्णदर्शनलालसाः
तांस्तथा कातरान्वीक्ष्य भगवान्माधवो बलः
प्रहस्य किञ्चिन्नोवाच प्रभावज्ञोऽनुजस्य सः
तेऽन्वेषमाणा दयितं कृष्णं सूचितया पदैः
भगवल्लक्षणैर्जग्मुः पदव्या यमुनातटम्‌
ते तत्र तत्राब्जयवाङ्कुशाशनि ध्वजोपपन्नानि पदानि विश्पतेः
मार्गे गवामन्यपदान्तरान्तरे निरीक्षमाणा ययुरङ्ग सत्वराः
अन्तर्ह्रदे भुजगभोगपरीतमारात्‌
कृष्णं निरीहमुपलभ्य जलाशयान्ते
गोपांश्च मूढधिषणान्परितः पशूंश्च
सङ्क्रन्दतः परमकश्मलमापुरार्ताः
गोप्योऽनुरक्तमनसो भगवत्यनन्ते
तत्सौहृदस्मितविलोकगिरः स्मरन्त्यः
ग्रस्तेऽहिना प्रियतमे भृशदुःखतप्ताः
शून्यं प्रियव्यतिहृतं ददृशुस्त्रिलोकम्‌
ताः कृष्णमातरमपत्यमनुप्रविष्टां
तुल्यव्यथाः समनुगृह्य शुचः स्रवन्त्यः
तास्ता व्रजप्रियकथाः कथयन्त्य आसन्‌
कृष्णाननेऽर्पितदृशो मृतकप्रतीकाः
कृष्णप्राणान्निर्विशतो नन्दादीन्वीक्ष्य तं ह्रदम्‌
प्रत्यषेधत्स भगवान्रामः कृष्णानुभाववित्‌
इत्थम्स्वगोकुलमनन्यगतिं निरीक्ष्य
सस्त्रीकुमारमतिदुःखितमात्महेतोः
आज्ञाय मर्त्यपदवीमनुवर्तमानः
स्थित्वा मुहूर्तमुदतिष्ठदुरङ्गबन्धात्‌
तत्प्रथ्यमानवपुषा व्यथितात्मभोगस्‌
त्यक्त्वोन्नमय्य कुपितः स्वफणान्भुजङ्गः
तस्थौ श्वसञ्छ्वसनरन्ध्रविषाम्बरीष
स्तब्धेक्षणोल्मुकमुखो हरिमीक्षमाणः
तं जिह्वया द्विशिखया परिलेलिहानं
द्वे सृक्वणी ह्यतिकरालविषाग्निदृष्टिम्‌
क्रीडन्नमुं परिससार यथा खगेन्द्रो
बभ्राम सोऽप्यवसरं प्रसमीक्षमाणः
एवं परिभ्रमहतौजसमुन्नतांसम्‌
आनम्य तत्पृथुशिरःस्वधिरूढ आद्यः
तन्मूर्धरत्ननिकरस्पर्शातिताम्र
पादाम्बुजोऽखिलकलादिगुरुर्ननर्त
तं नर्तुमुद्यतमवेक्ष्य तदा तदीय
गन्धर्वसिद्धमुनिचारणदेववध्वः
प्रीत्या मृदङ्गपणवानकवाद्यगीत
पुष्पोपहारनुतिभिः सहसोपसेदुः
यद्यच्छिरो न नमतेऽङ्ग शतैकशीर्ष्णस्‌
तत्तन्ममर्द खरदण्डधरोऽङ्घ्रिपातैः
क्षीणायुषो भ्रमत उल्बणमास्यतोऽसृङ्‌
नस्तो वमन्परमकश्मलमाप नागः
तस्याक्षिभिर्गरलमुद्वमतः शिरःसु
यद्यत्समुन्नमति निःश्वसतो रुषोच्चैः
नृत्यन्पदानुनमयन्दमयां बभूव
पुष्पैः प्रपूजित इवेह पुमान्पुराणः
तच्चित्रताण्डवविरुग्नफणासहस्रो
रक्तं मुखैरुरु वमन्नृप भग्नगात्रः
स्मृत्वा चराचरगुरुं पुरुषं पुराणं
नारायणं तमरणं मनसा जगाम
कृष्णस्य गर्भजगतोऽतिभरावसन्नं
पार्ष्णिप्रहारपरिरुग्नफणातपत्रम्‌
दृष्ट्वाहिमाद्यमुपसेदुरमुष्य पत्न्य
आर्ताः श्लथद्वसनभूषणकेशबन्धाः
तास्तं सुविग्नमनसोऽथ पुरस्कृतार्भाः
कायं निधाय भुवि भूतपतिं प्रणेमुः
साध्व्यः कृताञ्जलिपुटाः शमलस्य भर्तुर्‌
मोक्षेप्सवः शरणदं शरणं प्रपन्नाः
नागपत्न्य ऊचुः
न्याय्यो हि दण्डः कृतकिल्बिषेऽस्मिंस्‌
तवावतारः खलनिग्रहाय
रिपोः सुतानामपि तुल्यदृष्टिर्‌
धत्से दमं फलमेवानुशंसन्‌
अनुग्रहोऽयं भवतः कृतो हि नो दण्डोऽसतां ते खलु कल्मषापहः
यद्दन्दशूकत्वममुष्य देहिनः क्रोधोऽपि तेऽनुग्रह एव सम्मतः
तपः सुतप्तं किमनेन पूर्वं निरस्तमानेन च मानदेन
धर्मोऽथ वा सर्वजनानुकम्पया यतो भवांस्तुष्यति सर्वजीवः
कस्यानुभावोऽस्य न देव विद्महे तवाङ्घ्रिरेणुस्परशाधिकारः
यद्वाञ्छया श्रीर्ललनाचरत्तपो विहाय कामान्सुचिरं धृतव्रता
न नाकपृष्ठं न च सार्वभौमं
न पारमेष्ठ्यं न रसाधिपत्यम्‌
न योगसिद्धीरपुनर्भवं वा
वाञ्छन्ति यत्पादरजःप्रपन्नाः
तदेष नाथाप दुरापमन्यैस्तमोजनिः क्रोधवशोऽप्यहीशः
संसारचक्रे भ्रमतः शरीरिणो यदिच्छतः स्याद्विभवः समक्षः
नमस्तुभ्यं भगवते पुरुषाय महात्मने
भूतावासाय भूताय पराय परमात्मने
ज्ञानविज्ञाननीधये ब्रह्मणेऽनन्तशक्तये
अगुणायाविकाराय नमस्ते प्राकृताय च
कालाय कालनाभाय कालावयवसाक्षिणे
विश्वाय तदुपद्रष्ट्रे तत्कर्त्रे विश्वहेतवे
भूतमात्रेन्द्रियप्राण मनोबुद्ध्याशयात्मने
त्रिगुणेनाभिमानेन गूढस्वात्मानुभूतये
नमोऽनन्ताय सूक्ष्माय कूटस्थाय विपश्चिते
नानावादानुरोधाय वाच्यवाचकशक्तये
नमः प्रमाणमूलाय कवये शास्त्रयोनये
प्रवृत्ताय निवृत्ताय निगमाय नमो नमः
नमः कृष्णाय रामाय वसुदेवसुताय च
प्रद्युम्नायानिरुद्धाय सात्वतां पतये नमः
नमो गुणप्रदीपाय गुणात्मच्छादनाय च
गुणवृत्त्युपलक्ष्याय गुणद्रष्ट्रे स्वसंविदे
अव्याकृतविहाराय सर्वव्याकृतसिद्धये
हृषीकेश नमस्तेऽस्तु मुनये मौनशीलिने
परावरगतिज्ञाय सर्वाध्यक्षाय ते नमः
अविश्वाय च विश्वाय तद्द्रष्ट्रेऽस्य च हेतवे
त्वं ह्यस्य जन्मस्थितिसंयमान्विभो
गुणैरनीहोऽकृतकालशक्तिधृक्‌
तत्तत्स्वभावान्प्रतिबोधयन्सतः
समीक्षयामोघविहार ईहसे
तस्यैव तेऽमूस्तनवस्त्रिलोक्यां
शान्ता अशान्ता उत मूढयोनयः
शान्ताः प्रियास्ते ह्यधुनावितुं सतां
स्थातुश्च ते धर्मपरीप्सयेहतः
अपराधः सकृद्भर्त्रा सोढव्यः स्वप्रजाकृतः
क्षन्तुमर्हसि शान्तात्मन्मूढस्य त्वामजानतः
अनुगृह्णीष्व भगवन्प्राणांस्त्यजति पन्नगः
स्त्रीणां नः साधुशोच्यानां पतिः प्राणः प्रदीयताम्‌
विधेहि ते किङ्करीणामनुष्ठेयं तवाज्ञया
यच्छ्रद्धयानुतिष्ठन्वै मुच्यते सर्वतो भयात्‌
श्रीशुक उवाच
इत्थं स नागपत्नीभिर्भगवान्समभिष्टुतः
मूर्च्छितं भग्नशिरसं विससर्जाङ्घ्रिकुट्टनैः
प्रतिलब्धेन्द्रियप्राणः कालियः शनकैर्हरिम्‌
कृच्छ्रात्समुच्छ्वसन्दीनः कृष्णं प्राह कृताञ्जलिः
कालिय उवाच
वयं खलाः सहोत्पत्त्या तमसा दीर्घमन्यवः
स्वभावो दुस्त्यजो नाथ लोकानां यदसद्ग्रहः
त्वया सृष्टमिदं विश्वं धातर्गुणविसर्जनम्‌
नानास्वभाववीर्यौजो योनिबीजाशयाकृति
वयं च तत्र भगवन्सर्पा जात्युरुमन्यवः
कथं त्यजामस्त्वन्मायां दुस्त्यजां मोहिताः स्वयम्‌
भवान्हि कारणं तत्र सर्वज्ञो जगदीश्वरः
अनुग्रहं निग्रहं वा मन्यसे तद्विधेहि नः
श्रीशुक उवाच
इत्याकर्ण्य वचः प्राह भगवान्कार्यमानुषः
नात्र स्थेयं त्वया सर्प समुद्रं याहि मा चिरम्‌
स्वज्ञात्यपत्यदाराढ्यो गोनृभिर्भुज्यते नदी
य एतत्संस्मरेन्मर्त्यस्तुभ्यं मदनुशासनम्‌
कीर्तयन्नुभयोः सन्ध्योर्न युष्मद्भयमाप्नुयात्‌
योऽस्मिन्स्नात्वा मदाक्रीडे देवादींस्तर्पयेज्जलैः
उपोष्य मां स्मरन्नर्चेत्सर्वपापैः प्रमुच्यते
द्वीपं रमणकं हित्वा ह्रदमेतमुपाश्रितः
यद्भयात्स सुपर्णस्त्वां नाद्यान्मत्पादलाञ्छितम्‌
श्रीऋषिरुवाच
मुक्तो भगवता राजन्कृष्णेनाद्भुतकर्मणा
तं पूजयामास मुदा नागपत्न्यश्च सादरम्‌
दिव्याम्बरस्रङ्मणिभिः परार्ध्यैरपि भूषणैः
दिव्यगन्धानुलेपैश्च महत्योत्पलमालया
पूजयित्वा जगन्नाथं प्रसाद्य गरुडध्वजम्‌
ततः प्रीतोऽभ्यनुज्ञातः परिक्रम्याभिवन्द्य तम्‌
सकलत्रसुहृत्पुत्रो द्वीपमब्धेर्जगाम ह
तदैव सामृतजला यमुना निर्विषाभवत्‌
अनुग्रहाद्भगवतः क्रीडामानुषरूपिणः
श्रीराजोवाच
नागालयं रमणकं कथं तत्याज कालियः
कृतं किं वा सुपर्णस्य तेनैकेनासमञ्जसम्‌
श्रीशुक उवाच
उपहार्यैः सर्पजनैर्मासि मासीह यो बलिः
वानस्पत्यो महाबाहो नागानां प्राङ्निरूपितः
स्वं स्वं भागं प्रयच्छन्ति नागाः पर्वणि पर्वणि
गोपीथायात्मनः सर्वे सुपर्णाय महात्मने
विषवीर्यमदाविष्टः काद्रवेयस्तु कालियः
कदर्थीकृत्य गरुडं स्वयं तं बुभुजे बलिम्‌
तच्छ्रुत्वा कुपितो राजन्भगवान्भगवत्प्रियः
विजिघांसुर्महावेगः कालियं समपाद्रवत्‌
तमापतन्तं तरसा विषायुधः प्रत्यभ्ययादुत्थितनैकमस्तकः
दद्भिः सुपर्णं व्यदशद्ददायुधः करालजिह्रोच्छ्वसितोग्रलोचनः
तं तार्क्ष्यपुत्रः स निरस्य मन्युमान्‌
प्रचण्डवेगो मधुसूदनासनः
पक्षेण सव्येन हिरण्यरोचिषा
जघान कद्रुसुतमुग्रविक्रमः
सुपर्णपक्षाभिहतः कालियोऽतीव विह्वलः
ह्रदं विवेश कालिन्द्यास्तदगम्यं दुरासदम्‌
तत्रैकदा जलचरं गरुडो भक्ष्यमीप्सितम्‌
निवारितः सौभरिणा प्रसह्य क्षुधितोऽहरत्‌
मीनान्सुदुःखितान्दृष्ट्वा दीनान्मीनपतौ हते
कृपया सौभरिः प्राह तत्रत्यक्षेममाचरन्‌
अत्र प्रविश्य गरुडो यदि मत्स्यान्स खादति
सद्यः प्राणैर्वियुज्येत सत्यमेतद्ब्रवीम्यहम्‌
तत्कालियः परं वेद नान्यः कश्चन लेलिहः
अवात्सीद्गरुडाद्भीतः कृष्णेन च विवासितः
कृष्णं ह्रदाद्विनिष्क्रान्तं दिव्यस्रग्गन्धवाससम्‌
महामणिगणाकीर्णं जाम्बूनदपरिष्कृतम्‌
उपलभ्योत्थिताः सर्वे लब्धप्राणा इवासवः
प्रमोदनिभृतात्मानो गोपाः प्रीत्याभिरेभिरे
यशोदा रोहिणी नन्दो गोप्यो गोपाश्च कौरव
कृष्णं समेत्य लब्धेहा आसन्शुष्का नगा अपि
रामश्चाच्युतमालिङ्ग्य जहासास्यानुभाववित्‌
प्रेम्णा तमङ्कमारोप्य पुनः पुनरुदैक्षत
गावो वृषा वत्सतर्यो लेभिरे परमां मुदम्‌
नन्दं विप्राः समागत्य गुरवः सकलत्रकाः
ऊचुस्ते कालियग्रस्तो दिष्ट्या मुक्तस्तवात्मजः
देहि दानं द्विजातीनां कृष्णनिर्मुक्तिहेतवे
नन्दः प्रीतमना राजन्गाः सुवर्णं तदादिशत्‌
यशोदापि महाभागा नष्टलब्धप्रजा सती
परिष्वज्याङ्कमारोप्य मुमोचाश्रुकलां मुहुः
तां रात्रिं तत्र राजेन्द्र क्षुत्तृड्भ्यां श्रमकर्षिताः
ऊषुर्व्रयौकसो गावः कालिन्द्या उपकूलतः
तदा शुचिवनोद्भूतो दावाग्निः सर्वतो व्रजम्‌
सुप्तं निशीथ आवृत्य प्रदग्धुमुपचक्रमे
तत उत्थाय सम्भ्रान्ता दह्यमाना व्रजौकसः
कृष्णं ययुस्ते शरणं मायामनुजमीश्वरम्‌
कृष्ण कृष्ण महाभग हे रामामितविक्रम
एष घोरतमो वह्निस्तावकान्ग्रसते हि नः
सुदुस्तरान्नः स्वान्पाहि कालाग्नेः सुहृदः प्रभो
न शक्नुमस्त्वच्चरणं सन्त्यक्तुमकुतोभयम्‌
इत्थं स्वजनवैक्लव्यं निरीक्ष्य जगदीश्वरः
तमग्निमपिबत्तीव्रमनन्तोऽनन्तशक्तिधृक्‌
श्रीशुक उवाच
अथ कृष्णः परिवृतो ज्ञातिभिर्मुदितात्मभिः
अनुगीयमानो न्यविशद्व्रजं गोकुलमण्डितम्‌
व्रजे विक्रीडतोरेवं गोपालच्छद्ममायया
ग्रीष्मो नामर्तुरभवन्नातिप्रेयाञ्छरीरिणाम्‌
स च वृन्दावनगुणैर्वसन्त इव लक्षितः
यत्रास्ते भगवान्साक्षाद्रामेण सह केशवः
यत्र निर्झरनिर्ह्राद निवृत्तस्वनझिल्लिकम्‌
शश्वत्तच्छीकरर्जीष द्रुममण्डलमण्डितम्‌
सरित्सरःप्रस्रवणोर्मिवायुना कह्लारकञ्जोत्पलरेणुहारिणा
न विद्यते यत्र वनौकसां दवो निदाघवह्न्यर्कभवोऽतिशाद्वले
अगाधतोयह्रदिनीतटोर्मिभिर्द्रवत्पुरीष्याः पुलिनैः समन्ततः
न यत्र चण्डांशुकरा विषोल्बणा भुवो रसं शाद्वलितं च गृह्णते
वनं कुसुमितं श्रीमन्नदच्चित्रमृगद्विजम्‌
गायन्मयूरभ्रमरं कूजत्कोकिलसारसम्‌
क्रीडिष्यमाणस्तत्क्र्ष्णो भगवान्बलसंयुतः
वेणुं विरणयन्गोपैर्गोधनैः संवृतोऽविशत्‌
प्रवालबर्हस्तबक स्रग्धातुकृतभूषणाः
रामकृष्णादयो गोपा ननृतुर्युयुधुर्जगुः
कृष्णस्य नृत्यतः केचिज्जगुः केचिदवादयन्‌
वेणुपाणितलैः शृङ्गैः प्रशशंसुरथापरे
गोपजातिप्रतिच्छन्ना देवा गोपालरूपिणौ
ईडिरे कृष्णरामौ च नटा इव नटं नृप
भ्रमणैर्लङ्घनैः क्षेपैरास्फोटनविकर्षणैः
चिक्रीडतुर्नियुद्धेन काकपक्षधरौ क्वचित्‌
क्वचिन्नृत्यत्सु चान्येषु गायकौ वादकौ स्वयम्‌
शशंसतुर्महाराज साधु साध्विति वादिनौ
क्वचिद्बिल्वैः क्वचित्कुम्भैः क्वचामलकमुष्टिभिः
अस्पृश्यनेत्रबन्धाद्यैः क्वचिन्मृगखगेहया
क्वचिच्च दर्दुरप्लावैर्विविधैरुपहासकैः
कदाचित्स्यन्दोलिकया कर्हिचिन्नृपचेष्टया
एवं तौ लोकसिद्धाभिः क्रीडाभिश्चेरतुर्वने
नद्यद्रिद्रोणिकुञ्जेषु काननेषु सरःसु च
पशूंश्चारयतोर्गोपैस्तद्वने रामकृष्णयोः
गोपरूपी प्रलम्बोऽगादसुरस्तज्जिहीर्षया
तं विद्वानपि दाशार्हो भगवान्सर्वदर्शनः
अन्वमोदत तत्सख्यं वधं तस्य विचिन्तयन्‌
तत्रोपाहूय गोपालान्कृष्णः प्राह विहारवित्‌
हे गोपा विहरिष्यामो द्वन्द्वीभूय यथायथम्‌
तत्र चक्रुः परिवृढौ गोपा रामजनार्दनौ
कृष्णसङ्घट्टिनः केचिदासन्रामस्य चापरे
आचेरुर्विविधाः क्रीडा वाह्यवाहकलक्षणाः
यत्रारोहन्ति जेतारो वहन्ति च पराजिताः
वहन्तो वाह्यमानाश्च चारयन्तश्च गोधनम्‌
भाण्डीरकं नाम वटं जग्मुः कृष्णपुरोगमाः
रामसङ्घट्टिनो यर्हि श्रीदामवृषभादयः
क्रीडायां जयिनस्तांस्तानूहुः कृष्णादयो नृप
उवाह कृष्णो भगवान्श्रीदामानं पराजितः
वृषभं भद्रसेनस्तु प्रलम्बो रोहिणीसुतम्‌
अविषह्यं मन्यमानः कृष्णं दानवपुङ्गवः
वहन्द्रुततरं प्रागादवरोहणतः परम्‌
तमुद्वहन्धरणिधरेन्द्रगौरवं
महासुरो विगतरयो निजं वपुः
स आस्थितः पुरटपरिच्छदो बभौ
तडिद्द्युमानुडुपतिवाडिवाम्बुदः
निरीक्ष्य तद्वपुरलमम्बरे चरत्‌
प्रदीप्तदृग्भ्रुकुटितटोग्रदंष्ट्रकम्‌
ज्वलच्छिखं कटककिरीटकुण्डल
त्विषाद्भुतं हलधर ईषदत्रसत्‌
अथागतस्मृतिरभयो रिपुं बलो विहाय सार्थमिव हरन्तमात्मनः
रुषाहनच्छिरसि दृढेन मुष्टिना सुराधिपो गिरिमिव वज्ररंहसा
स आहतः सपदि विशीर्णमस्तको मुखाद्वमन्रुधिरमपस्मृतोऽसुरः
महारवं व्यसुरपतत्समीरयन्गिरिर्यथा मघवत आयुधाहतः
दृष्ट्वा प्रलम्बं निहतं बलेन बलशालिना
गोपाः सुविस्मिता आसन्साधु साध्विति वादिनः
आशिषोऽभिगृणन्तस्तं प्रशशंसुस्तदर्हणम्‌
प्रेत्यागतमिवालिङ्ग्य प्रेमविह्वलचेतसः
पापे प्रलम्बे निहते देवाः परमनिर्वृताः
अभ्यवर्षन्बलं माल्यैः शशंसुः साधु साध्विति
श्रीशुक उवाच
क्रीडासक्तेषु गोपेषु तद्गावो दूरचारिणीः
स्वैरं चरन्त्यो विविशुस्तृणलोभेन गह्वरम्‌
अजा गावो महिष्यश्च निर्विशन्त्यो वनाद्वनम्‌
ईषीकाटवीं निर्विविशुः क्रन्दन्त्यो दावतर्षिताः
तेऽपश्यन्तः पशून्गोपाः कृष्णरामादयस्तदा
जातानुतापा न विदुर्विचिन्वन्तो गवां गतिम्‌
तृणैस्तत्खुरदच्छिन्नैर्गोष्पदैरङ्कितैर्गवाम्‌
मार्गमन्वगमन्सर्वे नष्टाजीव्या विचेतसः
मुञ्जाटव्यां भ्रष्टमार्गं क्रन्दमानं स्वगोधनम्‌
सम्प्राप्य तृषिताः श्रान्तास्ततस्ते सन्न्यवर्तयन्‌
ता आहूता भगवता मेघगम्भीरया गिरा
स्वनाम्नां निनदं श्रुत्वा प्रतिनेदुः प्रहर्षिताः
ततः समन्ताद्दवधूमकेतुर्यदृच्छयाभूत्क्षयकृद्वनौकसाम्‌
समीरितः सारथिनोल्बणोल्मुकैर्विलेलिहानः स्थिरजङ्गमान्महान्‌
तमापतन्तं परितो दवाग्निं गोपाश्च गावः प्रसमीक्ष्य भीताः
ऊचुश्च कृष्णं सबलं प्रपन्ना यथा हरिं मृत्युभयार्दिता जनाः
कृष्ण कृष्ण महावीर हे रामामोघ विक्रम
दावाग्निना दह्यमानान्प्रपन्नांस्त्रातुमर्हथः
नूनं त्वद्बान्धवाः कृष्ण न चार्हन्त्यवसादितुम्‌
वयं हि सर्वधर्मज्ञ त्वन्नाथास्त्वत्परायणाः
श्रीशुक उवाच
वचो निशम्य कृपणं बन्धूनां भगवान्हरिः
निमीलयत मा भैष्ट लोचनानीत्यभाषत
तथेति मीलिताक्षेषु भगवानग्निमुल्बणम्‌
पीत्वा मुखेन तान्कृच्छ्राद्योगाधीशो व्यमोचयत्‌
ततश्च तेऽक्षीण्युन्मील्य पुनर्भाण्डीरमापिताः
निशम्य विस्मिता आसन्नात्मानं गाश्च मोचिताः
कृष्णस्य योगवीर्यं तद्योगमायानुभावितम्‌
दावाग्नेरात्मनः क्षेमं वीक्ष्य ते मेनिरेऽमरम्‌
गाः सन्निवर्त्य सायाह्ने सहरामो जनार्दनः
वेणुं विरणयन्गोष्ठमगाद्गोपैरभिष्टुतः
गोपीनां परमानन्द आसीद्गोविन्ददर्शने
क्षणं युगशतमिव यासां येन विनाभवत्‌
श्रीशुक उवाच
तयोस्तदद्भुतं कर्म दावाग्नेर्मोक्षमात्मनः
गोपाः स्त्रीभ्यः समाचख्युः प्रलम्बवधमेव च
गोपवृद्धाश्च गोप्यश्च तदुपाकर्ण्य विस्मिताः
मेनिरे देवप्रवरौ कृष्णरामौ व्रजं गतौ
ततः प्रावर्तत प्रावृट्सर्वसत्त्वसमुद्भवा
विद्योतमानपरिधिर्विस्फूर्जितनभस्तला
सान्द्रनीलाम्बुदैर्व्योम सविद्युत्स्तनयित्नुभिः
अस्पष्टज्योतिराच्छन्नं ब्रह्मेव सगुणं बभौ
अष्टौ मासान्निपीतं यद्भूम्याश्चोदमयं वसु
स्वगोभिर्मोक्तुमारेभे पर्जन्यः काल आगते
तडिद्वन्तो महामेघाश्चण्ड श्वसन वेपिताः
प्रीणनं जीवनं ह्यस्य मुमुचुः करुणा इव
तपःकृशा देवमीढा आसीद्वर्षीयसी मही
यथैव काम्यतपसस्तनुः सम्प्राप्य तत्फलम्‌
निशामुखेषु खद्योतास्तमसा भान्ति न ग्रहाः
यथा पापेन पाषण्डा न हि वेदाः कलौ युगे
श्रुत्वा पर्जन्यनिनदं मण्डुकाः ससृजुर्गिरः
तूष्णीं शयानाः प्राग्यद्वद्ब्राह्मणा नियमात्यये
आसन्नुत्पथगामिन्यः क्षुद्रनद्योऽनुशुष्यतीः
पुंसो यथास्वतन्त्रस्य देहद्रविण सम्पदः
हरिता हरिभिः शष्पैरिन्द्रगोपैश्च लोहिता
उच्छिलीन्ध्रकृतच्छाया नृणां श्रीरिव भूरभूत्‌
क्षेत्राणि शष्यसम्पद्भिः कर्षकाणां मुदं ददुः
मानिनामनुतापं वै दैवाधीनमजानताम्‌
जलस्थलौकसः सर्वे नववारिनिषेवया
अबिभ्रन्रुचिरं रूपं यथा हरिनिषेवया
सरिद्भिः सङ्गतः सिन्धुश्चुक्षोभ श्वसनोर्मिमान्‌
अपक्वयोगिनश्चित्तं कामाक्तं गुणयुग्यथा
गिरयो वर्षधाराभिर्हन्यमाना न विव्यथुः
अभिभूयमाना व्यसनैर्यथाधोक्षजचेतसः
मार्गा बभूवुः सन्दिग्धास्तृणैश्छन्ना ह्यसंस्कृताः
नाभ्यस्यमानाः श्रुतयो द्विजैः कालेन चाहताः
लोकबन्धुषु मेघेषु विद्युतश्चलसौहृदाः
स्थैर्यं न चक्रुः कामिन्यः पुरुषेषु गुणिष्विव
धनुर्वियति माहेन्द्रं निर्गुणं च गुणिन्यभात्‌
व्यक्ते गुणव्यतिकरेऽगुणवान्पुरुषो यथा
न रराजोडुपश्छन्नः स्वज्योत्स्नाराजितैर्घनैः
अहंमत्या भासितया स्वभासा पुरुषो यथा
मेघागमोत्सवा हृष्टाः प्रत्यनन्दञ्छिखण्डिनः
गृहेषु तप्तनिर्विण्णा यथाच्युतजनागमे
पीत्वापः पादपाः पद्भिरासन्नानात्ममूर्तयः
प्राक्क्षामास्तपसा श्रान्ता यथा कामानुसेवया
सरःस्वशान्तरोधःसु न्यूषुरङ्गापि सारसाः
गृहेष्वशान्तकृत्येषु ग्राम्या इव दुराशयाः
जलौघैर्निरभिद्यन्त सेतवो वर्षतीश्वरे
पाषण्डिनामसद्वादैर्वेदमार्गाः कलौ यथा
व्यमुञ्चन्वायुभिर्नुन्ना भूतेभ्यश्चामृतं घनाः
यथाशिषो विश्पतयः काले काले द्विजेरिताः
एवं वनं तद्वर्षिष्ठं पक्वखर्जुरजम्बुमत्‌
गोगोपालैर्वृतो रन्तुं सबलः प्राविशद्धरिः
धेनवो मन्दगामिन्य ऊधोभारेण भूयसा
ययुर्भगवताहूता द्रुतं प्रीत्या स्नुतस्तनाः
वनौकसः प्रमुदिता वनराजीर्मधुच्युतः
जलधारा गिरेर्नादादासन्ना ददृशे गुहाः
क्वचिद्वनस्पतिक्रोडे गुहायां चाभिवर्षति
निर्विश्य भगवान्रेमे कन्दमूलफलाशनः
दध्योदनं समानीतं शिलायां सलिलान्तिके
सम्भोजनीयैर्बुभुजे गोपैः सङ्कर्षणान्वितः
शाद्वलोपरि संविश्य चर्वतो मीलितेक्षणान्‌
तृप्तान्वृषान्वत्सतरान्गाश्च स्वोधोभरश्रमाः
प्रावृट्श्रियं च तां वीक्ष्य सर्वकालसुखावहाम्‌
भगवान्पूजयां चक्रे आत्मशक्त्युपबृंहिताम्‌
एवं निवसतोस्तस्मिन्रामकेशवयोर्व्रजे
शरत्समभवद्व्यभ्रा स्वच्छाम्ब्वपरुषानिला
शरदा नीरजोत्पत्त्या नीराणि प्रकृतिं ययुः
भ्रष्टानामिव चेतांसि पुनर्योगनिषेवया
व्योम्नोऽब्भ्रं भूतशाबल्यं भुवः पङ्कमपां मलम्‌
शरज्जहाराश्रमिणां कृष्णे भक्तिर्यथाशुभम्‌
सर्वस्वं जलदा हित्वा विरेजुः शुभ्रवर्चसः
यथा त्यक्तैषणाः शान्ता मुनयो मुक्तकिल्बिषाः
गिरयो मुमुचुस्तोयं क्वचिन्न मुमुचुः शिवम्‌
यथा ज्ञानामृतं काले ज्ञानिनो ददते न वा
नैवाविदन्क्षीयमाणं जलं गाधजलेचराः
यथायुरन्वहं क्षय्यं नरा मूढाः कुटुम्बिनः
गाधवारिचरास्तापमविन्दञ्छरदर्कजम्‌
यथा दरिद्रः कृपणः कुटुम्ब्यविजितेन्द्रियः
शनैः शनैर्जहुः पङ्कं स्थलान्यामं च वीरुधः
यथाहंममतां धीराः शरीरादिष्वनात्मसु
निश्चलाम्बुरभूत्तूष्णीं समुद्रः शरदागमे
आत्मन्युपरते सम्यङ्मुनिर्व्युपरतागमः
केदारेभ्यस्त्वपोऽगृह्णन्कर्षका दृढसेतुभिः
यथा प्राणैः स्रवज्ज्ञानं तन्निरोधेन योगिनः
शरदर्कांशुजांस्तापान्भूतानामुडुपोऽहरत्‌
देहाभिमानजं बोधो मुकुन्दो व्रजयोषिताम्‌
खमशोभत निर्मेघं शरद्विमलतारकम्‌
सत्त्वयुक्तं यथा चित्तं शब्दब्रह्मार्थदर्शनम्‌
अखण्डमण्डलो व्योम्नि रराजोडुगणैः शशी
यथा यदुपतिः कृष्णो वृष्णिचक्रावृतो भुवि
आश्लिष्य समशीतोष्णं प्रसूनवनमारुतम्‌
जनास्तापं जहुर्गोप्यो न कृष्णहृतचेतसः
गावो मृगाः खगा नार्यः पुष्पिण्यः शरदाभवन्‌
अन्वीयमानाः स्ववृषैः फलैरीशक्रिया इव
उदहृष्यन्वारिजानि सूर्योत्थाने कुमुद्विना
राज्ञा तु निर्भया लोका यथा दस्यून्विना नृप
पुरग्रामेष्वाग्रयणैरिन्द्रियैश्च महोत्सवैः
बभौ भूः पक्वशष्याढ्या कलाभ्यां नितरां हरेः
वणिङ्मुनिनृपस्नाता निर्गम्यार्थान्प्रपेदिरे
वर्षरुद्धा यथा सिद्धाः स्वपिण्डान्काल आगते
श्रीशुक उवाच
इत्थं शरत्स्वच्छजलं पद्माकरसुगन्धिना
न्यविशद्वायुना वातं स गोगोपालकोऽच्युतः
कुसुमितवनराजिशुष्मिभृङ्ग द्विजकुलघुष्टसरःसरिन्महीध्रम्‌
मधुपतिरवगाह्य चारयन्गाः सहपशुपालबलश्चुकूज वेणुम्‌
तद्व्रजस्त्रिय आश्रुत्य वेणुगीतं स्मरोदयम्‌
काश्चित्परोक्षं कृष्णस्य स्वसखीभ्योऽन्ववर्णयन्‌
तद्वर्णयितुमारब्धाः स्मरन्त्यः कृष्णचेष्टितम्‌
नाशकन्स्मरवेगेन विक्षिप्तमनसो नृप
बर्हापीडं नटवरवपुः कर्णयोः कर्णिकारं
बिभ्रद्वासः कनककपिशं वैजयन्तीं च मालाम्‌
रन्ध्रान्वेणोरधरसुधयापूरयन्गोपवृन्दैर्‌
वृन्दारण्यं स्वपदरमणं प्राविशद्गीतकीर्तिः
इति वेणुरवं राजन्सर्वभूतमनोहरम्‌
श्रुत्वा व्रजस्त्रियः सर्वा वर्णयन्त्योऽभिरेभिरे
श्रीगोप्य ऊचुः
अक्षण्वतां फलमिदं न परं विदामः
सख्यः पशूननविवेशयतोर्वयस्यैः
वक्त्रं व्रजेशसुतयोरनवेणुजुष्टं
यैर्वा निपीतमनुरक्तकटाक्षमोक्षम्‌
चूतप्रवालबर्हस्तबकोत्पलाब्ज मालानुपृक्तपरिधानविचित्रवेशौ
मध्ये विरेजतुरलं पशुपालगोष्ठ्यां रङ्गे यथा नटवरौ क्वच गायमानौ
गोप्यः किमाचरदयं कुशलं स्म वेणुर्‌
दामोदराधरसुधामपि गोपिकानाम्‌
भुङ्क्ते स्वयं यदवशिष्टरसं ह्रदिन्यो
हृष्यत्त्वचोऽश्रु मुमुचुस्तरवो यथार्यः
वृन्दावनं सखि भुवो वितनोति कीऋतिं
यद्देवकीसुतपदाम्बुजलब्धलक्ष्मि
गोविन्दवेणुमनु मत्तमयूरनृत्यं
प्रेक्ष्याद्रिसान्ववरतान्यसमस्तसत्त्वम्‌
धन्याः स्म मूढगतयोऽपि हरिण्य एता
या नन्दनन्दनमुपात्तविचित्रवेशम्‌
आकर्ण्य वेणुरणितं सहकृष्णसाराः
पूजां दधुर्विरचितां प्रणयावलोकैः
कृष्णं निरीक्ष्य वनितोत्सवरूपशीलं
श्रुत्वा च तत्क्वणितवेणुविविक्तगीतम्‌
देव्यो विमानगतयः स्मरनुन्नसारा
भ्रश्यत्प्रसूनकबरा मुमुहुर्विनीव्यः
गावश्च कृष्णमुखनिर्गतवेणुगीत
पीयूषमुत्तभितकर्णपुटैः पिबन्त्यः
शावाः स्नुतस्तनपयःकवलाः स्म तस्थुर्‌
गोविन्दमात्मनि दृशाश्रुकलाः स्पृशन्त्यः
प्रायो बताम्ब विहगा मुनयो वनेऽस्मिन्‌
कृष्णेक्षितं तदुदितं कलवेणुगीतम्‌
आरुह्य ये द्रुमभुजान्रुचिरप्रवालान्‌
शृण्वन्ति मीलितदृशो विगतान्यवाचः
नद्यस्तदा तदुपधार्य मुकुन्दगीतम्‌
आवर्तलक्षितमनोभवभग्नवेगाः
आलिङ्गनस्थगितमूर्मिभुजैर्मुरारेर्‌
गृह्णन्ति पादयुगलं कमलोपहाराः
दृष्ट्वातपे व्रजपशून्सह रामगोपैः
सञ्चारयन्तमनु वेणुमुदीरयन्तम्‌
प्रेमप्रवृद्ध उदितः कुसुमावलीभिः
सख्युर्व्यधात्स्ववपुषाम्बुद आतपत्रम्‌
पूर्णाः पुलिन्द्य उरुगायपदाब्जराग
श्रीकुङ्कुमेन दयितास्तनमण्डितेन
तद्दर्शनस्मररुजस्तृणरूषितेन
लिम्पन्त्य आननकुचेषु जहुस्तदाधिम्‌
हन्तायमद्रिरबला हरिदासवर्यो
यद्रामकृष्णचरणस्परशप्रमोदः
मानं तनोति सहगोगणयोस्तयोर्यत्‌
पानीयसूयवसकन्दरकन्दमूलैः
गा गोपकैरनुवनं नयतोरुदार
वेणुस्वनैः कलपदैस्तनुभृत्सु सख्यः
अस्पन्दनं गतिमतां पुलकस्तरुणां
निर्योगपाशकृतलक्षणयोर्विचित्रम्‌
एवंविधा भगवतो या वृन्दावनचारिणः
वर्णयन्त्यो मिथो गोप्यः क्रीडास्तन्मयतां ययुः
श्रीशुक उवाच
हेमन्ते प्रथमे मासि नन्दव्रजकमारिकाः
चेरुर्हविष्यं भुञ्जानाः कात्यायन्यर्चनव्रतम्‌
आप्लुत्याम्भसि कालिन्द्या जलान्ते चोदितेऽरुणे
कृत्वा प्रतिकृतिं देवीमानर्चुर्नृप सैकतीम्‌
गन्धैर्माल्यैः सुरभिभिर्बलिभिर्धूपदीपकैः
उच्चावचैश्चोपहारैः प्रवालफलतण्डुलैः
कात्यायनि महामाये महायोगिन्यधीश्वरि
नन्दगोपसुतं देवि पतिं मे कुरु ते नमः
इति मन्त्रं जपन्त्यस्ताः पूजां चक्रुः कमारिकाः
एवं मासं व्रतं चेरुः कुमार्यः कृष्णचेतसः
भद्रकालीं समानर्चुर्भूयान्नन्दसुतः पतिः
ऊषस्युत्थाय गोत्रैः स्वैरन्योन्याबद्धबाहवः
कृष्णमुच्चैर्जगुर्यान्त्यः कालिन्द्यां स्नातुमन्वहम्‌
नद्याः कदाचिदागत्य तीरे निक्षिप्य पूर्ववत्‌
वासांसि कृष्णं गायन्त्यो विजह्रुः सलिले मुदा
भगवांस्तदभिप्रेत्य कृष्नो योगेश्वरेश्वरः
वयस्यैरावृतस्तत्र गतस्तत्कर्मसिद्धये
तासां वासांस्युपादाय नीपमारुह्य सत्वरः
हसद्भिः प्रहसन्बालैः परिहासमुवाच ह
अत्रागत्याबलाः कामं स्वं स्वं वासः प्रगृह्यताम्‌
सत्यं ब्रवाणि नो नर्म यद्यूयं व्रतकर्शिताः
न मयोदितपूर्वं वा अनृतं तदिमे विदुः
एकैकशः प्रतीच्छध्वं सहैवेति सुमध्यमाः
तस्य तत्क्ष्वेलितं दृष्ट्वा गोप्यः प्रेमपरिप्लुताः
व्रीडिताः प्रेक्ष्य चान्योन्यं जातहासा न निर्ययुः
एवं ब्रुवति गोविन्दे नर्मणाक्षिप्तचेतसः
आकण्ठमग्नाः शीतोदे वेपमानास्तमब्रुवन्‌
मानयं भोः कृथास्त्वां तु नन्दगोपसुतं प्रियम्‌
जानीमोऽङ्ग व्रजश्लाघ्यं देहि वासांसि वेपिताः
श्यामसुन्दर ते दास्यः करवाम तवोदितम्‌
देहि वासांसि धर्मज्ञ नो चेद्राज्ञे ब्रुवाम हे
श्रीभगवानुवाच
भवत्यो यदि मे दास्यो मयोक्तं वा करिष्यथ
अत्रागत्य स्ववासांसि प्रतीच्छत शुचिस्मिताः
नो चेन्नाहं प्रदास्ये किं क्रुद्धो राजा करिष्यति
ततो जलाशयात्सर्वा दारिकाः शीतवेपिताः
पाणिभ्यां योनिमाच्छाद्य प्रोत्तेरुः शीतकर्शिताः
भगवानाहता वीक्ष्य शुद्ध भावप्रसादितः
स्कन्धे निधाय वासांसि प्रीतः प्रोवाच सस्मितम्‌
यूयं विवस्त्रा यदपो धृतव्रता व्यगाहतैतत्तदु देवहेलनम्‌
बद्ध्वाञ्जलिं मूर्ध्न्यपनुत्तयेऽंहसः कृत्वा नमोऽधोवसनं प्रगृह्यताम्‌
इत्यच्युतेनाभिहितं व्रजाबला मत्वा विवस्त्राप्लवनं व्रतच्युतिम्‌
तत्पूर्तिकामास्तदशेषकर्मणां साक्षात्कृतं नेमुरवद्यमृग्यतः
तास्तथावनता दृष्ट्वा भगवान्देवकीसुतः
वासांसि ताभ्यः प्रायच्छत्करुणस्तेन तोषितः
दृढं प्रलब्धास्त्रपया च हापिताः
प्रस्तोभिताः क्रीडनवच्च कारिताः
वस्त्राणि चैवापहृतान्यथाप्यमुं
ता नाभ्यसूयन्प्रियसङ्गनिर्वृताः
परिधाय स्ववासांसि प्रेष्ठसङ्गमसज्जिताः
गृहीतचित्ता नो चेलुस्तस्मिन्लज्जायितेक्षणाः
तासां विज्ञाय भगवान्स्वपादस्पर्शकाम्यया
धृतव्रतानां सङ्कल्पमाह दामोदरोऽबलाः
सङ्कल्पो विदितः साध्व्यो भवतीनां मदर्चनम्‌
मयानुमोदितः सोऽसौ सत्यो भवितुमर्हति
न मय्यावेशितधियां कामः कामाय कल्पते
भर्जिता क्वथिता धानाः प्रायो बीजाय नेशते
याताबला व्रजं सिद्धा मयेमा रंस्यथा क्षपाः
यदुद्दिश्य व्रतमिदं चेरुरार्यार्चनं सतीः
श्रीशुक उवाच
इत्यादिष्टा भगवता लब्धकामाः कुमारिकाः
ध्यायन्त्यस्तत्पदाम्भोजम्कृच्छ्रान्निर्विविशुर्व्रजम्‌
अथ गोपैः परिवृतो भगवान्देवकीसुतः
वृन्दावनाद्गतो दूरं चारयन्गाः सहाग्रजः
निदघार्कातपे तिग्मे छायाभिः स्वाभिरात्मनः
आतपत्रायितान्वीक्ष्य द्रुमानाह व्रजौकसः
हे स्तोककृष्ण हे अंशो श्रीदामन्सुबलार्जुन
विशाल वृषभौजस्विन्देवप्रस्थ वरूथप
पश्यतैतान्महाभागान्परार्थैकान्तजीवितान्‌
वातवर्षातपहिमान्सहन्तो वारयन्ति नः
अहो एषां वरं जन्म सर्व प्राण्युपजीवनम्‌
सुजनस्येव येषां वै विमुखा यान्ति नार्थिनः
पत्रपुष्पफलच्छाया मूलवल्कलदारुभिः
गन्धनिर्यासभस्मास्थि तोक्मैः कामान्वितन्वते
एतावज्जन्मसाफल्यं देहिनामिह देहिषु
प्राणैरर्थैर्धिया वाचा श्रेयआचरणं सदा
इति प्रवालस्तबक फलपुष्पदलोत्करैः
तरूणां नम्रशाखानां मध्यतो यमुनां गतः
तत्र गाः पाययित्वापः सुमृष्टाः शीतलाः शिवाः
ततो नृप स्वयं गोपाः कामं स्वादु पपुर्जलम्‌
तस्या उपवने कामं चारयन्तः पशून्नृप
कृष्णरामावुपागम्य क्षुधार्ता इदमब्रवन्‌
श्रीगोप ऊचुः
राम राम महाबाहो कृष्ण दुष्टनिबर्हण
एषा वै बाधते क्षुन्नस्तच्छान्तिं कर्तुमर्हथः
श्रीशुक उवाच
इति विज्ञापितो गोपैर्भगवान्देवकीसुतः
भक्ताया विप्रभार्यायाः प्रसीदन्निदमब्रवीत्‌
प्रयात देवयजनं ब्राह्मणा ब्रह्मवादिनः
सत्रमाङ्गिरसं नाम ह्यासते स्वर्गकाम्यया
तत्र गत्वौदनं गोपा याचतास्मद्विसर्जिताः
कीर्तयन्तो भगवत आर्यस्य मम चाभिधाम्‌
इत्यादिष्टा भगवता गत्वा याचन्त ते तथा
कृताञ्जलिपुटा विप्रान्दण्डवत्पतिता भुवि
हे भूमिदेवाः शृणुत कृष्णस्यादेशकारिणः
प्राप्ताञ्जानीत भद्रं वो गोपान्नो रामचोदितान्‌
गाश्चारयन्तावविदूर ओदनं रामाच्युतौ वो लषतो बुभुक्षितौ
तयोर्द्विजा ओदनमर्थिनोर्यदि श्रद्धा च वो यच्छत धर्मवित्तमाः
दीक्षायाः पशुसंस्थायाः सौत्रामण्याश्च सत्तमाः
अन्यत्र दीक्षितस्यापि नान्नमश्नन्हि दुष्यति
इति ते भगवद्याच्ञां शृण्वन्तोऽपि न शुश्रुवुः
क्षुद्राशा भूरिकर्माणो बालिशा वृद्धमानिनः
देशः कालः पृथग्द्रव्यं मन्त्रतन्त्रर्त्विजोऽग्नयः
देवता यजमानश्च क्रतुर्धर्मश्च यन्मयः
तं ब्रह्म परमं साक्षाद्भगवन्तमधोक्षजम्‌
मनुष्यदृष्ट्या दुष्प्रज्ञा मर्त्यात्मानो न मेनिरे
न ते यदोमिति प्रोचुर्न नेति च परन्तप
गोपा निराशाः प्रत्येत्य तथोचुः कृष्णरामयोः
तदुपाकर्ण्य भगवान्प्रहस्य जगदीश्वरः
व्याजहार पुनर्गोपान्दर्शयन्लौकिकीं गतिम्‌
मां ज्ञापयत पत्नीभ्यः ससङ्कर्षणमागतम्‌
दास्यन्ति काममन्नं वः स्निग्धा मय्युषिता धिया
गत्वाथ पत्नीशालायां दृष्ट्वासीनाः स्वलङ्कृताः
नत्वा द्विजसतीर्गोपाः प्रश्रिता इदमब्रुवन्‌
नमो वो विप्रपत्नीभ्यो निबोधत वचांसि नः
इतोऽविदूरे चरता कृष्णेनेहेषिता वयम्‌
गाश्चारयन्स गोपालैः सरामो दूरमागतः
बुभुक्षितस्य तस्यान्नं सानुगस्य प्रदीयताम्‌
श्रुत्वाच्युतमुपायातं नित्यं तद्दर्शनोत्सुकाः
तत्कथाक्षिप्तमनसो बभूवुर्जातसम्भ्रमाः
चतुर्विधं बहुगुणमन्नमादाय भाजनैः
अभिसस्रुः प्रियं सर्वाः समुद्रमिव निम्नगाः
निषिध्यमानाः पतिभिर्भ्रातृभिर्बन्धुभिः सुतैः
भगवत्युत्तमश्लोके दीर्घश्रुत धृताशयाः
यमुनोपवनेऽशोक नवपल्लवमण्डिते
विचरन्तं वृतं गोपैः साग्रजं ददृशुः स्त्रियः
श्यामं हिरण्यपरिधिं वनमाल्यबर्ह
धातुप्रवालनटवेषमनव्रतांसे
विन्यस्तहस्तमितरेण धुनानमब्जं
कर्णोत्पलालककपोलमुखाब्जहासम्‌
प्रायःश्रुतप्रियतमोदयकर्णपूरैर्‌
यस्मिन्निमग्नमनसस्तमथाक्षिरन्द्रैः
अन्तः प्रवेश्य सुचिरं परिरभ्य तापं
प्राज्ञं यथाभिमतयो विजहुर्नरेन्द्र
तास्तथा त्यक्तसर्वाशाः प्राप्ता आत्मदिदृक्षया
विज्ञायाखिलदृग्द्रष्टा प्राह प्रहसिताननः
स्वागतं वो महाभागा आस्यतां करवाम किम्‌
यन्नो दिदृक्षया प्राप्ता उपपन्नमिदं हि वः
नन्वद्धा मयि कुर्वन्ति कुशलाः स्वार्थदर्शिनः
अहैतुक्यव्यवहितां भक्तिमात्मप्रिये यथा
प्राणबुद्धिमनःस्वात्म दारापत्यधनादयः
यत्सम्पर्कात्प्रिया आसंस्ततः को न्वपरः प्रियः
तद्यात देवयजनं पतयो वो द्विजातयः
स्वसत्रं पारयिष्यन्ति युष्माभिर्गृहमेधिनः
श्रीपत्न्य ऊचुः
मैवं विभोऽर्हति भवान्गदितुं न्र्शंसं
सत्यं कुरुष्व निगमं तव पदमूलम्‌
प्राप्ता वयं तुलसिदाम पदावसृष्टं
केशैर्निवोढुमतिलङ्घ्य समस्तबन्धून्‌
गृह्णन्ति नो न पतयः पितरौ सुता वा
न भ्रातृबन्धुसुहृदः कुत एव चान्ये
तस्माद्भवत्प्रपदयोः पतितात्मनां नो
नान्या भवेद्गतिररिन्दम तद्विधेहि
श्रीभगवानुवाच
पतयो नाभ्यसूयेरन्पितृभ्रातृसुतादयः
लोकाश्च वो मयोपेता देवा अप्यनुमन्वते
न प्रीतयेऽनुरागाय ह्यङ्गसङ्गो नृणामिह
तन्मनो मयि युञ्जाना अचिरान्मामवाप्स्यथ
श्रवणाद्दर्शनाद्ध्यानान्मयि भावोऽनुकीर्तनात्‌
न तथा सन्निकर्षेण प्रतियात ततो गृहान्‌
श्रीशुक उवाच
इत्युक्ता द्विजपत्न्यस्ता यज्ञवाटं पुनर्गताः
ते चानसूयवस्ताभिः स्त्रीभिः सत्रमपारयन्‌
तत्रैका विधृता भर्त्रा भगवन्तं यथाश्रुतम्‌
हृडोपगुह्य विजहौ देहं कर्मानुबन्धनम्‌
भगवानपि गोविन्दस्तेनैवान्नेन गोपकान्‌
चतुर्विधेनाशयित्वा स्वयं च बुभुजे प्रभुः
एवं लीलानरवपुर्न्र्लोकमनुशीलयन्‌
रेमे गोगोपगोपीनां रमयन्रूपवाक्कृतैः
अथानुस्मृत्य विप्रास्ते अन्वतप्यन्कृतागसः
यद्विश्वेश्वरयोर्याच्ञामहन्म नृविडम्बयोः
दृष्ट्वा स्त्रीणां भगवति कृष्णे भक्तिमलौकिकीम्‌
आत्मानं च तया हीनमनुतप्ता व्यगर्हयन्‌
धिग्जन्म नस्त्रिवृद्यत्तद्धिग्व्रतं धिग्बहुज्ञताम्‌
धिक्कुलं धिक्क्रियादाक्ष्यं विमुखा ये त्वधोक्षजे
नूनं भगवतो माया योगिनामपि मोहिनी
यद्वयं गुरवो नृणां स्वार्थे मुह्यामहे द्विजाः
अहो पश्यत नारीणामपि कृष्णे जगद्गुरौ
दुरन्तभावं योऽविध्यन्मृत्युपाशान्गृहाभिधान्‌
नासां द्विजातिसंस्कारो न निवासो गुरावपि
न तपो नात्ममीमांसा न शौचं न क्रियाः शुभाः
तथापि ह्युत्तमःश्लोके कृष्णे योगेश्वरेश्वरे
भक्तिर्दृढा न चास्माकं संस्कारादिमतामपि
ननु स्वार्थविमूढानां प्रमत्तानां गृहेहया
अहो नः स्मारयामास गोपवाक्यैः सतां गतिः
अन्यथा पूर्णकामस्य कैवल्याद्यशिषां पतेः
ईशितव्यैः किमस्माभिरीशस्यैतद्विडम्बनम्‌
हित्वान्यान्भजते यं श्रीः पादस्पर्शाशयासकृत्‌
स्वात्मदोषापवर्गेण तद्याच्ञा जनमोहिनी
देशः कालः पृथग्द्रव्यं मन्त्रतन्त्रर्त्विजोऽग्नयः
देवता यजमानश्च क्रतुर्धर्मश्च यन्मयः
स एव भगवान्साक्षाद्विष्णुर्योगेश्वरेश्वरः
जातो यदुष्वित्याशृण्म ह्यपि मूढा न विद्महे
तस्मै नमो भगवते कृष्णायाकुण्ठमेधसे
यन्मायामोहितधियो भ्रमामः कर्मवर्त्मसु
स वै न आद्यः पुरुषः स्वमायामोहितात्मनाम्‌
अविज्ञतानुभावानां क्षन्तुमर्हत्यतिक्रमम्‌
इति स्वाघमनुस्मृत्य कृष्णे ते कृतहेलनाः
दिदृक्षवो व्रजमथ कंसाद्भीता न चाचलन्‌
श्रीशुक उवाच
भगवानपि तत्रैव बलदेवेन संयुतः
अपश्यन्निवसन्गोपानिन्द्रयागकृतोद्यमान्‌
तदभिज्ञोऽपि भगवान्सर्वात्मा सर्वदर्शनः
प्रश्रयावनतोऽपृच्छद्वृद्धान्नन्दपुरोगमान्‌
कथ्यतां मे पितः कोऽयं सम्भ्रमो व उपागतः
किं फलं कस्य वोद्देशः केन वा साध्यते मखः
एतद्ब्रूहि महान्कामो मह्यं शुश्रूषवे पितः
न हि गोप्यं हि सधूनां कृत्यं सर्वात्मनामिह
अस्त्यस्वपरदृष्टीनाममित्रोदास्तविद्विषाम्‌
उदासीनोऽरिवद्वर्ज्य
आत्मवत्सुहृदुच्यते
ज्ञत्वाज्ञात्वा च कर्माणि जनोऽयमनुतिष्ठति
विदुषः कर्मसिद्धिः स्याद्यथा नाविदुषो भवेत्‌
तत्र तावत्क्रियायोगो भवतां किं विचारितः
अथ वा लौकिकस्तन्मे पृच्छतः साधु भण्यताम्‌
श्रीनन्द उवाच
पर्जन्यो भगवानिन्द्रो मेघास्तस्यात्ममूर्तयः
तेऽभिवर्षन्ति भूतानां प्रीणनं जीवनं पयः
तं तात वयमन्ये च वार्मुचां पतिमीश्वरम्‌
द्रव्यैस्तद्रेतसा सिद्धैर्यजन्ते क्रतुभिर्नराः
तच्छेषेणोपजीवन्ति त्रिवर्गफलहेतवे
पुंसां पुरुषकाराणां पर्जन्यः फलभावनः
य एनं विसृजेद्धर्मं परम्पर्यागतं नरः
कामाद्द्वेषाद्भयाल्लोभात्स वै नाप्नोति शोभनम्‌
श्रीशुक उवाच
वचो निशम्य नन्दस्य तथान्येषां व्रजौकसाम्‌
इन्द्राय मन्युं जनयन्पितरं प्राह केशवः
श्रीभगवानुवाच
कर्मणा जायते जन्तुः कर्मणैव प्रलीयते
सुखं दुःखं भयं क्षेमं कर्मणैवाभिपद्यते
अस्ति चेदीश्वरः कश्चित्फलरूप्यन्यकर्मणाम्‌
कर्तारं भजते सोऽपि न ह्यकर्तुः प्रभुर्हि सः
किमिन्द्रेणेह भूतानां स्वस्वकर्मानुवर्तिनाम्‌
अनीशेनान्यथा कर्तुं स्वभावविहितं नृणाम्‌
स्वभावतन्त्रो हि जनः स्वभावमनुवर्तते
स्वभावस्थमिदं सर्वं सदेवासुरमानुषम्‌
देहानुच्चावचाञ्जन्तुः प्राप्योत्सृजति कर्मणा
शत्रुर्मित्रमुदासीनः कर्मैव गुरुरीश्वरः
तस्मात्सम्पूजयेत्कर्म स्वभावस्थः स्वकर्मकृत्‌
अन्जसा येन वर्तेत तदेवास्य हि दैवतम्‌
आजीव्यैकतरं भावं यस्त्वन्यमुपजीवति
न तस्माद्विन्दते क्षेमं जारान्नार्यसती यथा
वर्तेत ब्रह्मणा विप्रो राजन्यो रक्षया भुवः
वैश्यस्तु वार्तया जीवेच्छूद्रस्तु द्विजसेवया
कृषिवाणिज्यगोरक्षा कुसीदं तूर्यमुच्यते
वार्ता चतुर्विधा तत्र वयं गोवृत्तयोऽनिशम्‌
सत्त्वं रजस्तम इति स्थित्युत्पत्त्यन्तहेतवः
रजसोत्पद्यते विश्वमन्योन्यं विविधं जगत्‌
रजसा चोदिता मेघा वर्षन्त्यम्बूनि सर्वतः
प्रजास्तैरेव सिध्यन्ति महेन्द्रः किं करिष्यति
न नः पुरोजनपदा न ग्रामा न गृहा वयम्‌
वनौकसस्तात नित्यं वनशैलनिवासिनः
तस्माद्गवां ब्राह्मणानामद्रेश्चारभ्यतां मखः
य इन्द्रयागसम्भारास्तैरयं साध्यतां मखः
पच्यन्तां विविधाः पाकाः सूपान्ताः पायसादयः
संयावापूपशष्कुल्यः सर्वदोहश्च गृह्यताम्‌
हूयन्तामग्नयः सम्यग्ब्राह्मणैर्ब्रह्मवादिभिः
अन्नं बहुगुणं तेभ्यो देयं वो धेनुदक्षिणाः
अन्येभ्यश्चाश्वचाण्डाल पतितेभ्यो यथार्हतः
यवसं च गवां दत्त्वा गिरये दीयतां बलिः
स्वलङ्कृता भुक्तवन्तः स्वनुलिप्ताः सुवाससः
प्रदक्षिणां च कुरुत गोविप्रानलपर्वतान्‌
एतन्मम मतं तात क्रियतां यदि रोचते
अयं गोब्राह्मणाद्रीणां मह्यं च दयितो मखः
श्रीशुक उवाच
कालात्मना भगवता शक्रदर्पजिघांसया
प्रोक्तं निशम्य नन्दाद्याः साध्वगृह्णन्त तद्वचः
तथा च व्यदधुः सर्वं यथाह मधुसूदनः
वाचयित्वा स्वस्त्ययनं तद्द्रव्येण गिरिद्विजान्‌
उपहृत्य बलीन्सम्यगादृता यवसं गवाम्‌
गोधनानि पुरस्कृत्य गिरिं चक्रुः प्रदक्षिणम्‌
अनांस्यनडुद्युक्तानि ते चारुह्य स्वलङ्कृताः
गोप्यश्च कृष्णवीर्याणि गायन्त्यः सद्विजाशिषः
कृष्णस्त्वन्यतमं रूपं गोपविश्रम्भणं गतः
शैलोऽस्मीति ब्रुवन्भूरि बलिमादद्बृहद्वपुः
तस्मै नमो व्रजजनैः सह चक्र आत्मनात्मने
अहो पश्यत शैलोऽसौ रूपी नोऽनुग्रहं व्यधात्‌
एषोऽवजानतो मर्त्यान्कामरूपी वनौकसः
हन्ति ह्यस्मै नमस्यामः शर्मणे आत्मनो गवाम्‌
इत्यद्रिगोद्विजमखं वासुदेवप्रचोदिताः
यथा विधाय ते गोपा सहकृष्णा व्रजं ययुः
श्रीशुक उवाच
इन्द्रस्तदात्मनः पूजां विज्ञाय विहतां नृप
गोपेभ्यः कृष्णनाथेभ्यो नन्दादिभ्यश्चुकोप ह
गणं सांवर्तकं नाम मेघानां चान्तकारीणाम्‌
इन्द्रः प्रचोदयत्क्रुद्धो वाक्यं चाहेशमान्युत
अहो श्रीमदमाहात्म्यं गोपानां काननौकसाम्‌
कृष्णं मर्त्यमुपाश्रित्य ये चक्रुर्देवहेलनम्‌
यथादृढैः कर्ममयैः क्रतुभिर्नामनौनिभैः
विद्यामान्वीक्षिकीं हित्वा तितीर्षन्ति भवार्णवम्‌
वाचालं बालिशं स्तब्धमज्ञं पण्डितमानिनम्‌
कृष्णं मर्त्यमुपाश्रित्य गोपा मे चक्रुरप्रियम्‌
एषां श्रियावलिप्तानां कृष्णेनाध्मापितात्मनाम्‌
धुनुत श्रीमदस्तम्भं पशून्नयत सङ्क्षयम्‌
अहं चैरावतं नागमारुह्यानुव्रजे व्रजम्‌
मरुद्गणैर्महावेगैर्नन्दगोष्ठजिघांसया
श्रीशुक उवाच
इत्थं मघवताज्ञप्ता मेघा निर्मुक्तबन्धनाः
नन्दगोकुलमासारैः पीडयामासुरोजसा
विद्योतमाना विद्युद्भिः स्तनन्तः स्तनयित्नुभिः
तीव्रैर्मरुद्गणैर्नुन्ना ववृषुर्जलशर्कराः
स्थूणास्थूला वर्षधारा मुञ्चत्स्वभ्रेष्वभीक्ष्णशः
जलौघैः प्लाव्यमाना भूर्नादृश्यत नतोन्नतम्‌
अत्यासारातिवातेन पशवो जातवेपनाः
गोपा गोप्यश्च शीतार्ता गोविन्दं शरणं ययुः
शिरः सुतांश्च कायेन प्रच्छाद्यासारपीडिताः
वेपमाना भगवतः पादमूलमुपाययुः
कृष्ण कृष्ण महाभाग त्वन्नाथं गोकुलं प्रभो
त्रातुमर्हसि देवान्नः कुपिताद्भक्तवत्सल
शिलावर्षातिवातेन हन्यमानमचेतनम्‌
निरीक्ष्य भगवान्मेने कुपितेन्द्रकृतं हरिः
अपर्त्वत्युल्बणं वर्षमतिवातं शिलामयम्‌
स्वयागे विहतेऽस्माभिरिन्द्रो नाशाय वर्षति
तत्र प्रतिविधिं सम्यगात्मयोगेन साधये
लोकेशमानिनां मौढ्याद्धनिष्ये श्रीमदं तमः
न हि सद्भावयुक्तानां सुराणामीशविस्मयः
मत्तोऽसतां मानभङ्गः प्रशमायोपकल्पते
तस्मान्मच्छरणं गोष्ठं मन्नाथं मत्परिग्रहम्‌
गोपाये स्वात्मयोगेन सोऽयं मे व्रत आहितः
इत्युक्त्वैकेन हस्तेन कृत्वा गोवर्धनाचलम्‌
दधार लीलया विष्णुश्छत्राकमिव बालकः
अथाह भगवान्गोपान्हेऽम्ब तात व्रजौकसः
यथोपजोषं विशत गिरिगर्तं सगोधनाः
न त्रास इह वः कार्यो मद्धस्ताद्रिनिपातनात्‌
वातवर्षभयेनालं तत्त्राणं विहितं हि वः
तथा निर्विविशुर्गर्तं कृष्णाश्वासितमानसः
यथावकाशं सधनाः सव्रजाः सोपजीविनः
क्षुत्तृड्व्यथां सुखापेक्षां हित्वा तैर्व्रजवासिभिः
वीक्ष्यमाणो दधाराद्रिं सप्ताहं नाचलत्पदात्‌
कृष्णयोगानुभावं तं निशम्येन्द्रोऽतिविस्मितः
निस्तम्भो भ्रष्टसङ्कल्पः स्वान्मेघान्सन्न्यवारयत्‌
खं व्यभ्रमुदितादित्यं वातवर्षं च दारुणम्‌
निशम्योपरतं गोपान्गोवर्धनधरोऽब्रवीत्‌
निर्यात त्यजत त्रासं गोपाः सस्त्रीधनार्भकाः
उपारतं वातवर्षं व्युदप्रायाश्च निम्नगाः
ततस्ते निर्ययुर्गोपाः स्वं स्वमादाय गोधनम्‌
शकटोढोपकरणं स्त्रीबालस्थविराः शनैः
भगवानपि तं शैलं स्वस्थाने पूर्ववत्प्रभुः
पश्यतां सर्वभूतानां स्थापयामास लीलया
तं प्रेमवेगान्निर्भृता व्रजौकसो
यथा समीयुः परिरम्भणादिभिः
गोप्यश्च सस्नेहमपूजयन्मुदा
दध्यक्षताद्भिर्युयुजुः सदाशिषः
यशोदा रोहिणी नन्दो रामश्च बलिनां वरः
कृष्णमालिङ्ग्य युयुजुराशिषः स्नेहकातराः
दिवि देवगणाः सिद्धाः साध्या गन्धर्वचारणाः
तुष्टुवुर्मुमुचुस्तुष्टाः पुष्पवर्षाणि पार्थिव
शङ्खदुन्दुभयो नेदुर्दिवि देवप्रचोदिताः
जगुर्गन्धर्वपतयस्तुम्बुरुप्रमुखा नृप
ततोऽनुरक्तैः पशुपैः परिश्रितो राजन्स्वगोष्ठं सबलोऽव्रजद्धरिः
तथाविधान्यस्य कृतानि गोपिका गायन्त्य ईयुर्मुदिता हृदिस्पृशः
श्रीशुक उवाच
एवंविधानि कर्माणि गोपाः कृष्णस्य वीक्ष्य ते
अतद्वीर्यविदः प्रोचुः समभ्येत्य सुविस्मिताः
बालकस्य यदेतानि कर्माण्यत्यद्भुतानि वै
कथमर्हत्यसौ जन्म ग्राम्येष्वात्मजुगुप्सितम्‌
यः सप्तहायनो बालः करेणैकेन लीलया
कथं बिभ्रद्गिरिवरं पुष्करं गजराडिव
तोकेनामीलिताक्षेण पूतनाया महौजसः
पीतः स्तनः सह प्राणैः कालेनेव वयस्तनोः
हिन्वतोऽधः शयानस्य मास्यस्य चरणावुदक्‌
अनोऽपतद्विपर्यस्तं रुदतः प्रपदाहतम्‌
एकहायन आसीनो ह्रियमाणो विहायसा
दैत्येन यस्तृणावर्तमहन्कण्ठग्रहातुरम्‌
क्वचिद्धैयङ्गवस्तैन्ये मात्रा बद्ध उदूखले
गच्छन्नर्जुनयोर्मध्ये बाहुभ्यां तावपातयत्‌
वने सञ्चारयन्वत्सान्सरामो बालकैर्वृतः
हन्तुकामं बकं दोर्भ्यां मुखतोऽरिमपाटयत्‌
वत्सेषु वत्सरूपेण प्रविशन्तं जिघांसया
हत्वा न्यपातयत्तेन कपित्थानि च लीलया
हत्वा रासभदैतेयं तद्बन्धूंश्च बलान्वितः
चक्रे तालवनं क्षेमं परिपक्वफलान्वितम्‌
प्रलम्बं घातयित्वोग्रं बलेन बलशालिना
अमोचयद्व्रजपशून्गोपांश्चारण्यवह्नितः
आशीविषतमाहीन्द्रं दमित्वा विमदं ह्रदात्‌
प्रसह्योद्वास्य यमुनां चक्रेऽसौ निर्विषोदकाम्‌
दुस्त्यजश्चानुरागोऽस्मिन्सर्वेषां नो व्रजौकसाम्‌
नन्द ते तनयेऽस्मासु तस्याप्यौत्पत्तिकः कथम्‌
क्व सप्तहायनो बालः क्व महाद्रिविधारणम्‌
ततो नो जायते शङ्का व्रजनाथ तवात्मजे
श्रीनन्द उवाच
श्रूयतां मे वचो गोपा व्येतु शङ्का च वोऽर्भके
एनम्कुमारमुद्दिश्य गर्गो मे यदुवाच ह
वर्णास्त्रयः किलास्यासन्गृह्णतोऽनुयुगं तनूः
शुक्लो रक्तस्तथा पीत इदानीं कृष्णतां गतः
प्रागयं वसुदेवस्य क्वचिज्जातस्तवात्मजः
वासुदेव इति श्रीमानभिज्ञाः सम्प्रचक्षते
बहूनि सन्ति नामानि रूपाणि च सुतस्य ते
गुण कर्मानुरूपाणि तान्यहं वेद नो जनाः
एष वः श्रेय आधास्यद्गोपगोकुलनन्दनः
अनेन सर्वदुर्गाणि यूयमञ्जस्तरिष्यथ
पुरानेन व्रजपते साधवो दस्युपीडिताः
अराजके रक्ष्यमाणा जिग्युर्दस्यून्समेधिताः
य एतस्मिन्महाभागे प्रीतिं कुर्वन्ति मानवाः
नारयोऽभिभवन्त्येतान्विष्णुपक्षानिवासुराः
तस्मान्नन्द कुमारोऽयं नारायणसमो गुणैः
श्रिया कीर्त्यानुभावेन तत्कर्मसु न विस्मयः
इत्यद्धा मां समादिश्य गर्गे च स्वगृहं गते
मन्ये नारायणस्यांशं कृष्णमक्लिष्टकारिणम्‌
इति नन्दवचः श्रुत्वा गर्गगीतं तं व्रजौकसः
मुदिता नन्दमानर्चुः कृष्णं च गतविस्मयाः
देवे वर्षति यज्ञविप्लवरुषा वज्रास्मवर्षानिलैः
सीदत्पालपशुस्त्रियात्मशरणं दृष्ट्वानुकम्प्युत्स्मयन्‌
उत्पाट्यैककरेण शैलमबलो लीलोच्छिलीन्ध्रं यथा
बिभ्रद्गोष्ठमपान्महेन्द्रमदभित्प्रीयान्न इन्द्रो गवाम्‌
श्रीशुक उवाच
गोवर्धने धृते शैले आसाराद्रक्षिते व्रजे
गोलोकादाव्रजत्कृष्णं सुरभिः शक्र एव च
विविक्त उपसङ्गम्य व्रीडीतः कृतहेलनः
पस्पर्श पादयोरेनं किरीटेनार्कवर्चसा
दृष्टश्रुतानुभावोऽस्य कृष्णस्यामिततेजसः
नष्टत्रिलोकेशमद इदमाह कृताञ्जलिः
इन्द्र उवाच
विशुद्धसत्त्वं तव धाम शान्तं तपोमयं ध्वस्तरजस्तमस्कम्‌
मायामयोऽयं गुणसम्प्रवाहो न विद्यते ते ग्रहणानुबन्धः
कुतो नु तद्धेतव ईश तत्कृता लोभादयो येऽबुधलिन्गभावाः
तथापि दण्डं भगवान्बिभर्ति धर्मस्य गुप्त्यै खलनिग्रहाय
पिता गुरुस्त्वं जगतामधीशो दुरत्ययः काल उपात्तदण्डः
हिताय चेच्छातनुभिः समीहसे मानं विधुन्वन्जगदीशमानिनाम्‌
ये मद्विधाज्ञा जगदीशमानिनस्त्वां वीक्ष्य कालेऽभयमाशु तन्मदम्‌
हित्वार्यमार्गं प्रभजन्त्यपस्मया ईहा खलानामपि तेऽनुशासनम्‌
स त्वं ममैश्वर्यमदप्लुतस्य कृतागसस्तेऽविदुषः प्रभावम्‌
क्षन्तुं प्रभोऽथार्हसि मूढचेतसो मैवं पुनर्भून्मतिरीश मेऽसती
तवावतारोऽयमधोक्षजेह भुवो भराणामुरुभारजन्मनाम्‌
चमूपतीनामभवाय देव भवाय युष्मच्चरणानुवर्तिनाम्‌
नमस्तुभ्यं भगवते पुरुषाय महात्मने
वासुदेवाय कृष्णाय सात्वतां पतये नमः
स्वच्छन्दोपात्तदेहाय विशुद्धज्ञानमूर्तये
सर्वस्मै सर्वबीजाय सर्वभूतात्मने नमः
मयेदं भगवन्गोष्ठ नाशायासारवायुभिः
चेष्टितं विहते यज्ञे मानिना तीव्रमन्युना
त्वयेशानुगृहीतोऽस्मि ध्वस्तस्तम्भो वृथोद्यमः
ईश्वरं गुरुमात्मानं त्वामहं शरणं गतः
श्रीशुक उवाच
एवं सङ्कीर्तितः कृष्णो मघोना भगवानमुम्‌
मेघगम्भीरया वाचा प्रहसन्निदमब्रवीत्‌
श्रीभगवानुवाच
मया तेऽकारि मघवन्मखभङ्गोऽनुगृह्णता
मदनुस्मृतये नित्यं मत्तस्येन्द्रश्रिया भृशम्‌
मामैश्वर्यश्रीमदान्धो दण्ड पाणिं न पश्यति
तं भ्रंशयामि सम्पद्भ्यो यस्य चेच्छाम्यनुग्रहम्‌
गम्यतां शक्र भद्रं वः क्रियतां मेऽनुशासनम्‌
स्थीयतां स्वाधिकारेषु युक्तैर्वः स्तम्भवर्जितैः
अथाह सुरभिः कृष्णमभिवन्द्य मनस्विनी
स्वसन्तानैरुपामन्त्र्य गोपरूपिणमीश्वरम्‌
सुरभिरुवाच
कृष्ण कृष्ण महायोगिन्विश्वात्मन्विश्वसम्भव
भवता लोकनाथेन सनाथा वयमच्युत
त्वं नः परमकं दैवं त्वं न इन्द्रो जगत्पते
भवाय भव गोविप्र देवानां ये च साधवः
इन्द्रं नस्त्वाभिषेक्ष्यामो ब्रह्मणा चोदिता वयम्‌
अवतीर्णोऽसि विश्वात्मन्भूमेर्भारापनुत्तये
शृईशुक उवाच
एवं कृष्णमुपामन्त्र्य सुरभिः पयसात्मनः
जलैराकाशगङ्गाया ऐरावतकरोद्धृतैः
इन्द्रः सुरर्षिभिः साकं चोदितो देवमातृभिः
अभ्यसिञ्चत दाशार्हं गोविन्द इति चाभ्यधात्‌
तत्रागतास्तुम्बुरुनारदादयो गन्धर्वविद्याधरसिद्धचारणाः
जगुर्यशो लोकमलापहं हरेः सुराङ्गनाः सन्ननृतुर्मुदान्विताः
तं तुष्टुवुर्देवनिकायकेतवो ह्यवाकिरंश्चाद्भुतपुष्पवृष्टिभिः
लोकाः परां निर्वृतिमाप्नुवंस्त्रयो गावस्तदा गामनयन्पयोद्रुताम्‌
नानारसौघाः सरितो वृक्षा आसन्मधुस्रवाः
अकृष्टपच्यौषधयो गिरयोऽबिभ्रनुन्मणीन्‌
कृष्णेऽभिषिक्त एतानि सर्वाणि कुरुनन्दन
निर्वैराण्यभवंस्तात क्रूराण्यपि निसर्गतः
इति गोगोकुलपतिं गोविन्दमभिषिच्य सः
अनुज्ञातो ययौ शक्रो वृतो देवादिभिर्दिवम्‌
श्रीबादरायणिरुवाच
एकादश्यां निराहारः समभ्यर्च्य जनार्दनम्‌
स्नातुं नन्दस्तु कालिन्द्यां द्वादश्यां जलमाविशत्‌
तं गृहीत्वानयद्भृत्यो वरुणस्यासुरोऽन्तिकम्‌
अवज्ञायासुरीं वेलां प्रविष्टमुदकं निशि
चुक्रुशुस्तमपश्यन्तः कृष्ण रामेति गोपकाः
भगवांस्तदुपश्रुत्य पितरं वरुणाहृतम्‌
तदन्तिकं गतो राजन्स्वानामभयदो विभुः
प्राप्तं वीक्ष्य हृषीकेशं लोकपालः सपर्यया
महत्या पूजयित्वाह तद्दर्शनमहोत्सवः
श्रीवरुण उवाच
अद्य मे निभृतो देहोऽद्यैवार्थोऽधिगतः प्रभो
त्वत्पादभाजो भगवन्नवापुः पारमध्वनः
नमस्तुभ्यं भगवते ब्रह्मणे परमात्मने
न यत्र श्रूयते माया लोकसृष्टिविकल्पना
अजानता मामकेन मूढेनाकार्यवेदिना
आनीतोऽयं तव पिता तद्भवान्क्षन्तुमर्हति
ममाप्यनुग्रहं कृष्ण कर्तुमर्हस्यशेषदृक्‌
गोविन्द नीयतामेष पिता ते पितृवत्सल
श्रीशुक उवाच
एवं प्रसादितः कृष्णो भगवानीश्वरेश्वरः
आदायागात्स्वपितरं बन्धूनां चावहन्मुदम्‌
नन्दस्त्वतीन्द्रियं दृष्ट्वा लोकपालमहोदयम्‌
कृष्णे च सन्नतिं तेषां ज्ञातिभ्यो विस्मितोऽब्रवीत्‌
ते चौत्सुक्यधियो राजन्मत्वा गोपास्तमीश्वरम्‌
अपि नः स्वगतिं सूक्ष्मामुपाधास्यदधीश्वरः
इति स्वानां स भगवान्विज्ञायाखिलदृक्स्वयम्‌
सङ्कल्पसिद्धये तेषां कृपयैतदचिन्तयत्‌
जनो वै लोक एतस्मिन्नविद्याकामकर्मभिः
उच्चावचासु गतिषु न वेद स्वां गतिं भ्रमन्‌
इति सञ्चिन्त्य भगवान्महाकारुणिको हरिः
दर्शयामास लोकं स्वं गोपानां तमसः परम्‌
सत्यं ज्ञानमनन्तं यद्ब्रह्मज्योतिः सनातनम्‌
यद्धि पश्यन्ति मुनयो गुणापाये समाहिताः
ते तु ब्रह्मह्रदम्नीता मग्नाः कृष्णेन चोद्धृताः
ददृशुर्ब्रह्मणो लोकं यत्राक्रूरोऽध्यगात्पुरा
नन्दादयस्तु तं दृष्ट्वा परमानन्दनिवृताः
कृष्णं च तत्र च्छन्दोभिः स्तूयमानं सुविस्मिताः
श्रीबादरायणिरुवाच
भगवानपि ता रातृईः शारदोत्फुल्लमल्लिकाः
वीक्ष्य रन्तुं मनश्चक्रे योगमायामुपाश्रितः
तदोडुराजः ककुभः करैर्मुखं प्राच्या विलिम्पन्नरुणेन शन्तमैः
स चर्षणीनामुदगाच्छुचो मृजन्प्रियः प्रियाया इव दीर्घदर्शनः
दृष्ट्वा कुमुद्वन्तमखण्डमण्डलं
रमाननाभं नवकुङ्कुमारुणम्‌
वनं च तत्कोमलगोभी रञ्जितं
जगौ कलं वामदृशां मनोहरम्‌
निशम्य गीतां तदनङ्गवर्धनं व्रजस्त्रियः कृष्णगृहीतमानसाः
आजग्मुरन्योन्यमलक्षितोद्यमाः स यत्र कान्तो जवलोलकुण्डलाः
दुहन्त्योऽभिययुः काश्चिद्दोहं हित्वा समुत्सुकाः
पयोऽधिश्रित्य संयावमनुद्वास्यापरा ययुः
परिवेषयन्त्यस्तद्धित्वा पाययन्त्यः शिशून्पयः
शुश्रूषन्त्यः पतीन्काश्चिदश्नन्त्योऽपास्य भोजनम्‌
लिम्पन्त्यः प्रमृजन्त्योऽन्या अञ्जन्त्यः काश्च लोचने
व्यत्यस्तवस्त्राभरणाः काश्चित्कृष्णान्तिकं ययुः
ता वार्यमाणाः पतिभिः पितृभिर्भ्रातृबन्धुभिः
गोविन्दापहृतात्मानो न न्यवर्तन्त मोहिताः
अन्तर्गृहगताः काश्चिद्गोप्योऽलब्धविनिर्गमाः
कृष्णं तद्भावनायुक्ता दध्युर्मीलितलोचनाः
दुःसहप्रेष्ठविरह तीव्रतापधुताशुभाः
ध्यानप्राप्ताच्युताश्लेष निर्वृत्या क्षीणमङ्गलाः
तमेव परमात्मानं जारबुद्ध्यापि सङ्गताः
जहुर्गुणमयं देहं सद्यः प्रक्षीणबन्धनाः
श्रीपरीक्षिदुवाच
कृष्णं विदुः परं कान्तं न तु ब्रह्मतया मुने
गुणप्रवाहोपरमस्तासां गुणधियां कथम्‌
श्रीशुक उवाच
उक्तं पुरस्तादेतत्ते चैद्यः सिद्धिं यथा गतः
द्विषन्नपि हृषीकेशं किमुताधोक्षजप्रियाः
नृणां निःश्रेयसार्थाय व्यक्तिर्भगवतो नृप
अव्ययस्याप्रमेयस्य निर्गुणस्य गुणात्मनः
कामं क्रोधं भयं स्नेहमैक्यं सौहृदमेव च
नित्यं हरौ विदधतो यान्ति तन्मयतां हि ते
न चैवं विस्मयः कार्यो भवता भगवत्यजे
योगेश्वरेश्वरे कृष्णे यत एतद्विमुच्यते
ता दृष्ट्वान्तिकमायाता भगवान्व्रजयोषितः
अवदद्वदतां श्रेष्ठो वाचः पेशैर्विमोहयन्‌
श्रीभगवानुवाच
स्वागतं वो महाभागाः प्रियं किं करवाणि वः
व्रजस्यानामयं कच्चिद्ब्रूतागमनकारणम्‌
रजन्येषा घोररूपा घोरसत्त्वनिषेविता
प्रतियात व्रजं नेह स्थेयं स्त्रीभिः सुमध्यमाः
मातरः पितरः पुत्रा भ्रातरः पतयश्च वः
विचिन्वन्ति ह्यपश्यन्तो मा कृढ्वं बन्धुसाध्वसम्‌
दृष्टं वनं कुसुमितं राकेशकररञ्जितम्‌
यमुनानिललीलैजत्तरुपल्लवशोभितम्‌
तद्यात मा चिरं गोष्ठं शुश्रूषध्वं पतीन्सतीः
क्रन्दन्ति वत्सा बालाश्च तान्पाययत दुह्यत
अथ वा मदभिस्नेहाद्भवत्यो यन्त्रिताशयाः
आगता ह्युपपन्नं वः प्रीयन्ते मयि जन्तवः
भर्तुः शुश्रूषणं स्त्रीणां परो धर्मो ह्यमायया
तद्बन्धूनां च कल्याणः प्रजानां चानुपोषणम्‌
दुःशीलो दुर्भगो वृद्धो जडो रोग्यधनोऽपि वा
पतिः स्त्रीभिर्न हातव्यो लोकेप्सुभिरपातकी
अस्वर्ग्यमयशस्यं च फल्गु कृच्छ्रं भयावहम्‌
जुगुप्सितं च सर्वत्र ह्यौपपत्यं कुलस्त्रियः
श्रवणाद्दर्शनाद्ध्यानान्मयि भावोऽनुकीर्तनात्‌
न तथा सन्निकर्षेण प्रतियात ततो गृहान्‌
श्रीशुक उवाच
इति विप्रियमाकर्ण्य गोप्यो गोविन्दभाषितम्‌
विषण्णा भग्नसङ्कल्पाश्चिन्तामापुर्दुरत्ययाम्‌
कृत्वा मुखान्यव शुचः श्वसनेन शुष्यद्‌
बिम्बाधराणि चरणेन भुवः लिखन्त्यः
अस्रैरुपात्तमसिभिः कुचकुङ्कुमानि
तस्थुर्मृजन्त्य उरुदुःखभराः स्म तूष्णीम्‌
प्रेष्ठं प्रियेतरमिव प्रतिभाषमाणं
कृष्णं तदर्थविनिवर्तितसर्वकामाः
नेत्रे विमृज्य रुदितोपहते स्म किञ्चित्‌
संरम्भगद्गदगिरोऽब्रुवतानुरक्ताः
श्रीगोप्य ऊचुः
मैवं विभोऽर्हति भवान्गदितुं नृशंसं
सन्त्यज्य सर्वविषयांस्तव पादमूलम्‌
भक्ता भजस्व दुरवग्रह मा त्यजास्मान्‌
देवो यथादिपुरुषो भजते मुमुक्षून्‌
यत्पत्यपत्यसुहृदामनुवृत्तिरङ्ग
स्त्रीणां स्वधर्म इति धर्मविदा त्वयोक्तम्‌
अस्त्वेवमेतदुपदेशपदे त्वयीशे
प्रेष्ठो भवांस्तनुभृतां किल बन्धुरात्मा
कुर्वन्ति हि त्वयि रतिं कुशलाः स्व आत्मन्‌
नित्यप्रिये पतिसुतादिभिरार्तिदैः किम्‌
तन्नः प्रसीद परमेश्वर मा स्म छिन्द्या
आशां धृतां त्वयि चिरादरविन्दनेत्र
चित्तं सुखेन भवतापहृतं गृहेषु
यन्निर्विशत्युत करावपि गृह्यकृत्ये
पादौ पदं न चलतस्तव पादमूलाद्‌
यामः कथं व्रजमथो करवाम किं वा
सिञ्चाङ्ग नस्त्वदधरामृतपूरकेण
हासावलोककलगीतजहृच्छयाग्निम्‌
नो चेद्वयं विरहजाग्न्युपयुक्तदेहा
ध्यानेन याम पदयोः पदवीं सखे ते
यर्ह्यम्बुजाक्ष तव पादतलं रमाया
दत्तक्षणं क्वचिदरण्यजनप्रियस्य
अस्प्राक्ष्म तत्प्रभृति नान्यसमक्षमञ्जः
स्थातुंस्त्वयाभिरमिता बत पारयामः
श्रीर्यत्पदाम्बुजरजश्चकमे तुलस्या
लब्ध्वापि वक्षसि पदं किल भृत्यजुष्टम्‌
यस्याः स्ववीक्षण उतान्यसुरप्रयासस्‌
तद्वद्वयं च तव पादरजः प्रपन्नाः
तन्नः प्रसीद वृजिनार्दन तेऽन्घ्रिमूलं
प्राप्ता विसृज्य वसतीस्त्वदुपासनाशाः
त्वत्सुन्दरस्मितनिरीक्षणतीव्रकाम
तप्तात्मनां पुरुषभूषण देहि दास्यम्‌
वीक्ष्यालकावृतमुखं तव कुण्दलश्री
गण्डस्थलाधरसुधं हसितावलोकम्‌
दत्ताभयं च भुजदण्डयुगं विलोक्य
वक्षः श्रियैकरमणं च भवाम दास्यः
का स्त्र्यङ्ग ते कलपदायतवेणुगीत
सम्मोहितार्यचरितान्न चलेत्त्रिलोक्याम्‌
त्रैलोक्यसौभगमिदं च निरीक्ष्य रूपं
यद्गोद्विजद्रुममृगाः पुलकान्यबिभ्रन्‌
व्यक्तं भवान्व्रजभयार्तिहरोऽभिजातो
देवो यथादिपुरुषः सुरलोकगोप्ता
तन्नो निधेहि करपङ्कजमार्तबन्धो
तप्तस्तनेषु च शिरःसु च किङ्करीणाम्‌
श्रीशुक उवाच
इति विक्लवितं तासां श्रुत्वा योगेश्वरेश्वरः
प्रहस्य सदयं गोपीरात्मारामोऽप्यरीरमत्‌
ताभिः समेताभिरुदारचेष्टितः प्रियेक्षणोत्फुल्लमुखीभिरच्युतः
उदारहासद्विजकुन्ददीधतिर्व्यरोचतैणाङ्क इवोडुभिर्वृतः
उपगीयमान उद्गायन्वनिताशतयूथपः
मालां बिभ्रद्वैजयन्तीं व्यचरन्मण्डयन्वनम्‌
नद्याः पुलिनमाविश्य गोपीभिर्हिमवालुकम्‌
जुष्टं तत्तरलानन्दि कुमुदामोदवायुना
बाहुप्रसारपरिरम्भकरालकोरु नीवीस्तनालभननर्मनखाग्रपातैः
क्ष्वेल्यावलोकहसितैर्व्रजसुन्दरीणामुत्तम्भयन्रतिपतिं रमयां चकार
एवं भगवतः कृष्णाल्लब्धमाना महात्मनः
आत्मानं मेनिरे स्त्रीणां मानिन्यो ह्यधिकं भुवि
तासां तत्सौभगमदं वीक्ष्य मानं च केशवः
प्रशमाय प्रसादाय तत्रैवान्तरधीयत
श्रीशुक उवाच
अन्तर्हिते भगवति सहसैव व्रजाङ्गनाः
अतप्यंस्तमचक्षाणाः करिण्य इव यूथपम्‌
गत्यानुरागस्मितविभ्रमेक्षितैर्मनोरमालापविहारविभ्रमैः
आक्षिप्तचित्ताः प्रमदा रमापतेस्तास्ता विचेष्टा जगृहुस्तदात्मिकाः
गतिस्मितप्रेक्षणभाषणादिषु प्रियाः प्रियस्य प्रतिरूढमूर्तयः
असावहं त्वित्यबलास्तदात्मिका न्यवेदिषुः कृष्णविहारविभ्रमाः
गायन्त्य उच्चैरमुमेव संहता विचिक्युरुन्मत्तकवद्वनाद्वनम्‌
पप्रच्छुराकाशवदन्तरं बहिर्भूतेषु सन्तं पुरुषं वनस्पतीन्‌
दृष्टो वः कच्चिदश्वत्थ प्लक्ष न्यग्रोध नो मनः
नन्दसूनुर्गतो हृत्वा प्रेमहासावलोकनैः
कच्चित्कुरबकाशोक नागपुन्नागचम्पकाः
रामानुजो मानिनीनामितो दर्पहरस्मितः
कच्चित्तुलसि कल्याणि गोविन्दचरणप्रिये
सह त्वालिकुलैर्बिभ्रद्दृष्टस्तेऽतिप्रियोऽच्युतः
मालत्यदर्शि वः कच्चिन्मल्लिके जातियूथिके
प्रीतिं वो जनयन्यातः करस्पर्शेन माधवः
चूतप्रियालपनसासनकोविदार जम्ब्वर्कबिल्वबकुलाम्रकदम्बनीपाः
येऽन्ये परार्थभवका यमुनोपकूलाः शंसन्तु कृष्णपदवीं रहितात्मनां नः
किं ते कृतं क्षिति तपो बत केशवाङ्घ्रि
स्पर्शोत्सवोत्पुलकिताङ्गनहैर्विभासि
अप्यङ्घ्रिसम्भव उरुक्रमविक्रमाद्वा
आहो वराहवपुषः परिरम्भणेन
अप्येणपत्न्युपगतः प्रिययेह गात्रैस्‌
तन्वन्दृशां सखि सुनिर्वृतिमच्युतो वः
कान्ताङ्गसङ्गकुचकुङ्कुमरञ्जितायाः
कुन्दस्रजः कुलपतेरिह वाति गन्धः
बाहुं प्रियांस उपधाय गृहीतपद्मो
रामानुजस्तुलसिकालिकुलैर्मदान्धैः
अन्वीयमान इह वस्तरवः प्रणामं
किं वाभिनन्दति चरन्प्रणयावलोकैः
पृच्छतेमा लता बाहूनप्याश्लिष्टा वनस्पतेः
नूनं तत्करजस्पृष्टा बिभ्रत्युत्पुलकान्यहो
इत्युन्मत्तवचो गोप्यः कृष्णान्वेषणकातराः
लीला भगवतस्तास्ता ह्यनुचक्रुस्तदात्मिकाः
कस्याचित्पूतनायन्त्याः कृष्णायन्त्यपिबत्स्तनम्‌
तोकयित्वा रुदत्यन्या पदाहन्शकटायतीम्‌
दैत्यायित्वा जहारान्यामेको कृष्णार्भभावनाम्‌
रिङ्गयामास काप्यङ्घ्री कर्षन्ती घोषनिःस्वनैः
कृष्णरामायिते द्वे तु गोपायन्त्यश्च काश्चन
वत्सायतीं हन्ति चान्या तत्रैका तु बकायतीम्‌
आहूय दूरगा यद्वत्कृष्णस्तमनुवर्ततीम्‌
वेणुं क्वणन्तीं क्रीडन्तीमन्याः शंसन्ति साध्विति
कस्याञ्चित्स्वभुजं न्यस्य चलन्त्याहापरा ननु
कृष्णोऽहं पश्यत गतिं ललितामिति तन्मनाः
मा भैष्ट वातवर्षाभ्यां तत्त्राणं विहितं मय
इत्युक्त्वैकेन हस्तेन यतन्त्युन्निदधेऽम्बरम्‌
आरुह्यैका पदाक्रम्य शिरस्याहापरां नृप
दुष्टाहे गच्छ जातोऽहं खलानाम्ननु दण्डकृत्‌
तत्रैकोवाच हे गोपा दावाग्निं पश्यतोल्बणम्‌
चक्षूंष्याश्वपिदध्वं वो विधास्ये क्षेममञ्जसा
बद्धान्यया स्रजा काचित्तन्वी तत्र उलूखले
बध्नामि भाण्डभेत्तारं हैयङ्गवमुषं त्विति
भीता सुदृक्पिधायास्यं भेजे भीतिविडम्बनम्‌
एवं कृष्णं पृच्छमाना व्र्ण्दावनलतास्तरून्‌
व्यचक्षत वनोद्देशे पदानि परमात्मनः
पदानि व्यक्तमेतानि नन्दसूनोर्महात्मनः
लक्ष्यन्ते हि ध्वजाम्भोज वज्राङ्कुशयवादिभिः
तैस्तैः पदैस्तत्पदवीमन्विच्छन्त्योऽग्रतोऽबलाः
वध्वाः पदैः सुपृक्तानि विलोक्यार्ताः समब्रुवन्‌
कस्याः पदानि चैतानि याताया नन्दसूनुना
अंसन्यस्तप्रकोष्ठायाः करेणोः करिणा यथा
अनयाराधितो नूनं भगवान्हरिरीश्वरः
यन्नो विहाय गोविन्दः प्रीतो यामनयद्रहः
धन्या अहो अमी आल्यो गोविन्दाङ्घ्र्यब्जरेणवः
यान्ब्रह्मेशौ रमा देवी दधुर्मूर्ध्न्यघनुत्तये
तस्या अमूनि नः क्षोभं कुर्वन्त्युच्चैः पदानि यत्‌
यैकापहृत्य गोपीनाम्रहो भुन्क्तेऽच्युताधरम्‌
न लक्ष्यन्ते पदान्यत्र तस्या नूनं तृणाङ्कुरैः
खिद्यत्सुजाताङ्घ्रितलामुन्निन्ये प्रेयसीं प्रियः
इमान्यधिकमग्नानि पदानि वहतो वधूम्‌
गोप्यः पश्यत कृष्णस्य भाराक्रान्तस्य कामिनः
अत्रावरोपिता कान्ता पुष्पहेतोर्महात्मना
अत्र प्रसूनावचयः प्रियार्थे प्रेयसा कृतः
प्रपदाक्रमण एते पश्यतासकले पदे
केशप्रसाधनं त्वत्र कामिन्याः कामिना कृतम्‌
तानि चूडयता कान्तामुपविष्टमिह ध्रुवम्‌
रेमे तया चात्मरत आत्मारामोऽप्यखण्डितः
कामिनां दर्शयन्दैन्यं स्त्रीणां चैव दुरात्मताम्‌
इत्येवं दर्शयन्त्यस्ताश्चेरुर्गोप्यो विचेतसः
यां गोपीमनयत्कृष्णो विहायान्याः स्त्रियो वने
सा च मेने तदात्मानं वरिष्ठं सर्वयोषिताम्‌
हित्वा गोपीः कामयाना मामसौ भजते प्रियः
ततो गत्वा वनोद्देशं दृप्ता केशवमब्रवीत्‌
न पारयेऽहं चलितुं नय मां यत्र ते मनः
एवमुक्तः प्रियामाह स्कन्ध आरुह्यतामिति
ततश्चान्तर्दधे कृष्णः सा वधूरन्वतप्यत
हा नाथ रमण प्रेष्ठ क्वासि क्वासि महाभुज
दास्यास्ते कृपणाया मे सखे दर्शय सन्निधिम्‌
श्रीशुक उवाच
अन्विच्छन्त्यो भगवतो मार्गं गोप्योऽविदूरितः
ददृशुः प्रियविश्लेषान्मोहितां दुःखितां सखीम्‌
तया कथितमाकर्ण्य मानप्राप्तिं च माधवात्‌
अवमानं च दौरात्म्याद्विस्मयं परमं ययुः
ततोऽविशन्वनं चन्द्र ज्योत्स्ना यावद्विभाव्यते
तमः प्रविष्टमालक्ष्य ततो निववृतुः स्त्रियः
तन्मनस्कास्तदलापास्तद्विचेष्टास्तदात्मिकाः
तद्गुणानेव गायन्त्यो नात्मगाराणि सस्मरुः
पुनः पुलिनमागत्य कालिन्द्याः कृष्णभावनाः
समवेता जगुः कृष्णं तदागमनकाङ्क्षिताः
गोप्य ऊचुः
जयति तेऽधिकं जन्मना व्रजः श्रयत इन्दिरा शश्वदत्र हि
दयित दृश्यतां दिक्षु तावकास्त्वयि धृतासवस्त्वां विचिन्वते
शरदुदाशये साधुजातसत्‌ सरसिजोदरश्रीमुषा दृशा
सुरतनाथ तेऽशुल्कदासिका वरद निघ्नतो नेह किं वधः
विषजलाप्ययाद्व्यालराक्षसाद्वर्षमारुताद्वैद्युतानलात्‌
वृषमयात्मजाद्विश्वतो भयादृषभ ते वयं रक्षिता मुहुः
न खलु गोपीकानन्दनो भवानखिलदेहिनामन्तरात्मदृक्‌
विखनसार्थितो विश्वगुप्तये सख उदेयिवान्सात्वतां कुले
विरचिताभयं वृष्णिधूर्य ते चरणमीयुषां संसृतेर्भयात्‌
करसरोरुहं कान्त कामदं शिरसि धेहि नः श्रीकरग्रहम्‌
व्रजजनार्तिहन्वीर योषितां निजजनस्मयध्वंसनस्मित
भज सखे भवत्किङ्करीः स्म नो जलरुहाननं चारु दर्शय
प्रणतदेहिनां पापकर्षणं तृणचरानुगं श्रीनिकेतनम्‌
फणिफणार्पितं ते पदाम्बुजं कृणु कुचेषु नः कृन्धि हृच्छयम्‌
मधुरया गिरा वल्गुवाक्यया बुधमनोज्ञया पुष्करेक्षण
विधिकरीरिमा वीर मुह्यतीरधरसीधुनाप्याययस्व नः
तव कथामृतं तप्तजीवनं कविभिरीडितं कल्मषापहम्‌
श्रवणमङ्गलं श्रीमदाततं भुवि गृणन्ति ये भूरिदा जनाः
प्रहसितं प्रियप्रेमवीक्षणं विहरणं च ते ध्यानमङ्गलम्‌
रहसि संविदो या हृदि स्पृशः कुहक नो मनः क्षोभयन्ति हि
चलसि यद्व्रजाच्चारयन्पशून्नलिनसुन्दरं नाथ ते पदम्‌
शिलतृणाङ्कुरैः सीदतीति नः कलिलतां मनः कान्त गच्छति
दिनपरिक्षये नीलकुन्तलैर्वनरुहाननं बिभ्रदावृतम्‌
घनरजस्वलं दर्शयन्मुहुर्मनसि नः स्मरं वीर यच्छसि
प्रणतकामदं पद्मजार्चितं धरणिमण्डनं ध्येयमापदि
चरणपङ्कजं शन्तमं च ते रमण नः स्तनेष्वर्पयाधिहन्‌
सुरतवर्धनं शोकनाशनं स्वरितवेणुना सुष्ठु चुम्बितम्‌
इतररागविस्मारणं नृणां वितर वीर नस्तेऽधरामृतम्‌
अटति यद्भवानह्नि काननं त्रुटि युगायते त्वामपश्यताम्‌
कुटिलकुन्तलं श्रीमुखं च ते जड उदीक्षतां पक्ष्मकृद्दृशाम्‌
पतिसुतान्वयभ्रातृबान्धवानतिविलङ्घ्य तेऽन्त्यच्युतागताः
गतिविदस्तवोद्गीतमोहिताः कितव योषितः कस्त्यजेन्निशि
रहसि संविदं हृच्छयोदयं प्रहसिताननं प्रेमवीक्षणम्‌
बृहदुरः श्रियो वीक्ष्य धाम ते मुहुरतिस्पृहा मुह्यते मनः
व्रजवनौकसां व्यक्तिरङ्ग ते वृजिनहन्त्र्यलं विश्वमङ्गलम्‌
त्यज मनाक्च नस्त्वत्स्पृहात्मनां स्वजनहृद्रुजां यन्निषूदनम्‌
यत्ते सुजातचरणाम्बुरुहं स्तनेषु
भीताः शनैः प्रिय दधीमहि कर्कशेषु
तेनाटवीमटसि तद्व्यथते न किं स्वित्‌
कूर्पादिभिर्भ्रमति धीर्भवदायुषां नः
श्रीशुक उवाच
इति गोप्यः प्रगायन्त्यः प्रलपन्त्यश्च चित्रधा
रुरुदुः सुस्वरं राजन्कृष्णदर्शनलालसाः
तासामाविरभूच्छौरिः स्मयमानमुखाम्बुजः
पीताम्बरधरः स्रग्वी साक्षान्मन्मथमन्मथः
तं विलोक्यागतं प्रेष्ठं प्रीत्युत्फुल्लदृशोऽबलाः
उत्तस्थुर्युगपत्सर्वास्तन्वः प्राणमिवागतम्‌
काचित्कराम्बुजं शौरेर्जगृहेऽञ्जलिना मुदा
काचिद्दधार तद्बाहुमंसे चन्दनभूषितम्‌
काचिदञ्जलिनागृह्णात्तन्वी ताम्बूलचर्वितम्‌
एका तदङ्घ्रिकमलं सन्तप्ता स्तनयोरधात्‌
एका भ्रुकुटिमाबध्य प्रेमसंरम्भविह्वला
घ्नन्तीवैक्षत्कटाक्षेपैः सन्दष्टदशनच्छदा
अपरानिमिषद्दृग्भ्यां जुषाणा तन्मुखाम्बुजम्‌
आपीतमपि नातृप्यत्सन्तस्तच्चरणं यथा
तं काचिन्नेत्ररन्ध्रेण हृदि कृत्वा निमील्य च
पुलकाङ्ग्युपगुह्यास्ते योगीवानन्दसम्प्लुता
सर्वास्ताः केशवालोक परमोत्सवनिर्वृताः
जहुर्विरहजं तापं प्राज्ञं प्राप्य यथा जनाः
ताभिर्विधूतशोकाभिर्भगवानच्युतो वृतः
व्यरोचताधिकं तात पुरुषः शक्तिभिर्यथा
ताः समादाय कालिन्द्या निर्विश्य पुलिनं विभुः
विकसत्कुन्दमन्दार सुरभ्यनिलषट्पदम्‌
शरच्चन्द्रांशुसन्दोह ध्वस्तदोषातमः शिवम्‌
कृष्णाया हस्ततरला चितकोमलवालुकम्‌
तद्दर्शनाह्लादविधूतहृद्रुजो मनोरथान्तं श्रुतयो यथा ययुः
स्वैरुत्तरीयैः कुचकुङ्कुमाङ्कितैरचीकॢपन्नासनमात्मबन्धवे
तत्रोपविष्टो भगवान्स ईश्वरो योगेश्वरान्तर्हृदि कल्पितासनः
चकास गोपीपरिषद्गतोऽर्चितस्त्रैलोक्यलक्ष्म्येकपदं वपुर्दधत्‌
सभाजयित्वा तमनङ्गदीपनं सहासलीलेक्षणविभ्रमभ्रुवा
संस्पर्शनेनाङ्ककृताङ्घ्रिहस्तयोः संस्तुत्य ईषत्कुपिता बभाषिरे
श्रीगोप्य ऊचुः
भजतोऽनुभजन्त्येक एक एतद्विपर्ययम्‌
नोभयांश्च भजन्त्येक एतन्नो ब्रूहि साधु भोः
श्रीभगवानुवाच
मिथो भजन्ति ये सख्यः स्वार्थैकान्तोद्यमा हि ते
न तत्र सौहृदं धर्मः स्वार्थार्थं तद्धि नान्यथा
भजन्त्यभजतो ये वै करुणाः पितरौ यथा
धर्मो निरपवादोऽत्र सौहृदं च सुमध्यमाः
भजतोऽपि न वै केचिद्भजन्त्यभजतः कुतः
आत्मारामा ह्याप्तकामा अकृतज्ञा गुरुद्रुहः
नाहं तु सख्यो भजतोऽपि जन्तून्भजाम्यमीषामनुवृत्तिवृत्तये
यथाधनो लब्धधने विनष्टे तच्चिन्तयान्यन्निभृतो न वेद
एवं मदर्थोज्झितलोकवेद स्वानाम्हि वो मय्यनुवृत्तयेऽबलाः
मयापरोक्षं भजता तिरोहितं मासूयितुं मार्हथ तत्प्रियं प्रियाः
न पारयेऽहं निरवद्यसंयुजां स्वसाधुकृत्यं विबुधायुषापि वः
या माभजन्दुर्जरगेहशृङ्खलाः संवृश्च्य तद्वः प्रतियातु साधुना
श्रीशुक उवाच
इत्थं भगवतो गोप्यः श्रुत्वा वाचः सुपेशलाः
जहुर्विरहजं तापं तदङ्गोपचिताशिषः
तत्रारभत गोविन्दो रासक्रीडामनुव्रतैः
स्त्रीरत्नैरन्वितः प्रीतैरन्योन्याबद्धबाहुभिः
रासोत्सवः सम्प्रवृत्तो गोपीमण्डलमण्डितः
योगेश्वरेण कृष्णेन तासां मध्ये द्वयोर्द्वयोः
प्रविष्टेन गृहीतानां कण्ठे स्वनिकटं स्त्रियः
यं मन्येरन्नभस्तावद्विमानशतसङ्कुलम्‌
दिवौकसां सदाराणामौत्सुक्यापहृतात्मनाम्‌
ततो दुन्दुभयो नेदुर्निपेतुः पुष्पवृष्टयः
जगुर्गन्धर्वपतयः सस्त्रीकास्तद्यशोऽमलम्‌
वलयानां नूपुराणां किङ्किणीनां च योषिताम्‌
सप्रियाणामभूच्छब्दस्तुमुलो रासमण्डले
तत्रातिशुशुभे ताभिर्भगवान्देवकीसुतः
मध्ये मणीनां हैमानां महामरकतो यथा
पादन्यासैर्भुजविधुतिभिः सस्मितैर्भ्रूविलासैर्‌
भज्यन्मध्यैश्चलकुचपटैः कुण्डलैर्गण्डलोलैः
स्विद्यन्मुख्यः कवररसनाग्रन्थयः कृष्णवध्वो
गायन्त्यस्तं तडित इव ता मेघचक्रे विरेजुः
उच्चैर्जगुर्नृत्यमाना रक्तकण्ठ्यो रतिप्रियाः
कृष्णाभिमर्शमुदिता यद्गीतेनेदमावृतम्‌
काचित्समं मुकुन्देन स्वरजातीरमिश्रिताः
उन्निन्ये पूजिता तेन प्रीयता साधु साध्विति
तदेव ध्रुवमुन्निन्ये तस्यै मानं च बह्वदात्‌
काचिद्रासपरिश्रान्ता पार्श्वस्थस्य गदाभृतः
जग्राह बाहुना स्कन्धं श्लथद्वलयमल्लिका
तत्रैकांसगतं बाहुं कृष्णस्योत्पलसौरभम्‌
चन्दनालिप्तमाघ्राय हृष्टरोमा चुचुम्ब ह
कस्याश्चिन्नाट्यविक्षिप्त कुण्डलत्विषमण्डितम्‌
गण्डं गण्डे सन्दधत्याः प्रादात्ताम्बूलचर्वितम्‌
नृत्यती गायती काचित्कूजन्नूपुरमेखला
पार्श्वस्थाच्युतहस्ताब्जं श्रान्ताधात्स्तनयोः शिवम्‌
गोप्यो लब्ध्वाच्युतं कान्तं श्रिय एकान्तवल्लभम्‌
गृहीतकण्ठ्यस्तद्दोर्भ्यां गायन्त्यस्तम्विजह्रिरे
कर्णोत्पलालकविटङ्ककपोलघर्म
वक्त्रश्रियो वलयनूपुरघोषवाद्यैः
गोप्यः समं भगवता ननृतुः स्वकेश
स्रस्तस्रजो भ्रमरगायकरासगोष्ठ्याम्‌
एवं परिष्वङ्गकराभिमर्श स्निग्धेक्षणोद्दामविलासहासैः
रेमे रमेशो व्रजसुन्दरीभिर्यथार्भकः स्वप्रतिबिम्बविभ्रमः
तदङ्गसङ्गप्रमुदाकुलेन्द्रियाः केशान्दुकूलं कुचपट्टिकां वा
नाञ्जः प्रतिव्योढुमलं व्रजस्त्रियो विस्रस्तमालाभरणाः कुरूद्वह
कृष्णविक्रीडितं वीक्ष्य मुमुहुः खेचरस्त्रियः
कामार्दिताः शशाङ्कश्च सगणो विस्मितोऽभवत्‌
कृत्वा तावन्तमात्मानं यावतीर्गोपयोषितः
रेमे स भगवांस्ताभिरात्मारामोऽपि लीलया
तासां रतिविहारेण श्रान्तानां वदनानि सः
प्रामृजत्करुणः प्रेम्णा शन्तमेनाङ्ग पाणिना
गोप्यः स्फुरत्पुरटकुण्डलकुन्तलत्विड्‌
गण्डश्रिया सुधितहासनिरीक्षणेन
मानं दधत्य ऋषभस्य जगुः कृतानि
पुण्यानि तत्कररुहस्पर्शप्रमोदाः
ताभिर्युतः श्रममपोहितुमङ्गसङ्ग
घृष्टस्रजः स कुचकुङ्कुमरञ्जितायाः
गन्धर्वपालिभिरनुद्रुत आविशद्वाः
श्रान्तो गजीभिरिभराडिव भिन्नसेतुः
सोऽम्भस्यलं युवतिभिः परिषिच्यमानः
प्रेम्णेक्षितः प्रहसतीभिरितस्ततोऽङ्ग
वैमानिकैः कुसुमवर्षिभिरीद्यमानो
रेमे स्वयं स्वरतिरत्र गजेन्द्रलीलः
ततश्च कृष्णोपवने जलस्थल प्रसूनगन्धानिलजुष्टदिक्तटे
चचार भृङ्गप्रमदागणावृतो यथा मदच्युद्द्विरदः करेणुभिः
एवं शशाङ्कांशुविराजिता निशाः स सत्यकामोऽनुरताबलागणः
सिषेव आत्मन्यवरुद्धसौरतः सर्वाः शरत्काव्यकथारसाश्रयाः
श्रीपरीक्षिदुवाच
संस्थापनाय धर्मस्य प्रशमायेतरस्य च
अवतीर्णो हि भगवानंशेन जगदीश्वरः
स कथं धर्मसेतूनां वक्ता कर्ताभिरक्षिता
प्रतीपमाचरद्ब्रह्मन्परदाराभिमर्शनम्‌
आप्तकामो यदुपतिः कृतवान्वै जुगुप्सितम्‌
किमभिप्राय एतन्नः शंशयं छिन्धि सुव्रत
श्रीशुक उवाच
धर्मव्यतिक्रमो दृष्ट ईश्वराणां च साहसम्‌
तेजीयसां न दोषाय वह्नेः सर्वभुजो यथा
नैतत्समाचरेज्जातु मनसापि ह्यनीश्वरः
विनश्यत्याचरन्मौढ्याद्यथारुद्रोऽब्धिजं विषम्‌
ईश्वराणां वचः सत्यं तथैवाचरितं क्वचित्‌
तेषां यत्स्ववचोयुक्तं बुद्धिमांस्तत्समाचरेत्‌
कुशलाचरितेनैषामिह स्वार्थो न विद्यते
विपर्ययेण वानर्थो निरहङ्कारिणां प्रभो
किमुताखिलसत्त्वानां तिर्यङ्मर्त्यदिवौकसाम्‌
ईशितुश्चेशितव्यानां कुशलाकुशलान्वयः
यत्पादपङ्कजपरागनिषेवतृप्ता
योगप्रभावविधुताखिलकर्मबन्धाः
स्वैरं चरन्ति मुनयोऽपि न नह्यमानास्‌
तस्येच्छयात्तवपुषः कुत एव बन्धः
गोपीनां तत्पतीनां च सर्वेषामेव देहिनाम्‌
योऽन्तश्चरति सोऽध्यक्षः क्रीडनेनेह देहभाक्‌
अनुग्रहाय भक्तानां मानुषं देहमास्थितः
भजते तादृशीः क्रीड याः श्रुत्वा तत्परो भवेत्‌
नासूयन्खलु कृष्णाय मोहितास्तस्य मायया
मन्यमानाः स्वपार्श्वस्थान्स्वान्स्वान्दारान्व्रजौकसः
ब्रह्मरात्र उपावृत्ते वासुदेवानुमोदिताः
अनिच्छन्त्यो ययुर्गोप्यः स्वगृहान्भगवत्प्रियाः
विक्रीडितं व्रजवधूभिरिदं च विष्णोः
श्रद्धान्वितोऽनुशृणुयादथ वर्णयेद्यः
भक्तिं परां भगवति प्रतिलभ्य कामं
हृद्रोगमाश्वपहिनोत्यचिरेण धीरः
श्रीशुक उवाच
एकदा देवयात्रायां गोपाला जातकौतुकाः
अनोभिरनडुद्युक्तैः प्रययुस्तेऽम्बिकावनम्‌
तत्र स्नात्वा सरस्वत्यां देवं पशुपतिं विभुम्‌
आनर्चुरर्हणैर्भक्त्या देवीं च णृपतेऽम्बिकाम्‌
गावो हिरण्यं वासांसि मधु मध्वन्नमादृताः
ब्राह्मणेभ्यो ददुः सर्वे देवो नः प्रीयतामिति
ऊषुः सरस्वतीतीरे जलं प्राश्य यतव्रताः
रजनीं तां महाभागा नन्दसुनन्दकादयः
कश्चिन्महानहिस्तस्मिन्विपिनेऽतिबुभुक्षितः
यदृच्छयागतो नन्दं शयानमुरगोऽग्रसीत्‌
स चुक्रोशाहिना ग्रस्तः कृष्ण कृष्ण महानयम्‌
सर्पो मां ग्रसते तात प्रपन्नं परिमोचय
तस्य चाक्रन्दितं श्रुत्वा गोपालाः सहसोत्थिताः
ग्रस्तं च दृष्ट्वा विभ्रान्ताः सर्पं विव्यधुरुल्मुकैः
अलातैर्दह्यमानोऽपि नामुञ्चत्तमुरङ्गमः
तमस्पृशत्पदाभ्येत्य भगवान्सात्वतां पतिः
स वै भगवतः श्रीमत्पादस्पर्शहताशुभः
भेजे सर्पवपुर्हित्वा रूपं विद्याधरार्चितम्‌
तमपृच्छद्धृषीकेशः प्रणतं समवस्थितम्‌
दीप्यमानेन वपुषा पुरुषं हेममालिनम्‌
को भवान्परया लक्ष्म्या रोचतेऽद्भुतदर्शनः
कथं जुगुप्सितामेतां गतिं वा प्रापितोऽवशः
सर्प उवाच
अहं विद्याधरः कश्चित्सुदर्शन इति श्रुतः
श्रिया स्वरूपसम्पत्त्या विमानेनाचरन्दिशः
ऋषीन्विरूपाङ्गिरसः प्राहसं रूपदर्पितः
तैरिमां प्रापितो योनिं प्रलब्धैः स्वेन पाप्मना
शापो मेऽनुग्रहायैव कृतस्तैः करुणात्मभिः
यदहं लोकगुरुणा पदा स्पृष्टो हताशुभः
तं त्वाहं भवभीतानां प्रपन्नानां भयापहम्‌
आपृच्छे शापनिर्मुक्तः पादस्पर्शादमीवहन्‌
प्रपन्नोऽस्मि महायोगिन्महापुरुष सत्पते
अनुजानीहि मां देव सर्वलोकेश्वरेश्वर
ब्रह्मदण्डाद्विमुक्तोऽहं सद्यस्तेऽच्युत दर्शनात्‌
यन्नाम गृह्णन्नखिलान्श्रोत्ऱ्‌^ईनात्मानमेव च
सद्यः पुनाति किं भूयस्तस्य स्पृष्टः पदा हि ते
इत्यनुज्ञाप्य दाशार्हं परिक्रम्याभिवन्द्य च
सुदर्शनो दिवं यातः कृच्छ्रान्नन्दश्च मोचितः
निशाम्य कृष्णस्य तदात्मवैभवं
व्रजौकसो विस्मितचेतसस्ततः
समाप्य तस्मिन्नियमं पुनर्व्रजं
णृपाययुस्तत्कथयन्त आदृताः
कदाचिदथ गोविन्दो रामश्चाद्भुतविक्रमः
विजह्रतुर्वने रात्र्यां मध्यगौ व्रजयोषिताम्‌
उपगीयमानौ ललितं स्त्रीजनैर्बद्धसौहृदैः
स्वलङ्कृतानुलिप्ताङ्गौ स्रग्विनौ विरजोऽम्बरौ
निशामुखं मानयन्तावुदितोडुपतारकम्‌
मल्लिकागन्धमत्तालि जुष्टं कुमुदवायुना
जगतुः सर्वभूतानां मनःश्रवणमङ्गलम्‌
तौ कल्पयन्तौ युगपत्स्वरमण्डलमूर्च्छितम्‌
गोप्यस्तद्गीतमाकर्ण्य मूर्च्छिता नाविदन्नृप
स्रंसद्दुकूलमात्मानं स्रस्तकेशस्रजं ततः
एवं विक्रीडतोः स्वैरं गायतोः सम्प्रमत्तवत्‌
शङ्खचूड इति ख्यातो धनदानुचरोऽभ्यगात्‌
तयोर्निरीक्षतो राजंस्तन्नाथं प्रमदाजनम्‌
क्रोशन्तं कालयामास दिश्युदीच्यामशङ्कितः
क्रोशन्तं कृष्ण रामेति विलोक्य स्वपरिग्रहम्‌
यथा गा दस्युना ग्रस्ता भ्रातरावन्वधावताम्‌
मा भैष्टेत्यभयारावौ शालहस्तौ तरस्विनौ
आसेदतुस्तं तरसा त्वरितं गुह्यकाधमम्‌
स वीक्ष्य तावनुप्राप्तौ कालमृत्यू इवोद्विजन्‌
विषृज्य स्त्रीजनं मूढः प्राद्रवज्जीवितेच्छया
तमन्वधावद्गोविन्दो यत्र यत्र स धावति
जिहीर्षुस्तच्छिरोरत्नं तस्थौ रक्षन्स्त्रियो बलः
अविदूर इवाभ्येत्य शिरस्तस्य दुरात्मनः
जहार मुष्टिनैवाङ्ग सहचूडमणिं विभुः
शङ्खचूडं निहत्यैवं मणिमादाय भास्वरम्‌
अग्रजायाददात्प्रीत्या पश्यन्तीनां च योषिताम्‌
श्रीशुक उवाच
गोप्यः कृष्णे वनं याते तमनुद्रुतचेतसः
कृष्णलीलाः प्रगायन्त्यो निन्युर्दुःखेन वासरान्‌
श्रीगोप्य ऊचुः
वामबाहुकृतवामकपोलो वल्गितभ्रुरधरार्पितवेणुम्‌
कोमलाङ्गुलिभिराश्रितमार्गं गोप्य ईरयति यत्र मुकुन्दः
व्योमयानवनिताः सह सिद्धैर्विस्मितास्तदुपधार्य सलज्जाः
काममार्गणसमर्पितचित्ताः कश्मलं ययुरपस्मृतनीव्यः
हन्त चित्रमबलाः शृणुतेदं हारहास उरसि स्थिरविद्युत्‌
नन्दसूनुरयमार्तजनानां नर्मदो यर्हि कूजितवेणुः
वृन्दशो व्रजवृषा मृगगावो वेणुवाद्यहृतचेतस आरात्‌
दन्तदष्टकवला धृतकर्णा निद्रिता लिखितचित्रमिवासन्‌
बर्हिणस्तबकधातुपलाशैर्बद्धमल्लपरिबर्हविडम्बः
कर्हिचित्सबल आलि स गोपैर्गाः समाह्वयति यत्र मुकुन्दः
तर्हि भग्नगतयः सरितो वै तत्पदाम्बुजरजोऽनिलनीतम्‌
स्पृहयतीर्वयमिवाबहुपुण्याः प्रेमवेपितभुजाः स्तिमितापः
अनुचरैः समनुवर्णितवीर्य आदिपूरुष इवाचलभूतिः
वनचरो गिरितटेषु चरन्तीर्वेणुनाह्वयति गाः स यदा हि
वनलतास्तरव आत्मनि विष्णुं व्यञ्जयन्त्य इव पुष्पफलाढ्याः
प्रणतभारविटपा मधुधाराः प्रेमहृष्टतनवो ववृषुः स्म
दर्शनीयतिलको वनमाला दिव्यगन्धतुलसीमधुमत्तैः
अलिकुलैरलघु गीतामभीष्टमाद्रियन्यर्हि सन्धितवेणुः
सरसि सारसहंसविहङ्गाश्चारुगीताहृतचेतस एत्य
हरिमुपासत ते यतचित्ता हन्त मीलितदृशो धृतमौनाः
सहबलः स्रगवतंसविलासः सानुषु क्षितिभृतो व्रजदेव्यः
हर्षयन्यर्हि वेणुरवेण जातहर्ष उपरम्भति विश्वम्‌
महदतिक्रमणशङ्कितचेता मन्दमन्दमनुगर्जति मेघः
सुहृदमभ्यवर्षत्सुमनोभिश्छायया च विदधत्प्रतपत्रम्‌
विविधगोपचरणेषु विदग्धो वेणुवाद्य उरुधा निजशिक्षाः
तव सुतः सति यदाधरबिम्बे दत्तवेणुरनयत्स्वरजातीः
सवनशस्तदुपधार्य सुरेशाः शक्रशर्वपरमेष्ठिपुरोगाः
कवय आनतकन्धरचित्ताः कश्मलं ययुरनिश्चिततत्त्वाः
निजपदाब्जदलैर्ध्वजवज्र नीरजाङ्कुशविचित्रललामैः
व्रजभुवः शमयन्खुरतोदं वर्ष्मधुर्यगतिरीडितवेणुः
व्रजति तेन वयं सविलास वीक्षणार्पितमनोभववेगाः
कुजगतिं गमिता न विदामः कश्मलेन कवरं वसनं वा
मणिधरः क्वचिदागणयन्गा मालया दयितगन्धतुलस्याः
प्रणयिनोऽनुचरस्य कदांसे प्रक्षिपन्भुजमगायत यत्र
क्वणितवेणुरववञ्चितचित्ताः कृष्णमन्वसत कृष्णगृहिण्यः
गुणगणार्णमनुगत्य हरिण्यो गोपिका इव विमुक्तगृहाशाः
कुन्ददामकृतकौतुकवेषो गोपगोधनवृतो यमुनायाम्‌
नन्दसूनुरनघे तव वत्सो नर्मदः प्रणयिणां विजहार
मन्दवायुरुपवात्यनकूलं मानयन्मलयजस्पर्शेन
वन्दिनस्तमुपदेवगणा ये वाद्यगीतबलिभिः परिवव्रुः
वत्सलो व्रजगवां यदगध्रो वन्द्यमानचरणः पथि वृद्धैः
कृत्स्नगोधनमुपोह्य दिनान्ते गीतवेणुरनुगेडितकीर्तिः
उत्सवं श्रमरुचापि दृशीनामुन्नयन्खुररजश्छुरितस्रक्‌
दित्सयैति सुहृदासिष एष देवकीजठरभूरुडुराजः
मदविघूर्णितलोचन ईषत्मानदः स्वसुहृदां वनमाली
बदरपाण्डुवदनो मृदुगण्डं मण्डयन्कनककुण्डललक्ष्म्या
यदुपतिर्द्विरदराजविहारो यामिनीपतिरिवैष दिनान्ते
मुदितवक्त्र उपयाति दुरन्तं मोचयन्व्रजगवां दिनतापम्‌
श्रीशुक उवाच
एवं व्रजस्त्रियो राजन्कृष्णलीलानुगायतीः
रेमिरेऽहःसु तच्चित्तास्तन्मनस्का महोदयाः
श्री बादरायणिरुवाच
अथ तर्ह्यागतो गोष्ठमरिष्टो वृषभासुरः
महीम्महाककुत्कायः कम्पयन्खुरविक्षताम्‌
रम्भमाणः खरतरं पदा च विलिखन्महीम्‌
उद्यम्य पुच्छं वप्राणि विषाणाग्रेण चोद्धरन्‌
किञ्चित्किञ्चिच्छकृन्मुञ्चन्मूत्रयन्स्तब्धलोचनः
यस्य निर्ह्रादितेनाङ्ग निष्ठुरेण गवां नृणाम्‌
पतन्त्यकालतो गर्भाः स्रवन्ति स्म भयेन वै
निर्विशन्ति घना यस्य ककुद्यचलशङ्कया
तं तीक्ष्णशृङ्गमुद्वीक्ष्य गोप्यो गोपाश्च तत्रसुः
पशवो दुद्रुवुर्भीता राजन्सन्त्यज्य गोकुलम्‌
कृष्ण कृष्णेति ते सर्वे गोविन्दं शरणं ययुः
भगवानपि तद्वीक्ष्य गोकुलं भयविद्रुतम्‌
मा भैष्टेति गिराश्वास्य वृषासुरमुपाह्वयत्‌
गोपालैः पशुभिर्मन्द त्रासितैः किमसत्तम
मयि शास्तरि दुष्टानां त्वद्विधानां दुरात्मनाम्‌
इत्यास्फोत्याच्युतोऽरिष्टं तलशब्देन कोपयन्‌
सख्युरंसे भुजाभोगं प्रसार्यावस्थितो हरिः
सोऽप्येवं कोपितोऽरिष्टः खुरेणावनिमुल्लिखन्‌
उद्यत्पुच्छभ्रमन्मेघः क्रुद्धः कृष्णमुपाद्रवत्‌
अग्रन्यस्तविषाणाग्रः स्तब्धासृग्लोचनोऽच्युतम्‌
कटाक्षिप्याद्रवत्तूर्णमिन्द्रमुक्तोऽशनिर्यथा
गृहीत्वा शृङ्गयोस्तं वा अष्टादश पदानि सः
प्रत्यपोवाह भगवान्गजः प्रतिगजं यथा
सोऽपविद्धो भगवता पुनरुत्थाय सत्वरम्‌
आपतत्स्विन्नसर्वाङ्गो निःश्वसन्क्रोधमूर्च्छितः
तमापतन्तं स निगृह्य शृङ्गयोः पदा समाक्रम्य निपात्य भूतले
निष्पीडयामास यथार्द्रमम्बरं कृत्वा विषाणेन जघान सोऽपतत्‌
असृग्वमन्मूत्रशकृत्समुत्सृजन्क्षिपंश्च पादाननवस्थितेक्षणः
जगाम कृच्छ्रं निरृतेरथ क्षयं पुष्पैः किरन्तो हरिमीडिरे सुराः
एवं कुकुद्मिनं हत्वा स्तूयमानः द्विजातिभिः
विवेश गोष्ठं सबलो गोपीनां नयनोत्सवः
अरिष्टे निहते दैत्ये कृष्णेनाद्भुतकर्मणा
कंसायाथाह भगवान्नारदो देवदर्शनः
यशोदायाः सुतां कन्यां देवक्याः कृष्णमेव च
रामं च रोहिणीपुत्रं वसुदेवेन बिभ्यता
न्यस्तौ स्वमित्रे नन्दे वै याभ्यां ते पुरुषा हताः
निशम्य तद्भोजपतिः कोपात्प्रचलितेन्द्रियः
निशातमसिमादत्त वसुदेवजिघांसया
निवारितो नारदेन तत्सुतौ मृत्युमात्मनः
ज्ञात्वा लोहमयैः पाशैर्बबन्ध सह भार्यया
प्रतियाते तु देवर्षौ कंस आभाष्य केशिनम्‌
प्रेषयामास हन्येतां भवता रामकेशवौ
ततो मुष्टिकचाणूर शलतोशलकादिकान्‌
अमात्यान्हस्तिपांश्चैव समाहूयाह भोजराट्‌
भो भो निशम्यतामेतद्वीरचाणूरमुष्टिकौ
नन्दव्रजे किलासाते सुतावानकदुन्दुभेः
रामकृष्णौ ततो मह्यं मृत्युः किल निदर्शितः
भवद्भ्यामिह सम्प्राप्तौ हन्येतां मल्ललीलया
मञ्चाः क्रियन्तां विविधा मल्लरङ्गपरिश्रिताः
पौरा जानपदाः सर्वे पश्यन्तु स्वैरसंयुगम्‌
महामात्र त्वया भद्र रङ्गद्वार्युपनीयताम्‌
द्विपः कुवलयापीडो जहि तेन ममाहितौ
आरभ्यतां धनुर्यागश्चतुर्दश्यां यथाविधि
विशसन्तु पशून्मेध्यान्भूतराजाय मीढुषे
इत्याज्ञाप्यार्थतन्त्रज्ञ आहूय यदुपुङ्गवम्‌
गृहीत्वा पाणिना पाणिं ततोऽक्रूरमुवाच ह
भो भो दानपते मह्यं क्रियतां मैत्रमादृतः
नान्यस्त्वत्तो हिततमो विद्यते भोजवृष्णिषु
अतस्त्वामाश्रितः सौम्य कार्यगौरवसाधनम्‌
यथेन्द्रो विष्णुमाश्रित्य स्वार्थमध्यगमद्विभुः
गच्छ नन्दव्रजं तत्र सुतावानकदुन्दुभेः
आसाते ताविहानेन रथेनानय मा चिरम्‌
निसृष्टः किल मे मृत्युर्देवैर्वैकुण्ठसंश्रयैः
तावानय समं गोपैर्नन्दाद्यैः साभ्युपायनैः
घातयिष्य इहानीतौ कालकल्पेन हस्तिना
यदि मुक्तौ ततो मल्लैर्घातये वैद्युतोपमैः
तयोर्निहतयोस्तप्तान्वसुदेवपुरोगमान्‌
तद्बन्धून्निहनिष्यामि वृष्णिभोजदशार्हकान्‌
उग्रसेनं च पितरं स्थविरं राज्यकामुकं
तद्भ्रातरं देवकं च ये चान्ये विद्विषो मम
ततश्चैषा मही मित्र
भवित्री नष्टकण्टका
जरासन्धो मम गुरुर्द्विविदो दयितः सखा
शम्बरो नरको बाणो मय्येव कृतसौहृदाः
तैरहं सुरपक्षीयान्हत्वा भोक्ष्ये महीं नृपान्‌
एतज्ज्ञात्वानय क्षिप्रं रामकृष्णाविहार्भकौ
धनुर्मखनिरीक्षार्थं द्रष्टुं यदुपुरश्रियम्‌
श्रीअक्रूर उवाच
राजन्मनीषितं सध्र्यक्तव स्वावद्यमार्जनम्‌
सिद्ध्यसिद्ध्योः समं कुर्याद्दैवं हि फलसाधनम्‌
मनोरथान्करोत्युच्चैर्जनो दैवहतानपि
युज्यते हर्षशोकाभ्यां तथाप्याज्ञां करोमि ते
श्रीशुक उवाच
एवमादिश्य चाक्रूरं मन्त्रिणश्च विषृज्य सः
प्रविवेश गृहं कंसस्तथाक्रूरः स्वमालयम्‌
श्रीशुक उवाच
केशी तु कंसप्रहितः खुरैर्महीं
महाहयो निर्जरयन्मनोजवः
सटावधूताभ्रविमानसङ्कुलं
कुर्वन्नभो हेषितभीषिताखिलः
तं त्रासयन्तं भगवान्स्वगोकुलं
तद्धेषितैर्वालविघूर्णिताम्बुदम्‌
आत्मानमाजौ मृगयन्तमग्रणीर्‌
उपाह्वयत्स व्यनदन्मृगेन्द्रवत्‌
स तं निशाम्याभिमुखो मखेन खं
पिबन्निवाभ्यद्रवदत्यमर्षणः
जघान पद्भ्यामरविन्दलोचनं
दुरासदश्चण्डजवो दुरत्ययः
तद्वञ्चयित्वा तमधोक्षजो रुषा प्रगृह्य दोर्भ्यां परिविध्य पादयोः
सावज्ञमुत्सृज्य धनुःशतान्तरे यथोरगं तार्क्ष्यसुतो व्यवस्थितः
सः लब्धसंज्ञः पुनरुत्थितो रुषा
व्यादाय केशी तरसापतद्धरिम्‌
सोऽप्यस्य वक्त्रे भुजमुत्तरं स्मयन्‌
प्रवेशयामास यथोरगं बिले
दन्ता निपेतुर्भगवद्भुजस्पृशस्‌
ते केशिनस्तप्तमयस्पृशो यथा
बाहुश्च तद्देहगतो महात्मनो
यथामयः संववृधे उपेक्षितः
समेधमानेन स कृष्णबाहुना निरुद्धवायुश्चरणांश्च विक्षिपन्‌
प्रस्विन्नगात्रः परिवृत्तलोचनः पपात लण्डं विसृजन्क्षितौ व्यसुः
तद्देहतः कर्कटिकाफलोपमाद्व्यसोरपाकृष्य भुजं महाभुजः
अविस्मितोऽयत्नहतारिकः सुरैः प्रसूनवर्षैर्वर्षद्भिरीडितः
देवर्षिरुपसङ्गम्य भागवतप्रवरो नृप
कृष्णमक्लिष्टकर्माणं रहस्येतदभाषत
कृष्ण कृष्णाप्रमेयात्मन्योगेश जगदीश्वर
वासुदेवाखिलावास सात्वतां प्रवर प्रभो
त्वमात्मा सर्वभूतानामेको ज्योतिरिवैधसाम्‌
गूढो गुहाशयः साक्षी महापुरुष ईश्वरः
आत्मनात्माश्रयः पूर्वं मायया ससृजे गुणान्‌
तैरिदं सत्यसङ्कल्पः सृजस्यत्स्यवसीश्वरः
स त्वं भूधरभूतानां दैत्यप्रमथरक्षसाम्‌
अवतीर्णो विनाशाय साधुनां रक्षणाय च
दिष्ट्या ते निहतो दैत्यो लीलयायं हयाकृतिः
यस्य हेषितसन्त्रस्तास्त्यजन्त्यनिमिषा दिवम्‌
चाणूरं मुष्टिकं चैव मल्लानन्यांश्च हस्तिनम्‌
कंसं च निहतं द्रक्ष्ये परश्वोऽहनि ते विभो
तस्यानु शङ्खयवन मुराणां नरकस्य च
पारिजातापहरणमिन्द्रस्य च पराजयम्‌
उद्वाहं वीरकन्यानां वीर्यशुल्कादिलक्षणम्‌
नृगस्य मोक्षणं शापाद्द्वारकायां जगत्पते
स्यमन्तकस्य च मणेरादानं सह भार्यया
मृतपुत्रप्रदानं च ब्राह्मणस्य स्वधामतः
पौण्ड्रकस्य वधं पश्चात्काशिपुर्याश्च दीपनम्‌
दन्तवक्रस्य निधनं चैद्यस्य च महाक्रतौ
यानि चान्यानि वीर्याणि द्वारकामावसन्भवान्‌
कर्ता द्रक्ष्याम्यहं तानि गेयानि कविभिर्भुवि
अथ ते कालरूपस्य क्षपयिष्णोरमुष्य वै
अक्षौहिणीनां निधनं द्रक्ष्याम्यर्जुनसारथेः
विशुद्धविज्ञानघनं स्वसंस्थया
समाप्तसर्वार्थममोघवाञ्छितम्‌
स्वतेजसा नित्यनिवृत्तमाया
गुणप्रवाहं भगवन्तमीमहि
त्वामीश्वरं स्वाश्रयमात्ममायया विनिर्मिताशेषविशेषकल्पनम्‌
क्रीडार्थमद्यात्तमनुष्यविग्रहं नतोऽस्मि धुर्यं यदुवृष्णिसात्वताम्‌
श्रीशुक उवाच
एवं यदुपतिं कृष्णं भागवतप्रवरो मुनिः
प्रणिपत्याभ्यनुज्ञातो ययौ तद्दर्शनोत्सवः
भगवानपि गोविन्दो हत्वा केशिनमाहवे
पशूनपालयत्पालैः प्रीतैर्व्रजसुखावहः
एकदा ते पशून्पालाश्चारयन्तोऽद्रिसानुषु
चक्रुर्निलायनक्रीडाश्चोरपालापदेशतः
तत्रासन्कतिचिच्चोराः पालाश्च कतिचिन्नृप
मेषायिताश्च तत्रैके विजह्रुरकुतोभयाः
मयपुत्रो महामायो व्योमो गोपालवेषधृक्‌
मेषायितानपोवाह प्रायश्चोरायितो बहून्‌
गिरिदर्यां विनिक्षिप्य नीतं नीतं महासुरः
शिलया पिदधे द्वारं चतुःपञ्चावशेषिताः
तस्य तत्कर्म विज्ञाय कृष्णः शरणदः सताम्‌
गोपान्नयन्तं जग्राह वृकं हरिरिवौजसा
स निजं रूपमास्थाय गिरीन्द्रसदृशं बली
इच्छन्विमोक्तुमात्मानं नाशक्नोद्ग्रहणातुरः
तं निगृह्याच्युतो दोर्भ्यां पातयित्वा महीतले
पश्यतां दिवि देवानां पशुमारममारयत्‌
गुहापिधानं निर्भिद्य गोपान्निःसार्य कृच्छ्रतः
स्तूयमानः सुरैर्गोपैः प्रविवेश स्वगोकुलम्‌
श्रीशुक उवाच
अक्रूरोऽपि च तां रात्रिं मधुपुर्यां महामतिः
उषित्वा रथमास्थाय प्रययौ नन्दगोकुलम्‌
गच्छन्पथि महाभागो भगवत्यम्बुजेक्षणे
भक्तिं परामुपगत एवमेतदचिन्तयत्‌
किं मयाचरितं भद्रं किं तप्तं परमं तपः
किं वाथाप्यर्हते दत्तं यद्द्रक्ष्याम्यद्य केशवम्‌
ममैतद्दुर्लभं मन्य उत्तमःश्लोकदर्शनम्‌
विषयात्मनो यथा ब्रह्म कीर्तनं शूद्रजन्मनः
मैवं ममाधमस्यापि स्यादेवाच्युतदर्शनम्‌
ह्रियमाणः कलनद्या क्वचित्तरति कश्चन
ममाद्यामङ्गलं नष्टं फलवांश्चैव मे भवः
यन्नमस्ये भगवतो योगिध्येयान्घ्रिपङ्कजम्‌
कंसो बताद्याकृत मेऽत्यनुग्रहं द्रक्ष्येऽङ्घ्रिपद्मं प्रहितोऽमुना हरेः
कृतावतारस्य दुरत्ययं तमः पूर्वेऽतरन्यन्नखमण्डलत्विषा
यदर्चितं ब्रह्मभवादिभिः सुरैः
श्रिया च देव्या मुनिभिः ससात्वतैः
गोचारणायानुचरैश्चरद्वने
यद्गोपिकानां कुचकुङ्कुमाङ्कितम्‌
द्रक्ष्यामि नूनं सुकपोलनासिकं स्मितावलोकारुणकञ्जलोचनम्‌
मुखं मुकुन्दस्य गुडालकावृतं प्रदक्षिणं मे प्रचरन्ति वै मृगाः
अप्यद्य विष्णोर्मनुजत्वमीयुषो भारावताराय भुवो निजेच्छया
लावण्यधाम्नो भवितोपलम्भनं मह्यं न न स्यात्फलमञ्जसा दृशः
य ईक्षिताहंरहितोऽप्यसत्सतोः स्वतेजसापास्ततमोभिदाभ्रमः
स्वमाययात्मन्रचितैस्तदीक्षया प्राणाक्षधीभिः सदनेष्वभीयते
यस्याखिलामीवहभिः सुमङ्गलैः वाचो विमिश्रा गुणकर्मजन्मभिः
प्राणन्ति शुम्भन्ति पुनन्ति वै जगत्यास्तद्विरक्ताः शवशोभना मताः
स चावतीर्णः किल सत्वतान्वये स्वसेतुपालामरवर्यशर्मकृत्‌
यशो वितन्वन्व्रज आस्त ईश्वरो गायन्ति देवा यदशेषमङ्गलम्‌
तं त्वद्य नूनं महतां गतिं गुरुं
त्रैलोक्यकान्तं दृशिमन्महोत्सवम्‌
रूपं दधानं श्रिय ईप्सितास्पदं
द्रक्ष्ये ममासन्नुषसः सुदर्शनाः
अथावरूढः सपदीशयो रथात्प्रधानपुंसोश्चरणं स्वलब्धये
धिया धृतं योगिभिरप्यहं ध्रुवं नमस्य आभ्यां च सखीन्वनौकसः
अप्यङ्घ्रिमूले पतितस्य मे विभुः
शिरस्यधास्यन्निजहस्तपङ्कजम्‌
दत्ताभयं कालभुजाङ्गरंहसा
प्रोद्वेजितानां शरणैषिणां णृनाम्‌
समर्हणं यत्र निधाय कौशिकस्तथा बलिश्चाप जगत्त्रयेन्द्रताम्‌
यद्वा विहारे व्रजयोषितां श्रमं स्पर्शेन सौगन्धिकगन्ध्यपानुदत्‌
न मय्युपैष्यत्यरिबुद्धिमच्युतः
कंसस्य दूतः प्रहितोऽपि विश्वदृक्‌
योऽन्तर्बहिश्चेतस एतदीहितं
क्षेत्रज्ञ ईक्षत्यमलेन चक्षुषा
अप्यङ्घ्रिमूलेऽवहितं कृताञ्जलिं
मामीक्षिता सस्मितमार्द्रया दृशा
सपद्यपध्वस्तसमस्तकिल्बिषो
वोढा मुदं वीतविशङ्क ऊर्जिताम्‌
सुहृत्तमं ज्ञातिमनन्यदैवतं दोर्भ्यां बृहद्भ्यां परिरप्स्यतेऽथ माम्‌
आत्मा हि तीर्थीक्रियते तदैव मे बन्धश्च कर्मात्मक उच्छ्वसित्यतः
लब्ध्वाङ्गसङ्गम्प्रणतम्कृताञ्जलिं
मां वक्ष्यतेऽक्रूर ततेत्युरुश्रवाः
तदा वयं जन्मभृतो महीयसा
नैवादृतो यो धिगमुष्य जन्म तत्‌
न तस्य कश्चिद्दयितः सुहृत्तमो न चाप्रियो द्वेष्य उपेक्ष्य एव वा
तथापि भक्तान्भजते यथा तथा सुरद्रुमो यद्वदुपाश्रितोऽर्थदः
किं चाग्रजो मावनतं यदूत्तमः स्मयन्परिष्वज्य गृहीतमञ्जलौ
गृहं प्रवेष्याप्तसमस्तसत्कृतं सम्प्रक्ष्यते कंसकृतं स्वबन्धुषु
श्रीशुक उवाच
इति सञ्चिन्तयन्कृष्णं श्वफल्कतनयोऽध्वनि
रथेन गोकुलं प्राप्तः सूर्यश्चास्तगिरिं नृप
पदानि तस्याखिललोकपाल किरीटजुष्टामलपादरेणोः
ददर्श गोष्ठे क्षितिकौतुकानि विलक्षितान्यब्जयवाङ्कुशाद्यैः
तद्दर्शनाह्लादविवृद्धसम्भ्रमः
प्रेम्णोर्ध्वरोमाश्रुकलाकुलेक्षणः
रथादवस्कन्द्य स तेष्वचेष्टत
प्रभोरमून्यङ्घ्रिरजांस्यहो इति
देहंभृतामियानर्थो हित्वा दम्भं भियं शुचम्‌
सन्देशाद्यो हरेर्लिङ्ग दर्शनश्रवणादिभिः
ददर्श कृष्णं रामं च व्रजे गोदोहनं गतौ
पीतनीलाम्बरधरौ शरदम्बुरहेक्षणौ
किशोरौ श्यामलश्वेतौ श्रीनिकेतौ बृहद्भुजौ
सुमुखौ सुन्दरवरौ बलद्विरदविक्रमौ
ध्वजवज्राङ्कुशाम्भोजैश्चिह्नितैरङ्घ्रिभिर्व्रजम्‌
शोभयन्तौ महात्मानौ सानुक्रोशस्मितेक्षणौ
उदाररुचिरक्रीडौ स्रग्विणौ वनमालिनौ
पुण्यगन्धानुलिप्ताङ्गौ स्नातौ विरजवाससौ
प्रधानपुरुषावाद्यौ जगद्धेतू जगत्पती
अवतीर्णौ जगत्यर्थे स्वांशेन बलकेशवौ
दिशो वितिमिरा राजन्कुर्वाणौ प्रभया स्वया
यथा मारकतः शैलो रौप्यश्च कनकाचितौ
रथात्तूर्णमवप्लुत्य सोऽक्रूरः स्नेहविह्वलः
पपात चरणोपान्ते दण्डवद्रामकृष्णयोः
भगवद्दर्शनाह्लाद बाष्पपर्याकुलेक्षणः
पुलकचिताङ्ग औत्कण्ठ्यात्स्वाख्याने नाशकन्नृप
भगवांस्तमभिप्रेत्य रथाङ्गाङ्कितपाणिना
परिरेभेऽभ्युपाकृष्य प्रीतः प्रणतवत्सलः
सङ्कर्षणश्च प्रणतमुपगुह्य महामनाः
गृहीत्वा पाणिना पाणी अनयत्सानुजो गृहम्‌
पृष्ट्वाथ स्वागतं तस्मै निवेद्य च वरासनम्‌
प्रक्षाल्य विधिवत्पादौ मधुपर्कार्हणमाहरत्‌
निवेद्य गां चातिथये संवाह्य श्रान्तमाडृतः
अन्नं बहुगुणं मेध्यं श्रद्धयोपाहरद्विभुः
तस्मै भुक्तवते प्रीत्या रामः परमधर्मवित्‌
मखवासैर्गन्धमाल्यैः परां प्रीतिं व्यधात्पुनः
पप्रच्छ सत्कृतं नन्दः कथं स्थ निरनुग्रहे
कंसे जीवति दाशार्ह सौनपाला इवावयः
योऽवधीत्स्वस्वसुस्तोकान्क्रोशन्त्या असुतृप्खलः
किं नु स्वित्तत्प्रजानां वः कुशलं विमृशामहे
इत्थं सूनृतया वाचा नन्देन सुसभाजितः
अक्रूरः परिपृष्टेन जहावध्वपरिश्रमम्‌
श्रीशुक उवाच
सुखोपविष्टः पर्यङ्के रमकृष्णोरुमानितः
लेभे मनोरथान्सर्वान्पथि यान्स चकार ह
किमलभ्यं भगवति प्रसन्ने श्रीनिकेतने
तथापि तत्परा राजन्न हि वाञ्छन्ति किञ्चन
सायन्तनाशनं कृत्वा भगवान्देवकीसुतः
सुहृत्सु वृत्तं कंसस्य पप्रच्छान्यच्चिकीर्षितम्‌
श्रीभगवानुवाच
तात सौम्यागतः कच्चित्स्वागतं भद्रमस्तु वः
अपि स्वज्ञातिबन्धूनामनमीवमनामयम्‌
किं नु नः कुशलं पृच्छे एधमाने कुलामये
कंसे मातुलनाम्नाङ्ग स्वानां नस्तत्प्रजासु च
अहो अस्मदभूद्भूरि पित्रोर्वृजिनमार्ययोः
यद्धेतोः पुत्रमरणं यद्धेतोर्बन्धनं तयोः
दिष्ट्याद्य दर्शनं स्वानां मह्यं वः सौम्य काङ्क्षितम्‌
सञ्जातं वर्ण्यतां तात तवागमनकारणम्‌
श्रीशुक उवाच
पृष्टो भगवता सर्वं वर्णयामास माधवः
वैरानुबन्धं यदुषु वसुदेववधोद्यमम्‌
यत्सन्देशो यदर्थं वा दूतः सम्प्रेषितः स्वयम्‌
यदुक्तं नारदेनास्य स्वजन्मानकदुन्दुभेः
श्रुत्वाक्रूरवचः कृष्णो बलश्च परवीरहा
प्रहस्य नन्दं पितरं राज्ञा दिष्टं विजज्ञतुः
गोपान्समादिशत्सोऽपि गृह्यतां सर्वगोरसः
उपायनानि गृह्णीध्वं युज्यन्तां शकटानि च
यास्यामः श्वो मधुपुरीं दास्यामो नृपते रसान्‌
द्रक्ष्यामः सुमहत्पर्व यान्ति जानपदाः किल
एवमाघोषयत्क्षत्रा नन्दगोपः स्वगोकुले
गोप्यस्तास्तदुपश्रुत्य बभूवुर्व्यथिता भृशम्‌
रामकृष्णौ पुरीं नेतुमक्रूरं व्रजमागतम्‌
काश्चित्तत्कृतहृत्ताप श्वासम्लानमुखश्रियः
स्रंसद्दुकूलवलय केशग्रन्थ्यश्च काश्चन
अन्याश्च तदनुध्यान निवृत्ताशेषवृत्तयः
नाभ्यजानन्निमं लोकमात्मलोकं गता इव
स्मरन्त्यश्चापराः शौरेरनुरागस्मितेरिताः
हृदिस्पृशश्चित्रपदा गिरः सम्मुमुहुः स्त्रियः
गतिं सुललितां चेष्टां स्निग्धहासावलोकनम्‌
शोकापहानि नर्माणि प्रोद्दामचरितानि च
चिन्तयन्त्यो मुकुन्दस्य भीता विरहकातराः
समेताः सङ्घशः प्रोचुरश्रुमुख्योऽच्युताशयाः
श्रीगोप्य ऊचुः
अहो विधातस्तव न क्वचिद्दया संयोज्य मैत्र्या प्रणयेन देहिनः
तांश्चाकृतार्थान्वियुनङ्क्ष्यपार्थकं विक्रीडितं तेऽर्भकचेष्टितं यथा
यस्त्वं प्रदर्श्यासितकुन्तलावृतं
मुकुन्दवक्त्रं सुकपोलमुन्नसम्‌
शोकापनोदस्मितलेशसुन्दरं
करोषि पारोक्ष्यमसाधु ते कृतम्‌
क्रूरस्त्वमक्रूरसमाख्यया स्म नश्‌
चक्षुर्हि दत्तं हरसे बताज्ञवत्‌
येनैकदेशेऽखिलसर्गसौष्ठवं
त्वदीयमद्राक्ष्म वयं मधुद्विषः
न नन्दसूनुः क्षणभङ्गसौहृदः
समीक्षते नः स्वकृतातुरा बत
विहाय गेहान्स्वजनान्सुतान्पतींस्‌
तद्दास्यमद्धोपगता नवप्रियः
सुखं प्रभाता रजनीयमाशिषः सत्या बभूवुः पुरयोषितां ध्रुवम्‌
याः सम्प्रविष्टस्य मुखं व्रजस्पतेः पास्यन्त्यपाङ्गोत्कलितस्मितासवम्‌
तासां मुकुन्दो मधुमञ्जुभाषितैर्‌
गृहीतचित्तः परवान्मनस्व्यपि
कथं पुनर्नः प्रतियास्यतेऽबला
ग्राम्याः सलज्जस्मितविभ्रमैर्भ्रमन्‌
अद्य ध्रुवं तत्र दृशो भविष्यते दाशार्हभोजान्धकवृष्णिसात्वताम्‌
महोत्सवः श्रीरमणं गुणास्पदं द्रक्ष्यन्ति ये चाध्वनि देवकीसुतम्‌
मैतद्विधस्याकरुणस्य नाम भूदक्रूर इत्येतदतीव दारुणः
योऽसावनाश्वास्य सुदुःखितम्जनं प्रियात्प्रियं नेष्यति पारमध्वनः
अनार्द्रधीरेष समास्थितो रथं तमन्वमी च त्वरयन्ति दुर्मदाः
गोपा अनोभिः स्थविरैरुपेक्षितं दैवं च नोऽद्य प्रतिकूलमीहते
निवारयामः समुपेत्य माधवं किं नोऽकरिष्यन्कुलवृद्धबान्धवाः
मुकुन्दसङ्गान्निमिषार्धदुस्त्यजाद्दैवेन विध्वंसितदीनचेतसाम्‌
यस्यानुरागललितस्मितवल्गुमन्त्र
लीलावलोकपरिरम्भणरासगोष्ठाम्‌
नीताः स्म नः क्षणमिव क्षणदा विना तं
गोप्यः कथं न्वतितरेम तमो दुरन्तम्‌
योऽह्नः क्षये व्रजमनन्तसखः परीतो
गोपैर्विशन्खुररजश्छुरितालकस्रक्‌
वेणुं क्वणन्स्मितकताक्षनिरीक्षणेन
चित्तं क्षिणोत्यमुमृते नु कथं भवेम
श्रीशुक उवाच
एवं ब्रुवाणा विरहातुरा भृशं व्रजस्त्रियः कृष्णविषक्तमानसाः
विसृज्य लज्जां रुरुदुः स्म सुस्वरं गोविन्द दामोदर माधवेति
स्त्रीणामेवं रुदन्तीनामुदिते सवितर्यथ
अक्रूरश्चोदयामास कृतमैत्रादिको रथम्‌
गोपास्तमन्वसज्जन्त नन्दाद्याः शकटैस्ततः
आदायोपायनं भूरि कुम्भान्गोरससम्भृतान्‌
गोप्यश्च दयितं कृष्णमनुव्रज्यानुरञ्जिताः
प्रत्यादेशं भगवतः काङ्क्षन्त्यश्चावतस्थिरे
तास्तथा तप्यतीर्वीक्ष्य स्वप्रस्थाणे यदूत्तमः
सान्त्वयामस सप्रेमैरायास्य इति दौत्यकैः
यावदालक्ष्यते केतुर्यावद्रेणू रथस्य च
अनुप्रस्थापितात्मानो लेख्यानीवोपलक्षिताः
ता निराशा निववृतुर्गोविन्दविनिवर्तने
विशोका अहनी निन्युर्गायन्त्यः प्रियचेष्टितम्‌
भगवानपि सम्प्राप्तो रामाक्रूरयुतो नृप
रथेन वायुवेगेन कालिन्दीमघनाशिनीम्‌
तत्रोपस्पृश्य पानीयं पीत्वा मृष्टं मणिप्रभम्‌
वृक्षषण्डमुपव्रज्य सरामो रथमाविशत्‌
अक्रूरस्तावुपामन्त्र्य निवेश्य च रथोपरि
कालिन्द्या ह्रदमागत्य स्नानं विधिवदाचरत्‌
निमज्ज्य तस्मिन्सलिले जपन्ब्रह्म सनातनम्‌
तावेव ददृशेऽक्रूरो रामकृष्णौ समन्वितौ
तौ रथस्थौ कथमिह सुतावानकदुन्दुभेः
तर्हि स्वित्स्यन्दने न स्त इत्युन्मज्ज्य व्यचष्ट सः
तत्रापि च यथापूर्वमासीनौ पुनरेव सः
न्यमज्जद्दर्शनं यन्मे मृषा किं सलिले तयोः
भूयस्तत्रापि सोऽद्राक्षीत्स्तूयमानमहीश्वरम्‌
सिद्धचारणगन्धर्वैरसुरैर्नतकन्धरैः
सहस्रशिरसं देवं सहस्रफणमौलिनम्‌
नीलाम्बरं विसश्वेतं शृङ्गैः श्वेतमिव स्थितम्‌
तस्योत्सङ्गे घनस्यामं पीतकौशेयवाससम्‌
पुरुषं चतुर्भुजं शान्तम्पद्मपत्रारुणेक्षणम्‌
चारुप्रसन्नवदनं चारुहासनिरीक्षणम्‌
सुभ्रून्नसं चरुकर्णं सुकपोलारुणाधरम्‌
प्रलम्बपीवरभुजं तुङ्गांसोरःस्थलश्रियम्‌
कम्बुकण्ठं निम्ननाभिं वलिमत्पल्लवोदरम्‌
बृहत्कतिततश्रोणि करभोरुद्वयान्वितम्‌
चारुजानुयुगं चारु जङ्घायुगलसंयुतम्‌
तुङ्गगुल्फारुणनख व्रातदीधितिभिर्वृतम्‌
नवाङ्गुल्यङ्गुष्ठदलैर्विलसत्पादपङ्कजम्‌
सुमहार्हमणिव्रात किरीटकटकाङ्गदैः
कटिसूत्रब्रह्मसूत्र हारनूपुरकुण्डलैः
भ्राजमानं पद्मकरं शङ्खचक्रगदाधरम्‌
श्रीवत्सवक्षसं भ्राजत्कौस्तुभं वनमालिनम्‌
सुनन्दनन्दप्रमुखैः पर्षदैः सनकादिभिः
सुरेशैर्ब्रह्मरुद्राद्यैर्नवभिश्च द्विजोत्तमैः
प्रह्रादनारदवसु प्रमुखैर्भागवतोत्तमैः
स्तूयमानं पृथग्भावैर्वचोभिरमलात्मभिः
श्रिया पुष्ट्या गिरा कान्त्या कीर्त्या तुष्ट्येलयोर्जया
विद्ययाविद्यया शक्त्या मायया च निषेवितम्‌
विलोक्य सुभृशं प्रीतो भक्त्या परमया युतः
हृष्यत्तनूरुहो भाव परिक्लिन्नात्मलोचनः
गिरा गद्गदयास्तौषीत्सत्त्वमालम्ब्य सात्वतः
प्रणम्य मूर्ध्नावहितः कृताञ्जलिपुटः शनैः
श्रीअक्रूर उवाच
नतोऽस्म्यहं त्वाखिलहेतुहेतुं नारायणं पूरुषमाद्यमव्ययम्‌
यन्नाभिजातादरविन्दकोषाद्ब्रह्माविरासीद्यत एष लोकः
भूस्तोयमग्निः पवनं खमादिर्महानजादिर्मन इन्द्रियाणि
सर्वेन्द्रियार्था विबुधाश्च सर्वे ये हेतवस्ते जगतोऽङ्गभूताः
नैते स्वरूपं विदुरात्मनस्ते ह्यजादयोऽनात्मतया गृहीतः
अजोऽनुबद्धः स गुणैरजाया गुणात्परं वेद न ते स्वरूपम्‌
त्वां योगिनो यजन्त्यद्धा महापुरुषमीश्वरम्‌
साध्यात्मं साधिभूतं च साधिदैवं च साधवः
त्रय्या च विद्यया केचित्त्वां वै वैतानिका द्विजाः
यजन्ते विततैर्यज्ञैर्नानारूपामराख्यया
एके त्वाखिलकर्माणि सन्न्यस्योपशमं गताः
ज्ञानिनो ज्ञानयज्ञेन यजन्ति ज्ञानविग्रहम्‌
अन्ये च संस्कृतात्मानो विधिनाभिहितेन ते
यजन्ति त्वन्मयास्त्वां वै बहुमूर्त्येकमूर्तिकम्‌
त्वामेवान्ये शिवोक्तेन मार्गेण शिवरूपिणम्‌
बह्वाचार्यविभेदेन भगवन्तर्नुपासते
सर्व एव यजन्ति त्वां सर्वदेवमयेश्वरम्‌
येऽप्यन्यदेवताभक्ता यद्यप्यन्यधियः प्रभो
यथाद्रिप्रभवा नद्यः पर्जन्यापूरिताः प्रभो
विशन्ति सर्वतः सिन्धुं तद्वत्त्वां गतयोऽन्ततः
सत्त्वं रजस्तम इति भवतः प्रकृतेर्गुणाः
तेषु हि प्राकृताः प्रोता आब्रह्मस्थावरादयः
तुभ्यं नमस्ते त्वविषक्तदृष्टये
सर्वात्मने सर्वधियां च साक्षिणे
गुणप्रवाहोऽयमविद्यया कृतः
प्रवर्तते देवनृतिर्यगात्मसु
अग्निर्मुखं तेऽवनिरङ्घ्रिरीक्षणं
सूर्यो नभो नाभिरथो दिशः श्रुतिः
द्यौः कं सुरेन्द्रास्तव बाहवोऽर्णवाः
कुक्षिर्मरुत्प्राणबलं प्रकल्पितम्‌
रोमाणि वृक्षौषधयः शिरोरुहा
मेघाः परस्यास्थिनखानि तेऽद्रयः
निमेषणं रात्र्यहनी प्रजापतिर्‌
मेढ्रस्तु वृष्टिस्तव वीर्यमिष्यते
त्वय्यव्ययात्मन्पुरुषे प्रकल्पिता लोकाः सपाला बहुजीवसङ्कुलाः
यथा जले सञ्जिहते जलौकसोऽप्युदुम्बरे वा मशका मनोमये
यानि यानीह रूपाणि क्रीडनार्थं बिभर्षि हि
तैरामृष्टशुचो लोका मुदा गायन्ति ते यशः
नमः कारणमत्स्याय प्रलयाब्धिचराय च
हयशीर्ष्णे नमस्तुभ्यं मधुकैटभमृत्यवे
अकूपाराय बृहते नमो मन्दरधारिणे
क्षित्युद्धारविहाराय नमः शूकरमूर्तये
नमस्तेऽद्भुतसिंहाय साधुलोकभयापह
वामनाय नमस्तुभ्यं क्रान्तत्रिभुवनाय च
नमो भृगुणां पतये दृप्तक्षत्रवनच्छिदे
नमस्ते रघुवर्याय रावणान्तकराय च
नमस्ते वासुदेवाय नमः सङ्कर्षणाय च
प्रद्युम्नायनिरुद्धाय सात्वतां पतये नमः
नमो बुद्धाय शुद्धाय दैत्यदानवमोहिने
म्लेच्छप्रायक्षत्रहन्त्रे नमस्ते कल्किरूपिणे
भगवन्जीवलोकोऽयं मोहितस्तव मायया
अहं ममेत्यसद्ग्राहो भ्राम्यते कर्मवर्त्मसु
अहं चात्मात्मजागार दारार्थस्वजनादिषु
भ्रमामि स्वप्नकल्पेषु मूढः सत्यधिया विभो
अनित्यानात्मदुःखेषु विपर्ययमतिर्ह्यहम्‌
द्वन्द्वारामस्तमोविष्टो न जाने त्वात्मनः प्रियम्‌
यथाबुधो जलं हित्वा प्रतिच्छन्नं तदुद्भवैः
अभ्येति मृगतृष्णां वै तद्वत्त्वाहं पराङ्मुखः
नोत्सहेऽहं कृपणधीः कामकर्महतं मनः
रोद्धुं प्रमाथिभिश्चाक्षैर्ह्रियमाणमितस्ततः
सोऽहं तवाङ्घ्र्युपगतोऽस्म्यसतां दुरापं
तच्चाप्यहं भवदनुग्रह ईश मन्ये
पुंसो भवेद्यर्हि संसरणापवर्गस्‌
त्वय्यब्जनाभ सदुपासनया मतिः स्यात्‌
नमो विज्ञानमात्राय सर्वप्रत्ययहेतवे
पुरुषेशप्रधानाय ब्रह्मणेऽनन्तशक्तये
नमस्ते वासुदेवाय सर्वभूतक्षयाय च
हृषीकेश नमस्तुभ्यं प्रपन्नं पाहि मां प्रभो
श्रीशुक उवाच
स्तुवतस्तस्य भगवान्दर्शयित्वा जले वपुः
भूयः समाहरत्कृष्णो नटो नाट्यमिवात्मनः
सोऽपि चान्तर्हितं वीक्ष्य जलादुन्मज्य सत्वरः
कृत्वा चावश्यकं सर्वं विस्मितो रथमागमत्‌
तमपृच्छद्धृषीकेशः किं ते दृष्टमिवाद्भुतम्‌
भूमौ वियति तोये वा तथा त्वां लक्षयामहे
श्रीअक्रूर उवाच
अद्भुतानीह यावन्ति भूमौ वियति वा जले
त्वयि विश्वात्मके तानि किं मेऽदृष्टं विपश्यतः
यत्राद्भुतानि सर्वाणि भूमौ वियति वा जले
तं त्वानुपश्यतो ब्रह्मन्किं मे दृष्टमिहाद्भुतम्‌
इत्युक्त्वा चोदयामास स्यन्दनं गान्दिनीसुतः
मथुरामनयद्रामं कृष्णं चैव दिनात्यये
मार्गे ग्रामजना राजंस्तत्र तत्रोपसङ्गताः
वसुदेवसुतौ वीक्ष्य प्रीता दृष्टिं न चाददुः
तावद्व्रजौकसस्तत्र नन्दगोपादयोऽग्रतः
पुरोपवनमासाद्य प्रतीक्षन्तोऽवतस्थिरे
तान्समेत्याह भगवानक्रूरं जगदीश्वरः
गृहीत्वा पाणिना पाणिं प्रश्रितं प्रहसन्निव
भवान्प्रविशतामग्रे सहयानः पुरीं गृहम्‌
वयं त्विहावमुच्याथ ततो द्रक्ष्यामहे पुरीम्‌
श्रीअक्रूर उवाच
नाहं भवद्भ्यां रहितः प्रवेक्ष्ये मथुरां प्रभो
त्यक्तुं नार्हसि मां नाथ भक्तं ते भक्तवत्सल
आगच्छ याम गेहान्नः सनाथान्कुर्वधोक्षज
सहाग्रजः सगोपालैः सुहृद्भिश्च सुहृत्तम
पुनीहि पादरजसा गृहान्नो गृहमेधिनाम्‌
यच्छौचेनानुतृप्यन्ति पितरः साग्नयः सुराः
अवनिज्याङ्घ्रियुगलमासीत्श्लोक्यो बलिर्महान्‌
ऐश्वर्यमतुलं लेभे गतिं चैकान्तिनां तु या
आपस्तेऽङ्घ्र्यवनेजन्यस्त्रींल्लोकान्शुचयोऽपुनन्‌
शिरसाधत्त याः शर्वः स्वर्याताः सगरात्मजाः
देवदेव जगन्नाथ पुण्यश्रवणकीर्तन
यदूत्तमोत्तमःश्लोक नारायण नमोऽस्तु ते
श्रीभगवनुवाच
आयास्ये भवतो गेहमहमर्यसमन्वितः
यदुचक्रद्रुहं हत्वा वितरिष्ये सुहृत्प्रियम्‌
श्रीशुक उवाच
एवमुक्तो भगवता सोऽक्रूरो विमना इव
पुरीं प्रविष्टः कंसाय कर्मावेद्य गृहं ययौ
अथापराह्ने भगवान्कृष्णः सङ्कर्षणान्वितः
मथुरां प्राविशद्गोपैर्दिदृक्षुः परिवारितः
ददर्श तां स्फाटिकतुण्गगोपुर द्वारां बृहद्धेमकपाटतोरणाम्‌
ताम्रारकोष्ठां परिखादुरासदामुद्यानरम्योपवनोपशोभिताम्‌
सौवर्णशृङ्गाटकहर्म्यनिष्कुटैः श्रेणीसभाभिर्भवनैरुपस्कृताम्‌
वैदूर्यवज्रामलनीलविद्रुमैर्मुक्ताहरिद्भिर्वलभीषु वेदिषु
जुष्टेषु जालामुखरन्ध्रकुट्टिमेष्वाविष्टपारावतबर्हिनादिताम्‌
संसिक्तरथ्यापणमार्गचत्वरां प्रकीर्णमाल्याङ्कुरलाजतण्डुलाम्‌
आपूर्णकुम्भैर्दधिचन्दनोक्षितैः प्रसूनदीपावलिभिः सपल्लवैः
सवृन्दरम्भाक्रमुकैः सकेतुभिः स्वलङ्कृतद्वारगृहां सपट्टिकैः
तां सम्प्रविष्टौ वसुदेवनन्दनौ वृतौ वयस्यैर्नरदेववर्त्मना
द्रष्टुं समीयुस्त्वरिताः पुरस्त्रियो हर्म्याणि चैवारुरुहुर्नृपोत्सुकाः
काश्चिद्विपर्यग्धृतवस्त्रभूषणा
विस्मृत्य चैकं युगलेष्वथापराः
कृतैकपत्रश्रवनैकनूपुरा
नाङ्क्त्वा द्वितीयं त्वपराश्च लोचनम्‌
अश्नन्त्य एकास्तदपास्य सोत्सवा अभ्यज्यमाना अकृतोपमज्जनाः
स्वपन्त्य उत्थाय निशम्य निःस्वनं प्रपाययन्त्योऽर्भमपोह्य मातरः
मनांसि तासामरविन्दलोचनः प्रगल्भलीलाहसितावलोकैः
जहार मत्तद्विरदेन्द्रविक्रमो दृशां ददच्छ्रीरमणात्मनोत्सवम्‌
दृष्ट्वा मुहुः श्रुतमनुद्रुतचेतसस्तं
तत्प्रेक्षणोत्स्मितसुधोक्षणलब्धमानाः
आनन्दमूर्तिमुपगुह्य दृशात्मलब्धं
हृष्यत्त्वचो जहुरनन्तमरिन्दमाधिम्‌
प्रासादशिखरारूढाः प्रीत्युत्फुल्लमुखाम्बुजाः
अभ्यवर्षन्सौमनस्यैः प्रमदा बलकेशवौ
दध्यक्षतैः सोदपात्रैः स्रग्गन्धैरभ्युपायनैः
तावानर्चुः प्रमुदितास्तत्र तत्र द्विजातयः
ऊचुः पौरा अहो गोप्यस्तपः किमचरन्महत्‌
या ह्येतावनुपश्यन्ति नरलोकमहोत्सवौ
रजकं कञ्चिदायान्तं रङ्गकारं गदाग्रजः
दृष्ट्वायाचत वासांसि धौतान्यत्युत्तमानि च
देह्यावयोः समुचितान्यङ्ग वासांसि चार्हतोः
भविष्यति परं श्रेयो दातुस्ते नात्र संशयः
स याचितो भगवता परिपूर्णेन सर्वतः
साक्षेपं रुषितः प्राह भृत्यो राज्ञः सुदुर्मदः
ईदृशान्येव वासांसी नित्यं गिरिवनेचरः
परिधत्त किमुद्वृत्ता राजद्रव्याण्यभीप्सथ
याताशु बालिशा मैवं प्रार्थ्यं यदि जिजीवीषा
बध्नन्ति घ्नन्ति लुम्पन्ति दृप्तं राजकुलानि वै
एवं विकत्थमानस्य कुपितो देवकीसुतः
रजकस्य कराग्रेण शिरः कायादपातयत्‌
तस्यानुजीविनः सर्वे वासःकोशान्विसृज्य वै
दुद्रुवुः सर्वतो मार्गं वासांसि जगृहेऽच्युतः
वसित्वात्मप्रिये वस्त्रे कृष्णः सङ्कर्षणस्तथा
शेषाण्यादत्त गोपेभ्यो विसृज्य भुवि कानिचित्‌
ततस्तु वायकः प्रीतस्तयोर्वेषमकल्पयत्‌
विचित्रवर्णैश्चैलेयैराकल्पैरनुरूपतः
नानालक्षणवेषाभ्यां कृष्णरामौ विरेजतुः
स्वलङ्कृतौ बालगजौ पर्वणीव सितेतरौ
तस्य प्रसन्नो भगवान्प्रादात्सारूप्यमात्मनः
श्रियं च परमां लोके बलैश्वर्यस्मृतीन्द्रियम्‌
ततः सुदाम्नो भवनं मालाकारस्य जग्मतुः
तौ दृष्ट्वा स समुत्थाय ननाम शिरसा भुवि
तयोरासनमानीय पाद्यं चार्घ्यार्हणादिभिः
पूजां सानुगयोश्चक्रे स्रक्ताम्बूलानुलेपनैः
प्राह नः सार्थकं जन्म पावितं च कुलं प्रभो
पितृदेवर्षयो मह्यं तुष्टा ह्यागमनेन वाम्‌
भवन्तौ किल विश्वस्य जगतः कारणं परम्‌
अवतीर्णाविहांशेन क्षेमाय च भवाय च
न हि वां विषमा दृष्टिः सुहृदोर्जगदात्मनोः
समयोः सर्वभूतेषु भजन्तं भजतोरपि
तावज्ञापयतं भृत्यं किमहं करवाणि वाम्‌
पुंसोऽत्यनुग्रहो ह्येष भवद्भिर्यन्नियुज्यते
इत्यभिप्रेत्य राजेन्द्र सुदामा प्रीतमानसः
शस्तैः सुगन्धैः कुसुमैर्माला विरचिता ददौ
ताभिः स्वलङ्कृतौ प्रीतौ कृष्णरामौ सहानुगौ
प्रणताय प्रपन्नाय ददतुर्वरदौ वरान्‌
सोऽपि वव्रेऽचलां भक्तिं तस्मिन्नेवाखिलात्मनि
तद्भक्तेषु च सौहार्दं भूतेषु च दयां पराम्‌
इति तस्मै वरं दत्त्वा श्रियं चान्वयवर्धिनीम्‌
बलमायुर्यशः कान्तिं निर्जगाम सहाग्रजः
श्रीशुक उवाच
अथ व्रजन्राजपथेन माधवः स्त्रियं गृहीताङ्गविलेपभाजनाम्‌
विलोक्य कुब्जां युवतीं वराननां पप्रच्छ यान्तीं प्रहसन्रसप्रदः
का त्वं वरोर्वेतदु हानुलेपनं कस्याङ्गने वा कथयस्व साधु नः
देह्यावयोरङ्गविलेपमुत्तमं श्रेयस्ततस्ते न चिराद्भविष्यति
सैरन्ध्र्युवाच
दास्यस्म्यहं सुन्दर कंससम्मता
त्रिवक्रनामा ह्यनुलेपकर्मणि
मद्भावितं भोजपतेरतिप्रियं
विना युवां कोऽन्यतमस्तदर्हति
रूपपेशलमाधुर्य हसितालापवीक्षितैः
धर्षितात्मा ददौ सान्द्रमुभयोरनुलेपनम्‌
ततस्तावङ्गरागेण स्ववर्णेतरशोभिना
सम्प्राप्तपरभागेन शुशुभातेऽनुरञ्जितौ
प्रसन्नो भगवान्कुब्जां त्रिवक्रां रुचिराननाम्‌
ऋज्वीं कर्तुं मनश्चक्रे दर्शयन्दर्शने फलम्‌
पद्भ्यामाक्रम्य प्रपदे द्र्यङ्गुल्युत्तानपाणिना
प्रगृह्य चिबुकेऽध्यात्ममुदनीनमदच्युतः
सा तदर्जुसमानाङ्गी बृहच्छ्रोणिपयोधरा
मुकुन्दस्पर्शनात्सद्यो बभूव प्रमदोत्तमा
ततो रूपगुणौदार्य सम्पन्ना प्राह केशवम्‌
उत्तरीयान्तमकृष्य स्मयन्ती जातहृच्छया
एहि वीर गृहं यामो न त्वां त्यक्तुमिहोत्सहे
त्वयोन्मथितचित्तायाः प्रसीद पुरुषर्षभ
एवं स्त्रिया याच्यमानः कृष्णो रामस्य पश्यतः
मुखं वीक्ष्यानु गोपानां प्रहसंस्तामुवाच ह
एष्यामि ते गृहं सुभ्रु पुंसामाधिविकर्शनम्‌
साधितार्थोऽगृहाणां नः पान्थानां त्वं परायणम्‌
विसृज्य माध्व्या वाण्या ताम्व्रजन्मार्गे वणिक्पथैः
नानोपायनताम्बूल स्रग्गन्धैः साग्रजोऽर्चितः
तद्दर्शनस्मरक्षोभादात्मानं नाविदन्स्त्रियः
विस्रस्तवासःकवर वलया लेख्यमूर्तयः
ततः पौरान्पृच्छमानो धनुषः स्थानमच्युतः
तस्मिन्प्रविष्टो ददृशे धनुरैन्द्रमिवाद्भुतम्‌
पुरुषैर्बहुभिर्गुप्तमर्चितं परमर्द्धिमत्‌
वार्यमाणो नृभिः कृष्णः प्रसह्य धनुराददे
करेण वामेन सलीलमुद्धृतं सज्यं च कृत्वा निमिषेण पश्यताम्‌
नृणां विकृष्य प्रबभञ्ज मध्यतो यथेक्षुदण्डं मदकर्युरुक्रमः
धनुषो भज्यमानस्य शब्दः खं रोदसी दिशः
पूरयामास यं श्रुत्वा कंसस्त्रासमुपागमत्‌
तद्रक्षिणः सानुचरं कुपिता आततायिनः
गृहीतुकामा आवव्रुर्गृह्यतां वध्यतामिति
अथ तान्दुरभिप्रायान्विलोक्य बलकेशवौ
क्रुद्धौ धन्वन आदाय शकले तांश्च जघ्नतुः
बलं च कंसप्रहितं हत्वा शालामुखात्ततः
निष्क्रम्य चेरतुर्हृष्टौ निरीक्ष्य पुरसम्पदः
तयोस्तदद्भुतं वीर्यं निशाम्य पुरवासिनः
तेजः प्रागल्भ्यं रूपं च मेनिरे विबुधोत्तमौ
तयोर्विचरतोः स्वैरमादित्योऽस्तमुपेयिवान्‌
कृष्णरामौ वृतौ गोपैः पुराच्छकटमीयतुः
गोप्यो मुकुन्दविगमे विरहातुरा या आशासताशिष ऋता मधुपुर्यभूवन्‌
सम्पश्यतां पुरुषभूषणगात्रलक्ष्मीं हित्वेतरान्नु भजतश्चकमेऽयनं श्रीः
अवनिक्ताङ्घ्रियुगलौ भुक्त्वा क्षीरोपसेचनम्‌
ऊषतुस्तां सुखं रात्रिं ज्ञात्वा कंसचिकीर्षितम्‌
कंसस्तु धनुषो भङ्गं रक्षिणां स्वबलस्य च
वधं निशम्य गोविन्द रामविक्रीडितं परम्‌
दीर्घप्रजागरो भीतो दुर्निमित्तानि दुर्मतिः
बहून्यचष्टोभयथा मृत्योर्दौत्यकराणि च
अदर्शनं स्वशिरसः प्रतिरूपे च सत्यपि
असत्यपि द्वितीये च द्वैरूप्यं ज्योतिषां तथा
छिद्रप्रतीतिश्छायायां प्राणघोषानुपश्रुतिः
स्वर्णप्रतीतिर्वृक्षेषु स्वपदानामदर्शनम्‌
स्वप्ने प्रेतपरिष्वङ्गः खरयानं विषादनम्‌
यायान्नलदमाल्येकस्तैलाभ्यक्तो दिगम्बरः
अन्यानि चेत्थंभूतानि स्वप्नजागरितानि च
पश्यन्मरणसन्त्रस्तो निद्रां लेभे न चिन्तया
व्युष्टायां निशि कौरव्य सूर्ये चाद्भ्यः समुत्थिते
कारयामास वै कंसो मल्लक्रीडामहोत्सवम्‌
आनर्चुः पुरुषा रङ्गं तूर्यभेर्यश्च जघ्निरे
मञ्चाश्चालङ्कृताः स्रग्भिः पताकाचैलतोरणैः
तेषु पौरा जानपदा ब्रह्मक्षत्रपुरोगमाः
यथोपजोषं विविशू राजानश्च कृतासनाः
कंसः परिवृतोऽमात्यै राजमञ्च उपाविशत्‌
मण्डलेश्वरमध्यस्थो हृदयेन विदूयता
वाद्यमानेसु तूर्येषु मल्लतालोत्तरेषु च
मल्लाः स्वलङ्कृताः दृप्ताः सोपाध्यायाः समासत
चाणूरो मुष्टिकः कूतः शलस्तोशल एव च
त आसेदुरुपस्थानं वल्गुवाद्यप्रहर्षिताः
नन्दगोपादयो गोपा भोजराजसमाहुताः
निवेदितोपायनास्त एकस्मिन्मञ्च आविशन्‌
श्रीशुक उवाच
अथ कृष्णश्च रामश्च कृतशौचौ परन्तप
मल्लदुन्दुभिनिर्घोषं श्रुत्वा द्रष्टुमुपेयतुः
रङ्गद्वारं समासाद्य तस्मिन्नागमवस्थितम्‌
अपश्यत्कुवलयापीडं कृष्णोऽम्बष्ठप्रचोदितम्‌
बद्ध्वा परिकरं शौरिः समुह्य कुटिलालकान्‌
उवाच हस्तिपं वाचा मेघनादगभीरया
अम्बष्ठाम्बष्ठ मार्गं नौ देह्यपक्रम मा चिरम्‌
नो चेत्सकुञ्जरं त्वाद्य नयामि यमसादनम्‌
एवं निर्भर्त्सितोऽम्बष्ठः कुपितः कोपितं गजम्‌
चोदयामास कृष्णाय कालान्तकयमोपमम्‌
करीन्द्रस्तमभिद्रुत्य करेण तरसाग्रहीत्‌
कराद्विगलितः सोऽमुं निहत्याङ्घ्रिष्वलीयत
सङ्क्रुद्धस्तमचक्षाणो घ्राणदृष्टिः स केशवम्‌
परामृशत्पुष्करेण स प्रसह्य विनिर्गतः
पुच्छे प्रगृह्यातिबलं धनुषः पञ्चविंशतिम्‌
विचकर्ष यथा नागं सुपर्ण इव लीलया
स पर्यावर्तमानेन सव्यदक्षिणतोऽच्युतः
बभ्राम भ्राम्यमाणेन गोवत्सेनेव बालकः
ततोऽभिमखमभ्येत्य पाणिनाहत्य वारणम्‌
प्राद्रवन्पातयामास स्पृश्यमानः पदे पदे
स धावन्कृईदया भूमौ पतित्वा सहसोत्थितः
तम्मत्वा पतितं क्रुद्धो दन्ताभ्यां सोऽहनत्क्षितिम्‌
स्वविक्रमे प्रतिहते कुञ्जरेन्द्रोऽत्यमर्षितः
चोद्यमानो महामात्रैः कृष्णमभ्यद्रवद्रुषा
तमापतन्तमासाद्य भगवान्मधुसूदनः
निगृह्य पाणिना हस्तं पातयामास भूतले
पतितस्य पदाक्रम्य मृगेन्द्र इव लीलया
दन्तमुत्पाट्य तेनेभं हस्तिपांश्चाहनद्धरिः
मृतकं द्विपमुत्सृज्य दन्तपाणिः समाविशत्‌
अंसन्यस्तविषाणोऽसृङ्‌ मदबिन्दुभिरङ्कितः
विरूढस्वेदकणिका वदनाम्बुरुहो बभौ
वृतौ गोपैः कतिपयैर्बलदेवजनार्दनौ
रङ्गं विविशतू राजन्गजदन्तवरायुधौ
मल्लानामशनिर्नृणां नरवरः स्त्रीणां स्मरो मूर्तिमान्‌
गोपानां स्वजनोऽसतां क्षितिभुजां शास्ता स्वपित्रोः शिशुः
मृत्युर्भोजपतेर्विराडविदुषां तत्त्वं परं योगिनां
वृष्णीनां परदेवतेति विदितो रङ्गं गतः साग्रजः
हतं कुवलयापीडं दृष्ट्वा तावपि दुर्जयौ
कंसो मनस्यपि तदा भृशमुद्विविजे नृप
तौ रेजतू रङ्गगतौ महाभुजौ विचित्रवेषाभरणस्रगम्बरौ
यथा नटावुत्तमवेषधारिणौ मनः क्षिपन्तौ प्रभया निरीक्षताम्‌
निरीक्ष्य तावुत्तमपूरुषौ जना मञ्चस्थिता नागरराष्ट्रका नृप
प्रहर्षवेगोत्कलितेक्षणाननाः पपुर्न तृप्ता नयनैस्तदाननम्‌
पिबन्त इव चक्षुर्भ्यां लिहन्त इव जिह्वया
जिघ्रन्त इव नासाभ्यां श्लिष्यन्त इव बाहुभिः
ऊचुः परस्परं ते वै यथादृष्टं यथाश्रुतम्‌
तद्रूपगुणमाधुर्य प्रागल्भ्यस्मारिता इव
एतौ भगवतः साक्षाद्धरेर्नारायणस्य हि
अवतीर्णाविहांशेन वसुदेवस्य वेश्मनि
एष वै किल देवक्यां जातो नीतश्च गोकुलम्‌
कालमेतं वसन्गूढो ववृधे नन्दवेश्मनि
पूतनानेन नीतान्तं चक्रवातश्च दानवः
अर्जुनौ गुह्यकः केशी धेनुकोऽन्ये च तद्विधाः
गावः सपाला एतेन दावाग्नेः परिमोचिताः
कालियो दमितः सर्प इन्द्रश्च विमदः कृतः
सप्ताहमेकहस्तेन धृतोऽद्रिप्रवरोऽमुना
वर्षवाताशनिभ्यश्च परित्रातं च गोकुलम्‌
गोप्योऽस्य नित्यमुदित हसितप्रेक्षणं मुखम्‌
पश्यन्त्यो विविधांस्तापांस्तरन्ति स्माश्रमं मुदा
वदन्त्यनेन वंशोऽयं यदोः सुबहुविश्रुतः
श्रियं यशो महत्वं च लप्स्यते परिरक्षितः
अयं चास्याग्रजः श्रीमान्रामः कमललोचनः
प्रलम्बो निहतो येन वत्सको ये बकादयः
जनेष्वेवं ब्रुवाणेषु तूर्येषु निनदत्सु च
कृष्णरामौ समाभाष्य चाणूरो वाक्यमब्रवीत्‌
हे नन्दसूनो हे राम भवन्तौ वीरसम्मतौ
नियुद्धकुशलौ श्रुत्वा राज्ञाहूतौ दिदृक्षुणा
प्रियं राज्ञः प्रकुर्वत्यः श्रेयो विन्दन्ति वै प्रजाः
मनसा कर्मणा वाचा विपरीतमतोऽन्यथा
नित्यं प्रमुदिता गोपा वत्सपाला यथास्फुटम्‌
वनेषु मल्लयुद्धेन क्रीडन्तश्चारयन्ति गाः
तस्माद्राज्ञः प्रियं यूयं वयं च करवाम हे
भूतानि नः प्रसीदन्ति सर्वभूतमयो नृपः
तन्निशम्याब्रवीत्कृष्णो देशकालोचितं वचः
नियुद्धमात्मनोऽभीष्टं मन्यमानोऽभिनन्द्य च
प्रजा भोजपतेरस्य वयं चापि वनेचराः
करवाम प्रियं नित्यं तन्नः परमनुग्रहः
बाला वयं तुल्यबलैः क्रीडिष्यामो यथोचितम्‌
भवेन्नियुद्धं माधर्मः स्पृशेन्मल्लसभासदः
चाणूर उवाच
न बालो न किशोरस्त्वं बलश्च बलिनां वरः
लीलयेभो हतो येन सहस्रद्विपसत्त्वभृत्‌
तस्माद्भवद्भ्यां बलिभिर्योद्धव्यं नानयोऽत्र वै
मयि विक्रम वार्ष्णेय बलेन सह मुष्टिकः
श्रीशुक उवाच
एवं चर्चितसङ्कल्पो भगवान्मधुसूदनः
आससादाथ चणूरं मुष्ट्तिकं रोहिणीसुतः
हस्ताभ्यां हस्तयोर्बद्ध्वा पद्भ्यामेव च पादयोः
विचकर्षतुरन्योन्यं प्रसह्य विजिगीषया
अरत्नी द्वे अरत्निभ्यां जानुभ्यां चैव जानुनी
शिरः शीर्ष्णोरसोरस्तावन्योन्यमभिजघ्नतुः
परिभ्रामणविक्षेप परिरम्भावपातनैः
उत्सर्पणापसर्पणैश्चान्योन्यं प्रत्यरुन्धताम्‌
उत्थापनैरुन्नयनैश्चालनैः स्थापनैरपि
परस्परं जिगीषन्तावपचक्रतुरात्मनः
तद्बलाबलवद्युद्धं समेताः सर्वयोषितः
ऊचुः परस्परं राजन्सानुकम्पा वरूथशः
महानयं बताधर्म एषां राजसभासदाम्‌
ये बलाबलवद्युद्धं राज्ञोऽन्विच्छन्ति पश्यतः
क्व वज्रसारसर्वाङ्गौ मल्लौ शैलेन्द्रसन्निभौ
क्व चातिसुकुमाराङ्गौ किशोरौ नाप्तयौवनौ
धर्मव्यतिक्रमो ह्यस्य समाजस्य ध्रुवं भवेत्‌
यत्राधर्मः समुत्तिष्ठेन्न स्थेयं तत्र कर्हिचित्‌
न सभां प्रविशेत्प्राज्ञः सभ्यदोषाननुस्मरन्‌
अब्रुवन्विब्रुवन्नज्ञो नरः किल्बिषमश्नुते
वल्गतः शत्रुमभितः कृष्णस्य वदनाम्बुजम्‌
वीक्ष्यतां श्रमवार्युप्तं पद्मकोशमिवाम्बुभिः
किं न पश्यत रामस्य मुखमाताम्रलोचनम्‌
मुष्टिकं प्रति सामर्षं हाससंरम्भशोभितम्‌
पुण्या बत व्रजभुवो यदयं नृलिङ्ग
गूढः पुराणपुरुषो वनचित्रमाल्यः
गाः पालयन्सहबलः क्वणयंश्च वेणुं
विक्रीदयाञ्चति गिरित्ररमार्चिताङ्घ्रिः
गोप्यस्तपः किमचरन्यदमुष्य रूपं
लावण्यसारमसमोर्ध्वमनन्यसिद्धम्‌
दृग्भिः पिबन्त्यनुसवाभिनवं दुरापम्‌
एकान्तधाम यशसः श्रीय ऐश्वरस्य
या दोहनेऽवहनने मथनोपलेप प्रेङ्खेङ्खनार्भरुदितोक्षणमार्जनादौ
गायन्ति चैनमनुरक्तधियोऽश्रुकण्ठ्यो धन्या व्रजस्त्रिय उरुक्रमचित्तयानाः
प्रातर्व्रजाद्व्रजत आविशतश्च सायं
गोभिः समं क्वणयतोऽस्य निशम्य वेणुम्‌
निर्गम्य तूर्णमबलाः पथि भूरिपुण्याः
पश्यन्ति सस्मितमुखं सदयावलोकम्‌
एवं प्रभाषमाणासु स्त्रीषु योगेश्वरो हरिः
शत्रुं हन्तुं मनश्चक्रे भगवान्भरतर्षभ
सभयाः स्त्रीगिरः श्रुत्वा पुत्रस्नेहशुचातुरौ
पितरावन्वतप्येतां पुत्रयोरबुधौ बलम्‌
तैस्तैर्नियुद्धविधिभिर्विविधैरच्युतेतरौ
युयुधाते यथान्योन्यं तथैव बलमुष्टिकौ
भगवद्गात्रनिष्पातैर्वज्रनीष्पेषनिष्ठुरैः
चाणूरो भज्यमानाङ्गो मुहुर्ग्लानिमवाप ह
स श्येनवेग उत्पत्य मुष्टीकृत्य करावुभौ
भगवन्तं वासुदेवं क्रुद्धो वक्षस्यबाधत
नाचलत्तत्प्रहारेण मालाहत इव द्विपः
बाह्वोर्निगृह्य चाणूरं बहुशो भ्रामयन्हरिः
भूपृष्ठे पोथयामास तरसा क्षीण जीवितम्‌
विस्रस्ताकल्पकेशस्रगिन्द्रध्वज इवापतत्‌
तथैव मुष्टिकः पूर्वं स्वमुष्ट्याभिहतेन वै
बलभद्रेण बलिना तलेनाभिहतो भृशम्‌
प्रवेपितः स रुधिरमुद्वमन्मुखतोऽर्दितः
व्यसुः पपातोर्व्युपस्थे वाताहत इवाङ्घ्रिपः
ततः कूटमनुप्राप्तं रामः प्रहरतां वरः
अवधील्लीलया राजन्सावज्ञं वाममुष्टिना
तर्ह्येव हि शलः कृष्ण प्रपदाहतशीर्षकः
द्विधा विदीर्णस्तोशलक उभावपि निपेततुः
चाणूरे मुष्टिके कूटे शले तोशलके हते
शेषाः प्रदुद्रुवुर्मल्लाः सर्वे प्राणपरीप्सवः
गोपान्वयस्यानाकृष्य तैः संसृज्य विजह्रतुः
वाद्यमानेषु तूर्येषु वल्गन्तौ रुतनूपुरौ
जनाः प्रजहृषुः सर्वे कर्मणा रामकृष्णयोः
ऋते कंसं विप्रमुख्याः साधवः साधु साध्विति
हतेषु मल्लवर्येषु विद्रुतेषु च भोजराट्‌
न्यवारयत्स्वतूर्याणि वाक्यं चेदमुवाच ह
निःसारयत दुर्वृत्तौ वसुदेवात्मजौ पुरात्‌
धनं हरत गोपानां नन्दं बध्नीत दुर्मतिम्‌
वसुदेवस्तु दुर्मेधा हन्यतामाश्वसत्तमः
उग्रसेनः पिता चापि सानुगः परपक्षगः
एवं विकत्थमाने वै कंसे प्रकुपितोऽव्ययः
लघिम्नोत्पत्य तरसा मञ्चमुत्तुङ्गमारुहत्‌
तमाविशन्तमालोक्य मृत्युमात्मन आसनात्‌
मनस्वी सहसोत्थाय जगृहे सोऽसिचर्मणी
तं खड्गपाणिं विचरन्तमाशु श्येनं यथा दक्षिणसव्यमम्बरे
समग्रहीद्दुर्विषहोग्रतेजा यथोरगं तार्क्ष्यसुतः प्रसह्य
प्रगृह्य केशेषु चलत्किरीतं निपात्य रङ्गोपरि तुङ्गमञ्चात्‌
तस्योपरिष्टात्स्वयमब्जनाभः पपात विश्वाश्रय आत्मतन्त्रः
तं सम्परेतं विचकर्ष भूमौ हरिर्यथेभं जगतो विपश्यतः
हा हेति शब्दः सुमहांस्तदाभूदुदीरितः सर्वजनैर्नरेन्द्र
स नित्यदोद्विग्नधिया तमीश्वरं पिबन्नदन्वा विचरन्स्वपन्श्वसन्‌
ददर्श चक्रायुधमग्रतो यतस्तदेव रूपं दुरवापमाप
तस्यानुजा भ्रातरोऽष्टौ कङ्कन्यग्रोधकादयः
अभ्यधावन्नतिक्रुद्धा भ्रातुर्निर्वेशकारिणः
तथातिरभसांस्तांस्तु संयत्तान्रोहिणीसुतः
अहन्परिघमुद्यम्य पशूनिव मृगाधिपः
नेदुर्दुन्दुभयो व्योम्नि ब्रह्मेशाद्या विभूतयः
पुष्पैः किरन्तस्तं प्रीताः शशंसुर्ननृतुः स्त्रियः
तेषां स्त्रियो महाराज सुहृन्मरणदुःखिताः
तत्राभीयुर्विनिघ्नन्त्यः शीर्षाण्यश्रुविलोचनाः
शयानान्वीरशयायां पतीनालिङ्ग्य शोचतीः
विलेपुः सुस्वरं नार्यो विसृजन्त्यो मुहुः शुचः
हा नाथ प्रिय धर्मज्ञ करुणानाथवत्सल
त्वया हतेन निहता वयं ते सगृहप्रजाः
त्वया विरहिता पत्या पुरीयं पुरुषर्षभ
न शोभते वयमिव निवृत्तोत्सवमङ्गला
अनागसां त्वं भूतानां कृतवान्द्रोहमुल्बणम्‌
तेनेमां भो दशां नीतो भूतध्रुक्को लभेत शम्‌
सर्वेषामिह भूतानामेष हि प्रभवाप्ययः
गोप्ता च तदवध्यायी न क्वचित्सुखमेधते
श्रीशुक उवाच
राजयोषित आश्वास्य भगवांल्लोकभावनः
यामाहुर्लौकिकीं संस्थां हतानां समकारयत्‌
मातरं पितरं चैव मोचयित्वाथ बन्धनात्‌
कृष्णरामौ ववन्दाते शिरसा स्पृश्य पादयोः
देवकी वसुदेवश्च विज्ञाय जगदीश्वरौ
कृतसंवन्दनौ पुत्रौ सस्वजाते न शङ्कितौ
श्रीशुक उवाच
पितरावुपलब्धार्थौ विदित्वा पुरुषोत्तमः
मा भूदिति निजां मायां ततान जनमोहिनीम्‌
उवाच पितरावेत्य साग्रजः सात्वनर्षभः
प्रश्रयावनतः प्रीणन्नम्ब तातेति सादरम्‌
नास्मत्तो युवयोस्तात नित्योत्कण्ठितयोरपि
बाल्यपौगण्डकैशोराः पुत्राभ्यामभवन्क्वचित्‌
न लब्धो दैवहतयोर्वासो नौ भवदन्तिके
यां बालाः पितृगेहस्था विन्दन्ते लालिता मुदम्‌
सर्वार्थसम्भवो देहो जनितः पोषितो यतः
न तयोर्याति निर्वेशं पित्रोर्मर्त्यः शतायुषा
यस्तयोरात्मजः कल्प आत्मना च धनेन च
वृत्तिं न दद्यात्तं प्रेत्य स्वमांसं खादयन्ति हि
मातरं पितरं वृद्धं भार्यां साध्वीं सुतम्शिशुम्‌
गुरुं विप्रं प्रपन्नं च कल्पोऽबिभ्रच्छ्वसन्मृतः
तन्नावकल्पयोः कंसान्नित्यमुद्विग्नचेतसोः
मोघमेते व्यतिक्रान्ता दिवसा वामनर्चतोः
तत्क्षन्तुमर्हथस्तात मातर्नौ परतन्त्रयोः
अकुर्वतोर्वां शुश्रूषां क्लिष्टयोर्दुर्हृदा भृशम्‌
श्रीशुक उवाच
इति मायामनुष्यस्य हरेर्विश्वात्मनो गिरा
मोहितावङ्कमारोप्य परिष्वज्यापतुर्मुदम्‌
सिञ्चन्तावश्रुधाराभिः स्नेहपाशेन चावृतौ
न किञ्चिदूचतू राजन्बाष्पकण्ठौ विमोहितौ
एवमाश्वास्य पितरौ भगवान्देवकीसुतः
मातामहं तूग्रसेनं यदूनामकरोन्णृपम्‌
आह चास्मान्महाराज प्रजाश्चाज्ञप्तुमर्हसि
ययातिशापाद्यदुभिर्नासितव्यं नृपासने
मयि भृत्य उपासीने भवतो विबुधादयः
बलिं हरन्त्यवनताः किमुतान्ये नराधिपाः
सर्वान्स्वान्ज्ञतिसम्बन्धान्दिग्भ्यः कंसभयाकुलान्‌
यदुवृष्ण्यन्धकमधु दाशार्हकुकुरादिकान्‌
सभाजितान्समाश्वास्य विदेशावासकर्शितान्‌
न्यवासयत्स्वगेहेषु वित्तैः सन्तर्प्य विश्वकृत्‌
कृष्णसङ्कर्षणभुजैर्गुप्ता लब्धमनोरथाः
गृहेषु रेमिरे सिद्धाः कृष्णरामगतज्वराः
वीक्षन्तोऽहरहः प्रीता मुकुन्दवदनाम्बुजम्‌
नित्यं प्रमुदितं श्रीमत्सदयस्मितवीक्षणम्‌
तत्र प्रवयसोऽप्यासन्युवानोऽतिबलौजसः
पिबन्तोऽक्षैर्मुकुन्दस्य मुखाम्बुजसुधां मुहुः
अथ नन्दं समसाद्य भगवान्देवकीसुतः
सङ्कर्षणश्च राजेन्द्र परिष्वज्येदमूचतुः
पितर्युवाभ्यां स्निग्धाभ्यां पोषितौ लालितौ भृशम्‌
पित्रोरभ्यधिका प्रीतिरात्मजेष्वात्मनोऽपि हि
स पिता सा च जननी यौ पुष्णीतां स्वपुत्रवत्‌
शिशून्बन्धुभिरुत्सृष्टानकल्पैः पोषरक्षणे
यात यूयं व्रजंन्तात वयं च स्नेहदुःखितान्‌
ज्ञातीन्वो द्रष्टुमेष्यामो विधाय सुहृदां सुखम्‌
एवं सान्त्वय्य भगवान्नन्दं सव्रजमच्युतः
वासोऽलङ्कारकुप्याद्यैरर्हयामास सादरम्‌
इत्युक्तस्तौ परिष्वज्य नन्दः प्रणयविह्वलः
पूरयन्नश्रुभिर्नेत्रे सह गोपैर्व्रजं ययौ
अथ शूरसुतो राजन्पुत्रयोः समकारयत्‌
पुरोधसा ब्राह्मणैश्च यथावद्द्विजसंस्कृतिम्‌
तेभ्योऽदाद्दक्षिणा गावो रुक्ममालाः स्वलङ्कृताः
स्वलङ्कृतेभ्यः सम्पूज्य सवत्साः क्षौममालिनीः
याः कृष्णरामजन्मर्क्षे मनोदत्ता महामतिः
ताश्चाददादनुस्मृत्य कंसेनाधर्मतो हृताः
ततश्च लब्धसंस्कारौ द्विजत्वं प्राप्य सुव्रतौ
गर्गाद्यदुकुलाचार्याद्गायत्रं व्रतमास्थितौ
प्रभवौ सर्वविद्यानां सर्वज्ञौ जगदीश्वरौ
नान्यसिद्धामलं ज्ञानं गूहमानौ नरेहितैः
अथो गुरुकुले वासमिच्छन्तावुपजग्मतुः
काश्यं सान्दीपनिं नाम ह्यवन्तिपुरवासिनम्‌
यथोपसाद्य तौ दान्तौ गुरौ वृत्तिमनिन्दिताम्‌
ग्राहयन्तावुपेतौ स्म भक्त्या देवमिवादृतौ
तयोर्द्विजवरस्तुष्टः शुद्धभावानुवृत्तिभिः
प्रोवाच वेदानखिलान्सङ्गोपनिषदो गुरुः
सरहस्यं धनुर्वेदं धर्मान्न्यायपथांस्तथा
तथा चान्वीक्षिकीं विद्यां राजनीतिं च षड्विधाम्‌
सर्वं नरवरश्रेष्ठौ सर्वविद्याप्रवर्तकौ
सकृन्निगदमात्रेण तौ सञ्जगृहतुर्नृप
अहोरात्रैश्चतुःषष्ट्या संयत्तौ तावतीः कलाः
गुरुदक्षिणयाचार्यं छन्दयामासतुर्नृप
द्विजस्तयोस्तं महिमानमद्भुतं
संलोक्ष्य राजन्नतिमानुसीं मतिम्‌
सम्मन्त्र्य पत्न्या स महार्णवे मृतं
बालं प्रभासे वरयां बभूव ह
तेथेत्यथारुह्य महारथौ रथं
प्रभासमासाद्य दुरन्तविक्रमौ
वेलामुपव्रज्य निषीदतुः क्षनं
सिन्धुर्विदित्वार्हनमाहरत्तयोः
तमाह भगवानाशु गुरुपुत्रः प्रदीयताम्‌
योऽसाविह त्वया ग्रस्तो बालको महतोर्मिणा
श्रीसमुद्र उवाच
न चाहार्षमहं देव दैत्यः पञ्चजनो महान्‌
अन्तर्जलचरः कृष्ण शङ्खरूपधरोऽसुरः
आस्ते तेनाहृतो नूनं तच्छ्रुत्वा सत्वरं प्रभुः
जलमाविश्य तं हत्वा नापश्यदुदरेऽर्भकम्‌
तदङ्गप्रभवं शङ्खमादाय रथमागमत्‌
ततः संयमनीं नाम यमस्य दयितां पुरीम्‌
गत्वा जनार्दनः शङ्खं प्रदध्मौ सहलायुधः
शङ्खनिर्ह्रादमाकर्ण्य प्रजासंयमनो यमः
तयोः सपर्यां महतीं चक्रे भक्त्युपबृंहिताम्‌
उवाचावनतः कृष्णं सर्वभूताशयालयम्‌
लीलामनुष्ययोर्विष्णो युवयोः करवाम किम्‌
श्रीभगवानुवाच
गुरुपुत्रमिहानीतं निजकर्मनिबन्धनम्‌
आनयस्व महाराज मच्छासनपुरस्कृतः
तथेति तेनोपानीतं गुरुपुत्रं यदूत्तमौ
दत्त्वा स्वगुरवे भूयो वृणीष्वेति तमूचतुः
श्रीगुरुरुवाच
सम्यक्सम्पादितो वत्स भवद्भ्यां गुरुनिष्क्रयः
को नु युष्मद्विधगुरोः कामानामवशिष्यते
गच्छतं स्वगृहं वीरौ कीर्तिर्वामस्तु पावनी
छन्दांस्ययातयामानि भवन्त्विह परत्र च
गुरुणैवमनुज्ञातौ रथेनानिलरंहसा
आयातौ स्वपुरं तात पर्जन्यनिनदेन वै
समनन्दन्प्रजाः सर्वा दृष्ट्वा रामजनार्दनौ
अपश्यन्त्यो बह्वहानि नष्टलब्धधना इव
श्रीशुक उवाच
वृष्णीनां प्रवरो मन्त्री कृष्णस्य दयितः सखा
शिष्यो बृहस्पतेः साक्षादुद्धवो बुद्धिसत्तमः
तमाह भगवान्प्रेष्ठं भक्तमेकान्तिनं क्वचित्‌
गृहीत्वा पाणिना पाणिं प्रपन्नार्तिहरो हरिः
गच्छोद्धव व्रजं सौम्य पित्रोर्नौ प्रीतिमावह
गोपीनां मद्वियोगाधिं मत्सन्देशैर्विमोचय
ता मन्मनस्का तृष्ट्प्राणा मदर्थे त्यक्तदैहिकाः
मामेव दयितं प्रेष्ठमात्मानं मनसा गताः
ये त्यक्तलोकधर्माश्च मदर्थे तान्बिभर्म्यहम्‌
मयि ताः प्रेयसां प्रेष्ठे दूरस्थे गोकुलस्त्रियः
स्मरन्त्योऽङ्ग विमुह्यन्ति विरहौत्कण्ठ्यविह्वलाः
धारयन्त्यतिकृच्छ्रेण प्रायः प्राणान्कथञ्चन
प्रत्यागमनसन्देशैर्बल्लव्यो मे मदात्मिकाः
श्रीशुक उवाच
इत्युक्त उद्धवो राजन्सन्देशं भर्तुरादृतः
आदाय रथमारुह्य प्रययौ नन्दगोकुलम्‌
प्राप्तो नन्दव्रजं श्रीमान्निम्लोचति विभावसौ
छन्नयानः प्रविशतां पशूनां खुररेणुभिः
वासितार्थेऽभियुध्यद्भिर्नादितं शुश्मिभिर्वृषैः
धावन्तीभिश्च वास्राभिरुधोभारैः स्ववत्सकान्‌
इतस्ततो विलङ्घद्भिर्गोवत्सैर्मण्डितं सितैः
गोदोहशब्दाभिरवं वेणूनां निःस्वनेन च
गायन्तीभिश्च कर्माणि शुभानि बलकृष्णयोः
स्वलङ्कृताभिर्गोपीभिर्गोपैश्च सुविराजितम्‌
अग्न्यर्कातिथिगोविप्र पितृदेवार्चनान्वितैः
धूपदीपैश्च माल्यैश्च गोपावासैर्मनोरमम्‌
सर्वतः पुष्पितवनं द्विजालिकुलनादितम्‌
हंसकारण्डवाकीर्णैः पद्मषण्डैश्च मण्डितम्‌
तमागतं समागम्य कृष्णस्यानुचरं प्रियम्‌
नन्दः प्रीतः परिष्वज्य वासुदेवधियार्चयत्‌
भोजितं परमान्नेन संविष्टं कशिपौ सुखम्‌
गतश्रमं पर्यपृच्छत्पादसंवाहनादिभिः
कच्चिदङ्ग महाभाग सखा नः शूरनन्दनः
आस्ते कुशल्यपत्याद्यैर्युक्तो मुक्तः सुहृद्व्रतः
दिष्ट्या कंसो हतः पापः सानुगः स्वेन पाप्मना
साधूनां धर्मशीलानां यदूनां द्वेष्टि यः सदा
अपि स्मरति नः कृष्णो मातरं सुहृदः सखीन्‌
गोपान्व्रजं चात्मनाथं गावो वृन्दावनं गिरिम्‌
अप्यायास्यति गोविन्दः स्वजनान्सकृदीक्षितुम्‌
तर्हि द्रक्ष्याम तद्वक्त्रं सुनसं सुस्मितेक्षणम्‌
दावाग्नेर्वातवर्षाच्च वृषसर्पाच्च रक्षिताः
दुरत्ययेभ्यो मृत्युभ्यः कृष्णेन सुमहात्मना
स्मरतां कृष्णवीर्याणि लीलापाङ्गनिरीक्षितम्‌
हसितं भाषितं चाङ्ग सर्वा नः शिथिलाः क्रियाः
सरिच्छैलवनोद्देशान्मुकुन्दपदभूषितान्‌
आक्रीडानीक्ष्यमाणानां मनो याति तदात्मताम्‌
मन्ये कृष्णं च रामं च प्राप्ताविह सुरोत्तमौ
सुराणां महदर्थाय गर्गस्य वचनं यथा
कंसं नागायुतप्राणं मल्लौ गजपतिं यथा
अवधिष्टां लीलयैव पशूनिव मृगाधिपः
तालत्रयं महासारं धनुर्यष्टिमिवेभराट्‌
बभञ्जैकेन हस्तेन सप्ताहमदधाद्गिरिम्‌
प्रलम्बो धेनुकोऽरिष्टस्तृणावर्तो बकादयः
दैत्याः सुरासुरजितो हता येनेह लीलया
श्रीशुक उवाच
इति संस्मृत्य संस्मृत्य नन्दः कृष्णानुरक्तधीः
अत्युत्कण्ठोऽभवत्तूष्णीं प्रेमप्रसरविह्वलः
यशोदा वर्ण्यमानानि पुत्रस्य चरितानि च
शृण्वन्त्यश्रूण्यवास्राक्षीत्स्नेहस्नुतपयोधरा
तयोरित्थं भगवति कृष्णे नन्दयशोदयोः
वीक्ष्यानुरागं परमं नन्दमाहोद्धवो मुदा
श्रीउद्धव उवाच
युवां श्लाघ्यतमौ नूनं देहिनामिह मानद
नारायणेऽखिलगुरौ यत्कृता मतिरीदृशी
एतौ हि विश्वस्य च बीजयोनी रामो मुकुन्दः पुरुषः प्रधानम्‌
अन्वीय भूतेषु विलक्षणस्य ज्ञानस्य चेशात इमौ पुराणौ
यस्मिन्जनः प्राणवियोगकाले क्षनं समावेश्य मनोऽविशुद्धम्‌
निर्हृत्य कर्माशयमाशु याति परां गतिं ब्रह्ममयोऽर्कवर्णः
तस्मिन्भवन्तावखिलात्महेतौ नारायणे कारणमर्त्यमूर्तौ
भावं विधत्तां नितरां महात्मन्किं वावशिष्टं युवयोः सुकृत्यम्‌
आगमिष्यत्यदीर्घेण कालेन व्रजमच्युतः
प्रियं विधास्यते पित्रोर्भगवान्सात्वतां पतिः
हत्वा कंसं रङ्गमध्ये प्रतीपं सर्वसात्वताम्‌
यदाह वः समागत्य कृष्णः सत्यं करोति तत्‌
मा खिद्यतं महाभागौ द्रक्ष्यथः कृष्णमन्तिके
अन्तर्हृदि स भूतानामास्ते ज्योतिरिवैधसि
न ह्यस्यास्ति प्रियः कश्चिन्नाप्रियो वास्त्यमानिनः
नोत्तमो नाधमो वापि समानस्यासमोऽपि वा
न माता न पिता तस्य न भार्या न सुतादयः
नात्मीयो न परश्चापि न देहो जन्म एव च
न चास्य कर्म वा लोके सदसन्मिश्रयोनिषु
क्रीडार्थं सोऽपि साधूनां परित्राणाय कल्पते
सत्त्वं रजस्तम इति भजते निर्गुणो गुणान्‌
क्रीडन्नतीतोऽपि गुणैः सृजत्यवन्हन्त्यजः
यथा भ्रमरिकादृष्ट्या भ्राम्यतीव महीयते
चित्ते कर्तरि तत्रात्मा कर्तेवाहंधिया स्मृतः
युवयोरेव नैवायमात्मजो भगवान्हरिः
सर्वेषामात्मजो ह्यात्मा पिता माता स ईश्वरः
दृष्टं श्रुतं भूतभवद्भविष्यत्‌
स्थास्नुश्चरिष्णुर्महदल्पकं च
विनाच्युताद्वस्तु तरां न वाच्यं
स एव सर्वं परमात्मभूतः
एवं निशा सा ब्रुवतोर्व्यतीता नन्दस्य कृष्णानुचरस्य राजन्‌
गोप्यः समुत्थाय निरूप्य दीपान्वास्तून्समभ्यर्च्य दौधीन्यमन्थुन्‌
ता दीपदीप्तैर्मणिभिर्विरेजू रज्जूर्विकर्षद्भुजकङ्कणस्रजः
चलन्नितम्बस्तनहारकुण्डल त्विषत्कपोलारुणकुङ्कुमाननाः
उद्गायतीनामरविन्दलोचनं व्रजाङ्गनानां दिवमस्पृशद्ध्वनिः
दध्नश्च निर्मन्थनशब्दमिश्रितो निरस्यते येन दिशाममङ्गलम्‌
भगवत्युदिते सूर्ये नन्दद्वारि व्रजौकसः
दृष्ट्वा रथं शातकौम्भं कस्यायमिति चाब्रुवन्‌
अक्रूर आगतः किं वा यः कंसस्यार्थसाधकः
येन नीतो मधुपुरीं कृष्णः कमललोचनः
किं साधयिष्यत्यस्माभिर्भर्तुः प्रीतस्य निष्कृतिम्‌
ततः स्त्रीणां वदन्तीनामुद्धवोऽगात्कृताह्निकः
श्रीशुक उवाच
तं वीक्ष्य कृषानुचरं व्रजस्त्रियः
प्रलम्बबाहुं नवकञ्जलोचनम्‌
पीताम्बरं पुष्करमालिनं लसन्‌
मुखारविन्दं परिमृष्टकुण्डलम्‌
सुविस्मिताः कोऽयमपीव्यदर्शनः
कुतश्च कस्याच्युतवेषभूषणः
इति स्म सर्वाः परिवव्रुरुत्सुकास्‌
तमुत्तमःश्लोकपदाम्बुजाश्रयम्‌
तं प्रश्रयेणावनताः सुसत्कृतं सव्रीडहासेक्षणसूनृतादिभिः
रहस्यपृच्छन्नुपविष्टमासने विज्ञाय सन्देशहरं रमापतेः
जानीमस्त्वां यदुपतेः पार्षदं समुपागतम्‌
भर्त्रेह प्रेषितः पित्रोर्भवान्प्रियचिकीर्षया
अन्यथा गोव्रजे तस्य स्मरणीयं न चक्ष्महे
स्नेहानुबन्धो बन्धूनां मुनेरपि सुदुस्त्यजः
अन्येष्वर्थकृता मैत्री यावदर्थविडम्बनम्‌
पुम्भिः स्त्रीषु कृता यद्वत्सुमनःस्विव षट्पदैः
निःस्वं त्यजन्ति गणिका अकल्पं नृपतिं प्रजाः
अधीतविद्या आचार्यमृत्विजो दत्तदक्षिणम्‌
खगा वीतफलं वृक्षं भुक्त्वा चातिथयो गृहम्‌
दग्धं मृगास्तथारण्यं जारा भुक्त्वा रतां स्त्रियम्‌
इति गोप्यो हि गोविन्दे गतवाक्कायमानसाः
कृष्णदूते समायाते उद्धवे त्यक्तलौकिकाः
गायन्त्यः प्रीयकर्माणि रुदन्त्यश्च गतह्रियः
तस्य संस्मृत्य संस्मृत्य यानि कैशोरबाल्ययोः
काचिन्मधुकरं दृष्ट्वा ध्यायन्ती कृष्णसङ्गमम्‌
प्रियप्रस्थापितं दूतं कल्पयित्वेदमब्रवीत्‌
गोप्युवाच
मधुप कितवबन्धो मा स्पृशङ्घ्रिं सपत्न्याः
कुचविलुलितमालाकुङ्कुमश्मश्रुभिर्नः
वहतु मधुपतिस्तन्मानिनीनां प्रसादं
यदुसदसि विडम्ब्यं यस्य दूतस्त्वमीदृक्‌
सकृदधरसुधां स्वां मोहिनीं पाययित्वा
सुमनस इव सद्यस्तत्यजेऽस्मान्भवादृक्‌
परिचरति कथं तत्पादपद्मं नु पद्मा
ह्यपि बत हृतचेता ह्युत्तमःश्लोकजल्पैः
किमिह बहु षडङ्घ्रे गायसि त्वं यदूनाम्‌
अधिपतिमगृहाणामग्रतो नः पुराणम्‌
विजयसखसखीनां गीयतां तत्प्रसङ्गः
क्षपितकुचरुजस्ते कल्पयन्तीष्टमिष्टाः
दिवि भुवि च रसायां काः स्त्रियस्तद्दुरापाः
कपटरुचिरहासभ्रूविजृम्भस्य याः स्युः
चरणरज उपास्ते यस्य भूतिर्वयं का
अपि च कृपणपक्षे ह्युत्तमःश्लोकशब्दः
विसृज शिरसि पादं वेद्म्यहं चातुकारैर्‌
अनुनयविदुषस्तेऽभ्येत्य दौत्यैर्मुकुन्दात्‌
स्वकृत इह विषृष्टापत्यपत्यन्यलोका
व्यसृजदकृतचेताः किं नु सन्धेयमस्मिन्‌
मृगयुरिव कपीन्द्रं विव्यधे लुब्धधर्मा
स्त्रियमकृत विरूपां स्त्रीजितः कामयानाम्‌
बलिमपि बलिमत्त्वावेष्टयद्ध्वाङ्क्षवद्यस्‌
तदलमसितसख्यैर्दुस्त्यजस्तत्कथार्थः
यदनुचरितलीलाकर्णपीयूषविप्रुट्‌
सकृददनविधूतद्वन्द्वधर्मा विनष्टाः
सपदि गृहकुटुम्बं दीनमुत्सृज्य दीना
बहव इह विहङ्गा भिक्षुचर्यां चरन्ति
वयमृतमिव जिह्मव्याहृतं श्रद्दधानाः
कुलिकरुतमिवाज्ञाः कृष्णवध्वो हरिण्यः
ददृशुरसकृदेतत्तन्नखस्पर्शतीव्र
स्मररुज उपमन्त्रिन्भण्यतामन्यवार्ता
प्रियसख पुनरागाः प्रेयसा प्रेषितः किं
वरय किमनुरुन्धे माननीयोऽसि मेऽङ्ग
नयसि कथमिहास्मान्दुस्त्यजद्वन्द्वपार्श्वं
सततमुरसि सौम्य श्रीर्वधूः साकमास्ते
अपि बत मधुपुर्यामार्यपुत्रोऽधुनास्ते
स्मरति स पितृगेहान्सौम्य बन्धूंश्च गोपान्‌
क्वचिदपि स कथा नः किङ्करीणां गृणीते
भुजमगुरुसुगन्धं मूर्ध्न्यधास्यत्कदा नु
श्रीशुक उवाच
अथोद्धवो निशम्यैवं कृष्णदर्शनलालसाः
सान्त्वयन्प्रियसन्देशैर्गोपीरिदमभाषत
श्रीउद्धव उवाच
अहो यूयं स्म पूर्णार्था भवत्यो लोकपूजिताः
वासुदेवे भगवति यासामित्यर्पितं मनः
दानव्रततपोहोम जपस्वाध्यायसंयमैः
श्रेयोभिर्विविधैश्चान्यैः कृष्णे भक्तिर्हि साध्यते
भगवत्युत्तमःश्लोके भवतीभिरनुत्तमा
भक्तिः प्रवर्तिता दिष्ट्या मुनीनामपि दुर्लभा
दिष्ट्या पुत्रान्पतीन्देहान्स्वजनान्भवनानि च
हित्वावृनीत यूयं यत्कृष्णाख्यं पुरुषं परम्‌
सर्वात्मभावोऽधिकृतो भवतीनामधोक्षजे
विरहेण महाभागा महान्मेऽनुग्रहः कृतः
श्रूयतां प्रियसन्देशो भवतीनां सुखावहः
यमादायागतो भद्रा अहं भर्तू रहस्करः
श्रीभगवानुवाच
भवतीनां वियोगो मे न हि सर्वात्मना क्वचित्‌
यथा भूतानि भूतेषु खं वाय्वग्निर्जलं मही
तथाहं च मनःप्राण भूतेन्द्रियगुणाश्रयः
आत्मन्येवात्मनात्मानं सृजे हन्म्यनुपालये
आत्ममायानुभावेन भूतेन्द्रियगुणात्मना
आत्मा ज्ञानमयः शुद्धो व्यतिरिक्तोऽगुणान्वयः
सुषुप्तिस्वप्नजाग्रद्भिर्मायावृत्तिभिरीयते
येनेन्द्रियार्थान्ध्यायेत मृषा स्वप्नवदुत्थितः
तन्निरुन्ध्यादिन्द्रियाणि विनिद्रः प्रत्यपद्यत
एतदन्तः समाम्नायो योगः साङ्ख्यं मनीषिणाम्‌
त्यागस्तपो दमः सत्यं समुद्रान्ता इवापगाः
यत्त्वहं भवतीनां वै दूरे वर्ते प्रियो दृशाम्‌
मनसः सन्निकर्षार्थं मदनुध्यानकाम्यया
यथा दूरचरे प्रेष्ठे मन आविश्य वर्तते
स्त्रीणां च न तथा चेतः सन्निकृष्टेऽक्षिगोचरे
मय्यावेश्य मनः कृत्स्नं विमुक्ताशेषवृत्ति यत्‌
अनुस्मरन्त्यो मां नित्यमचिरान्मामुपैष्यथ
या मया क्रीडता रात्र्यां वनेऽस्मिन्व्रज आस्थिताः
अलब्धरासाः कल्याण्यो मापुर्मद्वीर्यचिन्तया
श्रीशुक उवाच
एवं प्रियतमादिष्टमाकर्ण्य व्रजयोषितः
ता ऊचुरुद्धवं प्रीतास्तत्सन्देशागतस्मृतीः
गोप्य ऊचुः
दिष्ट्याहितो हतः कंसो यदूनां सानुगोऽघकृत्‌
दिष्ट्याप्तैर्लब्धसर्वार्थैः कुशल्यास्तेऽच्युतोऽधुना
कच्चिद्गदाग्रजः सौम्य करोति पुरयोषिताम्‌
प्रीतिं नः स्निग्धसव्रीड हासोदारेक्षणार्चितः
कथं रतिविशेषज्ञः प्रियश्च पुरयोषिताम्‌
नानुबध्येत तद्वाक्यैर्विभ्रमैश्चानुभाजितः
अपि स्मरति नः साधो गोविन्दः प्रस्तुते क्वचित्‌
गोष्ठिमध्ये पुरस्त्रीणाम्ग्राम्याः स्वैरकथान्तरे
ताः किं निशाः स्मरति यासु तदा प्रियाभिर्‌
वृन्दावने कुमुदकुन्दशशाङ्करम्ये
रेमे क्वणच्चरणनूपुररासगोष्ठ्याम्‌
अस्माभिरीडितमनोज्ञकथः कदाचित्‌
अप्येष्यतीह दाशार्हस्तप्ताः स्वकृतया शुचा
सञ्जीवयन्नु नो गात्रैर्यथेन्द्रो वनमम्बुदैः
कस्मात्कृष्ण इहायाति प्राप्तराज्यो हताहितः
नरेन्द्रकन्या उद्वाह्य प्रीतः सर्वसुहृद्वृतः
किमस्माभिर्वनौकोभिरन्याभिर्वा महात्मनः
श्रीपतेराप्तकामस्य क्रियेतार्थः कृतात्मनः
परं सौख्यं हि नैराश्यं स्वैरिण्यप्याह पिङ्गला
तज्जानतीनां नः कृष्णे तथाप्याशा दुरत्यया
क उत्सहेत सन्त्यक्तुमुत्तमःश्लोकसंविदम्‌
अनिच्छतोऽपि यस्य श्रीरङ्गान्न च्यवते क्वचित्‌
सरिच्छैलवनोद्देशा गावो वेणुरवा इमे
सङ्कर्षणसहायेन कृष्णेनाचरिताः प्रभो
पुनः पुनः स्मारयन्ति नन्दगोपसुतं बत
श्रीनिकेतैस्तत्पदकैर्विस्मर्तुं नैव शक्नुमः
गत्या ललितयोदार हासलीलावलोकनैः
माध्व्या गिरा हृतधियः कथं तं विस्मराम हे
हे नाथ हे रमानाथ व्रजनाथार्तिनाशन
मग्नमुद्धर गोविन्द गोकुलं वृजिनार्णवात्‌
श्रीशुक उवाच
ततस्ताः कृष्णसन्देशैर्व्यपेतविरहज्वराः
उद्धवं पूजयां चक्रुर्ज्ञात्वात्मानमधोक्षजम्‌
उवास कतिचिन्मासान्गोपीनां विनुदन्शुचः
कृष्णलीलाकथां गायन्रमयामास गोकुलम्‌
यावन्त्यहानि नन्दस्य व्रजेऽवात्सीत्स उद्धवः
व्रजौकसां क्षणप्रायाण्यासन्कृष्णस्य वार्तया
सरिद्वनगिरिद्रोणीर्वीक्षन्कुसुमितान्द्रुमान्‌
कृष्णं संस्मारयन्रेमे हरिदासो व्रजौकसाम्‌
दृष्ट्वैवमादि गोपीनां कृष्णावेशात्मविक्लवम्‌
उद्धवः परमप्रीतस्ता नमस्यन्निदं जगौ
एताः परं तनुभृतो भुवि गोपवध्वो
गोविन्द एव निखिलात्मनि रूढभावाः
वाञ्छन्ति यद्भवभियो मुनयो वयं च
किं ब्रह्मजन्मभिरनन्तकथारसस्य
क्वेमाः स्त्रियो वनचरीर्व्यभिचारदुष्टाः
कृष्णे क्व चैष परमात्मनि रूढभावः
नन्वीश्वरोऽनुभजतोऽविदुषोऽपि साक्षाच्‌
छ्रेयस्तनोत्यगदराज इवोपयुक्तः
नायं श्रियोऽङ्ग उ नितान्तरतेः प्रसादः
स्वर्योषितां नलिनगन्धरुचां कुतोऽन्याः
रासोत्सवेऽस्य भुजदण्डगृहीतकण्ठ
लब्धाशिषां य उदगाद्व्रजवल्लभीनाम्‌
आसामहो चरणरेणुजुषामहं स्यां
वृन्दावने किमपि गुल्मलतौषधीनाम्‌
या दुस्त्यजं स्वजनमार्यपथं च हित्वा
भेजुर्मुकुन्दपदवीं श्रुतिभिर्विमृग्याम्‌
या वै श्रियार्चितमजादिभिराप्तकामैर्‌
योगेश्वरैरपि यदात्मनि रासगोष्ठ्याम्‌
कृष्णस्य तद्भगवतः चरणारविन्दं
न्यस्तं स्तनेषु विजहुः परिरभ्य तापम्‌
वन्दे नन्दव्रजस्त्रीणां पादरेणुमभीक्ष्णशः
यासां हरिकथोद्गीतं पुनाति भुवनत्रयम्‌
श्रीशुक उवाच
अथ गोपीरनुज्ञाप्य यशोदां नन्दमेव च
गोपानामन्त्र्य दाशार्हो यास्यन्नारुरुहे रथम्‌
तं निर्गतं समासाद्य नानोपायनपाणयः
नन्दादयोऽनुरागेण प्रावोचन्नश्रुलोचनाः
मनसो वृत्तयो नः स्युः कृष्ण पादाम्बुजाश्रयाः
वाचोऽभिधायिनीर्नाम्नां कायस्तत्प्रह्वणादिषु
कर्मभिर्भ्राम्यमाणानां यत्र क्वापीश्वरेच्छया
मङ्गलाचरितैर्दानै रतिर्नः कृष्ण ईश्वरे
एवं सभाजितो गोपैः कृष्णभक्त्या नराधिप
उद्धवः पुनरागच्छन्मथुरां कृष्णपालिताम्‌
कृष्णाय प्रणिपत्याह भक्त्युद्रेकं व्रजौकसाम्‌
वसुदेवाय रामाय राज्ञे चोपायनान्यदात्‌
श्रीशुक उवाच
अथ विज्ञाय भगवान्सर्वात्मा सर्वदर्शनः
सैरन्ध्र्याः कामतप्तायाः प्रियमिच्छन्गृहं ययौ
महार्होपस्करैराढ्यं कामोपायोपबृंहितम्‌
मुक्तादामपताकाभिर्वितानशयनासनैः
धूपैः सुरभिभिर्दीपैः स्रग्गन्धैरपि मण्डितम्‌
गृहं तमायान्तमवेक्ष्य सासनात्सद्यः समुत्थाय हि जातसम्भ्रमा
यथोपसङ्गम्य सखीभिरच्युतं सभाजयामास सदासनादिभिः
तथोद्धवः साधुतयाभिपूजितो न्यषीददुर्व्यामभिमृश्य चासनम्‌
कृष्णोऽपि तूर्णं शयनं महाधनं विवेश लोकाचरितान्यनुव्रतः
सा मज्जनालेपदुकूलभूषण स्रग्गन्धताम्बूलसुधासवादिभिः
प्रसाधितात्मोपससार माधवं सव्रीडलीलोत्स्मितविभ्रमेक्षितैः
आहूय कान्तां नवसङ्गमह्रिया विशङ्कितां कङ्कणभूषिते करे
प्रगृह्य शय्यामधिवेश्य रामया रेमेऽनुलेपार्पणपुण्यलेशया
सानङ्गतप्तकुचयोरुरसस्तथाक्ष्णोर्‌
जिघ्रन्त्यनन्तचरणेन रुजो मृजन्ती
दोर्भ्यां स्तनान्तरगतं परिरभ्य कान्तम्‌
आनन्दमूर्तिमजहादतिदीर्घतापम्‌
सैवं कैवल्यनाथं तं प्राप्य दुष्प्राप्यमीश्वरम्‌
अङ्गरागार्पणेनाहो दुर्भगेदमयाचत
सहोष्यतामिह प्रेष्ठ दिनानि कतिचिन्मया
रमस्व नोत्सहे त्यक्तुं सङ्गं तेऽम्बुरुहेक्षण
तस्यै कामवरं दत्त्वा मानयित्वा च मानदः
सहोद्धवेन सर्वेशः स्वधामागमदृद्धिमत्‌
दुरार्ध्यं समाराध्य विष्णुं सर्वेश्वरेश्वरम्‌
यो वृणीते मनोग्राह्यमसत्त्वात्कुमनीष्यसौ
अक्रूरभवनं कृष्णः सहरामोद्धवः प्रभुः
किञ्चिच्चिकीर्षयन्प्रागादक्रूरप्रीयकाम्यया
स तान्नरवरश्रेष्ठानाराद्वीक्ष्य स्वबान्धवान्‌
प्रत्युत्थाय प्रमुदितः परिष्वज्याभिनन्द्य च
ननाम कृष्णं रामं च स तैरप्यभिवादितः
पूजयामास विधिवत्कृतासनपरिग्रहान्‌
पादावनेजनीरापो धारयन्शिरसा नृप
अर्हणेनाम्बरैर्दिव्यैर्गन्धस्रग्भूषणोत्तमैः
अर्चित्वा शिरसानम्य पादावङ्कगतौ मृजन्‌
प्रश्रयावनतोऽक्रूरः कृष्णरामावभाषत
दिष्ट्या पापो हतः कंसः सानुगो वामिदं कुलम्‌
भवद्भ्यामुद्धृतं कृच्छ्राद्दुरन्ताच्च समेधितम्‌
युवां प्रधानपुरुषौ जगद्धेतू जगन्मयौ
भवद्भ्यां न विना किञ्चित्परमस्ति न चापरम्‌
आत्मसृष्टमिदं विश्वमन्वाविश्य स्वशक्तिभिः
ईयते बहुधा ब्रह्मन्श्रु तप्रत्यक्षगोचरम्‌
यथा हि भूतेषु चराचरेषु मह्यादयो योनिषु भान्ति नाना
एवं भवान्केवल आत्मयोनिष्वात्मात्मतन्त्रो बहुधा विभाति
सृजस्यथो लुम्पसि पासि विश्वं रजस्तमःसत्त्वगुणैः स्वशक्तिभिः
न बध्यसे तद्गुणकर्मभिर्वा ज्ञानात्मनस्ते क्व च बन्धहेतुः
देहाद्युपाधेरनिरूपितत्वाद्भवो न साक्षान्न भिदात्मनः स्यात्‌
अतो न बन्धस्तव नैव मोक्षः स्याताम्निकामस्त्वयि नोऽविवेकः
त्वयोदितोऽयं जगतो हिताय यदा यदा वेदपथः पुराणः
बाध्येत पाषण्डपथैरसद्भिस्तदा भवान्सत्त्वगुणं बिभर्ति
स त्वम्प्रभोऽद्य वसुदेवगृहेऽवतीर्णः
स्वांशेन भारमपनेतुमिहासि भूमेः
अक्षौहिणीशतवधेन सुरेतरांश
राज्ञाममुष्य च कुलस्य यशो वितन्वन्‌
अद्येश नो वसतयः खलु भूरिभागा
यः सर्वदेवपितृभूतनृदेवमूर्तिः
यत्पादशौचसलिलं त्रिजगत्पुनाति
स त्वं जगद्गुरुरधोक्षज याः प्रविष्टः
कः पण्डितस्त्वदपरं शरणं समीयाद्‌
भक्तप्रियादृतगिरः सुहृदः कृतज्ञात्‌
सर्वान्ददाति सुहृदो भजतोऽभिकामान्‌
आत्मानमप्युपचयापचयौ न यस्य
दिष्ट्या जनार्दन भवानिह नः प्रतीतो
योगेश्वरैरपि दुरापगतिः सुरेशैः
छिन्ध्याशु नः सुतकलत्रधनाप्तगेह
देहादिमोहरशनां भवदीयमायाम्‌
इत्यर्चितः संस्तुतश्च भक्तेन भगवान्हरिः
अक्रूरं सस्मितं प्राह गीर्भिः सम्मोहयन्निव
श्रीभगवानुवाच
त्वं नो गुरुः पितृव्यश्च श्लाघ्यो बन्धुश्च नित्यदा
वयं तु रक्ष्याः पोष्याश्च अनुकम्प्याः प्रजा हि वः
भवद्विधा महाभागा निषेव्या अर्हसत्तमाः
श्रेयस्कामैर्नृभिर्नित्यं देवाः स्वार्था न साधवः
न ह्यम्मयानि तीर्थानि न देवा मृच्छिलामयाः
ते पुनन्त्युरुकालेन दर्शनादेव साधवः
स भवान्सुहृदां वै नः श्रेयान्श्रेयश्चिकीर्षया
जिज्ञासार्थं पाण्डवानां गच्छस्व त्वं गजाह्वयम्‌
पितर्युपरते बालाः सह मात्रा सुदुःखिताः
आनीताः स्वपुरं राज्ञा वसन्त इति शुश्रुम
तेषु राजाम्बिकापुत्रो भ्रातृपुत्रेषु दीनधीः
समो न वर्तते नूनं दुष्पुत्रवशगोऽन्धदृक्‌
गच्छ जानीहि तद्वृत्तमधुना साध्वसाधु वा
विज्ञाय तद्विधास्यामो यथा शं सुहृदां भवेत्‌
इत्यक्रूरं समादिश्य भगवान्हरिरीश्वरः
सङ्कर्षणोद्धवाभ्यां वै ततः स्वभवनं ययौ
श्रीशुक उवाच
स गत्वा हास्तिनपुरं पौरवेन्द्रयशोऽङ्कितम्‌
ददर्श तत्राम्बिकेयं सभीष्मं विदुरं पृथाम्‌
सहपुत्रं च बाह्लीकं भारद्वाजं सगौतमम्‌
कर्नं सुयोधनं द्रौणिं पाण्डवान्सुहृदोऽपरान्‌
यथावदुपसङ्गम्य बन्धुभिर्गान्दिनीसुतः
सम्पृष्टस्तैः सुहृद्वार्तां स्वयं चापृच्छदव्ययम्‌
उवास कतिचिन्मासान्राज्ञो वृत्तविवित्सया
दुष्प्रजस्याल्पसारस्य खलच्छन्दानुवर्तिनः
तेज ओजो बलं वीर्यं प्रश्रयादींश्च सद्गुणान्‌
प्रजानुरागं पार्थेषु न सहद्भिश्चिकीऋषितम्‌
कृतं च धार्तराष्ट्रैर्यद्गरदानाद्यपेशलम्‌
आचख्यौ सर्वमेवास्मै पृथा विदुर एव च
पृथा तु भ्रातरं प्राप्तमक्रूरमुपसृत्य तम्‌
उवाच जन्मनिलयं स्मरन्त्यश्रुकलेक्षणा
अपि स्मरन्ति नः सौम्य पितरौ भ्रातरश्च मे
भगिन्यौ भ्रातृपुत्राश्च जामयः सख्य एव च
भ्रात्रेयो भगवान्कृष्णः शरण्यो भक्तवत्सलः
पैतृष्वस्रेयान्स्मरति रामश्चाम्बुरुहेक्षणः
सपत्नमध्ये शोचन्तीं वृकानां हरिणीमिव
सान्त्वयिष्यति मां वाक्यैः पितृहीनांश्च बालकान्‌
कृष्ण कृष्ण महायोगिन्विश्वात्मन्विश्वभावन
प्रपन्नां पाहि गोविन्द शिशुभिश्चावसीदतीम्‌
नान्यत्तव पदाम्भोजात्पश्यामि शरणं नृणाम्‌
बिभ्यतां मृत्युसंसारादीस्वरस्यापवर्गिकात्‌
नमः कृष्णाय शुद्धाय ब्रह्मणे परमात्मने
योगेश्वराय योगाय त्वामहं शरणं गता
श्रीशुक उवाच
इत्यनुस्मृत्य स्वजनं कृष्णं च जगदीश्वरम्‌
प्रारुदद्दुःखिता राजन्भवतां प्रपितामही
समदुःखसुखोऽक्रूरो विदुरश्च महायशाः
सान्त्वयामासतुः कुन्तीं तत्पुत्रोत्पत्तिहेतुभिः
यास्यन्राजानमभ्येत्य विषमं पुत्रलालसम्‌
अवदत्सुहृदां मध्ये बन्धुभिः सौहृदोदितम्‌
अक्रूर उवाच
भो भो वैचित्रवीर्य त्वं कुरूणां कीर्तिवर्धन
भ्रातर्युपरते पाण्डावधुनासनमास्थितः
धर्मेण पालयन्नुर्वीं प्रजाः शीलेन रञ्जयन्‌
वर्तमानः समः स्वेषु श्रेयः कीर्तिमवाप्स्यसि
अन्यथा त्वाचरंल्लोके गर्हितो यास्यसे तमः
तस्मात्समत्वे वर्तस्व पाण्डवेष्वात्मजेषु च
नेह चात्यन्तसंवासः कस्यचित्केनचित्सह
राजन्स्वेनापि देहेन किमु जायात्मजादिभिः
एकः प्रसूयते जन्तुरेक एव प्रलीयते
एकोऽनुभुङ्क्ते सुकृतमेक एव च दुष्कृतम्‌
अधर्मोपचितं वित्तं हरन्त्यन्येऽल्पमेधसः
सम्भोजनीयापदेशैर्जलानीव जलौकसः
पुष्णाति यानधर्मेण स्वबुद्ध्या तमपण्डितम्‌
तेऽकृतार्थं प्रहिण्वन्ति प्राणा रायः सुतादयः
स्वयं किल्बिषमादाय तैस्त्यक्तो नार्थकोविदः
असिद्धार्थो विशत्यन्धं स्वधर्मविमुखस्तमः
तस्माल्लोकमिमं राजन्स्वप्नमायामनोरथम्‌
वीक्ष्यायम्यात्मनात्मानं समः शान्तो भव प्रभो
धृतराष्ट्र उवाच
यथा वदति कल्याणीं वाचं दानपते भवान्‌
तथानया न तृप्यामि मर्त्यः प्राप्य यथामृतम्‌
तथापि सूनृता सौम्य हृदि न स्थीयते चले
पुत्रानुरागविषमे विद्युत्सौदामनी यथा
ईश्वरस्य विधिं को नु विधुनोत्यन्यथा पुमान्‌
भूमेर्भारावताराय योऽवतीर्णो यदोः कुले
यो दुर्विमर्शपथया निजमाययेदं
सृष्ट्वा गुणान्विभजते तदनुप्रविष्टः
तस्मै नमो दुरवबोधविहारतन्त्र
संसारचक्रगतये परमेश्वराय
श्रीशुक उवाच
इत्यभिप्रेत्य नृपतेरभिप्रायं स यादवः
सुहृद्भिः समनुज्ञातः पुनर्यदुपुरीमगात्‌
शशंस रामकृष्णाभ्यां धृतराष्ट्रविचेष्टितम्‌
पाण्दवान्प्रति कौरव्य यदर्थं प्रेषितः स्वयम्‌
श्रीशुक उवाच
अस्तिः प्राप्तिश्च कंसस्य महिष्यौ भरतर्षभ
मृते भर्तरि दुःखार्ते ईयतुः स्म पितुर्गृहान्‌
पित्रे मगधराजाय जरासन्धाय दुःखिते
वेदयां चक्रतुः सर्वमात्मवैधव्यकारणम्‌
स तदप्रियमाकर्ण्य शोकामर्षयुतो नृप
अयादवीं महीं कर्तुं चक्रे परममुद्यमम्‌
अक्षौहिणीभिर्विंशत्या तिसृभिश्चापि संवृतः
यदुराजधानीं मथुरां न्यरुधत्सर्वतो दिशम्‌
निरीक्ष्य तद्बलं कृष्ण उद्वेलमिव सागरम्‌
स्वपुरं तेन संरुद्धं स्वजनं च भयाकुलम्‌
चिन्तयामास भगवान्हरिः कारणमानुषः
तद्देशकालानुगुणं स्वावतारप्रयोजनम्‌
हनिष्यामि बलं ह्येतद्भुवि भारं समाहितम्‌
मागधेन समानीतं वश्यानां सर्वभूभुजाम्‌
अक्षौहिणीभिः सङ्ख्यातं भटाश्वरथकुञ्जरैः
मागधस्तु न हन्तव्यो भूयः कर्ता बलोद्यमम्‌
एतदर्थोऽवतारोऽयं भूभारहरणाय मे
संरक्षणाय साधूनां कृतोऽन्येषां वधाय च
अन्योऽपि धर्मरक्षायै देहः संभ्रियते मया
विरामायाप्यधर्मस्य काले प्रभवतः क्वचित्‌
एवं ध्यायति गोविन्द आकाशात्सूर्यवर्चसौ
रथावुपस्थितौ सद्यः ससूतौ सपरिच्छदौ
आयुधानि च दिव्यानि पुराणानि यदृच्छया
दृष्ट्वा तानि हृषीकेशः सङ्कर्षणमथाब्रवीत्‌
पश्यार्य व्यसनं प्राप्तं यदूनां त्वावतां प्रभो
एष ते रथ आयातो दयितान्यायुधानि च
एतदर्थं हि नौ जन्म साधूनामीश शर्मकृत्‌
त्रयोविंशत्यनीकाख्यं भूमेर्भारमपाकुरु
एवं सम्मन्त्र्य दाशार्हौ दंशितौ रथिनौ पुरात्‌
निर्जग्मतुः स्वायुधाढ्यौ बलेनाल्पीयसा वृतौ
शङ्खं दध्मौ विनिर्गत्य हरिर्दारुकसारथिः
ततोऽभूत्परसैन्यानां हृदि वित्रासवेपथुः
तावाह मागधो वीक्ष्य हे कृष्ण पुरुषाधम
न त्वया योद्धुमिच्छामि बालेनैकेन लज्जया
गुप्तेन हि त्वया मन्द न योत्स्ये याहि बन्धुहन्‌
तव राम यदि श्रद्धा युध्यस्व धैर्यमुद्वह
हित्वा वा मच्छरैश्छिन्नं देहं स्वर्याहि मां जहि
श्रीभगवानुवाच
न वै शूरा विकत्थन्ते दर्शयन्त्येव पौरुषम्‌
न गृह्णीमो वचो राजन्नातुरस्य मुमूर्षतः
श्रीशुक उवाच
जरासुतस्तावभिसृत्य माधवौ महाबलौघेन बलीयसावृनोत्‌
ससैन्ययानध्वजवाजिसारथी सूर्यानलौ वायुरिवाभ्ररेणुभिः
सुपर्णतालध्वजचिहित्नौ रथाव्‌
अलक्षयन्त्यो हरिरामयोर्मृधे
स्त्रियः पुराट्टालकहर्म्यगोपुरं
समाश्रिताः सम्मुमुहुः शुचार्दितः
हरिः परानीकपयोमुचां मुहुः शिलीमुखात्युल्बणवर्षपीडितम्‌
स्वसैन्यमालोक्य सुरासुरार्चितं व्यस्फूर्जयच्छार्ङ्गशरासनोत्तमम्‌
गृह्णन्निशङ्गादथ सन्दधच्छरान्‌
विकृष्य मुञ्चन्शितबाणपूगान्‌
निघ्नन्रथान्कुञ्जरवाजिपत्तीन्‌
निरन्तरं यद्वदलातचक्रम्‌
निर्भिन्नकुम्भाः करिणो निपेतुरनेकशोऽश्वाः शरवृक्णकन्धराः
रथा हताश्वध्वजसूतनायकाः पदायतश्छिन्नभुजोरुकन्धराः
सञ्छिद्यमानद्विपदेभवाजिनामङ्गप्रसूताः शतशोऽसृगापगाः
भुजाहयः पूरुषशीर्षकच्छपा हतद्विपद्वीपहय ग्रहाकुलाः
करोरुमीना नरकेशशैवला धनुस्तरङ्गायुधगुल्मसङ्कुलाः
अच्छूरिकावर्तभयानका महा मणिप्रवेकाभरणाश्मशर्कराः
प्रवर्तिता भीरुभयावहा मृधे मनस्विनां हर्षकरीः परस्परम्‌
विनिघ्नतारीन्मुषलेन दुर्मदान्सङ्कर्षणेनापरीमेयतेजसा
बलं तदङ्गार्णवदुर्गभैरवं दुरन्तपारं मगधेन्द्रपालितम्‌
क्षयं प्रणीतं वसुदेवपुत्रयोर्विक्रीडितं तज्जगदीशयोः परम्‌
स्थित्युद्भवान्तं भुवनत्रयस्य यः
समीहितेऽनन्तगुणः स्वलीलया
न तस्य चित्रं परपक्षनिग्रहस्‌
तथापि मर्त्यानुविधस्य वर्ण्यते
जग्राह विरथं रामो जरासन्धं महाबलम्‌
हतानीकावशिष्टासुं सिंहः सिंहमिवौजसा
बध्यमानं हतारातिं पाशैर्वारुणमानुषैः
वारयामास गोविन्दस्तेन कार्यचिकीर्षया
सा मुक्तो लोकनाथाभ्यां व्रीडितो वीरसम्मतः
तपसे कृतसङ्कल्पो वारितः पथि राजभिः
वाक्यैः पवित्रार्थपदैर्नयनैः प्राकृतैरपि
स्वकर्मबन्धप्राप्तोऽयं यदुभिस्ते पराभवः
हतेषु सर्वानीकेषु नृपो बार्हद्रथस्तदा
उपेक्षितो भगवता मगधान्दुर्मना ययौ
मुकुन्दोऽप्यक्षतबलो निस्तीर्णारिबलार्णवः
विकीर्यमाणः कुसुमैस्त्रीदशैरनुमोदितः
माथुरैरुपसङ्गम्य विज्वरैर्मुदितात्मभिः
उपगीयमानविजयः सूतमागधवन्दिभिः
शङ्खदुन्दुभयो नेदुर्भेरीतूर्याण्यनेकशः
वीणावेणुमृदङ्गानि पुरं प्रविशति प्रभौ
सिक्तमार्गां हृष्टजनां पताकाभिरभ्यलङ्कृताम्‌
निर्घुष्टां ब्रह्मघोषेण कौतुकाबद्धतोरणाम्‌
निचीयमानो नारीभिर्माल्यदध्यक्षताङ्कुरैः
निरीक्ष्यमाणः सस्नेहं प्रीत्युत्कलितलोचनैः
आयोधनगतं वित्तमनन्तं वीरभूषणम्‌
यदुराजाय तत्सर्वमाहृतं प्रादिशत्प्रभुः
एवं सप्तदशकृत्वस्तावत्यक्षौहिणीबलः
युयुधे मागधो राजा यदुभिः कृष्णपालितैः
अक्षिण्वंस्तद्बलं सर्वं वृष्णयः कृष्णतेजसा
हतेषु स्वेष्वनीकेषु त्यक्तोऽगादरिभिर्नृपः
अष्टादशम सङ्ग्राम आगामिनि तदन्तरा
नारदप्रेषितो वीरो यवनः प्रत्यदृश्यत
रुरोध मथुरामेत्य तिसृभिर्म्लेच्छकोटिभिः
नृलोके चाप्रतिद्वन्द्वो वृष्णीन्श्रुत्वात्मसम्मितान्‌
तं दृष्ट्वाचिन्तयत्कृष्णः सङ्कर्षण सहायवान्‌
अहो यदूनां वृजिनं प्राप्तं ह्युभयतो महत्‌
यवनोऽयं निरुन्धेऽस्मानद्य तावन्महाबलः
मागधोऽप्यद्य वा श्वो वा परश्वो वागमिष्यति
आवयोः युध्यतोरस्य यद्यागन्ता जरासुतः
बन्धून्हनिष्यत्यथ वा नेष्यते स्वपुरं बली
तस्मादद्य विधास्यामो दुर्गं द्विपददुर्गमम्‌
तत्र ज्ञातीन्समाधाय यवनं घातयामहे
इति सम्मन्त्र्य भगवान्दुर्गं द्वादशयोजनम्‌
अन्तःसमुद्रे नगरं कृत्स्नाद्भुतमचीकरत्‌
दृश्यते यत्र हि त्वाष्ट्रं विज्ञानं शिल्पनैपुणम्‌
रथ्याचत्वरवीथीभिर्यथावास्तु विनिर्मितम्‌
सुरद्रुमलतोद्यान विचित्रोपवनान्वितम्‌
हेमशृङ्गैर्दिविस्पृग्भिः स्फटिकाट्टालगोपुरैः
राजतारकुटैः कोष्ठैर्हेमकुम्भैरलङ्कृतैः
रत्नकूतैर्गृहैर्हेमैर्महामारकतस्थलैः
वास्तोष्पतीनां च गृहैर्वल्लभीभिश्च निर्मितम्‌
चातुर्वर्ण्यजनाकीर्णं यदुदेवगृहोल्लसत्‌
सुधर्मां पारिजातं च महेन्द्रः प्राहिणोद्धरेः
यत्र चावस्थितो मर्त्यो मर्त्यधर्मैर्न युज्यते
श्यामैकवर्णान्वरुणो हयान्शुक्लान्मनोजवान्‌
अष्टौ निधिपतिः कोशान्लोकपालो निजोदयान्‌
यद्यद्भगवता दत्तमाधिपत्यं स्वसिद्धये
सर्वं प्रत्यर्पयामासुर्हरौ भूमिगते नृप
तत्र योगप्रभावेन नीत्वा सर्वजनं हरिः
प्रजापालेन रामेण कृष्णः समनुमन्त्रितः
निर्जगाम पुरद्वारात्पद्ममाली निरायुधः
श्रीशुक उवाच
तं विलोक्य विनिष्क्रान्तमुज्जिहानमिवोडुपम्‌
दर्शनीयतमं श्यामं पीतकौशेयवाससम्‌
श्रीवत्सवक्षसं भ्राजत्कौस्तुभामुक्तकन्धरम्‌
पृथुदीर्घचतुर्बाहुं नवकञ्जारुणेक्षणम्‌
नित्यप्रमुदितं श्रीमत्सुकपोलं शुचिस्मितम्‌
मुखारविन्दं बिभ्राणं स्फुरन्मकरकुण्डलम्‌
वासुदेवो ह्ययमिति पुमान्श्रीवत्सलाञ्छनः
चतुर्भुजोऽरविन्दाक्षो वनमाल्यतिसुन्दरः
लक्षणैर्नारदप्रोक्तैर्नान्यो भवितुमर्हति
निरायुधश्चलन्पद्भ्यां योत्स्येऽनेन निरायुधः
इति निश्चित्य यवनः प्राद्रवद्तं पराङ्मुखम्‌
अन्वधावज्जिघृक्षुस्तं दुरापमपि योगिनाम्‌
हस्तप्राप्तमिवात्मानं हरीणा स पदे पदे
नीतो दर्शयता दूरं यवनेशोऽद्रिकन्दरम्‌
पलायनं यदुकुले जातस्य तव नोचितम्‌
इति क्षिपन्ननुगतो नैनं प्रापाहताशुभः
एवं क्षिप्तोऽपि भगवान्प्राविशद्गिरिकन्दरम्‌
सोऽपि प्रविष्टस्तत्रान्यं शयानं ददृशे नरम्‌
नन्वसौ दूरमानीय शेते मामिह साधुवत्‌
इति मत्वाच्युतं मूढस्तं पदा समताडयत्‌
स उत्थाय चिरं सुप्तः शनैरुन्मील्य लोचने
दिशो विलोकयन्पार्श्वे तमद्राक्षीदवस्थितम्‌
स तावत्तस्य रुष्टस्य दृष्टिपातेन भारत
देहजेनाग्निना दग्धो भस्मसादभवत्क्षणात्‌
श्रीराजोवाच
को नाम स पुमान्ब्रह्मन्कस्य किंवीर्य एव च
कस्माद्गुहां गतः शिष्ये किंतेजो यवनार्दनः
श्रीशुक उवाच
स इक्ष्वाकुकुले जातो मान्धातृतनयो महान्‌
मुचुकुन्द इति ख्यातो ब्रह्मण्यः सत्यसङ्गरः
स याचितः सुरगणैरिन्द्राद्यैरात्मरक्षणे
असुरेभ्यः परित्रस्तैस्तद्रक्षां सोऽकरोच्चिरम्‌
लब्ध्वा गुहं ते स्वःपालं मुचुकुन्दमथाब्रुवन्‌
राजन्विरमतां कृच्छ्राद्भवान्नः परिपालनात्‌
नरलोकं परित्यज्य राज्यं निहतकण्टकम्‌
अस्मान्पालयतो वीर कामास्ते सर्व उज्झिताः
सुता महिष्यो भवतो ज्ञातयोऽमात्यमन्त्रिनः
प्रजाश्च तुल्यकालीना नाधुना सन्ति कालिताः
कालो बलीयान्बलिनां भगवानीश्वरोऽव्ययः
प्रजाः कालयते क्रीडन्पशुपालो यथा पशून्‌
वरं वृणीष्व भद्रं ते ऋते कैवल्यमद्य नः
एक एवेश्वरस्तस्य भगवान्विष्णुरव्ययः
एवमुक्तः स वै देवानभिवन्द्य महायशाः
अशयिष्ट गुहाविष्टो निद्रया देवदत्तया
यवने भस्मसान्नीते भगवान्सात्वतर्षभः
आत्मानं दर्शयामास मुचुकुन्दाय धीमते
तमालोक्य घनश्यामं पीतकौशेयवाससम्‌
श्रीवत्सवक्षसं भ्राजत्कौस्तुभेन विराजितम्‌
चतुर्भुजं रोचमानं वैजयन्त्या च मालया
चारुप्रसन्नवदनं स्फुरन्मकरकुण्डलम्‌
प्रेक्षणीयं नृलोकस्य सानुरागस्मितेक्षणम्‌
अपीव्यवयसं मत्त मृगेन्द्रोदारविक्रमम्‌
पर्यपृच्छन्महाबुद्धिस्तेजसा तस्य धर्षितः
शङ्कितः शनकै राजा दुर्धर्षमिव तेजसा
श्रीमुचुकुन्द उवाच
को भवानिह सम्प्राप्तो विपिने गिरिगह्वरे
पद्भ्यां पद्मपलाशाभ्यां विचरस्युरुकण्टके
किं स्वित्तेजस्विनां तेजो भगवान्वा विभावसुः
सूर्यः सोमो महेन्द्रो वा लोकपालो परोऽपि वा
मन्ये त्वां देवदेवानां त्रयाणां पुरुषर्षभम्‌
यद्बाधसे गुहाध्वान्तं प्रदीपः प्रभया यथा
शुश्रूषतामव्यलीकमस्माकं नरपुङ्गव
स्वजन्म कर्म गोत्रं वा कथ्यतां यदि रोचते
वयं तु पुरुषव्याघ्र ऐक्ष्वाकाः क्षत्रबन्धवः
मुचुकुन्द इति प्रोक्तो यौवनाश्वात्मजः प्रभो
चिरप्रजागरश्रान्तो निद्रयापहतेन्द्रियः
शयेऽस्मिन्विजने कामं केनाप्युत्थापितोऽधुना
सोऽपि भस्मीकृतो नूनमात्मीयेनैव पाप्मना
अनन्तरं भवान्श्रीमांल्लक्षितोऽमित्रशासनः
तेजसा तेऽविषह्येण भूरि द्रष्टुं न शक्नुमः
हतौजसा महाभाग माननीयोऽसि देहिनाम्‌
एवं सम्भाषितो राज्ञा भगवान्भूतभावनः
प्रत्याह प्रहसन्वाण्या मेघनादगभीरया
श्रीभगवानुवाच
जन्मकर्माभिधानानि सन्ति मेऽङ्ग सहस्रशः
न शक्यन्तेऽनुसङ्ख्यातुमनन्तत्वान्मयापि हि
क्वचिद्रजांसि विममे पार्थिवान्युरुजन्मभिः
गुणकर्माभिधानानि न मे जन्मानि कर्हिचित्‌
कालत्रयोपपन्नानि जन्मकर्माणि मे नृप
अनुक्रमन्तो नैवान्तं गच्छन्ति परमर्षयः
तथाप्यद्यतनान्यङ्ग शृनुष्व गदतो मम
विज्ञापितो विरिञ्चेन पुराहं धर्मगुप्तये
भूमेर्भारायमाणानामसुराणां क्षयाय च
अवतीर्णो यदुकुले गृह आनकदुन्दुभेः
वदन्ति वासुदेवेति वसुदेवसुतं हि माम्‌
कालनेमिर्हतः कंसः प्रलम्बाद्याश्च सद्द्विषः
अयं च यवनो दग्धो राजंस्ते तिग्मचक्षुषा
सोऽहं तवानुग्रहार्थं गुहामेतामुपागतः
प्रार्थितः प्रचुरं पूर्वं त्वयाहं भक्तवत्सलः
वरान्वृणीष्व राजर्षे सर्वान्कामान्ददामि ते
मां प्रसन्नो जनः कश्चिन्न भूयोऽर्हति शोचितुम्‌
श्रीशुक उवाच
इत्युक्तस्तं प्रणम्याह मुचुकुन्दो मुदान्वितः
ज्ञात्वा नारायणं देवं गर्गवाक्यमनुस्मरन्‌
श्रीमुचुकुन्द उवाच
विमोहितोऽयं जन ईश मायया त्वदीयया त्वां न भजत्यनर्थदृक्‌
सुखाय दुःखप्रभवेषु सज्जते गृहेषु योषित्पुरुषश्च वञ्चितः
लब्ध्वा जनो दुर्लभमत्र मानुषं
कथञ्चिदव्यङ्गमयत्नतोऽनघ
पादारविन्दं न भजत्यसन्मतिर्‌
गृहान्धकूपे पतितो यथा पशुः
ममैष कालोऽजित निष्फलो गतो राज्यश्रियोन्नद्धमदस्य भूपतेः
मर्त्यात्मबुद्धेः सुतदारकोशभूष्वासज्जमानस्य दुरन्तचिन्तया
कलेवरेऽस्मिन्घटकुड्यसन्निभे
निरूढमानो नरदेव इत्यहम्‌
वृतो रथेभाश्वपदात्यनीकपैर्‌
गां पर्यटंस्त्वागणयन्सुदुर्मदः
प्रमत्तमुच्चैरितिकृत्यचिन्तया प्रवृद्धलोभं विषयेषु लालसम्‌
त्वमप्रमत्तः सहसाभिपद्यसे क्षुल्लेलिहानोऽहिरिवाखुमन्तकः
पुरा रथैर्हेमपरिष्कृतैश्चरन्‌
मतंगजैर्वा नरदेवसंज्ञितः
स एव कालेन दुरत्ययेन ते
कलेवरो विट्कृमिभस्मसंज्ञितः
निर्जित्य दिक्चक्रमभूतविग्रहो वरासनस्थः समराजवन्दितः
गृहेषु मैथुन्यसुखेषु योषितां क्रीडामृगः पूरुष ईश नीयते
करोति कर्माणि तपःसुनिष्ठितो निवृत्तभोगस्तदपेक्षयाददत्‌
पुनश्च भूयासमहं स्वराडिति प्रवृद्धतर्षो न सुखाय कल्पते
भवापवर्गो भ्रमतो यदा भवेज्जनस्य तर्ह्यच्युत सत्समागमः
सत्सङ्गमो यर्हि तदैव सद्गतौ परावरेशे त्वयि जायते मतिः
मन्ये ममानुग्रह ईश ते कृतो राज्यानुबन्धापगमो यदृच्छया
यः प्रार्थ्यते साधुभिरेकचर्यया वनं विविक्षद्भिरखण्डभूमिपैः
न कामयेऽन्यं तव पादसेवनादकिञ्चनप्रार्थ्यतमाद्वरं विभो
आराध्य कस्त्वां ह्यपवर्गदं हरे वृणीत आर्यो वरमात्मबन्धनम्‌
तस्माद्विसृज्याशिष ईश सर्वतो रजस्तमःसत्त्वगुणानुबन्धनाः
निरञ्जनं निर्गुणमद्वयं परं त्वां ज्ञाप्तिमात्रं पुरुषं व्रजाम्यहम्‌
चिरमिह वृजिनार्तस्तप्यमानोऽनुतापैर्‌
अवितृषषडमित्रोऽलब्धशान्तिः कथञ्चित्‌
शरणद समुपेतस्त्वत्पदाब्जं परात्मन्‌
अभयमृतमशोकं पाहि मापन्नमीश
श्रीभगवानुवाच
सार्वभौम महाराज मतिस्ते विमलोर्जिता
वरैः प्रलोभितस्यापि न कामैर्विहता यतः
प्रलोभितो वरैर्यत्त्वमप्रमादाय विद्धि तत्‌
न धीरेकान्तभक्तानामाशीर्भिर्भिद्यते क्वचित्‌
युञ्जानानामभक्तानां प्राणायामादिभिर्मनः
अक्षीणवासनं राजन्दृश्यते पुनरुत्थितम्‌
विचरस्व महीं कामं मय्यावेशितमानसः
अस्त्वेवं नित्यदा तुभ्यं भक्तिर्मय्यनपायिनी
क्षात्रधर्मस्थितो जन्तून्न्यवधीर्मृगयादिभिः
समाहितस्तत्तपसा जह्यघं मदुपाश्रितः
जन्मन्यनन्तरे राजन्सर्वभूतसुहृत्तमः
भूत्वा द्विजवरस्त्वं वै मामुपैष्यसि केवलम्‌
श्रीशुक उवाच
इत्थं सोऽनग्रहीतोऽन्ग कृष्णेनेक्ष्वाकु नन्दनः
तं परिक्रम्य सन्नम्य निश्चक्राम गुहामुखात्‌
संवीक्ष्य क्षुल्लकान्मर्त्यान्पशून्वीरुद्वनस्पतीन्‌
मत्वा कलियुगं प्राप्तं जगाम दिशमुत्तराम्‌
तपःश्रद्धायुतो धीरो निःसङ्गो मुक्तसंशयः
समाधाय मनः कृष्णे प्राविशद्गन्धमादनम्‌
बदर्याश्रममासाद्य नरनारायणालयम्‌
सर्वद्वन्द्वसहः शान्तस्तपसाराधयद्धरिम्‌
भगवान्पुनराव्रज्य पुरीं यवनवेष्टिताम्‌
हत्वा म्लेच्छबलं निन्ये तदीयं द्वारकां धनम्‌
नीयमाने धने गोभिर्नृभिश्चाच्युतचोदितैः
आजगाम जरासन्धस्त्रयोविंशत्यनीकपः
विलोक्य वेगरभसं रिपुसैन्यस्य माधवौ
मनुष्यचेष्टामापन्नौ राजन्दुद्रुवतुर्द्रुतम्‌
विहाय वित्तं प्रचुरमभीतौ भीरुभीतवत्‌
पद्भ्यां पलाशाभ्यां चेलतुर्बहुयोजनम्‌
पलायमानौ तौ दृष्ट्वा मागधः प्रहसन्बली
अन्वधावद्रथानीकैरीशयोरप्रमाणवित्‌
प्रद्रुत्य दूरं संश्रान्तौ तुङ्गमारुहतां गिरिम्‌
प्रवर्षणाख्यं भगवान्नित्यदा यत्र वर्षति
गिरौ निलीनावाज्ञाय नाधिगम्य पदं नृप
ददाह गिरिमेधोभिः समन्तादग्निमुत्सृजन्‌
तत उत्पत्य तरसा दह्यमानतटादुभौ
दशैकयोजनात्तुङ्गान्निपेततुरधो भुवि
अलक्ष्यमाणौ रिपुणा सानुगेन यदूत्तमौ
स्वपुरं पुनरायातौ समुद्रपरिखां नृप
सोऽपि दग्धाविति मृषा मन्वानो बलकेशवौ
बलमाकृष्य सुमहन्मगधान्मागधो ययौ
आनर्ताधिपतिः श्रीमान्रैवतो रैवतीं सुताम्‌
ब्रह्मणा चोदितः प्रादाद्बलायेति पुरोदितम्‌
भगवानपि गोविन्द उपयेमे कुरूद्वह
वैदर्भीं भीष्मकसुतां श्रियो मात्रां स्वयंवरे
प्रमथ्य तरसा राज्ञः शाल्वादींश्चैद्यपक्षगान्‌
पश्यतां सर्वलोकानां तार्क्ष्यपुत्रः सुधामिव
श्रीराजोवाच
भगवान्भीष्मकसुतां रुक्मिणीं रुचिराननाम्‌
राक्षसेन विधानेन उपयेम इति श्रुतम्‌
भगवन्श्रोतुमिच्छामि कृष्णस्यामिततेजसः
यथा मागधशाल्वादीन्जित्वा कन्यामुपाहरत्‌
ब्रह्मन्कृष्णकथाः पुण्या माध्वीर्लोकमलापहाः
को नु तृप्येत शृण्वानः श्रुतज्ञो नित्यनूतनाः
श्रीबादरायणिरुवाच
राजासीद्भीष्मको नाम विदर्भाधिपतिर्महान्‌
तस्य पन्चाभवन्पुत्राः कन्यैका च वरानना
रुक्म्यग्रजो रुक्मरथो रुक्मबाहुरनन्तरः
रुक्मकेशो रुक्ममाली रुक्मिण्येषा स्वसा सती
सोपश्रुत्य मुकुन्दस्य रूपवीर्यगुणश्रियः
गृहागतैर्गीयमानास्तं मेने सदृशं पतिम्‌
तां बुद्धिलक्षणौदार्य रूपशीलगुणाश्रयाम्‌
कृष्णश्च सदृशीं भार्यां समुद्वोढुं मनो दधे
बन्धूनामिच्छतां दातुं कृष्णाय भगिनीं नृप
ततो निवार्य कृष्णद्विड्रुक्मी चैद्यममन्यत
तदवेत्यासितापाङ्गी वैदर्भी दुर्मना भृशम्‌
विचिन्त्याप्तं द्विजं कञ्चित्कृष्णाय प्राहिणोद्द्रुतम्‌
द्वारकां स समभ्येत्य प्रतीहारैः प्रवेशितः
अपश्यदाद्यं पुरुषमासीनं काञ्चनासने
दृष्ट्वा ब्रह्मण्यदेवस्तमवरुह्य निजासनात्‌
उपवेश्यार्हयां चक्रे यथात्मानं दिवौकसः
तं भुक्तवन्तं विश्रान्तमुपगम्य सतां गतिः
पाणिनाभिमृशन्पादावव्यग्रस्तमपृच्छत
कच्चिद्द्विजवरश्रेष्ठ धर्मस्ते वृद्धसम्मतः
वर्तते नातिकृच्छ्रेण सन्तुष्टमनसः सदा
सन्तुष्टो यर्हि वर्तेत ब्राह्मणो येन केनचित्‌
अहीयमानः स्वद्धर्मात्स ह्यस्याखिलकामधुक्‌
असन्तुष्टोऽसकृल्लोकानाप्नोत्यपि सुरेश्वरः
अकिञ्चनोऽपि सन्तुष्टः शेते सर्वाङ्गविज्वरः
विप्रान्स्वलाभसन्तुष्टान्साधून्भूतसुहृत्तमान्‌
निरहङ्कारिणः शान्तान्नमस्ये शिरसासकृत्‌
कच्चिद्वः कुशलं ब्रह्मन्राजतो यस्य हि प्रजाः
सुखं वसन्ति विषये पाल्यमानाः स मे प्रियः
यतस्त्वमागतो दुर्गं निस्तीर्येह यदिच्छया
सर्वं नो ब्रूह्यगुह्यं चेत्किं कार्यं करवाम ते
एवं सम्पृष्टसम्प्रश्नो ब्राह्मणः परमेष्ठिना
लीलागृहीतदेहेन तस्मै सर्वमवर्णयत्‌
श्रीरुक्मिण्युवाच
श्रुत्वा गुणान्भुवनसुन्दर शृण्वतां ते
निर्विश्य कर्णविवरैर्हरतोऽङ्गतापम्‌
रूपं दृशां दृशिमतामखिलार्थलाभं
त्वय्यच्युताविशति चित्तमपत्रपं मे
का त्वा मुकुन्द महती कुलशीलरूप
विद्यावयोद्रविणधामभिरात्मतुल्यम्‌
धीरा पतिं कुलवती न वृणीत कन्या
काले नृसिंह नरलोकमनोऽभिरामम्‌
तन्मे भवान्खलु वृतः पतिरङ्ग जायाम्‌
आत्मार्पितश्च भवतोऽत्र विभो विधेहि
मा वीरभागमभिमर्शतु चैद्य आराद्‌
गोमायुवन्मृगपतेर्बलिमम्बुजाक्ष
पूर्तेष्टदत्तनियमव्रतदेवविप्र
गुर्वर्चनादिभिरलं भगवान्परेशः
आराधितो यदि गदाग्रज एत्य पाणिं
गृह्णातु मे न दमघोषसुतादयोऽन्ये
श्वो भाविनि त्वमजितोद्वहने विदर्भान्‌
गुप्तः समेत्य पृतनापतिभिः परीतः
निर्मथ्य चैद्यमगधेन्द्रबलं प्रसह्य
मां राक्षसेन विधिनोद्वह वीर्यशुल्काम्‌
अन्तःपुरान्तरचरीमनिहत्य बन्धून्‌
त्वामुद्वहे कथमिति प्रवदाम्युपायम्‌
पूर्वेद्युरस्ति महती कुलदेवयात्रा
यस्यां बहिर्नववधूर्गिरिजामुपेयात्‌
यस्याङ्घ्रिपङ्कजरजःस्नपनं महान्तो
वाञ्छन्त्युमापतिरिवात्मतमोऽपहत्यै
यर्ह्यम्बुजाक्ष न लभेय भवत्प्रसादं
जह्यामसून्व्रतकृशान्शतजन्मभिः स्यात्‌
ब्राह्मण उवाच
इत्येते गुह्यसन्देशा यदुदेव मयाहृताः
विमृश्य कर्तुं यच्चात्र क्रियतां तदनन्तरम्‌
श्रीशुक उवाच
वैदर्भ्याः स तु सन्देशं निशम्य यदुनन्दनः
प्रगृह्य पाणिना पाणिं प्रहसन्निदमब्रवीत्‌
श्रीभगवानुवाच
तथाहमपि तच्चित्तो निद्रां च न लभे निशि
वेदाहम्रुक्मिणा द्वेषान्ममोद्वाहो निवारितः
तामानयिष्य उन्मथ्य राजन्यापसदान्मृधे
मत्परामनवद्याङ्गीमेधसोऽग्निशिखामिव
श्रीशुक उवाच
उद्वाहर्क्षं च विज्ञाय रुक्मिण्या मधुसूदनः
रथः संयुज्यतामाशु दारुकेत्याह सारथिम्‌
स चाश्वैः शैब्यसुग्रीव मेघपुष्पबलाहकैः
युक्तं रथमुपानीय तस्थौ प्राञ्जलिरग्रतः
आरुह्य स्यन्दनं शौरिर्द्विजमारोप्य तूर्णगैः
आनर्तादेकरात्रेण विदर्भानगमद्धयैः
राजा स कुण्डिनपतिः पुत्रस्नेहवशानुगः
शिशुपालाय स्वां कन्यां दास्यन्कर्माण्यकारयत्‌
पुरं सम्मृष्टसंसिक्त मार्गरथ्याचतुष्पथम्‌
चित्रध्वजपताकाभिस्तोरणैः समलङ्कृतम्‌
स्रग्गन्धमाल्याभरणैर्विरजोऽम्बरभूषितैः
जुष्टं स्त्रीपुरुषैः श्रीमद्‌ गृहैरगुरुधूपितैः
पित्ऱ्‌^ईन्देवान्समभ्यर्च्य विप्रांश्च विधिवन्नृप
भोजयित्वा यथान्यायं वाचयामास मङ्गलम्‌
सुस्नातां सुदतीं कन्यां कृतकौतुकमङ्गलाम्‌
आहतांशुकयुग्मेन भूषितां भूषणोत्तमैः
चक्रुः सामर्ग्यजुर्मन्त्रैर्वध्वा रक्षां द्विजोत्तमाः
पुरोहितोऽथर्वविद्वै जुहाव ग्रहशान्तये
हिरण्यरूप्य वासांसि तिलांश्च गुडमिश्रितान्‌
प्रादाद्धेनूश्च विप्रेभ्यो राजा विधिविदां वरः
एवं चेदिपती राजा दमघोषः सुताय वै
कारयामास मन्त्रज्ञैः सर्वमभ्युदयोचितम्‌
मदच्युद्भिर्गजानीकैः स्यन्दनैर्हेममालिभिः
पत्त्यश्वसङ्कुलैः सैन्यैः परीतः कुण्दीनं ययौ
तं वै विदर्भाधिपतिः समभ्येत्याभिपूज्य च
निवेशयामास मुदा कल्पितान्यनिवेशने
तत्र शाल्वो जरासन्धो दन्तवक्रो विदूरथः
आजग्मुश्चैद्यपक्षीयाः पौण्ड्रकाद्याः सहस्रशः
कृष्णरामद्विषो यत्ताः कन्यां चैद्याय साधितुम्‌
यद्यागत्य हरेत्कृष्नो रामाद्यैर्यदुभिर्वृतः
योत्स्यामः संहतास्तेन इति निश्चितमानसाः
आजग्मुर्भूभुजः सर्वे समग्रबलवाहनाः
श्रुत्वैतद्भगवान्रामो विपक्षीय नृपोद्यमम्‌
कृष्णं चैकं गतं हर्तुं कन्यां कलहशङ्कितः
बलेन महता सार्धं भ्रातृस्नेहपरिप्लुतः
त्वरितः कुण्डिनं प्रागाद्गजाश्वरथपत्तिभिः
भीष्मकन्या वरारोहा काङ्क्षन्त्यागमनं हरेः
प्रत्यापत्तिमपश्यन्ती द्विजस्याचिन्तयत्तदा
अहो त्रियामान्तरित उद्वाहो मेऽल्पराधसः
नागच्छत्यरविन्दाक्षो नाहं वेद्म्यत्र कारणम्‌
सोऽपि नावर्ततेऽद्यापि मत्सन्देशहरो द्विजः
अपि मय्यनवद्यात्मा दृष्ट्वा किञ्चिज्जुगुप्सितम्‌
मत्पाणिग्रहणे नूनं नायाति हि कृतोद्यमः
दुर्भगाया न मे धाता नानुकूलो महेश्वरः
देवी वा विमुखी गौरी रुद्राणी गिरिजा सती
एवं चिन्तयती बाला गोविन्दहृतमानसा
न्यमीलयत कालज्ञा नेत्रे चाश्रुकलाकुले
एवं वध्वाः प्रतीक्षन्त्या गोविन्दागमनं नृप
वाम ऊरुर्भुजो नेत्रमस्फुरन्प्रियभाषिणः
अथ कृष्णविनिर्दिष्टः स एव द्विजसत्तमः
अन्तःपुरचरीं देवीं राजपुत्रीम्ददर्श ह
सा तं प्रहृष्टवदनमव्यग्रात्मगतिं सती
आलक्ष्य लक्षणाभिज्ञा समपृच्छच्छुचिस्मिता
तस्या आवेदयत्प्राप्तं शशंस यदुनन्दनम्‌
उक्तं च सत्यवचनमात्मोपनयनं प्रति
तमागतं समाज्ञाय वैदर्भी हृष्टमानसा
न पश्यन्ती ब्राह्मणाय प्रियमन्यन्ननाम सा
प्राप्तौ श्रुत्वा स्वदुहितुरुद्वाहप्रेक्षणोत्सुकौ
अभ्ययात्तूर्यघोषेण रामकृष्णौ समर्हणैः
मधुपर्कमुपानीय वासांसि विरजांसि सः
उपायनान्यभीष्टानि विधिवत्समपूजयत्‌
तयोर्निवेशनं श्रीमदुपाकल्प्य महामतिः
ससैन्ययोः सानुगयोरातिथ्यं विदधे यथा
एवं राज्ञां समेतानां यथावीर्यं यथावयः
यथाबलं यथावित्तं सर्वैः कामैः समर्हयत्‌
कृष्णमागतमाकर्ण्य विदर्भपुरवासिनः
आगत्य नेत्राञ्जलिभिः पपुस्तन्मुखपङ्कजम्‌
अस्यैव भार्या भवितुं रुक्मिण्यर्हति नापरा
असावप्यनवद्यात्मा भैष्म्याः समुचितः पतिः
किञ्चित्सुचरितं यन्नस्तेन तुष्टस्त्रिलोककृत्‌
अनुगृह्णातु गृह्णातु वैदर्भ्याः पाणिमच्युतः
एवं प्रेमकलाबद्धा वदन्ति स्म पुरौकसः
कन्या चान्तःपुरात्प्रागाद्भटैर्गुप्ताम्बिकालयम्‌
पद्भ्यां विनिर्ययौ द्रष्टुं भवान्याः पादपल्लवम्‌
सा चानुध्यायती सम्यङ्मुकुन्दचरणाम्बुजम्‌
यतवाङ्मातृभिः सार्धं सखीभिः परिवारिता
गुप्ता राजभटैः शूरैः सन्नद्धैरुद्यतायुधैः
मृडङ्गशङ्खपणवास्तूर्यभेर्यश्च जघ्निरे
नानोपहार बलिभिर्वारमुख्याः सहस्रशः
स्रग्गन्धवस्त्राभरणैर्द्विजपत्न्यः स्वलङ्कृताः
गायन्त्यश्च स्तुवन्तश्च गायका वाद्यवादकाः
परिवार्य वधूं जग्मुः सूतमागधवन्दिनः
आसाद्य देवीसदनं धौतपादकराम्बुजा
उपस्पृश्य शुचिः शान्ता प्रविवेशाम्बिकान्तिकम्‌
तां वै प्रवयसो बालां विधिज्ञा विप्रयोषितः
भवानीं वन्दयां चक्रुर्भवपत्नीं भवान्विताम्‌
नमस्ये त्वाम्बिकेऽभीक्ष्णं स्वसन्तानयुतां शिवाम्‌
भूयात्पतिर्मे भगवान्कृष्णस्तदनुमोदताम्‌
अद्भिर्गन्धाक्षतैर्धूपैर्वासःस्रङ्माल्य भूषणैः
नानोपहारबलिभिः प्रदीपावलिभिः पृथक्‌
विप्रस्त्रियः पतिमतीस्तथा तैः समपूजयत्‌
लवणापूपताम्बूल कण्ठसूत्रफलेक्षुभिः
तस्यै स्त्रियस्ताः प्रददुः शेषां युयुजुराशिषः
ताभ्यो देव्यै नमश्चक्रे शेषां च जगृहे वधूः
मुनिव्रतमथ त्यक्त्वा निश्चक्रामाम्बिकागृहात्‌
प्रगृह्य पाणिना भृत्यां रत्नमुद्रोपशोभिना
तां देवमायामिव धीरमोहिनीं सुमध्यमां कुण्डलमण्डिताननाम्‌
श्यामां नितम्बार्पितरत्नमेखलां व्यञ्जत्स्तनीं कुन्तलशङ्कितेक्षणाम्‌
शुचिस्मितां बिम्बफलाधरद्युति शोणायमानद्विजकुन्दकुड्मलाम्‌
पदा चलन्तीं कलहंसगामिनीं सिञ्जत्कलानूपुरधामशोभिना
विलोक्य वीरा मुमुहुः समागता यशस्विनस्तत्कृतहृच्छयार्दिताः
यां वीक्ष्य ते नृपतयस्तदुदारहास व्रीदावलोकहृतचेतस उज्झितास्त्राः
पेतुः क्षितौ गजरथाश्वगता विमूढा यात्राच्छलेन हरयेऽर्पयतीं स्वशोभाम्‌
सैवं शनैश्चलयती चलपद्मकोशौ प्राप्तिं तदा भगवतः प्रसमीक्षमाणा
उत्सार्य वामकरजैरलकानपङ्गैः प्राप्तान्ह्रियैक्षत नृपान्ददृशेऽच्युतं च
तां राजकन्यां रथमारुरक्षतीं जहार कृष्णो द्विषतां समीक्षताम्‌
रथं समारोप्य सुपर्णलक्षणं राजन्यचक्रं परिभूय माधवः
ततो ययौ रामपुरोगमः शनैः शृगालमध्यादिव भागहृद्धरिः
तं मानिनः स्वाभिभवं यशःक्षयं
परे जरासन्धमुखा न सेहिरे
अहो धिगस्मान्यश आत्तधन्वनां
गोपैर्हृतं केशरिणां मृगैरिव
श्रीशुक उवाच
इति सर्वे सुसंरब्धा वाहानारुह्य दंशिताः
स्वैः स्वैर्बलैः परिक्रान्ता अन्वीयुर्धृतकार्मुकाः
तानापतत आलोक्य यादवानीकयूथपाः
तस्थुस्तत्सम्मुखा राजन्विस्फूर्ज्य स्वधनूंषि ते
अश्वपृष्ठे गजस्कन्धे रथोपस्थेऽस्त्र कोविदाः
मुमुचुः शरवर्षाणि मेघा अद्रिष्वपो यथा
पत्युर्बलं शरासारैश्छन्नं वीक्ष्य सुमध्यमा
सव्रीड्मैक्षत्तद्वक्त्रं भयविह्वललोचना
प्रहस्य भगवानाह मा स्म भैर्वामलोचने
विनङ्क्ष्यत्यधुनैवैतत्तावकैः शात्रवं बलम्‌
तेषां तद्विक्रमं वीरा गदसङ्कर्षनादयः
अमृष्यमाणा नाराचैर्जघ्नुर्हयगजान्रथान्‌
पेतुः शिरांसि रथिनामश्विनां गजिनां भुवि
सकुण्डलकिरीटानि सोष्णीषाणि च कोटिशः
हस्ताः सासिगदेष्वासाः करभा ऊरवोऽङ्घ्रयः
अश्वाश्वतरनागोष्ट्र खरमर्त्यशिरांसि च
हन्यमानबलानीका वृष्णिभिर्जयकाङ्क्षिभिः
राजानो विमुखा जग्मुर्जरासन्धपुरःसराः
शिशुपालं समभ्येत्य हृतदारमिवातुरम्‌
नष्टत्विषं गतोत्साहं शुष्यद्वदनमब्रुवन्‌
भो भोः पुरुषशार्दूल दौर्मनस्यमिदं त्यज
न प्रियाप्रिययो राजन्निष्ठा देहिषु दृश्यते
यथा दारुमयी योषित्नृत्यते कुहकेच्छया
एवमीश्वरतन्त्रोऽयमीहते सुखदुःखयोः
शौरेः सप्तदशाहं वै संयुगानि पराजितः
त्रयोविंशतिभिः सैन्यैर्जिग्ये एकमहं परम्‌
तथाप्यहं न शोचामि न प्रहृष्यामि कर्हिचित्‌
कालेन दैवयुक्तेन जानन्विद्रावितं जगत्‌
अधुनापि वयं सर्वे वीरयूथपयूथपाः
पराजिताः फल्गुतन्त्रैर्यदुभिः कृष्णपालितैः
रिपवो जिग्युरधुना काल आत्मानुसारिणि
तदा वयं विजेष्यामो यदा कालः प्रदक्षिणः
श्रीशुक उवाच
एवं प्रबोधितो मित्रैश्चैद्योऽगात्सानुगः पुरम्‌
हतशेषाः पुनस्तेऽपि ययुः स्वं स्वं पुरं नृपाः
रुक्मी तु राक्षसोद्वाहं कृष्णद्विडसहन्स्वसुः
पृष्ठतोऽन्वगमत्कृष्णमक्षौहिण्या वृतो बली
रुक्म्यमर्षी सुसंरब्धः शृण्वतां सर्वभूभुजाम्‌
प्रतिजज्ञे महाबाहुर्दंशितः सशरासनः
अहत्वा समरे कृष्णमप्रत्यूह्य च रुक्मिणीम्‌
कुण्डिनं न प्रवेक्ष्यामि सत्यमेतद्ब्रवीमि वः
इत्युक्त्वा रथमारुह्य सारथिं प्राह सत्वरः
चोदयाश्वान्यतः कृष्णः तस्य मे संयुगं भवेत्‌
अद्याहं निशितैर्बाणैर्गोपालस्य सुदुर्मतेः
नेष्ये वीर्यमदं येन स्वसा मे प्रसभं हृता
विकत्थमानः कुमतिरीश्वरस्याप्रमाणवित्‌
रथेनैकेन गोविन्दं तिष्ठ तिष्ठेत्यथाह्वयत्‌
धनुर्विकृष्य सुदृढं जघ्ने कृष्णं त्रिभिः शरैः
आह चात्र क्षणं तिष्ठ यदूनां कुलपांसन
यत्र यासि स्वसारं मे मुषित्वा ध्वाङ्क्षवद्धविः
हरिष्येऽद्य मदं मन्द मायिनः कूटयोधिनः
यावन्न मे हतो बाणैः शयीथा मुञ्च दारीकाम्‌
स्मयन्कृष्णो धनुश्छित्त्वा षड्भिर्विव्याध रुक्मिणम्‌
अष्टभिश्चतुरो वाहान्द्वाभ्यां सूतं ध्वजं त्रिभिः
स चान्यद्धनुराधाय कृष्णं विव्याध पञ्चभिः
तैस्तादितः शरौघैस्तु चिच्छेद धनुरच्युतः
पुनरन्यदुपादत्त तदप्यच्छिनदव्ययः
परिघं पट्टिशं शूलं चर्मासी शक्तितोमरौ
यद्यदायुधमादत्त तत्सर्वं सोऽच्छिनद्धरिः
ततो रथादवप्लुत्य खड्गपाणिर्जिघांसया
कृष्णमभ्यद्रवत्क्रुद्धः पतङ्ग इव पावकम्‌
तस्य चापततः खड्गं तिलशश्चर्म चेषुभिः
छित्त्वासिमाददे तिग्मं रुक्मिणं हन्तुमुद्यतः
दृष्ट्वा भ्रातृवधोद्योगं रुक्मिणी भयविह्वला
पतित्वा पादयोर्भर्तुरुवाच करुणं सती
श्रीरुक्मिण्युवाच
योगेश्वराप्रमेयात्मन्देवदेव जगत्पते
हन्तुं नार्हसि कल्याण भ्रातरं मे महाभुज
श्रीशुक उवाच
तया परित्रासविकम्पिताङ्गया शुचावशुष्यन्मुखरुद्धकण्ठया
कातर्यविस्रंसितहेममालया गृहीतपादः करुणो न्यवर्तत
चैलेन बद्ध्वा तमसाधुकारीणं सश्मश्रुकेशं प्रवपन्व्यरूपयत्‌
तावन्ममर्दुः परसैन्यमद्भुतं यदुप्रवीरा नलिनीं यथा गजाः
कृष्णान्तिकमुपव्रज्य ददृशुस्तत्र रुक्मिणम्‌
तथाभूतं हतप्रायं दृष्ट्वा सङ्कर्षणो विभुः
विमुच्य बद्धं करुणो भगवान्कृष्णमब्रवीत्‌
असाध्विदं त्वया कृष्ण कृतमस्मज्जुगुप्सितम्‌
वपनं श्मश्रुकेशानां वैरूप्यं सुहृदो वधः
मैवास्मान्साध्व्यसूयेथा भ्रातुर्वैरूप्यचिन्तया
सुखदुःखदो न चान्योऽस्ति यतः स्वकृतभुक्पुमान्‌
बन्धुर्वधार्हदोषोऽपि न बन्धोर्वधमर्हति
त्याज्यः स्वेनैव दोषेण हतः किं हन्यते पुनः
क्षत्रियाणामयं धर्मः प्रजापतिविनिर्मितः
भ्रातापि भ्रातरं हन्याद्येन घोरतमस्ततः
राज्यस्य भूमेर्वित्तस्य स्त्रियो मानस्य तेजसः
मानिनोऽन्यस्य वा हेतोः श्रीमदान्धाः क्षिपन्ति हि
तवेयं विषमा बुद्धिः सर्वभूतेषु दुर्हृदाम्‌
यन्मन्यसे सदाभद्रं सुहृदां भद्रमज्ञवत्‌
आत्ममोहो नृणामेव कल्पते देवमायया
सुहृद्दुर्हृदुदासीन इति देहात्ममानिनाम्‌
एक एव परो ह्यात्मा सर्वेषामपि देहिनाम्‌
नानेव गृह्यते मूढैर्यथा ज्योतिर्यथा नभः
देह आद्यन्तवानेष द्रव्यप्राणगुणात्मकः
आत्मन्यविद्यया कॢप्तः संसारयति देहिनम्‌
नात्मनोऽन्येन संयोगो वियोगश्चसतः सति
तद्धेतुत्वात्तत्प्रसिद्धेर्दृग्रूपाभ्यां यथा रवेः
जन्मादयस्तु देहस्य विक्रिया नात्मनः क्वचित्‌
कलानामिव नैवेन्दोर्मृतिर्ह्यस्य कुहूरिव
यथा शयान आत्मानं विषयान्फलमेव च
अनुभुङ्क्तेऽप्यसत्यर्थे तथाप्नोत्यबुधो भवम्‌
तस्मादज्ञानजं शोकमात्मशोषविमोहनम्‌
तत्त्वज्ञानेन निर्हृत्य स्वस्था भव शुचिस्मिते
श्रीशुक उवाच
एवं भगवता तन्वी रामेण प्रतिबोधिता
वैमनस्यं परित्यज्य मनो बुद्ध्या समादधे
प्राणावशेष उत्सृष्टो द्विड्भिर्हतबलप्रभः
स्मरन्विरूपकरणं वितथात्ममनोरथः
चक्रे भोजकटं नाम निवासाय महत्पुरम्‌
अहत्वा दुर्मतिं कृष्णमप्रत्यूह्य यवीयसीम्‌
कुण्डिनं न प्रवेक्ष्यामीत्युक्त्वा तत्रावसद्रुषा
भगवान्भीष्मकसुतामेवं निर्जित्य भूमिपान्‌
पुरमानीय विधिवदुपयेमे कुरूद्वह
तदा महोत्सवो न्ऱ्‌^ईणां यदुपुर्यां गृहे गृहे
अभूदनन्यभावानां कृष्णे यदुपतौ नृप
नरा नार्यश्च मुदिताः प्रमृष्टमणिकुण्डलाः
पारिबर्हमुपाजह्रुर्वरयोश्चित्रवाससोः
सा वृष्णिपुर्युत्तम्भितेन्द्रकेतुभिर्‌
विचित्रमाल्याम्बररत्नतोरणैः
बभौ प्रतिद्वार्युपकॢप्तमङ्गलैर्‌
आपूर्णकुम्भागुरुधूपदीपकैः
सिक्तमार्गा मदच्युद्भिराहूतप्रेष्ठभूभुजाम्‌
गजैर्द्वाःसु परामृष्ट रम्भापूगोपशोभिता
कुरुसृञ्जयकैकेय विदर्भयदुकुन्तयः
मिथो मुमुदिरे तस्मिन्सम्भ्रमात्परिधावताम्‌
रुक्मिण्या हरणं श्रुत्वा गीयमानं ततस्ततः
राजानो राजकन्याश्च बभूवुर्भृशविस्मिताः
द्वारकायामभूद्राजन्महामोदः पुरौकसाम्‌
रुक्मिण्या रमयोपेतं दृष्ट्वा कृष्णं श्रियः पतिम्‌
श्रीशुक उवाच
कामस्तु वासुदेवांशो दग्धः प्राग्रुद्रमन्युना
देहोपपत्तये भूयस्तमेव प्रत्यपद्यत
स एव जातो वैदर्भ्यां कृष्णवीर्यसमुद्भवः
प्रद्युम्न इति विख्यातः सर्वतोऽनवमः पितुः
तं शम्बरः कामरूपी हृत्वा तोकमनिर्दशम्‌
स विदित्वात्मनः शत्रुं प्रास्योदन्वत्यगाद्गृहम्‌
तं निर्जगार बलवान्मीनः सोऽप्यपरैः सह
वृतो जालेन महता गृहीतो मत्स्यजीविभिः
तं शम्बराय कैवर्ता उपाजह्रुरुपायनम्‌
सूदा महानसं नीत्वा वद्यन्सुधितिनाद्भुतम्‌
दृष्ट्वा तदुदरे बालम्मायावत्यै न्यवेदयन्‌
नारदोऽकथयत्सर्वं तस्याः शङ्कितचेतसः
बालस्य तत्त्वमुत्पत्तिं मत्स्योदरनिवेशनम्‌
सा च कामस्य वै पत्नी रतिर्नाम यशस्विनी
पत्युर्निर्दग्धदेहस्य देहोत्पत्तिम्प्रतीक्षती
निरूपिता शम्बरेण सा सूदौदनसाधने
कामदेवं शिशुं बुद्ध्वा चक्रे स्नेहं तदार्भके
नातिदीर्घेण कालेन स कार्ष्णि रूढयौवनः
जनयामास नारीणां वीक्षन्तीनां च विभ्रमम्‌
सा तम्पतिं पद्मदलायतेक्षणं प्रलम्बबाहुं नरलोकसुन्दरम्‌
सव्रीडहासोत्तभितभ्रुवेक्षती प्रीत्योपतस्थे रतिरङ्ग सौरतैः
तामह भगवान्कार्ष्णिर्मातस्ते मतिरन्यथा
मातृभावमतिक्रम्य वर्तसे कामिनी यथा
रतिरुवाच
भवान्नारायणसुतः शम्बरेण हृतो गृहात्‌
अहं तेऽधिकृता पत्नी रतिः कामो भवान्प्रभो
एष त्वानिर्दशं सिन्धावक्षिपच्छम्बरोऽसुरः
मत्स्योऽग्रसीत्तदुदरादितः प्राप्तो भवान्प्रभो
तमिमं जहि दुर्धर्षं दुर्जयं शत्रुमात्मनः
मायाशतविदं तं च मायाभिर्मोहनादिभिः
परीशोचति ते माता कुररीव गतप्रजा
पुत्रस्नेहाकुला दीना विवत्सा गौरिवातुरा
प्रभाष्यैवं ददौ विद्यां प्रद्युम्नाय महात्मने
मायावती महामायां सर्वमायाविनाशिनीम्‌
स च शम्बरमभ्येत्य संयुगाय समाह्वयत्‌
अविषह्यैस्तमाक्षेपैः क्षिपन्सञ्जनयन्कलिम्‌
सोऽधिक्षिप्तो दुर्वाचोभिः पदाहत इवोरगः
निश्चक्राम गदापाणिरमर्षात्ताम्रलोचनः
गदामाविध्य तरसा प्रद्युम्नाय महात्मने
प्रक्षिप्य व्यनदन्नादं वज्रनिष्पेषनिष्ठुरम्‌
तामापतन्तीं भगवान्प्रद्युम्नो गदया गदाम्‌
अपास्य शत्रवे क्रुद्धः प्राहिणोत्स्वगदां नृप
स च मायां समाश्रित्य दैतेयीं मयदर्शितम्‌
मुमुचेऽस्त्रमयं वर्षं कार्ष्णौ वैहायसोऽसुरः
बाध्यमानोऽस्त्रवर्षेण रौक्मिणेयो महारथः
सत्त्वात्मिकां महाविद्यां सर्वमायोपमर्दिनीम्‌
ततो गौह्यकगान्धर्व पैशाचोरगराक्षसीः
प्रायुङ्क्त शतशो दैत्यः कार्ष्णिर्व्यधमयत्स ताः
निशातमसिमुद्यम्य सकिरीटं सकुण्डलम्‌
शम्बरस्य शिरः कायात्ताम्रश्मश्र्वोजसाहरत्‌
आकीर्यमाणो दिविजैः स्तुवद्भिः कुसुमोत्करैः
भार्ययाम्बरचारिण्या पुरं नीतो विहायसा
अन्तःपुरवरं राजन्ललनाशतसङ्कुलम्‌
विवेश पत्न्या गगनाद्विद्युतेव बलाहकः
तं दृष्ट्वा जलदश्यामं पीतकौशेयवाससम्‌
प्रलम्बबाहुं ताम्राक्षं सुस्मितं रुचिराननम्‌
स्वलङ्कृतमुखाम्भोजं नीलवक्रालकालिभिः
कृष्णं मत्वा स्त्रियो ह्रीता निलिल्युस्तत्र तत्र ह
अवधार्य शनैरीषद्वैलक्षण्येन योषितः
उपजग्मुः प्रमुदिताः सस्त्री रत्नं सुविस्मिताः
अथ तत्रासितापाङ्गी वैदर्भी वल्गुभाषिणी
अस्मरत्स्वसुतं नष्टं स्नेहस्नुतपयोधरा
को न्वयम्नरवैदूर्यः कस्य वा कमलेक्षणः
धृतः कया वा जठरे केयं लब्धा त्वनेन वा
मम चाप्यात्मजो नष्टो नीतो यः सूतिकागृहात्‌
एतत्तुल्यवयोरूपो यदि जीवति कुत्रचित्‌
कथं त्वनेन सम्प्राप्तं सारूप्यं शार्ङ्गधन्वनः
आकृत्यावयवैर्गत्या स्वरहासावलोकनैः
स एव वा भवेन्नूनं यो मे गर्भे धृतोऽर्भकः
अमुष्मिन्प्रीतिरधिका वामः स्फुरति मे भुजः
एवं मीमांसमणायां वैदर्भ्यां देवकीसुतः
देवक्यानकदुन्दुभ्यामुत्तमःश्लोक आगमत्‌
विज्ञातार्थोऽपि भगवांस्तूष्णीमास जनार्दनः
नारदोऽकथयत्सर्वं शम्बराहरणादिकम्‌
तच्छ्रुत्वा महदाश्चर्यं कृष्णान्तःपुरयोषितः
अभ्यनन्दन्बहूनब्दान्नष्टं मृतमिवागतम्‌
देवकी वसुदेवश्च कृष्णरामौ तथा स्त्रियः
दम्पती तौ परिष्वज्य रुक्मिणी च ययुर्मुदम्‌
नष्टं प्रद्युम्नमायातमाकर्ण्य द्वारकौकसः
अहो मृत इवायातो बालो दिष्ट्येति हाब्रुवन्‌
यं वै मुहुः पितृसरूपनिजेशभावास्‌
तन्मातरो यदभजन्रहरूढभावाः
चित्रं न तत्खलु रमास्पदबिम्बबिम्बे
कामे स्मरेऽक्षविषये किमुतान्यनार्यः
श्रीशुक उवाच
सत्राजितः स्वतनयां कृष्णाय कृतकिल्बिषः
स्यमन्तकेन मणिना स्वयमुद्यम्य दत्तवान्‌
श्रीराजोवाच
सत्राजितः किमकरोद्ब्रह्मन्कृष्णस्य किल्बिषः
स्यमन्तकः कुतस्तस्य कस्माद्दत्ता सुता हरेः
श्रीशुक उवाच
आसीत्सत्राजितः सूर्यो भक्तस्य परमः सखा
प्रीतस्तस्मै मणिं प्रादात्स च तुष्टः स्यमन्तकम्‌
स तं बिभ्रन्मणिं कण्ठे भ्राजमानो यथा रविः
प्रविष्टो द्वारकां राजन्तेजसा नोपलक्षितः
तं विलोक्य जना दूरात्तेजसा मुष्टदृष्टयः
दीव्यतेऽक्षैर्भगवते शशंसुः सूर्यशङ्किताः
नारायण नमस्तेऽस्तु शङ्खचक्रगदाधर
दामोदरारविन्दाक्ष गोविन्द यदुनन्दन
एष आयाति सविता त्वां दिदृक्षुर्जगत्पते
मुष्णन्गभस्तिचक्रेण नृणां चक्षूंषि तिग्मगुः
नन्वन्विच्छन्ति ते मार्गं त्रीलोक्यां विबुधर्षभाः
ज्ञात्वाद्य गूढं यदुषु द्रष्टुं त्वां यात्यजः प्रभो
श्रीशुक उवाच
निशम्य बालवचनं प्रहस्याम्बुजलोचनः
प्राह नासौ रविर्देवः सत्राजिन्मणिना ज्वलन्‌
सत्राजित्स्वगृहं श्रीमत्कृतकौतुकमङ्गलम्‌
प्रविश्य देवसदने मणिं विप्रैर्न्यवेशयत्‌
दिने दिने स्वर्णभारानष्टौ स सृजति प्रभो
दुर्भिक्षमार्यरिष्टानि सर्पाधिव्याधयोऽशुभाः
न सन्ति मायिनस्तत्र यत्रास्तेऽभ्यर्चितो मणिः
स याचितो मणिं क्वापि यदुराजाय शौरिणा
नैवार्थकामुकः प्रादाद्याच्ञाभङ्गमतर्कयन्‌
तमेकदा मणिं कण्ठे प्रतिमुच्य महाप्रभम्‌
प्रसेनो हयमारुह्य मृगायां व्यचरद्वने
प्रसेनं सहयं हत्वा मणिमाच्छिद्य केशरी
गिरिं विशन्जाम्बवता निहतो मणिमिच्छता
सोऽपि चक्रे कुमारस्य मणिं क्रीडनकं बिले
अपश्यन्भ्रातरं भ्राता सत्राजित्पर्यतप्यत
प्रायः कृष्णेन निहतो मणिग्रीवो वनं गतः
भ्राता ममेति तच्छ्रुत्वा कर्णे कर्णेऽजपन्जनाः
भगवांस्तदुपश्रुत्य दुर्यशो लिप्तमात्मनि
मार्ष्टुं प्रसेनपदवीमन्वपद्यत नागरैः
हतं प्रसेनं अश्वं च वीक्ष्य केशरिणा वने
तं चाद्रिपृष्ठे निहतमृक्षेण ददृशुर्जनाः
ऋक्षराजबिलं भीममन्धेन तमसावृतम्‌
एको विवेश भगवानवस्थाप्य बहिः प्रजाः
तत्र दृष्ट्वा मणिप्रेष्ठं बालक्रीडनकं कृतम्‌
हर्तुं कृतमतिस्तस्मिन्नवतस्थेऽर्भकान्तिके
तमपूर्वं नरं दृष्ट्वा धात्री चुक्रोश भीतवत्‌
तच्छ्रुत्वाभ्यद्रवत्क्रुद्धो जाम्बवान्बलिनां वरः
स वै भगवता तेन युयुधे स्वामीनात्मनः
पुरुषम्प्राकृतं मत्वा कुपितो नानुभाववित्‌
द्वन्द्वयुद्धं सुतुमुलमुभयोर्विजिगीषतोः
आयुधाश्मद्रुमैर्दोर्भिः क्रव्यार्थे श्येनयोरिव
आसीत्तदष्टाविम्शाहमितरेतरमुष्टिभिः
वज्रनिष्पेषपरुषैरविश्रममहर्निशम्‌
कृष्णमुष्टिविनिष्पात निष्पिष्टाङ्गोरु बन्धनः
क्षीणसत्त्वः स्विन्नगात्रस्तमाहातीव विस्मितः
जाने त्वां सऋवभूतानां प्राण ओजः सहो बलम्‌
विष्णुं पुराणपुरुषं प्रभविष्णुमधीश्वरम्‌
त्वं हि विश्वसृजाम्स्रष्टा सृष्टानामपि यच्च सत्‌
कालः कलयतामीशः पर आत्मा तथात्मनाम्‌
यस्येषदुत्कलितरोषकटाक्षमोक्षैर्‌
वर्त्मादिशत्क्षुभितनक्रतिमिङ्गलोऽब्धिः
सेतुः कृतः स्वयश उज्ज्वलिता च लङ्का
रक्षःशिरांसि भुवि पेतुरिषुक्षतानि
इति विज्ञातवीज्ञानमृक्षराजानमच्युतः
व्याजहार महाराज भगवान्देवकीसुतः
अभिमृश्यारविन्दाक्षः पाणिना शंकरेण तम्‌
कृपया परया भक्तं मेघगम्भीरया गिरा
मणिहेतोरिह प्राप्ता वयमृक्षपते बिलम्‌
मिथ्याभिशापं प्रमृजन्नात्मनो मणिनामुना
इत्युक्तः स्वां दुहितरं कन्यां जाम्बवतीं मुदा
अर्हणार्थम्स मणिना कृष्णायोपजहार ह
अदृष्ट्वा निर्गमं शौरेः प्रविष्टस्य बिलं जनाः
प्रतीक्ष्य द्वादशाहानि दुःखिताः स्वपुरं ययुः
निशम्य देवकी देवी रक्मिण्यानकदुन्दुभिः
सुहृदो ज्ञातयोऽशोचन्बिलात्कृष्णमनिर्गतम्‌
सत्राजितं शपन्तस्ते दुःखिता द्वारकौकसः
उपतस्थुश्चन्द्रभागां दुर्गां कृष्णोपलब्धये
तेषां तु देव्युपस्थानात्प्रत्यादिष्टाशिषा स च
प्रादुर्बभूव सिद्धार्थः सदारो हर्षयन्हरिः
उपलभ्य हृषीकेशं मृतं पुनरिवागतम्‌
सह पत्न्या मणिग्रीवं सर्वे जातमहोत्सवाः
सत्राजितं समाहूय सभायां राजसन्निधौ
प्राप्तिं चाख्याय भगवान्मणिं तस्मै न्यवेदयत्‌
स चातिव्रीडितो रत्नं गृहीत्वावाङ्मुखस्ततः
अनुतप्यमानो भवनमगमत्स्वेन पाप्मना
सोऽनुध्यायंस्तदेवाघं बलवद्विग्रहाकुलः
कथं मृजाम्यात्मरजः प्रसीदेद्वाच्युतः कथम्‌
किम्कृत्वा साधु मह्यं स्यान्न शपेद्वा जनो यथा
अदीर्घदर्शनं क्षुद्रं मूढं द्रविणलोलुपम्‌
दास्ये दुहितरं तस्मै स्त्रीरत्नं रत्नमेव च
उपायोऽयं समीचीनस्तस्य शान्तिर्न चान्यथा
एवं व्यवसितो बुद्ध्या सत्राजित्स्वसुतां शुभाम्‌
मणिं च स्वयमुद्यम्य कृष्णायोपजहार ह
तां सत्यभामां भगवानुपयेमे यथाविधि
बहुभिर्याचितां शील रूपौदार्यगुणान्विताम्‌
भगवानाह न मणिं प्रतीच्छामो वयं नृप
तवास्तां देवभक्तस्य वयं च फलभागिनः
श्रीबादरायणिरुवाच
विज्ञातार्थोऽपि गोविन्दो दग्धानाकर्ण्य पाण्डवान्‌
कुन्तीं च कुल्यकरणे सहरामो ययौ कुरून्‌
भीष्मं कृपं स विदुरं गान्धारीं द्रोणमेव च
तुल्यदुःखौ च सङ्गम्य हा कष्टमिति होचतुः
लब्ध्वैतदन्तरं राजन्शतधन्वानमूचतुः
अक्रूरकृतवर्माणौ मनिः कस्मान्न गृह्यते
योऽस्मभ्यं सम्प्रतिश्रुत्य कन्यारत्नं विगर्ह्य नः
कृष्णायादान्न सत्राजित्कस्माद्भ्रातरमन्वियात्‌
एवं भिन्नमतिस्ताभ्यां सत्राजितमसत्तमः
शयानमवधील्लोभात्स पापः क्षीण जीवितः
स्त्रीणां विक्रोशमानानां क्रन्दन्तीनामनाथवत्‌
हत्वा पशून्सौनिकवन्मणिमादाय जग्मिवान्‌
सत्यभामा च पितरं हतं वीक्ष्य शुचार्पिता
व्यलपत्तात तातेति हा हतास्मीति मुह्यती
तैलद्रोण्यां मृतं प्रास्य जगाम गजसाह्वयम्‌
कृष्णाय विदितार्थाय तप्ताचख्यौ पितुर्वधम्‌
तदाकर्ण्येश्वरौ राजन्ननुसृत्य नृलोकताम्‌
अहो नः परमं कष्टमित्यस्राक्षौ विलेपतुः
आगत्य भगवांस्तस्मात्सभार्यः साग्रजः पुरम्‌
शतधन्वानमारेभे हन्तुं हर्तुं मणिं ततः
सोऽपि कृतोद्यमं ज्ञात्वा भीतः प्राणपरीप्सया
साहाय्ये कृतवर्माणमयाचत स चाब्रवीत्‌
नाहमीस्वरयोः कुर्यां हेलनं रामकृष्णयोः
को नु क्षेमाय कल्पेत तयोर्वृजिनमाचरन्‌
कंसः सहानुगोऽपीतो यद्द्वेषात्त्याजितः श्रिया
जरासन्धः सप्तदश संयुगाद्विरथो गतः
प्रत्याख्यातः स चाक्रूरं पार्ष्णिग्राहमयाचत
सोऽप्याह को विरुध्येत विद्वानीश्वरयोर्बलम्‌
य इदं लीलया विश्वं सृजत्यवति हन्ति च
चेष्टां विश्वसृजो यस्य न विदुर्मोहिताजया
यः सप्तहायनः शैलमुत्पाट्यैकेन पाणिना
दधार लीलया बाल उच्छिलीन्ध्रमिवार्भकः
नमस्तस्मै भगवते कृष्णायाद्भुतकर्मणे
अनन्तायादिभूताय कूटस्थायात्मने नमः
प्रत्याख्यातः स तेनापि शतधन्वा महामणिम्‌
तस्मिन्न्यस्याश्वमारुह्य शतयोजनगं ययौ
गरुडध्वजमारुह्य रथं रामजनार्दनौ
अन्वयातां महावेगैरश्वै राजन्गुरुद्रुहम्‌
मिथिलायामुपवने विसृज्य पतितं हयम्‌
पद्भ्यामधावत्सन्त्रस्तः कृष्णोऽप्यन्वद्रवद्रुषा
पदातेर्भगवांस्तस्य पदातिस्तिग्मनेमिना
चक्रेण शिर उत्कृत्य वाससोर्व्यचिनोन्मणिम्‌
अलब्धमणिरागत्य कृष्ण आहाग्रजान्तिकम्‌
वृथा हतः शतधनुर्मणिस्तत्र न विद्यते
तत आह बलो नूनं स मणिः शतधन्वना
कस्मिंश्चित्पुरुषे न्यस्तस्तमन्वेष पुरं व्रज
अहं वैदेहमिच्छामि द्रष्टुं प्रियतमं मम
इत्युक्त्वा मिथिलां राजन्विवेश यदनन्दनः
तं दृष्ट्वा सहसोत्थाय मैथिलः प्रीतमानसः
अर्हयां आस विधिवदर्हणीयं समर्हणैः
उवास तस्यां कतिचिन्मिथिलायां समा विभुः
मानितः प्रीतियुक्तेन जनकेन महात्मना
ततोऽशिक्षद्गदां काले धार्तराष्ट्रः सुयोधनः
केशवो द्वारकामेत्य निधनं शतधन्वनः
अप्राप्तिं च मणेः प्राह प्रियायाः प्रियकृद्विभुः
ततः स कारयामास क्रिया बन्धोर्हतस्य वै
साकं सुहृद्भिर्भगवान्या याः स्युः साम्परायिकीः
अक्रूरः कृतवर्मा च श्रुत्वा शतधनोर्वधम्‌
व्यूषतुर्भयवित्रस्तौ द्वारकायाः प्रयोजकौ
अक्रूरे प्रोषितेऽरिष्टान्यासन्वै द्वारकौकसाम्‌
शारीरा मानसास्तापा मुहुर्दैविकभौतिकाः
इत्यङ्गोपदिशन्त्येके विस्मृत्य प्रागुदाहृतम्‌
मुनिवासनिवासे किं घटेतारिष्टदर्शनम्‌
देवेऽवर्षति काशीशः श्वफल्कायागताय वै
स्वसुतां गाण्दिनीं प्रादात्ततोऽवर्षत्स्म काशिषु
तत्सुतस्तत्प्रभावोऽसावक्रूरो यत्र यत्र ह
देवोऽभिवर्षते तत्र नोपतापा न मारीकाः
इति वृद्धवचः श्रुत्वा नैतावदिह कारणम्‌
इति मत्वा समानाय्य प्राहाक्रूरं जनार्दनः
पूजयित्वाभिभाष्यैनं कथयित्वा प्रियाः कथाः
विज्ञताखिलचित्त ज्ञः स्मयमान उवाच ह
ननु दानपते न्यस्तस्त्वय्यास्ते शतधन्वना
स्यमन्तको मनिः श्रीमान्विदितः पूर्वमेव नः
सत्राजितोऽनपत्यत्वाद्गृह्णीयुर्दुहितुः सुताः
दायं निनीयापः पिण्डान्विमुच्यर्णं च शेषितम्‌
तथापि दुर्धरस्त्वन्यैस्त्वय्यास्तां सुव्रते मणिः
किन्तु मामग्रजः सम्यङ्न प्रत्येति मणिं प्रति
दर्शयस्व महाभाग बन्धूनां शान्तिमावह
अव्युच्छिन्ना मखास्तेऽद्य वर्तन्ते रुक्मवेदयः
एवं सामभिरालब्धः श्वफल्कतनयो मणिम्‌
आदाय वाससाच्छन्नः ददौ सूर्यसमप्रभम्‌
स्यमन्तकं दर्शयित्वा ज्ञातिभ्यो रज आत्मनः
विमृज्य मणिना भूयस्तस्मै प्रत्यर्पयत्प्रभुः
यस्त्वेतद्भगवत ईश्वरस्य विष्णोर्‌
वीर्याढ्यं वृजिनहरं सुमङ्गलं च
आख्यानं पठति शृणोत्यनुस्मरेद्वा
दुष्कीर्तिं दुरितमपोह्य याति शान्तिम्‌
श्रीशुक उवाच
एकदा पाण्डवान्द्रष्टुं प्रतीतान्पुरुषोत्तमः
इन्द्रप्रस्थं गतः शृईमान्युयुधानादिभिर्वृतः
दृष्ट्वा तमागतं पार्था मुकुन्दमखिलेश्वरम्‌
उत्तस्थुर्युगपद्वीराः प्राणा मुख्यमिवागतम्‌
परिष्वज्याच्युतं वीरा अङ्गसङ्गहतैनसः
सानुरागस्मितं वक्त्रं वीक्ष्य तस्य मुदं ययुः
युधिष्ठिरस्य भीमस्य कृत्वा पादाभिवन्दनम्‌
फाल्गुनं परिरभ्याथ यमाभ्यां चाभिवन्दितः
परमासन आसीनं कृष्णा कृष्णमनिन्दिता
नवोढा व्रीडिता किञ्चिच्छनैरेत्याभ्यवन्दत
तथैव सात्यकिः पार्थैः पूजितश्चाभिवन्दितः
निषसादासनेऽन्ये च पूजिताः पर्युपासत
पृथाम्समागत्य कृताभिवादनस्तयातिहार्दार्द्रदृशाभिरम्भितः
आपृष्टवांस्तां कुशलं सहस्नुषां पितृष्वसारम्परिपृष्टबान्धवः
तमाह प्रेमवैक्लव्य रुद्धकण्ठाश्रुलोचना
स्मरन्ती तान्बहून्क्लेशान्क्लेशापायात्मदर्शनम्‌
तदैव कुशलं नोऽभूत्सनाथास्ते कृता वयम्‌
ज्ञतीन्नः स्मरता कृष्ण भ्राता मे प्रेषितस्त्वया
न तेऽस्ति स्वपरभ्रान्तिर्विश्वस्य सुहृदात्मनः
तथापि स्मरतां शश्वत्क्लेशान्हंसि हृदि स्थितः
युधिष्ठिर उवाच
किं न आचरितं श्रेयो न वेदाहमधीश्वर
योगेश्वराणां दुर्दर्शो यन्नो दृष्टः कुमेधसाम्‌
इति वै वार्षिकान्मासान्राज्ञा सोऽभ्यर्थितः सुखम्‌
जनयन्नयनानन्दमिन्द्रप्रस्थौकसां विभुः
एकदा रथमारुह्य विजयो वानरध्वजम्‌
गाण्डीवं धनुरादाय तूणौ चाक्षयसायकौ
साकं कृष्णेन सन्नद्धो विहर्तुं विपिनं महत्‌
बहुव्यालमृगाकीर्णं प्राविशत्परवीरहा
तत्राविध्यच्छरैर्व्याघ्रान्शूकरान्महिषान्रुरून्‌
शरभान्गवयान्खड्गान्हरिणान्शशशल्लकान्‌
तान्निन्युः किङ्करा राज्ञे मेध्यान्पर्वण्युपागते
तृट्परीतः परिश्रान्तो बिभत्सुर्यमुनामगात्‌
तत्रोपस्पृश्य विशदं पीत्वा वारि महारथौ
कृष्णौ ददृशतुः कन्यां चरन्तीं चारुदर्शनाम्‌
तामासाद्य वरारोहां सुद्विजां रुचिराननाम्‌
पप्रच्छ प्रेषितः सख्या फाल्गुनः प्रमदोत्तमाम्‌
का त्वं कस्यासि सुश्रोणि कुतो वा किं चिकीर्षसि
मन्ये त्वां पतिमिच्छन्तीं सर्वं कथय शोभने
श्रीकालिन्द्युवाच
अहं देवस्य सवितुर्दुहिता पतिमिच्छती
विष्णुं वरेण्यं वरदं तपः परममास्थितः
नान्यं पतिं वृणे वीर तमृते श्रीनिकेतनम्‌
तुष्यतां मे स भगवान्मुकुन्दोऽनाथसंश्रयः
कालिन्दीति समाख्याता वसामि यमुनाजले
निर्मिते भवने पित्रा यावदच्युतदर्शनम्‌
तथावदद्गुडाकेशो वासुदेवाय सोऽपि ताम्‌
रथमारोप्य तद्विद्वान्धर्मराजमुपागमत्‌
यदैव कृष्णः सन्दिष्टः पार्थानां परमाद्बुतम्‌
कारयामास नगरं विचित्रं विश्वकर्मणा
भगवांस्तत्र निवसन्स्वानां प्रियचिकीर्षया
अग्नये खाण्डवं दातुमर्जुनस्यास सारथिः
सोऽग्निस्तुष्टो धनुरदाद्धयान्श्वेतान्रथं नृप
अर्जुनायाक्षयौ तूणौ वर्म चाभेद्यमस्त्रिभिः
मयश्च मोचितो वह्नेः सभां सख्य उपाहरत्‌
यस्मिन्दुर्योधनस्यासीज्जलस्थलदृशिभ्रमः
स तेन समनुज्ञातः सुहृद्भिश्चानुमोदितः
आययौ द्वारकां भूयः सात्यकिप्रमखैर्वृतः
अथोपयेमे कालिन्दीं सुपुण्यर्त्वृक्ष ऊर्जिते
वितन्वन्परमानन्दं स्वानां परममङ्गलः
विन्द्यानुविन्द्यावावन्त्यौ दुर्योधनवशानुगौ
स्वयंवरे स्वभगिनीं कृष्णे सक्तां न्यषेधताम्‌
राजाधिदेव्यास्तनयां मित्रविन्दां पितृष्वसुः
प्रसह्य हृतवान्कृष्णो राजन्राज्ञां प्रपश्यताम्‌
नग्नजिन्नाम कौशल्य आसीद्राजातिधार्मिकः
तस्य सत्याभवत्कन्या देवी नाग्नजिती नृप
न तां शेकुर्नृपा वोढुमजित्वा सप्तगोवृषान्‌
तीक्ष्णशृङ्गान्सुदुर्धर्षान्वीर्यगन्धासहान्खलान्‌
तां श्रुत्वा वृषजिल्लभ्यां भगवान्सात्वतां पतिः
जगाम कौशल्यपुरं सैन्येन महता वृतः
स कोशलपतिः प्रीतः प्रत्युत्थानासनादिभिः
अर्हणेनापि गुरुणा पूजयन्प्रतिनन्दितः
वरं विलोक्याभिमतं समागतं नरेन्द्रकन्या चकमे रमापतिम्‌
भूयादयं मे पतिराशिषोऽनलः करोतु सत्या यदि मे धृतो व्रतः
यत्पादपङ्कजरजः शिरसा बिभर्ति
शृईरब्यजः सगिरिशः सह लोकपालैः
लीलातनुः स्वकृतसेतुपरीप्सया यः
कालेऽदधत्स भगवान्मम केन तुष्येत्‌
अर्चितं पुनरित्याह नारायण जगत्पते
आत्मानन्देन पूर्णस्य करवाणि किमल्पकः
श्रीशुक उवाच
तमाह भगवान्हृष्टः कृतासनपरिग्रहः
मेघगम्भीरया वाचा सस्मितं कुरुनन्दन
श्रीभगवानुवाच
नरेन्द्र याच्ञा कविभिर्विगर्हिता राजन्यबन्धोर्निजधर्मवर्तिनः
तथापि याचे तव सौहृदेच्छया कन्यां त्वदीयां न हि शुल्कदा वयम्‌
श्रीराजोवाच
कोऽन्यस्तेऽभ्यधिको नाथ कन्यावर इहेप्सितः
गुणैकधाम्नो यस्याङ्गे श्रीर्वसत्यनपायिनी
किन्त्वस्माभिः कृतः पूर्वं समयः सात्वतर्षभ
पुंसां वीर्यपरीक्षार्थं कन्यावरपरीप्सया
सप्तैते गोवृषा वीर दुर्दान्ता दुरवग्रहाः
एतैर्भग्नाः सुबहवो भिन्नगात्रा नृपात्मजाः
यदिमे निगृहीताः स्युस्त्वयैव यदुनन्दन
वरो भवानभिमतो दुहितुर्मे श्रियःपते
एवं समयमाकर्ण्य बद्ध्वा परिकरं प्रभुः
आत्मानं सप्तधा कृत्वा न्यगृह्णाल्लीलयैव तान्‌
बद्ध्वा तान्दामभिः शौरिर्भग्नदर्पान्हतौजसः
व्यकर्सल्लीलया बद्धान्बालो दारुमयान्यथा
ततः प्रीतः सुतां राजा ददौ कृष्णाय विस्मितः
तां प्रत्यगृह्णाद्भगवान्विधिवत्सदृशीं प्रभुः
राजपत्न्यश्च दुहितुः कृष्णं लब्ध्वा प्रियं पतिम्‌
लेभिरे परमानन्दं जातश्च परमोत्सवः
शङ्खभेर्यानका नेदुर्गीतवाद्यद्विजाशिषः
नरा नार्यः प्रमुदिताः सुवासःस्रगलङ्कृताः
दशधेनुसहस्राणि पारिबर्हमदाद्विभुः
युवतीनां त्रिसाहस्रं निष्कग्रीवसुवाससम्‌
नवनागसहस्राणि नागाच्छतगुणान्रथान्‌
रथाच्छतगुणानश्वानश्वाच्छतगुणान्नरान्‌
दम्पती रथमारोप्य महत्या सेनया वृतौ
स्नेहप्रक्लिन्नहृदयो यापयामास कोशलः
श्रुत्वैतद्रुरुधुर्भूपा नयन्तं पथि कन्यकाम्‌
भग्नवीर्याः सुदुर्मर्षा यदुभिर्गोवृषैः पुरा
तानस्यतः शरव्रातान्बन्धुप्रियकृदर्जुनः
गाण्डीवी कालयामास सिंहः क्षुद्रमृगानिव
पारिबर्हमुपागृह्य द्वारकामेत्य सत्यया
रेमे यदूनामृषभो भगवान्देवकीसुतः
श्रुतकीर्तेः सुतां भद्रां उपयेमे पितृष्वसुः
कैकेयीं भ्रातृभिर्दत्तां कृष्णः सन्तर्दनादिभिः
सुतां च मद्राधिपतेर्लक्ष्मणां लक्षणैर्यताम्‌
स्वयंवरे जहारैकः स सुपर्णः सुधामिव
अन्याश्चैवंविधा भार्याः कृष्णस्यासन्सहस्रशः
भौमं हत्वा तन्निरोधादाहृताश्चारुदर्शनाः
श्रीराजोवाच यथा हतो भगवता भौमो येने च ताः स्त्रियः
निरुद्धा एतदाचक्ष्व विक्रमं शार्ङ्गधन्वनः
श्रीशुक उवाच
इन्द्रेण हृतछत्रेण हृतकुण्डलबन्धुना
हृतामराद्रिस्थानेन ज्ञापितो भौमचेष्टितम्‌
सभार्यो गरुडारूढः प्राग्ज्योतिषपुरं ययौ
गिरिदुर्गैः शस्त्रदुर्गैर्जलाग्न्यनिलदुर्गमम्‌
मुरपाशायुतैर्घोरैर्दृढैः सर्वत आवृतम्‌
गदया निर्बिभेदाद्रीन्शस्त्रदुर्गाणि सायकैः
चक्रेणाग्निं जलं वायुं मुरपाशांस्तथासिना
शङ्खनादेन यन्त्राणि हृदयानि मनस्विनाम्‌
प्राकारं गदया गुर्व्या निर्बिभेद गदाधरः
पाञ्चजन्यध्वनिं श्रुत्वा युगान्तशनिभीषणम्‌
मुरः शयान उत्तस्थौ दैत्यः पञ्चशिरा जलात्‌
त्रिशूलमुद्यम्य सुदुर्निरीक्षणो युगान्तसूर्यानलरोचिरुल्बणः
ग्रसंस्त्रिलोकीमिव पञ्चभिर्मुखैरभ्यद्रवत्तार्क्ष्यसुतं यथोरगः
आविध्य शूलं तरसा गरुत्मते निरस्य वक्त्रैर्व्यनदत्स पञ्चभिः
स रोदसी सर्वदिशोऽम्बरं महानापूरयन्नण्डकटाहमावृणोत्‌
तदापतद्वै त्रिशिखं गरुत्मते हरिः शराभ्यामभिनत्त्रिधोजसा
मुखेषु तं चापि शरैरताडयत्तस्मै गदां सोऽपि रुषा व्यमुञ्चत
तामापतन्तीं गदया गदां मृधे गदाग्रजो निर्बिभिदे सहस्रधा
उद्यम्य बाहूनभिधावतोऽजितः शिरांसि चक्रेण जहार लीलया
व्यसुः पपाताम्भसि कृत्तशीर्षो निकृत्तशृङ्गोऽद्रिरिवेन्द्रतेजसा
तस्यात्मजाः सप्त पितुर्वधातुराः प्रतिक्रियामर्षजुषः समुद्यताः
ताम्रोऽन्तरिक्षः श्रवणो विभावसुर्‌
वसुर्नभस्वानरुणश्च सप्तमः
पीठं पुरस्कृत्य चमूपतिं मृधे
भौमप्रयुक्ता निरगन्धृतायुधाः
प्रायुञ्जतासाद्य शरानसीन्गदाः शक्त्यृष्टिशूलान्यजिते रुषोल्बणाः
तच्छस्त्रकूटं भगवान्स्वमार्गणैरमोघवीर्यस्तिलशश्चकर्त ह
तान्पीठमुख्याननयद्यमक्षयं
निकृत्तशीर्षोरुभुजाङ्घ्रिवर्मणः
स्वानीकपानच्युतचक्रसायकैस्‌
तथा निरस्तान्नरको धरासुतः
निरीक्ष्य दुर्मर्षण आस्रवन्मदैर्‌
गजैः पयोधिप्रभवैर्निराक्रमात्‌
दृष्ट्वा सभार्यं गरुडोपरि स्थितं
सूर्योपरिष्टात्सतडिद्घनं यथा
कृष्णं स तस्मै व्यसृजच्छतघ्नीं
योधाश्च सर्वे युगपच्च विव्यधुः
तद्भौमसैन्यं भगवान्गदाग्रजो
विचित्रवाजैर्निशितैः शिलीमुखैः
निकृत्तबाहूरुशिरोध्रविग्रहं
चकार तर्ह्येव हताश्वकुञ्जरम्‌
यानि योधैः प्रयुक्तानि शस्त्रास्त्राणि कुरूद्वह
हरिस्तान्यच्छिनत्तीक्ष्णैः शरैरेकैकशस्त्रीभिः
उह्यमानः सुपर्णेन पक्षाभ्यां निघ्नता गजान्‌
गुरुत्मता हन्यमानास्तुण्डपक्षनखेर्गजाः
पुरमेवाविशन्नार्ता नरको युध्ययुध्यत
दृष्ट्वा विद्रावितं सैन्यं गरुडेनार्दितं स्वकं
तं भौमः प्राहरच्छक्त्या वज्रः प्रतिहतो यतः
नाकम्पत तया विद्धो मालाहत इव द्विपः
शूलं भौमोऽच्युतं हन्तुमाददे वितथोद्यमः
तद्विसर्गात्पूर्वमेव नरकस्य शिरो हरिः
अपाहरद्गजस्थस्य चक्रेण क्षुरनेमिना
सकुण्डलं चारुकिरीटभूषणं बभौ पृथिव्यां पतितम्समुज्ज्वलम्‌
ह हेति साध्वित्यृषयः सुरेश्वरा माल्यैर्मुकुन्दं विकिरन्त ईदिरे
ततश्च भूः कृष्णमुपेत्य कुण्डले
प्रतप्तजाम्बूनदरत्नभास्वरे
सवैजयन्त्या वनमालयार्पयत्‌
प्राचेतसं छत्रमथो महामणिम्‌
अस्तौषीदथ विश्वेशं देवी देववरार्चितम्‌
प्राञ्जलिः प्रणता राजन्भक्तिप्रवणया धिया
भूमिरुवाच
नमस्ते देवदेवेश शङ्खचक्रगदाधर
भक्तेच्छोपात्तरूपाय परमात्मन्नमोऽस्तु ते
नमः पङ्कजनाभाय नमः पङ्कजमालिने
नमः पङ्कजनेत्राय नमस्तेपङ्कजाङ्घ्रये
नमो भगवते तुभ्यं वासुदेवाय विष्णवे
पुरुषायादिबीजाय पूर्णबोधाय ते नमः
अजाय जनयित्रेऽस्य ब्रह्मणेऽनन्तशक्तये
परावरात्मन्भूतात्मन्परमात्मन्नमोऽस्तु ते
त्वं वै सिसृक्षुरज उत्कटं प्रभो
तमो निरोधाय बिभर्ष्यसंवृतः
स्थानाय सत्त्वं जगतो जगत्पते
कालः प्रधानं पुरुषो भवान्परः
अहं पयो ज्योतिरथानिलो नभो मात्राणि देवा मन इन्द्रियाणि
कर्ता महानित्यखिलं चराचरं त्वय्यद्वितीये भगवनयं भ्रमः
तस्यात्मजोऽयं तव पादपङ्कजं भीतः प्रपन्नार्तिहरोपसादितः
तत्पालयैनं कुरु हस्तपङ्कजं शिरस्यमुष्याखिलकल्मषापहम्‌
श्रीशुक उवाच
इति भूम्यर्थितो वाग्भिर्भगवान्भक्तिनम्रया
दत्त्वाभयं भौमगृहम्प्राविशत्सकलर्द्धिमत्‌
तत्र राजन्यकन्यानां षट्सहस्राधिकायुतम्‌
भौमाहृतानां विक्रम्य राजभ्यो ददृशे हरिः
तम्प्रविष्टं स्त्रियो वीक्ष्य नरवर्यं विमोहिताः
मनसा वव्रिरेऽभीष्टं पतिं दैवोपसादितम्‌
भूयात्पतिरयं मह्यं धाता तदनुमोदताम्‌
इति सर्वाः पृथक्कृष्णे भावेन हृदयं दधुः
ताः प्राहिणोद्द्वारवतीं सुमृष्टविरजोऽम्बराः
नरयानैर्महाकोशान्रथाश्वान्द्रविणं महात्‌
ऐरावतकुलेभांश्च चतुर्दन्तांस्तरस्विनः
पाण्डुरांश्च चतुःषष्टिं प्रेरयामास केशवः
गत्वा सुरेन्द्रभवनं दत्त्वादित्यै च कुण्डले
पूजितस्त्रिदशेन्द्रेण महेन्द्र्याण्या च सप्रियः
चोदितो भार्ययोत्पाट्य पारीजातं गरुत्मति
आरोप्य सेन्द्रान्विबुधान्निर्जित्योपानयत्पुरम्‌
स्थापितः सत्यभामाया गृहोद्यानोपशोभनः
अन्वगुर्भ्रमराः स्वर्गात्तद्गन्धासवलम्पटाः
ययाच आनम्य किरीटकोटिभिः पादौ स्पृशन्नच्युतमर्थसाधनम्‌
सिद्धार्थ एतेन विगृह्यते महानहो सुराणां च तमो धिगाढ्यताम्‌
अथो मुहूर्त एकस्मिन्नानागारेषु ताः स्त्रियः
यथोपयेमे भगवान्तावद्रूपधरोऽव्ययः
गृहेषु तासामनपाय्यतर्ककृन्निरस्तसाम्यातिशयेष्ववस्थितः
रेमे रमाभिर्निजकामसम्प्लुतो यथेतरो गार्हकमेधिकांश्चरन्‌
इत्थं रमापतिमवाप्य पतिं स्त्रियस्ता
ब्रह्मादयोऽपि न विदुः पदवीं यदीयाम्‌
भेजुर्मुदाविरतमेधितयानुराग
हासावलोकनवसङ्गमजल्पलज्जाः
प्रत्युद्गमासनवरार्हणपदशौच
ताम्बूलविश्रमणवीजनगन्धमाल्यैः
केशप्रसारशयनस्नपनोपहार्यैः
दासीशता अपि विभोर्विदधुः स्म दास्यम्‌
श्रीबादरायणिरुवाच
कर्हिचित्सुखमासीनं स्वतल्पस्थं जगद्गुरुम्‌
पतिं पर्यचरद्भैष्मी व्यजनेन सखीजनैः
यस्त्वेतल्लीलया विश्वं सृजत्यत्त्यवतीश्वरः
स हि जातः स्वसेतूनां गोपीथाय यदुष्वजः
तस्मिनन्तर्गृहे भ्राजन्‌ मुक्तादामविलम्बिना
विराजिते वितानेन दीपैर्मणिमयैरपि
मल्लिकादामभिः पुष्पैर्द्विरेफकुलनादिते
जालरन्ध्रप्रविष्टैश्च गोभिश्चन्द्रमसोऽमलैः
पारिजातवनामोद वायुनोद्यानशालिना
धूपैरगुरुजै राजन्जालरन्ध्रविनिर्गतैः
पयःफेननिभे शुभ्रे पर्यङ्के कशिपूत्तमे
उपतस्थे सुखासीनं जगतामीश्वरं पतिम्‌
वालव्यजनमादाय रत्नदण्डं सखीकरात्‌
तेन वीजयती देवी उपासां चक्र ईश्वरम्‌
सोपाच्युतं क्वणयती मणिनूपुराभ्यां
रेजेऽङ्गुलीयवलयव्यजनाग्रहस्ता
वस्त्रान्तगूढकुचकुङ्कुमशोणहार
भासा नितम्बधृतया च परार्ध्यकाञ्च्या
तां रूपिणीं श्रीयमनन्यगतिं निरीक्ष्य
या लीलया धृततनोरनुरूपरूपा
प्रीतः स्मयन्नलककुण्डलनिष्ककण्ठ
वक्त्रोल्लसत्स्मितसुधां हरिराबभाषे
श्रीभगवानुवाच
राजपुत्रीप्सिता भूपैर्लोकपालविभूतिभिः
महानुभावैः श्रीमद्भी रूपौदार्यबलोर्जितैः
तान्प्राप्तानर्थिनो हित्वा चैद्यादीन्स्मरदुर्मदान्‌
दत्ता भ्रात्रा स्वपित्रा च कस्मान्नो ववृषेऽसमान्‌
राजभ्यो बिभ्यतः सुभ्रु समुद्रं शरणं गतान्‌
बलवद्भिः कृतद्वेषान्प्रायस्त्यक्तनृपासनान्‌
अस्पष्टवर्त्मनाम्पुंसामलोकपथमीयुषाम्‌
आस्थिताः पदवीं सुभ्रु प्रायः सीदन्ति योषितः
निष्किञ्चना वयं शश्वन्निष्किञ्चनजनप्रियाः
तस्मा त्प्रायेण न ह्याढ्या मां भजन्ति सुमध्यमे
ययोरात्मसमं वित्तं जन्मैश्वर्याकृतिर्भवः
तयोर्विवाहो मैत्री च नोत्तमाधमयोः क्वचित्‌
वैदर्भ्येतदविज्ञाय त्वयादीर्घसमीक्षया
वृता वयं गुणैर्हीना भिक्षुभिः श्लाघिता मुधा
अथात्मनोऽनुरूपं वै भजस्व क्षत्रियर्षभम्‌
येन त्वमाशिषः सत्या इहामुत्र च लप्स्यसे
चैद्यशाल्वजरासन्ध दन्तवक्रादयो नृपाः
मम द्विषन्ति वामोरु रुक्मी चापि तवाग्रजः
तेषां वीर्यमदान्धानां दृप्तानां स्मयनुत्तये
आनितासि मया भद्रे तेजोपहरतासताम्‌
उदासीना वयं नूनं न स्त्र्यपत्यार्थकामुकाः
आत्मलब्ध्यास्महे पूर्णा गेहयोर्ज्योतिरक्रियाः
श्रीशुक उवाच
एतावदुक्त्वा भगवानात्मानं वल्लभामिव
मन्यमानामविश्लेषात्तद्दर्पघ्न उपारमत्‌
इति त्रिलोकेशपतेस्तदात्मनः प्रियस्य देव्यश्रुतपूर्वमप्रियम्‌
आश्रुत्य भीता हृदि जातवेपथुश्चिन्तां दुरन्तां रुदती जगाम ह
पदा सुजातेन नखारुणश्रीया भुवं लिखन्त्यश्रुभिरञ्जनासितैः
आसिञ्चती कुङ्कुमरूषितौ स्तनौ तस्थावधोमुख्यतिदुःखरुद्धवाक्‌
तस्याः सुदुःखभयशोकविनष्टबुद्धेर्‌
हस्ताच्छ्लथद्वलयतो व्यजनं पपात
देहश्च विक्लवधियः सहसैव मुह्यन्‌
रम्भेव वायुविहतो प्रविकीर्य केशान्‌
तद्दृष्ट्वा भगवान्कृष्णः प्रियायाः प्रेमबन्धनम्‌
हास्यप्रौढिमजानन्त्याः करुणः सोऽन्वकम्पत
पर्यङ्कादवरुह्याशु तामुत्थाप्य चतुर्भुजः
केशान्समुह्य तद्वक्त्रं प्रामृजत्पद्मपाणिना
प्रमृज्याश्रुकले नेत्रे स्तनौ चोपहतौ शुचा
आश्लिष्य बाहुना राजननन्यविषयां सतीम्‌
सान्त्वयामास सान्त्वज्ञः कृपया कृपणां प्रभुः
हास्यप्रौढिभ्रमच्चित्तामतदर्हां सतां गतिः
श्रीभगवानुवाच
मा मा वैदर्भ्यसूयेथा जाने त्वां मत्परायणाम्‌
त्वद्वचः श्रोतुकामेन क्ष्वेल्याचरितमङ्गने
मुखं च प्रेमसंरम्भ स्फुरिताधरमीक्षितुम्‌
कटाक्षेपारुणापाङ्गं सुन्दरभ्रुकुटीतटम्‌
अयं हि परमो लाभो गृहेषु गृहमेधिनाम्‌
यन्नर्मैरीयते यामः प्रियया भीरु भामिनि
श्रीशुक उवाच
सैवं भगवता राजन्वैदर्भी परिसान्त्विता
ज्ञात्वा तत्परिहासोक्तिं प्रियत्यागभयं जहौ
बभाष ऋषभं पुंसां वीक्षन्ती भगवन्मुखम्‌
सव्रीडहासरुचिर स्निग्धापाङ्गेन भारत
श्रीरुक्मिण्युवाच
नन्वेवमेतदरविन्दविलोचनाह यद्वै भवान्भगवतोऽसदृशी विभूम्नः
क्व स्वे महिम्न्यभिरतो भगवांस्त्र्यधीशः क्वाहं गुणप्रकृतिरज्ञगृहीतपादा
सत्यं भयादिव गुणेभ्य उरुक्रमान्तः
शेते समुद्र उपलम्भनमात्र आत्मा
नित्यं कदिन्द्रियगणैः कृतविग्रहस्त्वं
त्वत्सेवकैर्नृपपदं विधुतं तमोऽन्धम्‌
त्वत्पादपद्ममकरन्दजुषां मुनीनां
वर्त्मास्फुटं न्र्पशुभिर्ननु दुर्विभाव्यम्‌
यस्मादलौकिकमिवेहितमीश्वरस्य
भूमंस्तवेहितमथो अनु ये भवन्तम्‌
निष्किञ्चनो ननु भवान्न यतोऽस्ति किञ्चिद्‌
यस्मै बलिं बलिभुजोऽपि हरन्त्यजाद्याः
न त्वा विदन्त्यसुतृपोऽन्तकमाढ्यतान्धाः
प्रेष्ठो भवान्बलिभुजामपि तेऽपि तुभ्यम्‌
त्वं वै समस्तपुरुषार्थमयः फलात्मा
यद्वाञ्छया सुमतयो विसृजन्ति कृत्स्नम्‌
तेषां विभो समुचितो भवतः समाजः
पुंसः स्त्रियाश्च रतयोः सुखदुःखिनोर्न
त्वं न्यस्तदण्डमुनिभिर्गदितानुभाव
आत्मात्मदश्च जगतामिति मे वृतोऽसि
हित्वा भवद्भ्रुव उदीरितकालवेग
ध्वस्ताशिषोऽब्जभवनाकपतीन्कुतोऽन्ये
जाड्यं वचस्तव गदाग्रज यस्तु भूपान्‌
विद्राव्य शार्ङ्गनिनदेन जहर्थ मां त्वम्‌
सिंहो यथा स्वबलिमीश पशून्स्वभागं
तेभ्यो भयाद्यदुदधिं शरणं प्रपन्नः
यद्वाञ्छया नृपशिखामणयोऽन्गवैन्य
जायन्तनाहुषगयादय ऐक्यपत्यम्‌
राज्यं विसृज्य विविशुर्वनमम्बुजाक्ष
सीदन्ति तेऽनुपदवीं त इहास्थिताः किम्‌
कान्यं श्रयेत तव पादसरोजगन्धम्‌
आघ्राय सन्मुखरितं जनतापवर्गम्‌
लक्ष्म्यालयं त्वविगणय्य गुणालयस्य
मर्त्या सदोरुभयमर्थविवीतदृष्टिः
तं त्वानुरूपमभजं जगतामधीशम्‌
आत्मानमत्र च परत्र च कामपूरम्‌
स्यान्मे तवाङ्घ्रिररणं सृतिभिर्भ्रमन्त्या
यो वै भजन्तमुपयात्यनृतापवर्गः
तस्याः स्युरच्युत नृपा भवतोपदिष्टाः
स्त्रीणां गृहेषु खरगोश्वविडालभृत्याः
यत्कर्णमूलमन्कर्षण नोपयायाद्‌
युष्मत्कथा मृडविरिञ्चसभासु गीता
त्वक्ष्मश्रुरोमनखकेशपिनद्धमन्तर्‌
मांसास्थिरक्तकृमिविट्कफपित्तवातम्‌
जीवच्छवं भजति कान्तमतिर्विमूढा
या ते पदाब्जमकरन्दमजिघ्रती स्त्री
अस्त्वम्बुजाक्ष मम ते चरणानुराग
आत्मन्रतस्य मयि चानतिरिक्तदृष्टेः
यर्ह्यस्य वृद्धय उपात्तरजोऽतिमात्रो
मामीक्षसे तदु ह नः परमानुकम्पा
नैवालीकमहं मन्ये वचस्ते मधुसूदन
अम्बाया एव हि प्रायः कन्यायाः स्याद्रतिः क्वचित्‌
व्यूढायाश्चापि पुंश्चल्या मनोऽभ्येति नवं नवम्‌
बुधोऽसतीं न बिभृयात्तां बिभ्रदुभयच्युतः
श्रीभगवानुवाच
साध्व्येतच्छ्रोतुकामैस्त्वं राजपुत्री प्रलम्भिता
मयोदितं यदन्वात्थ सर्वं तत्सत्यमेव हि
यान्यान्कामयसे कामान्मय्यकामाय भामिनि
सन्ति ह्येकान्तभक्तायास्तव कल्याणि नित्यद
उपलब्धं पतिप्रेम पातिव्रत्यं च तेऽनघे
यद्वाक्यैश्चाल्यमानाया न धीर्मय्यपकर्षिता
ये मां भजन्ति दाम्पत्ये तपसा व्रतचर्यया
कामात्मानोऽपवर्गेशं मोहिता मम मायया
मां प्राप्य मानिन्यपवर्गसम्पदं
वाञ्छन्ति ये सम्पद एव तत्पतिम्‌
ते मन्दभागा निरयेऽपि ये नृणां
मात्रात्मकत्वात्निरयः सुसङ्गमः
दिष्ट्या गृहेश्वर्यसकृन्मयि त्वया कृतानुवृत्तिर्भवमोचनी खलैः
सुदुष्करासौ सुतरां दुराशिषो ह्यसुंभराया निकृतिं जुषः स्त्रियाः
न त्वादृशीम्प्रणयिनीं गृहिणीं गृहेषु
पश्यामि मानिनि यया स्वविवाहकाले
प्राप्तान्नृपान्न विगणय्य रहोहरो मे
प्रस्थापितो द्विज उपश्रुतसत्कथस्य
भ्रातुर्विरूपकरणं युधि निर्जितस्य
प्रोद्वाहपर्वणि च तद्वधमक्षगोष्ठ्याम्‌
दुःखं समुत्थमसहोऽस्मदयोगभीत्या
नैवाब्रवीः किमपि तेन वयं जितास्ते
दूतस्त्वयात्मलभने सुविविक्तमन्त्रः
प्रस्थापितो मयि चिरायति शून्यमेतत्‌
मत्वा जिहास इदं अङ्गमनन्ययोग्यं
तिष्ठेत तत्त्वयि वयं प्रतिनन्दयामः
श्रीशुक उवाच
एवं सौरतसंलापैर्भगवान्जगदीश्वरः
स्वरतो रमया रेमे नरलोकं विडम्बयन्‌
तथान्यासामपि विभुर्गृहेसु गृहवानिव
आस्थितो गृहमेधीयान्धर्मान्लोकगुरुर्हरिः
श्रीशुक उवाच
एकैकशस्ताः कृष्णस्य पुत्रान्दशदशाबआः
अजीजनन्ननवमान्पितुः सर्वात्मसम्पदा
गृहादनपगं वीक्ष्य राजपुत्र्योऽच्युतं स्थितम्‌
प्रेष्ठं न्यमंसत स्वं स्वं न तत्तत्त्वविदः स्त्रियः
चार्वब्जकोशवदनायतबाहुनेत्र
सप्रेमहासरसवीक्षितवल्गुजल्पैः
सम्मोहिता भगवतो न मनो विजेतुं
स्वैर्विभ्रमैः समशकन्वनिता विभूम्नः
स्मायावलोकलवदर्शितभावहारि
भ्रूमण्डलप्रहितसौरतमन्त्रशौण्डैः
पत्न्यस्तु शोडशसहस्रमनङ्गबाणैर्‌
यस्येन्द्रियं विमथितुम्करणैर्न शेकुः
इत्थं रमापतिमवाप्य पतिं स्त्रियस्ता
ब्रह्मादयोऽपि न विदुः पदवीं यदीयाम्‌
भेजुर्मुदाविरतमेधितयानुराग
हासावलोकनवसङ्गमलालसाद्यम्‌
प्रत्युद्गमासनवरार्हणपादशौच
ताम्बूलविश्रमणवीजनगन्धमाल्यैः
केशप्रसारशयनस्नपनोपहार्यैः
दासीशता अपि विभोर्विदधुः स्म दास्यम्‌
तासां या दशपुत्राणां कृष्णस्त्रीणां पुरोदिताः
अष्टौ महिष्यस्तत्पुत्रान्प्रद्युम्नादीन्गृणामि ते
चारुदेष्णः सुदेष्णश्च चारुदेहश्च वीर्यवान्‌
सुचारुश्चारुगुप्तश्च भद्रचारुस्तथापरः
चारुचन्द्रो विचारुश्च चारुश्च दशमो हरेः
प्रद्युम्नप्रमुखा जाता रुक्मिण्यां नावमाः पितुः
भानुः सुभानुः स्वर्भानुः प्रभानुर्भानुमांस्तथा
चन्द्रभानुर्बृहद्भानुरतिभानुस्तथाष्टमः
श्रीभानुः प्रतिभानुश्च सत्यभामात्मजा दश
साम्बः सुमित्रः पुरुजिच्छतजिच्च सहस्रजित्‌
विययश्चित्रकेतुश्च वसुमान्द्रविडः क्रतुः
जाम्बवत्याः सुता ह्येते साम्बाद्याः पितृसम्मताः
वीरश्चन्द्रोऽश्वसेनश्च चित्रगुर्वेगवान्वृषः
आमः शङ्कुर्वसुः श्रीमान्कुन्तिर्नाग्नजितेः सुताः
श्रुतः कविर्वृषो वीरः सुबाहुर्भद्र एकलः
शान्तिर्दर्शः पूर्णमासः कालिन्द्याः सोमकोऽवरः
प्रघोषो गात्रवान्सिंहो बलः प्रबल ऊर्धगः
माद्र्याः पुत्रा महाशक्तिः सह ओजोऽपराजितः
वृको हर्षोऽनिलो गृध्रो वर्धनोन्नाद एव च
महांसः पावनो वह्निर्मित्रविन्दात्मजाः क्षुधिः
सङ्ग्रामजिद्बृहत्सेनः शूरः प्रहरणोऽरिजित्‌
जयः सुभद्रो भद्राया वाम आयुश्च सत्यकः
दीप्तिमांस्ताम्रतप्ताद्या रोहिण्यास्तनया हरेः
प्रद्यम्नाच्चानिरुद्धोऽभूद्रुक्मवत्यां महाबलः
पुत्र्यां तु रुक्मिणो राजन्नाम्ना भोजकटे पुरे
एतेषां पुत्रपौत्राश्च बभूवुः कोटिशो नृप
मातरः कृष्णजातीनां सहस्राणि च षोडश
श्रीराजोवाच
कथं रुक्म्यरीपुत्राय प्रादाद्दुहितरं युधि
कृष्णेन परिभूतस्तं हन्तुं रन्ध्रं प्रतीक्षते
एतदाख्याहि मे विद्वन्द्विषोर्वैवाहिकं मिथः
अनागतमतीतं च वर्तमानमतीन्द्रियम्‌
विप्रकृष्टं व्यवहितं सम्यक्पश्यन्ति योगिनः
श्रीशुक उवाच
वृतः स्वयंवरे साक्षादनण्गोऽण्गयुतस्तया
राज्ञः समेतान्निर्जित्य जहारैकरथो युधि
यद्यप्यनुस्मरन्वैरं रुक्मी कृष्णावमानितः
व्यतरद्भागिनेयाय सुतां कुर्वन्स्वसुः प्रियम्‌
रुक्मिण्यास्तनयां राजन्कृतवर्मसुतो बली
उपयेमे विशालाक्षीं कन्यां चारुमतीं किल
दौहित्रायानिरुद्धाय पौत्रीं रुक्म्याददाद्धरेः
रोचनां बद्धवैरोऽपि स्वसुः प्रियचिकीर्षया
जानन्नधर्मं तद्यौनं स्नेहपाशानुबन्धनः
तस्मिन्नभ्युदये राजन्रुक्मिणी रामकेशवौ
पुरं भोजकटं जग्मुः साम्बप्रद्युम्नकादयः
तस्मिन्निवृत्त उद्वाहे कालिङ्गप्रमुखा नृपाः
दृप्तास्ते रुक्मिणं प्रोचुर्बलमक्षैर्विनिर्जय
अनक्षज्ञो ह्ययं राजन्नपि तद्व्यसनं महत्‌
इत्युक्तो बलमाहूय तेनाक्षैर्रुक्म्यदीव्यत
शतं सहस्रमयुतं रामस्तत्राददे पणम्‌
तं तु रुक्म्यजयत्तत्र कालिङ्गः प्राहसद्बलम्‌
दन्तान्सन्दर्शयन्नुच्चैर्नामृष्यत्तद्धलायुधः
ततो लक्षं रुक्म्यगृह्णाद्ग्लहं तत्राजयद्बलः
जितवानहमित्याह रुक्मी कैतवमाश्रितः
मन्युना क्षुभितः श्रीमान्समुद्र इव पर्वणि
जात्यारुणाक्षोऽतिरुषा न्यर्बुदं ग्लहमाददे
तं चापि जितवान्रामो धर्मेण छलमाश्रितः
रुक्मी जितं मयात्रेमे वदन्तु प्राश्निका इति
तदाब्रवीन्नभोवाणी बलेनैव जितो ग्लहः
धर्मतो वचनेनैव रुक्मी वदति वै मृषा
तामनादृत्य वैदर्भो दुष्टराजन्यचोदितः
सङ्कर्षणं परिहसन्बभाषे कालचोदितः
नैवाक्षकोविदा यूयं गोपाला वनगोचराः
अक्षैर्दीव्यन्ति राजानो बाणैश्च न भवादृशाः
रुक्मिणैवमधिक्षिप्तो राजभिश्चोपहासितः
क्रुद्धः परिघमुद्यम्य जघ्ने तं नृम्णसंसदि
कलिङ्गराजं तरसा गृहीत्वा दशमे पदे
दन्तानपातयत्क्रुद्धो योऽहसद्विवृतैर्द्विजैः
अन्ये निर्भिन्नबाहूरु शिरसो रुधिरोक्षिताः
राजानो दुद्रवर्भीता बलेन पङ्घार्दिताः
निहते रुक्मिणि श्याले नाब्रवीत्साध्वसाधु वा
रक्मिणीबलयो राजन्स्नेहभङ्गभयाद्धरिः
ततोऽनिरुद्धं सह सूर्यया वरं रथं समारोप्य ययुः कुशस्थलीम्‌
रामादयो भोजकटाद्दशार्हाः सिद्धाखिलार्था मधुसूदनाश्रयाः
श्रीराजोवाच
बाणस्य तनयामूषामुपयेमे यदूत्तमः
तत्र युद्धमभूद्घोरं हरिशङ्करयोर्महत्‌
एतत्सर्वं महायोगिन्समाख्यातुं त्वमर्हसि
श्रीशुक उवाच
बाणः पुत्रशतज्येष्ठो बलेरासीन्महात्मनः
येन वामनरूपाय हरयेऽदायि मेदिनी
तस्यौरसः सुतो बानः शिवभक्तिरतः सदा
मान्यो वदान्यो धीमांश्च सत्यसन्धो दृढव्रतः
शोणिताख्ये पुरे रम्ये स राज्यमकरोत्पुरा
तस्य शम्भोः प्रसादेन किङ्करा इव तेऽमराः
सहस्रबाहुर्वाद्येन ताण्दवेऽतोषयन्मृडम्‌
भगवान्सर्वभूतेशः शरण्यो भक्तवत्सलः
वरेण छन्दयामास स तं वव्रे पुराधिपम्‌
स एकदाह गिरिशं पार्श्वस्थं वीर्यदुर्मदः
किरीटेनार्कवर्णेन संस्पृशंस्तत्पदाम्बुजम्‌
नमस्ये त्वां महादेव लोकानां गुरुमीश्वरम्‌
पुंसामपूर्णकामानां कामपूरामराङ्घ्रिपम्‌
दोःसहस्रं त्वया दत्तं परं भाराय मेऽभवत्‌
त्रिलोक्यां प्रतियोद्धारं न लभे त्वदृते समम्‌
कण्डूत्या निभृतैर्दोर्भिर्युयुत्सुर्दिग्गजानहम्‌
आद्यायां चूर्णयन्नद्रीन्भीतास्तेऽपि प्रदुद्रुवुः
तच्छ्रुत्वा भगवान्क्रुद्धः केतुस्ते भज्यते यदा
त्वद्दर्पघ्नं भवेन्मूढ संयुगं मत्समेन ते
इत्युक्तः कुमतिर्हृष्टः स्वगृहं प्राविशन्नृप
प्रतीक्षन्गिरिशादेशं स्ववीर्यनशनम्कुधीः
तस्योषा नाम दुहिता स्वप्ने प्राद्युम्निना रतिम्‌
कन्यालभत कान्तेन प्रागदृष्टश्रुतेन सा
सा तत्र तमपश्यन्ती क्वासि कान्तेति वादिनी
सखीनां मध्य उत्तस्थौ विह्वला व्रीडिता भृशम्‌
बाणस्य मन्त्री कुम्भाण्डश्चित्रलेखा च तत्सुता
सख्यपृच्छत्सखीमूषां कौतूहलसमन्विता
कं त्वं मृगयसे सुभ्रु कीदृशस्ते मनोरथः
हस्तग्राहं न तेऽद्यापि राजपुत्र्युपलक्षये
दृष्टः कश्चिन्नरः स्वप्ने श्यामः कमललोचनः
पीतवासा बृहद्बाहुर्योषितां हृदयंगमः
तमहं मृगये कान्तं पाययित्वाधरं मधु
क्वापि यातः स्पृहयतीं क्षिप्त्वा मां वृजिनार्णवे
चित्रलेखोवाच
व्यसनं तेऽपकर्षामि त्रिलोक्यां यदि भाव्यते
तमानेष्ये वरं यस्ते मनोहर्ता तमादिश
इत्युक्त्वा देवगन्धर्व सिद्धचारणपन्नगान्‌
दैत्यविद्याधरान्यक्षान्मनुजांश्च यथालिखत्‌
मनुजेषु च सा वृष्नीन्शूरमानकदुन्दुभिम्‌
व्यलिखद्रामकृष्णौ च प्रद्युम्नं वीक्ष्य लज्जिता
अनिरुद्धं विलिखितं वीक्ष्योषावाङ्मुखी ह्रिया
सोऽसावसाविति प्राह स्मयमाना महीपते
चित्रलेखा तमाज्ञाय पौत्रं कृष्णस्य योगिनी
ययौ विहायसा राजन्द्वारकां कृष्णपालिताम्‌
तत्र सुप्तं सुपर्यङ्के प्राद्युम्निं योगमास्थिता
गृहीत्वा शोणितपुरं सख्यै प्रियमदर्शयत्‌
सा च तं सुन्दरवरं विलोक्य मुदितानना
दुष्प्रेक्ष्ये स्वगृहे पुम्भी रेमे प्राद्युम्निना समम्‌
परार्ध्यवासःस्रग्गन्ध धूपदीपासनादिभिः
पानभोजनभक्ष्यैश्च वाक्यैः शुश्रूषणार्चितः
गूढः कन्यापुरे शश्वत्‌ प्रवृद्धस्नेहया तया
नाहर्गणान्स बुबुधे ऊषयापहृतेन्द्रियः
तां तथा यदुवीरेण भुज्यमानां हतव्रताम्‌
हेतुभिर्लक्षयां चक्रुरापृईतां दुरवच्छदैः
भटा आवेदयां चक्रू राजंस्ते दुहितुर्वयम्‌
विचेष्टितं लक्षयाम कन्यायाः कुलदूषणम्‌
अनपायिभिरस्माभिर्गुप्तायाश्च गृहे प्रभो
कन्याया दूषणं पुम्भिर्दुष्प्रेक्ष्याया न विद्महे
ततः प्रव्यथितो बाणो दुहितुः श्रुतदूषणः
त्वरितः कन्यकागारं प्राप्तोऽद्राक्षीद्यदूद्वहम्‌
कामात्मजं तं भुवनैकसुन्दरं श्यामं पिशङ्गाम्बरमम्बुजेक्षणम्‌
बृहद्भुजं कुण्डलकुन्तलत्विषा स्मितावलोकेन च मण्डिताननम्‌
दीव्यन्तमक्षैः प्रिययाभिनृम्णया तदङ्गसङ्गस्तनकुङ्कुमस्रजम्‌
बाह्वोर्दधानं मधुमल्लिकाश्रितां तस्याग्र आसीनमवेक्ष्य विस्मितः
स तं प्रविष्टं वृतमाततायिभिर्भटैरनीकैरवलोक्य माधवः
उद्यम्य मौर्वं परिघं व्यवस्थितो यथान्तको दण्डधरो जिघांसया
जिघृक्षया तान्परितः प्रसर्पतः शुनो यथा शूकरयूथपोऽहनत्‌
ते हन्यमाना भवनाद्विनिर्गता निर्भिन्नमूर्धोरुभुजाः प्रदुद्रुवुः
तं नागपाशैर्बलिनन्दनो बली घ्नन्तं स्वसैन्यं कुपितो बबन्ध ह
ऊषा भृशं शोकविषादविह्वला बद्धं निशम्याश्रुकलाक्ष्यरौत्सीत्‌
शृईशुक उवाच
अपश्यतां चानिरुद्धं तद्बन्धूनां च भारत
चत्वारो वार्षिका मासा व्यतीयुरनुशोचताम्‌
नारदात्तदुपाकर्ण्य वार्तां बद्धस्य कर्म च
प्रययुः शोणितपुरं वृष्णयः कृष्णदैवताः
प्रद्युम्नो युयुधानश्च गदः साम्बोऽथ सारणः
नन्दोपनन्दभद्राद्या रामकृष्णानुवर्तिनः
अक्षौहिणीभिर्द्वादशभिः समेताः सर्वतो दिशम्‌
रुरुधुर्बाणनगरं समन्तात्सात्वतर्षभाः
भज्यमानपुरोद्यान प्राकाराट्टालगोपुरम्‌
प्रेक्षमाणो रुषाविष्टस्तुल्यसैन्योऽभिनिर्ययौ
बाणार्थे भगवान्रुद्रः ससुतः प्रमथैर्वृतः
आरुह्य नन्दिवृषभं युयुधे रामकृष्णयोः
आसीत्सुतुमुलं युद्धमद्भुतं रोमहर्षणम्‌
कृष्णशङ्करयो राजन्प्रद्युम्नगुहयोरपि
कुम्भाण्डकूपकर्णाभ्यां बलेन सह संयुगः
साम्बस्य बाणपुत्रेण बाणेन सह सात्यकेः
ब्रह्मादयः सुराधीशा मुनयः सिद्धचारणाः
गन्धर्वाप्सरसो यक्षा विमानैर्द्रष्टुमागमन्‌
शङ्करानुचरान्शौरिर्भूतप्रमथगुह्यकान्‌
डाकिनीर्यातुधानांश्च वेतालान्सविनायकान्‌
प्रेतमातृपिशाचांश्च कुष्माण्डान्ब्रह्मराक्षसान्‌
द्रावयामास तीक्ष्णाग्रैः शरैः शार्ङ्गधनुश्च्युतैः
पृथग्विधानि प्रायुङ्क्त पिणाक्यस्त्राणि शार्ङ्गिणे
प्रत्यस्त्रैः शमयामास शार्ङ्गपाणिरविस्मितः
ब्रह्मास्त्रस्य च ब्रह्मास्त्रं वायव्यस्य च पार्वतम्‌
आग्नेयस्य च पार्जन्यं नैजं पाशुपतस्य च
मोहयित्वा तु गिरिशं जृम्भणास्त्रेण जृम्भितम्‌
बाणस्य पृतनां शौरिर्जघानासिगदेषुभिः
स्कन्दः प्रद्युम्नबाणौघैरर्द्यमानः समन्ततः
असृग्विमुञ्चन्गात्रेभ्यः शिखिनापक्रमद्रणात्‌
कुम्भाण्डकूपकर्णश्च पेततुर्मुषलार्दितौ
दुद्रुवुस्तदनीकनि हतनाथानि सर्वतः
विशीर्यमाणम्स्वबलं दृष्ट्वा बाणोऽत्यमर्षितः
कृष्णमभ्यद्रवत्सङ्ख्ये रथी हित्वैव सात्यकिम्‌
धनूंष्याकृष्य युगपद्बाणः पञ्चशतानि वै
एकैकस्मिन्शरौ द्वौ द्वौ सन्दधे रणदुर्मदः
तानि चिच्छेद भगवान्धनूंसि युगपद्धरिः
सारथिं रथमश्वांश्च हत्वा शङ्खमपूरयत्‌
तन्माता कोटरा नाम नग्ना मक्तशिरोरुहा
पुरोऽवतस्थे कृष्णस्य पुत्रप्राणरिरक्षया
ततस्तिर्यङ्मुखो नग्नामनिरीक्षन्गदाग्रजः
बाणश्च तावद्विरथश्छिन्नधन्वाविशत्पुरम्‌
विद्राविते भूतगणे ज्वरस्तु त्रीशिरास्त्रीपात्‌
अभ्यधावत दाशार्हं दहन्निव दिशो दश
अथ नारायणः देवः तं दृष्ट्वा व्यसृजज्ज्वरम्‌
माहेश्वरो वैष्णवश्च युयुधाते ज्वरावुभौ
माहेश्वरः समाक्रन्दन्वैष्णवेन बलार्दितः
अलब्ध्वाभयमन्यत्र भीतो माहेश्वरो ज्वरः
शरणार्थी हृषीकेशं तुष्टाव प्रयताञ्जलिः
ज्वर उवाच
नमामि त्वानन्तशक्तिं परेशम्सर्वात्मानं केवलं ज्ञप्तिमात्रम्‌
विश्वोत्पत्तिस्थानसंरोधहेतुं यत्तद्ब्रह्म ब्रह्मलिङ्गम्प्रशान्तम्‌
कालो दैवं कर्म जीवः स्वभावो द्रव्यं क्षेत्रं प्राण आत्मा विकारः
तत्सङ्घातो बीजरोहप्रवाहस्त्वन्मायैषा तन्निषेधं प्रपद्ये
नानाभावैर्लीलयैवोपपन्नैर्देवान्साधून्लोकसेतून्बिभर्षि
हंस्युन्मार्गान्हिंसया वर्तमानान्जन्मैतत्ते भारहाराय भूमेः
तप्तोऽहम्ते तेजसा दुःसहेन शान्तोग्रेणात्युल्बणेन ज्वरेण
तावत्तापो देहिनां तेऽन्घ्रिमूलं नो सेवेरन्यावदाशानुबद्धाः
श्रीभगवानुवाच
त्रिशिरस्ते प्रसन्नोऽस्मि व्येतु ते मज्ज्वराद्भयम्‌
यो नौ स्मरति संवादं तस्य त्वन्न भवेद्भयम्‌
इत्युक्तोऽच्युतमानम्य गतो माहेश्वरो ज्वरः
बाणस्तु रथमारूढः प्रागाद्योत्स्यन्जनार्दनम्‌
ततो बाहुसहस्रेण नानायुधधरोऽसुरः
मुमोच परमक्रुद्धो बाणांश्चक्रायुधे नृप
तस्यास्यतोऽस्त्राण्यसकृच्चक्रेण क्षुरनेमिना
चिच्छेद भगवान्बाहून्शाखा इव वनस्पतेः
बाहुषु छिद्यमानेषु बाणस्य भगवान्भवः
भक्तानकम्प्युपव्रज्य चक्रायुधमभाषत
श्रीरुद्र उवाच
त्वं हि ब्रह्म परं ज्योतिर्गूढं ब्रह्मणि वाङ्मये
यं पश्यन्त्यमलात्मान आकाशमिव केवलम्‌
नाभिर्नभोऽग्निर्मुखमम्बु रेतो
द्यौः शीर्षमाशाः श्रुतिरङ्घ्रिरुर्वी
चन्द्रो मनो यस्य दृगर्क आत्मा
अहं समुद्रो जठरं भुजेन्द्रः
रोमाणि यस्यौषधयोऽम्बुवाहाः
केशा विरिञ्चो धिषणा विसर्गः
प्रजापतिर्हृदयं यस्य धर्मः
स वै भवान्पुरुषो लोककल्पः
तवावतारोऽयमकुण्ठधामन्धर्मस्य गुप्त्यै जगतो हिताय
वयं च सर्वे भवतानुभाविता विभावयामो भुवनानि सप्त
त्वमेक आद्यः पुरुषोऽद्वितीयस्तुर्यः स्वदृग्धेतुरहेतुरीशः
प्रतीयसेऽथापि यथाविकारं स्वमायया सर्वगुणप्रसिद्ध्यै
यथैव सूर्यः पिहितश्छायया स्वया
छायां च रूपाणि च सञ्चकास्ति
एवं गुणेनापिहितो गुणांस्त्वम्‌
आत्मप्रदीपो गुणिनश्च भूमन्‌
यन्मायामोहितधियः पुत्रदारगृहादिषु
उन्मज्जन्ति निमज्जन्ति प्रसक्ता वृजिनार्णवे
देवदत्तमिमं लब्ध्वा नृलोकमजितेन्द्रियः
यो नाद्रियेत त्वत्पादौ स शोच्यो ह्यात्मवञ्चकः
यस्त्वां विसृजते मर्त्य आत्मानं प्रियमीश्वरम्‌
विपर्ययेन्द्रियार्थार्थं विषमत्त्यमृतं त्यजन्‌
अहं ब्रह्माथ विबुधा मुनयश्चामलाशयाः
सर्वात्मना प्रपन्नास्त्वामात्मानं प्रेष्ठमीश्वरम्‌
तं त्वा जगत्स्थित्युदयान्तहेतुं
समं प्रसान्तं सुहृदात्मदैवम्‌
अनन्यमेकं जगदात्मकेतं
भवापवर्गाय भजाम देवम्‌
अयं ममेष्टो दयितोऽनुवर्ती मयाभयं दत्तममुष्य देव
सम्पाद्यतां तद्भवतः प्रसादो यथा हि ते दैत्यपतौ प्रसादः
श्रीभगवानुवाच
यदात्थ भगवंस्त्वं नः करवाम प्रियं तव
भवतो यद्व्यवसितं तन्मे साध्वनुमोदितम्‌
अवध्योऽयं ममाप्येष वैरोचनिसुतोऽसुरः
प्रह्रादाय वरो दत्तो न वध्यो मे तवान्वयः
दर्पोपशमनायास्य प्रवृक्णा बाहवो मया
सूदितं च बलं भूरि यच्च भारायितं भुवः
चत्वारोऽस्य भुजाः शिष्टा भविष्यत्यजरामरः
पार्षदमुख्यो भवतो न कुतश्चिद्भयोऽसुरः
इति लब्ध्वाभयं कृष्णं प्रणम्य शिरसासुरः
प्राद्युम्निं रथमारोप्य सवध्वो समुपानयत्‌
अक्षौहिण्या परिवृतं सुवासःसमलङ्कृतम्‌
सपत्नीकं पुरस्कृत्य ययौ रुद्रानुमोदितः
स्वराजधानीं समलङ्कृतां ध्वजैः
सतोरणैरुक्षितमार्गचत्वराम्‌
विवेश शङ्खानकदुन्दुभिस्वनैर्‌
अभ्युद्यतः पौरसुहृद्द्विजातिभिः
य एवं कृष्णविजयं शङ्करेण च संयुगम्‌
संस्मरेत्प्रातरुत्थाय न तस्य स्यात्पराजयः
श्रीबादरायणिरुवाच
एकदोपवनं राजन्जग्मुर्यदुकुमारकाः
विहर्तुं साम्बप्रद्युम्न चारुभानुगदादयः
क्रीडित्वा सुचिरं तत्र विचिन्वन्तः पिपासिताः
जलं निरुदके कूपे ददृशुः सत्त्वमद्भुतम्‌
कृकलासं गिरिनिभं वीक्ष्य विस्मितमानसाः
तस्य चोद्धरणे यत्नं चक्रुस्ते कृपयान्विताः
चर्मजैस्तान्तवैः पाशैर्बद्ध्वा पतितमर्भकाः
नाशक्नुरन्समुद्धर्तुं कृष्णायाचख्युरुत्सुकाः
तत्रागत्यारविन्दाक्षो भगवान्विश्वभावनः
वीक्ष्योज्जहार वामेन तं करेण स लीलया
स उत्तमःश्लोककराभिमृष्टो विहाय सद्यः कृकलासरूपम्‌
सन्तप्तचामीकरचारुवर्णः स्वर्ग्यद्भुतालङ्करणाम्बरस्रक्‌
पप्रच्छ विद्वानपि तन्निदानं जनेषु विख्यापयितुं मुकुन्दः
कस्त्वं महाभाग वरेण्यरूपो देवोत्तमं त्वां गणयामि नूनम्‌
दशामिमां वा कतमेन कर्मणा सम्प्रापितोऽस्यतदर्हः सुभद्र
आत्मानमाख्याहि विवित्सतां नो यन्मन्यसे नः क्षममत्र वक्तुम्‌
श्रीशुक उवाच
इति स्म राजा सम्पृष्टः कृष्णेनानन्तमूर्तिना
माधवं प्रणिपत्याह किरीटेनार्कवर्चसा
नृग उवाच
नृगो नाम नरेन्द्रोऽहमिक्ष्वाकुतनयः प्रभो
दानिष्वाख्यायमानेषु यदि ते कर्णमस्पृशम्‌
किं नु तेऽविदितं नाथ सर्वभूतात्मसाक्षिणः
कालेनाव्याहतदृशो वक्ष्येऽथापि तवाज्ञया
यावत्यः सिकता भूमेर्यावत्यो दिवि तारकाः
यावत्यो वर्षधाराश्च तावतीरददं स्म गाः
पयस्विनीस्तरुणीः शीलरूप गुणोपपन्नाः कपिला हेमसृङ्गीः
न्यायार्जिता रूप्यखुराः सवत्सा दुकूलमालाभरणा ददावहम्‌
स्वलङ्कृतेभ्यो गुणशीलवद्भ्यः सीदत्कुटुम्बेभ्य ऋतव्रतेभ्यः
तपःश्रुतब्रह्मवदान्यसद्भ्यः प्रादां युवभ्यो द्विजपुङ्गवेभ्यः
गोभूहिरण्यायतनाश्वहस्तिनः कन्याः सदासीस्तिलरूप्यशय्याः
वासांसि रत्नानि परिच्छदान्रथानिष्टं च यज्ञैश्चरितं च पूर्तम्‌
कस्यचिद्द्विजमुख्यस्य भ्रष्टा गौर्मम गोधने
सम्पृक्ताविदुषा सा च मया दत्ता द्विजातये
तां नीयमानां तत्स्वामी दृष्ट्रोवाच ममेति तम्‌
ममेति परिग्राह्याह नृगो मे दत्तवानिति
विप्रौ विवदमानौ मामूचतुः स्वार्थसाधकौ
भवान्दातापहर्तेति तच्छ्रुत्वा मेऽभवद्भ्रमः
अनुनीतावुभौ विप्रौ धर्मकृच्छ्रगतेन वै
गवां लक्षं प्रकृष्टानां दास्याम्येषा प्रदीयताम्‌
भवन्तावनुगृह्णीतां किङ्करस्याविजानतः
समुद्धरतं मां कृच्छ्रात्पतन्तं निरयेऽशुचौ
नाहं प्रतीच्छे वै राजन्नित्युक्त्वा स्वाम्यपाक्रमत्‌
नान्यद्गवामप्ययुतमिच्छामीत्यपरो ययौ
एतस्मिन्नन्तरे यामैर्दूतैर्नीतो यमक्षयम्‌
यमेन पृष्टस्तत्राहं देवदेव जगत्पते
पूर्वं त्वमशुभं भुङ्क्ष उताहो नृपते शुभम्‌
नान्तं दानस्य धर्मस्य पश्ये लोकस्य भास्वतः
पूर्वं देवाशुभं भुञ्ज इति प्राह पतेति सः
तावदद्राक्षमात्मानं कृकलासं पतन्प्रभो
ब्रह्मण्यस्य वदान्यस्य तव दासस्य केशव
स्मृतिर्नाद्यापि विध्वस्ता भवत्सन्दर्शनार्थिनः
स त्वं कथं मम विभोऽक्षिपथः परात्मा
योगेश्वरः श्रुतिदृशामलहृद्विभाव्यः
साक्षादधोक्षज उरुव्यसनान्धबुद्धेः
स्यान्मेऽनुदृश्य इह यस्य भवापवर्गः
देवदेव जगन्नाथ गोविन्द पुरुषोत्तम
नारायण हृषीकेश पुण्यश्लोकाच्युताव्यय
अनुजानीहि मां कृष्ण यान्तं देवगतिं प्रभो
यत्र क्वापि सतश्चेतो भूयान्मे त्वत्पदास्पदम्‌
नमस्ते सर्वभावाय ब्रह्मणेऽनन्तशक्तये
कृष्णाय वासुदेवाय योगानां पतये नमः
इत्युक्त्वा तं परिक्रम्य पादौ स्पृष्ट्वा स्वमौलिना
अनुज्ञातो विमानाग्र्यमारुहत्पश्यतां नृणाम्‌
कृष्णः परिजनं प्राह भगवान्देवकीसुतः
ब्रह्मण्यदेवो धर्मात्मा राजन्याननुशिक्षयन्‌
दुर्जरं बत ब्रह्मस्वं भुक्तमग्नेर्मनागपि
तेजीयसोऽपि किमुत राज्ञां ईश्वरमानिनाम्‌
नाहं हालाहलं मन्ये विषं यस्य प्रतिक्रिया
ब्रह्मस्वं हि विषं प्रोक्तं नास्य प्रतिविधिर्भुवि
हिनस्ति विषमत्तारं वह्निरद्भिः प्रशाम्यति
कुलं समूलं दहति ब्रह्मस्वारणिपावकः
ब्रह्मस्वं दुरनुज्ञातं भुक्तं हन्ति त्रिपूरुषम्‌
प्रसह्य तु बलाद्भुक्तं दश पूर्वान्दशापरान्‌
राजानो राजलक्ष्म्यान्धा नात्मपातं विचक्षते
निरयं येऽभिमन्यन्ते ब्रह्मस्वं साधु बालिशाः
गृह्णन्ति यावतः पांशून्क्रन्दतामश्रुबिन्दवः
विप्राणां हृतवृत्तीनाम्वदान्यानां कुटुम्बिनाम्‌
राजानो राजकुल्याश्च तावतोऽब्दान्निरङ्कुशाः
कुम्भीपाकेषु पच्यन्ते ब्रह्मदायापहारिणः
स्वदत्तां परदत्तां वा ब्रह्मवृत्तिं हरेच्च यः
षष्टिवर्षसहस्राणि विष्ठायां जायते कृमिः
न मे ब्रह्मधनं भूयाद्यद्गृध्वाल्पायुषो नराः
पराजिताश्च्युता राज्याद्भवन्त्युद्वेजिनोऽहयः
विप्रं कृतागसमपि नैव द्रुह्यत मामकाः
घ्नन्तं बहु शपन्तं वा नमस्कुरुत नित्यशः
यथाहं प्रणमे विप्राननुकालं समाहितः
तथा नमत यूयं च योऽन्यथा मे स दण्डभाक्‌
ब्राह्मणार्थो ह्यपहृतो हर्तारं पातयत्यधः
अजानन्तमपि ह्येनं नृगं ब्राह्मणगौरिव
एवं विश्राव्य भगवान्मुकुन्दो द्वारकौकसः
पावनः सर्वलोकानां विवेश निजमन्दिरम्‌
श्रीशुक उवाच
बलभद्रः कुरुश्रेष्ठ भगवान्रथमास्थितः
सुहृद्दिदृक्षुरुत्कण्ठः प्रययौ नन्दगोकुलम्‌
परिष्वक्तश्चिरोत्कण्ठैर्गोपैर्गोपीभिरेव च
रामोऽभिवाद्य पितरावाशीर्भिरभिनन्दितः
चिरं नः पाहि दाशार्ह सानुजो जगदीश्वरः
इत्यारोप्याङ्कमालिङ्ग्य नेत्रैः सिषिचतुर्जलैः
गोपवृद्धांश्च विधिवद्यविष्ठैरभिवन्दितः
यथावयो यथासख्यं यथासम्बन्धमात्मनः
समुपेत्याथ गोपालान्हास्यहस्तग्रहादिभिः
विश्रान्तम्सुखमासीनं पप्रच्छुः पर्युपागताः
पृष्टाश्चानामयं स्वेषु प्रेमगद्गदया गिरा
कृष्णे कमलपत्राक्षे सन्न्यस्ताखिलराधसः
कच्चिन्नो बान्धवा राम सर्वे कुशलमासते
कच्चित्स्मरथ नो राम यूयं दारसुतान्विताः
दिष्ट्या कंसो हतः पापो दिष्ट्या मुक्ताः सुहृज्जनाः
निहत्य निर्जित्य रिपून्दिष्ट्या दुर्गं समाश्रीताः
गोप्यो हसन्त्यः पप्रच्छू रामसन्दर्शनादृताः
कच्चिदास्ते सुखं कृष्णः पुरस्त्रीजनवल्लभः
कच्चित्स्मरति वा बन्धून्पितरं मातरं च सः
अप्यसौ मातरं द्रष्टुं सकृदप्यागमिष्यति
अपि वा स्मरतेऽस्माकमनुसेवां महाभुजः
मातरं पितरं भ्रात्ऱ्‌^ईन्पतीन्पुत्रान्स्वस्ऱ्‌^ईनपि
यदर्थे जहिम दाशार्ह दुस्त्यजान्स्वजनान्प्रभो
ता नः सद्यः परित्यज्य गतः सञ्छिन्नसौहृदः
कथं नु तादृशं स्त्रीभिर्न श्रद्धीयेत भाषितम्‌
कथं नु गृह्णन्त्यनवस्थितात्मनो
वचः कृतघ्नस्य बुधाः पुरस्त्रियः
गृह्णन्ति वै चित्रकथस्य सुन्दर
स्मितावलोकोच्छ्वसितस्मरातुराः
किं नस्तत्कथया गोप्यः कथाः कथयतापराः
यात्यस्माभिर्विना कालो यदि तस्य तथैव नः
इति प्रहसितं शौरेर्जल्पितं चारुवीक्षितम्‌
गतिं प्रेमपरिष्वङ्गं स्मरन्त्यो रुरुदुः स्त्रियः
सङ्कर्षणस्ताः कृष्णस्य सन्देशैर्हृदयंगमैः
सान्त्वयामास भगवान्नानानुनयकोविदः
द्वौ मासौ तत्र चावात्सीन्मधुं माधवं एव च
रामः क्षपासु भगवान्गोपीनां रतिमावहन्‌
पूर्णचन्द्रकलामृष्टे कौमुदीगन्धवायुना
यमुनोपवने रेमे सेविते स्त्रीगणैर्वृतः
वरुणप्रेषिता देवी वारुणी वृक्षकोटरात्‌
पतन्ती तद्वनं सर्वं स्वगन्धेनाध्यवासयत्‌
तं गन्धं मधुधाराया वायुनोपहृतं बलः
आघ्रायोपगतस्तत्र ललनाभिः समं पपौ
उपगीयमानो गन्धर्वैर्वनिताशोभिमण्डले
रेमे करेणुयूथेशो माहेन्द्र इव वारणः
नेदुर्दुन्दुभयो व्योम्नि ववृषुः कुसुमैर्मुदा
गन्धर्वा मुनयो रामं तद्वीर्यैरीडिरे तदा
उपगीयमानचरितो वनिताभिर्हलायुध
वनेषु व्यचरत्क्षीवो मदविह्वललोचनः
स्रग्व्येककुण्डलो मत्तो वैजयन्त्या च मालया
बिभ्रत्स्मितमुखाम्भोजं स्वेदप्रालेयभूषितम्‌
स आजुहाव यमुनां जलक्रीडार्थमीश्वरः
निजं वाक्यमनादृत्य मत्त इत्यापगां बलः
अनागतां हलाग्रेण कुपितो विचकर्ष ह
पापे त्वं मामवज्ञाय यन्नायासि मयाहुता
नेष्ये त्वां लाङ्गलाग्रेण शतधा कामचारिणीम्‌
एवं निर्भर्त्सिता भीता यमुना यदुनन्दनम्‌
उवाच चकिता वाचं पतिता पादयोर्नृप
राम राम महाबाहो न जाने तव विक्रमम्‌
यस्यैकांशेन विधृता जगती जगतः पते
परं भावं भगवतो भगवन्मामजानतीम्‌
मोक्तुमर्हसि विश्वात्मन्प्रपन्नां भक्तवत्सल
ततो व्यमुञ्चद्यमुनां याचितो भगवान्बलः
विजगाह जलं स्त्रीभिः करेणुभिरिवेभराट्‌
कामं विहृत्य सलिलादुत्तीर्णायासीताम्बरे
भूषणानि महार्हाणि ददौ कान्तिः शुभां स्रजम्‌
वसित्वा वाससी नीले मालां आमुच्य काञ्चनीम्‌
रेये स्वलङ्कृतो लिप्तो माहेन्द्र इव वारणः
अद्यापि दृश्यते राजन्यमुनाकृष्टवर्त्मना
बलस्यानन्तवीर्यस्य वीर्यं सूचयतीव हि
एवं सर्वा निशा याता एकेव रमतो व्रजे
रामस्याक्षिप्तचित्तस्य माधुर्यैर्व्रजयोषिताम्‌
श्रीशुक उवाच
नन्दव्रजं गते रामे करूषाधिपतिर्नृप
वासुदेवोऽहमित्यज्ञो दूतं कृष्णाय प्राहिणोत्‌
त्वं वासुदेवो भगवानवतीऋनो जगत्पतिः
इति प्रस्तोभितो बालैर्मेन आत्मानमच्युतम्‌
दूतं च प्राहिणोन्मन्दः कृष्णायाव्यक्तवर्त्मने
द्वारकायां यथा बालो नृपो बालकृतोऽबुधः
दूतस्तु द्वारकामेत्य सभायामास्थितं प्रभुम्‌
कृष्णं कमलपत्राक्षं राजसन्देशमब्रवीत्‌
वासुदेवोऽवतीर्नोऽहमेक एव न चापरः
भूतानामनुकम्पार्थं त्वं तु मिथ्याभिधां त्यज
यानि त्वमस्मच्चिह्नानि मौढ्याद्बिभर्षि सात्वत
त्यक्त्वैहि मां त्वं शरणं नो चेद्देहि ममाहवम्‌
श्रीशुक उवाच
कत्थनं तदुपाकर्ण्य पौण्ड्रकस्याल्पमेधसः
उग्रसेनादयः सभ्या उच्चकैर्जहसुस्तदा
उवाच दूतं भगवान्परिहासकथामनु
उत्स्रक्ष्ये मूढ चिह्नानि यैस्त्वमेवं विकत्थसे
मुखं तदपिधायाज्ञ कङ्कगृध्रवटैर्वृतः
शयिष्यसे हतस्तत्र भविता शरणं शुनाम्‌
इति दूतस्तमाक्षेपं स्वामिने सर्वमाहरत्‌
कृष्णोऽपि रथमास्थाय काशीमुपजगाम ह
पौण्ड्रकोऽपि तदुद्योगमुपलभ्य महारथः
अक्षौहिणीभ्यां संयुक्तो निश्चक्राम पुराद्द्रुतम्‌
तस्य काशीपतिर्मित्रं पार्ष्णिग्राहोऽन्वयान्नृप
अक्षौहिणीभिस्तिसृभिरपश्यत्पौण्ड्रकं हरिः
शङ्खार्यसिगदाशार्ङ्ग श्रीवत्साद्युपलक्षितम्‌
बिभ्राणं कौस्तुभमणिं वनमालाविभूषितम्‌
कौशेयवाससी पीते वसानं गरुडध्वजम्‌
अमूल्यमौल्याभरणं स्फुरन्मकरकुण्डलम्‌
दृष्ट्वा तमात्मनस्तुल्यं वेषं कृत्रिममास्थितम्‌
यथा नटं रङ्गगतं विजहास भृशं हरीः
शुलैर्गदाभिः परिघैः शक्त्यृष्टिप्रासतोमरैः
असिभिः पट्टिशैर्बाणैः प्राहरन्नरयो हरिम्‌
कृष्णस्तु तत्पौण्ड्रककाशिराजयोर्‌
बलं गजस्यन्दनवाजिपत्तिमत्‌
गदासिचक्रेषुभिरार्दयद्भृशं
यथा युगान्ते हुतभुक्पृथक्प्रजाः
आयोधनं तद्रथवाजिकुञ्जर द्विपत्खरोष्ट्रैररिणावखण्डितैः
बभौ चितं मोदवहं मनस्विनामाक्रीडनं भूतपतेरिवोल्बणम्‌
अथाह पौण्ड्रकं शौरिर्भो भो पौण्ड्रक यद्भवान्‌
दूतवाक्येन मामाह तान्यस्त्रण्युत्सृजामि ते
त्याजयिष्येऽभिधानं मे यत्त्वयाज्ञ मृषा धृतम्‌
व्रजामि शरनं तेऽद्य यदि नेच्छामि संयुगम्‌
इति क्षिप्त्वा शितैर्बाणैर्विरथीकृत्य पौण्ड्रकम्‌
शिरोऽवृश्चद्रथाङ्गेन वज्रेणेन्द्रो यथा गिरेः
तथा काशीपतेः कायाच्छिर उत्कृत्य पत्रिभिः
न्यपातयत्काशीपुर्यां पद्मकोशमिवानिलः
एवं मत्सरिणम्हत्वा पौण्ड्रकं ससखं हरिः
द्वारकामाविशत्सिद्धैर्गीयमानकथामृतः
स नित्यं भगवद्ध्यान प्रध्वस्ताखिलबन्धनः
बिभ्राणश्च हरे राजन्स्वरूपं तन्मयोऽभवत्‌
शिरः पतितमालोक्य राजद्वारे सकुण्डलम्‌
किमिदं कस्य वा वक्त्रमिति संशिशिरे जनाः
राज्ञः काशीपतेर्ज्ञात्वा महिष्यः पुत्रबान्धवाः
पौराश्च हा हता राजन्नाथ नाथेति प्रारुदन्‌
सुदक्षिणस्तस्य सुतः कृत्वा संस्थाविधिं पतेः
निहत्य पितृहन्तारं यास्याम्यपचितिं पितुः
इत्यात्मनाभिसन्धाय सोपाध्यायो महेश्वरम्‌
सुदक्षिणोऽर्चयामास परमेण समाधिना
प्रीतोऽविमुक्ते भगवांस्तस्मै वरमदाद्विभुः
पितृहन्तृवधोपायं स वव्रे वरमीप्सितम्‌
दक्षिणाग्निं परिचर ब्राह्मणैः सममृत्विजम्‌
अभिचारविधानेन स चाग्निः प्रमथैर्वृतः
साधयिष्यति सङ्कल्पमब्रह्मण्ये प्रयोजितः
इत्यादिष्टस्तथा चक्रे कृष्णायाभिचरन्व्रती
ततोऽग्निरुत्थितः कुण्डान्मूर्तिमानतिभीषणः
तप्तताम्रशिखाश्मश्रुरङ्गारोद्गारिलोचनः
दंष्ट्रोग्रभ्रुकुटीदण्ड कठोरास्यः स्वजिह्वया
आलिहन्सृक्वणी नग्नो विधुन्वंस्त्रिशिखं ज्वलत्‌
पद्भ्यां तालप्रमाणाभ्यां कम्पयन्नवनीतलम्‌
सोऽभ्यधावद्वृतो भूतैर्द्वारकां प्रदहन्दिशः
तमाभिचारदहनमायान्तं द्वारकौकसः
विलोक्य तत्रसुः सर्वे वनदाहे मृगा यथा
अक्षैः सभायां क्रीडन्तं भगवन्तं भयातुराः
त्राहि त्राहि त्रिलोकेश वह्नेः प्रदहतः पुरम्‌
श्रुत्वा तज्जनवैक्लव्यं दृष्ट्वा स्वानां च साध्वसम्‌
शरण्यः सम्प्रहस्याह मा भैष्टेत्यवितास्म्यहम्‌
सर्वस्यान्तर्बहिःसाक्षी कृत्यां माहेश्वरीं विभुः
विज्ञाय तद्विघातार्थं पार्श्वस्थं चक्रमादिशत्‌
तत्सूर्यकोटिप्रतिमं सुदर्शनं जाज्वल्यमानं प्रलयानलप्रभम्‌
स्वतेजसा खं ककुभोऽथ रोदसी चक्रं मुकुन्दास्त्रं अथाग्निमार्दयत्‌
कृत्यानलः प्रतिहतः स रथान्गपाणेर्‌
अस्त्रौजसा स नृप भग्नमुखो निवृत्तः
वाराणसीं परिसमेत्य सुदक्षिणं तं
सर्त्विग्जनं समदहत्स्वकृतोऽभिचारः
चक्रं च विष्णोस्तदनुप्रविष्टं वारानसीं साट्टसभालयापणाम्‌
सगोपुराट्टालककोष्ठसङ्कुलां सकोशहस्त्यश्वरथान्नशालिनीम्‌
दग्ध्वा वाराणसीं सर्वां विष्णोश्चक्रं सुदर्शनम्‌
भूयः पार्श्वमुपातिष्ठत्कृष्णस्याक्लिष्टकर्मणः
य एनं श्रावयेन्मर्त्य उत्तमःश्लोकविक्रमम्‌
समाहितो वा शृणुयात्सर्वपापैः प्रमुच्यते
श्रीराजोवाच
भुयोऽहं श्रोतुमिच्छामि रामस्याद्भुतकर्मणः
अनन्तस्याप्रमेयस्य यदन्यत्कृतवान्प्रभुः
श्रीशुक उवाच
नरकस्य सखा कश्चिद्द्विविदो नाम वानरः
सुग्रीवसचिवः सोऽथ भ्राता मैन्दस्य वीर्यवान्‌
सख्युः सोऽपचितिं कुर्वन्वानरो राष्ट्रविप्लवम्‌
पुरग्रामाकरान्घोषानदहद्वह्निमुत्सृजन्‌
क्वचित्स शैलानुत्पाट्य तैर्देशान्समचूर्णयत्‌
आनर्तान्सुतरामेव यत्रास्ते मित्रहा हरिः
क्वचित्समुद्रमध्यस्थो दोर्भ्यामुत्क्षिप्य तज्जलम्‌
देशान्नागायुतप्राणो वेलाकूले न्यमज्जयत्‌
आश्रमानृषिमुख्यानां कृत्वा भग्नवनस्पतीन्‌
अदूषयच्छकृन्मूत्रैरग्नीन्वैतानिकान्खलः
पुरुषान्योषितो दृप्तः क्ष्माभृद्द्रोनीगुहासु सः
निक्षिप्य चाप्यधाच्छैलैः पेशष्कारीव कीटकम्‌
एवं देशान्विप्रकुर्वन्दूषयंश्च कुलस्त्रियः
श्रुत्वा सुललितं गीतं गिरिं रैवतकं ययौ
तत्रापश्यद्यदुपतिं रामं पुष्करमालिनम्‌
सुदर्शनीयसर्वाङ्गं ललनायूथमध्यगम्‌
गायन्तं वारुणीं पीत्वा मदविह्वललोचनम्‌
विभ्राजमानं वपुषा प्रभिन्नमिव वारणम्‌
दुष्टः शाखामृगः शाखामारूढः कम्पयन्द्रुमान्‌
चक्रे किलकिलाशब्दमात्मानं सम्प्रदर्शयन्‌
तस्य धार्ष्ट्यं कपेर्वीक्ष्य तरुण्यो जातिचापलाः
हास्यप्रिया विजहसुर्बलदेवपरिग्रहाः
ता हेलयामास कपिर्भ्रूक्षेपैर्सम्मुखादिभिः
दर्शयन्स्वगुदं तासां रामस्य च निरीक्षितः
तं ग्राव्णा प्राहरत्क्रुद्धो बलः प्रहरतां वरः
स वञ्चयित्वा ग्रावाणं मदिराकलशं कपिः
गृहीत्वा हेलयामास धूर्तस्तं कोपयन्हसन्‌
निर्भिद्य कलशं दुष्टो वासांस्यास्फालयद्बलम्‌
कदर्थीकृत्य बलवान्विप्रचक्रे मदोद्धतः
तं तस्याविनयं दृष्ट्वा देशांश्च तदुपद्रुतान्‌
क्रुद्धो मुषलमादत्त हलं चारिजिघांसया
द्विविदोऽपि महावीर्यः शालमुद्यम्य पाणिना
अभ्येत्य तरसा तेन बलं मूर्धन्यताडयत्‌
तं तु सङ्कर्षणो मूर्ध्नि पतन्तमचलो यथा
प्रतिजग्राह बलवान्सुनन्देनाहनच्च तम्‌
मूषलाहतमस्तिष्को विरेजे रक्तधारया
गिरिर्यथा गैरिकया प्रहारं नानुचिन्तयन्‌
पुनरन्यं समुत्क्षिप्य कृत्वा निष्पत्रमोजसा
तेनाहनत्सुसङ्क्रुद्धस्तं बलः शतधाच्छिनत्‌
ततोऽन्येन रुषा जघ्ने तं चापि शतधाच्छिनत्‌
एवं युध्यन्भगवता भग्ने भग्ने पुनः पुनः
आकृष्य सर्वतो वृक्षान्निर्वृक्षमकरोद्वनम्‌
ततोऽमुञ्चच्छिलावर्षं बलस्योपर्यमर्षितः
तत्सर्वं चूर्णयां आस लीलया मुषलायुधः
स बाहू तालसङ्काशौ मुष्टीकृत्य कपीश्वरः
आसाद्य रोहिणीपुत्रं ताभ्यां वक्षस्यरूरुजत्‌
यादवेन्द्रोऽपि तं दोर्भ्यां त्यक्त्वा मुषललाङ्गले
जत्रावभ्यर्दयत्क्रुद्धः सोऽपतद्रुधिरं वमन्‌
चकम्पे तेन पतता सटङ्कः सवनस्पतिः
पर्वतः कुरुशार्दूल वायुना नौरिवाम्भसि
जयशब्दो नमःशब्दः साधु साध्विति चाम्बरे
सुरसिद्धमुनीन्द्राणामासीत्कुसुमवर्षिणाम्‌
एवं निहत्य द्विविदं जगद्व्यतिकरावहम्‌
संस्तूयमानो भगवान्जनैः स्वपुरमाविशत्‌
श्रीशुक उवाच
दुर्योधनसुतां राजन्लक्ष्मणां समितिंजयः
स्वयंवरस्थामहरत्साम्बो जाम्बवतीसुतः
कौरवाः कुपिता ऊचुर्दुर्विनीतोऽयमर्भकः
कदर्थीकृत्य नः कन्यामकामामहरद्बलात्‌
बध्नीतेमं दुर्विनीतं किं करिष्यन्ति वृष्णयः
येऽस्मत्प्रसादोपचितां दत्तां नो भुञ्जते महीम्‌
निगृहीतं सुतं श्रुत्वा यद्येष्यन्तीह वृष्णयः
भग्नदर्पाः शमं यान्ति प्राणा इव सुसंयताः
इति कर्णः शलो भूरिर्यज्ञकेतुः सुयोधनः
साम्बमारेभिरे योद्धुं कुरुवृद्धानुमोदिताः
दृष्ट्वानुधावतः साम्बो धार्तराष्ट्रान्महारथः
प्रगृह्य रुचिरं चापं तस्थौ सिंह इवैकलः
तं ते जिघृक्षवः क्रुद्धास्तिष्ठ तिष्ठेति भाषिणः
आसाद्य धन्विनो बाणैः कर्णाग्रण्यः समाकिरन्‌
सोऽपविद्धः कुरुश्रेष्ठ कुरुभिर्यदुनन्दनः
नामृष्यत्तदचिन्त्यार्भः सिंह क्षुद्रमृगैरिव
विस्फूर्ज्य रुचिरं चापं सर्वान्विव्याध सायकैः
कर्णादीन्षड्रथान्वीरस्तावद्भिर्युगपत्पृथक्‌
चतुर्भिश्चतुरो वाहानेकैकेन च सारथीन्‌
रथिनश्च महेष्वासांस्तस्य तत्तेऽभ्यपूजयन्‌
तं तु ते विरथं चक्रुश्चत्वारश्चतुरो हयान्‌
एकस्तु सारथिं जघ्ने चिच्छेदण्यः शरासनम्‌
तं बद्ध्वा विरथीकृत्य कृच्छ्रेण कुरवो युधि
कुमारं स्वस्य कन्यां च स्वपुरं जयिनोऽविशन्‌
तच्छ्रुत्वा नारदोक्तेन राजन्सञ्जातमन्यवः
कुरून्प्रत्युद्यमं चक्रुरुग्रसेनप्रचोदिताः
सान्त्वयित्वा तु तान्रामः सन्नद्धान्वृष्णिपुङ्गवान्‌
नैच्छत्कुरूणां वृष्णीनां कलिं कलिमलापहः
जगाम हास्तिनपुरं रथेनादित्यवर्चसा
ब्राह्मणैः कुलवृद्धैश्च वृतश्चन्द्र इव ग्रहैः
गत्वा गजाह्वयं रामो बाह्योपवनमास्थितः
उद्धवं प्रेषयामास धृतराष्ट्रं बुभुत्सया
सोऽभिवन्द्याम्बिकापुत्रं भीष्मं द्रोणं च बाह्लिकम्‌
दुर्योधनं च विधिवद्राममागतं अब्रवीत्‌
तेऽतिप्रीतास्तमाकर्ण्य प्राप्तं रामं सुहृत्तमम्‌
तमर्चयित्वाभिययुः सर्वे मङ्गलपाणयः
तं सङ्गम्य यथान्यायं गामर्घ्यं च न्यवेदयन्‌
तेषां ये तत्प्रभावज्ञाः प्रणेमुः शिरसा बलम्‌
बन्धून्कुशलिनः श्रुत्वा पृष्ट्वा शिवमनामयम्‌
परस्परमथो रामो बभाषेऽविक्लवं वचः
उग्रसेनः क्षितेशेशो यद्व आज्ञापयत्प्रभुः
तदव्यग्रधियः श्रुत्वा कुरुध्वमविलम्बितम्‌
यद्यूयं बहवस्त्वेकं जित्वाधर्मेण धार्मिकम्‌
अबध्नीताथ तन्मृष्ये बन्धूनामैक्यकाम्यया
वीर्यशौर्यबलोन्नद्धमात्मशक्तिसमं वचः
कुरवो बलदेवस्य निशम्योचुः प्रकोपिताः
अहो महच्चित्रमिदं कालगत्या दुरत्यया
आरुरुक्षत्युपानद्वै शिरो मुकुटसेवितम्‌
एते यौनेन सम्बद्धाः सहशय्यासनाशनाः
वृष्णयस्तुल्यतां नीता अस्मद्दत्तनृपासनाः
चामरव्यजने शङ्खमातपत्रं च पाण्डुरम्‌
किरीटमासनं शय्यां भुञ्जतेऽस्मदुपेक्षया
अलं यदूनां नरदेवलाञ्छनैर्दातुः प्रतीपैः फणिनामिवामृतम्‌
येऽस्मत्प्रसादोपचिता हि यादवा आज्ञापयन्त्यद्य गतत्रपा बत
कथमिन्द्रोऽपि कुरुभिर्भीष्मद्रोणार्जुनादिभिः
अदत्तमवरुन्धीत सिंहग्रस्तमिवोरणः
श्रीबादरायणिरुवाच
जन्मबन्धुश्रीयोन्नद्ध मदास्ते भरतर्षभ
आश्राव्य रामं दुर्वाच्यमसभ्याः पुरमाविशन्‌
दृष्ट्वा कुरूनां दौःशील्यं श्रुत्वावाच्यानि चाच्युतः
अवोचत्कोपसंरब्धो दुष्प्रेक्ष्यः प्रहसन्मुहुः
नूनं नानामदोन्नद्धाः शान्तिं नेच्छन्त्यसाधवः
तेषां हि प्रशमो दण्डः पशूनां लगुडो यथा
अहो यदून्सुसंरब्धान्कृष्णं च कुपितं शनैः
सान्त्वयित्वाहमेतेषां शममिच्छन्निहागतः
त इमे मन्दमतयः कलहाभिरताः खलाः
तं मामवज्ञाय मुहुर्दुर्भाषान्मानिनोऽब्रुवन्‌
नोग्रसेनः किल विभुर्भोजवृष्ण्यन्धकेश्वरः
शक्रादयो लोकपाला यस्यादेशानुवर्तिनः
सुधर्माक्रम्यते येन पारिजातोऽमराङ्घ्रिपः
आनीय भुज्यते सोऽसौ न किलाध्यासनार्हणः
यस्य पादयुगं साक्षाच्छ्रीरुपास्तेऽखिलेश्वरी
स नार्हति किल श्रीशो नरदेवपरिच्छदान्‌
यस्याङ्घ्रिपङ्कजरजोऽखिललोकपालैर्‌
मौल्युत्तमैर्धृतमुपासिततीर्थतीर्थम्‌
ब्रह्मा भवोऽहमपि यस्य कलाः कलायाः
श्रीश्चोद्वहेम चिरमस्य नृपासनं क्व
भुञ्जते कुरुभिर्दत्तं भूखण्डं वृष्णयः किल
उपानहः किल वयं स्वयं तु कुरवः शिरः
अहो ऐश्वर्यमत्तानां मत्तानामिव मानिनाम्‌
असम्बद्धा गिऋओ रुक्षाः कः सहेतानुशासीता
अद्य निष्कौरवं पृथ्वीं करिष्यामीत्यमर्षितः
गृहीत्वा हलमुत्तस्थौ दहन्निव जगत्त्रयम्‌
लाङ्गलाग्रेण नगरमुद्विदार्य गजाह्वयम्‌
विचकर्ष स गङ्गायां प्रहरिष्यन्नमर्षितः
जलयानमिवाघूर्णं गङ्गायां नगरं पतत्‌
आकृष्यमाणमालोक्य कौरवाः जातसम्भ्रमाः
तमेव शरणं जग्मुः सकुटुम्बा जिजीविषवः
सलक्ष्मणं पुरस्कृत्य साम्बं प्राञ्जलयः प्रभुम्‌
राम रामाखिलाधार प्रभावं न विदाम ते
मूढानां नः कुबुद्धीनां क्षन्तुमर्हस्यतिक्रमम्‌
स्थित्युत्पत्त्यप्ययानां त्वमेको हेतुर्निराश्रयः
लोकान्क्रीडनकानीश क्रीडतस्ते वदन्ति हि
त्वमेव मूर्ध्नीदमनन्त लीलया भूमण्डलं बिभर्षि सहस्रमूर्धन्‌
अन्ते च यः स्वात्मनिरुद्धविश्वः शेषेऽद्वितीयः परिशिष्यमाणः
कोपस्तेऽखिलशिक्षार्थं न द्वेषान्न च मत्सरात्‌
बिभ्रतो भगवन्सत्त्वं स्थितिपालनतत्परः
नमस्ते सर्वभूतात्मन्सर्वशक्तिधराव्यय
विश्वकर्मन्नमस्तेऽस्तु त्वां वयं शरणं गताः
श्रीशुक उवाच
एवं प्रपन्नैः संविग्नैर्वेपमानायनैर्बलः
प्रसादितः सुप्रसन्नो मा भैष्टेत्यभयं ददौ
दुर्योधनः पारिबर्हं कुञ्जरान्षष्टिहायनान्‌
ददौ च द्वादशशतान्ययुतानि तुरङ्गमान्‌
रथानां षट्सहस्राणि रौक्माणां सूर्यवर्चसाम्‌
दासीनां निष्ककण्ठीनां सहस्रं दुहितृवत्सलः
प्रतिगृह्य तु तत्सर्वं भगवान्सात्वतर्षभः
ससुतः सस्नुषः प्रायात्सुहृद्भिरभिनन्दितः
ततः प्रविष्टः स्वपुरं हलायुधः
समेत्य बन्धूननुरक्तचेतसः
शशंस सर्वं यदुपुङ्गवानां
मध्ये सभायां कुरुषु स्वचेष्टितम्‌
अद्यापि च पुरं ह्येतत्सूचयद्रामविक्रमम्‌
समुन्नतं दक्षिणतो गङ्गायामनुदृश्यते
श्रीशुक उवाच
नरकं निहतं श्रुत्वा तथोद्वाहं च योषिताम्‌
कृष्णेनैकेन बह्वीनां तद्दिदृक्षुः स्म नारदः
चित्रं बतैतदेकेन वपुषा युगपत्पृथक्‌
गृहेषु द्व्यष्टसाहस्रं स्त्रिय एक उदावहत्‌
इत्युत्सुको द्वारवतीं देवर्षिर्द्रष्टुमागमत्‌
पुष्पितोपवनाराम द्विजालिकुलनादिताम्‌
उत्फुल्लेन्दीवराम्भोज कह्लारकुमुदोत्पलैः
छुरितेषु सरःसूच्चैः कूजितां हंससारसैः
प्रासादलक्षैर्नवभिर्जुष्टां स्फाटिकराजतैः
महामरकतप्रख्यैः स्वर्णरत्नपरिच्छदैः
विभक्तरथ्यापथचत्वरापणैः शालासभाभी रुचिरां सुरालयैः
संसिक्तमार्गाङ्गनवीथिदेहलीं पतत्पताकध्वजवारितातपाम्‌
तस्यामन्तःपुरं श्रीमदर्चितं सर्वधिष्ण्यपैः
हरेः स्वकौशलं यत्र त्वष्ट्रा कार्त्स्न्येन दर्शितम्‌
तत्र षोडशभिः सद्म सहस्रैः समलङ्कृतम्‌
विवेशैकतोमं शौरेः पत्नीनां भवनं महत्‌
विष्टब्धं विद्रुमस्तम्भैर्वैदूर्यफलकोत्तमैः
इन्द्रनीलमयैः कुड्यैर्जगत्या चाहतत्विषा
वितानैर्निर्मितैस्त्वष्ट्रा मुक्तादामविलम्बिभिः
दान्तैरासनपर्यङ्कैर्मण्युत्तमपरिष्कृतैः
दासीभिर्निष्ककण्ठीभिः सुवासोभिरलङ्कृतम्‌
पुम्भिः सकञ्चुकोष्णीष सुवस्त्रमणिकुण्डलैः
रत्नप्रदीपनिकरद्युतिभिर्निरस्त ध्वान्तं विचित्रवलभीषु शिखण्डिनोऽङ्ग
नृत्यन्ति यत्र विहितागुरुधूपमक्षैर्निर्यान्तमीक्ष्य घनबुद्धय उन्नदन्तः
तस्मिन्समानगुणरूपवयःसुवेष
दासीसहस्रयुतयानुसवं गृहिण्या
विप्रो ददर्श चमरव्यजनेन रुक्म
दण्डेन सात्वतपतिं परिवीजयन्त्या
तं सन्निरीक्ष्य भगवान्सहसोत्थितश्री
पर्यङ्कतः सकलधर्मभृतां वरिष्ठः
आनम्य पादयुगलं शिरसा किरीट
जुष्टेन साञ्जलिरवीविशदासने स्वे
तस्यावनिज्य चरणौ तदपः स्वमूर्ध्ना
बिभ्रज्जगद्गुरुतमोऽपि सतां पतिर्हि
ब्रह्मण्यदेव इति यद्गुणनाम युक्तं
तस्यैव यच्चरणशौचमशेषतीर्थम्‌
सम्पूज्य देवऋषिवर्यमृषिः पुराणो
नारायणो नरसखो विधिनोदितेन
वाण्याभिभाष्य मितयामृतमिष्टया तं
प्राह प्रभो भगवते करवाम हे किम्‌
श्रीनारद उवाच
नैवाद्भुतं त्वयि विभोऽखिललोकनाथे
मैत्री जनेषु सकलेषु दमः खलानाम्‌
निःश्रेयसाय हि जगत्स्थितिरक्षणाभ्यां
स्वैरावतार उरुगाय विदाम सुष्ठु
दृष्टं तवाङ्घ्रियुगलं जनतापवर्गं
ब्रह्मादिभिर्हृदि विचिन्त्यमगाधबोधैः
संसारकूपपतितोत्तरणावलम्बं
ध्यायंश्चराम्यनुगृहाण यथा स्मृतिः स्यात्‌
ततोऽन्यदाविशद्गेहं कृष्णपत्न्याः स नारदः
योगेश्वरेश्वरस्याङ्ग योगमायाविवित्सया
दीव्यन्तमक्षैस्तत्रापि प्रियया चोद्धवेन च
पूजितः परया भक्त्या प्रत्युत्थानासनादिभिः
पृष्टश्चाविदुषेवासौ कदायातो भवानिति
क्रियते किं नु पूर्णानामपूर्णैरस्मदादिभिः
अथापि ब्रूहि नो ब्रह्मन्जन्मैतच्छोभनं कुरु
स तु विस्मित उत्थाय तूष्णीमन्यदगाद्गृहम्‌
तत्राप्यचष्ट गोविन्दं लालयन्तं सुतान्शिशून्‌
ततोऽन्यस्मिन्गृहेऽपश्यन्मज्जनाय कृतोद्यमम्‌
जुह्वन्तं च वितानाग्नीन्यजन्तं पञ्चभिर्मखैः
भोजयन्तं द्विजान्क्वापि भुञ्जानमवशेषितम्‌
क्वापि सन्ध्यामुपासीनं जपन्तं ब्रह्म वाग्यतम्‌
एकत्र चासिचर्माभ्यां चरन्तमसिवर्त्मसु
अश्वैर्गजै रथैः क्वापि विचरन्तं गदाग्रजम्‌
क्वचिच्छयानं पर्यङ्के स्तूयमानं च वन्दिभिः
मन्त्रयन्तं च कस्मिंश्चिन्मन्त्रिभिश्चोद्धवादिभिः
जलक्रीडारतं क्वापि वारमुख्याबलावृतम्‌
कुत्रचिद्द्विजमुख्येभ्यो ददतं गाः स्वलङ्कृताः
इतिहासपुराणानि शृण्वन्तं मङ्गलानि च
हसन्तं हासकथया कदाचित्प्रियया गृहे
क्वापि धर्मं सेवमानमर्थकामौ च कुत्रचित्‌
ध्यायन्तमेकमासीनं पुरुषं प्रकृतेः परम्‌
शुश्रूषन्तं गुरून्क्वापि कामैर्भोगैः सपर्यया
कुर्वन्तं विग्रहं कैश्चित्सन्धिं चान्यत्र केशवम्‌
कुत्रापि सह रामेण चिन्तयन्तं सतां शिवम्‌
पुत्राणां दुहित्ऱ्‌^ईणां च काले विध्युपयापनम्‌
दारैर्वरैस्तत्सदृशैः कल्पयन्तं विभूतिभिः
प्रस्थापनोपनयनैरपत्यानां महोत्सवान्‌
वीक्ष्य योगेश्वरेशस्य येषां लोका विसिस्मिरे
यजन्तं सकलान्देवान्क्वापि क्रतुभिरूर्जितैः
पूर्तयन्तं क्वचिद्धर्मं कूर्पाराममठादिभिः
चरन्तं मृगयां क्वापि हयमारुह्य सैन्धवम्‌
घ्नन्तं तत्र पशून्मेध्यान्परीतं यदुपुङ्गवैः
अव्यक्तलिन्गं प्रकृतिष्वन्तःपुरगृहादिषु
क्वचिच्चरन्तं योगेशं तत्तद्भावबुभुत्सया
अथोवाच हृषीकेशं नारदः प्रहसन्निव
योगमायोदयं वीक्ष्य मानुषीमीयुषो गतिम्‌
विदाम योगमायास्ते दुर्दर्शा अपि मायिनाम्‌
योगेश्वरात्मन्निर्भाता भवत्पादनिषेवया
अनुजानीहि मां देव लोकांस्ते यशसाप्लुतान्‌
पर्यटामि तवोद्गायन्लीला भुवनपावनीः
श्रीभगवानुवाच
ब्रह्मन्धन्नस्य वक्ताहं कर्ता तदनुमोदिता
तच्छिक्षयन्लोकमिममास्थितः पुत्र मा खिदः
श्रीशुक उवाच
इत्याचरन्तं सद्धर्मान्पावनान्गृहमेधिनाम्‌
तमेव सर्वगेहेषु सन्तमेकं ददर्श ह
कृष्णस्यानन्तवीर्यस्य योगमायामहोदयम्‌
मुहुर्दृष्ट्वा ऋषिरभूद्विस्मितो जातकौतुकः
इत्यर्थकामधर्मेषु कृष्णेन श्रद्धितात्मना
सम्यक्सभाजितः प्रीतस्तमेवानुस्मरन्ययौ
एवं मनुष्यपदवीमनुवर्तमानो नारायणोऽखिलभवाय गृहीतशक्तिः
रेमेऽण्ग षोडशसहस्रवराङ्गनानां सव्रीडसौहृदनिरीक्षणहासजुष्टः
यानीह विश्वविलयोद्भववृत्तिहेतुः
कर्माण्यनन्यविषयाणि हरीश्चकार
यस्त्वङ्ग गायति शृणोत्यनुमोदते वा
भक्तिर्भवेद्भगवति ह्यपवर्गमार्गे
श्रीशुक उवाच
अथोषस्युपवृत्तायां कुक्कुटान्कूजतोऽशपन्‌
गृहीतकण्ठ्यः पतिभिर्माधव्यो विरहातुराः
वयांस्यरोरुवन्कृष्णं बोधयन्तीव वन्दिनः
गायत्स्वलिष्वनिद्राणि मन्दारवनवायुभिः
मुहूर्तं तं तु वैदर्भी नामृष्यदतिशोभनम्‌
परिरम्भणविश्लेषात्प्रियबाह्वन्तरं गता
ब्राह्मे मुहूर्त उत्थाय वार्युपस्पृश्य माधवः
दध्यौ प्रसन्नकरण आत्मानं तमसः परम्‌
एकं स्वयंज्योतिरनन्यमव्ययं स्वसंस्थया नित्यनिरस्तकल्मषम्‌
ब्रह्माख्यमस्योद्भवनाशहेतुभिः स्वशक्तिभिर्लक्षितभावनिर्वृतिम्‌
अथाप्लुतोऽम्भस्यमले यथाविधि
क्रियाकलापं परिधाय वाससी
चकार सन्ध्योपगमादि सत्तमो
हुतानलो ब्रह्म जजाप वाग्यतः
उपस्थायार्कमुद्यन्तं तर्पयित्वात्मनः कलाः
देवानृषीन्पित्ऱ्‌^ईन्वृद्धान्विप्रानभ्यर्च्य चात्मवान्‌
धेनूनां रुक्मशृङ्गीनां साध्वीनां मौक्तिकस्रजाम्‌
पयस्विनीनां गृष्टीनां सवत्सानां सुवाससाम्‌
ददौ रूप्यखुराग्राणां क्षौमाजिनतिलैः सह
अलङ्कृतेभ्यो विप्रेभ्यो बद्वं बद्वं दिने दिने
गोविप्रदेवतावृद्ध गुरून्भूतानि सर्वशः
नमस्कृत्यात्मसम्भूतीर्मङ्गलानि समस्पृशत्‌
आत्मानं भूषयामास नरलोकविभूषणम्‌
वासोभिर्भूषणैः स्वीयैर्दिव्यस्रगनुलेपनैः
अवेक्ष्याज्यं तथादर्शं गोवृषद्विजदेवताः
कामांश्च सर्ववर्णानां पौरान्तःपुरचारिणाम्‌
प्रदाप्य प्रकृतीः कामैः प्रतोष्य प्रत्यनन्दत
संविभज्याग्रतो विप्रान्स्रक्ताम्बूलानुलेपनैः
सुहृदः प्रकृतीर्दारानुपायुङ्क्त ततः स्वयम्‌
तावत्सूत उपानीय स्यन्दनं परमाद्भुतम्‌
सुग्रीवाद्यैर्हयैर्युक्तं प्रणम्यावस्थितोऽग्रतः
गृहीत्वा पाणिना पाणी सारथेस्तमथारुहत्‌
सात्यक्युद्धवसंयुक्तः पूर्वाद्रिमिव भास्करः
ईक्षितोऽन्तःपुरस्त्रीणां सव्रीडप्रेमवीक्षितैः
कृच्छ्राद्विसृष्टो निरगाज्जातहासो हरन्मनः
सुधर्माख्यां सभां सर्वैर्वृष्णिभिः परिवारितः
प्राविशद्यन्निविष्टानां न सन्त्यङ्ग षडूर्मयः
तत्रोपविस्तः परमासने विभुर्बभौ स्वभासा ककुभोऽवभासयन्‌
वृतो नृसिंहैर्यदुभिर्यदूत्तमो यथोडुराजो दिवि तारकागणैः
तत्रोपमन्त्रिणो राजन्नानाहास्यरसैर्विभुम्‌
उपतस्थुर्नटाचार्या नर्तक्यस्ताण्डवैः पृथक्‌
मृदङ्गवीणामुरज वेणुतालदरस्वनैः
ननृतुर्जगुस्तुष्टुवुश्च सूतमागधवन्दिनः
तत्राहुर्ब्राह्मणाः केचिदासीना ब्रह्मवादिनः
पूर्वेषां पुण्ययशसां राज्ञां चाकथयन्कथाः
तत्रैकः पुरुषो राजन्नागतोऽपूर्वदर्शनः
विज्ञापितो भगवते प्रतीहारैः प्रवेशितः
स नमस्कृत्य कृष्णाय परेशाय कृताञ्जलिः
राज्ञामावेदयद्दुःखं जरासन्धनिरोधजम्‌
ये च दिग्विजये तस्य सन्नतिं न ययुर्नृपाः
प्रसह्य रुद्धास्तेनासन्नयुते द्वे गिरिव्रजे
राजान ऊचुः
कृष्ण कृष्णाप्रमेयात्मन्प्रपन्नभयभञ्जन
वयं त्वां शरणं यामो भवभीताः पृथग्धियः
लोको विकर्मनिरतः कुशले प्रमत्तः
कर्मण्ययं त्वदुदिते भवदर्चने स्वे
यस्तावदस्य बलवानिह जीविताशां
सद्यश्छिनत्त्यनिमिषाय नमोऽस्तु तस्मै
लोके भवाञ्जगदिनः कलयावतीर्णः
सद्रक्षणाय खलनिग्रहणाय चान्यः
कश्चित्त्वदीयमतियाति निदेशमीश
किं वा जनः स्वकृतमृच्छति तन्न विद्मः
स्वप्नायितं नृपसुखं परतन्त्रमीश
शश्वद्भयेन मृतकेन धुरं वहामः
हित्वा तदात्मनि सुखं त्वदनीहलभ्यं
क्लिश्यामहेऽतिकृपणास्तव माययेह
तन्नो भवान्प्रणतशोकहराङ्घ्रियुग्मो
बद्धान्वियुङ्क्ष्व मगधाह्वयकर्मपाशात्‌
यो भूभुजोऽयुतमतङ्गजवीर्यमेको
बिभ्रद्रुरोध भवने मृगराडिवावीः
यो वै त्वया द्विनवकृत्व उदात्तचक्र
भग्नो मृधे खलु भवन्तमनन्तवीर्यम्‌
जित्वा नृलोकनिरतं सकृदूढदर्पो
युष्मत्प्रजा रुजति नोऽजित तद्विधेहि
दूत उवाच
इति मागधसंरुद्धा भवद्दर्शनकङ्क्षिणः
प्रपन्नाः पादमूलं ते दीनानां शं विधीयताम्‌
श्रीशुक उवाच
राजदूते ब्रुवत्येवं देवर्षिः परमद्युतिः
बिभ्रत्पिङ्गजटाभारं प्रादुरासीद्यथा रविः
तं दृष्ट्वा भगवान्कृष्णः सर्वलोकेश्वरेश्वरः
ववन्द उत्थितः शीर्ष्णा ससभ्यः सानुगो मुदा
सभाजयित्वा विधिवत्कृतासनपरिग्रहम्‌
बभाषे सुनृतैर्वाक्यैः श्रद्धया तर्पयन्मुनिम्‌
अपि स्विदद्य लोकानां त्रयाणामकुतोभयम्‌
ननु भूयान्भगवतो लोकान्पर्यटतो गुणः
न हि तेऽविदितं किञ्चिल्लोकेष्वीश्वरकर्तृषु
अथ पृच्छामहे युष्मान्पाण्डवानां चिकीर्षितम्‌
श्रीनारद उवाच
दृष्टा माया ते बहुशो दुरत्यया माया विभो विश्वसृजश्च मायिनः
भूतेषु भूमंश्चरतः स्वशक्तिभिर्वह्नेरिव च्छन्नरुचो न मेऽद्भुतम्‌
तवेहितं कोऽर्हति साधु वेदितुं स्वमाययेदं सृजतो नियच्छतः
यद्विद्यमानात्मतयावभासते तस्मै नमस्ते स्वविलक्षणात्मने
जीवस्य यः संसरतो विमोक्षणं न जानतोऽनर्थवहाच्छरीरतः
लीलावतारैः स्वयशः प्रदीपकं प्राज्वालयत्त्वा तमहं प्रपद्ये
अथाप्याश्रावये ब्रह्म नरलोकविडम्बनम्‌
राज्ञः पैतृष्वस्रेयस्य भक्तस्य च चिकीर्षितम्‌
यक्ष्यति त्वां मखेन्द्रेण राजसूयेन पाण्डवः
पारमेष्ठ्यकामो नृपतिस्तद्भवाननुमोदताम्‌
तस्मिन्देव क्रतुवरे भवन्तं वै सुरादयः
दिदृक्षवः समेष्यन्ति राजानश्च यशस्विनः
श्रवणात्कीर्तनाद्ध्यानात्पूयन्तेऽन्तेवसायिनः
तव ब्रह्ममयस्येश किमुतेक्षाभिमर्शिनः
यस्यामलं दिवि यशः प्रथितं रसायां
भूमौ च ते भुवनमङ्गल दिग्वितानम्‌
मन्दाकिनीति दिवि भोगवतीति चाधो
गङ्गेति चेह चरणाम्बु पुनाति विश्वम्‌
श्रीशुक उवाच
तत्र तेष्वात्मपक्षेष्व गृणत्सु विजिगीषया
वाचः पेशैः स्मयन्भृत्यमुद्धवं प्राह केशवः
श्रीभगवानुवाच
त्वं हि नः परमं चक्षुः सुहृन्मन्त्रार्थतत्त्ववित्‌
अथात्र ब्रूह्यनुष्ठेयं श्रद्दध्मः करवाम तत्‌
इत्युपामन्त्रितो भर्त्रा सर्वज्ञेनापि मुग्धवत्‌
निदेशं शिरसाधाय उद्धवः प्रत्यभाषत
श्रीशुक उवाच
इत्युदीरितमाकर्ण्य देवऋषेरुद्धवोऽब्रवीत्‌
सभ्यानां मतमाज्ञाय कृष्णस्य च महामतिः
श्रीउद्धव उवाच
यदुक्तमृषिना देव साचिव्यं यक्ष्यतस्त्वया
कार्यं पैतृष्वस्रेयस्य रक्षा च शरणैषिणाम्‌
यष्टव्यम्राजसूयेन दिक्चक्रजयिना विभो
अतो जरासुतजय उभयार्थो मतो मम
अस्माकं च महानर्थो ह्येतेनैव भविष्यति
यशश्च तव गोविन्द राज्ञो बद्धान्विमुञ्चतः
स वै दुर्विषहो राजा नागायुतसमो बले
बलिनामपि चान्येषां भीमं समबलं विना
द्वैरथे स तु जेतव्यो मा शताक्षौहिणीयुतः
ब्राह्मण्योऽभ्यर्थितो विप्रैर्न प्रत्याख्याति कर्हिचित्‌
ब्रह्मवेषधरो गत्वा तं भिक्षेत वृकोदरः
हनिष्यति न सन्देहो द्वैरथे तव सन्निधौ
निमित्तं परमीशस्य विश्वसर्गनिरोधयोः
हिरण्यगर्भः शर्वश्च कालस्यारूपिणस्तव
गायन्ति ते विशदकर्म गृहेषु देव्यो
राज्ञां स्वशत्रुवधमात्मविमोक्षणं च
गोप्यश्च कुञ्जरपतेर्जनकात्मजायाः
पित्रोश्च लब्धशरणा मुनयो वयं च
जरासन्धवधः कृष्ण भूर्यर्थायोपकल्पते
प्रायः पाकविपाकेन तव चाभिमतः क्रतुः
श्रीशुक उवाच
इत्युद्धववचो राजन्सर्वतोभद्रमच्युतम्‌
देवर्षिर्यदुवृद्धाश्च कृष्णश्च प्रत्यपूजयन्‌
अथादिशत्प्रयाणाय भगवान्देवकीसुतः
भृत्यान्दारुकजैत्रादीननुज्ञाप्य गुरून्विभुः
निर्गमय्यावरोधान्स्वान्ससुतान्सपरिच्छदान्‌
सङ्कर्षणमनुज्ञाप्य यदुराजं च शत्रुहन्‌
सूतोपनीतं स्वरथमारुहद्गरुडध्वजम्‌
ततो रथद्विपभटसादिनायकैः
करालया परिवृत आत्मसेनया
मृदङ्गभेर्यानकशङ्खगोमुखैः
प्रघोषघोषितककुभो निरक्रमत्‌
नृवाजिकाञ्चनशिबिकाभिरच्युतं सहात्मजाः पतिमनु सुव्रता ययुः
वराम्बराभरणविलेपनस्रजः सुसंवृता नृभिरसिचर्मपाणिभिः
नरोष्ट्रगोमहिषखराश्वतर्यनः
करेणुभिः परिजनवारयोषितः
स्वलङ्कृताः कटकुटिकम्बलाम्बराद्य्‌
उपस्करा ययुरधियुज्य सर्वतः
बलं बृहद्ध्वजपटछत्रचामरैर्‌
वरायुधाभरणकिरीटवर्मभिः
दिवांशुभिस्तुमुलरवं बभौ रवेर्‌
यथार्णवः क्षुभिततिमिङ्गिलोर्मिभिः
अथो मुनिर्यदुपतिना सभाजितः प्रणम्य तं हृदि विदधद्विहायसा
निशम्य तद्व्यवसितमाहृतार्हणो मुकुन्दसन्दरशननिर्वृतेन्द्रियः
राजदूतमुवाचेदं भगवान्प्रीणयन्गिरा
मा भैष्ट दूत भद्रं वो घातयिष्यामि मागधम्‌
इत्युक्तः प्रस्थितो दूतो यथावदवदन्नृपान्‌
तेऽपि सन्दर्शनं शौरेः प्रत्यैक्षन्यन्मुमुक्षवः
आनर्तसौवीरमरूंस्तीर्त्वा विनशनं हरिः
गिरीन्नदीरतीयाय पुरग्रामव्रजाकरान्‌
ततो दृषद्वतीं तीर्त्वा मुकुन्दोऽथ सरस्वतीम्‌
पञ्चालानथ मत्स्यांश्च शक्रप्रस्थमथागमत्‌
तमुपागतमाकर्ण्य प्रीतो दुर्दर्शनं नृनाम्‌
अजातशत्रुर्निरगात्सोपध्यायः सुहृद्वृतः
गीतवादित्रघोषेण ब्रह्मघोषेण भूयसा
अभ्ययात्स हृषीकेशं प्राणाः प्राणमिवादृतः
दृष्ट्वा विक्लिन्नहृदयः कृष्णं स्नेहेन पाण्डवः
चिराद्दृष्टं प्रियतमं सस्वजेऽथ पुनः पुनः
दोर्भ्यां परिष्वज्य रमामलालयं मुकुन्दगात्रं नृपतिर्हताशुभः
लेभे परां निर्वृतिमश्रुलोचनो हृष्यत्तनुर्विस्मृतलोकविभ्रमः
तं मातुलेयं परिरभ्य निर्वृतो भीमः स्मयन्प्रेमजलाकुलेन्द्रियः
यमौ किरीटी च सुहृत्तमं मुदा प्रवृद्धबाष्पाः परिरेभिरेऽच्युतम्‌
अर्जुनेन परिष्वक्तो यमाभ्यामभिवादितः
ब्राह्मणेभ्यो नमस्कृत्य वृद्धेभ्यश्च यथार्हतः
मानिनो मानयामास कुरुसृञ्जयकैकयान्‌
सूतमागधगन्धर्वा वन्दिनश्चोपमन्त्रिणः
मृदङ्गशङ्खपटह वीणापणवगोमुखैः
ब्राह्मणाश्चारविन्दाक्षं तुष्टुवुर्ननृतुर्जगुः
एवं सुहृद्भिः पर्यस्तः पुण्यश्लोकशिखामणिः
संस्तूयमानो भगवान्विवेशालङ्कृतं पुरम्‌
संसिक्तवर्त्म करिणां मदगन्धतोयैश्‌
चित्रध्वजैः कनकतोरणपूर्णकुम्भैः
मृष्टात्मभिर्नवदुकूलविभूषणस्रग्‌
गन्धैर्नृभिर्युवतिभिश्च विराजमानम्‌
उद्दीप्तदीपबलिभिः प्रतिसद्म जाल
निर्यातधूपरुचिरं विलसत्पताकम्‌
मूर्धन्यहेमकलशै रजतोरुशृङ्गैर्‌
जुष्टं ददर्श भवनैः कुरुराजधाम
प्राप्तं निशम्य नरलोचनपानपात्रम्‌
औत्सुक्यविश्लथितकेशदुकूलबन्धाः
सद्यो विसृज्य गृहकर्म पतींश्च तल्पे
द्रष्टुं ययुर्युवतयः स्म नरेन्द्रमार्गे
तस्मिन्सुसङ्कुल इभाश्वरथद्विपद्भिः
कृष्णम्सभार्यमुपलभ्य गृहाधिरूढाः
नार्यो विकीर्य कुसुमैर्मनसोपगुह्य
सुस्वागतं विदधुरुत्स्मयवीक्षितेन
ऊचुः स्त्रियः पथि निरीक्ष्य मुकुन्दपत्नीस्‌
तारा यथोडुपसहाः किमकार्यमूभिः
यच्चक्षुषां पुरुषमौलिरुदारहास
लीलावलोककलयोत्सवमातनोति
तत्र तत्रोपसङ्गम्य पौरा मङ्गलपाणयः
चक्रुः सपर्यां कृष्णाय श्रेणीमुख्या हतैनसः
अन्तःपुरजनैः प्रीत्या मुकुन्दः फुल्ललोचनैः
ससम्भ्रमैरभ्युपेतः प्राविशद्राजमन्दिरम्‌
पृथा विलोक्य भ्रात्रेयं कृष्णं त्रिभुवनेश्वरम्‌
प्रीतात्मोत्थाय पर्यङ्कात्सस्नुषा परिषस्वजे
गोविन्दं गृहमानीय देवदेवेशमादृतः
पूजायां नाविदत्कृत्यं प्रमोदोपहतो नृपः
पितृस्वसुर्गुरुस्त्रीणां कृष्णश्चक्रेऽभिवादनम्‌
स्वयं च कृष्णया राजन्भगिन्या चाभिवन्दितः
श्वशृवा सञ्चोदिता कृष्णा कृष्णपत्नीश्च सर्वशः
आनर्च रुक्मिणीं सत्यां भद्रां जाम्बवतीं तथा
कालिन्दीं मित्रविन्दां च शैब्यां नाग्नजितीं सतीम्‌
अन्याश्चाभ्यागता यास्तु वासःस्रङ्मण्डनादिभिः
सुखं निवासयामास धर्मराजो जनार्दनम्‌
ससैन्यं सानुगामत्यं सभार्यं च नवं नवम्‌
तर्पयित्वा खाण्डवेन वह्निं फाल्गुनसंयुतः
मोचयित्वा मयं येन राज्ञे दिव्या सभा कृता
उवास कतिचिन्मासान्राज्ञः प्रियचिकीर्षया
विहरन्रथमारुह्य फाल्गुनेन भटैर्वृतः
श्रीशुक उवाच
एकदा तु सभामध्य आस्थितो मुनिभिर्वृतः
ब्राह्मणैः क्षत्रियैर्वैश्यैर्भ्रातृभिश्च युधिष्ठिरः
आचार्यैः कुलवृद्धैश्च ज्ञातिसम्बन्धिबान्धवैः
शृण्वतामेव चैतेषामाभाष्येदमुवाच ह
श्रीयुधिष्ठिर उवाच
क्रतुराजेन गोविन्द राजसूयेन पावनीः
यक्ष्ये विभूतीर्भवतस्तत्सम्पादय नः प्रभो
त्वत्पादुके अविरतं परि ये चरन्ति
ध्यायन्त्यभद्रनशने शुचयो गृणन्ति
विन्दन्ति ते कमलनाभ भवापवर्गम्‌
आशासते यदि त आशिष ईश नान्ये
तद्देवदेव भवतश्चरणारविन्द
सेवानुभावमिह पश्यतु लोक एषः
ये त्वां भजन्ति न भजन्त्युत वोभयेषां
निष्ठां प्रदर्शय विभो कुरुसृञ्जयानाम्‌
न ब्रह्मणः स्वपरभेदमतिस्तव स्यात्‌
सर्वात्मनः समदृशः स्वसुखानुभूतेः
संसेवतां सुरतरोरिव ते प्रसादः
सेवानुरूपमुदयो न विपर्ययोऽत्र
श्रीभगवानुवाच
सम्यग्व्यवसितं राजन्भवता शत्रुकर्शन
कल्याणी येन ते कीर्तिर्लोकाननुभविष्यति
ऋषीणां पितृदेवानां सुहृदामपि नः प्रभो
सर्वेषामपि भूतानामीप्सितः क्रतुराडयम्‌
विजित्य नृपतीन्सर्वान्कृत्वा च जगतीं वशे
सम्भृत्य सर्वसम्भारानाहरस्व महाक्रतुम्‌
एते ते भ्रातरो राजंल्लोकपालांशसम्भवाः
जितोऽस्म्यात्मवता तेऽहं दुर्जयो योऽकृतात्मभिः
न कश्चिन्मत्परं लोके तेजसा यशसा श्रिया
विभूतिभिर्वाभिभवेद्देवोऽपि किमु पार्थिवः
श्रीशुक उवाच
निशम्य भगवद्गीतं प्रीतः फुल्लमुखाम्बुजः
भ्रात्ऱ्‌^ईन्दिग्विजयेऽयुङ्क्त विष्णुतेजोपबृंहितान्‌
सहदेवं दक्षिणस्यामादिशत्सह सृञ्जयैः
दिशि प्रतीच्यां नकुलमुदीच्यां सव्यसाचिनम्‌
प्राच्यां वृकोदरं मत्स्यैः केकयैः सह मद्रकैः
ते विजित्य नृपान्वीरा आजह्रुर्दिग्भ्य ओजसा
अजातशत्रवे भूरि द्रविणं नृप यक्ष्यते
श्रुत्वाजितं जरासन्धं नृपतेर्ध्यायतो हरिः
आहोपायं तमेवाद्य उद्धवो यमुवाच ह
भीमसेनोऽर्जुनः कृष्णो ब्रह्मलिन्गधरास्त्रयः
जग्मुर्गिरिव्रजं तात बृहद्रथसुतो यतः
ते गत्वातिथ्यवेलायां गृहेषु गृहमेधिनम्‌
ब्रह्मण्यं समयाचेरन्राजन्या ब्रह्मलिङ्गिनः
राजन्विद्ध्यतिथीन्प्राप्तानर्थिनो दूरमागतान्‌
तन्नः प्रयच्छ भद्रं ते यद्वयं कामयामहे
किं दुर्मर्षं तितिक्षूणां किमकार्यमसाधुभिः
किं न देयं वदान्यानां कः परः समदर्शिनाम्‌
योऽनित्येन शरीरेण सतां गेयं यशो ध्रुवम्‌
नाचिनोति स्वयं कल्पः स वाच्यः शोच्य एव सः
हरिश्चन्द्रो रन्तिदेव उञ्छवृत्तिः शिबिर्बलिः
व्याधः कपोतो बहवो ह्यध्रुवेण ध्रुवं गताः
श्रीशुक उवाच
स्वरैराकृतिभिस्तांस्तु प्रकोष्ठैर्ज्याहतैरपि
राजन्यबन्धून्विज्ञाय दृष्टपूर्वानचिन्तयत्‌
राजन्यबन्धवो ह्येते ब्रह्मलिङ्गानि बिभ्रति
ददानि भिक्षितं तेभ्य आत्मानमपि दुस्त्यजम्‌
बलेर्नु श्रूयते कीर्तिर्वितता दिक्ष्वकल्मषा
ऐश्वर्याद्भ्रंशितस्यापि विप्रव्याजेन विष्णुना
श्रियं जिहीर्षतेन्द्रस्य विष्णवे द्विजरूपिणे
जानन्नपि महीम्प्रादाद्वार्यमाणोऽपि दैत्यराट्‌
जीवता ब्राह्मणार्थाय को न्वर्थः क्षत्रबन्धुना
देहेन पतमानेन नेहता विपुलं यशः
इत्युदारमतिः प्राह कृष्णार्जुनवृकोदरान्‌
हे विप्रा व्रियतां कामो ददाम्यात्मशिरोऽपि वः
श्रीभगवानुवाच
युद्धं नो देहि राजेन्द्र द्वन्द्वशो यदि मन्यसे
युद्धार्थिनो वयं प्राप्ता राजन्या नान्यकाङ्क्षिणः
असौ वृकोदरः पार्थस्तस्य भ्रातार्जुनो ह्ययम्‌
अनयोर्मातुलेयं मां कृष्णं जानीहि ते रिपुम्‌
एवमावेदितो राजा जहासोच्चैः स्म मागधः
आह चामर्षितो मन्दा युद्धं तर्हि ददामि वः
न त्वया भीरुणा योत्स्ये युधि विक्लवतेजसा
मथुरां स्वपुरीं त्यक्त्वा समुद्रं शरणं गतः
अयं तु वयसातुल्यो नातिसत्त्वो न मे समः
अर्जुनो न भवेद्योद्धा भीमस्तुल्यबलो मम
इत्युक्त्वा भीमसेनाय प्रादाय महतीं गदाम्‌
द्वितीयां स्वयमादाय निर्जगाम पुराद्बहिः
ततः समेखले वीरौ संयुक्तावितरेतरम्‌
जघ्नतुर्वज्रकल्पाभ्यां गदाभ्यां रणदुर्मदौ
मण्डलानि विचित्राणि सव्यं दक्षिणमेव च
चरतोः शुशुभे युद्धं नटयोरिव रङ्गिणोः
ततश्चटचटाशब्दो वज्रनिष्पेससन्निभः
गदयोः क्षिप्तयो राजन्दन्तयोरिव दन्तिनोः
ते वै गदे भुजजवेन निपात्यमाने
अन्योन्यतोऽंसकटिपादकरोरुजत्रुम्‌
चूर्णीबभूवतुरुपेत्य यथार्कशाखे
संयुध्यतोर्द्विरदयोरिव दीप्तमन्व्योः
इत्थं तयोः प्रहतयोर्गदयोर्नृवीरौ
क्रुद्धौ स्वमुष्टिभिरयःस्परशैरपिष्टाम्‌
शब्दस्तयोः प्रहरतोरिभयोरिवासीन्‌
निर्घातवज्रपरुषस्तलताडनोत्थः
तयोरेवं प्रहरतोः समशिक्षाबलौजसोः
निर्विशेषमभूद्युद्धमक्षीणजवयोर्नृप
शत्रोर्जन्ममृती विद्वाञ्जीवितं च जराकृतम्‌
पार्थमाप्याययन्स्वेन तेजसाचिन्तयद्धरिः
सञ्चिन्त्यारीवधोपायं भीमस्यामोघदर्शनः
दर्शयामास विटपं पाटयन्निव संज्ञया
तद्विज्ञाय महासत्त्वो भीमः प्रहरतां वरः
गृहीत्वा पादयोः शत्रुं पातयामास भूतले
एकम्पादं पदाक्रम्य दोर्भ्यामन्यं प्रगृह्य सः
गुदतः पाटयामास शाखमिव महागजः
एकपादोरुवृषण कटिपृष्ठस्तनांसके
एकबाह्वक्षिभ्रूकर्णे शकले ददृशुः प्रजाः
हाहाकारो महानासीन्निहते मगधेश्वरे
पूजयामासतुर्भीमं परिरभ्य जयाच्यतौ
सहदेवं तत्तनयं भगवान्भूतभावनः
अभ्यषिञ्चदमेयात्मा मगधानां पतिं प्रभुः
मोचयामास राजन्यान्संरुद्धा मागधेन ये
श्रीशुक उवाच
अयुते द्वे शतान्यष्टौ निरुद्धा युधि निर्जिताः
ते निर्गता गिरिद्रोण्यां मलिना मलवाससः
क्षुत्क्षामाः शुष्कवदनाः संरोधपरिकर्शिताः
ददृशुस्ते घनश्यामं पीतकौशेयवाससम्‌
श्रीवत्साङ्कं चतुर्बाहुं पद्मगर्भारुणेक्षणम्‌
चारुप्रसन्नवदनं स्फुरन्मकरकुण्डलम्‌
पद्महस्तं गदाशङ्ख रथाङ्गैरुपलक्षितम्‌
किरीटहारकटक कटिसूत्राङ्गदाञ्चितम्‌
भ्राजद्वरमणिग्रीवं निवीतं वनमालया
पिबन्त इव चक्षुर्भ्यां लिहन्त इव जिह्वया
जिघ्रन्त इव नासाभ्यां रम्भन्त इव बाहुभिः
प्रणेमुर्हतपाप्मानो मूर्धभिः पादयोर्हरेः
कृष्णसन्दर्शनाह्लाद ध्वस्तसंरोधनक्लमाः
प्रशशंसुर्हृषीकेशं गीर्भिः प्राञ्जलयो नृपाः
राजान ऊचुः
नमस्ते देवदेवेश प्रपन्नार्तिहराव्यय
प्रपन्नान्पाहि नः कृष्ण निर्विण्णान्घोरसंसृतेः
नैनं नाथानुसूयामो मागधं मधुसूदन
अनुग्रहो यद्भवतो राज्ञां राज्यच्युतिर्विभो
राज्यैश्वर्यमदोन्नद्धो न श्रेयो विन्दते नृपः
त्वन्मायामोहितोऽनित्या मन्यते सम्पदोऽचलाः
मृगतृष्णां यथा बाला मन्यन्त उदकाशयम्‌
एवं वैकारिकीं मायामयुक्ता वस्तु चक्षते
वयं पुरा श्रीमदनष्टदृष्टयो जिगीषयास्या इतरेतरस्पृधः
घ्नन्तः प्रजाः स्वा अतिनिर्घृणाः प्रभो मृत्युं पुरस्त्वाविगणय्य दुर्मदाः
त एव कृष्णाद्य गभीररंहसा दुरन्तेवीर्येण विचालिताः श्रियः
कालेन तन्वा भवतोऽनुकम्पया विनष्टदर्पाश्चरणौ स्मराम ते
अथो न राज्यम्मृगतृष्णिरूपितं देहेन शश्वत्पतता रुजां भुवा
उपासितव्यं स्पृहयामहे विभो क्रियाफलं प्रेत्य च कर्णरोचनम्‌
तं नः समादिशोपायं येन ते चरणाब्जयोः
स्मृतिर्यथा न विरमेदपि संसरतामिह
कृष्णाय वासुदेवाय हरये परमात्मने
प्रणतक्लेशनाशाय गोविन्दाय नमो नमः
श्रीशुक उवाच
संस्तूयमानो भगवान्राजभिर्मुक्तबन्धनैः
तानाह करुणस्तात शरण्यः श्लक्ष्णया गिरा
श्रीभगवानुवाच
अद्य प्रभृति वो भूपा मय्यात्मन्यखिलेश्वरे
सुदृढा जायते भक्तिर्बाढमाशंसितं तथा
दिष्ट्या व्यवसितं भूपा भवन्त ऋतभाषिणः
श्रीयैश्वर्यमदोन्नाहं पश्य उन्मादकं नृणाम्‌
हैहयो नहुषो वेणो रावणो नरकोऽपरे
श्रीमदाद्भ्रंशिताः स्थानाद्देवदैत्यनरेश्वराः
भवन्त एतद्विज्ञाय देहाद्युत्पाद्यमन्तवत्‌
मां यजन्तोऽध्वरैर्युक्ताः प्रजा धर्मेण रक्ष्यथ
सन्तन्वन्तः प्रजातन्तून्सुखं दुःखं भवाभवौ
प्राप्तं प्राप्तं च सेवन्तो मच्चित्ता विचरिष्यथ
उदासीनाश्च देहादावात्मारामा धृतव्रताः
मय्यावेश्य मनः सम्यङ्मामन्ते ब्रह्म यास्यथ
श्रीशुक उवाच
इत्यादिश्य नृपान्कृष्णो भगवान्भुवनेश्वरः
तेषां न्ययुङ्क्त पुरुषान्स्त्रियो मज्जनकर्मणि
सपर्यां कारयामास सहदेवेन भारत
नरदेवोचितैर्वस्त्रैर्भूषणैः स्रग्विलेपनैः
भोजयित्वा वरान्नेन सुस्नातान्समलङ्कृतान्‌
भोगैश्च विविधैर्युक्तांस्ताम्बूलाद्यैर्नृपोचितैः
ते पूजिता मुकुन्देन राजानो मृष्टकुण्डलाः
विरेजुर्मोचिताः क्लेशात्प्रावृडन्ते यथा ग्रहाः
रथान्सदश्वानारोप्य मणिकाञ्चनभूषितान्‌
प्रीणय्य सुनृतैर्वाक्यैः स्वदेशान्प्रत्ययापयत्‌
त एवं मोचिताः कृच्छ्रात्कृष्णेन सुमहात्मना
ययुस्तमेव ध्यायन्तः कृतानि च जगत्पतेः
जगदुः प्रकृतिभ्यस्ते महापुरुषचेष्टितम्‌
यथान्वशासद्भगवांस्तथा चक्रुरतन्द्रिताः
जरासन्धं घातयित्वा भीमसेनेन केशवः
पार्थाभ्यां संयुतः प्रायात्सहदेवेन पूजितः
गत्वा ते खाण्डवप्रस्थं शङ्खान्दध्मुर्जितारयः
हर्षयन्तः स्वसुहृदो दुर्हृदां चासुखावहाः
तच्छ्रुत्वा प्रीतमनस इन्द्रप्रस्थनिवासिनः
मेनिरे मागधं शान्तं राजा चाप्तमनोरथः
अभिवन्द्याथ राजानं भीमार्जुनजनार्दनाः
सर्वमाश्रावयां चक्रुरात्मना यदनुष्ठितम्‌
निशम्य धर्मराजस्तत्केशवेनानुकम्पितम्‌
आनन्दाश्रुकलां मुञ्चन्प्रेम्णा नोवाच किञ्चन
श्रीशुक उवाच
एवं युधिष्ठिरो राजा जरासन्धवधं विभोः
कृष्णस्य चानुभावं तं श्रुत्वा प्रीतस्तमब्रवीत्‌
श्रीयुधिष्ठिर उवाच
ये स्युस्त्रैलोक्यगुरवः सर्वे लोका महेश्वराः
वहन्ति दुर्लभं लब्द्वा शिरसैवानुशासनम्‌
स भवानरविन्दाक्षो दीनानामीशमानिनाम्‌
धत्तेऽनुशासनं भूमंस्तदत्यन्तविडम्बनम्‌
न ह्येकस्याद्वितीयस्य ब्रह्मणः परमात्मनः
कर्मभिर्वर्धते तेजो ह्रसते च यथा रवेः
न वै तेऽजित भक्तानां ममाहमिति माधव
त्वं तवेति च नानाधीः पशूनामिव वैकृती
श्रीशुक उवाच
इत्युक्त्वा यज्ञिये काले वव्रे युक्तान्स ऋत्विजः
कृष्णानुमोदितः पार्थो ब्राह्मणान्ब्रह्मवादिनः
द्वैपायनो भरद्वाजः सुमन्तुर्गोतमोऽसितः
वसिष्ठश्च्यवनः कण्वो मैत्रेयः कवषस्त्रितः
विश्वामित्रो वामदेवः सुमतिर्जैमिनिः क्रतुः
पैलः पराशरो गर्गो वैशम्पायन एव च
अथर्वा कश्यपो धौम्यो रामो भार्गव आसुरिः
वीतिहोत्रो मधुच्छन्दा वीरसेनोऽकृतव्रणः
उपहूतास्तथा चान्ये द्रोणभीष्मकृपादयः
धृतराष्ट्रः सहसुतो विदुरश्च महामतिः
ब्राह्मणाः क्षत्रिया वैश्याः शूद्रा यज्ञदिदृक्षवः
तत्रेयुः सर्वराजानो राज्ञां प्रकृतयो नृप
ततस्ते देवयजनं ब्राह्मणाः स्वर्णलाङ्गलैः
कृष्ट्वा तत्र यथाम्नायं दीक्षयां चक्रिरे नृपम्‌
हैमाः किलोपकरणा वरुणस्य यथा पुरा
इन्द्रादयो लोकपाला विरिञ्चिभवसंयुताः
सगणाः सिद्धगन्धर्वा विद्याधरमहोरगाः
मुनयो यक्षरक्षांसि खगकिन्नरचारणाः
राजानश्च समाहूता राजपत्न्यश्च सर्वशः
राजसूयं समीयुः स्म राज्ञः पाण्डुसुतस्य वै
मेनिरे कृष्णभक्तस्य सूपपन्नमविस्मिताः
अयाजयन्महाराजं याजका देववर्चसः
राजसूयेन विधिवत्प्रचेतसमिवामराः
सूत्येऽहन्यवनीपालो याजकान्सदसस्पतीन्‌
अपूजयन्महाभागान्यथावत्सुसमाहितः
सदस्याग्र्यार्हणार्हं वै विमृशन्तः सभासदः
नाध्यगच्छन्ननैकान्त्यात्सहदेवस्तदाब्रवीत्‌
अर्हति ह्यच्युतः श्रैष्ठ्यं भगवान्सात्वतां पतिः
एष वै देवताः सर्वा देशकालधनादयः
यदात्मकमिदं विश्वं क्रतवश्च यदात्मकाः
अग्निराहुतयो मन्त्रा साङ्ख्यं योगश्च यत्परः
एक एवाद्वितीयोऽसावैतदात्म्यमिदं जगत्‌
आत्मनात्माश्रयः सभ्याः सृजत्यवति हन्त्यजः
विविधानीह कर्माणि जनयन्यदवेक्षया
ईहते यदयं सर्वः श्रेयो धर्मादिलक्षणम्‌
तस्मात्कृष्णाय महते दीयतां परमार्हणम्‌
एवं चेत्सर्वभूतानामात्मनश्चार्हणं भवेत्‌
सर्वभूतात्मभूताय कृष्णायानन्यदर्शिने
देयं शान्ताय पूर्णाय दत्तस्यानन्त्यमिच्छता
इत्युक्त्वा सहदेवोऽभूत्तूष्णीं कृष्णानुभाववित्‌
तच्छ्रुत्वा तुष्टुवुः सर्वे साधु साध्विति सत्तमाः
श्रुत्वा द्विजेरितं राजा ज्ञात्वा हार्दं सभासदाम्‌
समर्हयद्धृषीकेशं प्रीतः प्रणयविह्वलः
तत्पादाववनिज्यापः शिरसा लोकपावनीः
सभार्यः सानुजामात्यः सकुटुम्बो वहन्मुदा
वासोभिः पीतकौषेयैर्भूषणैश्च महाधनैः
अर्हयित्वाश्रुपूर्णाक्षो नाशकत्समवेक्षितुम्‌
इत्थं सभाजितं वीक्ष्य सर्वे प्राञ्जलयो जनाः
नमो जयेति नेमुस्तं निपेतुः पुष्पवृष्टयः
इत्थं निशम्य दमघोषसुतः स्वपीठाद्‌
उत्थाय कृष्णगुणवर्णनजातमन्युः
उत्क्षिप्य बाहुमिदमाह सदस्यमर्षी
संश्रावयन्भगवते परुषाण्यभीतः
ईशो दुरत्ययः काल इति सत्यवती स्रुतिः
वृद्धानामपि यद्बुद्धिर्बालवाक्यैर्विभिद्यते
यूयं पात्रविदां श्रेष्ठा मा मन्ध्वं बालभाषीतम्‌
सदसस्पतयः सर्वे कृष्णो यत्सम्मतोऽर्हणे
तपोविद्याव्रतधरान्ज्ञानविध्वस्तकल्मषान्‌
परमऋषीन्ब्रह्मनिष्ठांल्लोकपालैश्च पूजितान्‌
सदस्पतीनतिक्रम्य गोपालः कुलपांसनः
यथा काकः पुरोडाशं सपर्यां कथमर्हति
वर्णाश्रमकुलापेतः सर्वधर्मबहिष्कृतः
स्वैरवर्ती गुणैर्हीनः सपर्यां कथमर्हति
ययातिनैषां हि कुलं शप्तं सद्भिर्बहिष्कृतम्‌
वृथापानरतं शश्वत्सपर्यां कथमर्हति
ब्रह्मर्षिसेवितान्देशान्हित्वैतेऽब्रह्मवर्चसम्‌
समुद्रं दुर्गमाश्रित्य बाधन्ते दस्यवः प्रजाः
एवमादीन्यभद्राणि बभाषे नष्टमङ्गलः
नोवाच किञ्चिद्भगवान्यथा सिंहः शिवारुतम्‌
भगवन्निन्दनं श्रुत्वा दुःसहं तत्सभासदः
कर्णौ पिधाय निर्जग्मुः शपन्तश्चेदिपं रुषा
निन्दां भगवतः शृण्वंस्तत्परस्य जनस्य वा
ततो नापैति यः सोऽपि यात्यधः सुकृताच्च्युतः
ततः पाण्डुसुताः क्रुद्धा मत्स्यकैकयसृञ्जयाः
उदायुधाः समुत्तस्थुः शिशुपालजिघांसवः
ततश्चैद्यस्त्वसम्भ्रान्तो जगृहे खड्गचर्मणी
भर्त्सयन्कृष्णपक्षीयान्राज्ञः सदसि भारत
तावदुत्थाय भगवान्स्वान्निवार्य स्वयं रुषा
शिरः क्षुरान्तचक्रेण जहार पततो रिपोः
शब्दः कोलाहलोऽथासीच्छिशुपाले हते महान्‌
तस्यानुयायिनो भूपा दुद्रुवुर्जीवितैषिणः
चैद्यदेहोत्थितं ज्योतिर्वासुदेवमुपाविशत्‌
पश्यतां सर्वभूतानामुल्केव भुवि खाच्च्युता
जन्मत्रयानुगुणित वैरसंरब्धया धिया
ध्यायंस्तन्मयतां यातो भावो हि भवकारणम्‌
ऋत्विग्भ्यः ससदस्येभ्यो दक्षिनां विपुलामदात्‌
सर्वान्सम्पूज्य विधिवच्चक्रेऽवभृथमेकराट्‌
साधयित्वा क्रतुः राज्ञः कृष्णो योगेश्वरेश्वरः
उवास कतिचिन्मासान्सुहृद्भिरभियाचितः
ततोऽनुज्ञाप्य राजानमनिच्छन्तमपीश्वरः
ययौ सभार्यः सामात्यः स्वपुरं देवकीसुतः
वर्णितं तदुपाख्यानं मया ते बहुविस्तरम्‌
वैकुण्ठवासिनोर्जन्म विप्रशापात्पुनः पुनः
राजसूयावभृथ्येन स्नातो राजा युधिष्ठिरः
ब्रह्मक्षत्रसभामध्ये शुशुभे सुरराडिव
राज्ञा सभाजिताः सर्वे सुरमानवखेचराः
कृष्णं क्रतुं च शंसन्तः स्वधामानि ययुर्मुदा
दुर्योधनमृते पापं कलिं कुरुकुलामयम्‌
यो न सेहे श्रीयं स्फीतां दृष्ट्वा पाण्डुसुतस्य ताम्‌
य इदं कीर्तयेद्विष्णोः कर्म चैद्यवधादिकम्‌
राजमोक्षं वितानं च सर्वपापैः प्रमुच्यते
श्रीराजोवाच
अजातशत्रोस्तम्दृष्ट्वा राजसूयमहोदयम्‌
सर्वे मुमुदिरे ब्रह्मन्नृदेवा ये समागताः
दुर्योधनं वर्जयित्वा राजानः सर्षयः सुराः
इति श्रुतं नो भगवंस्तत्र कारणमुच्यताम्‌
श्रीबादरायणिरुवाच
पितामहस्य ते यज्ञे राजसूये महात्मनः
बान्धवाः परिचर्यायां तस्यासन्प्रेमबन्धनाः
भीमो महानसाध्यक्षो धनाध्यक्षः सुयोधनः
सहदेवस्तु पूजायां नकुलो द्रव्यसाधने
गुरुशुश्रूषणे जिष्णुः कृष्णः पादावनेजने
परिवेषणे द्रुपदजा कर्णो दाने महामनाः
युयुधानो विकर्णश्च हार्दिक्यो विदुरादयः
बाह्लीकपुत्रा भूर्याद्या ये च सन्तर्दनादयः
निरूपिता महायज्ञे नानाकर्मसु ते तदा
प्रवर्तन्ते स्म राजेन्द्र राज्ञः प्रियचिकीर्षवः
ऋत्विक्सदस्यबहुवित्सु सुहृत्तमेषु
स्विष्टेषु सूनृतसमर्हणदक्षिणाभिः
चैद्ये च सात्वतपतेश्चरणं प्रविष्टे
चक्रुस्ततस्त्ववभृथस्नपनं द्युनद्याम्‌
मृदङ्गशङ्खपणव धुन्धुर्यानकगोमुखाः
वादित्राणि विचित्राणि नेदुरावभृथोत्सवे
नार्तक्यो ननृतुर्हृष्टा गायका यूथशो जगुः
वीणावेणुतलोन्नादस्तेषां स दिवमस्पृशत्‌
चित्रध्वजपताकाग्रैरिभेन्द्रस्यन्दनार्वभिः
स्वलङ्कृतैर्भटैर्भूपा निर्ययू रुक्ममालिनः
यदुसृञ्जयकाम्बोज कुरुकेकयकोशलाः
कम्पयन्तो भुवं सैन्यैर्ययमानपुरःसराः
सदस्यर्त्विग्द्विजश्रेष्ठा ब्रह्मघोषेण भूयसा
देवर्षिपितृगन्धर्वास्तुष्टुवुः पुष्पवर्षिणः
स्वलण्कृता नरा नार्यो गन्धस्रग्भूषणाम्बरैः
विलिम्पन्त्योऽभिसिञ्चन्त्यो विजह्रुर्विविधै रसैः
तैलगोरसगन्धोद हरिद्रासान्द्रकुङ्कुमैः
पुम्भिर्लिप्ताः प्रलिम्पन्त्यो विजह्रुर्वारयोषितः
गुप्ता नृभिर्निरगमन्नुपलब्धुमेतद्‌
देव्यो यथा दिवि विमानवरैर्नृदेव्यो
ता मातुलेयसखिभिः परिषिच्यमानाः
सव्रीडहासविकसद्वदना विरेजुः
ता देवरानुत सखीन्सिषिचुर्दृतीभिः
क्लिन्नाम्बरा विवृतगात्रकुचोरुमध्याः
औत्सुक्यमुक्तकवराच्च्यवमानमाल्याः
क्षोभं दधुर्मलधियां रुचिरैर्विहारैः
स सम्राड्रथमारुढः सदश्वं रुक्ममालिनम्‌
व्यरोचत स्वपत्नीभिः क्रियाभिः क्रतुराडिव
पत्नीसम्याजावभृथ्यैश्चरित्वा ते तमृत्विजः
आचान्तं स्नापयां चक्रुर्गङ्गायां सह कृष्णया
देवदुन्दुभयो नेदुर्नरदुन्दुभिभिः समम्‌
मुमुचुः पुष्पवर्षाणि देवर्षिपितृमानवाः
सस्नुस्तत्र ततः सर्वे वर्णाश्रमयुता नराः
महापातक्यपि यतः सद्यो मुच्येत किल्बिषात्‌
अथ राजाहते क्षौमे परिधाय स्वलङ्कृतः
ऋत्विक्सदस्यविप्रादीनानर्चाभरणाम्बरैः
बन्धूञ्ज्ञातीन्नृपान्मित्र सुहृदोऽन्यांश्च सर्वशः
अभीक्ष्नं पूजयामास नारायणपरो नृपः
सर्वे जनाः सुररुचो मणिकुण्डलस्रग्‌
उष्णीषकञ्चुकदुकूलमहार्घ्यहाराः
नार्यश्च कुण्डलयुगालकवृन्दजुष्ट
वक्त्रश्रियः कनकमेखलया विरेजुः
अथर्त्विजो महाशीलाः सदस्या ब्रह्मवादिनः
ब्रह्मक्षत्रियविट्शुद्रा राजानो ये समागताः
देवर्षिपितृभूतानि लोकपालाः सहानुगाः
पूजितास्तमनुज्ञाप्य स्वधामानि ययुर्नृप
हरिदासस्य राजर्षे राजसूयमहोदयम्‌
नैवातृप्यन्प्रशंसन्तः पिबन्मर्त्योऽमृतं यथा
ततो युधिष्ठिरो राजा सुहृत्सम्बन्धिबान्धवान्‌
प्रेम्णा निवारयामास कृष्णं च त्यागकातरः
भगवानपि तत्राङ्ग न्यावात्सीत्तत्प्रियंकरः
प्रस्थाप्य यदुवीरांश्च साम्बादींश्च कुशस्थलीम्‌
इत्थं राजा धर्मसुतो मनोरथमहार्णवम्‌
सुदुस्तरं समुत्तीर्य कृष्णेनासीद्गतज्वरः
एकदान्तःपुरे तस्य वीक्ष्य दुर्योधनः श्रियम्‌
अतप्यद्राजसूयस्य महित्वं चाच्युतात्मनः
यस्मिंस्नरेन्द्रदितिजेन्द्रसुरेन्द्रलक्ष्मीर्‌
नाना विभान्ति किल विश्वसृजोपकॢप्ताः
ताभिः पतीन्द्रुपदराजसुतोपतस्थे
यस्यां विषक्तहृदयः कुरुराडतप्यत्‌
यस्मिन्तदा मधुपतेर्महिषीसहस्रं
श्रोणीभरेण शनकैः क्वणदङ्घ्रिशोभम्‌
मध्ये सुचारु कुचकुङ्कुमशोणहारं
श्रीमन्मुखं प्रचलकुण्डलकुन्तलाढ्यम्‌
सभायां मयकॢप्तायां क्वापि धर्मसुतोऽधिराट्‌
वृतोऽनुगैर्बन्धुभिश्च कृष्णेनापि स्वचक्षुषा
आसीनः काञ्चने साक्षादासने मघवानिव
पारमेष्ठ्यश्रीया जुष्टः स्तूयमानश्च वन्दिभिः
तत्र दुर्योधनो मानी परीतो भ्रातृभिर्नृप
किरीटमाली न्यविशदसिहस्तः क्षिपन्रुषा
स्थलेऽभ्यगृह्णाद्वस्त्रान्तं जलं मत्वा स्थलेऽपतत्‌
जले च स्थलवद्भ्रान्त्या मयमायाविमोहितः
जहास भीमस्तं दृष्ट्वा स्त्रियो नृपतयो परे
निवार्यमाणा अप्यङ्ग राज्ञा कृष्णानुमोदिताः
स व्रीडितोऽवग्वदनो रुषा ज्वलन्निष्क्रम्य तूष्णीं प्रययौ गजाह्वयम्‌
हाहेति शब्दः सुमहानभूत्सतामजातशत्रुर्विमना इवाभवत्‌
बभूव तूष्णीं भगवान्भुवो भरं समुज्जिहीर्षुर्भ्रमति स्म यद्दृशा
एतत्तेऽभिहितं राजन्यत्पृष्टोऽहमिह त्वया
सुयोधनस्य दौरात्म्यं राजसूये महाक्रतौ
श्रीशुक उवाच
अथान्यदपि कृष्णस्य शृणु कर्माद्भुतं नृप
क्रीडानरशरीरस्य यथा सौभपतिर्हतः
शिशुपालसखः शाल्वो रुक्मिण्युद्वाह आगतः
यदुभिर्निर्जितः सङ्ख्ये जरासन्धादयस्तथा
शाल्वः प्रतिज्ञामकरोच्छृण्वतां सर्वभूभुजाम्‌
अयादवां क्ष्मां करिष्ये पौरुषं मम पश्यत
इति मूढः प्रतिज्ञाय देवं पशुपतिं प्रभुम्‌
आराधयामास नृपः पांशुमुष्टिं सकृद्ग्रसन्‌
संवत्सरान्ते भगवानाशुतोष उमापतिः
वरेण च्छन्दयामास शाल्वं शरणमागतम्‌
देवासुरमनुष्याणां गन्धर्वोरगरक्षसाम्‌
अभेद्यं कामगं वव्रे स यानं वृष्णिभीषणम्‌
तथेति गिरिशादिष्टो मयः परपुरंजयः
पुरं निर्माय शाल्वाय प्रादात्सौभमयस्मयम्‌
स लब्ध्वा कामगं यानं तमोधाम दुरासदम्‌
ययस्द्वारवतीं शाल्वो वैरं वृष्णिकृतं स्मरन्‌
निरुध्य सेनया शाल्वो महत्या भरतर्षभ
पुरीं बभञ्जोपवनानुद्यानानि च सर्वशः
सगोपुराणि द्वाराणि प्रासादाट्टालतोलिकाः
विहारान्स विमानाग्र्यान्निपेतुः शस्त्रवृष्टयः
शिलाद्रुमाश्चाशनयः सर्पा आसारशर्कराः
प्रचण्डश्चक्रवातोऽभूद्रजसाच्छादिता दिशः
इत्यर्द्यमाना सौभेन कृष्णस्य नगरी भृशम्‌
नाभ्यपद्यत शं राजंस्त्रिपुरेण यथा मही
प्रद्युम्नो भगवान्वीक्ष्य बाध्यमाना निजाः प्रजाः
म भैष्टेत्यभ्यधाद्वीरो रथारूढो महायशाः
सात्यकिश्चारुदेष्णश्च साम्बोऽक्रूरः सहानुजः
हार्दिक्यो भानुविन्दश्च गदश्च शुकसारणौ
अपरे च महेष्वासा रथयूथपयूथपाः
निर्ययुर्दंशिता गुप्ता रथेभाश्वपदातिभिः
ततः प्रववृते युद्धं शाल्वानां यदुभिः सह
यथासुराणां विबुधैस्तुमुलं लोमहर्षणम्‌
ताश्च सौभपतेर्माया दिव्यास्त्रै रुक्मिणीसुतः
क्षणेन नाशयामास नैशं तम इवोष्णगुः
विव्याध पञ्चविंशत्या स्वर्णपुङ्खैरयोमुखैः
शाल्वस्य ध्वजिनीपालं शरैः सन्नतपर्वभिः
शतेनाताडयच्छाल्वमेकैकेनास्य सैनिकान्‌
दशभिर्दशभिर्नेत्ऱ्‌^ईन्वाहनानि त्रिभिस्त्रिभिः
तदद्भुतं महत्कर्म प्रद्युम्नस्य महात्मनः
दृष्ट्वा तं पूजयामासुः सर्वे स्वपरसैनिकाः
बहुरूपैकरूपं तद्दृश्यते न च दृश्यते
मायामयं मयकृतं दुर्विभाव्यं परैरभूत्‌
क्वचिद्भूमौ क्वचिद्व्योम्नि गिरिमूर्ध्नि जले क्वचित्‌
अलातचक्रवद्भ्राम्यत्सौभं तद्दुरवस्थितम्‌
यत्र यत्रोपलक्ष्येत ससौभः सहसैनिकः
शाल्वस्ततस्ततोऽमुञ्चञ्छरान्सात्वतयूथपाः
शरैरग्न्यर्कसंस्पर्शैराशीविषदुरासदैः
पीड्यमानपुरानीकः शाल्वोऽमुह्यत्परेरितैः
शाल्वानीकपशस्त्रौघैर्वृष्णिवीरा भृशार्दिताः
न तत्यजू रणं स्वं स्वं लोकद्वयजिगीषवः
शाल्वामात्यो द्युमान्नाम प्रद्युम्नं प्रक्प्रपीडितः
आसाद्य गदया मौर्व्या व्याहत्य व्यनदद्बली
प्रद्युम्नं गदया सीर्ण वक्षःस्थलमरिंदमम्‌
अपोवाह रणात्सूतो धर्मविद्दारुकात्मजः
लब्धसम्ज्ञो मुहूर्तेन कार्ष्णिः सारथिमब्रवीत्‌
अहो असाध्विदं सूत यद्रणान्मेऽपसर्पणम्‌
न यदूनां कुले जातः श्रूयते रणविच्युतः
विना मत्क्लीबचित्तेन सूतेन प्राप्तकिल्बिषात्‌
किं नु वक्ष्येऽभिसङ्गम्य पितरौ रामकेशवौ
युद्धात्सम्यगपक्रान्तः पृष्टस्तत्रात्मनः क्षमम्‌
व्यक्तं मे कथयिष्यन्ति हसन्त्यो भ्रातृजामयः
क्लैब्यं कथं कथं वीर तवान्यैः कथ्यतां मृधे
सारथिरुवाच
धर्मं विजानतायुष्मन्कृतमेतन्मया विभो
सूतः कृच्छ्रगतं रक्षेद्रथिनं सारथिं रथी
एतद्विदित्वा तु भवान्मयापोवाहितो रणात्‌
उपसृष्टः परेणेति मूर्च्छितो गदया हतः
श्रीशुक उवाच
स उपस्पृश्य सलिलं दंशितो धृतकार्मुकः
नय मां द्युमतः पार्श्वं वीरस्येत्याह सारथिम्‌
विधमन्तं स्वसैन्यानि द्युमन्तं रुक्मिणीसुतः
प्रतिहत्य प्रत्यविध्यान्नाराचैरष्टभिः स्मयन्‌
चतुर्भिश्चतुरो वाहान्सूतमेकेन चाहनत्‌
द्वाभ्यं धनुश्च केतुं च शरेणान्येन वै शिरः
गदसात्यकिसाम्बाद्या जघ्नुः सौभपतेर्बलम्‌
पेतुः समुद्रे सौभेयाः सर्वे सञ्छिन्नकन्धराः
एवं यदूनां शाल्वानां निघ्नतामितरेतरम्‌
युद्धं त्रिनवरात्रं तदभूत्तुमुलमुल्बणम्‌
इन्द्रप्रस्थं गतः कृष्ण आहूतो धर्मसूनुना
राजसूयेऽथ निवृत्ते शिशुपाले च संस्थिते
कुरुवृद्धाननुज्ञाप्य मुनींश्च ससुतां पृथाम्‌
निमित्तान्यतिघोराणि पश्यन्द्वारवतीं ययौ
आह चाहमिहायात आर्यमिश्राभिसङ्गतः
राजन्याश्चैद्यपक्षीया नूनं हन्युः पुरीं मम
वीक्ष्य तत्कदनं स्वानां निरूप्य पुररक्षणम्‌
सौभं च शाल्वराजं च दारुकं प्राह केशवः
रथं प्रापय मे सूत शाल्वस्यान्तिकमाशु वै
सम्भ्रमस्ते न कर्तव्यो मायावी सौभराडयम्‌
इत्युक्तश्चोदयामास रथमास्थाय दारुकः
विशन्तं ददृशुः सर्वे स्वे परे चारुणानुजम्‌
शाल्वश्च कृष्णमालोक्य हतप्रायबलेश्वरः
प्राहरत्कृष्णसूतय शक्तिं भीमरवां मृधे
तामापतन्तीं नभसि महोल्कामिव रंहसा
भासयन्तीं दिशः शौरिः सायकैः शतधाच्छिनत्‌
तं च षोडशभिर्विद्ध्वा बानैः सौभं च खे भ्रमत्‌
अविध्यच्छरसन्दोहैः खं सूर्य इव रश्मिभिः
शाल्वः शौरेस्तु दोः सव्यं सशार्ङ्गं शार्ङ्गधन्वनः
बिभेद न्यपतद्धस्ताच्छार्ङ्गमासीत्तदद्भुतम्‌
हाहाकारो महानासीद्भूतानां तत्र पश्यताम्‌
निनद्य सौभराडुच्चैरिदमाह जनार्दनम्‌
यत्त्वया मूढ नः सख्युर्भ्रातुर्भार्या हृतेक्षताम्‌
प्रमत्तः स सभामध्ये त्वया व्यापादितः सखा
तं त्वाद्य निशितैर्बाणैरपराजितमानिनम्‌
नयाम्यपुनरावृत्तिं यदि तिष्ठेर्ममाग्रतः
श्रीभगवानुवाच
वृथा त्वं कत्थसे मन्द न पश्यस्यन्तिकेऽन्तकम्‌
पौरुसं दर्शयन्ति स्म शूरा न बहुभाषिणः
इत्युक्त्वा भगवाञ्छाल्वं गदया भीमवेगया
तताड जत्रौ संरब्धः स चकम्पे वमन्नसृक्‌
गदायां सन्निवृत्तायां शाल्वस्त्वन्तरधीयत
ततो मुहूर्त आगत्य पुरुषः शिरसाच्युतम्‌
देवक्या प्रहितोऽस्मीति नत्वा प्राह वचो रुदन्‌
कृष्ण कृष्ण महाबाहो पिता ते पितृवत्सल
बद्ध्वापनीतः शाल्वेन सौनिकेन यथा पशुः
निशम्य विप्रियं कृष्णो मानुसीं प्रकृतिं गतः
विमनस्को घृणी स्नेहाद्बभाषे प्राकृतो यथा
कथं राममसम्भ्रान्तं जित्वाजेयं सुरासुरैः
शाल्वेनाल्पीयसा नीतः पिता मे बलवान्विधिः
इति ब्रुवाणे गोविन्दे सौभराट्प्रत्युपस्थितः
वसुदेवमिवानीय कृष्णं चेदमुवाच सः
एष ते जनिता तातो यदर्थमिह जीवसि
वधिष्ये वीक्षतस्तेऽमुमीशश्चेत्पाहि बालिश
एवं निर्भर्त्स्य मायावी खड्गेनानकदुन्दुभेः
उत्कृत्य शिर आदाय खस्थं सौभं समाविशत्‌
ततो मुहूर्तं प्रकृतावुपप्लुतः स्वबोध आस्ते स्वजनानुषङ्गतः
महानुभावस्तदबुध्यदासुरीं मायां स शाल्वप्रसृतां मयोदिताम्‌
न तत्र दूतं न पितुः कलेवरं प्रबुद्ध आजौ समपश्यदच्युतः
स्वाप्नं यथा चाम्बरचारिणं रिपुं सौभस्थमालोक्य निहन्तुमुद्यतः
एवं वदन्ति राजर्षे ऋषयः के च नान्विताः
यत्स्ववाचो विरुध्येत नूनं ते न स्मरन्त्युत
क्व शोकमोहौ स्नेहो वा भयं वा येऽज्ञसम्भवाः
क्व चाखण्डितविज्ञान ज्ञानैश्वर्यस्त्वखण्डितः
यत्पादसेवोर्जितयात्मविद्यया हिन्वन्त्यनाद्यात्मविपर्ययग्रहम्‌
लभन्त आत्मीयमनन्तमैश्वरं कुतो नु मोहः परमस्य सद्गतेः
तं शस्त्रपूगैः प्रहरन्तमोजसा
शाल्वं शरैः शौरिरमोघविक्रमः
विद्ध्वाच्छिनद्वर्म धनुः शिरोमणिं
सौभं च शत्रोर्गदया रुरोज ह
तत्कृष्णहस्तेरितया विचूर्णितं पपात तोये गदया सहस्रधा
विसृज्य तद्भूतलमास्थितो गदामुद्यम्य शाल्वोऽच्युतमभ्यगाद्द्रुतम्‌
आधावतः सगदं तस्य बाहुं भल्लेन छित्त्वाथ रथाङ्गमद्भुतम्‌
वधाय शाल्वस्य लयार्कसन्निभं बिभ्रद्बभौ सार्क इवोदयाचलः
जहार तेनैव शिरः सकुण्डलं किरीटयुक्तं पुरुमायिनो हरिः
वज्रेण वृत्रस्य यथा पुरन्दरो बभूव हाहेति वचस्तदा नृणाम्‌
तस्मिन्निपतिते पापे सौभे च गदया हते
नेदुर्दुन्दुभयो राजन्दिवि देवगणेरिताः
सखीनामपचितिं कुर्वन्दन्तवक्रो रुषाभ्यगात्‌
श्रीशुक उवाच
शिशुपालस्य शाल्वस्य पौण्ड्रकस्यापि दुर्मतिः
परलोकगतानां च कुर्वन्पारोक्ष्यसौहृदम्‌
एकः पदातिः सङ्क्रुद्धो गदापाणिः प्रकम्पयन्‌
पद्भ्यामिमां महाराज महासत्त्वो व्यदृश्यत
तं तथायान्तमालोक्य गदामादाय सत्वरः
अवप्लुत्य रथात्कृष्णः सिन्धुं वेलेव प्रत्यधात्‌
गदामुद्यम्य कारूषो मुकुन्दं प्राह दुर्मदः
दिष्ट्या दिष्ट्या भवानद्य मम दृष्टिपथं गतः
त्वं मातुलेयो नः कृष्ण मित्रध्रुङ्मां जिघांससि
अतस्त्वां गदया मन्द हनिष्ये वज्रकल्पया
तर्ह्यानृण्यमुपैम्यज्ञ मित्राणां मित्रवत्सलः
बन्धुरूपमरिं हत्वा व्याधिं देहचरं यथा
एवं रूक्षैस्तुदन्वाक्यैः कृष्णं तोत्रैरिव द्विपम्‌
गदयाताडयन्मूर्ध्नि सिंहवद्व्यनदच्च सः
गदयाभिहतोऽप्याजौ न चचाल यदूद्वहः
कृष्णोऽपि तमहन्गुर्व्या कौमोदक्या स्तनान्तरे
गदानिर्भिन्नहृदय उद्वमन्रुधिरं मुखात्‌
प्रसार्य केशबाह्वङ्घ्रीन्धरण्यां न्यपतद्व्यसुः
ततः सूक्ष्मतरं ज्योतिः कृष्णमाविशदद्भुतम्‌
पश्यतां सर्वभूतानां यथा चैद्यवधे नृप
विदूरथस्तु तद्भ्राता भ्रातृशोकपरिप्लुतः
आगच्छदसिचर्माभ्यामुच्छ्वसंस्तज्जिघांसया
तस्य चापततः कृष्णश्चक्रेण क्षुरनेमिना
शिरो जहार राजेन्द्र सकिरीटं सकुण्डलम्‌
एवं सौभं च शाल्वं च दन्तवक्रं सहानुजम्‌
हत्वा दुर्विषहानन्यैरीडितः सुरमानवैः
मुनिभिः सिद्धगन्धर्वैर्विद्याधरमहोरगैः
अप्सरोभिः पितृगणैर्यक्षैः किन्नरचारणैः
उपगीयमानविजयः कुसुमैरभिवर्षितः
वृतश्च वृष्णिप्रवरैर्विवेशालङ्कृतां पुरीम्‌
एवं योगेश्वरः कृष्णो भगवान्जगदीश्वरः
ईयते पशुदृष्टीनां निर्जितो जयतीति सः
श्रुत्वा युद्धोद्यमं रामः कुरूणां सह पाण्डवैः
तीर्थाभिषेकव्याजेन मध्यस्थः प्रययौ किल
स्नात्वा प्रभासे सन्तर्प्य देवर्षिपितृमानवान्‌
सरस्वतीं प्रतिस्रोतं ययौ ब्राह्मणसंवृतः
पृथूदकं बिन्दुसरस्त्रितकूपं सुदर्शनम्‌
विशालं ब्रह्मतीर्थं च चक्रं प्राचीं सरस्वतीम्‌
यमुनामनु यान्येव गङ्गामनु च भारत
जगाम नैमिषं यत्र ऋषयः सत्रमासते
तमागतमभिप्रेत्य मुनयो दीर्घसत्रिणः
अभिनन्द्य यथान्यायं प्रणम्योत्थाय चार्चयन्‌
सोऽर्चितः सपरीवारः कृतासनपरिग्रहः
रोमहर्षणमासीनं महर्षेः शिष्यमैक्षत
अप्रत्युत्थायिनं सूतमकृतप्रह्वणाञ्जलिम्‌
अध्यासीनं च तान्विप्रांश्चुकोपोद्वीक्ष्य माधवः
यस्मादसाविमान्विप्रानध्यास्ते प्रतिलोमजः
धर्मपालांस्तथैवास्मान्वधमर्हति दुर्मतिः
ऋषेर्भगवतो भूत्वा शिष्योऽधीत्य बहूनि च
सेतिहासपुराणानि धर्मशास्त्राणि सर्वशः
अदान्तस्याविनीतस्य वृथा पण्डितमानिनः
न गुणाय भवन्ति स्म नटस्येवाजितात्मनः
एतदर्थो हि लोकेऽस्मिन्नवतारो मया कृतः
वध्या मे धर्मध्वजिनस्ते हि पातकिनोऽधिकाः
एतावदुक्त्वा भगवान्निवृत्तोऽसद्वधादपि
भावित्वात्तं कुशाग्रेण करस्थेनाहनत्प्रभुः
हाहेतिवादिनः सर्वे मुनयः खिन्नमानसाः
ऊचुः सङ्कर्षणं देवमधर्मस्ते कृतः प्रभो
अस्य ब्रह्मासनं दत्तमस्माभिर्यदुनन्दन
आयुश्चात्माक्लमं तावद्यावत्सत्रं समाप्यते
अजानतैवाचरितस्त्वया ब्रह्मवधो यथा
योगेश्वरस्य भवतो नाम्नायोऽपि नियामकः
यद्येतद्ब्रह्महत्यायाः पावनं लोकपावन
चरिष्यति भवांल्लोक सङ्ग्रहोऽनन्यचोदितः
श्रीभगवानुवाच
चरिष्ये वधनिर्वेशं लोकानुग्रहकाम्यया
नियमः प्रथमे कल्पे यावान्स तु विधीयताम्‌
दीर्घमायुर्बतैतस्य सत्त्वमिन्द्रियमेव च
आशासितं यत्तद्ब्रूते साधये योगमायया
ऋषय ऊचुः
अस्त्रस्य तव वीर्यस्य मृत्योरस्माकमेव च
यथा भवेद्वचः सत्यं तथा राम विधीयताम्‌
श्रीभगवानुवाच
आत्मा वै पुत्र उत्पन्न इति वेदानुशासनम्‌
तस्मादस्य भवेद्वक्ता आयुरिन्द्रियसत्त्ववान्‌
किं वः कामो मुनिश्रेष्ठा ब्रूताहं करवाण्यथ
अजानतस्त्वपचितिं यथा मे चिन्त्यतां बुधाः
ऋषय ऊचुः
इल्वलस्य सुतो घोरो बल्वलो नाम दानवः
स दूषयति नः सत्रमेत्य पर्वणि पर्वणि
तं पापं जहि दाशार्ह तन्नः शुश्रूषणं परम्‌
पूयशोणितविन्मूत्र सुरामांसाभिवर्षिणम्‌
ततश्च भारतं वर्षं परीत्य सुसमाहितः
चरित्वा द्वादशमासांस्तीर्थस्नायी विशुध्यसि
श्रीशुक उवाच
ततः पर्वण्युपावृत्ते प्रचण्डः पांशुवर्षणः
भीमो वायुरभूद्राजन्पूयगन्धस्तु सर्वशः
ततोऽमेध्यमयं वर्षं बल्वलेन विनिर्मितम्‌
अभवद्यज्ञशालायां सोऽन्वदृश्यत शूलधृक्‌
तं विलोक्य बृहत्कायं भिन्नाञ्जनचयोपमम्‌
तप्तताम्रशिखाश्मश्रुं दंष्ट्रोग्रभ्रुकुटीमुखम्‌
सस्मार मूषलं रामः परसैन्यविदारणम्‌
हलं च दैत्यदमनं ते तूर्णमुपतस्थतुः
तमाकृष्य हलाग्रेण बल्वलं गगनेचरम्‌
मूषलेनाहनत्क्रुद्धो मूर्ध्नि ब्रह्मद्रुहं बलः
सोऽपतद्भुवि निर्भिन्न ललाटोऽसृक्समुत्सृजन्‌
मुञ्चन्नार्तस्वरं शैलो यथा वज्रहतोऽरुणः
संस्तुत्य मुनयो रामं प्रयुज्यावितथाशिषः
अभ्यषिञ्चन्महाभागा वृत्रघ्नं विबुधा यथा
वैजयन्तीं ददुर्मालां श्रीधामाम्लानपङ्कजां
रामाय वाससी दिव्ये दिव्यान्याभरणानि च
अथ तैरभ्यनुज्ञातः कौशिकीमेत्य ब्राह्मणैः
स्नात्वा सरोवरमगाद्यतः सरयूरास्रवत्‌
अनुस्रोतेन सरयूं प्रयागमुपगम्य सः
स्नात्वा सन्तर्प्य देवादीन्जगाम पुलहाश्रमम्‌
गोमतीं गण्डकीं स्नात्वा विपाशां शोण आप्लुतः
गयां गत्वा पित्ऱ्‌^ईनिष्ट्वा गङ्गासागरसङ्गमे
उपस्पृश्य महेन्द्राद्रौ रामं दृष्ट्वाभिवाद्य च
सप्तगोदावरीं वेणां पम्पां भीमरथीं ततः
स्कन्दं दृष्ट्वा ययौ रामः श्रीशैलं गिरिशालयम्‌
द्रविडेषु महापुण्यं दृष्ट्वाद्रिं वेङ्कटं प्रभुः
कामकोष्णीं पुरीं काञ्चीं कावेरीं च सरिद्वराम्‌
श्रीरन्गाख्यं महापुण्यं यत्र सन्निहितो हरिः
ऋषभाद्रिं हरेः क्षेत्रं दक्षिणां मथुरां तथा
सामुद्रं सेतुमगमत्महापातकनाशनम्‌
तत्रायुतमदाद्धेनूर्ब्राह्मणेभ्यो हलायुधः
कृतमालां ताम्रपर्णीं मलयं च कुलाचलम्‌
तत्रागस्त्यं समासीनं नमस्कृत्याभिवाद्य च
योजितस्तेन चाशीर्भिरनुज्ञातो गतोऽर्णवम्‌
दक्षिणं तत्र कन्याख्यां दुर्गां देवीं ददर्श सः
ततः फाल्गुनमासाद्य पञ्चाप्सरसमुत्तमम्‌
विष्णुः सन्निहितो यत्र स्नात्वास्पर्शद्गवायुतम्‌
ततोऽभिव्रज्य भगवान्केरलांस्तु त्रिगर्तकान्‌
गोकर्णाख्यं शिवक्षेत्रं सान्निध्यं यत्र धूर्जटेः
आर्यां द्वैपायनीं दृष्ट्वा शूर्पारकमगाद्बलः
तापीं पयोष्णीं निर्विन्ध्यामुपस्पृश्याथ दण्डकम्‌
प्रविश्य रेवामगमद्यत्र माहिष्मती पुरी
मनुतीर्थमुपस्पृश्य प्रभासं पुनरागमत्‌
श्रुत्वा द्विजैः कथ्यमानं कुरुपाण्डवसंयुगे
सर्वराजन्यनिधनं भारं मेने हृतं भुवः
स भीमदुर्योधनयोर्गदाभ्यां युध्यतोर्मृधे
वारयिष्यन्विनशनं जगाम यदुनन्दनः
युधिष्ठिरस्तु तं दृष्ट्वा यमौ कृष्णार्जुनावपि
अभिवाद्याभवंस्तुष्णीं किं विवक्षुरिहागतः
गदापाणी उभौ दृष्ट्वा संरब्धौ विजयैषिणौ
मण्डलानि विचित्राणि चरन्ताविदमब्रवीत्‌
युवां तुल्यबलौ वीरौ हे राजन्हे वृकोदर
एकं प्राणाधिकं मन्ये उतैकं शिक्षयाधिकम्‌
तस्मादेकतरस्येह युवयोः समवीर्ययोः
न लक्ष्यते जयोऽन्यो वा विरमत्वफलो रणः
न तद्वाक्यं जगृहतुर्बद्धवैरौ नृपार्थवत्‌
अनुस्मरन्तावन्योन्यं दुरुक्तं दुष्कृतानि च
दिष्टं तदनुमन्वानो रामो द्वारवतीं ययौ
उग्रसेनादिभिः प्रीतैर्ज्ञातिभिः समुपागतः
तं पुनर्नैमिषं प्राप्तमृषयोऽयाजयन्मुदा
क्रत्वङ्गं क्रतुभिः सर्वैर्निवृत्ताखिलविग्रहम्‌
तेभ्यो विशुद्धं विज्ञानं भगवान्व्यतरद्विभुः
येनैवात्मन्यदो विश्वमात्मानं विश्वगं विदुः
स्वपत्यावभृथस्नातो ज्ञातिबन्धुसुहृद्वृतः
रेजे स्वज्योत्स्नयेवेन्दुः सुवासाः सुष्ठ्वलङ्कृतः
ईदृग्विधान्यसङ्ख्यानि बलस्य बलशालिनः
अनन्तस्याप्रमेयस्य मायामर्त्यस्य सन्ति हि
योऽनुस्मरेत रामस्य कर्माण्यद्भुतकर्मणः
सायं प्रातरनन्तस्य विष्णोः स दयितो भवेत्‌
श्रीराजोवाच
भगवन्यानि चान्यानि मुकुन्दस्य महात्मनः
वीर्याण्यनन्तवीर्यस्य श्रोतुमिच्छामि हे प्रभो
को नु श्रुत्वासकृद्ब्रह्मन्नुत्तमःश्लोकसत्कथाः
विरमेत विशेषज्ञो विषण्णः काममार्गणैः
सा वाग्यया तस्य गुणान्गृणीते करौ च तत्कर्मकरौ मनश्च
स्मरेद्वसन्तं स्थिरजङ्गमेषु शृणोति तत्पुण्यकथाः स कर्णः
शिरस्तु तस्योभयलिङ्गमानमेत्तदेव यत्पश्यति तद्धि चक्षुः
अङ्गानि विष्णोरथ तज्जनानां पादोदकं यानि भजन्ति नित्यम्‌
सूत उवाच
विष्णुरातेन सम्पृष्टो भगवान्बादरायणिः
वासुदेवे भगवति निमग्नहृदयोऽब्रवीत्‌
श्रीशुक उवाच
कृष्णस्यासीत्सखा कश्चिद्ब्राह्मणो ब्रह्मवित्तमः
विरक्त इन्द्रियार्थेषु प्रशान्तात्मा जितेन्द्रियः
यदृच्छयोपपन्नेन वर्तमानो गृहाश्रमी
तस्य भार्या कुचैलस्य क्षुत्क्षामा च तथाविधा
पतिव्रता पतिं प्राह म्लायता वदनेन सा
दरिद्रं सीदमाना वै वेपमानाभिगम्य च
ननु ब्रह्मन्भगवतः सखा साक्षाच्छ्रियः पतिः
ब्रह्मण्यश्च शरण्यश्च भगवान्सात्वतर्षभः
तमुपैहि महाभाग साधूनां च परायणम्‌
दास्यति द्रविणं भूरि सीदते ते कुटुम्बिने
आस्तेऽधुना द्वारवत्यां भोजवृष्ण्यन्धकेश्वरः
स्मरतः पादकमलमात्मानमपि यच्छति
किं न्वर्थकामान्भजतो नात्यभीष्टान्जगद्गुरुः
स एवं भार्यया विप्रो बहुशः प्रार्थितो मुहुः
अयं हि परमो लाभ उत्तमःश्लोकदर्शनम्‌
इति सञ्चिन्त्य मनसा गमनाय मतिं दधे
अप्यस्त्युपायनं किञ्चिद्गृहे कल्याणि दीयताम्‌
याचित्वा चतुरो मुष्टीन्विप्रान्पृथुकतण्डुलान्‌
चैलखण्डेन तान्बद्ध्वा भर्त्रे प्रादादुपायनम्‌
स तानादाय विप्राग्र्यः प्रययौ द्वारकां किल
कृष्णसन्दर्शनं मह्यं कथं स्यादिति चिन्तयन्‌
त्रीणि गुल्मान्यतीयाय तिस्रः कक्षाश्च सद्विजः
विप्रोऽगम्यान्धकवृष्णीनां गृहेष्वच्युतधर्मिणाम्‌
गृहं द्व्यष्टसहस्राणां महिषीणां हरेर्द्विजः
विवेशैकतमं श्रीमद्ब्रह्मानन्दं गतो यथा
तं विलोक्याच्युतो दूरात्प्रियापर्यङ्कमास्थितः
सहसोत्थाय चाभ्येत्य दोर्भ्यां पर्यग्रहीन्मुदा
सख्युः प्रियस्य विप्रर्षेरङ्गसङ्गातिनिर्वृतः
प्रीतो व्यमुञ्चदब्बिन्दून्नेत्राभ्यां पुष्करेक्षणः
अथोपवेश्य पर्यङ्के स्वयम्सख्युः समर्हणम्‌
उपहृत्यावनिज्यास्य पादौ पादावनेजनीः
अग्रहीच्छिरसा राजन्भगवांल्लोकपावनः
व्यलिम्पद्दिव्यगन्धेन चन्दनागुरुकुङ्कमैः
धूपैः सुरभिभिर्मित्रं प्रदीपावलिभिर्मुदा
अर्चित्वावेद्य ताम्बूलं गां च स्वागतमब्रवीत्‌
कुचैलं मलिनं क्षामं द्विजं धमनिसन्ततम्‌
देवी पर्यचरत्साक्षाच्चामरव्यजनेन वै
अन्तःपुरजनो दृष्ट्वा कृष्णेनामलकीर्तिना
विस्मितोऽभूदतिप्रीत्या अवधूतं सभाजितम्‌
किमनेन कृतं पुण्यमवधूतेन भिक्षुणा
श्रिया हीनेन लोकेऽस्मिन्गर्हितेनाधमेन च
योऽसौ त्रिलोकगुरुणा श्रीनिवासेन सम्भृतः
पर्यङ्कस्थां श्रियं हित्वा परिष्वक्तोऽग्रजो यथा
कथयां चक्रतुर्गाथाः पूर्वा गुरुकुले सतोः
आत्मनोर्ललिता राजन्करौ गृह्य परस्परम्‌
श्रीभगवानुवाच
अपि ब्रह्मन्गुरुकुलाद्भवता लब्धदक्षिणात्‌
समावृत्तेन धर्मज्ञ भार्योढा सदृशी न वा
प्रायो गृहेषु ते चित्तमकामविहितं तथा
नैवातिप्रीयसे विद्वन्धनेषु विदितं हि मे
केचित्कुर्वन्ति कर्माणि कामैरहतचेतसः
त्यजन्तः प्रकृतीर्दैवीर्यथाहं लोकसङ्ग्रहम्‌
कच्चिद्गुरुकुले वासं ब्रह्मन्स्मरसि नौ यतः
द्विजो विज्ञाय विज्ञेयं तमसः पारमश्नुते
स वै सत्कर्मणां साक्षाद्द्विजातेरिह सम्भवः
आद्योऽङ्ग यत्राश्रमिणां यथाहं ज्ञानदो गुरुः
नन्वर्थकोविदा ब्रह्मन्वर्णाश्रमवतामिह
ये मया गुरुणा वाचा तरन्त्यञ्जो भवार्णवम्‌
नाहमिज्याप्रजातिभ्यां तपसोपशमेन वा
तुष्येयं सर्वभूतात्मा गुरुशुश्रूषया यथा
अपि नः स्मर्यते ब्रह्मन्वृत्तं निवसतां गुरौ
गुरुदारैश्चोदितानामिन्धनानयने क्वचित्‌
प्रविष्टानां महारण्यमपर्तौ सुमहद्द्विज
वातवर्षमभूत्तीव्रं निष्ठुराः स्तनयित्नवः
सूर्यश्चास्तं गतस्तावत्तमसा चावृता दिशः
निम्नं कूलं जलमयं न प्राज्ञायत किञ्चन
वयं भृशम्तत्र महानिलाम्बुभिर्निहन्यमाना महुरम्बुसम्प्लवे
दिशोऽविदन्तोऽथ परस्परं वने गृहीतहस्ताः परिबभ्रिमातुराः
एतद्विदित्वा उदिते रवौ सान्दीपनिर्गुरुः
अन्वेषमाणो नः शिष्यानाचार्योऽपश्यदातुरान्‌
अहो हे पुत्रका यूयमस्मदर्थेऽतिदुःखिताः
आत्मा वै प्राणिनाम्प्रेष्ठस्तमनादृत्य मत्पराः
एतदेव हि सच्छिष्यैः कर्तव्यं गुरुनिष्कृतम्‌
यद्वै विशुद्धभावेन सर्वार्थात्मार्पणं गुरौ
तुष्टोऽहं भो द्विजश्रेष्ठाः सत्याः सन्तु मनोरथाः
छन्दांस्ययातयामानि भवन्त्विह परत्र च
इत्थंविधान्यनेकानि वसतां गुरुवेश्मनि
गुरोरनुग्रहेणैव पुमान्पूर्णः प्रशान्तये
श्रीब्राह्मण उवाच
किमस्माभिरनिर्वृत्तं देवदेव जगद्गुरो
भवता सत्यकामेन येषां वासो गुरोरभूत्‌
यस्य च्छन्दोमयं ब्रह्म देह आवपनं विभो
श्रेयसां तस्य गुरुषु वासोऽत्यन्तविडम्बनम्‌
श्रीशुक उवाच
स इत्थं द्विजमुख्येन सह सङ्कथयन्हरिः
सर्वभूतमनोऽभिज्ञः स्मयमान उवाच तम्‌
ब्रह्मण्यो ब्राह्मणं कृष्णो भगवान्प्रहसन्प्रियम्‌
प्रेम्णा निरीक्षणेनैव प्रेक्षन्खलु सतां गतिः
श्रीभगवानुवाच
किमुपायनमानीतं ब्रह्मन्मे भवता गृहात्‌
अण्वप्युपाहृतं भक्तैः प्रेम्णा भुर्येव मे भवेत्‌
भूर्यप्यभक्तोपहृतं न मे तोषाय कल्पते
पत्रं पुष्पं फलं तोयं यो मे भक्त्या प्रयच्छति
तदहं भक्त्युपहृतमश्नामि प्रयतात्मनः
इत्युक्तोऽपि द्वियस्तस्मै व्रीडितः पतये श्रियः
पृथुकप्रसृतिं राजन्न प्रायच्छदवाङ्मुखः
सर्वभूतात्मदृक्साक्षात्तस्यागमनकारणम्‌
विज्ङायाचिन्तयन्नायं श्रीकामो माभजत्पुरा
पत्न्याः पतिव्रतायास्तु सखा प्रियचिकीर्षया
प्राप्तो मामस्य दास्यामि सम्पदोऽमर्त्यदुर्लभाः
इत्थं विचिन्त्य वसनाच्चीरबद्धान्द्विजन्मनः
स्वयं जहार किमिदमिति पृथुकतण्डुलान्‌
नन्वेतदुपनीतं मे परमप्रीणनं सखे
तर्पयन्त्यङ्ग मां विश्वमेते पृथुकतण्डुलाः
इति मुष्टिं सकृज्जग्ध्वा द्वितीयां जग्धुमाददे
तावच्छ्रीर्जगृहे हस्तं तत्परा परमेष्ठिनः
एतावतालं विश्वात्मन्सर्वसम्पत्समृद्धये
अस्मिन्लोकेऽथ वामुष्मिन्पुंसस्त्वत्तोषकारणम्‌
ब्राह्मणस्तां तु रजनीमुषित्वाच्युतमन्दिरे
भुक्त्वा पीत्वा सुखं मेने आत्मानं स्वर्गतं यथा
श्वोभूते विश्वभावेन स्वसुखेनाभिवन्दितः
जगाम स्वालयं तात पथ्यनव्रज्य नन्दितः
स चालब्ध्वा धनं कृष्णान्न तु याचितवान्स्वयम्‌
स्वगृहान्व्रीडितोऽगच्छन्महद्दर्शननिर्वृतः
अहो ब्रह्मण्यदेवस्य दृष्टा ब्रह्मण्यता मया
यद्दरिद्रतमो लक्ष्मीमाश्लिष्टो बिभ्रतोरसि
क्वाहं दरिद्रः पापीयान्क्व कृष्णः श्रीनिकेतनः
ब्रह्मबन्धुरिति स्माहं बाहुभ्यां परिरम्भितः
निवासितः प्रियाजुष्टे पर्यङ्के भ्रातरो यथा
महिष्या वीजितः श्रान्तो बालव्यजनहस्तया
शुश्रूषया परमया पादसंवाहनादिभिः
पूजितो देवदेवेन विप्रदेवेन देववत्‌
स्वर्गापवर्गयोः पुंसां रसायां भुवि सम्पदाम्‌
सर्वासामपि सिद्धीनां मूलं तच्चरणार्चनम्‌
अधनोऽयं धनं प्राप्य माद्यन्नुच्चैर्न मां स्मरेत्‌
इति कारुणिको नूनं धनं मेऽभूरि नाददात्‌
इति तच्चिन्तयन्नन्तः प्राप्तो नियगृहान्तिकम्‌
सूर्यानलेन्दुसङ्काशैर्विमानैः सर्वतो वृतम्‌
विचित्रोपवनोद्यानैः कूजद्द्विजकुलाकुलैः
प्रोत्फुल्लकमुदाम्भोज कह्लारोत्पलवारिभिः
जुष्टं स्वलङ्कृतैः पुम्भिः स्त्रीभिश्च हरिणाक्षिभिः
किमिदं कस्य वा स्थानं कथं तदिदमित्यभूत्‌
एवं मीमांसमानं तं नरा नार्योऽमरप्रभाः
प्रत्यगृह्णन्महाभागं गीतवाद्येन भूयसा
पतिमागतमाकर्ण्य पत्न्युद्धर्षातिसम्भ्रमा
निश्चक्राम गृहात्तूर्णं रूपिणी श्रीरिवालयात्‌
पतिव्रता पतिं दृष्ट्वा प्रेमोत्कण्ठाश्रुलोचना
मीलिताक्ष्यनमद्बुद्ध्या मनसा परिषस्वजे
पत्नीं वीक्ष्य विस्फुरन्तीं देवीं वैमानिकीमिव
दासीनां निष्ककण्ठीनां मध्ये भान्तीं स विस्मितः
प्रीतः स्वयं तया युक्तः प्रविष्टो निजमन्दिरम्‌
मणिस्तम्भशतोपेतं महेन्द्रभवनं यथा
पयःफेननिभाः शय्या दान्ता रुक्मपरिच्छदाः
पर्यङ्का हेमदण्डानि चामरव्यजनानि च
आसनानि च हैमानि मृदूपस्तरणानि च
मुक्तादामविलम्बीनि वितानानि द्युमन्ति च
स्वच्छस्फटिककुड्येषु महामारकतेषु च
रत्नदीपान्भ्राजमानान्ललना रत्नसंयुताः
विलोक्य ब्राह्मणस्तत्र समृद्धीः सर्वसम्पदाम्‌
तर्कयामास निर्व्यग्रः स्वसमृद्धिमहैतुकीम्‌
नूनं बतैतन्मम दुर्भगस्य शश्वद्दरिद्रस्य समृद्धिहेतुः
महाविभूतेरवलोकतोऽन्यो नैवोपपद्येत यदूत्तमस्य
नन्वब्रुवाणो दिशते समक्षं याचिष्णवे भूर्यपि भूरिभोजः
पर्जन्यवत्तत्स्वयमीक्षमाणो दाशार्हकाणामृषभः सखा मे
किञ्चित्करोत्युर्वपि यत्स्वदत्तं
सुहृत्कृतं फल्ग्वपि भूरिकारी
मयोपणीतं पृथुकैकमुष्टिं
प्रत्यग्रहीत्प्रीतियुतो महात्मा
तस्यैव मे सौहृदसख्यमैत्री दास्यं पुनर्जन्मनि जन्मनि स्यात्‌
महानुभावेन गुणालयेन विषज्जतस्तत्पुरुषप्रसङ्गः
भक्ताय चित्रा भगवान्हि सम्पदो राज्यं विभूतीर्न समर्थयत्यजः
अदीर्घबोधाय विचक्षणः स्वयं पश्यन्निपातं धनिनां मदोद्भवम्‌
इत्थं व्यवसितो बुद्ध्या भक्तोऽतीव जनार्दने
विषयान्जायया त्यक्ष्यन्बुभुजे नातिलम्पटः
तस्य वै देवदेवस्य हरेर्यज्ञपतेः प्रभोः
ब्राह्मणाः प्रभवो दैवं न तेभ्यो विद्यते परम्‌
एवं स विप्रो भगवत्सुहृत्तदा दृष्ट्वा स्वभृत्यैरजितं पराजितम्‌
तद्ध्यानवेगोद्ग्रथितात्मबन्धनस्तद्धाम लेभेऽचिरतः सतां गतिम्‌
एतद्ब्रह्मण्यदेवस्य श्रुत्वा ब्रह्मण्यतां नरः
लब्धभावो भगवति कर्मबन्धाद्विमुच्यते
श्रीशुक उवाच
अथैकदा द्वारवत्यां वसतो रामकृष्णयोः
सूर्योपरागः सुमहानासीत्कल्पक्षये यथा
तं ज्ञात्वा मनुजा राजन्पुरस्तादेव सर्वतः
समन्तपञ्चकं क्षेत्रं ययुः श्रेयोविधित्सया
निःक्षत्रियां महीं कुर्वन्रामः शस्त्रभृतां वरः
नृपाणां रुधिरौघेण यत्र चक्रे महाह्रदान्‌
ईजे च भगवान्रामो यत्रास्पृष्टोऽपि कर्मणा
लोकं सङ्ग्राहयन्नीशो यथान्योऽघापनुत्तये
महत्यां तीर्थयात्रायां तत्रागन्भारतीः प्रजाः
वृष्णयश्च तथाक्रूर वसुदेवाहुकादयः
ययुर्भारत तत्क्षेत्रं स्वमघं क्षपयिष्णवः
गदप्रद्युम्नसाम्बाद्याः सुचन्द्रशुकसारणैः
आस्तेऽनिरुद्धो रक्षायां कृतवर्मा च यूथपः
ते रथैर्देवधिष्ण्याभैर्हयैश्च तरलप्लवैः
गजैर्नदद्भिरभ्राभैर्नृभिर्विद्याधरद्युभिः
व्यरोचन्त महातेजाः पथि काञ्चनमालिनः
दिव्यस्रग्वस्त्रसन्नाहाः कलत्रैः खेचरा इव
तत्र स्नात्वा महाभागा उपोष्य सुसमाहिताः
ब्राह्मणेभ्यो ददुर्धेनूर्वासःस्रग्रुक्ममालिनीः
रामह्रदेषु विधिवत्पुनराप्लुत्य वृष्णयः
ददः स्वन्नं द्विजाग्र्येभ्यः कृष्णे नो भक्तिरस्त्विति
स्वयं च तदनुज्ञाता वृष्णयः कृष्णदेवताः
भुक्त्वोपविविशुः कामं स्निग्धच्छायाङ्घ्रिपाङ्घ्रिषु
तत्रागतांस्ते ददृशुः सुहृत्सम्बन्धिनो नृपान्‌
मत्स्योशीनरकौशल्य विदर्भकुरुसृञ्जयान्‌
काम्बोजकैकयान्मद्रान्कुन्तीनानर्तकेरलान्‌
अन्यांश्चैवात्मपक्षीयान्परांश्च शतशो नृप
नन्दादीन्सुहृदो गोपान्गोपीश्चोत्कण्ठिताश्चिरम्‌
अन्योन्यसन्दर्शनहर्षरंहसा प्रोत्फुल्लहृद्वक्त्रसरोरुहश्रियः
आश्लिष्य गाढं नयनैः स्रवज्जला हृष्यत्त्वचो रुद्धगिरो ययुर्मुदम्‌
स्त्रियश्च संवीक्ष्य मिथोऽतिसौहृद
स्मितामलापाङ्गदृशोऽभिरेभिरे
स्तनैः स्तनान्कुङ्कुमपङ्करूषितान्‌
निहत्य दोर्भिः प्रणयाश्रुलोचनाः
ततोऽभिवाद्य ते वृद्धान्यविष्ठैरभिवादिताः
स्वागतं कुशलं पृष्ट्वा चक्रुः कृष्णकथा मिथः
पृथा भ्रात्ऱ्‌^ईन्स्वस्ऱ्‌^ईर्वीक्ष्य तत्पुत्रान्पितरावपि
भ्रातृपत्नीर्मुकुन्दं च जहौ सङ्कथया शुचः
कुन्त्युवाच
आर्य भ्रातरहं मन्ये आत्मानमकृताशिषम्‌
यद्वा आपत्सु मद्वार्तां नानुस्मरथ सत्तमाः
सुहृदो ज्ञातयः पुत्रा भ्रातरः पितरावपि
नानुस्मरन्ति स्वजनं यस्य दैवमदक्षिणम्‌
श्रीवसुदेव उवाच
अम्ब मास्मानसूयेथा दैवक्रीडनकान्नरान्‌
ईशस्य हि वशे लोकः कुरुते कार्यतेऽथ वा
कंसप्रतापिताः सर्वे वयं याता दिशं दिशम्‌
एतर्ह्येव पुनः स्थानं दैवेनासादिताः स्वसः
श्रीशुक उवाच
वसुदेवोग्रसेनाद्यैर्यदुभिस्तेऽर्चिता नृपाः
आसन्नच्युतसन्दर्श परमानन्दनिर्वृताः
भीष्मो द्रोणोऽम्बिकापुत्रो गान्धारी ससुता तथा
सदाराः पाण्डवाः कुन्ती सञ्जयो विदुरः कृपः
कुन्तीभोजो विराटश्च भीष्मको नग्नजिन्महान्‌
पुरुजिद्द्रुपदः शल्यो धृष्टकेतुः स काशिराट्‌
दमघोषो विशालाक्षो मैथिलो मद्रकेकयौ
युधामन्युः सुशर्मा च ससुता बाह्लिकादयः
राजानो ये च राजेन्द्र युधिष्ठिरमनुव्रताः
श्रीनिकेतं वपुः शौरेः सस्त्रीकं वीक्ष्य विस्मिताः
अथ ते रामकृष्णाभ्यां सम्यक्प्राप्तसमर्हणाः
प्रशशंसुर्मुदा युक्ता वृष्णीन्कृष्णपरिग्रहान्‌
अहो भोजपते यूयं जन्मभाजो नृणामिह
यत्पश्यथासकृत्कृष्णं दुर्दर्शमपि योगिनाम्‌
यद्विश्रुतिः श्रुतिनुतेदमलं पुनाति
पादावनेजनपयश्च वचश्च शास्त्रम्‌
भूः कालभर्जितभगापि यदङ्घ्रिपद्म
स्पर्शोत्थशक्तिरभिवर्षति नोऽखिलार्थान्‌
तद्दर्शनस्पर्शनानुपथप्रजल्प
शय्यासनाशनसयौनसपिण्डबन्धः
येषां गृहे निरयवर्त्मनि वर्ततां वः
स्वर्गापवर्गविरमः स्वयमास विष्णुः
श्रीशुक उवाच
नन्दस्तत्र यदून्प्राप्तान्ज्ञात्वा कृष्णपुरोगमान्‌
तत्रागमद्वृतो गोपैरनःस्थार्थैर्दिदृक्षया
तं दृष्ट्वा वृष्णयो हृष्टास्तन्वः प्राणमिवोत्थिताः
परिषस्वजिरे गाढं चिरदर्शनकातराः
वसुदेवः परिष्वज्य सम्प्रीतः प्रेमविह्वलः
स्मरन्कंसकृतान्क्लेशान्पुत्रन्यासं च गोकुले
कृष्णरामौ परिष्वज्य पितरावभिवाद्य च
न किञ्चनोचतुः प्रेम्णा साश्रुकण्ठौ कुरूद्वह
तावात्मासनमारोप्य बाहुभ्यां परिरभ्य च
यशोदा च महाभागा सुतौ विजहतुः शुचः
रोहिणी देवकी चाथ परिष्वज्य व्रजेश्वरीम्‌
स्मरन्त्यौ तत्कृतां मैत्रीं बाष्पकण्ठ्यौ समूचतुः
का विस्मरेत वां मैत्रीमनिवृत्तां व्रजेश्वरि
अवाप्याप्यैन्द्रमैश्वर्यं यस्या नेह प्रतिक्रिया
एतावदृष्टपितरौ युवयोः स्म पित्रोः
सम्प्रीणनाभ्युदयपोषणपालनानि
प्राप्योषतुर्भवति पक्ष्म ह यद्वदक्ष्णोर्‌
न्यस्तावकुत्र च भयौ न सतां परः स्वः
श्रीशुक उवाच
गोप्यश्च कृष्णमुपलभ्य चिरादभीष्टं
यत्प्रेक्षणे दृशिषु पक्ष्मकृतं शपन्ति
दृग्भिर्हृदीकृतमलं परिरभ्य सर्वास्‌
तद्भावमापुरपि नित्ययुजां दुरापम्‌
भगवांस्तास्तथाभूता विविक्त उपसङ्गतः
आश्लिष्यानामयं पृष्ट्वा प्रहसन्निदमब्रवीत्‌
अपि स्मरथ नः सख्यः स्वानामर्थचिकीर्षया
गतांश्चिरायिताञ्छत्रु पक्षक्षपणचेतसः
अप्यवध्यायथास्मान्स्विदकृतज्ञाविशङ्कया
नूनं भूतानि भगवान्युनक्ति वियुनक्ति च
वायुर्यथा घनानीकं तृणं तूलं रजांसि च
संयोज्याक्षिपते भूयस्तथा भूतानि भूतकृत्‌
मयि भक्तिर्हि भूतानाममृतत्वाय कल्पते
दिष्ट्या यदासीन्मत्स्नेहो भवतीनां मदापनः
अहं हि सर्वभूतानामादिरन्तोऽन्तरं बहिः
भौतिकानां यथा खं वार्भूर्वायुर्ज्योतिरङ्गनाः
एवं ह्येतानि भूतानि भूतेष्वात्मात्मना ततः
उभयं मय्यथ परे पश्यताभातमक्षरे
श्रीशुक उवाच
अध्यात्मशिक्षया गोप्य एवं कृष्णेन शिक्षिताः
तदनुस्मरणध्वस्त जीवकोशास्तमध्यगन्‌
आहुश्च ते नलिननाभ पदारविन्दं
योगेश्वरैर्हृदि विचिन्त्यमगाधबोधैः
संसारकूपपतितोत्तरणावलम्बं
गेहं जुषामपि मनस्युदियात्सदा नः
श्रीशुक उवाच
तथानुगृह्य भगवान्गोपीनां स गुरुर्गतिः
युधिष्ठिरमथापृच्छत्सर्वांश्च सुहृदोऽव्ययम्‌
त एवं लोकनाथेन परिपृष्टाः सुसत्कृताः
प्रत्यूचुर्हृष्टमनसस्तत्पादेक्षाहतांहसः
कुतोऽशिवं त्वच्चरणाम्बुजासवं महन्मनस्तो मुखनिःसृतं क्वचित्‌
पिबन्ति ये कर्णपुटैरलं प्रभो देहंभृतां देहकृदस्मृतिच्छिदम्‌
हि त्वात्म धामविधुतात्मकृतत्र्यवस्थाम्‌
आनन्दसम्प्लवमखण्डमकुण्ठबोधम्‌
कालोपसृष्टनिगमावन आत्तयोग
मायाकृतिं परमहंसगतिं नताः स्म
श्रीऋषिरुवाच
इत्युत्तमःश्लोकशिखामणिं जनेष्व्‌
अभिष्टुवत्स्वन्धककौरवस्त्रियः
समेत्य गोविन्दकथा मिथोऽगृनंस्‌
त्रिलोकगीताः शृणु वर्णयामि ते
श्रीद्रौपद्युवाच
हे वैदर्भ्यच्युतो भद्रे हे जाम्बवति कौशले
हे सत्यभामे कालिन्दि शैब्ये रोहिणि लक्ष्मणे
हे कृष्णपत्न्य एतन्नो ब्रूते वो भगवान्स्वयम्‌
उपयेमे यथा लोकमनुकुर्वन्स्वमायया
श्रीरुक्मिण्युवाच
चैद्याय मार्पयितुमुद्यतकार्मुकेषु
राजस्वजेयभटशेखरिताङ्घ्रिरेणुः
निन्ये मृगेन्द्र इव भागमजावियूथात्‌
तच्छ्रीनिकेतचरणोऽस्तु ममार्चनाय
श्रीसत्यभामोवाच
यो मे सनाभिवधतप्तहृदा ततेन
लिप्ताभिशापमपमार्ष्टुमुपाजहार
जित्वर्क्षराजमथ रत्नमदात्स तेन
भीतः पितादिशत मां प्रभवेऽपि दत्ताम्‌
श्रीजाम्बवत्युवाच
प्राज्ञाय देहकृदमुं निजनाथदैवं
सीतापतिं त्रिनवहान्यमुनाभ्ययुध्यत्‌
ज्ञात्वा परीक्षित उपाहरदर्हणं मां
पादौ प्रगृह्य मणिनाहममुष्य दासी
श्रीकालिन्द्युवाच
तपश्चरन्तीमाज्ञाय स्वपादस्पर्शनाशया
सख्योपेत्याग्रहीत्पाणिं योऽहं तद्गृहमार्जनी
श्रीमित्रविन्दोवाच
यो मां स्वयंवर उपेत्य विजित्य भूपान्‌
निन्ये श्वयूथगं इवात्मबलिं द्विपारिः
भ्रात्ऱ्‌^ईंश्च मेऽपकुरुतः स्वपुरं श्रियौकस्‌
तस्यास्तु मेऽनुभवमङ्घ्र्यवनेजनत्वम्‌
श्रीसत्योवाच
सप्तोक्षणोऽतिबलवीर्यसुतीक्ष्णशृङ्गान्‌
पित्रा कृतान्क्षितिपवीर्यपरीक्षणाय
तान्वीरदुर्मदहनस्तरसा निगृह्य
क्रीडन्बबन्ध ह यथा शिशवोऽजतोकान्‌
य इत्थं वीर्यशुल्कां मां
दासीभिश्चतुरन्गिणीम्‌
पथि निर्जित्य राजन्यान्‌
निन्ये तद्दास्यमस्तु मे
पिता मे मातुलेयाय स्वयमाहूय दत्तवान्‌
कृष्णे कृष्णाय तच्चित्तामक्षौहिण्या सखीजनैः
अस्य मे पादसंस्पर्शो भवेज्जन्मनि जन्मनि
कर्मभिर्भ्राम्यमाणाया येन तच्छ्रेय आत्मनः
श्रीलक्ष्मणोवाच
ममापि राज्ञ्यच्युतजन्मकर्म श्रुत्वा मुहुर्नारदगीतमास ह
चित्तं मुकुन्दे किल पद्महस्तया वृतः सुसम्मृश्य विहाय लोकपान्‌
ज्ञात्वा मम मतं साध्वि पिता दुहितृवत्सलः
बृहत्सेन इति ख्यातस्तत्रोपायमचीकरत्‌
यथा स्वयंवरे राज्ञि मत्स्यः पार्थेप्सया कृतः
अयं तु बहिराच्छन्नो दृश्यते स जले परम्‌
श्रुत्वैतत्सर्वतो भूपा आययुर्मत्पितुः पुरम्‌
सर्वास्त्रशस्त्रतत्त्वज्ञाः सोपाध्यायाः सहस्रशः
पित्रा सम्पूजिताः सर्वे यथावीर्यं यथावयः
आददुः सशरं चापं वेद्धुं पर्षदि मद्धियः
आदाय व्यसृजन्केचित्सज्यं कर्तुमनीश्वराः
आकोष्ठं ज्यां समुत्कृष्य पेतुरेकेऽमुनाहताः
सज्यं कृत्वापरे वीरा मागधाम्बष्ठचेदिपाः
भीमो दुर्योधनः कर्णो नाविदंस्तदवस्थितिम्‌
मत्स्याभासं जले वीक्ष्य ज्ञात्वा च तदवस्थितिम्‌
पार्थो यत्तोऽसृजद्बाणं नाच्छिनत्पस्पृशे परम्‌
राजन्येषु निवृत्तेषु भग्नमानेषु मानिषु
भगवान्धनुरादाय सज्यं कृत्वाथ लीलया
तस्मिन्सन्धाय विशिखं मत्स्यं वीक्ष्य सकृज्जले
छित्त्वेषुणापातयत्तं सूर्ये चाभिजिति स्थिते
दिवि दुन्दुभयो नेदुर्जयशब्दयुता भुवि
देवाश्च कुसुमासारान्मुमुचुर्हर्षविह्वलाः
तद्रङ्गमाविशमहं कलनूपुराभ्यां
पद्भ्यां प्रगृह्य कनकोइज्वलरत्नमालाम्‌
नूत्ने निवीय परिधाय च कौशिकाग्र्ये
सव्रीडहासवदना कवरीधृतस्रक्‌
उन्नीय वक्त्रमुरुकुन्तलकुण्डलत्विड्‌
गण्डस्थलं शिशिरहासकटाक्षमोक्षैः
राज्ञो निरीक्ष्य परितः शनकैर्मुरारेर्‌
अंसेऽनुरक्तहृदया निदधे स्वमालाम्‌
तावन्मृदङ्गपटहाः शङ्खभेर्यानकादयः
निनेदुर्नटनर्तक्यो ननृतुर्गायका जगुः
एवं वृते भगवति मयेशे नृपयूथपाः
न सेहिरे याज्ञसेनि स्पर्धन्तो हृच्छयातुराः
मां तावद्रथमारोप्य हयरत्नचतुष्टयम्‌
शार्ङ्गमुद्यम्य सन्नद्धस्तस्थावाजौ चतुर्भुजः
दारुकश्चोदयामास काञ्चनोपस्करं रथम्‌
मिषतां भूभुजां राज्ञि मृगाणां मृगराडिव
तेऽन्वसज्जन्त राजन्या निषेद्धुं पथि केचन
संयत्ता उद्धृतेष्वासा ग्रामसिंहा यथा हरिम्‌
ते शार्ङ्गच्युतबाणौघैः कृत्तबाह्वङ्घ्रिकन्धराः
निपेतुः प्रधने केचिदेके सन्त्यज्य दुद्रुवुः
ततः पुरीं यदुपतिरत्यलङ्कृतां
रविच्छदध्वजपटचित्रतोरणाम्‌
कुशस्थलीं दिवि भुवि चाभिसंस्तुतां
समाविशत्तरणिरिव स्वकेतनम्‌
पिता मे पूजयामास सुहृत्सम्बन्धिबान्धवान्‌
महार्हवासोऽलङ्कारैः शय्यासनपरिच्छदैः
दासीभिः सर्वसम्पद्भिर्भटेभरथवाजिभिः
आयुधानि महार्हाणि ददौ पूर्णस्य भक्तितः
आत्मारामस्य तस्येमा वयं वै गृहदासिकाः
सर्वसङ्गनिवृत्त्याद्धा तपसा च बभूविम
महिष्य ऊचुः
भौमं निहत्य सगणं युधि तेन रुद्धा
ज्ञात्वाथ नः क्षितिजये जितराजकन्याः
निर्मुच्य संसृतिविमोक्षमनुस्मरन्तीः
पादाम्बुजं परिणिनाय य आप्तकामः
न वयं साध्वि साम्राज्यं स्वाराज्यं भौज्यमप्युत
वैराज्यं पारमेष्ठ्यं च आनन्त्यं वा हरेः पदम्‌
कामयामह एतस्य श्रीमत्पादरजः श्रियः
कुचकुङ्कुमगन्धाढ्यं मूर्ध्ना वोढुं गदाभृतः
व्रजस्त्रियो यद्वाञ्छन्ति पुलिन्द्यस्तृणवीरुधः
गावश्चारयतो गोपाः पदस्पर्शं महात्मनः
श्रीशुक उवाच
श्रुत्वा पृथा सुबलपुत्र्यथ याज्ञसेनी
माधव्यथ क्षितिपपत्न्य उत स्वगोप्यः
कृष्णेऽखिलात्मनि हरौ प्रणयानुबन्धं
सर्वा विसिस्म्युरलमश्रुकलाकुलाक्ष्यः
इति सम्भाषमाणासु स्त्रीभिः स्त्रीषु नृभिर्नृषु
आययुर्मुनयस्तत्र कृष्णरामदिदृक्षया
द्वैपायनो नारदश्च च्यवनो देवलोऽसितः
विश्वामित्रः शतानन्दो भरद्वाजोऽथ गौतमः
रामः सशिष्यो भगवान्वसिष्ठो गालवो भृगुः
पुलस्त्यः कश्यपोऽत्रिश्च मार्कण्डेयो बृहस्पतिः
द्वितस्त्रितश्चैकतश्च ब्रह्मपुत्रास्तथाङ्गिराः
अगस्त्यो याज्ञवल्क्यश्च वामदेवादयोऽपरे
तान्दृष्ट्वा सहसोत्थाय प्रागासीना नृपादयः
पाण्डवाः कृष्णरामौ च प्रणेमुर्विश्ववन्दितान्‌
तानानर्चुर्यथा सर्वे सहरामोऽच्युतोऽर्चयत्‌
स्वागतासनपाद्यार्घ्य माल्यधूपानुलेपनैः
उवाच सुखमासीनान्भगवान्धर्मगुप्तनुः
सदसस्तस्य महतो यतवाचोऽनुशृण्वतः
श्रीभगवानुवाच
अहो वयं जन्मभृतो लब्धं कार्त्स्न्येन तत्फलम्‌
देवानामपि दुष्प्रापं यद्योगेश्वरदर्शनम्‌
किं स्वल्पतपसां न्ऱ्‌^ईणामर्चायां देवचक्षुषाम्‌
दर्शनस्पर्शनप्रश्न प्रह्वपादार्चनादिकम्‌
न ह्यम्मयानि तीर्थानि न देवा मृच्छिलामयाः
ते पुनन्त्युरुकालेन दर्शनादेव साधवः
नाग्निर्न सूर्यो न च चन्द्रतारका
न भूर्जलं खं श्वसनोऽथ वाङ्मनः
उपासिता भेदकृतो हरन्त्यघं
विपश्चितो घ्नन्ति मुहूर्तसेवया
यस्यात्मबुद्धिः कुणपे त्रिधातुके
स्वधीः कलत्रादिषु भौम इज्यधीः
यत्तीर्थबुद्धिः सलिले न कर्हिचिज्‌
जनेष्वभिज्ञेषु स एव गोखरः
श्रीशुक उवाच
निशम्येत्थं भगवतः कृष्णस्याकुण्थमेधसः
वचो दुरन्वयं विप्रास्तूष्णीमासन्भ्रमद्धियः
चिरं विमृश्य मुनय ईश्वरस्येशितव्यताम्‌
जनसङ्ग्रह इत्यूचुः स्मयन्तस्तं जगद्गुरुम्‌
श्रीमुनय ऊचुः
यन्मायया तत्त्वविदुत्तमा वयं विमोहिता विश्वसृजामधीश्वराः
यदीशितव्यायति गूढ ईहया अहो विचित्रम्भगवद्विचेष्टितम्‌
अनीह एतद्बहुधैक आत्मना सृजत्यवत्यत्ति न बध्यते यथा
भौमैर्हि भूमिर्बहुनामरूपिणी अहो विभूम्नश्चरितं विडम्बनम्‌
अथापि काले स्वजनाभिगुप्तये बिभर्षि सत्त्वं खलनिग्रहाय च
स्वलीलया वेदपथं सनातनं वर्णाश्रमात्मा पुरुषः परो भवान्‌
ब्रह्म ते हृदयं शुक्लं तपःस्वाध्यायसंयमैः
यत्रोपलब्धं सद्व्यक्तमव्यक्तं च ततः परम्‌
तस्माद्ब्रह्मकुलं ब्रह्मन्शास्त्रयोनेस्त्वमात्मनः
सभाजयसि सद्धाम तद्ब्रह्मण्याग्रणीर्भवान्‌
अद्य नो जन्मसाफल्यं विद्यायास्तपसो दृशः
त्वया सङ्गम्य सद्गत्या यदन्तः श्रेयसां परः
नमस्तस्मै भगवते कृष्णायाकुण्ठमेधसे
स्वयोगमाययाच्छन्न महिम्ने परमात्मने
न यं विदन्त्यमी भूपा एकारामाश्च वृष्णयः
मायाजवनिकाच्छन्नमात्मानं कालमीश्वरम्‌
यथा शयानः पुरुष आत्मानं गुणतत्त्वदृक्‌
नाममात्रेन्द्रियाभातं न वेद रहितं परम्‌
एवं त्वा नाममात्रेषु विषयेष्विन्द्रियेहया
मायया विभ्रमच्चित्तो न वेद स्मृत्युपप्लवात्‌
तस्याद्य ते ददृशिमाङ्घ्रिमघौघमर्ष
तीर्थास्पदं हृदि कृतं सुविपक्वयोगैः
उत्सिक्तभक्त्युपहताशय जीवकोशा
आपुर्भवद्गतिमथानुगृहान भक्तान्‌
श्रीशुक उवाच
इत्यनुज्ञाप्य दाशार्हं धृतराष्ट्रं युधिष्ठिरम्‌
राजर्षे स्वाश्रमान्गन्तुं मुनयो दधिरे मनः
तद्वीक्ष्य तानुपव्रज्य वसुदेवो महायशाः
प्रणम्य चोपसङ्गृह्य बभाषेदं सुयन्त्रितः
श्रीवसुदेव उवाच
नमो वः सर्वदेवेभ्य ऋषयः श्रोतुमर्हथ
कर्मणा कर्मनिर्हारो यथा स्यान्नस्तदुच्यताम्‌
श्रीनारद उवाच
नातिचित्रमिदं विप्रा वसुदेवो बुभुत्सया
कृष्णम्मत्वार्भकं यन्नः पृच्छति श्रेय आत्मनः
सन्निकर्षोऽत्र मर्त्यानामनादरणकारणम्‌
गाङ्गं हित्वा यथान्याम्भस्तत्रत्यो याति शुद्धये
यस्यानुभूतिः कालेन लयोत्पत्त्यादिनास्य वै
स्वतोऽन्यस्माच्च गुणतो न कुतश्चन रिष्यति
तं क्लेशकर्मपरिपाकगुणप्रवाहैरव्याहतानुभवमीश्वरमद्वितीयम्‌
प्राणादिभिः स्वविभवैरुपगूढमन्यो मन्येत सूर्यमिव मेघहिमोपरागैः
अथोचुर्मुनयो राजन्नाभाष्यानल्सदुन्दभिम्‌
सर्वेषां शृण्वतां राज्ञां तथैवाच्युतरामयोः
कर्मणा कर्मनिर्हार एष साधुनिरूपितः
यच्छ्रद्धया यजेद्विष्णुं सर्वयज्ञेश्वरं मखैः
चित्तस्योपशमोऽयं वै कविभिः शास्त्रचक्षुसा
दर्शितः सुगमो योगो धर्मश्चात्ममुदावहः
अयं स्वस्त्ययनः पन्था द्विजातेर्गृहमेधिनः
यच्छ्रद्धयाप्तवित्तेन शुक्लेनेज्येत पूरुषः
वित्तैषणां यज्ञदानैर्गृहैर्दारसुतैषणाम्‌
आत्मलोकैषणां देव कालेन विसृजेद्बुधः
ग्रामे त्यक्तैषणाः सर्वे ययुर्धीरास्तपोवनम्‌
ऋणैस्त्रिभिर्द्विजो जातो देवर्षिपित्ऱ्‌^ईणां प्रभो
यज्ञाध्ययनपुत्रैस्तान्यनिस्तीर्य त्यजन्पतेत्‌
त्वं त्वद्य मुक्तो द्वाभ्यां वै ऋषिपित्रोर्महामते
यज्ञैर्देवर्णमुन्मुच्य निरृणोऽशरणो भव
वसुदेव भवान्नूनं भक्त्या परमया हरिम्‌
जगतामीश्वरं प्रार्चः स यद्वां पुत्रतां गतः
श्रीशुक उवाच
इति तद्वचनं श्रुत्वा वसुदेवो महामनाः
तानृषीनृत्विजो वव्रे मूर्ध्नानम्य प्रसाद्य च
त एनमृषयो राजन्वृता धर्मेण धार्मिकम्‌
तस्मिन्नयाजयन्क्षेत्रे मखैरुत्तमकल्पकैः
तद्दीक्षायां प्रवृत्तायां वृष्णयः पुष्करस्रजः
स्नाताः सुवाससो राजन्राजानः सुष्ठ्वलङ्कृताः
तन्महिष्यश्च मुदिता निष्ककण्ठ्यः सुवाससः
दीक्षाशालामुपाजग्मुरालिप्ता वस्तुपाणयः
नेदुर्मृदङ्गपटह शङ्खभेर्यानकादयः
ननृतुर्नटनर्तक्यस्तुष्टुवुः सूतमागधाः
जगुः सुकण्ठ्यो गन्धर्व्यः सङ्गीतं सहभर्तृकाः
तमभ्यषिञ्चन्विधिवदक्तमभ्यक्तमृत्विजः
पत्नीभिरष्टादशभिः सोमराजमिवोडुभिः
ताभिर्दुकूलवलयैर्हारनूपुरकुण्डलैः
स्वलङ्कृताभिर्विबभौ दीक्षितोऽजिनसंवृतः
तस्यर्त्विजो महाराज रत्नकौशेयवाससः
ससदस्या विरेजुस्ते यथा वृत्रहणोऽध्वरे
तदा रामश्च कृष्णश्च स्वैः स्वैर्बन्धुभिरन्वितौ
रेजतुः स्वसुतैर्दारैर्जीवेशौ स्वविभूतिभिः
ईजेऽनुयज्ञं विधिना अग्निहोत्रादिलक्षणैः
प्राकृतैर्वैकृतैर्यज्ञैर्द्रव्यज्ञानक्रियेश्वरम्‌
अथर्त्विग्भ्योऽददात्काले यथाम्नातं स दक्षिणाः
स्वलङ्कृतेभ्योऽलङ्कृत्य गोभूकन्या महाधनाः
पत्नीसंयाजावभृथ्यैश्चरित्वा ते महर्षयः
सस्नू रामह्रदे विप्रा यजमानपुरःसराः
स्नातोऽलङ्कारवासांसि वन्दिभ्योऽदात्तथा स्त्रियः
ततः स्वलङ्कृतो वर्णानाश्वभ्योऽन्नेन पूजयत्‌
बन्धून्सदारान्ससुतान्पारिबर्हेण भूयसा
विदर्भकोशलकुरून्काशिकेकयसृञ्जयान्‌
सदस्यर्त्विक्सुरगणान्नृभूतपितृचारणान्‌
श्रीनिकेतमनुज्ञाप्य शंसन्तः प्रययुः क्रतुम्‌
धृतराष्ट्रोऽनुजः पार्था भीष्मो द्रोणः पृथा यमौ
नारदो भगवान्व्यासः सुहृत्सम्बन्धिबान्धवाः
बन्धून्परिष्वज्य यदून्सौहृदाक्लिन्नचेतसः
ययुर्विरहकृच्छ्रेण स्वदेशांश्चापरे जनाः
नन्दस्तु सह गोपालैर्बृहत्या पूजयार्चितः
कृष्णरामोग्रसेनाद्यैर्न्यवात्सीद्बन्धुवत्सलः
वसुदेवोऽञ्जसोत्तीर्य मनोरथमहार्णवम्‌
सुहृद्वृतः प्रीतमना नन्दमाह करे स्पृशन्‌
श्रीवसुदेव उवाच
भ्रातरीशकृतः पाशो नृनां यः स्नेहसंज्ञितः
तं दुस्त्यजमहं मन्ये शूराणामपि योगिनाम्‌
अस्मास्वप्रतिकल्पेयं यत्कृताज्ञेषु सत्तमैः
मैत्र्यर्पिताफला चापि न निवर्तेत कर्हिचित्‌
प्रागकल्पाच्च कुशलं भ्रातर्वो नाचराम हि
अधुना श्रीमदान्धाक्षा न पश्यामः पुरः सतः
मा राज्यश्रीरभूत्पुंसः श्रेयस्कामस्य मानद
स्वजनानुत बन्धून्वा न पश्यति ययान्धदृक्‌
श्रीशुक उवाच
एवं सौहृदशैथिल्य चित्त आनकदुन्दुभिः
रुरोद तत्कृतां मैत्रीं स्मरन्नश्रुविलोचनः
नन्दस्तु सख्युः प्रियकृत्प्रेम्णा गोविन्दरामयोः
अद्य श्व इति मासांस्त्रीन्यदुभिर्मानितोऽवसत्‌
ततः कामैः पूर्यमाणः सव्रजः सहबान्धवः
परार्ध्याभरणक्षौम नानानर्घ्यपरिच्छदैः
वसुदेवोग्रसेनाभ्यां कृष्णोद्धवबलादिभिः
दत्तमादाय पारिबर्हं यापितो यदुभिर्ययौ
नन्दो गोपाश्च गोप्यश्च गोविन्दचरणाम्बुजे
मनः क्षिप्तं पुनर्हर्तुमनीशा मथुरां ययुः
बन्धुषु प्रतियातेषु वृष्णयः कृष्णदेवताः
वीक्ष्य प्रावृषमासन्नाद्ययुर्द्वारवतीं पुनः
जनेभ्यः कथयां चक्रुर्यदुदेवमहोत्सवम्‌
यदासीत्तीर्थयात्रायां सुहृत्सन्दर्शनादिकम्‌
श्रीबादरायणिरुवाच
अथैकदात्मजौ प्राप्तौ कृतपादाभिवन्दनौ
वसुदेवोऽभिनन्द्याह प्रीत्या सङ्कर्षणाच्युतौ
मुनीनां स वचः श्रुत्वा पुत्रयोर्धामसूचकम्‌
तद्वीर्यैर्जातविश्रम्भः परिभाष्याभ्यभाषत
कृष्ण कृष्ण महायोगिन्सङ्कर्षण सनातन
जाने वामस्य यत्साक्षात्प्रधानपुरुषौ परौ
यत्र येन यतो यस्य यस्मै यद्यद्यथा यदा
स्यादिदं भगवान्साक्षात्प्रधानपुरुषेश्वरः
एतन्नानाविधं विश्वमात्मसृष्टमधोक्षज
आत्मनानुप्रविश्यात्मन्प्राणो जीवो बिभर्ष्यज
प्राणादीनां विश्वसृजां शक्तयो याः परस्य ताः
पारतन्त्र्याद्वैसादृष्याद्द्वयोश्चेष्टैव चेष्टताम्‌
कान्तिस्तेजः प्रभा सत्ता चन्द्राग्न्यर्कर्क्षविद्युताम्‌
यत्स्थैर्यं भूभृतां भूमेर्वृत्तिर्गन्धोऽर्थतो भवान्‌
तर्पणं प्राणनमपां देव त्वं ताश्च तद्रसः
ओजः सहो बलं चेष्टा गतिर्वायोस्तवेश्वर
दिशां त्वमवकाशोऽसि दिशः खं स्फोट आश्रयः
नादो वर्णस्त्वमोंकार आकृतीनां पृथक्कृतिः
इन्द्रियं त्विन्द्रियाणां त्वं देवाश्च तदनुग्रहः
अवबोधो भवान्बुद्धेर्जीवस्यानुस्मृतिः सती
भूतानामसि भूतादिरिन्द्रियाणां च तैजसः
वैकारिको विकल्पानां प्रधानमनुशायिनम्‌
नश्वरेष्विह भावेषु तदसि त्वमनश्वरम्‌
यथा द्रव्यविकारेषु द्रव्यमात्रं निरूपितम्‌
सत्त्वम्रजस्तम इति गुणास्तद्वृत्तयश्च याः
त्वय्यद्धा ब्रह्मणि परे कल्पिता योगमायया
तस्मान्न सन्त्यमी भावा यर्हि त्वयि विकल्पिताः
त्वं चामीषु विकारेषु ह्यन्यदाव्यावहारिकः
गुणप्रवाह एतस्मिन्नबुधास्त्वखिलात्मनः
गतिं सूक्ष्मामबोधेन संसरन्तीह कर्मभिः
यदृच्छया नृतां प्राप्य सुकल्पामिह दुर्लभाम्‌
स्वार्थे प्रमत्तस्य वयो गतं त्वन्माययेश्वर
असावहम्ममैवैते देहे चास्यान्वयादिषु
स्नेहपाशैर्निबध्नाति भवान्सर्वमिदं जगत्‌
युवां न नः सुतौ साक्षात्प्रधानपुरुषेश्वरौ
भूभारक्षत्रक्षपण अवतीर्णौ तथात्थ ह
तत्ते गतोऽस्म्यरणमद्य पदारविन्दम्‌
आपन्नसंसृतिभयापहमार्तबन्धो
एतावतालमलमिन्द्रियलालसेन
मर्त्यात्मदृक्त्वयि परे यदपत्यबुद्धिः
सूतीगृहे ननु जगाद भवानजो नौ
सञ्जज्ञ इत्यनुयुगं निजधर्मगुप्त्यै
नानातनूर्गगनवद्विदधज्जहासि
को वेद भूम्न उरुगाय विभूतिमायाम्‌
श्रीशुक उवाच
आकर्ण्येत्थं पितुर्वाक्यं भगवान्सात्वतर्षभः
प्रत्याह प्रश्रयानम्रः प्रहसन्श्लक्ष्णया गिरा
श्रीभगवानुवाच
वचो वः समवेतार्थं तातैतदुपमन्महे
यन्नः पुत्रान्समुद्दिश्य तत्त्वग्राम उदाहृतः
अहं यूयमसावार्य इमे च द्वारकाउकसः
सर्वेऽप्येवं यदुश्रेष्ठ विमृग्याः सचराचरम्‌
आत्मा ह्येकः स्वयंज्योतिर्नित्योऽन्यो निर्गुणो गुणैः
आत्मसृष्टैस्तत्कृतेषु भूतेषु बहुधेयते
खं वायुर्ज्योतिरापो भूस्तत्कृतेषु यथाशयम्‌
आविस्तिरोऽल्पभूर्येको नानात्वं यात्यसावपि
श्रीशुक उवाच
एवं भगवता राजन्वसुदेव उदाहृतः
श्रुत्वा विनष्टनानाधीस्तूष्णीं प्रीतमना अभूत्‌
अथ तत्र कुरुश्रेष्ठ देवकी सर्वदेवता
श्रुत्वानीतं गुरोः पुत्रमात्मजाभ्यां सुविस्मिता
कृष्णरामौ समाश्राव्य पुत्रान्कंसविहिंसितान्‌
स्मरन्ती कृपणं प्राह वैक्लव्यादश्रुलोचना
श्रीदेवक्युवाच
राम रामाप्रमेयात्मन्कृष्ण योगेश्वरेश्वर
वेदाहं वां विश्वसृजामीश्वरावादिपूरुषौ
कलविध्वस्तसत्त्वानां राज्ञामुच्छास्त्रवर्तिनाम्‌
भूमेर्भारायमाणानामवतीर्णौ किलाद्य मे
यस्यांशांशांशभागेन विश्वोत्पत्तिलयोदयाः
भवन्ति किल विश्वात्मंस्तं त्वाद्याहं गतिं गता
चिरान्मृतसुतादाने गुरुणा किल चोदितौ
आनिन्यथुः पितृस्थानाद्गुरवे गुरुदक्षिणाम्‌
तथा मे कुरुतं कामं युवां योगेश्वरेश्वरौ
भोजराजहतान्पुत्रान्कामये द्रष्टुमाहृतान्‌
ऋषिरुवाच
एवं सञ्चोदितौ मात्रा रामः कृष्णश्च भारत
सुतलं संविविशतुर्योगमायामुपाश्रितौ
तस्मिन्प्रविष्टावुपलभ्य दैत्यराड्‌
विश्वात्मदैवं सुतरां तथात्मनः
तद्दर्शनाह्लादपरिप्लुताशयः
सद्यः समुत्थाय ननाम सान्वयः
तयोः समानीय वरासनं मुदा निविष्टयोस्तत्र महात्मनोस्तयोः
दधार पादाववनिज्य तज्जलं सवृन्द आब्रह्म पुनद्यदम्बु ह
समर्हयामास स तौ विभूतिभिर्महार्हवस्त्राभरणानुलेपनैः
ताम्बूलदीपामृतभक्षणादिभिः स्वगोत्रवित्तात्मसमर्पणेन च
स इन्द्रसेनो भगवत्पदाम्बुजं बिभ्रन्मुहुः प्रेमविभिन्नया धिया
उवाच हानन्दजलाकुलेक्षणः प्रहृष्टरोमा नृप गद्गदाक्षरम्‌
बलिरुवाच
नमोऽनन्ताय बृहते नमः कृष्णाय वेधसे
साङ्ख्ययोगवितानाय ब्रह्मणे परमात्मने
दर्शनं वां हि भूतानां दुष्प्रापं चाप्यदुर्लभम्‌
रजस्तमःस्वभावानां यन्नः प्राप्तौ यदृच्छया
दैत्यदानवगन्धर्वाः सिद्धविद्याध्रचारणाः
यक्षरक्षःपिशाचाश्च भूतप्रमथनायकाः
विशुद्धसत्त्वधाम्न्यद्धा त्वयि शास्त्रशरीरिणि
नित्यं निबद्धवैरास्ते वयं चान्ये च तादृशाः
केचनोद्बद्धवैरेण भक्त्या केचन कामतः
न तथा सत्त्वसंरब्धाः सन्निकृष्टाः सुरादयः
इदमित्थमिति प्रायस्तव योगेश्वरेश्वर
न विदन्त्यपि योगेशा योगमायां कुतो वयम्‌
तन्नः प्रसीद निरपेक्षविमृग्ययुष्मत्‌
पादारविन्दधिषणान्यगृहान्धकूपात्‌
निष्क्रम्य विश्वशरणाङ्घ्र्युपलब्धवृत्तिः
शान्तो यथैक उत सर्वसखैश्चरामि
शाध्यस्मानीशितव्येश निष्पापान्कुरु नः प्रभो
पुमान्यच्छ्रद्धयातिष्ठंश्चोदनाया विमुच्यते
श्रीभगवानुवाच
आसन्मरीचेः षट्पुत्रा ऊर्णायां प्रथमेऽन्तरे
देवाः कं जहसुर्वीक्ष्य सुतं यभितुमुद्यतम्‌
तेनासुरीमगन्योनिमधुनावद्यकर्मणा
हिरण्यकशिपोर्जाता नीतास्ते योगमायया
देवक्या उदरे जाता राजन्कंसविहिंसिताः
सा तान्शोचत्यात्मजान्स्वांस्त इमेऽध्यासतेऽन्तिके
इत एतान्प्रणेष्यामो मातृशोकापनुत्तये
ततः शापाद्विनिर्मक्ता लोकं यास्यन्ति विज्वराः
स्मरोद्गीथः परिष्वङ्गः पतङ्गः क्षुद्रभृद्घृणी
षडिमे मत्प्रसादेन पुनर्यास्यन्ति सद्गतिम्‌
इत्युक्त्वा तान्समादाय इन्द्रसेनेन पूजितौ
पुनर्द्वारवतीमेत्य मातुः पुत्रानयच्छताम्‌
तान्दृष्ट्वा बालकान्देवी पुत्रस्नेहस्नुतस्तनी
परिष्वज्याङ्कमारोप्य मूर्ध्न्यजिघ्रदभीक्ष्णशः
अपाययत्स्तनं प्रीता सुतस्पर्शपरिस्नुतम्‌
मोहिता मायया विष्णोर्यया सृष्टिः प्रवर्तते
पीत्वामृतं पयस्तस्याः पीतशेषं गदाभृतः
नारायणाङ्गसंस्पर्श प्रतिलब्धात्मदर्शनाः
ते नमस्कृत्य गोविन्दं देवकीं पितरं बलम्‌
मिषतां सर्वभूतानां ययुर्धाम दिवौकसाम्‌
तं दृष्ट्वा देवकी देवी मृतागमननिर्गमम्‌
मेने सुविस्मिता मायां कृष्णस्य रचितां नृप
एवंविधान्यद्भुतानि कृष्णस्य परमात्मनः
वीर्याण्यनन्तवीर्यस्य सन्त्यनन्तानि भारत
श्रीसूत उवाच
य इदमनुशृणोति श्रावयेद्वा मुरारेश्‌
चरितममृतकीर्तेर्वर्णितं व्यासपुत्रैः
जगदघभिदलं तद्भक्तसत्कर्णपूरं
भगवति कृतचित्तो याति तत्क्षेमधाम
श्रीराजोवाच
ब्रह्मन्वेदितुमिच्छामः स्वसारां रामकृष्णयोः
यथोपयेमे विजयो या ममासीत्पितामही
श्रीशुक उवाच
अर्जुनस्तीर्थयात्रायां पर्यटन्नवनीं प्रभुः
गतः प्रभासमशृणोन्मातुलेयीं स आत्मनः
दुर्योधनाय रामस्तां दास्यतीति न चापरे
तल्लिप्सुः स यतिर्भूत्वा त्रिदण्डी द्वारकामगात्‌
तत्र वै वार्षितान्मासानवात्सीत्स्वार्थसाधकः
पौरैः सभाजितोऽभीक्ष्णं रामेणाजानता च सः
एकदा गृहमानीय आतिथ्येन निमन्त्र्य तम्‌
श्रद्धयोपहृतं भैक्ष्यं बलेन बुभुजे किल
सोऽपश्यत्तत्र महतीं कन्यां वीरमनोहराम्‌
प्रीत्युत्फुल्लेक्षणस्तस्यां भावक्षुब्धं मनो दधे
सापि तं चकमे वीक्ष्य नारीणां हृदयंगमम्‌
हसन्ती व्रीडितापङ्गी तन्न्यस्तहृदयेक्षणा
तां परं समनुध्यायन्नन्तरं प्रेप्सुरर्जुनः
न लेभे शं भ्रमच्चित्तः कामेनातिबलीयसा
महत्यां देवयात्रायां रथस्थां दुर्गनिर्गतां
जहारानुमतः पित्रोः कृष्णस्य च महारथः
रथस्थो धनुरादाय शूरांश्चारुन्धतो भटान्‌
विद्राव्य क्रोशतां स्वानां स्वभागं मृगराडिव
तच्छ्रुत्वा क्षुभितो रामः पर्वणीव महार्णवः
गृहीतपादः कृष्णेन सुहृद्भिश्चानुसान्त्वितः
प्राहिणोत्पारिबर्हाणि वरवध्वोर्मुदा बलः
महाधनोपस्करेभ रथाश्वनरयोषितः
श्रीशुक उवाच
कृष्णस्यासीद्द्विजश्रेष्ठः श्रुतदेव इति श्रुतः
कृष्णैकभक्त्या पूर्णार्थः शान्तः कविरलम्पतः
स उवास विदेहेषु मिथिलायां गृहाश्रमी
अनीहयागताहार्य निर्वर्तितनिजक्रियः
यात्रामात्रं त्वहरहर्दैवादुपनमत्युत
नाधिकं तावता तुष्टः क्रिया चक्रे यथोचिताः
तथा तद्राष्ट्रपालोऽङ्ग बहुलाश्व इति श्रुतः
मैथिलो निरहम्मान उभावप्यच्युतप्रियौ
तयोः प्रसन्नो भगवान्दारुकेणाहृतं रथम्‌
आरुह्य साकं मुनिभिर्विदेहान्प्रययौ प्रभुः
नारदो वामदेवोऽत्रिः कृष्णो रामोऽसितोऽरुणिः
अहं बृहस्पतिः कण्वो मैत्रेयश्च्यवनादयः
तत्र तत्र तमायान्तं पौरा जानपदा नृप
उपतस्थुः सार्घ्यहस्ता ग्रहैः सूर्यमिवोदितम्‌
आनर्तधन्वकुरुजाङ्गलकङ्कमत्स्य
पाञ्चालकुन्तिमधुकेकयकोशलार्णाः
अन्ये च तन्मुखसरोजमुदारहास
स्निग्धेक्षणं नृप पपुर्दृशिभिर्न्र्नार्यः
तेभ्यः स्ववीक्षणविनष्टतमिस्रदृग्भ्यः
क्षेमं त्रिलोकगुरुरर्थदृशं च यच्छन्‌
शृण्वन्दिगन्तधवलं स्वयशोऽशुभघ्नं
गीतं सुरैर्नृभिरगाच्छनकैर्विदेहान्‌
तेऽच्युतं प्राप्तमाकर्ण्य पौरा जानपदा नृप
अभीयुर्मुदितास्तस्मै गृहीतार्हणपाणयः
दृष्ट्वा त उत्तमःश्लोकं प्रीत्युत्फुलाननाशयाः
कैर्धृताञ्जलिभिर्नेमुः श्रुतपूर्वांस्तथा मुनीन्‌
स्वानुग्रहाय सम्प्राप्तं मन्वानौ तं जगद्गुरुम्‌
मैथिलः श्रुतदेवश्च पादयोः पेततुः प्रभोः
न्यमन्त्रयेतां दाशार्हमातिथ्येन सह द्विजैः
मैथिलः श्रुतदेवश्च युगपत्संहताञ्जली
भगवांस्तदभिप्रेत्य द्वयोः प्रियचिकीर्षया
उभयोराविशद्गेहमुभाभ्यां तदलक्षितः
श्रान्तानप्यथ तान्दूराज्जनकः स्वगृहागतान्‌
आनीतेष्वासनाग्र्येषु सुखासीनान्महामनाः
प्रवृद्धभक्त्या उद्धर्ष हृदयास्राविलेक्षणः
नत्वा तदङ्घ्रीन्प्रक्षाल्य तदपो लोकपावनीः
सकुटुम्बो वहन्मूर्ध्ना पूजयां चक्र ईश्वरान्‌
गन्धमाल्याम्बराकल्प धूपदीपार्घ्यगोवृषैः
वाचा मधुरया प्रीणन्निदमाहान्नतर्पितान्‌
पादावङ्कगतौ विष्णोः संस्पृशञ्छनकैर्मुदा
श्रीबहुलाश्व उवाच
भवान्हि सर्वभूतानामात्मा साक्षी स्वदृग्विभो
अथ नस्त्वत्पदाम्भोजं स्मरतां दर्शनं गतः
स्ववचस्तदृतं कर्तुमस्मद्दृग्गोचरो भवान्‌
यदात्थैकान्तभक्तान्मे नानन्तः श्रीरजः प्रियः
को नु त्वच्चरणाम्भोजमेवंविद्विसृजेत्पुमान्‌
निष्किञ्चनानां शान्तानां मुनीनां यस्त्वमात्मदः
योऽवतीर्य यदोर्वंशे नृणां संसरतामिह
यशो वितेने तच्छान्त्यै त्रैलोक्यवृजिनापहम्‌
नमस्तुभ्यं भगवते कृष्णायाकुण्ठमेधसे
नारायणाय ऋषये सुशान्तं तप ईयुषे
दिनानि कतिचिद्भूमन्गृहान्नो निवस द्विजैः
समेतः पादरजसा पुनीहीदं निमेः कुलम्‌
इत्युपामन्त्रितो राज्ञा भगवांल्लोकभावनः
उवास कुर्वन्कल्याणं मिथिलानरयोषिताम्‌
श्रुतदेवोऽच्युतं प्राप्तं स्वगृहाञ्जनको यथा
नत्वा मुनीन्सुसंहृष्टो धुन्वन्वासो ननर्त ह
तृणपीठबृषीष्वेतानानीतेषूपवेश्य सः
स्वागतेनाभिनन्द्याङ्घ्रीन्सभार्योऽवनिजे मुदा
तदम्भसा महाभाग आत्मानं सगृहान्वयम्‌
स्नापयां चक्र उद्धर्षो लब्धसर्वमनोरथः
फलार्हणोशीरशिवामृताम्बुभिर्मृदा सुरभ्या तुलसीकुशाम्बुयैः
आराधयामास यथोपपन्नया सपर्यया सत्त्वविवर्धनान्धसा
स तर्कयामास कुतो ममान्वभूत्गृहान्धकुपे पतितस्य सङ्गमः
यः सर्वतीर्थास्पदपादरेणुभिः कृष्णेन चास्यात्मनिकेतभूसुरैः
सूपविष्टान्कृतातिथ्यान्श्रुतदेव उपस्थितः
सभार्यस्वजनापत्य उवाचाङ्घ्र्यभिमर्शनः
श्रुतदेव उवाच
नाद्य नो दर्शनं प्राप्तः परं परमपूरुषः
यर्हीदं शक्तिभिः सृष्ट्वा प्रविष्टो ह्यात्मसत्तया
यथा शयानः पुरुषो मनसैवात्ममायया
सृष्ट्वा लोकं परं स्वाप्नमनुविश्यावभासते
शृण्वतां गदतां शश्वदर्चतां त्वाभिवन्दताम्‌
णृणां संवदतामन्तर्हृदि भास्यमलात्मनाम्‌
हृदिस्थोऽप्यतिदूरस्थः कर्मविक्षिप्तचेतसाम्‌
आत्मशक्तिभिरग्राह्योऽप्यन्त्युपेतगुणात्मनाम्‌
नमोऽस्तु तेऽध्यात्मविदां परात्मने
अनात्मने स्वात्मविभक्तमृत्यवे
सकारणाकारणलिङ्गमीयुषे
स्वमाययासंवृतरुद्धदृष्टये
स त्वं शाधि स्वभृत्यान्नः किं देव करवाम हे
एतदन्तो नृणां क्लेशो यद्भवानक्षिगोचरः
श्रीशुक उवाच
तदुक्तमित्युपाकर्ण्य भगवान्प्रणतार्तिहा
गृहीत्वा पाणिना पाणिं प्रहसंस्तमुवाच ह
श्रीभगवानुवाच
ब्रह्मंस्तेऽनुग्रहार्थाय सम्प्राप्तान्विद्ध्यमून्मुनीन्‌
सञ्चरन्ति मया लोकान्पुनन्तः पादरेणुभिः
देवाः क्षेत्राणि तीर्थानि दर्शनस्पर्शनार्चनैः
शनैः पुनन्ति कालेन तदप्यर्हत्तमेक्षया
ब्राह्मणो जन्मना श्रेयान्सर्वेषाम्प्राणिनामिह
तपसा विद्यया तुष्ट्या किमु मत्कलया युतः
न ब्राह्मणान्मे दयितं रूपमेतच्चतुर्भुजम्‌
सर्ववेदमयो विप्रः सर्वदेवमयो ह्यहम्‌
दुष्प्रज्ञा अविदित्वैवमवजानन्त्यसूयवः
गुरुं मां विप्रमात्मानमर्चादाविज्यदृष्टयः
चराचरमिदं विश्वं भावा ये चास्य हेतवः
मद्रूपाणीति चेतस्याधत्ते विप्रो मदीक्षया
तस्माद्ब्रह्मऋषीनेतान्ब्रह्मन्मच्छ्रद्धयार्चय
एवं चेदर्चितोऽस्म्यद्धा नान्यथा भूरिभूतिभिः
श्रीशुक उवाच
स इत्थं प्रभुनादिष्टः सहकृष्णान्द्विजोत्तमान्‌
आराध्यैकात्मभावेन मैथिलश्चाप सद्गतिम्‌
एवं स्वभक्तयो राजन्भगवान्भक्तभक्तिमान्‌
उषित्वादिश्य सन्मार्गं पुनर्द्वारवतीमगात्‌
श्रीपरीक्षिदुवाच
ब्रह्मन्ब्रह्मण्यनिर्देश्ये निर्गुणे गुणवृत्तयः
कथं चरन्ति श्रुतयः साक्षात्सदसतः परे
श्रीशुक उवाच
बुद्धीन्द्रियमनःप्राणान्जनानामसृजत्प्रभुः
मात्रार्थं च भवार्थं च आत्मनेऽकल्पनाय च
सैषा ह्युपनिषद्ब्राह्मी पूर्वेशां पूर्वजैर्धृता
श्र्रद्धया धारयेद्यस्तां क्षेमं गच्छेदकिञ्चनः
अत्र ते वर्णयिष्यामि गाथां नारायणान्विताम्‌
नारदस्य च संवादमृषेर्नारायणस्य च
एकदा नारदो लोकान्पर्यटन्भगवत्प्रियः
सनातनमृषिं द्रष्टुं ययौ नारायणाश्रमम्‌
यो वै भारतवर्षेऽस्मिन्क्षेमाय स्वस्तये नृणाम्‌
धर्मज्ञानशमोपेतमाकल्पादास्थितस्तपः
तत्रोपविष्टमृषिभिः कलापग्रामवासिभिः
परीतं प्रणतोऽपृच्छदिदमेव कुरूद्वह
तस्मै ह्यवोचद्भगवानृषीणां शृण्वतामिदम्‌
यो ब्रह्मवादः पूर्वेषां जनलोकनिवासिनाम्‌
श्रीभगवानुवाच
स्वायम्भुव ब्रह्मसत्रं जनलोकेऽभवत्पुरा
तत्रस्थानां मानसानां मुनीनामूर्ध्वरेतसाम्‌
श्वेतद्वीपं गतवति त्वयि द्रष्टुं तदीश्वरम्‌
ब्रह्मवादः सुसंवृत्तः श्रुतयो यत्र शेरते
तत्र हायमभूत्प्रश्नस्त्वं मां यमनुपृच्छसि
तुल्यश्रुततपःशीलास्तुल्यस्वीयारिमध्यमाः
अपि चक्रुः प्रवचनमेकं शुश्रूषवोऽपरे
श्रीसनन्दन उवाच
स्वसृष्टमिदमापीय शयानं सह शक्तिभिः
तदन्ते बोधयां चक्रुस्तल्लिङ्गैः श्रुतयः परम्‌
यथा शयानं संराजं वन्दिनस्तत्पराक्रमैः
प्रत्यूषेऽभेत्य सुश्लोकैर्बोधयन्त्यनुजीविनः
श्रीश्रुतय ऊचुः
जय जय जह्यजामजित दोषगृभीतगुणां
त्वमसि यदात्मना समवरुद्धसमस्तभगः
अगजगदोकसामखिलशक्त्यवबोधक ते
क्वचिदजयात्मना च चरतोऽनुचरेन्निगमः
बृहदुपलब्धमेतदवयन्त्यवशेषतया
यत उदयास्तमयौ विकृतेर्मृदि वाविकृतात्‌
अत ऋषयो दधुस्त्वयि मनोवचनाचरितं
कथमयथा भवन्ति भुवि दत्तपदानि नृणाम्‌
इति तव सूरयस्त्र्यधिपतेऽखिललोकमल
क्षपणकथामृताब्धिमवगाह्य तपांसि जहुः
किमुत पुनः स्वधामविधुताशयकालगुणाः
परम भजन्ति ये पदमजस्रसुखानुभवम्‌
दृतय इव श्वसन्त्यसुभृतो यदि तेऽनुविधा
महदहमादयोऽण्डमसृजन्यदनुग्रहतः
पुरुषविधोऽन्वयोऽत्र चरमोऽन्नमयादिषु यः
सदसतः परं त्वमथ यदेष्ववशेषमृतम्‌
उदरमुपासते य ऋषिवर्त्मसु कूर्पदृशः
परिसरपद्धतिं हृदयमारुणयो दहरम्‌
तत उदगादनन्त तव धाम शिरः परमं
पुनरिह यत्समेत्य न पतन्ति कृतान्तमुखे
स्वकृतविचित्रयोनिषु विशन्निव हेतुतया
तरतमतश्चकास्स्यनलवत्स्वकृतानुकृतिः
अथ वितथास्वमूष्ववितथां तव धाम समं
विरजधियोऽनुयन्त्यभिविपण्यव एकरसम्‌
स्वकृतपुरेष्वमीष्वबहिरन्तरसंवरणं
तव पुरुषं वदन्त्यखिलशक्तिधृतोऽंशकृतम्‌
इति नृगतिं विविच्य कवयो निगमावपनं
भवत उपासतेऽङ्घ्रिमभवम्भुवि विश्वसिताः
दुरवगमात्मतत्त्वनिगमाय तवात्ततनोश्‌
चरितमहामृताब्धिपरिवर्तपरिश्रमणाः
न परिलषन्ति केचिदपवर्गमपीश्वर ते
चरणसरोजहंसकुलसङ्गविसृष्टगृहाः
त्वदनुपथं कुलायमिदमात्मसुहृत्प्रियवच्‌
चरति तथोन्मुखे त्वयि हिते प्रिय आत्मनि च
न बत रमन्त्यहो असदुपासनयात्महनो
यदनुशया भ्रमन्त्युरुभये कुशरीरभृतः
निभृतमरुन्मनोऽक्षदृढयोगयुजो हृदि यन्‌
मुनय उपासते तदरयोऽपि ययुः स्मरणात्‌
स्त्रिय उरगेन्द्रभोगभुजदण्डविषक्तधियो
वयमपि ते समाः समदृशोऽङ्घ्रिसरोजसुधाः
क इह नु वेद बतावरजन्मलयोऽग्रसरं
यत उदगादृषिर्यमनु देवगणा उभये
तर्हि न सन्न चासदुभयं न च कालजवः
किमपि न तत्र शास्त्रमवकृष्य शयीत यदा
जनिमसतः सतो मृतिमुतात्मनि ये च भिदां
विपणमृतं स्मरन्त्युपदिशन्ति त आरुपितैः
त्रिगुणमयः पुमानिति भिदा यदबोधकृता
त्वयि न ततः परत्र स भवेदवबोधरसे
सदिव मनस्त्रिवृत्त्वयि विभात्यसदामनुजात्‌
सदभिमृशन्त्यशेषमिदमात्मतयात्मविदः
न हि विकृतिं त्यजन्ति कनकस्य तदात्मतया
स्वकृतमनुप्रविष्टमिदमात्मतयावसितम्‌
तव परि ये चरन्त्यखिलसत्त्वनिकेततया
त उत पदाक्रमन्त्यविगणय्य शिरो निरृतेः
परिवयसे पशूनिव गिरा विबुधानपि तांस्‌
त्वयि कृतसौहृदाः खलु पुनन्ति न ये विमुखाः
त्वमकरणः स्वराडखिलकारकशक्तिधरस्‌
तव बलिमुद्वहन्ति समदन्त्यजयानिमिषाः
वर्षभुजोऽखिलक्षितिपतेरिव विश्वसृजो
विदधति यत्र ये त्वधिकृता भवतश्चकिताः
स्थिरचरजातयः स्युरजयोत्थनिमित्तयुजो
विहर उदीक्षया यदि परस्य विमुक्त ततः
न हि परमस्य कश्चिदपरो न परश्च भवेद्‌
वियत इवापदस्य तव शून्यतुलां दधतः
अपरिमिता ध्रुवास्तनुभृतो यदि सर्वगतास्‌
तर्हि न शास्यतेति नियमो ध्रव नेतरथा
अजनि च यन्मयं तदविमुच्य नियन्तृ भवेत्‌
सममनुजानतां यदमतं मतदुष्टतया
न घटत उद्भवः प्रकृतिपूरुषयोरजयोर्‌
उभययुजा भवन्त्यसुभृतो जलबुद्बुदवत्‌
त्वयि त इमे ततो विविधनामगुणैः परमे
सरित इवार्णवे मधुनि लिल्युरशेषरसाः
नृषु तव मयया भ्रमममीष्ववगत्य भृशं
त्वयि सुधियोऽभवे दधति भावमनुप्रभवम्‌
कथमनुवर्ततां भवभयं तव यद्भ्रुकुटिः
सृजति मुहुस्त्रिनेमिरभवच्छरणेषु भयम्‌
विजितहृषीकवायुभिरदान्तमनस्तुरगं
य इह यतन्ति यन्तुमतिलोलमुपायखिदः
व्यसनशतान्विताः समवहाय गुरोश्चरणं
वणिज इवाज सन्त्यकृतकर्णधरा जलधौ
स्वजनसुतात्मदारधनधामधरासुरथैस्‌
त्वयि सति किं नृणाम्श्रयत आत्मनि सर्वरसे
इति सदजानतां मिथुनतो रतये चरतां
सुखयति को न्विह स्वविहते स्वनिरस्तभगे
भुवि पुरुपुण्यतीर्थसदनान्यृषयो विमदास्‌
त उत भवत्पदाम्बुजहृदोऽघभिदङ्घ्रिजलाः
दधति सकृन्मनस्त्वयि य आत्मनि नित्यसुखे
न पुनरुपासते पुरुषसारहरावसथान्‌
सत इदं उत्थितं सदिति चेन्ननु तर्कहतं
व्यभिचरति क्व च क्व च मृषा न तथोभययुक्‌
व्यवहृतये विकल्प इषितोऽन्धपरम्परया
भ्रमयति भारती त उरुवृत्तिभिरुक्थजडान्‌
न यदिदमग्र आस न भविष्यदतो निधनाद्‌
अनु मितमन्तरा त्वयि विभाति मृषैकरसे
अत उपमीयते द्रविणजातिविकल्पपथैर्‌
वितथमनोविलासमृतमित्यवयन्त्यबुधाः
स यदजया त्वजामनुशयीत गुणांश्च जुषन्‌
भजति सरूपतां तदनु मृत्युमपेतभगः
त्वमुत जहासि तामहिरिव त्वचमात्तभगो
महसि महीयसेऽष्टगुणितेऽपरिमेयभगः
यदि न समुद्धरन्ति यतयो हृदि कामजटा
दुरधिगमोऽसतां हृदि गतोऽस्मृतकण्ठमणिः
असुतृपयोगिनामुभयतोऽप्यसुखं भगवन्न्‌
अनपगतान्तकादनधिरूढपदाद्भवतः
त्वदवगमी न वेत्ति भवदुत्थशुभाशुभयोर्‌
गुणविगुणान्वयांस्तर्हि देहभृतां च गिरः
अनुयुगमन्वहं सगुण गीतपरम्परया
श्रवणभृतो यतस्त्वमपवर्गगतिर्मनुजैः
द्युपतय एव ते न ययुरन्तमनन्ततया
त्वमपि यदन्तराण्डनिचया ननु सावरणाः
ख इव रजांसि वान्ति वयसा सह यच्छ्रुतयस्‌
त्वयि हि फलन्त्यतन्निरसनेन भवन्निधनाः
श्रीभगवानुवाच
इत्येतद्ब्रह्मणः पुत्रा आश्रुत्यात्मानुशासनम्‌
सनन्दनमथानर्चुः सिद्धा ज्ञात्वात्मनो गतिम्‌
इत्यशेषसमाम्नाय पुराणोपनिषद्रसः
समुद्धृतः पूर्वजातैर्व्योमयानैर्महात्मभिः
त्वं चैतद्ब्रह्मदायाद श्रद्धयात्मानुशासनम्‌
धारयंश्चर गां कामं कामानां भर्जनं नृणाम्‌
श्रीशुक उवाच
एवं स ऋषिणादिष्टं गृहीत्वा श्रद्धयात्मवान्‌
पूर्णः श्रुतधरो राजन्नाह वीरव्रतो मुनिः
श्रीनारद उवाच
नमस्तस्मै भगवते कृष्णायामलकीर्तये
यो धत्ते सर्वभूतानामभवायोशतीः कलाः
इत्याद्यमृषिमानम्य तच्छिष्यांश्च महात्मनः
ततोऽगादाश्रमं साक्षात्पितुर्द्वैपायनस्य मे
सभाजितो भगवता कृतासनपरिग्रहः
तस्मै तद्वर्णयामास नारायणमुखाच्छ्रुतम्‌
इत्येतद्वर्णितं राजन्यन्नः प्रश्नः कृतस्त्वया
यथा ब्रह्मण्यनिर्देश्ये नीऋगुणेऽपि मनश्चरेत्‌
योऽस्योत्प्रेक्षक आदिमध्यनिधने योऽव्यक्तजीवेश्वरो
यः सृष्ट्वेदमनुप्रविश्य ऋषिणा चक्रे पुरः शास्ति ताः
यं सम्पद्य जहात्यजामनुशयी सुप्तः कुलायं यथा
तं कैवल्यनिरस्तयोनिमभयं ध्यायेदजस्रं हरिम्‌
श्रीराजोवाच
देवासुरमनुष्येसु ये भजन्त्यशिवं शिवम्‌
प्रायस्ते धनिनो भोजा न तु लक्ष्म्याः पतिं हरिम्‌
एतद्वेदितुमिच्छामः सन्देहोऽत्र महान्हि नः
विरुद्धशीलयोः प्रभ्वोर्विरुद्धा भजतां गतिः
श्रीशुक उवाच
शिवः शक्तियुतः शश्वत्त्रिलिङ्गो गुणसंवृतः
वैकारिकस्तैजसश्च तामसश्चेत्यहं त्रिधा
ततो विकारा अभवन्षोडशामीषु कञ्चन
उपधावन्विभूतीनां सर्वासामश्नुते गतिम्‌
हरिर्हि निर्गुणः साक्षात्पुरुषः प्रकृतेः परः
स सर्वदृगुपद्रष्टा तं भजन्निर्गुणो भवेत्‌
निवृत्तेष्वश्वमेधेषु राजा युष्मत्पितामहः
शृण्वन्भगवतो धर्मानपृच्छदिदमच्युतम्‌
स आह भगवांस्तस्मै प्रीतः शुश्रूषवे प्रभुः
नृणां निःश्रेयसार्थाय योऽवतीर्णो यदोः कुले
श्रीभगवानुवाच
यस्याहमनुगृह्णामि हरिष्ये तद्धनं शनैः
ततोऽधनं त्यजन्त्यस्य स्वजना दुःखदुःखितम्‌
स यदा वितथोद्योगो निर्विण्णः स्याद्धनेहया
मत्परैः कृतमैत्रस्य करिष्ये मदनुग्रहम्‌
तद्ब्रह्म परमं सूक्ष्मं चिन्मात्रं सदनन्तकम्‌
विज्ञायात्मतया धीरः संसारात्परिमुच्यते
अतो मां सुदुराराध्यं हित्वान्यान्भजते जनः
ततस्त आशुतोषेभ्यो लब्धराज्यश्रियोद्धताः
मत्ताः प्रमत्ता वरदान्विस्मयन्त्यवजानते
श्रीशुक उवाच
शापप्रसादयोरीशा ब्रह्मविष्णुशिवादयः
सद्यः शापप्रसादोऽङ्ग शिवो ब्रह्मा न चाच्युतः
अत्र चोदाहरन्तीममितिहासं पुरातनम्‌
वृकासुराय गिरिशो वरं दत्त्वाप सङ्कटम्‌
वृको नामासुरः पुत्रः शकुनेः पथि नारदम्‌
दृष्ट्वाशुतोषं पप्रच्छ देवेषु त्रिषु दुर्मतिः
स आह देवं गिरिशमुपाधावाशु सिद्ध्यसि
योऽल्पाभ्यां गुणदोषाभ्यामाशु तुष्यति कुप्यति
दशास्यबाणयोस्तुष्टः स्तुवतोर्वन्दिनोरिव
ऐश्वर्यमतुलं दत्त्वा तत आप सुसङ्कटम्‌
इत्यादिष्टस्तमसुर उपाधावत्स्वगात्रतः
केदार आत्मक्रव्येण जुह्वानो ग्निमुखं हरम्‌
देवोपलब्धिमप्राप्य निर्वेदात्सप्तमेऽहनि
शिरोऽवृश्चत्सुधितिना तत्तीर्थक्लिन्नमूर्धजम्‌
तदा महाकारुणिको स धूर्जटिर्यथा वयं चाग्निरिवोत्थितोऽनलात्‌
निगृह्य दोर्भ्यां भुजयोर्न्यवारयत्तत्स्पर्शनाद्भूय उपस्कृताकृतिः
तमाह चाङ्गालमलं वृणीष्व मे यथाभिकामं वितरामि ते वरम्‌
प्रीयेय तोयेन नृणां प्रपद्यतामहो त्वयात्मा भृशमर्द्यते वृथा
देवं स वव्रे पापीयान्वरं भूतभयावहम्‌
यस्य यस्य करं शीर्ष्णि धास्ये स म्रियतामिति
तच्छ्रुत्वा भगवान्रुद्रो दुर्मना इव भारत
ओं इति प्रहसंस्तस्मै ददेऽहेरमृतं यथा
स तद्वरपरीक्षार्थं शम्भोर्मूर्ध्नि किलासुरः
स्वहस्तं धातुमारेभे सोऽबिभ्यत्स्वकृताच्छिवः
तेनोपसृष्टः सन्त्रस्तः पराधावन्सवेपथुः
यावदन्तं दिवो भूमेः कष्ठानामुदगादुदक्‌
अजानन्तः प्रतिविधिं तूष्णीमासन्सुरेश्वराः
ततो वैकुण्ठमगमद्भास्वरं तमसः परम्‌
यत्र नारायणः साक्षान्न्यासिनां परमो गतिः
शान्तानां न्यस्तदण्डानां यतो नावर्तते गतः
तं तथा व्यसनं दृष्ट्वा भगवान्वृजिनार्दनः
दूरात्प्रत्युदियाद्भूत्वा बटुको योगमायया
मेखलाजिनदण्डाक्षैस्तेजसाग्निरिव ज्वलन्‌
अभिवादयामास च तं कुशपाणिर्विनीतवत्‌
श्रीभगवानुवाच
शाकुनेय भवान्व्यक्तं श्रान्तः किं दूरमागतः
क्षणं विश्रम्यतां पुंस आत्मायं सर्वकामधुक्‌
यदि नः श्रवणायालं युष्मद्व्यवसितं विभो
भण्यतां प्रायशः पुम्भिर्धृतैः स्वार्थान्समीहते
श्रीशुक उवाच
एवं भगवता पृष्टो वचसामृतवर्षिणा
गतक्लमोऽब्रवीत्तस्मै यथापूर्वमनुष्ठितम्‌
श्रीभगवानुवाच
एवं चेत्तर्हि तद्वाक्यं न वयं श्रद्दधीमहि
यो दक्षशापात्पैशाच्यं प्राप्तः प्रेतपिशाचराट्‌
यदि वस्तत्र विश्रम्भो दानवेन्द्र जगद्गुरौ
तर्ह्यङ्गाशु स्वशिरसि हस्तं न्यस्य प्रतीयताम्‌
यद्यसत्यं वचः शम्भोः कथञ्चिद्दानवर्षभ
तदैनं जह्यसद्वाचं न यद्वक्तानृतं पुनः
इत्थं भगवतश्चित्रैर्वचोभिः स सुपेशलैः
भिन्नधीर्विस्मृतः शीर्ष्णि स्वहस्तं कुमतिर्न्यधात्‌
अथापतद्भिन्नशिराः व्रजाहत इव क्षणात्‌
जयशब्दो नमःशब्दः साधुशब्दोऽभवद्दिवि
मुमुचुः पुष्पवर्षाणि हते पापे वृकासुरे
देवर्षिपितृगन्धर्वा मोचितः सङ्कटाच्छिवः
मुक्तं गिरिशमभ्याह भगवान्पुरुषोत्तमः
अहो देव महादेव पापोऽयं स्वेन पाप्मना
हतः को नु महत्स्वीश जन्तुर्वै कृतकिल्बिषः
क्षेमी स्यात्किमु विश्वेशे कृतागस्को जगद्गुरौ
य एवमव्याकृतशक्त्युदन्वतः परस्य साक्षात्परमात्मनो हरेः
गिरित्रमोक्षं कथयेच्छृणोति वा विमुच्यते संसृतिभिस्तथारिभिः
श्रीशुक उवाच
सरस्वत्यास्तटे राजन्नृषयः सत्रमासत
वितर्कः समभूत्तेषां त्रिष्वधीशेषु को महान्‌
तस्य जिज्ञासया ते वै भृगुं ब्रह्मसुतं नृप
तज्ज्ञप्त्यै प्रेषयामासुः सोऽभ्जगाद्ब्रह्मणः सभाम्‌
न तस्मै प्रह्वणं स्तोत्रं चक्रे सत्त्वपरीक्षया
तस्मै चुक्रोध भगवान्प्रज्वलन्स्वेन तेजसा
स आत्मन्युत्थितम्मन्युमात्मजायात्मना प्रभुः
अशीशमद्यथा वह्निं स्वयोन्या वारिणात्मभूः
ततः कैलासमगमत्स तं देवो महेश्वरः
परिरब्धुं समारेभ उत्थाय भ्रातरं मुदा
नैच्छत्त्वमस्युत्पथग इति देवश्चुकोप ह
शूलमुद्यम्य तं हन्तुमारेभे तिग्मलोचनः
पतित्वा पादयोर्देवी सान्त्वयामास तं गिरा
अथो जगाम वैकुण्ठं यत्र देवो जनार्दनः
शयानं श्रिय उत्सङ्गे पदा वक्षस्यताडयत्‌
तत उत्थाय भगवान्सह लक्ष्म्या सतां गतिः
स्वतल्पादवरुह्याथ ननाम शिरसा मुनिम्‌
आह ते स्वागतं ब्रह्मन्निषीदात्रासने क्षणम्‌
अजानतामागतान्वः क्षन्तुमर्हथ नः प्रभो
पुनीहि सहलोकं मां लोकपालांश्च मद्गतान्‌
पादोदकेन भवतस्तीर्थानां तीर्थकारिणा
अद्याहं भगवंल्लक्ष्म्या आसमेकान्तभाजनम्‌
वत्स्यत्युरसि मे भूतिर्भवत्पादहतांहसः
श्रीशुक उवाच
एवं ब्रुवाणे वैकुण्ठे भृगुस्तन्मन्द्रया गिरा
निर्वृतस्तर्पितस्तूष्णीं भक्त्युत्कण्ठोऽश्रुलोचनः
पुनश्च सत्रमाव्रज्य मुनीनां ब्रह्मवादिनाम्‌
स्वानुभूतमशेषेण राजन्भृगुरवर्णयत्‌
तन्निशम्याथ मुनयो विस्मिता मुक्तसंशयाः
भूयांसं श्रद्दधुर्विष्णुं यतः शान्तिर्यतोऽभयम्‌
धर्मः साक्षाद्यतो ज्ञानं वैराग्यं च तदन्वितम्‌
ऐश्वर्यं चाष्टधा यस्माद्यशश्चात्ममलापहम्‌
मुनीनां न्यस्तदण्डानां शान्तानां समचेतसाम्‌
अकिञ्चनानां साधूनां यमाहुः परमां गतिम्‌
सत्त्वं यस्य प्रिया मूर्तिर्ब्राह्मणास्त्विष्टदेवताः
भजन्त्यनाशिषः शान्ता यं वा निपुणबुद्धयः
त्रिविधाकृतयस्तस्य राक्षसा असुराः सुराः
गुणिन्या मायया सृष्टाः सत्त्वं तत्तीर्थसाधनम्‌
श्रीशुक उवाच
इत्थं सारस्वता विप्रा नृणाम्संशयनुत्तये
पुरुषस्य पदाम्भोज सेवया तद्गतिं गताः
श्रीसूत उवाच
इत्येतन्मुनितनयास्यपद्मगन्ध
पीयूषं भवभयभित्परस्य पुंसः
सुश्लोकं श्रवणपुटैः पिबत्यभीक्ष्णम्‌
पान्थोऽध्वभ्रमणपरिश्रमं जहाति
श्रीशुक उवाच
एकदा द्वारवत्यां तु विप्रपत्न्याः कुमारकः
जातमात्रो भुवं स्पृष्ट्वा ममार किल भारत
विप्रो गृहीत्वा मृतकं राजद्वार्युपधाय सः
इदं प्रोवाच विलपन्नातुरो दीनमानसः
ब्रह्मद्विषः शठधियो लुब्धस्य विषयात्मनः
क्षत्रबन्धोः कर्मदोषात्पञ्चत्वं मे गतोऽर्भकः
हिंसाविहारं नृपतिं दुःशीलमजितेन्द्रियम्‌
प्रजा भजन्त्यः सीदन्ति दरिद्रा नित्यदुःखिताः
एवं द्वितीयं विप्रर्षिस्तृतीयं त्वेवमेव च
विसृज्य स नृपद्वारि तां गाथां समगायत
तामर्जुन उपश्रुत्य कर्हिचित्केशवान्तिके
परेते नवमे बाले ब्राह्मणं समभाषत
किं स्विद्ब्रह्मंस्त्वन्निवासे इह नास्ति धनुर्धरः
राजन्यबन्धुरेते वै ब्राह्मणाः सत्रमासते
धनदारात्मजापृक्ता यत्र शोचन्ति ब्राह्मणाः
ते वै राजन्यवेषेण नटा जीवन्त्यसुम्भराः
अहं प्रजाः वां भगवन्रक्षिष्ये दीनयोरिह
अनिस्तीर्णप्रतिज्ञोऽग्निं प्रवेक्ष्ये हतकल्मषः
श्रीब्राह्मण उवाच
सङ्कर्षणो वासुदेवः प्रद्युम्नो धन्विनां वरः
अनिरुद्धोऽप्रतिरथो न त्रातुं शक्नुवन्ति यत्‌
तत्कथं नु भवान्कर्म दुष्करं जगदीश्वरैः
त्वं चिकीर्षसि बालिश्यात्तन्न श्रद्दध्महे वयम्‌
श्रीअर्जुन उवाच
नाहं सङ्कर्षणो ब्रह्मन्न कृष्णः कार्ष्णिरेव च
अहं वा अर्जुनो नाम गाण्डीवं यस्य वै धनुः
मावमंस्था मम ब्रह्मन्वीर्यं त्र्यम्बकतोषणम्‌
मृत्युं विजित्य प्रधने आनेष्ये ते प्रजाः प्रभो
एवं विश्रम्भितो विप्रः फाल्गुनेन परन्तप
जगाम स्वगृहं प्रीतः पार्थवीर्यं निशामयन्‌
प्रसूतिकाल आसन्ने भार्याया द्विजसत्तमः
पाहि पाहि प्रजां मृत्योरित्याहार्जुनमातुरः
स उपस्पृश्य शुच्यम्भो नमस्कृत्य महेश्वरम्‌
दिव्यान्यस्त्राणि संस्मृत्य सज्यं गाण्डीवमाददे
न्यरुणत्सूतिकागारं शरैर्नानास्त्रयोजितैः
तिर्यगूर्ध्वमधः पार्थश्चकार शरपञ्जरम्‌
ततः कुमारः सञ्जातो विप्रपत्न्या रुदन्मुहुः
सद्योऽदर्शनमापेदे सशरीरो विहायसा
तदाह विप्रो विजयं विनिन्दन्कृष्णसन्निधौ
मौढ्यं पश्यत मे योऽहं श्रद्दधे क्लीबकत्थनम्‌
न प्रद्युम्नो नानिरुद्धो न रामो न च केशवः
यस्य शेकुः परित्रातुं कोऽन्यस्तदवितेश्वरः
धिगर्जुनं मृषावादं धिगात्मश्लाघिनो धनुः
दैवोपसृष्टं यो मौढ्यादानिनीषति दुर्मतिः
एवं शपति विप्रर्षौ विद्यामास्थाय फाल्गुनः
ययौ संयमनीमाशु यत्रास्ते भगवान्यमः
विप्रापत्यमचक्षाणस्तत ऐन्द्रीमगात्पुरीम्‌
आग्नेयीं नैरृतीं सौम्यां वायव्यां वारुणीमथ
रसातलं नाकपृष्ठं धिष्ण्यान्यन्यान्युदायुधः
ततोऽलब्धद्विजसुतो ह्यनिस्तीर्णप्रतिश्रुतः
अग्निं विविक्षुः कृष्णेन प्रत्युक्तः प्रतिषेधता
दर्शये द्विजसूनूंस्ते मावज्ञात्मानमात्मना
ये ते नः कीर्तिं विमलां मनुष्याः स्थापयिष्यन्ति
इति सम्भाष्य भगवानर्जुनेन सहेश्वरः
दिव्यं स्वरथमास्थाय प्रतीचीं दिशमाविशत्‌
सप्त द्वीपान्ससिन्धूंश्च सप्त सप्त गिरीनथ
लोकालोकं तथातीत्य विवेश सुमहत्तमः
तत्राश्वाः शैब्यसुग्रीव मेघपुष्पबलाहकाः
तमसि भ्रष्टगतयो बभूवुर्भरतर्षभ
तान्दृष्ट्वा भगवान्कृष्णो महायोगेश्वरेश्वरः
सहस्रादित्यसङ्काशं स्वचक्रं प्राहिणोत्पुरः
तमः सुघोरं गहनं कृतं महद्‌
विदारयद्भूरितरेण रोचिषा
मनोजवं निर्विविशे सुदर्शनं
गुणच्युतो रामशरो यथा चमूः
द्वारेण चक्रानुपथेन तत्तमः परं परं ज्योतिरनन्तपारम्‌
समश्नुवानं प्रसमीक्ष्य फाल्गुनः प्रताडिताक्षो पिदधेऽक्षिणी उभे
ततः प्रविष्टः सलिलं नभस्वता बलीयसैजद्बृहदूर्मिभूषणम्‌
तत्राद्भुतं वै भवनं द्युमत्तमं भ्राजन्मणिस्तम्भसहस्रशोभितम्‌
तस्मिन्महाभोगमनन्तमद्भुतं
सहस्रमूर्धन्यफणामणिद्युभिः
विभ्राजमानं द्विगुणेक्षणोल्बणं
सिताचलाभं शितिकण्ठजिह्वम्‌
ददर्श तद्भोगसुखासनं विभुं
महानुभावं पुरुषोत्तमोत्तमम्‌
सान्द्राम्बुदाभं सुपिशङ्गवाससं
प्रसन्नवक्त्रं रुचिरायतेक्षणम्‌
महामणिव्रातकिरीटकुण्डल
प्रभापरिक्षिप्तसहस्रकुन्तलम्‌
प्रलम्बचार्वष्टभुजं सकौस्तुभं
श्रीवत्सलक्ष्मं वनमालयावृतम्‌
महामणिव्रातकिरीटकुण्डल
प्रभापरिक्षिप्तसहस्रकुन्तलम्‌
प्रलम्बचार्वष्टभुजं सकौस्तुभं
श्रीवत्सलक्ष्मं वनमालयावृतम्‌
ववन्द आत्मानमनन्तमच्युतो जिष्णुश्च तद्दर्शनजातसाध्वसः
तावाह भूमा परमेष्ठिनां प्रभुर्बेद्धाञ्जली सस्मितमूर्जया गिरा
द्विजात्मजा मे युवयोर्दिदृक्षुणा मयोपनीता भुवि धर्मगुप्तये
कलावतीर्णाववनेर्भरासुरान्हत्वेह भूयस्त्वरयेतमन्ति मे
पूर्णकामावपि युवां नरनारायणावृषी
धर्ममाचरतां स्थित्यै ऋषभौ लोकसङ्ग्रहम्‌
इत्यादिष्टौ भगवता तौ कृष्णौ परमेष्ठिना
ओं इत्यानम्य भूमानमादाय द्विजदारकान्‌
न्यवर्तेतां स्वकं धाम सम्प्रहृष्टौ यथागतम्‌
विप्राय ददतुः पुत्रान्यथारूपं यथावयः
निशाम्य वैष्णवं धाम पार्थः परमविस्मितः
यत्किञ्चित्पौरुषं पुंसां मेने कृष्णानुकम्पितम्‌
इतीदृशान्यनेकानि वीर्याणीह प्रदर्शयन्‌
बुभुजे विषयान्ग्राम्यानीजे चात्युर्जितैर्मखैः
प्रववर्षाखिलान्कामान्प्रजासु ब्राह्मणादिषु
यथाकालं यथैवेन्द्रो भगवान्श्रैष्ठ्यमास्थितः
हत्वा नृपानधर्मिष्ठान्घाटयित्वार्जुनादिभिः
अञ्जसा वर्तयामास धर्मं धर्मसुतादिभिः
श्रीशुक उवाच
सुखं स्वपुर्यां निवसन्द्वारकायां श्रियः पतिः
सर्वसम्पत्समृद्धायां जुष्टायां वृष्णिपुङ्गवैः
स्त्रीभिश्चोत्तमवेषाभिर्नवयौवनकान्तिभिः
कन्दुकादिभिर्हर्म्येषु क्रीडन्तीभिस्तडिद्द्युभिः
नित्यं सङ्कुलमार्गायां मदच्युद्भिर्मतङ्गजैः
स्वलङ्कृतैर्भटैरश्वै रथैश्च कनकोज्ज्वलैः
उद्यानोपवनाढ्यायां पुष्पितद्रुमराजिषु
निर्विशद्भृङ्गविहगैर्नादितायां समन्ततः
रेमे षोडशसाहस्र पत्नीनां एकवल्लभः
तावद्विचित्ररूपोऽसौ तद्गेहेषु महर्द्धिषु
प्रोत्फुल्लोत्पलकह्लार कुमुदाम्भोजरेणुभिः
वासितामलतोयेषु कूजद्द्विजकुलेषु च
विजहार विगाह्याम्भो ह्रदिनीषु महोदयः
कुचकुङ्कुमलिप्ताङ्गः परिरब्धश्च योषिताम्‌
उपगीयमानो गन्धर्वैर्मृदङ्गपणवानकान्‌
वादयद्भिर्मुदा वीणां सूतमागधवन्दिभिः
सिच्यमानोऽच्युतस्ताभिर्हसन्तीभिः स्म रेचकैः
प्रतिषिञ्चन्विचिक्रीडे यक्षीभिर्यक्षराडिव
ताः क्लिन्नवस्त्रविवृतोरुकुचप्रदेशाः
सिञ्चन्त्य उद्धृतबृहत्कवरप्रसूनाः
कान्तं स्म रेचकजिहीर्षययोपगुह्य
जातस्मरोत्स्मयलसद्वदना विरेजुः
कृष्णस्तु तत्स्तनविषज्जितकुङ्कुमस्रक्‌
क्रीडाभिषङ्गधुतकुन्तलवृन्दबन्धः
सिञ्चन्मुहुर्युवतिभिः प्रतिषिच्यमानो
रेमे करेणुभिरिवेभपतिः परीतः
नटानां नर्तकीनां च गीतवाद्योपजीविनाम्‌
क्रीडालङ्कारवासांसि कृष्णोऽदात्तस्य च स्त्रियः
कृष्णस्यैवं विहरतो गत्यालापेक्षितस्मितैः
नर्मक्ष्वेलिपरिष्वङ्गैः स्त्रीणां किल हृता धियः
ऊचुर्मुकुन्दैकधियो गिर उन्मत्तवज्जडम्‌
चिन्तयन्त्योऽरविन्दाक्षं तानि मे गदतः शृणु
महिष्य ऊचुः
कुररि विलपसि त्वं वीतनिद्रा न शेषे
स्वपिति जगति रात्र्यामीश्वरो गुप्तबोधः
वयमिव सखि कच्चिद्गाढनिर्विद्धचेता
नलिननयनहासोदारलीलेक्षितेन
नेत्रे निमीलयसि नक्तमदृष्टबन्धुस्‌
त्वं रोरवीषि करुणं बत चक्रवाकि
दास्यं गत वयमिवाच्युतपादजुष्टां
किं वा स्रजं स्पृहयसे कवरेण वोढुम्‌
भो भोः सदा निष्टनसे उदन्वन्नलब्धनिद्रोऽधिगतप्रजागरः
किम्वा मुकुन्दापहृतात्मलाञ्छनः प्राप्तां दशां त्वं च गतो दुरत्ययाम्‌
त्वं यक्ष्मणा बलवतासि गृहीत इन्दो
क्षीणस्तमो न निजदीधितिभिः क्षिणोषि
कच्चिन्मुकुन्दगदितानि यथा वयं त्वं
विस्मृत्य भोः स्थगितगीरुपलक्ष्यसे नः
किं न्वाचरितमस्माभिर्मलयानिल तेऽप्रियम्‌
गोविन्दापाङ्गनिर्भिन्ने हृदीरयसि नः स्मरम्‌
मेघ श्रीमंस्त्वमसि दयितो यादवेन्द्रस्य नूनं
श्रीवत्साङ्कं वयमिव भवान्ध्यायति प्रेमबद्धः
अत्युत्कण्ठः शवलहृदयोऽस्मद्विधो बाष्पधाराः
स्मृत्वा स्मृत्वा विसृजसि मुहुर्दुःखदस्तत्प्रसङ्गः
प्रियरावपदानि भाषसे मृतसञ्जीविकयानया गिरा
करवाणि किमद्य ते प्रियं वद मे वल्गितकण्ठ कोकिल
न चलसि न वदस्युदारबुद्धे क्षितिधर चिन्तयसे महान्तमर्थम्‌
अपि बत वसुदेवनन्दनाङ्घ्रिं वयमिव कामयसे स्तनैर्विधर्तुम्‌
शुष्यद्ध्रदाः करशिता बत सिन्धुपत्न्यः
सम्प्रत्यपास्तकमलश्रिय इष्टभर्तुः
यद्वद्वयं मधुपतेः प्रणयावलोकम्‌
अप्राप्य मुष्टहृदयाः पुरुकर्शिताः स्म
हंस स्वागतमास्यतां पिब पयो ब्रूह्यङ्ग शौरेः कथां
दूतं त्वां नु विदाम कच्चिदजितः स्वस्त्यास्त उक्तं पुरा
किं वा नश्चलसौहृदः स्मरति तं कस्माद्भजामो वयं
क्षौद्रालापय कामदं श्रियमृते सैवैकनिष्ठा स्त्रियाम्‌
श्रीशुक उवाच
इतीदृशेन भावेन कृष्णे योगेश्वरेश्वरे
क्रियमाणेन माधव्यो लेभिरे परमां गतिम्‌
श्रुतमात्रोऽपि यः स्त्रीणां प्रसह्याकर्षते मनः
उरुगायोरुगीतो वा पश्यन्तीनां च किं पुनः
याः सम्पर्यचरन्प्रेम्णा पादसंवाहनादिभिः
जगद्गुरुं भर्तृबुद्ध्या तासां किम्वर्ण्यते तपः
एवं वेदोदितं धर्ममनुतिष्ठन्सतां गतिः
गृहं धर्मार्थकामानां मुहुश्चादर्शयत्पदम्‌
आस्थितस्य परं धर्मं कृष्णस्य गृहमेधिनाम्‌
आसन्षोडशसाहस्रं महिष्यश्च शताधिकम्‌
तासां स्त्रीरत्नभूतानामष्टौ याः प्रागुदाहृताः
रुक्मिणीप्रमुखा राजंस्तत्पुत्राश्चानुपूर्वशः
एकैकस्यां दश दश कृष्णोऽजीजनदात्मजान्‌
यावत्य आत्मनो भार्या अमोघगतिरीश्वरः
तेषामुद्दामवीर्याणामष्टादश महारथाः
आसन्नुदारयशसस्तेषां नामानि मे शृणु
प्रद्युम्नश्चानिरुद्धश्च दीप्तिमान्भानुरेव च
साम्बो मधुर्बृहद्भानुश्चित्रभानुर्वृकोऽरुणः
पुष्करो वेदबाहुश्च श्रुतदेवः सुनन्दनः
चित्रबाहुर्विरूपश्च कविर्न्यग्रोध एव च
एतेषामपि राजेन्द्र तनुजानां मधुद्विषः
प्रद्युम्न आसीत्प्रथमः पितृवद्रुक्मिणीसुतः
स रुक्मिणो दुहितरमुपयेमे महारथः
तस्यां ततोऽनिरुद्धोऽभूत्नागायतबलान्वितः
स चापि रुक्मिणः पौत्रीं दौहित्रो जगृहे ततः
वज्रस्तस्याभवद्यस्तु मौषलादवशेषितः
प्रतिबाहुरभूत्तस्मात्सुबाहुस्तस्य चात्मजः
सुबाहोः शान्तसेनोऽभूच्छतसेनस्तु तत्सुतः
न ह्येतस्मिन्कुले जाता अधना अबहुप्रजाः
अल्पायुषोऽल्पवीर्याश्च अब्रह्मण्याश्च जज्ञिरे
यदुवंशप्रसूतानां पुंसां विख्यातकर्मणाम्‌
सङ्ख्या न शक्यते कर्तुमपि वर्षायुतैर्नृप
तिस्रः कोट्यः सहस्राणामष्टाशीतिशतानि च
आसन्यदुकुलाचार्याः कुमाराणामिति श्रुतम्‌
सङ्ख्यानं यादवानां कः करिष्यति महात्मनाम्‌
यत्रायुतानामयुत लक्षेणास्ते स आहुकः
देवासुराहवहता दैतेया ये सुदारुणाः
ते चोत्पन्ना मनुष्येषु प्रजा दृप्ता बबाधिरे
तन्निग्रहाय हरिणा प्रोक्ता देवा यदोः कुले
अवतीर्णाः कुलशतं तेषामेकाधिकं नृप
तेषां प्रमाणं भगवान्प्रभुत्वेनाभवद्धरिः
ये चानुवर्तिनस्तस्य ववृधुः सर्वयादवाः
शय्यासनाटनालाप क्रीडास्नानादिकर्मसु
न विदुः सन्तमात्मानं वृष्णयः कृष्णचेतसः
तीर्थं चक्रे नृपोनं यदजनि यदुषु स्वःसरित्पादशौचं
विद्विट्स्निग्धाः स्वरूपं ययुरजितपर श्रीर्यदर्थेऽन्ययत्नः
यन्नामामङ्गलघ्नं श्रुतमथ गदितं यत्कृतो गोत्रधर्मः
कृष्णस्यैतन्न चित्रं क्षितिभरहरणं कालचक्रायुधस्य
जयति जननिवासो देवकीजन्मवादो
यदुवरपरिषत्स्वैर्दोर्भिरस्यन्नधर्मम्‌
स्थिरचरवृजिनघ्नः सुस्मितश्रीमुखेन
व्रजपुरवनितानां वर्धयन्कामदेवम्‌
इत्थं परस्य निजवर्त्मरिरक्षयात्त
लीलातनोस्तदनुरूपविडम्बनानि
कर्माणि कर्मकषणानि यदूत्तमस्य
श्रूयादमुष्य पदयोरनुवृत्तिमिच्छन्‌
मर्त्यस्तयानुसवमेधितया मुकुन्द
श्रीमत्कथाश्रवणकीर्तनचिन्तयैति
तद्धाम दुस्तरकृतान्तजवापवर्गं
ग्रामाद्वनं क्षितिभुजोऽपि ययुर्यदर्थाः
श्रीशुक उवाच
कृत्वा दैत्यवधं कृष्णः सरामो यदुभिर्वृतः
भुवोऽवतारयद्भारं जविष्ठं जनयन्कलिम्‌
ये कोपिताः सुबहु पाण्डुसुताः सपत्नैर्‌
दुर्द्यूतहेलनकचग्रहणादिभिस्तान्‌
कृत्वा निमित्तमितरेतरतः समेतान्‌
हत्वा नृपान्निरहरत्क्षितिभारमीशः
भूभारराजपृतना यदुभिर्निरस्य
गुप्तैः स्वबाहुभिरचिन्तयदप्रमेयः
मन्येऽवनेर्ननु गतोऽप्यगतं हि भारं
यद्यादवं कुलमहो अविषह्यमास्ते
नैवान्यतः परिभवोऽस्य भवेत्कथञ्चिन्‌
मत्संश्रयस्य विभवोन्नहनस्य नित्यम्‌
अन्तः कलिं यदुकुलस्य विधाय वेणु
स्तम्बस्य वह्निमिव शान्तिमुपैमि धाम
एवं व्यवसितो राजन्सत्यसङ्कल्प ईश्वरः
शापव्याजेन विप्राणां सञ्जह्रे स्वकुलं विभुः
स्वमूर्त्या लोकलावण्य निर्मुक्त्या लोचनं नृणाम्‌
गीर्भिस्ताः स्मरतां चित्तं पदैस्तानीक्षतां क्रियाः
आच्छिद्य कीर्तिं सुश्लोकां वितत्य ह्यञ्जसा नु कौ
तमोऽनया तरिष्यन्तीत्यगात्स्वं पदमीश्वरः
श्रीराजोवाच
ब्रह्मण्यानां वदान्यानां नित्यं वृद्धोपसेविनाम्‌
विप्रशापः कथमभूद्वृष्णीनां कृष्णचेतसाम्‌
यन्निमित्तः स वै शापो यादृशो द्विजसत्तम
कथमेकात्मनां भेद एतत्सर्वं वदस्व मे
श्रीबादरायणिरुवाच
बिभ्रद्वपुः सकलसुन्दरसन्निवेशं
कर्माचरन्भुवि सुमङ्गलमाप्तकामः
आस्थाय धाम रममाण उदारकीऋतिः
संहर्तुमैच्छत कुलं स्थितकृत्यशेषः
कर्मानि पुण्यनिवहानि सुमङ्गलानि
गायज्जगत्कलिमलापहराणि कृत्वा
कालात्मना निवसता यदुदेवगेहे
पिण्डारकं समगमन्मुनयो निसृष्टाः
विश्वामित्रोऽसितः कण्वो
दुर्वासा भृगुरङ्गिराः
कश्यपो वामदेवोऽत्रिर्‌
वसिष्ठो नारदादयः
क्रीडन्तस्तानुपव्रज्य कुमारा यदुनन्दनाः
उपसङ्गृह्य पप्रच्छुरविनीता विनीतवत्‌
ते वेषयित्वा स्त्रीवेषैः साम्बं जाम्बवतीसुतम्‌
एषा पृच्छति वो विप्रा अन्तर्वत्न्यसितेक्षणा
प्रष्टुं विलज्जती साक्षात्प्रब्रूतामोघदर्शनाः
प्रसोष्यन्ती पुत्रकामा किं स्वित्सञ्जनयिष्यति
एवं प्रलब्धा मुनयस्तानूचुः कुपिता नृप
जनयिष्यति वो मन्दा मुषलं कुलनाशनम्‌
तच्छ्रुत्वा तेऽतिसन्त्रस्ता विमुच्य सहसोदरम्‌
साम्बस्य ददृशुस्तस्मिन्मुषलं खल्वयस्मयम्‌
किं कृतं मन्दभाग्यैर्नः किं वदिष्यन्ति नो जनाः
इति विह्वलिता गेहानादाय मुषलं ययुः
तच्चोपनीय सदसि परिम्लानमुखश्रियः
राज्ञ आवेदयां चक्रुः सर्वयादवसन्निधौ
श्रुत्वामोघं विप्रशापं दृष्ट्वा च मुषलं नृप
विस्मिता भयसन्त्रस्ता बभूवुर्द्वारकौकसः
तच्चूर्णयित्वा मुषलं यदुराजः स आहुकः
समुद्रसलिले प्रास्यल्लोहं चास्यावशेषितम्‌
कश्चिन्मत्स्योऽग्रसील्लोहं चूर्णानि तरलैस्ततः
उह्यमानानि वेलायां लग्नान्यासन्किलैरकाः
मत्स्यो गृहीतो मत्स्यघ्नैर्जालेनान्यैः सहार्णवे
तस्योदरगतं लोहं स शल्ये लुब्धकोऽकरोत्‌
भगवान्ज्ञातसर्वार्थ ईश्वरोऽपि तदन्यथा
कर्तुं नैच्छद्विप्रशापं कालरूप्यन्वमोदत
श्रीशुक उवाच
गोविन्दभुजगुप्तायां द्वारवत्यां कुरूद्वह
अवात्सीन्नारदोऽभीक्ष्णं कृष्णोपासनलालसः
को नु राजन्निन्द्रियवान्मुकुन्दचरणाम्बुजम्‌
न भजेत्सर्वतोमृत्युरुपास्यममरोत्तमैः
तमेकदा तु देवर्षिं वसुदेवो गृहागतम्‌
अर्चितं सुखमासीनमभिवाद्येदमब्रवीत्‌
श्रीवसुदेव उवाच
भगवन्भवतो यात्रा स्वस्तये सर्वदेहिनाम्‌
कृपणानां यथा पित्रोरुत्तमश्लोकवर्त्मनाम्‌
भूतानां देवचरितं दुःखाय च सुखाय च
सुखायैव हि साधूनां त्वादृशामच्युतात्मनाम्‌
भजन्ति ये यथा देवान्देवा अपि तथैव तान्‌
छायेव कर्मसचिवाः साधवो दीनवत्सलाः
ब्रह्मंस्तथापि पृच्छामो धर्मान्भागवतांस्तव
यान्श्रुत्वा श्रद्धया मर्त्यो मुच्यते सर्वतो भयात्‌
अहं किल पुरानन्तं प्रजार्थो भुवि मुक्तिदम्‌
अपूजयं न मोक्षाय मोहितो देवमायया
यथा विचित्रव्यसनाद्भवद्भिर्विश्वतोभयात्‌
मुच्येम ह्यञ्जसैवाद्धा तथा नः शाधि सुव्रत
श्रीशुक उवाच
राजन्नेवं कृतप्रश्नो वसुदेवेन धीमता
प्रीतस्तमाह देवर्षिर्हरेः संस्मारितो गुणैः
श्रीनारद उवाच
सम्यगेतद्व्यवसितं भवता सात्वतर्षभ
यत्पृच्छसे भागवतान्धर्मांस्त्वं विश्वभावनान्‌
श्रुतोऽनुपठितो ध्यात आदृतो वानुमोदितः
सद्यः पुनाति सद्धर्मो देवविश्वद्रुहोऽपि हि
त्वया परमकल्याणः पुण्यश्रवणकीर्तनः
स्मारितो भगवानद्य देवो नारायणो मम
अत्राप्युदाहरन्तीममितिहासं पुरातनम्‌
आर्षभाणां च संवादं विदेहस्य महात्मनः
प्रियव्रतो नाम सुतो मनोः स्वायम्भुवस्य यः
तस्याग्नीध्रस्ततो नाभिरृषभस्तत्सुतः स्मृतः
तमाहुर्वासुदेवांशं मोक्षधर्मविवक्षया
अवतीर्णं सुतशतं तस्यासीद्ब्रह्मपारगम्‌
तेषां वै भरतो ज्येष्ठो नारायणपरायणः
विख्यातं वर्षमेतद्यन्‌ नाम्ना भारतमद्भुतम्‌
स भुक्तभोगां त्यक्त्वेमां निर्गतस्तपसा हरिम्‌
उपासीनस्तत्पदवीं लेभे वै जनृनभिस्त्रिभिः
तेषां नव नवद्वीप पतयोऽस्य समन्ततः
कर्मतन्त्रप्रणेतार एकाशीतिर्द्विजातयः
नवाभवन्महाभागा मुनयो ह्यर्थशंसिनः
श्रमणा वातरसना आत्मविद्याविशारदाः
कविर्हविरन्तरीक्षः प्रबुद्धः पिप्पलायनः
आविर्होत्रोऽथ द्रुमिलश्चमसः करभाजनः
त एते भगवद्रूपं विश्वं सदसदात्मकम्‌
आत्मनोऽव्यतिरेकेण पश्यन्तो व्यचरन्महीम्‌
अव्याहतेष्टगतयः सुरसिद्धसाध्य
गन्धर्वयक्षनरकिन्नरनागलोकान्‌
मुक्ताश्चरन्ति मुनिचारणभूतनाथ
विद्याधरद्विजगवां भुवनानि कामम्‌
त एकदा निमेः सत्रमुपजग्मुर्यदृच्छया
वितायमानमृषिभिरजनाभे महात्मनः
तान्दृष्ट्वा सूर्यसङ्काशान्महाभागवतान्नृप
यजमानोऽग्नयो विप्राः सर्व एवोपतस्थिरे
विदेहस्तानभिप्रेत्य नारायणपरायणान्‌
प्रीतः सम्पूजयां चक्रे आसनस्थान्यथार्हतः
तान्रोचमानान्स्वरुचा ब्रह्मपुत्रोपमान्नव
पप्रच्छ परमप्रीतः प्रश्रयावनतो नृपः
श्रीविदेह उवाच
मन्ये भगवतः साक्षात्पार्षदान्वो मधुद्विसः
विष्णोर्भूतानि लोकानां पावनाय चरन्ति हि
दुर्लभो मानुषो देहो देहिनां क्षणभङ्गुरः
तत्रापि दुर्लभं मन्ये वैकुण्ठप्रियदर्शनम्‌
अत आत्यन्तिकं क्षेमं पृच्छामो भवतोऽनघाः
संसारेऽस्मिन्क्षणार्धोऽपि सत्सङ्गः शेवधिर्नृणाम्‌
धर्मान्भागवतान्ब्रूत यदि नः श्रुतये क्षमम्‌
यैः प्रसन्नः प्रपन्नाय दास्यत्यात्मानमप्यजः
श्रीनारद उवाच
एवं ते निमिना पृष्टा वसुदेव महत्तमाः
प्रतिपूज्याब्रुवन्प्रीत्या ससदस्यर्त्विजं नृपम्‌
श्रीकविरुवाच
मन्येऽकुतश्चिद्भयमच्युतस्य पादाम्बुजोपासनमत्र नित्यम्‌
उद्विग्नबुद्धेरसदात्मभावाद्विश्वात्मना यत्र निवर्तते भीः
ये वै भगवता प्रोक्ता उपाया ह्यात्मलब्धये
अञ्जः पुंसामविदुषां विद्धि भागवतान्हि तान्‌
यानास्थाय नरो राजन्न प्रमाद्येत कर्हिचित्‌
धावन्निमील्य वा नेत्रे न स्खलेन्न पतेदिह
कायेन वाचा मनसेन्द्रियैर्वा बुद्ध्यात्मना वानुसृतस्वभावात्‌
करोति यद्यत्सकलं परस्मै नारायणायेति समर्पयेत्तत्‌
भयं द्वितीयाभिनिवेशतः स्यादीशादपेतस्य विपर्ययोऽस्मृतिः
तन्माययातो बुध आभजेत्तं भक्त्यैकयेशं गुरुदेवतात्मा
अविद्यमानोऽप्यवभाति हि द्वयो ध्यातुर्धिया स्वप्नमनोरथौ यथा
तत्कर्मसङ्कल्पविकल्पकं मनो बुधो निरुन्ध्यादभयं ततः स्यात्‌
शृण्वन्सुभद्राणि रथाङ्गपाणेर्जन्मानि कर्माणि च यानि लोके
गीतानि नामानि तदर्थकानि गायन्विलज्जो विचरेदसङ्गः
एवंव्रतः स्वप्रियनामकीर्त्या जातानुरागो द्रुतचित्त उच्चैः
हसत्यथो रोदिति रौति गायत्युन्मादवन्नृत्यति लोकबाह्यः
खं वायुमग्निं सलिलं महीं च ज्योतींषि सत्त्वानि दिशो द्रुमादीन्‌
सरित्समुद्रांश्च हरेः शरीरं यत्किं च भूतं प्रणमेदनन्यः
भक्तिः परेशानुभवो विरक्तिरन्यत्र चैष त्रिक एककालः
प्रपद्यमानस्य यथाश्नतः स्युस्तुष्टिः पुष्टिः क्षुदपायोऽनुघासम्‌
इत्यच्युताङ्घ्रिं भजतोऽनुवृत्त्या भक्तिर्विरक्तिर्भगवत्प्रबोधः
भवन्ति वै भागवतस्य राजंस्ततः परां शान्तिमुपैति साक्षात्‌
श्रीराजोवाच
अथ भागवतं ब्रूत यद्धर्मो यादृशो नृणाम्‌
यथाचरति यद्ब्रूते यैर्लिङ्गैर्भगवत्प्रियः
श्रीहविरुवाच
सर्वभूतेषु यः पश्येद्भगवद्भावमात्मनः
भूतानि भगवत्यात्मन्येष भागवतोत्तमः
ईस्वरे तदधीनेषु बालिशेषु द्विषत्सु च
प्रेममैत्रीकृपोपेक्षा यः करोति स मध्यमः
अर्चायामेव हरये पूजां यः श्रद्धयेहते
न तद्भक्तेषु चान्येषु स भक्तः प्राकृतः स्मृतः
गृहीत्वापीन्द्रियैरर्थान्यो न द्वेष्टि न हृष्यति
विष्णोर्मायामिदं पश्यन्स वै भागवतोत्तमः
देहेन्द्रियप्राणमनोधियां यो जन्माप्ययक्षुद्भयतर्षकृच्छ्रैः
संसारधर्मैरविमुह्यमानः स्मृत्या हरेर्भागवतप्रधानः
न कामकर्मबीजानां यस्य चेतसि सम्भवः
वासुदेवैकनिलयः स वै भागवतोत्तमः
न यस्य जन्मकर्मभ्यां न वर्णाश्रमजातिभिः
सज्जतेऽस्मिन्नहंभावो देहे वै स हरेः प्रियः
न यस्य स्वः पर इति वित्तेष्वात्मनि वा भिदा
सर्वभूतसमः शान्तः स वै भागवतोत्तमः
त्रिभुवनविभवहेतवेऽप्यकुण्ठ
स्मृतिरजितात्मसुरादिभिर्विमृग्यात्‌
न चलति भगवत्पदारविन्दाल्‌
लवनिमिषार्धमपि यः स वैष्णवाग्र्यः
भगवत उरुविक्रमाङ्घ्रिशाखा नखमणिचन्द्रिकया निरस्ततापे
हृदि कथमुपसीदतां पुनः स प्रभवति चन्द्र इवोदितेऽर्कतापः
विसृजति हृदयं न यस्य साक्षाद्धरिरवशाभिहितोऽप्यघौघनाशः
प्रणयरसनया धृताङ्घ्रिपद्मः स भवति भागवतप्रधान उक्तः
श्रीराजोवाच
परस्य विष्णोरीशस्य मायिनामपि मोहिनीम्‌
मायां वेदितुमिच्छामो भगवन्तो ब्रुवन्तु नः
नानुतृप्ये जुषन्युष्मद्‌ वचो हरिकथामृतम्‌
संसारतापनिस्तप्तो मर्त्यस्तत्तापभेषजम्‌
श्रीअन्तरीक्ष उवाच
एभिर्भूतानि भूतात्मा महाभूतैर्महाभुज
ससर्जोच्चावचान्याद्यः स्वमात्रात्मप्रसिद्धये
एवं सृष्टानि भूतानि प्रविष्टः पञ्चधातुभिः
एकधा दशधात्मानं विभजन्जुषते गुणान्‌
गुणैर्गुणान्स भुञ्जान आत्मप्रद्योतितैः प्रभुः
मन्यमान इदं सृष्टमात्मानमिह सज्जते
कर्माणि कर्मभिः कुर्वन्सनिमित्तानि देहभृत्‌
तत्तत्कर्मफलं गृह्णन्भ्रमतीह सुखेतरम्‌
इत्थं कर्मगतीर्गच्छन्बह्वभद्रवहाः पुमान्‌
आभूतसम्प्लवात्सर्ग प्रलयावश्नुतेऽवशः
धातूपप्लव आसन्ने व्यक्तं द्रव्यगुणात्मकम्‌
अनादिनिधनः कालो ह्यव्यक्तायापकर्षति
शतवर्षा ह्यनावृष्टिर्भविष्यत्युल्बणा भुवि
तत्कालोपचितोष्णार्को लोकांस्त्रीन्प्रतपिष्यति
पातालतलमारभ्य सङ्कर्षणमुखानलः
दहन्नूर्ध्वशिखो विष्वग्वर्धते वायुनेरितः
संवर्तको मेघगणो वर्षति स्म शतं समाः
धाराभिर्हस्तिहस्ताभिर्लीयते सलिले विराट्‌
ततो विराजमुत्सृज्य्‌ वैराजः पुरुषो नृप
अव्यक्तं विशते सूक्ष्मं निरिन्धन इवानलः
वायुना हृतगन्धा भूः सलिलत्वाय कल्पते
सलिलं तद्धृतरसं ज्योतिष्ट्वायोपकल्पते
हृतरूपं तु तमसा वायौ ज्योतिः प्रलीयते
हृतस्पर्शोऽवकाशेन वायुर्नभसि लीयते
कालात्मना हृतगुणं नभ आत्मनि लीयते
इन्द्रियाणि मनो बुद्धिः सह वैकारिकैर्नृप
प्रविशन्ति ह्यहङ्कारं स्वगुणैरहमात्मनि
एषा माया भगवतः सर्गस्थित्यन्तकारिणी
त्रिवर्णा वर्णितास्माभिः किं भूयः श्रोतुमिच्छसि
श्रीराजोवाच
यथैतामैश्वरीं मायां दुस्तरामकृतात्मभिः
तरन्त्यञ्जः स्थूलधियो महर्ष इदमुच्यताम्‌
श्रीप्रबुद्ध उवाच
कर्माण्यारभमाणानां दुःखहत्यै सुखाय च
पश्येत्पाकविपर्यासं मिथुनीचारिणां नृणाम्‌
नित्यार्तिदेन वित्तेन दुर्लभेनात्ममृत्युना
गृहापत्याप्तपशुभिः का प्रीतिः साधितैश्चलैः
एवं लोकं परम्विद्यान्नश्वरं कर्मनिर्मितम्‌
सतुल्यातिशयध्वंसं यथा मण्डलवर्तिनाम्‌
तस्माद्गुरुं प्रपद्येत जिज्ञासुः श्रेय उत्तमम्‌
शाब्दे परे च निष्णातं ब्रह्मण्युपशमाश्रयम्‌
तत्र भागवतान्धर्मान्शिक्षेद्गुर्वात्मदैवतः
अमाययानुवृत्त्या यैस्तुष्येदात्मात्मदो हरिः
सर्वतो मनसोऽसङ्गमादौ सङ्गं च साधुषु
दयां मैत्रीं प्रश्रयं च भूतेष्वद्धा यथोचितम्‌
शौचं तपस्तितिक्षां च मौनं स्वाध्यायमार्जवम्‌
ब्रह्मचर्यमहिंसां च समत्वं द्वन्द्वसंज्ञयोः
सर्वत्रात्मेश्वरान्वीक्षां कैवल्यमनिकेतताम्‌
विविक्तचीरवसनं सन्तोषं येन केनचित्‌
श्रद्धां भागवते शास्त्रेऽनिन्दामन्यत्र चापि हि
मनोवाक्कर्मदण्डं च सत्यं शमदमावपि
श्रवणं कीर्तनं ध्यानं हरेरद्भुतकर्मणः
जन्मकर्मगुणानां च तदर्थेऽखिलचेष्टितम्‌
इष्टं दत्तं तपो जप्तं वृत्तं यच्चात्मनः प्रियम्‌
दारान्सुतान्गृहान्प्राणान्यत्परस्मै निवेदनम्‌
एवं कृष्णात्मनाथेषु मनुष्येषु च सौहृदम्‌
परिचर्यां चोभयत्र महत्सु नृषु साधुषु
परस्परानुकथनं पावनं भगवद्यशः
मिथो रतिर्मिथस्तुष्टिर्निवृत्तिर्मिथ आत्मनः
स्मरन्तः स्मारयन्तश्च मिथोऽघौघहरं हरिम्‌
भक्त्या सञ्जातया भक्त्या बिभ्रत्युत्पुलकां तनुम्‌
क्वचिद्रुदन्त्यच्युतचिन्तया क्वचिद्‌
धसन्ति नन्दन्ति वदन्त्यलौकिकाः
नृत्यन्ति गायन्त्यनुशीलयन्त्यजं
भवन्ति तूष्णीं परमेत्य निर्वृताः
इति भागवतान्धर्मान्शिक्षन्भक्त्या तदुत्थया
नारायणपरो मायामञ्जस्तरति दुस्तराम्‌
श्रीराजोवाच
नारायणाभिधानस्य ब्रह्मणः परमात्मनः
निष्ठामर्हथ नो वक्तुं यूयं हि ब्रह्मवित्तमाः
श्रीपिप्पलायन उवाच
स्थित्युद्भवप्रलयहेतुरहेतुरस्य
यत्स्वप्नजागरसुषुप्तिषु सद्बहिश्च
देहेन्द्रियासुहृदयानि चरन्ति येन
सञ्जीवितानि तदवेहि परं नरेन्द्र
नैतन्मनो विशति वागुत चक्षुरात्मा
प्राणेन्द्रियाणि च यथानलमर्चिषः स्वाः
शब्दोऽपि बोधकनिषेधतयात्ममूलम्‌
अर्थोक्तमाह यदृते न निषेधसिद्धिः
सत्त्वं रजस्तम इति त्रिवृदेकमादौ
सूत्रं महानहमिति प्रवदन्ति जीवम्‌
ज्ञानक्रियार्थफलरूपतयोरुशक्ति
ब्रह्मैव भाति सदसच्च तयोः परं यत्‌
नात्मा जजान न मरिष्यति नैधतेऽसौ
न क्षीयते सवनविद्व्यभिचारिणां हि
सर्वत्र शश्वदनपाय्युपलब्धिमात्रं
प्राणो यथेन्द्रियबलेन विकल्पितं सत्‌
अण्डेषु पेशिषु तरुष्वविनिश्चितेषु प्राणो हि जीवमुपधावति तत्र तत्र
सन्ने यदिन्द्रियगणेऽहमि च प्रसुप्ते कूटस्थ आशयमृते तदनुस्मृतिर्नः
यर्ह्यब्जनाभचरणैषणयोरुभक्त्या
चेतोमलानि विधमेद्गुणकर्मजानि
तस्मिन्विशुद्ध उपलभ्यत आत्मतत्त्वं
शाक्षाद्यथामलदृशोः सवितृप्रकाशः
श्रीराजोवाच
कर्मयोगं वदत नः पुरुषो येन संस्कृतः
विधूयेहाशु कर्माणि नैष्कर्म्यं विन्दते परम्‌
एवं प्रश्नमृषीन्पूर्वमपृच्छं पितुरन्तिके
नाब्रुवन्ब्रह्मणः पुत्रास्तत्र कारणमुच्यताम्‌
श्रीआविर्होत्र उवाच
कर्माकर्म विकर्मेति वेदवादो न लौकिकः
वेदस्य चेश्वरात्मत्वात्तत्र मुह्यन्ति सूरयः
परोक्षवादो वेदोऽयं बालानामनुशासनम्‌
कर्ममोक्षाय कर्माणि विधत्ते ह्यगदं यथा
नाचरेद्यस्तु वेदोक्तं स्वयमज्ञोऽजितेन्द्रियः
विकर्मणा ह्यधर्मेण मृत्योर्मृत्युमुपैति सः
वेदोक्तमेव कुर्वाणो निःसङ्गोऽर्पितमीश्वरे
नैष्कर्म्यं लभते सिद्धिं रोचनार्था फलश्रुतिः
य आशु हृदयग्रन्थिं निर्जिहीऋषुः परात्मनः
विधिनोपचरेद्देवं तन्त्रोक्तेन च केशवम्‌
लब्ध्वानुग्रह आचार्यात्तेन सन्दर्शितागमः
महापुरुषमभ्यर्चेन्मूर्त्याभिमतयात्मनः
शुचिः सम्मुखमासीनः प्राणसंयमनादिभिः
पिण्डं विशोध्य सन्न्यास कृतरक्षोऽर्चयेद्धरिम्‌
अर्चादौ हृदये चापि यथालब्धोपचारकैः
द्रव्यक्षित्यात्मलिण्गानि निष्पाद्य प्रोक्ष्य चासनम्‌
पाद्यादीनुपकल्प्याथ सन्निधाप्य समाहितः
हृदादिभिः कृतन्यासो मूलमन्त्रेण चार्चयेत्‌
साङ्गोपाङ्गां सपार्षदां तां तां मूर्तिं स्वमन्त्रतः
पाद्यार्घ्याचमनीयाद्यैः स्नानवासोविभूषणैः
गन्धमाल्याक्षतस्रग्भिर्धूपदीपोपहारकैः
साङ्गम्सम्पूज्य विधिवत्स्तवैः स्तुत्वा नमेद्धरिम्‌
आत्मानम्तन्मयम्ध्यायन्मूर्तिं सम्पूजयेद्धरेः
शेषामाधाय शिरसा स्वधाम्न्युद्वास्य सत्कृतम्‌
एवमग्न्यर्कतोयादावतिथौ हृदये च यः
यजतीश्वरमात्मानमचिरान्मुच्यते हि सः
श्रीराजोवाच
यानि यानीह कर्माणि यैर्यैः स्वच्छन्दजन्मभिः
चक्रे करोति कर्ता वा हरिस्तानि ब्रुवन्तु नः
श्रीद्रुमिल उवाच
यो वा अनन्तस्य गुनाननन्ताननुक्रमिष्यन्स तु बालबुद्धिः
रजांसि भूमेर्गणयेत्कथञ्चित्कालेन नैवाखिलशक्तिधाम्नः
भूतैर्यदा पञ्चभिरात्मसृष्टैः
पुरं विराजं विरचय्य तस्मिन्‌
स्वांशेन विष्टः पुरुषाभिधानम्‌
अवाप नारायण आदिदेवः
यत्काय एष भुवनत्रयसन्निवेशो
यस्येन्द्रियैस्तनुभृतामुभयेन्द्रियाणि
ज्ञानं स्वतः श्वसनतो बलमोज ईहा
सत्त्वादिभिः स्थितिलयोद्भव आदिकर्ता
आदावभूच्छतधृती रजसास्य सर्गे
विष्णुः स्थितौ क्रतुपतिर्द्विजधर्मसेतुः
रुद्रोऽप्ययाय तमसा पुरुषः स आद्य
इत्युद्भवस्थितिलयाः सततं प्रजासु
धर्मस्य दक्षदुहितर्यजनिष्ट मूर्त्यां
नारायणो नर ऋषिप्रवरः प्रशान्तः
नैष्कर्म्यलक्षणमुवाच चचार कर्म
योऽद्यापि चास्त ऋषिवर्यनिषेविताङ्घ्रिः
इन्द्रो विशङ्क्य मम धाम जिघृक्षतीति
कामं न्ययुङ्क्त सगणं स बदर्युपाख्यम्‌
गत्वाप्सरोगणवसन्तसुमन्दवातैः
स्त्रीप्रेक्षणेषुभिरविध्यदतन्महिज्ञः
विज्ञाय शक्रकृतमक्रममादिदेवः
प्राह प्रहस्य गतविस्मय एजमानान्‌
मा भैर्विभो मदन मारुत देववध्वो
गृह्णीत नो बलिमशून्यमिमं कुरुध्वम्‌
इत्थं ब्रुवत्यभयदे नरदेव देवाः
सव्रीडनम्रशिरसः सघृणं तमूचुः
नैतद्विभो त्वयि परेऽविकृते विचित्रं
स्वारामधीरनिकरानतपादपद्मे
त्वां सेवतां सुरकृता बहवोऽन्तरायाः
स्वौको विलङ्घ्य परमं व्रजतां पदं ते
नान्यस्य बर्हिषि बलीन्ददतः स्वभागान्‌
धत्ते पदं त्वमविता यदि विघ्नमूर्ध्नि
क्षुत्तृट्त्रिकालगुणमारुतजैह्वशैष्णान्‌
अस्मानपारजलधीनतितीर्य केचित्‌
क्रोधस्य यान्ति विफलस्य वशं पदे गोर्‌
मज्जन्ति दुश्चरतपश्च वृथोत्सृजन्ति
इति प्रगृणतां तेषां स्त्रियोऽत्यद्भुतदर्शनाः
दर्शयामास शुश्रूषां स्वर्चिताः कुर्वतीर्विभुः
ते देवानुचरा दृष्ट्वा स्त्रियः श्रीरिव रूपिणीः
गन्धेन मुमुहुस्तासां रूपौदार्यहतश्रियः
तानाह देवदेवेशः प्रणतान्प्रहसन्निव
आसामेकतमां वृङ्ध्वं सवर्णां स्वर्गभूषणाम्‌
ओमित्यादेशमादाय नत्वा तं सुरवन्दिनः
उर्वशीमप्सरःश्रेष्ठां पुरस्कृत्य दिवं ययुः
इन्द्रायानम्य सदसि शृण्वतां त्रिदिवौकसाम्‌
ऊचुर्नारायणबलं शक्रस्तत्रास विस्मितः
हंसस्वरूप्यवददच्युत आत्मयोगं
दत्तः कुमार ऋषभो भगवान्पिता नः
विष्णुः शिवाय जगतां कलयावतिर्णस्‌
तेनाहृता मधुभिदा श्रुतयो हयास्ये
गुप्तोऽप्यये मनुरिलौषधयश्च मात्स्ये
क्रौडे हतो दितिज उद्धरताम्भसः क्ष्माम्‌
कौर्मे धृतोऽद्रिरमृतोन्मथने स्वपृष्ठे
ग्राहात्प्रपन्नमिभराजममुञ्चदार्तम्‌
संस्तुन्वतो निपतितान्श्रमणानृषींश्च
शक्रं च वृत्रवधतस्तमसि प्रविष्टम्‌
देवस्त्रियोऽसुरगृहे पिहिता अनाथा
जघ्नेऽसुरेन्द्रमभयाय सतां नृसिंहे
देवासुरे युधि च दैत्यपतीन्सुरार्थे
हत्वान्तरेषु भुवनान्यदधात्कलाभिः
भूत्वाथ वामन इमामहरद्बलेः क्ष्मां
याच्ञाच्छलेन समदाददितेः सुतेभ्यः
निःक्षत्रियामकृत गां च त्रिःसप्तकृत्वो
रामस्तु हैहयकुलाप्ययभार्गवाग्निः
सोऽब्धिं बबन्ध दशवक्त्रमहन्सलङ्कं
सीतापतिर्जयति लोकमलघ्नकीऋतिः
भूमेर्भरावतरणाय यदुष्वजन्मा
जातः करिष्यति सुरैरपि दुष्कराणि
वादैर्विमोहयति यज्ञकृतोऽतदर्हान्‌
शूद्रान्कलौ क्षितिभुजो न्यहनिष्यदन्ते
एवंविधानि जन्मानि कर्माणि च जगत्पतेः
भूरीणि भूरियशसो वर्णितानि महाभुज
श्रीराजोवाच
भगवन्तं हरिं प्रायो न भजन्त्यात्मवित्तमाः
तेषामशान्तकामानां क निष्ठाविजितात्मनाम्‌
श्रीचमस उवाच
मुखबाहूरुपादेभ्यः पुरुषस्याश्रमैः सह
चत्वारो जज्ञिरे वर्णा गुणैर्विप्रादयः पृथक्‌
य एषां पुरुषं साक्षादात्मप्रभवमीश्वरम्‌
न भजन्त्यवजानन्ति स्थानाद्भ्रष्टाः पतन्त्यधः
दूरे हरिकथाः केचिद्दूरे चाच्युतकीर्तनाः
स्त्रियः शूद्रादयश्चैव तेऽनुकम्प्या भवादृशाम्‌
विप्रो राजन्यवैश्यौ वा हरेः प्राप्ताः पदान्तिकम्‌
श्रौतेन जन्मनाथापि मुह्यन्त्याम्नायवादिनः
कर्मण्यकोविदाः स्तब्धा मूर्खाः पण्डितमानिनः
वदन्ति चाटुकान्मूढा यया माध्व्या गिरोत्सुकाः
रजसा घोरसङ्कल्पाः कामुका अहिमन्यवः
दाम्भिका मानिनः पापा विहसन्त्यच्युतप्रियान्‌
वदन्ति तेऽन्योन्यमुपासितस्त्रियो गृहेषु मैथुन्यपरेषु चाशिषः
यजन्त्यसृष्टान्नविधानदक्षिणं वृत्त्यै परं घ्नन्ति पशूनतद्विदः
श्रिया विभूत्याभिजनेन विद्यया त्यागेन रूपेण बलेन कर्मणा
जातस्मयेनान्धधियः सहेश्वरान्सतोऽवमन्यन्ति हरिप्रियान्खलाः
सर्वेषु शश्वत्तनुभृत्स्ववस्थितं
यथा खमात्मानमभीष्टमीश्वरम्‌
वेदोपगीतं च न शृण्वतेऽबुधा
मनोरथानां प्रवदन्ति वार्तया
लोके व्यवायामिषमद्यसेवा नित्या हि जन्तोर्न हि तत्र चोदना
व्यवस्थितिस्तेषु विवाहयज्ञ सुराग्रहैरासु निवृत्तिरिष्टा
धनं च धर्मैकफलं यतो वै
ज्ञानं सविज्ञानमनुप्रशान्ति
गृहेषु युञ्जन्ति कलेवरस्य
मृत्युं न पश्यन्ति दुरन्तवीर्यम्‌
यद्घ्राणभक्षो विहितः सुरायास्तथा पशोरालभनं न हिंसा
एवं व्यवायः प्रजया न रत्या इमं विशुद्धं न विदुः स्वधर्मम्‌
ये त्वनेवंविदोऽसन्तः स्तब्धाः सदभिमानिनः
पशून्द्रुह्यन्ति विश्रब्धाः प्रेत्य खादन्ति ते च तान्‌
द्विषन्तः परकायेषु स्वात्मानं हरिमीश्वरम्‌
मृतके सानुबन्धेऽस्मिन्बद्धस्नेहाः पतन्त्यधः
ये कैवल्यमसम्प्राप्ता ये चातीताश्च मूढताम्‌
त्रैवर्गिका ह्यक्षणिका आत्मानं घातयन्ति ते
एत आत्महनोऽशान्ता अज्ञाने ज्ञानमानिनः
सीदन्त्यकृतकृत्या वै कालध्वस्तमनोरथाः
हित्वात्ममायारचिता गृहापत्यसुहृत्स्त्रियः
तमो विशन्त्यनिच्छन्तो वासुदेवपराङ्मुखाः
श्री राजोवाच
कस्मिन्काले स भगवान्किं वर्णः कीदृशो नृभिः
नाम्ना वा केन विधिना पूज्यते तदिहोच्यताम्‌
श्रीकरभाजन उवाच
कृतं त्रेता द्वापरं च कलिरित्येषु केशवः
नानावर्णाभिधाकारो नानैव विधिनेज्यते
कृते शुक्लश्चतुर्बाहुर्जटिलो वल्कलाम्बरः
कृष्णाजिनोपवीताक्षान्बिभ्रद्दण्डकमण्डलू
मनुष्यास्तु तदा शान्ता निर्वैराः सुहृदः समाः
यजन्ति तपसा देवं शमेन च दमेन च
हंसः सुपर्णो वैकुण्ठो धर्मो योगेश्वरोऽमलः
ईश्वरः पुरुषोऽव्यक्तः परमात्मेति गीयते
त्रेतायां रक्तवर्णोऽसौ चतुर्बाहुस्त्रिमेखलः
हिरण्यकेशस्त्रय्यात्मा स्रुक्स्रुवाद्युपलक्षणः
तं तदा मनुजा देवं सर्वदेवमयं हरिम्‌
यजन्ति विद्यया त्रय्या धर्मिष्ठा ब्रह्मवादिनः
विष्णुर्यज्ञः पृश्निगर्भः सर्वदेव उरुक्रमः
वृषाकपिर्जयन्तश्च उरुगाय इतीर्यते
द्वापरे भगवाञ्श्यामः पीतवासा निजायुधः
श्रीवत्सादिभिरङ्कैश्च लक्षणैरुपलक्षितः
तं तदा पुरुषं मर्त्या महाराजोपलक्षणम्‌
यजन्ति वेदतन्त्राभ्यां परं जिज्ञासवो नृप
नमस्ते वासुदेवाय नमः सङ्कर्षणाय च
प्रद्युम्नायानिरुद्धाय तुभ्यं भगवते नमः
नारायणाय ऋषये पुरुषाय महात्मने
विश्वेश्वराय विश्वाय सर्वभूतात्मने नमः
इति द्वापर उर्वीश स्तुवन्ति जगदीश्वरम्‌
नानातन्त्रविधानेन कलावपि तथा शृणु
कृष्णवर्णं त्विषाकृष्णं साङ्गोपाङ्गास्त्रपार्षदम्‌
यज्ञैः सङ्कीर्तनप्रायैर्यजन्ति हि सुमेधसः
ध्येयं सदा परिभवघ्नमभीष्टदोहं
तीर्थास्पदं शिवविरिञ्चिनुतं शरण्यम्‌
भृत्यार्तिहं प्रणतपाल भवाब्धिपोतं
वन्दे महापुरुष ते चरणारविन्दम्‌
त्यक्त्वा सुदुस्त्यजसुरेप्सितराज्यलक्ष्मीं
धर्मिष्ठ आर्यवचसा यदगादरण्यम्‌
मायामृगं दयितयेप्सितमन्वधावद्‌
वन्दे महापुरुष ते चरणारविन्दम्‌
एवं युगानुरूपाभ्यां भगवान्युगवर्तिभिः
मनुजैरिज्यते राजन्श्रेयसामीश्वरो हरिः
कलिं सभाजयन्त्यार्या गुण ज्ञाः सारभागिनः
यत्र सङ्कीर्तनेनैव सर्वस्वार्थोऽभिलभ्यते
न ह्यतः परमो लाभो देहिनां भ्राम्यतामिह
यतो विन्देत परमां शान्तिं नश्यति संसृतिः
कृतादिषु प्रजा राजन्कलाविच्छन्ति सम्भवम्‌
कलौ खलु भविष्यन्ति नारायणपरायणाः
क्वचित्क्वचिन्महाराज द्रविडेषु च भूरिशः
ताम्रपर्णी नदी यत्र कृतमाला पयस्विनी
कावेरी च महापुण्या प्रतीची च महानदी
ये पिबन्ति जलं तासां मनुजा मनुजेश्वर
प्रायो भक्ता भगवति वासुदेवेऽमलाशयाः
देवर्षिभूताप्तनृणां पित्ऱ्‌^ईणां न किङ्करो नायमृणी च राजन्‌
सर्वात्मना यः शरणं शरण्यं गतो मुकुन्दं परिहृत्य कर्तम्‌
स्वपादमूलम्भजतः प्रियस्य त्यक्तान्यभावस्य हरिः परेशः
विकर्म यच्चोत्पतितं कथञ्चिद्धुनोति सर्वं हृदि सन्निविष्टः
श्रीनारद उवाच
धर्मान्भागवतानित्थं श्रुत्वाथ मिथिलेश्वरः
जायन्तेयान्मुनीन्प्रीतः सोपाध्यायो ह्यपूजयत्‌
ततोऽन्तर्दधिरे सिद्धाः सर्वलोकस्य पश्यतः
राजा धर्मानुपातिष्ठन्नवाप परमां गतिम्‌
त्वमप्येतान्महाभाग धर्मान्भागवतान्श्रुतान्‌
आस्थितः श्रद्धया युक्तो निःसङ्गो यास्यसे परम्‌
युवयोः खलु दम्पत्योर्यशसा पूरितं जगत्‌
पुत्रतामगमद्यद्वां भगवानीश्वरो हरिः
दर्शनालिङ्गनालापैः शयनासनभोजनैः
आत्मा वां पावितः कृष्णे पुत्रस्नेहं प्रकुर्वतोः
वैरेण यं नृपतयः शिशुपालपौण्ड्र
शाल्वादयो गतिविलासविलोकनाद्यैः
ध्यायन्त आकृतधियः शयनासनादौ
तत्साम्यमापुरनुरक्तधियां पुनः किम्‌
मापत्यबुद्धिमकृथाः कृष्णे सर्वात्मनीश्वरे
मायामनुष्यभावेन गूढैश्वर्ये परेऽव्यये
भूभारासुरराजन्य हन्तवे गुप्तये सताम्‌
अवतीर्णस्य निर्वृत्यै यशो लोके वितन्यते
श्रीशुक उवाच
एतच्छ्रुत्वा महाभागो वसुदेवोऽतिविस्मितः
देवकी च महाभागा जहतुर्मोहमात्मनः
इतिहासमिमं पुण्यं धारयेद्यः समाहितः
स विधूयेह शमलं ब्रह्मभूयाय कल्पते
श्रीशुक उवाच
अथ ब्रह्मात्मजैः देवैः प्रजेशैरावृतोऽभ्यगात्‌
भवश्च भूतभव्येशो ययौ भूतगणैर्वृतः
इन्द्रो मरुद्भिर्भगवानादित्या वसवोऽश्विनौ
ऋभवोऽङ्गिरसो रुद्रा विश्वे साध्याश्च देवताः
गन्धर्वाप्सरसो नागाः सिद्धचारणगुह्यकाः
ऋषयः पितरश्चैव सविद्याधरकिन्नराः
द्वारकामुपसञ्जग्मुः सर्वे कृष्णदिदृक्षवः
वपुषा येन भगवान्नरलोकमनोरमः
यशो वितेने लोकेषु सर्वलोकमलापहम्‌
तस्यां विभ्राजमानायां समृद्धायां महर्द्धिभिः
व्यचक्षतावितृप्ताक्षाः कृष्णमद्भुतदर्शनम्‌
स्वर्गोद्यानोपगैर्माल्यैश्छादयन्तो युदूत्तमम्‌
गीर्भिश्चित्रपदार्थाभिस्तुष्टुवुर्जगदीश्वरम्‌
श्रीदेवा ऊचुः
नताः स्म ते नाथ पदारविन्दं बुद्धीन्द्रियप्राणमनोवचोभिः
यच्चिन्त्यतेऽन्तर्हृदि भावयुक्तैर्मुमुक्षुभिः कर्ममयोरुपाशात्‌
त्वं मायया त्रिगुणयात्मनि दुर्विभाव्यं
व्यक्तं सृजस्यवसि लुम्पसि तद्गुणस्थः
नैतैर्भवानजित कर्मभिरज्यते वै
यत्स्वे सुखेऽव्यवहितेऽभिरतोऽनवद्यः
शुद्धिर्नृणां न तु तथेड्य दुराशयानां
विद्याश्रुताध्ययनदानतपःक्रियाभिः
सत्त्वात्मनामृषभ ते यशसि प्रवृद्ध
सच्छ्रद्धया श्रवणसम्भृतया यथा स्यात्‌
स्यान्नस्तवाङ्घ्रिरशुभाशयधूमकेतुः
क्षेमाय यो मुनिभिरार्द्रहृदोह्यमानः
यः सात्वतैः समविभूतय आत्मवद्भिर्‌
व्यूहेऽर्चितः सवनशः स्वरतिक्रमाय
यस्चिन्त्यते प्रयतपाणिभिरध्वराग्नौ
त्रय्या निरुक्तविधिनेश हविर्गृहीत्वा
अध्यात्मयोग उत योगिभिरात्ममायां
जिज्ञासुभिः परमभागवतैः परीष्टः
पर्युष्टया तव विभो वनमालयेयं
संस्पार्धिनी भगवती प्रतिपत्नीवच्छ्रीः
यः सुप्रणीतममुयार्हणमाददन्नो
भूयात्सदाङ्घ्रिरशुभाशयधूमकेतुः
केतुस्त्रिविक्रमयुतस्त्रिपतत्पताको
यस्ते भयाभयकरोऽसुरदेवचम्वोः
स्वर्गाय साधुषु खलेष्वितराय भूमन्‌
पदः पुनातु भगवन्भजतामघं नः
नस्योतगाव इव यस्य वशे भवन्ति
ब्रह्मादयस्तनुभृतो मिथुरर्द्यमानाः
कालस्य ते प्रकृतिपूरुषयोः परस्य
शं नस्तनोतु चरणः पुरुषोत्तमस्य
अस्यासि हेतुरुदयस्थितिसंयमानाम्‌
अव्यक्तजीवमहतामपि कालमाहुः
सोऽयं त्रिणाभिरखिलापचये प्रवृत्तः
कालो गभीररय उत्तमपूरुषस्त्वम्‌
त्वत्तः पुमान्समधिगम्य ययास्य वीर्यं
धत्ते महान्तमिव गर्भममोघवीर्यः
सोऽयं तयानुगत आत्मन आण्डकोशं
हैमं ससर्ज बहिरावरणैरुपेतम्‌
तत्तस्थूषश्च जगतश्च भवानधीशो
यन्माययोत्थगुणविक्रिययोपनीतान्‌
अर्थाञ्जुषन्नपि हृषीकपते न लिप्तो
येऽन्ये स्वतः परिहृतादपि बिभ्यति स्म
स्मायावलोकलवदर्शितभावहारि
भ्रूमण्डलप्रहितसौरतमन्त्रशौण्डैः
पत्न्यस्तु षोडशसहस्रमनङ्गबाणैर्‌
यस्येन्द्रियं विमथितुं करणैर्न विभ्व्यः
विभ्व्यस्तवामृतकथोदवहास्त्रिलोक्याः
पादावनेजसरितः शमलानि हन्तुम्‌
आनुश्रवं श्रुतिभिरङ्घ्रिजमङ्गसङ्गैस्‌
तीर्थद्वयं शुचिषदस्त उपस्पृशन्ति
श्रीबादरायणिरुवाच
इत्यभिष्टूय विबुधैः सेशः शतधृतिर्हरिम्‌
अभ्यभाषत गोविन्दं प्रणम्याम्बरमाश्रितः
श्रीब्रह्मोवाच
भूमेर्भारावताराय पुरा विज्ञापितः प्रभो
त्वमस्माभिरशेषात्मन्तत्तथैवोपपादितम्‌
धर्मश्च स्थापितः सत्सु सत्यसन्धेषु वै त्वया
कीर्तिश्च दिक्षु विक्षिप्ता सर्वलोकमलापहा
अवतीर्य यदोर्वंशे बिभ्रद्रूपमनुत्तमम्‌
कर्माण्युद्दामवृत्तानि हिताय जगतोऽकृथाः
यानि ते चरितानीश मनुष्याः साधवः कलौ
शृण्वन्तः कीर्तयन्तश्च तरिष्यन्त्यञ्जसा तमः
यदुवंशेऽवतीर्णस्य भवतः पुरुषोत्तम
शरच्छतं व्यतीयाय पञ्चविंशाधिकं प्रभो
नाधुना तेऽखिलाधार देवकार्यावशेषितम्‌
कुलं च विप्रशापेन नष्टप्रायमभूदिदम्‌
ततः स्वधाम परमं विशस्व यदि मन्यसे
सलोकाल्लोकपालान्नः पाहि वैकुण्ठकिङ्करान्‌
श्रीभगवानुवाच
अवधारितमेतन्मे यदात्थ विबुधेश्वर
कृतं वः कार्यमखिलं भूमेर्भारोऽवतारितः
तदिदं यादवकुलं वीर्यशौर्यश्रियोद्धतम्‌
लोकं जिघृक्षद्रुद्धं मे वेलयेव महार्णवः
यद्यसंहृत्य दृप्तानां यदूनां विपुलं कुलम्‌
गन्तास्म्यनेन लोकोऽयमुद्वेलेन विनङ्क्ष्यति
इदानीं नाश आरब्धः कुलस्य द्विजशापजः
यास्यामि भवनं ब्रह्मन्नेतदन्ते तवानघ
श्रीशुक उवाच
इत्युक्तो लोकनाथेन स्वयम्भूः प्रणिपत्य तम्‌
सह देवगणैर्देवः स्वधाम समपद्यत
अथ तस्यां महोत्पातान्द्वारवत्यां समुत्थितान्‌
विलोक्य भगवानाह यदुवृद्धान्समागतान्‌
श्रीभगवानुवाच
एते वै सुमहोत्पाता व्युत्तिष्ठन्तीह सर्वतः
शापश्च नः कुलस्यासीद्ब्राह्मणेभ्यो दुरत्ययः
न वस्तव्यमिहास्माभिर्जिजीविषुभिरार्यकाः
प्रभासं सुमहत्पुण्यं यास्यामोऽद्यैव मा चिरम्‌
यत्र स्नात्वा दक्षशापाद्गृहीतो यक्ष्मणोदुराट्‌
विमुक्तः किल्बिषात्सद्यो भेजे भूयः कलोदयम्‌
वयं च तस्मिन्नाप्लुत्य तर्पयित्वा पित्ऱ्‌^ईन्सुरान्‌
भोजयित्वोषिजो विप्रान्नानागुणवतान्धसा
तेषु दानानि पात्रेषु श्रद्धयोप्त्वा महान्ति वै
वृजिनानि तरिष्यामो दानैर्नौभिरिवार्णवम्‌
श्रीशुक उवाच
एवं भगवतादिष्टा यादवाः कुरुनन्दन
गन्तुं कृतधियस्तीर्थं स्यन्दनान्समयूयुजन्‌
तन्निरीक्ष्योद्धवो राजन्श्रुत्वा भगवतोदितम्‌
दृष्ट्वारिष्टानि घोराणि नित्यं कृष्णमनुव्रतः
विविक्त उपसङ्गम्य जगतामीश्वरेश्वरम्‌
प्रणम्य शिरिसा पादौ प्राञ्जलिस्तमभाषत
श्रीउद्धव उवाच
देवदेवेश योगेश पुण्यश्रवणकीर्तन
संहृत्यैतत्कुलं नूनं लोकं सन्त्यक्ष्यते भवान्‌
विप्रशापं समर्थोऽपि प्रत्यहन्न यदीश्वरः
नाहं तवाङ्घ्रिकमलं क्षणार्धमपि केशव
त्यक्तुं समुत्सहे नाथ स्वधाम नय मामपि
तव विक्रीडितं कृष्ण नृनां परममङ्गलम्‌
कर्णपीयूषमासाद्य त्यजन्त्यन्यस्पृहां जनाः
शय्यासनाटनस्थान स्नानक्रीडाशनादिषु
कथं त्वां प्रियमात्मानं वयं भक्तास्त्यजेम हि
त्वयोपभुक्तस्रग्गन्ध वासोऽलङ्कारचर्चिताः
उच्छिष्टभोजिनो दासास्तव मायां जयेम हि
वातवसना य ऋषयः श्रमणा ऊर्ध्रमन्थिनः
ब्रह्माख्यं धाम ते यान्ति शान्ताः सन्न्यासीनोऽमलाः
वयं त्विह महायोगिन्भ्रमन्तः कर्मवर्त्मसु
त्वद्वार्तया तरिष्यामस्तावकैर्दुस्तरं तमः
स्मरन्तः कीर्तयन्तस्ते कृतानि गदितानि च
गत्युत्स्मितेक्षणक्ष्वेलि यन्नृलोकविडम्बनम्‌
श्रीशुक उवाच
एवं विज्ञापितो राजन्भगवान्देवकीसुतः
एकान्तिनं प्रियं भृत्यमुद्धवं समभाषत
श्रीभगवानुवाच
यदात्थ मां महाभाग तच्चिकीर्षितमेव मे
ब्रह्मा भवो लोकपालाः स्वर्वासं मेऽभिकाङ्क्षिणः
मया निष्पादितं ह्यत्र देवकार्यमशेषतः
यदर्थमवतीर्णोऽहमंशेन ब्रह्मणार्थितः
कुलं वै शापनिर्दग्धं नङ्क्ष्यत्यन्योन्यविग्रहात्‌
समुद्रः सप्तमे ह्येनां पुरीं च प्लावयिष्यति
यर्ह्येवायं मया त्यक्तो लोकोऽयं नष्टमङ्गलः
भविष्यत्यचिरात्साधो कलिनापि निराकृतः
न वस्तव्यं त्वयैवेह मया त्यक्ते महीतले
जनोऽभद्ररुचिर्भद्र भविष्यति कलौ युगे
त्वं तु सर्वं परित्यज्य स्नेहं स्वजनबन्धुषु
मय्यावेश्य मनः संयक्समदृग्विचरस्व गाम्‌
यदिदं मनसा वाचा चक्षुर्भ्यां श्रवणादिभिः
नश्वरं गृह्यमाणं च विद्धि मायामनोमयम्‌
पुंसोऽयुक्तस्य नानार्थो भ्रमः स गुणदोषभाक्‌
कर्माकर्मविकर्मेति गुणदोषधियो भिदा
तस्माद्युक्तेन्द्रियग्रामो युक्तचित्त इदम्जगत्‌
आत्मनीक्षस्व विततमात्मानं मय्यधीश्वरे
ज्ञानविज्ञानसंयुक्त आत्मभूतः शरीरिणाम्‌
अत्मानुभवतुष्टात्मा नान्तरायैर्विहन्यसे
दोषबुद्ध्योभयातीतो निषेधान्न निवर्तते
गुणबुद्ध्या च विहितं न करोति यथार्भकः
सर्वभूतसुहृच्छान्तो ज्ञानविज्ञाननिश्चयः
पश्यन्मदात्मकं विश्वं न विपद्येत वै पुनः
श्रीशुक उवाच
इत्यादिष्टो भगवता महाभागवतो नृप
उद्धवः प्रणिपत्याह तत्त्वं जिज्ञासुरच्युतम्‌
श्रीउद्धव उवाच
योगेश योगविन्यास योगात्मन्योगसम्भव
निःश्रेयसाय मे प्रोक्तस्त्यागः सन्न्यासलक्षणः
त्यागोऽयं दुष्करो भूमन्कामानां विषयात्मभिः
सुतरां त्वयि सर्वात्मन्नभक्तैरिति मे मतिः
सोऽहं ममाहमिति मूढमतिर्विगाढस्‌
त्वन्मायया विरचितात्मनि सानुबन्धे
तत्त्वञ्जसा निगदितं भवता यथाहं
संसाधयामि भगवन्ननुशाधि भृत्यम्‌
सत्यस्य ते स्वदृश आत्मन आत्मनोऽन्यं
वक्तारमीश विबुधेष्वपि नानुचक्षे
सर्वे विमोहितधियस्तव माययेमे
ब्रह्मादयस्तनुभृतो बहिरर्थभावाः
तस्माद्भवन्तमनवद्यमनन्तपारं
सर्वज्ञमीश्वरमकुण्ठविकुण्ठधिष्ण्यम्‌
निर्विण्णधीरहमु हे वृजिनाभितप्तो
नारायणं नरसखं शरणं प्रपद्ये
श्रीभगवानुवाच
प्रायेण मनुजा लोके लोकतत्त्वविचक्षणाः
समुद्धरन्ति ह्यात्मानमात्मनैवाशुभाशयात्‌
आत्मनो गुरुरात्मैव पुरुषस्य विशेषतः
यत्प्रत्यक्षानुमानाभ्यां श्रेयोऽसावनुविन्दते
पुरुषत्वे च मां धीराः साङ्ख्ययोगविशारदाः
आविस्तरां प्रपश्यन्ति सर्वशक्त्युपबृंहितम्‌
एकद्वित्रिचतुस्पादो बहुपादस्तथापदः
बह्व्यः सन्ति पुरः सृष्टास्तासां मे पौरुषी प्रिया
अत्र मां मृगयन्त्यद्धा युक्ता हेतुभिरीश्वरम्‌
गृह्यमाणैर्गुणैर्लिङ्गैरग्राह्यमनुमानतः
अत्राप्युदाहरन्तीममितिहासं पुरातनम्‌
अवधूतस्य संवादं यदोरमिततेजसः
अवधूतं द्वियं कञ्चिच्चरन्तमकुतोभयम्‌
कविं निरीक्ष्य तरुणं यदुः पप्रच्छ धर्मवित्‌
श्रीयदुरुवाच
कुतो बुद्धिरियं ब्रह्मन्नकर्तुः सुविशारदा
यामासाद्य भवाल्लोकं विद्वांश्चरति बालवत्‌
प्रायो धर्मार्थकामेषु विवित्सायां च मानवाः
हेतुनैव समीहन्त आयुषो यशसः श्रियः
त्वं तु कल्पः कविर्दक्षः सुभगोऽमृतभाषणः
न कर्ता नेहसे किञ्चिज्जडोन्मत्तपिशाचवत्‌
जनेषु दह्यमानेषु कामलोभदवाग्निना
न तप्यसेऽग्निना मुक्तो गङ्गाम्भःस्थ इव द्विपः
त्वं हि नः पृच्छतां ब्रह्मन्नात्मन्यानन्दकारणम्‌
ब्रूहि स्पर्शविहीनस्य भवतः केवलात्मनः
श्रीभगवानुवाच
यदुनैवं महाभागो ब्रह्मण्येन सुमेधसा
पृष्टः सभाजितः प्राह प्रश्रयावनतं द्विजः
श्रीब्राह्मण उवाच
सन्ति मे गुरवो राजन्बहवो बुद्ध्युपश्रिताः
यतो बुद्धिमुपादाय मुक्तोऽटामीह तान्शृणु
पृथिवी वायुराकाशमापोऽग्निश्चन्द्रमा रविः
कपोतोऽजगरः सिन्धुः पतङ्गो मधुकृद्गजः
मधुहा हरिणो मीनः पिङ्गला कुररोऽर्भकः
कुमारी शरकृत्सर्प ऊर्णनाभिः सुपेशकृत्‌
एते मे गुरवो राजन्चतुर्विंशतिराश्रिताः
शिक्षा वृत्तिभिरेतेषामन्वशिक्षमिहात्मनः
यतो यदनुशिक्षामि यथा वा नाहुषात्मज
तत्तथा पुरुषव्याघ्र निबोध कथयामि ते
भूतैराक्रम्यमाणोऽपि धीरो दैववशानुगैः
तद्विद्वान्न चलेन्मार्गादन्वशिक्षं क्षितेर्व्रतम्‌
शश्वत्परार्थसर्वेहः परार्थैकान्तसम्भवः
साधुः शिक्षेत भूभृत्तो नगशिष्यः परात्मताम्‌
प्राणवृत्त्यैव सन्तुष्येन्मुनिर्नैवेन्द्रियप्रियैः
ज्ञानं यथा न नश्येत नावकीर्येत वाङ्मनः
विषयेष्वाविशन्योगी नानाधर्मेषु सर्वतः
गुणदोषव्यपेतात्मा न विषज्जेत वायुवत्‌
पार्थिवेष्विह देहेषु प्रविष्टस्तद्गुणाश्रयः
गुणैर्न युज्यते योगी गन्धैर्वायुरिवात्मदृक्‌
अन्तर्हितश्च स्थिरजङ्गमेषु ब्रह्मात्मभावेन समन्वयेन
व्याप्त्याव्यवच्छेदमसङ्गमात्मनो मुनिर्नभस्त्वं विततस्य भावयेत्‌
तेजोऽबन्नमयैर्भावैर्मेघाद्यैर्वायुनेरितैः
न स्पृश्यते नभस्तद्वत्कालसृष्टैर्गुणैः पुमान्‌
स्वच्छः प्रकृतितः स्निग्धो माधुर्यस्तीर्थभूर्नृणाम्‌
मुनिः पुनात्यपां मित्रमीक्षोपस्पर्शकीर्तनैः
तेजस्वी तपसा दीप्तो दुर्धर्षोदरभाजनः
सर्वभक्ष्योऽपि युक्तात्मा नादत्ते मलमग्निवत्‌
क्वचिच्छन्नः क्वचित्स्पष्ट उपास्यः श्रेय इच्छताम्‌
भुङ्क्ते सर्वत्र दातृणां दहन्प्रागुत्तराशुभम्‌
स्वमायया सृष्टमिदं सदसल्लक्षणं विभुः
प्रविष्ट ईयते तत्तत्‌ स्वरूपोऽग्निरिवैधसि
विसर्गाद्याः श्मशानान्ता भावा देहस्य नात्मनः
कलानामिव चन्द्रस्य कालेनाव्यक्तवर्त्मना
कालेन ह्योघवेगेन भूतानां प्रभवाप्ययौ
नित्यावपि न दृश्येते आत्मनोऽग्नेर्यथार्चिषाम्‌
गुणैर्गुणानुपादत्ते यथाकालं विमुञ्चति
न तेषु युज्यते योगी गोभिर्गा इव गोपतिः
बुध्यते स्वे न भेदेन व्यक्तिस्थ इव तद्गतः
लक्ष्यते स्थूलमतिभिरात्मा चावस्थितोऽर्कवत्‌
नातिस्नेहः प्रसङ्गो वा कर्तव्यः क्वापि केनचित्‌
कुर्वन्विन्देत सन्तापं कपोत इव दीनधीः
कपोतः कश्चनारण्ये कृतनीडो वनस्पतौ
कपोत्या भार्यया सार्धमुवास कतिचित्समाः
कपोतौ स्नेहगुणित हृदयौ गृहधर्मिणौ
दृष्टिं दृष्ट्याङ्गमङ्गेन बुद्धिं बुद्ध्या बबन्धतुः
शय्यासनाटनस्थान वार्ताक्रीडाशनादिकम्‌
मिथुनीभूय विश्रब्धौ चेरतुर्वनराजिषु
यं यं वाञ्छति सा राजन्तर्पयन्त्यनुकम्पिता
तं तं समनयत्कामं कृच्छ्रेणाप्यजितेन्द्रियः
कपोती प्रथमं गर्भं गृह्णन्ती काल आगते
अण्डानि सुषुवे नीडे स्तपत्युः सन्निधौ सती
तेषु काले व्यजायन्त रचितावयवा हरेः
शक्तिभिर्दुर्विभाव्याभिः कोमलाङ्गतनूरुहाः
प्रजाः पुपुषतुः प्रीतौ दम्पती पुत्रवत्सलौ
शृण्वन्तौ कूजितं तासां निर्वृतौ कलभाषितैः
तासां पतत्रैः सुस्पर्शैः कूजितैर्मुग्धचेष्टितैः
प्रत्युद्गमैरदीनानां पितरौ मुदमापतुः
स्नेहानुबद्धहृदयावन्योन्यं विष्णुमायया
विमोहितौ दीनधियौ शिशून्पुपुषतुः प्रजाः
एकदा जग्मतुस्तासामन्नार्थं तौ कुटुम्बिनौ
परितः कानने तस्मिन्नर्थिनौ चेरतुश्चिरम्‌
दृष्ट्वा तान्लुब्धकः कश्चिद्यदृच्छातो वनेचरः
जगृहे जालमातत्य चरतः स्वालयान्तिके
कपोतश्च कपोती च प्रजापोषे सदोत्सुकौ
गतौ पोषणमादाय स्वनीडमुपजग्मतुः
कपोती स्वात्मजान्वीक्ष्य बालकान्जालसम्वृतान्‌
तानभ्यधावत्क्रोशन्ती क्रोशतो भृशदुःखिता
सासकृत्स्नेहगुणिता दीनचित्ताजमायया
स्वयं चाबध्यत शिचा बद्धान्पश्यन्त्यपस्मृतिः
कपोतः स्वात्मजान्बद्धानात्मनोऽप्यधिकान्प्रियान्‌
भार्यां चात्मसमां दीनो विललापातिदुःखितः
अहो मे पश्यतापायमल्पपुण्यस्य दुर्मतेः
अतृप्तस्याकृतार्थस्य गृहस्त्रैवर्गिको हतः
अनुरूपानुकूला च यस्य मे पतिदेवता
शून्ये गृहे मां सन्त्यज्य पुत्रैः स्वर्याति साधुभिः
सोऽहं शून्ये गृहे दीनो मृतदारो मृतप्रजः
जिजीविषे किमर्थं वा विधुरो दुःखजीवितः
तांस्तथैवावृतान्शिग्भिर्मृत्युग्रस्तान्विचेष्टतः
स्वयं च कृपणः शिक्षु पश्यन्नप्यबुधोऽपतत्‌
तं लब्ध्वा लुब्धकः क्रूरः कपोतं गृहमेधिनम्‌
कपोतकान्कपोतीं च सिद्धार्थः प्रययौ गृहम्‌
एवं कुटुम्ब्यशान्तात्मा द्वन्द्वारामः पतत्रिवत्‌
पुष्णन्कुटुम्बं कृपणः सानुबन्धोऽवसीदति
यः प्राप्य मानुषं लोकं मुक्तिद्वारमपावृतम्‌
गृहेषु खगवत्सक्तस्तमारूढच्युतं विदुः
श्रीब्राह्मण उवाच
सुखमैन्द्रियकं राजन्स्वर्गे नरक एव च
देहिनां यद्यथा दुःखं तस्मान्नेच्छेत तद्बुधः
ग्रासं सुमृष्टं विरसं महान्तं स्तोकमेव वा
यदृच्छयैवापतितं ग्रसेदाजगरोऽक्रियः
शयीताहानि भूरीणि निराहारोऽनुपक्रमः
यदि नोपनयेद्ग्रासो महाहिरिव दिष्टभुक्‌
ओजःसहोबलयुतं बिभ्रद्देहमकर्मकम्‌
शयानो वीतनिद्रश्च नेहेतेन्द्रियवानपि
मुनिः प्रसन्नगम्भीरो दुर्विगाह्यो दुरत्ययः
अनन्तपारो ह्यक्षोभ्यः स्तिमितोद इवार्णवः
समृद्धकामो हीनो वा नारायणपरो मुनिः
नोत्सर्पेत न शुष्येत सरिद्भिरिव सागरः
दृष्ट्वा स्त्रियं देवमायां तद्भावैरजितेन्द्रियः
प्रलोभितः पतत्यन्धे तमस्यग्नौ पतङ्गवत्‌
योषिद्धिरण्याभरणाम्बरादि द्रव्येषु मायारचितेषु मूढः
प्रलोभितात्मा ह्युपभोगबुद्ध्या पतङ्गवन्नश्यति नष्टदृष्टिः
स्तोकं स्तोकं ग्रसेद्ग्रासं देहो वर्तेत यावता
गृहानहिंसन्नातिष्ठेद्वृत्तिं माधुकरीं मुनिः
अणुभ्यश्च महद्भ्यश्च शास्त्रेभ्यः कुशलो नरः
सर्वतः सारमादद्यात्पुष्पेभ्य इव षट्पदः
सायन्तनं श्वस्तनं वा न सङ्गृह्णीत भिक्षितम्‌
पाणिपात्रोदरामत्रो मक्षिकेव न सङ्ग्रही
सायन्तनं श्वस्तनं वा न सङ्गृह्णीत भिक्षुकः
मक्षिका इव सङ्गृह्णन्सह तेन विनश्यति
पदापि युवतीं भिक्षुर्न स्पृशेद्दारवीमपि
स्पृशन्करीव बध्येत करिण्या अङ्गसङ्गतः
नाधिगच्छेत्स्त्रियं प्राज्ञः कर्हिचिन्मृत्युमात्मनः
बलाधिकैः स हन्येत गजैरन्यैर्गजो यथा
न देयं नोपभोग्यं च लुब्धैर्यद्दुःखसञ्चितम्‌
भुङ्क्ते तदपि तच्चान्यो मधुहेवार्थविन्मधु
सुदुःखोपार्जितैर्वित्तैराशासानां गृहाशिषः
मधुहेवाग्रतो भुङ्क्ते यतिर्वै गृहमेधिनाम्‌
ग्राम्यगीतं न शृणुयाद्यतिर्वनचरः क्वचित्‌
शिक्षेत हरिणाद्बद्धान्मृगयोर्गीतमोहितात्‌
नृत्यवादित्रगीतानि जुषन्ग्राम्याणि योषिताम्‌
आसां क्रीडनको वश्य ऋष्यशृङ्गो मृगीसुतः
जिह्वयातिप्रमाथिन्या जनो रसविमोहितः
मृत्युमृच्छत्यसद्बुद्धिर्मीनस्तु बडिशैर्यथा
इन्द्रियाणि जयन्त्याशु निराहारा मनीषिणः
वर्जयित्वा तु रसनं तन्निरन्नस्य वर्धते
तावज्जितेन्द्रियो न स्याद्विजितान्येन्द्रियः पुमान्‌
न जयेद्रसनं यावज्जितं सर्वं जिते रसे
पिङ्गला नाम वेश्यासीद्विदेहनगरे पुरा
तस्या मे शिक्षितं किञ्चिन्निबोध नृपनन्दन
सा स्वैरिण्येकदा कान्तं सङ्केत उपनेष्यती
अभूत्काले बहिर्द्वारे बिभ्रती रूपमुत्तमम्‌
मार्ग आगच्छतो वीक्ष्य पुरुषान्पुरुषर्षभ
तान्शुल्कदान्वित्तवतः कान्तान्मेनेऽर्थकामुकी
आगतेष्वपयातेषु सा सङ्केतोपजीविनी
अप्यन्यो वित्तवान्कोऽपि मामुपैष्यति भूरिदः
एवं दुराशया ध्वस्त निद्रा द्वार्यवलम्बती
निर्गच्छन्ती प्रविशती निशीथं समपद्यत
तस्या वित्ताशया शुष्यद्‌ वक्त्राया दीनचेतसः
निर्वेदः परमो जज्ञे चिन्ताहेतुः सुखावहः
तस्या निर्विण्णचित्ताया गीतं शृणु यथा मम
निर्वेद आशापाशानां पुरुषस्य यथा ह्यसिः
न ह्यङ्गाजातनिर्वेदो देहबन्धं जिहासति
यथा विज्ञानरहितो मनुजो ममतां नृप
पिङ्गलोवाच
अहो मे मोहविततिं पश्यताविजितात्मनः
या कान्तादसतः कामं कामये येन बालिशा
सन्तं समीपे रमणं रतिप्रदं वित्तप्रदं नित्यमिमं विहाय
अकामदं दुःखभयाधिशोक मोहप्रदं तुच्छमहं भजेऽज्ञा
अहो मयात्मा परितापितो वृथा साङ्केत्यवृत्त्यातिविगर्ह्यवार्तया
स्त्रैणान्नराद्यार्थतृषोऽनुशोच्यात्क्रीतेन वित्तं रतिमात्मनेच्छती
यदस्थिभिर्निर्मितवंशवंस्य
स्थूणं त्वचा रोमनखैः पिनद्धम्‌
क्षरन्नवद्वारमगारमेतद्‌
विण्मूत्रपूर्णं मदुपैति कान्या
विदेहानां पुरे ह्यस्मिन्नहमेकैव मूढधीः
यान्यमिच्छन्त्यसत्यस्मादात्मदात्काममच्युतात्‌
सुहृत्प्रेष्ठतमो नाथ आत्मा चायं शरीरिणाम्‌
तं विक्रीयात्मनैवाहं रमेऽनेन यथा रमा
कियत्प्रियं ते व्यभजन्कामा ये कामदा नराः
आद्यन्तवन्तो भार्याया देवा वा कालविद्रुताः
नूनं मे भगवान्प्रीतो विष्णुः केनापि कर्मणा
निर्वेदोऽयं दुराशाया यन्मे जातः सुखावहः
मैवं स्युर्मन्दभाग्यायाः क्लेशा निर्वेदहेतवः
येनानुबन्धं निर्हृत्य पुरुषः शममृच्छति
तेनोपकृतमादाय शिरसा ग्राम्यसङ्गताः
त्यक्त्वा दुराशाः शरणं व्रजामि तमधीश्वरम्‌
सन्तुष्टा श्रद्दधत्येतद्यथालाभेन जीवती
विहराम्यमुनैवाहमात्मना रमणेन वै
संसारकूपे पतितं विषयैर्मुषितेक्षणम्‌
ग्रस्तं कालाहिनात्मानं कोऽन्यस्त्रातुमधीश्वरः
आत्मैव ह्यात्मनो गोप्ता निर्विद्येत यदाखिलात्‌
अप्रमत्त इदं पश्येद्ग्रस्तं कालाहिना जगत्‌
श्रीब्राह्मण उवाच
एवं व्यवसितमतिर्दुराशां कान्ततर्षजाम्‌
छित्त्वोपशममास्थाय शय्यामुपविवेश सा
आशा हि परमं दुःखं नैराश्यं परमं सुखम्‌
यथा सञ्छिद्य कान्ताशां सुखं सुष्वाप पिङ्गला
श्रीब्राह्मण उवाच
परिग्रहो हि दुःखाय यद्यत्प्रियतमं नृणाम्‌
अनन्तं सुखमाप्नोति तद्विद्वान्यस्त्वकिञ्चनः
सामिषं कुररं जघ्नुर्बलिनोऽन्ये निरामिषाः
तदामिषं परित्यज्य स सुखं समविन्दत
न मे मानापमानौ स्तो न चिन्ता गेहपुत्रिणाम्‌
आत्मक्रीड आत्मरतिर्विचरामीह बालवत्‌
द्वावेव चिन्तया मुक्तौ परमानन्द आप्लुतौ
यो विमुग्धो जडो बालो यो गुणेभ्यः परं गतः
क्वचित्कुमारी त्वात्मानं वृणानान्गृहमागतान्‌
स्वयं तानर्हयामास क्वापि यातेषु बन्धुषु
तेषामभ्यवहारार्थं शालीन्रहसि पार्थिव
अवघ्नन्त्याः प्रकोष्ठस्थाश्चक्रुः शङ्खाः स्वनं महत्‌
सा तज्जुगुप्सितं मत्वा महती वृईडिता ततः
बभञ्जैकैकशः शङ्खान्द्वौ द्वौ पाण्योरशेषयत्‌
उभयोरप्यभूद्घोषो ह्यवघ्नन्त्याः स्वशङ्खयोः
तत्राप्येकं निरभिददेकस्मान्नाभवद्ध्वनिः
अन्वशिक्षमिमं तस्या उपदेशमरिन्दम
लोकाननुचरन्नेतान्लोकतत्त्वविवित्सया
वासे बहूनां कलहो भवेद्वार्ता द्वयोरपि
एक एव वसेत्तस्मात्कुमार्या इव कङ्कणः
मन एकत्र संयुञ्ज्याज्जितश्वासो जितासनः
वैराग्याभ्यासयोगेन ध्रियमाणमतन्द्रितः
यस्मिन्मनो लब्धपदं यदेतच्छनैः शनैर्मुञ्चति कर्मरेणून्‌
सत्त्वेन वृद्धेन रजस्तमश्च विधूय निर्वाणमुपैत्यनिन्धनम्‌
तदैवमात्मन्यवरुद्धचित्तो न वेद किञ्चिद्बहिरन्तरं वा
यथेषुकारो नृपतिं व्रजन्तमिषौ गतात्मा न ददर्श पार्श्वे
एकचार्यनिकेतः स्यादप्रमत्तो गुहाशयः
अलक्ष्यमाण आचारैर्मुनिरेकोऽल्पभाषणः
गृहारम्भो हि दुःखाय विफलश्चाध्रुवात्मनः
सर्पः परकृतं वेश्म प्रविश्य सुखमेधते
एको नारायणो देवः पूर्वसृष्टं स्वमायया
संहृत्य कालकलया कल्पान्त इदमीश्वरः
एक एवाद्वितीयोऽभूदात्माधारोऽखिलाश्रयः
कालेनात्मानुभावेन साम्यं नीतासु शक्तिषु
सत्त्वादिष्वादिपुरुषः प्रधानपुरुषेश्वरः
परावराणां परम आस्ते कैवल्यसंज्ञितः
केवलानुभवानन्द सन्दोहो निरुपाधिकः
केवलात्मानुभावेन स्वमायां त्रिगुणात्मिकाम्‌
सङ्क्षोभयन्सृजत्यादौ तया सूत्रमरिन्दम
तामाहुस्त्रिगुणव्यक्तिं सृजन्तीं विश्वतोमुखम्‌
यस्मिन्प्रोतमिदं विश्वं येन संसरते पुमान्‌
यथोर्णनाभिर्हृदयादूर्णां सन्तत्य वक्त्रतः
तया विहृत्य भूयस्तां ग्रसत्येवं महेश्वरः
यत्र यत्र मनो देही धारयेत्सकलं धिया
स्नेहाद्द्वेषाद्भयाद्वापि याति तत्तत्स्वरूपताम्‌
कीटः पेशस्कृतं ध्यायन्कुड्यां तेन प्रवेशितः
याति तत्सात्मतां राजन्पूर्वरूपमसन्त्यजन्‌
एवं गुरुभ्य एतेभ्य एषा मे शिक्षिता मतिः
स्वात्मोपशिक्षितां बुद्धिं शृणु मे वदतः प्रभो
देहो गुरुर्मम विरक्तिविवेकहेतुर्‌
बिभ्रत्स्म सत्त्वनिधनं सततार्त्युदर्कम्‌
तत्त्वान्यनेन विमृशामि यथा तथापि
पारक्यमित्यवसितो विचराम्यसङ्गः
जायात्मजार्थपशुभृत्यगृहाप्तवर्गान्‌
पुष्नाति यत्प्रियचिकीर्षया वितन्वन्‌
स्वान्ते सकृच्छ्रमवरुद्धधनः स देहः
सृष्ट्वास्य बीजमवसीदति वृक्षधर्मः
जिह्वैकतोऽमुमपकर्षति कर्हि तर्षा
शिश्नोऽन्यतस्त्वगुदरं श्रवणं कुतश्चित्‌
घ्राणोऽन्यतश्चपलदृक्क्व च कर्मशक्तिर्‌
बह्व्यः सपत्न्य इव गेहपतिं लुनन्ति
सृष्ट्वा पुराणि विविधान्यजयात्मशक्त्या
वृक्षान्सरीसृपपशून्खगदन्दशूकान्‌
तैस्तैरतुष्टहृदयः पुरुषं विधाय
ब्रह्मावलोकधिषणं मुदमाप देवः
लब्ध्वा सुदुर्लभमिदं बहुसम्भवान्ते
मानुष्यमर्थदमनित्यमपीह धीरः
तूर्णं यतेत न पतेदनुमृत्यु यावन्‌
निःश्रेयसाय विषयः खलु सर्वतः स्यात्‌
एवं सञ्जातवैराग्यो विज्ञानालोक आत्मनि
विचरामि महीमेतां मुक्तसङ्गोऽनहङ्कृतः
न ह्येकस्माद्गुरोर्ज्ञानं सुस्थिरं स्यात्सुपुष्कलम्‌
ब्रह्मैतदद्वितीयं वै गीयते बहुधर्षिभिः
श्रीभगवानुवाच
इत्युक्त्वा स यदुं विप्रस्तमामन्त्र्य गभीरधीः
वन्दितः स्वर्चितो राज्ञा ययौ प्रीतो यथागतम्‌
अवधूतवचः श्रुत्वा पूर्वेषां नः स पूर्वजः
सर्वसङ्गविनिर्मुक्तः समचित्तो बभूव ह
श्रीभगवानुवाच
मयोदितेष्ववहितः स्वधर्मेषु मदाश्रयः
वर्णाश्रमकुलाचारमकामात्मा समाचरेत्‌
अन्वीक्षेत विशुद्धात्मा देहिनां विषयात्मनाम्‌
गुणेषु तत्त्वध्यानेन सर्वारम्भविपर्ययम्‌
सुप्तस्य विषयालोको ध्यायतो वा मनोरथः
नानात्मकत्वाद्विफलस्तथा भेदात्मधीर्गुणैः
निवृत्तं कर्म सेवेत प्रवृत्तं मत्परस्त्यजेत्‌
जिज्ञासायां सम्प्रवृत्तो नाद्रियेत्कर्मचोदनाम्‌
यमानभीक्ष्णं सेवेत नियमान्मत्परः क्वचित्‌
मदभिज्ञं गुरुं शान्तमुपासीत मदात्मकम्‌
अमान्यमत्सरो दक्षो निर्ममो दृढसौहृदः
असत्वरोऽर्थजिज्ञासुरनसूयुरमोघवाक्‌
जायापत्यगृहक्षेत्र स्वजनद्रविणादिषु
उदासीनः समं पश्यन्सर्वेष्वर्थमिवात्मनः
विलक्षणः स्थूलसूक्ष्माद्देहादात्मेक्षिता स्वदृक्‌
यथाग्निर्दारुणो दाह्याद्दाहकोऽन्यः प्रकाशकः
निरोधोत्पत्त्यणुबृहन्‌ नानात्वं तत्कृतान्गुणान्‌
अन्तः प्रविष्ट आधत्त एवं देहगुणान्परः
योऽसौ गुणैर्विरचितो देहोऽयं पुरुषस्य हि
संसारस्तन्निबन्धोऽयं पुंसो विद्या च्छिदात्मनः
तस्माज्जिज्ञासयात्मानमात्मस्थं केवलं परम्‌
सङ्गम्य निरसेदेतद्वस्तुबुद्धिं यथाक्रमम्‌
आचार्योऽरणिराद्यः स्यादन्तेवास्युत्तरारणिः
तत्सन्धानं प्रवचनं विद्यासन्धिः सुखावहः
वैशारदी सातिविशुद्धबुद्धिर्धुनोति मायां गुणसम्प्रसूताम्‌
गुनांश्च सन्दह्य यदात्ममेतत्स्वयं च शांयत्यसमिद्यथाग्निः
अथैषाम्कर्मकर्त्ऱ्‌^ईणां भोक्त्ऱ्‌^ईणां सुखदुःखयोः
नानात्वमथ नित्यत्वं लोककालागमात्मनाम्‌
मन्यसे सर्वभावानां संस्था ह्यौत्पत्तिकी यथा
तत्तदाकृतिभेदेन जायते भिद्यते च धीः
एवमप्यङ्ग सर्वेषां देहिनां देहयोगतः
कालावयवतः सन्ति भावा जन्मादयोऽसकृत्‌
तत्रापि कर्मणां कर्तुरस्वातन्त्र्यं च लक्ष्यते
भोक्तुश्च दुःखसुखयोः को न्वर्थो विवशं भजेत्‌
न देहिनां सुखं किञ्चिद्विद्यते विदुषामपि
तथा च दुःखं मूढानां वृथाहङ्करणं परम्‌
यदि प्राप्तिं विघातं च जानन्ति सुखदुःखयोः
तेऽप्यद्धा न विदुर्योगं मृत्युर्न प्रभवेद्यथा
कोऽन्वर्थः सुखयत्येनं कामो वा मृत्युरन्तिके
आघातं नीयमानस्य वध्यस्येव न तुष्टिदः
श्रुतं च दृष्टवद्दुष्टं स्पर्धासूयात्ययव्ययैः
बह्वन्तरायकामत्वात्कृषिवच्चापि निष्फलम्‌
अन्तरायैरविहितो यदि धर्मः स्वनुष्ठितः
तेनापि निर्जितं स्थानं यथा गच्छति तच्छृणु
इष्ट्वेह देवता यज्ञैः स्वर्लोकं याति याज्ञिकः
भुञ्जीत देववत्तत्र भोगान्दिव्यान्निजार्जितान्‌
स्वपुण्योपचिते शुभ्रे विमान उपगीयते
गन्धर्वैर्विहरन्मध्ये देवीनां हृद्यवेषधृक्‌
स्त्रीभिः कामगयानेन किङ्किनीजालमालिना
क्रीडन्न वेदात्मपातं सुराक्रीडेषु निर्वृतः
तावत्स मोदते स्वर्गे यावत्पुण्यं समाप्यते
क्षीणपुन्यः पतत्यर्वागनिच्छन्कालचालितः
यद्यधर्मरतः सङ्गादसतां वाजितेन्द्रियः
कामात्मा कृपणो लुब्धः स्त्रैणो भूतविहिंसकः
पशूनविधिनालभ्य प्रेतभूतगणान्यजन्‌
नरकानवशो जन्तुर्गत्वा यात्युल्बणं तमः
कर्माणि दुःखोदर्काणि कुर्वन्देहेन तैः पुनः
देहमाभजते तत्र किं सुखं मर्त्यधर्मिणः
लोकानां लोकपालानां मद्भयं कल्पजीविनाम्‌
ब्रह्मणोऽपि भयं मत्तो द्विपरार्धपरायुषः
गुणाः सृजन्ति कर्माणि गुणोऽनुसृजते गुणान्‌
जीवस्तु गुणसंयुक्तो भुङ्क्ते कर्मफलान्यसौ
यावत्स्याद्गुणवैषम्यं तावन्नानात्वमात्मनः
नानात्वमात्मनो यावत्पारतन्त्र्यं तदैव हि
यावदस्यास्वतन्त्रत्वं तावदीश्वरतो भयम्‌
य एतत्समुपासीरंस्ते मुह्यन्ति शुचार्पिताः
काल आत्मागमो लोकः स्वभावो धर्म एव च
इति मां बहुधा प्राहुर्गुणव्यतिकरे सति
श्रीउद्धव उवाच
गुणेषु वर्तमानोऽपि देहजेष्वनपावृतः
गुणैर्न बध्यते देही बध्यते वा कथं विभो
कथं वर्तेत विहरेत्कैर्वा ज्ञायेत लक्षणैः
किं भुञ्जीतोत विसृजेच्छयीतासीत याति वा
एतदच्युत मे ब्रूहि प्रश्नं प्रश्नविदां वर
नित्यबद्धो नित्यमुक्त एक एवेति मे भ्रमः
श्रीभगवानुवाच
बद्धो मुक्त इति व्याख्या गुणतो मे न वस्तुतः
गुणस्य मायामूलत्वान्न मे मोक्षो न बन्धनम्‌
शोकमोहौ सुखं दुःखं देहापत्तिश्च मायया
स्वप्नो यथात्मनः ख्यातिः संसृतिर्न तु वास्तवी
विद्याविद्ये मम तनू विद्ध्युद्धव शरीरिणाम्‌
मोक्षबन्धकरी आद्ये मायया मे विनिर्मिते
एकस्यैव ममांशस्य जीवस्यैव महामते
बन्धोऽस्याविद्ययानादिर्विद्यया च तथेतरः
अथ बद्धस्य मुक्तस्य वैलक्षण्यं वदामि ते
विरुद्धधर्मिणोस्तात स्थितयोरेकधर्मिणि
सुपर्णावेतौ सदृशौ सखायौ यदृच्छयैतौ कृतनीडौ च वृक्षे
एकस्तयोः खादति पिप्पलान्नमन्यो निरन्नोऽपि बलेन भूयान्‌
आत्मानमन्यं च स वेद विद्वानपिप्पलादो न तु पिप्पलादः
योऽविद्यया युक्स तु नित्यबद्धो विद्यामयो यः स तु नित्यमुक्तः
देहस्थोऽपि न देहस्थो विद्वान्स्वप्नाद्यथोत्थितः
अदेहस्थोऽपि देहस्थः कुमतिः स्वप्नदृग्यथा
इन्द्रियैरिन्द्रियार्थेषु गुणैरपि गुणेषु च
गृह्यमाणेष्वहं कुर्यान्न विद्वान्यस्त्वविक्रियः
दैवाधीने शरीरेऽस्मिन्गुणभाव्येन कर्मणा
वर्तमानोऽबुधस्तत्र कर्तास्मीति निबध्यते
एवं विरक्तः शयन आसनाटनमज्जने
दर्शनस्पर्शनघ्राण भोजनश्रवणादिषु
न तथा बध्यते विद्वान्तत्र तत्रादयन्गुणान्‌
प्रकृतिस्थोऽप्यसंसक्तो यथा खं सवितानिलः
वैशारद्येक्षयासङ्ग शितया छिन्नसंशयः
प्रतिबुद्ध इव स्वप्नान्नानात्वाद्विनिवर्तते
यस्य स्युर्वीतसङ्कल्पाः प्राणेन्द्रियर्ननोधियाम्‌
वृत्तयः स विनिर्मुक्तो देहस्थोऽपि हि तद्गुणैः
यस्यात्मा हिंस्यते हिंस्रैर्येन किञ्चिद्यदृच्छया
अर्च्यते वा क्वचित्तत्र न व्यतिक्रियते बुधः
न स्तुवीत न निन्देत कुर्वतः साध्वसाधु वा
वदतो गुणदोषाभ्यां वर्जितः समदृङ्मुनिः
न कुर्यान्न वदेत्किञ्चिन्न ध्यायेत्साध्वसाधु वा
आत्मारामोऽनया वृत्त्या विचरेज्जडवन्मुनिः
शब्दब्रह्मणि निष्णातो न निष्णायात्परे यदि
श्रमस्तस्य श्रमफलो ह्यधेनुमिव रक्षतः
गां दुग्धदोहामसतीं च भार्यां देहं पराधीनमसत्प्रजां च
वित्तं त्वतीर्थीकृतमङ्ग वाचं हीनां मया रक्षति दुःखदुःखी
यस्यां न मे पावनमङ्ग कर्म स्थित्युद्भवप्राणनिरोधमस्य
लीलावतारेप्सितजन्म वा स्याद्वन्ध्यां गिरं तां बिभृयान्न धीरः
एवं जिज्ञासयापोह्य नानात्वभ्रममात्मनि
उपारमेत विरजं मनो मय्यर्प्य सर्वगे
यद्यनीशो धारयितुं मनो ब्रह्मणि निश्चलम्‌
मयि सर्वाणि कर्माणि निरपेक्षः समाचर
श्रद्धालुर्मत्कथाः शृण्वन्सुभद्रा लोकपावनीः
गायन्ननुस्मरन्कर्म जन्म चाभिनयन्मुहुः
मदर्थे धर्मकामार्थानाचरन्मदपाश्रयः
लभते निश्चलां भक्तिं मय्युद्धव सनातने
सत्सङ्गलब्धया भक्त्या मयि मां स उपासिता
स वै मे दर्शितं सद्भिरञ्जसा विन्दते पदम्‌
श्रीउद्धव उवाच
साधुस्तवोत्तमश्लोक मतः कीदृग्विधः प्रभो
भक्तिस्त्वय्युपयुज्येत कीदृशी सद्भिरादृता
एतन्मे पुरुषाध्यक्ष लोकाध्यक्ष जगत्प्रभो
प्रणतायानुरक्ताय प्रपन्नाय च कथ्यताम्‌
त्वं ब्रह्म परमं व्योम पुरुषः प्रकृतेः परः
अवतीर्नोऽसि भगवन्स्वेच्छोपात्तपृथग्वपुः
श्रीभगवानुवाच
कृपालुरकृतद्रोहस्तितिक्षुः सर्वदेहिनाम्‌
सत्यसारोऽनवद्यात्मा समः सर्वोपकारकः
कामैरहतधीर्दान्तो मृदुः शुचिरकिञ्चनः
अनीहो मितभुक्षान्तः स्थिरो मच्छरणो मुनिः
अप्रमत्तो गभीरात्मा धृतिमाञ्जितषड्गुणः
अमानी मानदः कल्यो मैत्रः कारुणिकः कविः
आज्ञायैवं गुणान्दोषान्मयादिष्टानपि स्वकान्‌
धर्मान्सन्त्यज्य यः सर्वान्मां भजेत स तु सत्तमः
ज्ञात्वाज्ञात्वाथ ये वै मां यावान्यश्चास्मि यादृशः
भजन्त्यनन्यभावेन ते मे भक्ततमा मताः
मल्लिङ्गमद्भक्तजन दर्शनस्पर्शनार्चनम्‌
परिचर्या स्तुतिः प्रह्व गुणकर्मानुकीर्तनम्‌
मत्कथाश्रवणे श्रद्धा मदनुध्यानमुद्धव
सर्वलाभोपहरणं दास्येनात्मनिवेदनम्‌
मज्जन्मकर्मकथनं मम पर्वानुमोदनम्‌
गीतताण्डववादित्र गोष्ठीभिर्मद्गृहोत्सवः
यात्रा बलिविधानं च सर्ववार्षिकपर्वसु
वैदिकी तान्त्रिकी दीक्षा मदीयव्रतधारणम्‌
ममार्चास्थापने श्रद्धा स्वतः संहत्य चोद्यमः
उद्यानोपवनाक्रीड पुरमन्दिरकर्मणि
सम्मार्जनोपलेपाभ्यां सेकमण्डलवर्तनैः
गृहशुश्रूषणं मह्यं दासवद्यदमायया
अमानित्वमदम्भित्वं कृतस्यापरिकीर्तनम्‌
अपि दीपावलोकं मे नोपयुञ्ज्यान्निवेदितम्‌
यद्यदिष्टतमं लोके यच्चातिप्रियमात्मनः
तत्तन्निवेदयेन्मह्यं तदानन्त्याय कल्पते
सूर्योऽग्निर्ब्राह्मणा गावो वैष्णवः खं मरुज्जलम्‌
भूरात्मा सर्वभूतानि भद्र पूजापदानि मे
सूर्ये तु विद्यया त्रय्या हविषाग्नौ यजेत माम्‌
आतिथ्येन तु विप्राग्र्ये गोष्वङ्ग यवसादिना
वैष्णवे बन्धुसत्कृत्या हृदि खे ध्याननिष्ठया
वायौ मुख्यधिया तोये द्रव्यैस्तोयपुरःसरैः
स्थण्डिले मन्त्रहृदयैर्भोगैरात्मानमात्मनि
क्षेत्रज्ञं सर्वभूतेषु समत्वेन यजेत माम्‌
धिष्ण्येष्वित्येषु मद्रूपं शङ्खचक्रगदाम्बुजैः
युक्तं चतुर्भुजं शान्तं ध्यायन्नर्चेत्समाहितः
इष्टापूर्तेन मामेवं यो यजेत समाहितः
लभते मयि सद्भक्तिं मत्स्मृतिः साधुसेवया
प्रायेण भक्तियोगेन सत्सङ्गेन विनोद्धव
नोपायो विद्यते सम्यक्प्रायणं हि सतामहम्‌
अथैतत्परमं गुह्यं शृण्वतो यदुनन्दन
सुगोप्यमपि वक्ष्यामि त्वं मे भृत्यः सुहृत्सखा
श्रीभगवानुवाच
न रोधयति मां योगो न साङ्ख्यं धर्म एव च
न स्वाध्यायस्तपस्त्यागो नेष्टापूर्तं न दक्षिणा
व्रतानि यज्ञश्छन्दांसि तीर्थानि नियमा यमाः
यथावरुन्धे सत्सङ्गः सर्वसङ्गापहो हि माम्‌
सत्सङ्गेन हि दैतेया यातुधाना मृगाः खगाः
गन्धर्वाप्सरसो नागाः सिद्धाश्चारणगुह्यकाः
विद्याधरा मनुष्येषु वैश्याः शूद्राः स्त्रियोऽन्त्यजाः
रजस्तमःप्रकृतयस्तस्मिंस्तस्मिन्युगे युगे
बहवो मत्पदं प्राप्तास्त्वाष्ट्रकायाधवादयः
वृषपर्वा बलिर्बाणो मयश्चाथ विभीषणः
सुग्रीवो हनुमानृक्षो गजो गृध्रो वणिक्पथः
व्याधः कुब्जा व्रजे गोप्यो यज्ञपत्न्यस्तथापरे
ते नाधीतश्रुतिगणा नोपासितमहत्तमाः
अव्रतातप्ततपसः मत्सङ्गान्मामुपागताः
केवलेन हि भावेन गोप्यो गावो नगा मृगाः
येऽन्ये मूढधियो नागाः सिद्धा मामीयुरञ्जसा
यं न योगेन साङ्ख्येन दानव्रततपोऽध्वरैः
व्याख्यास्वाध्यायसन्न्यासैः प्राप्नुयाद्यत्नवानपि
रामेण सार्धं मथुरां प्रणीते श्वाफल्किना मय्यनुरक्तचित्ताः
विगाढभावेन न मे वियोग तीव्राधयोऽन्यं ददृशुः सुखाय
तास्ताः क्षपाः प्रेष्ठतमेन नीता मयैव वृन्दावनगोचरेण
क्षणार्धवत्ताः पुनरङ्ग तासां हीना मया कल्पसमा बभूवुः
ता नाविदन्मय्यनुषङ्गबद्ध धियः स्वमात्मानमदस्तथेदम्‌
यथा समाधौ मुनयोऽब्धितोये नद्यः प्रविष्टा इव नामरूपे
मत्कामा रमणं जारमस्वरूपविदोऽबलाः
ब्रह्म मां परमं प्रापुः सङ्गाच्छतसहस्रशः
तस्मात्त्वमुद्धवोत्सृज्य चोदनां प्रतिचोदनाम्‌
प्रवृत्तिं च निवृत्तिं च श्रोतव्यं श्रुतमेव च
मामेकमेव शरणमात्मानं सर्वदेहिनाम्‌
याहि सर्वात्मभावेन मया स्या ह्यकुतोभयः
श्रीउद्धव उवाच
संशयः शृण्वतो वाचं तव योगेश्वरेश्वर
न निवर्तत आत्मस्थो येन भ्राम्यति मे मनः
श्रीभगवानुवाच
स एष जीवो विवरप्रसूतिः प्राणेन घोषेण गुहां प्रविष्टः
मनोमयं सूक्ष्ममुपेत्य रूपं मात्रा स्वरो वर्ण इति स्थविष्ठः
यथानलः खेऽनिलबन्धुरुष्मा बलेन दारुण्यधिमथ्यमानः
अणुः प्रजातो हविषा समेधते तथैव मे व्यक्तिरियं हि वाणी
एवं गदिः कर्म गतिर्विसर्गो घ्राणो रसो दृक्स्पर्शः श्रुतिश्च
सङ्कल्पविज्ञानमथाभिमानः सूत्रं रजःसत्त्वतमोविकारः
अयं हि जीवस्त्रिवृदब्जयोनिरव्यक्त एको वयसा स आद्यः
विश्लिष्टशक्तिर्बहुधेव भाति बीजानि योनिं प्रतिपद्य यद्वत्‌
यस्मिन्निदं प्रोतमशेषमोतं पटो यथा तन्तुवितानसंस्थः
य एष संसारतरुः पुराणः कर्मात्मकः पुष्पफले प्रसूते
द्वे अस्य बीजे शतमूलस्त्रिनालः पञ्चस्कन्धः पञ्चरसप्रसूतिः
दशैकशाखो द्विसुपर्णनीडस्त्रिवल्कलो द्विफलोऽर्कं प्रविष्टः
अदन्ति चैकं फलमस्य गृध्रा ग्रामेचरा एकमरण्यवासाः
हंसा य एकं बहुरूपमिज्यैर्मायामयं वेद स वेद वेदम्‌
एवं गुरूपासनयैकभक्त्या विद्याकुठारेण शितेन धीरः
विवृश्च्य जीवाशयमप्रमत्तः सम्पद्य चात्मानमथ त्यजास्त्रम्‌
श्रीभगवानुवाच
सत्त्वं रजस्तम इति गुणा बुद्धेर्न चात्मनः
सत्त्वेनान्यतमौ हन्यात्सत्त्वं सत्त्वेन चैव हि
सत्त्वाद्धर्मो भवेद्वृद्धात्पुंसो मद्भक्तिलक्षणः
सात्त्विकोपासया सत्त्वं ततो धर्मः प्रवर्तते
धर्मो रजस्तमो हन्यात्सत्त्ववृद्धिरनुत्तमः
आशु नश्यति तन्मूलो ह्यधर्म उभये हते
आगमोऽपः प्रजा देशः कालः कर्म च जन्म च
ध्यानं मन्त्रोऽथ संस्कारो दशैते गुणहेतवः
तत्तत्सात्त्विकमेवैषां यद्यद्वृद्धाः प्रचक्षते
निन्दन्ति तामसं तत्तद्राजसं तदुपेक्षितम्‌
सात्त्विकान्येव सेवेत पुमान्सत्त्वविवृद्धये
ततो धर्मस्ततो ज्ञानं यावत्स्मृतिरपोहनम्‌
वेणुसङ्घर्षजो वह्निर्दग्ध्वा शाम्यति तद्वनम्‌
एवं गुणव्यत्ययजो देहः शाम्यति तत्क्रियः
श्रीउद्धव उवाच
विदन्ति मर्त्याः प्रायेण विषयान्पदमापदाम्‌
तथापि भुञ्जते कृष्ण तत्कथं श्वखराजवत्‌
श्रीभगवानुवाच
अहमित्यन्यथाबुद्धिः प्रमत्तस्य यथा हृदि
उत्सर्पति रजो घोरं ततो वैकारिकं मनः
रजोयुक्तस्य मनसः सङ्कल्पः सविकल्पकः
ततः कामो गुणध्यानाद्दुःसहः स्याद्धि दुर्मतेः
करोति कामवशगः कर्माण्यविजितेन्द्रियः
दुःखोदर्काणि सम्पश्यन्रजोवेगविमोहितः
रजस्तमोभ्यां यदपि विद्वान्विक्षिप्तधीः पुनः
अतन्द्रितो मनो युञ्जन्दोषदृष्टिर्न सज्जते
अप्रमत्तोऽनुयुञ्जीत मनो मय्यर्पयञ्छनैः
अनिर्विण्णो यथाकालं जितश्वासो जितासनः
एतावान्योग आदिष्टो मच्छिष्यैः सनकादिभिः
सर्वतो मन आकृष्य मय्यद्धावेश्यते यथा
श्रीउद्धव उवाच
यदा त्वं सनकादिभ्यो येन रूपेण केशव
योगमादिष्टवानेतद्रूपमिच्छामि वेदितुम्‌
श्रीभगवानुवाच
पुत्रा हिरण्यगर्भस्य मानसाः सनकादयः
पप्रच्छुः पितरं सूक्ष्मां योगस्यैकान्तिकीम्गतिम्‌
सनकादय ऊचुः
गुणेष्वाविशते चेतो गुणाश्चेतसि च प्रभो
कथमन्योन्यसन्त्यागो मुमुक्षोरतितितीर्षोः
श्रीभगवानुवाच
एवं पृष्टो महादेवः स्वयम्भूर्भूतभावनः
ध्यायमानः प्रश्नबीजं नाभ्यपद्यत कर्मधीः
स मामचिन्तयद्देवः प्रश्नपारतितीर्षया
तस्याहं हंसरूपेण सकाशमगमं तदा
दृष्ट्वा माम्त उपव्रज्य कृत्व पादाभिवन्दनम्‌
ब्रह्माणमग्रतः कृत्वा पप्रच्छुः को भवानिति
इत्यहं मुनिभिः पृष्टस्तत्त्वजिज्ञासुभिस्तदा
यदवोचमहं तेभ्यस्तदुद्धव निबोध मे
वस्तुनो यद्यनानात्व आत्मनः प्रश्न ईदृशः
कथं घटेत वो विप्रा वक्तुर्वा मे क आश्रयः
पञ्चात्मकेषु भूतेषु समानेषु च वस्तुतः
को भवानिति वः प्रश्नो वाचारम्भो ह्यनर्थकः
मनसा वचसा दृष्ट्या गृह्यतेऽन्यैरपीन्द्रियैः
अहमेव न मत्तोऽन्यदिति बुध्यध्वमञ्जसा
गुणेष्वाविशते चेतो गुणाश्चेतसि च प्रजाः
जीवस्य देह उभयं गुणाश्चेतो मदात्मनः
गुणेषु चाविशच्चित्तमभीक्ष्णं गुणसेवया
गुणाश्च चित्तप्रभवा मद्रूप उभयं त्यजेत्‌
जाग्रत्स्वप्नः सुषुप्तं च गुणतो बुद्धिवृत्तयः
तासां विलक्षणो जीवः साक्षित्वेन विनिश्चितः
यर्हि संसृतिबन्धोऽयमात्मनो गुणवृत्तिदः
मयि तुर्ये स्थितो जह्यात्त्यागस्तद्गुणचेतसाम्‌
अहङ्कारकृतं बन्धमात्मनोऽर्थविपर्ययम्‌
विद्वान्निर्विद्य संसार चिन्तां तुर्ये स्थितस्त्यजेत्‌
यावन्नानार्थधीः पुंसो न निवर्तेत युक्तिभिः
जागर्त्यपि स्वपन्नज्ञः स्वप्ने जागरणं यथा
असत्त्वादात्मनोऽन्येषां भावानां तत्कृता भिदा
गतयो हेतवश्चास्य मृषा स्वप्नदृशो यथा
यो जागरे बहिरनुक्षणधर्मिणोऽर्थान्‌
भुङ्क्ते समस्तकरणैर्हृदि तत्सदृक्षान्‌
स्वप्ने सुषुप्त उपसंहरते स एकः
स्मृत्यन्वयात्त्रिगुणवृत्तिदृगिन्द्रियेशः
एवं विमृश्य गुणतो मनसस्त्र्यवस्था
मन्मायया मयि कृता इति निश्चितार्थाः
सञ्छिद्य हार्दमनुमानसदुक्तितीक्ष्ण
ज्ञानासिना भजत माखिलसंशयाधिम्‌
ईक्षेत विभ्रममिदं मनसो विलासं
दृष्टं विनष्टमतिलोलमलातचक्रम्‌
विज्ञानमेकमुरुधेव विभाति माया
स्वप्नस्त्रिधा गुणविसर्गकृतो विकल्पः
दृष्टिम्ततः प्रतिनिवर्त्य निवृत्ततृष्णस्‌
तूष्णीं भवेन्निजसुखानुभवो निरीहः
सन्दृश्यते क्व च यदीदमवस्तुबुद्ध्या
त्यक्तं भ्रमाय न भवेत्स्मृतिरानिपातात्‌
देहं च नश्वरमवस्थितमुत्थितं वा
सिद्धो न पश्यति यतोऽध्यगमत्स्वरूपम्‌
दैवादपेतमथ दैववशादुपेतं
वासो यथा परिकृतं मदिरामदान्धः
देहोऽपि दैववशगः खलु कर्म यावत्‌
स्वारम्भकं प्रतिसमीक्षत एव सासुः
तं सप्रपञ्चमधिरूढसमाधियोगः
स्वाप्नं पुनर्न भजते प्रतिबुद्धवस्तुः
मयैतदुक्तं वो विप्रा गुह्यं यत्साङ्ख्ययोगयोः
जानीत मागतं यज्ञं युष्मद्धर्मविवक्षया
अहं योगस्य साङ्ख्यस्य सत्यस्यर्तस्य तेजसः
परायणं द्विजश्रेष्ठाः श्रियः कीर्तेर्दमस्य च
मां भजन्ति गुणाः सर्वे निर्गुणं निरपेक्षकम्‌
सुहृदं प्रियमात्मानं साम्यासङ्गादयोऽगुणाः
इति मे छिन्नसन्देहा मुनयः सनकादयः
सभाजयित्वा परया भक्त्यागृणत संस्तवैः
तैरहं पूजितः संयक्संस्तुतः परमर्षिभिः
प्रत्येयाय स्वकं धाम पश्यतः परमेष्ठिनः
श्रीउद्धव उवाच
वदन्ति कृष्ण श्रेयांसि बहूनि ब्रह्मवादिनः
तेषां विकल्पप्राधान्यमुताहो एकमुख्यता
भवतोदाहृतः स्वामिन्भक्तियोगोऽनपेक्षितः
निरस्य सर्वतः सङ्गं येन त्वय्याविशेन्मनः
श्रीभगवानुवाच
कालेन नष्टा प्रलये वाणीयं वेदसंज्ञिता
मयादौ ब्रह्मणे प्रोक्ता धर्मो यस्यां मदात्मकः
तेन प्रोक्ता स्वपुत्राय मनवे पूर्वजाय सा
ततो भृग्वादयोऽगृह्णन्सप्त ब्रह्ममहर्षयः
तेभ्यः पितृभ्यस्तत्पुत्रा देवदानवगुह्यकाः
मनुष्याः सिद्धगन्धर्वाः सविद्याधरचारणाः
किन्देवाः किन्नरा नागा रक्षःकिम्पुरुषादयः
बह्व्यस्तेषां प्रकृतयो रजःसत्त्वतमोभुवः
याभिर्भूतानि भिद्यन्ते भूतानां पतयस्तथा
यथाप्रकृति सर्वेषां चित्रा वाचः स्रवन्ति हि
एवं प्रकृतिवैचित्र्याद्भिद्यन्ते मतयो नृणाम्‌
पारम्पर्येण केषाञ्चित्पाषण्डमतयोऽपरे
मन्मायामोहितधियः पुरुषाः पुरुषर्षभ
श्रेयो वदन्त्यनेकान्तं यथाकर्म यथारुचि
धर्ममेके यशश्चान्ये कामं सत्यं दमं शमम्‌
अन्ये वदन्ति स्वार्थं वा ऐश्वर्यं त्यागभोजनम्‌
केचिद्यज्ञं तपो दानं व्रतानि नियमान्यमान्‌
आद्यन्तवन्त एवैषां लोकाः कर्मविनिर्मिताः
दुःखोदर्कास्तमोनिष्ठाः क्षुद्रा मन्दाः शुचार्पिताः
मय्यर्पितात्मनः सभ्य निरपेक्षस्य सर्वतः
मयात्मना सुखं यत्तत्कुतः स्याद्विषयात्मनाम्‌
अकिञ्चनस्य दान्तस्य शान्तस्य समचेतसः
मया सन्तुष्टमनसः सर्वाः सुखमया दिशः
न पारमेष्ठ्यं न महेन्द्रधिष्ण्यं
न सार्वभौमं न रसाधिपत्यम्‌
न योगसिद्धीरपुनर्भवं वा
मय्यर्पितात्मेच्छति मद्विनान्यत्‌
न तथा मे प्रियतम आत्मयोनिर्न शङ्करः
न च सङ्कर्षणो न श्रीर्नैवात्मा च यथा भवान्‌
निरपेक्षं मुनिं शान्तं निर्वैरं समदर्शनम्‌
अनुव्रजाम्यहं नित्यं पूयेयेत्यङ्घ्रिरेणुभिः
निष्किञ्चना मय्यनुरक्तचेतसः शान्ता महान्तोऽखिलजीववत्सलाः
कामैरनालब्धधियो जुषन्ति ते यन्नैरपेक्ष्यं न विदुः सुखं मम
बाध्यमानोऽपि मद्भक्तो विषयैरजितेन्द्रियः
प्रायः प्रगल्भया भक्त्या विषयैर्नाभिभूयते
यथाग्निः सुसमृद्धार्चिः करोत्येधांसि भस्मसात्‌
तथा मद्विषया भक्तिरुद्धवैनांसि कृत्स्नशः
न साधयति मां योगो न साङ्ख्यं धर्म उद्धव
न स्वाध्यायस्तपस्त्यागो यथा भक्तिर्ममोर्जिता
भक्त्याहमेकया ग्राह्यः श्रद्धयात्मा प्रियः सताम्‌
भक्तिः पुनाति मन्निष्ठा श्वपाकानपि सम्भवात्‌
धर्मः सत्यदयोपेतो विद्या वा तपसान्विता
मद्भक्त्यापेतमात्मानं न सम्यक्प्रपुनाति हि
कथं विना रोमहर्षं द्रवता चेतसा विना
विनानन्दाश्रुकलया शुध्येद्भक्त्या विनाशयः
वाग्गद्गदा द्रवते यस्य चित्तं रुदत्यभीक्ष्णं हसति क्वचिच्च
विलज्ज उद्गायति नृत्यते च मद्भक्तियुक्तो भुवनं पुनाति
यथाग्निना हेम मलं जहाति ध्मातं पुनः स्वं भजते च रूपम्‌
आत्मा च कर्मानुशयं विधूय मद्भक्तियोगेन भजत्यथो माम्‌
यथा यथात्मा परिमृज्यतेऽसौ मत्पुण्यगाथाश्रवणाभिधानैः
तथा तथा पश्यति वस्तु सूक्ष्मं चक्षुर्यथैवाञ्जनसम्प्रयुक्तम्‌
विषयान्ध्यायतश्चित्तं विषयेषु विषज्जते
मामनुस्मरतश्चित्तं मय्येव प्रविलीयते
तस्मादसदभिध्यानं यथा स्वप्नमनोरथम्‌
हित्वा मयि समाधत्स्व मनो मद्भावभावितम्‌
स्त्रीणां स्त्रीसङ्गिनां सङ्गं त्यक्त्वा दूरत आत्मवान्‌
क्षेमे विविक्त आसीनश्चिन्तयेन्मामतन्द्रितः
न तथास्य भवेत्क्लेशो बन्धश्चान्यप्रसङ्गतः
योषित्सङ्गाद्यथा पुंसो यथा तत्सङ्गिसङ्गतः
श्रीउद्धव उवाच
यथा त्वामरविन्दाक्ष यादृशं वा यदात्मकम्‌
ध्यायेन्मुमुक्षुरेतन्मे ध्यानं त्वं वक्तुमर्हसि
श्रीभगवानुवाच
सम आसन आसीनः समकायो यथासुखम्‌
हस्तावुत्सङ्ग आधाय स्वनासाग्रकृतेक्षणः
प्राणस्य शोधयेन्मार्गं पूरकुम्भकरेचकैः
विपर्ययेणापि शनैरभ्यसेन्निर्जितेन्द्रियः
हृद्यविच्छिनमोंकारं घण्टानादं बिसोर्णवत्‌
प्राणेनोदीर्य तत्राथ पुनः संवेशयेत्स्वरम्‌
एवं प्रणवसंयुक्तं प्राणमेव समभ्यसेत्‌
दशकृत्वस्त्रिषवणं मासादर्वाग्जितानिलः
हृत्पुण्डरीकमन्तःस्थमूर्ध्वनालमधोमुखम्‌
ध्यात्वोर्ध्वमुखमुन्निद्रमष्टपत्रं सकर्णिकम्‌
कर्णिकायां न्यसेत्सूर्य सोमाग्नीनुत्तरोत्तरम्‌
वह्निमध्ये स्मरेद्रूपं ममैतद्ध्यानमङ्गलम्‌
समं प्रशान्तं सुमुखं दीर्घचारुचतुर्भुजम्‌
सुचारुसुन्दरग्रीवं सुकपोलं शुचिस्मितम्‌
समानकर्णविन्यस्त स्फुरन्मकरकुण्डलम्‌
हेमाम्बरं घनश्यामं श्रीवत्सश्रीनिकेतनम्‌
शङ्खचक्रगदापद्म वनमालाविभूषितम्‌
नूपुरैर्विलसत्पादं कौस्तुभप्रभया युतम्‌
द्युमत्किरीटकटक कटिसूत्राङ्गदायुतम्‌
सर्वाङ्गसुन्दरं हृद्यं प्रसादसुमुखेक्षनम्‌
सुकुमारमभिध्यायेत्सर्वाङ्गेषु मनो दधत्‌
इन्द्रियाणीन्द्रियार्थेभ्यो मनसाकृष्य तन्मनः
बुद्ध्या सारथिना धीरः प्रणयेन्मयि सर्वतः
तत्सर्वव्यापकं चित्तमाकृष्यैकत्र धारयेत्‌
नान्यानि चिन्तयेद्भूयः सुस्मितं भावयेन्मुखम्‌
तत्र लब्धपदं चित्तमाकृष्य व्योम्नि धारयेत्‌
तच्च त्यक्त्वा मदारोहो न किञ्चिदपि चिन्तयेत्‌
एवं समाहितमतिर्मामेवात्मानमात्मनि
विचष्टे मयि सर्वात्मन्ज्योतिर्ज्योतिषि संयुतम्‌
ध्यानेनेत्थं सुतीव्रेण युञ्जतो योगिनो मनः
संयास्यत्याशु निर्वाणं द्रव्य ज्ञानक्रियाभ्रमः
श्रीभगवानुवाच
जितेन्द्रियस्य युक्तस्य जितश्वासस्य योगिनः
मयि धारयतश्चेत उपतिष्ठन्ति सिद्धयः
श्रीउद्धव उवाच
कया धारणया का स्वित्कथं वा सिद्धिरच्युत
कति वा सिद्धयो ब्रूहि योगिनां सिद्धिदो भवान्‌
श्रीभगवानुवाच
सिद्धयोऽष्टादश प्रोक्ता धारणा योगपारगैः
तासामष्टौ मत्प्रधाना दशैव गुणहेतवः
अणिमा महिमा मूर्तेर्लघिमा प्राप्तिरिन्द्रियैः
प्राकाम्यं श्रुतदृष्टेषु शक्तिप्रेरणमीशिता
गुणेष्वसङ्गो वशिता यत्कामस्तदवस्यति
एता मे सिद्धयः सौम्य अष्टावौत्पत्तिका मताः
अनूर्मिमत्त्वं देहेऽस्मिन्दूरश्रवणदर्शनम्‌
मनोजवः कामरूपं परकायप्रवेशनम्‌
स्वच्छन्दमृत्युर्देवानां सहक्रीडानुदर्शनम्‌
यथासङ्कल्पसंसिद्धिराज्ञाप्रतिहता गतिः
त्रिकालज्ञत्वमद्वन्द्वं परचित्ताद्यभिज्ञता
अग्न्यर्काम्बुविषादीनां प्रतिष्टम्भोऽपराजयः
एताश्चोद्देशतः प्रोक्ता योगधारणसिद्धयः
यया धारणया या स्याद्यथा वा स्यान्निबोध मे
भूतसूक्ष्मात्मनि मयि तन्मात्रं धारयेन्मनः
अणिमानमवाप्नोति तन्मात्रोपासको मम
महत्तत्त्वात्मनि मयि यथासंस्थं मनो दधत्‌
महिमानमवाप्नोति भूतानां च पृथक्पृथक्‌
परमाणुमये चित्तं भूतानां मयि रञ्जयन्‌
कालसूक्ष्मार्थतां योगी लघिमानमवाप्नुयात्‌
धारयन्मय्यहंतत्त्वे मनो वैकारिकेऽखिलम्‌
सर्वेन्द्रियाणामात्मत्वं प्राप्तिं प्राप्नोति मन्मनाः
महत्यात्मनि यः सूत्रे धारयेन्मयि मानसम्‌
प्राकाम्यं पारमेष्ठ्यं मे विन्दतेऽव्यक्तजन्मनः
विष्णौ त्र्यधीश्वरे चित्तं धारयेत्कालविग्रहे
स ईशित्वमवाप्नोति क्षेत्रज्ञक्षेत्रचोदनाम्‌
नारायणे तुरीयाख्ये भगवच्छब्दशब्दिते
मनो मय्यादधद्योगी मद्धर्मा वशितामियात्‌
निर्गुणे ब्रह्मणि मयि धारयन्विशदं मनः
परमानन्दमाप्नोति यत्र कामोऽवसीयते
श्वेतद्वीपपतौ चित्तं शुद्धे धर्ममये मयि
धारयञ्छ्वेततां याति षडूर्मिरहितो नरः
मय्याकाशात्मनि प्राणे मनसा घोषमुद्वहन्‌
तत्रोपलब्धा भूतानां हंसो वाचः शृणोत्यसौ
चक्षुस्त्वष्टरि संयोज्य त्वष्टारमपि चक्षुषि
मां तत्र मनसा ध्यायन्विश्वं पश्यति दूरतः
मनो मयि सुसंयोज्य देहं तदनुवायुना
मद्धारणानुभावेन तत्रात्मा यत्र वै मनः
यदा मन उपादाय यद्यद्रूपं बुभूषति
तत्तद्भवेन्मनोरूपं मद्योगबलमाश्रयः
परकायं विशन्सिद्ध आत्मानं तत्र भावयेत्‌
पिण्डं हित्वा विशेत्प्राणो वायुभूतः षडङ्घ्रिवत्‌
पार्ष्ण्यापीड्य गुदं प्राणं हृदुरःकण्ठमूर्धसु
आरोप्य ब्रह्मरन्ध्रेण ब्रह्म नीत्वोत्सृजेत्तनुम्‌
विहरिष्यन्सुराक्रीडे मत्स्थं सत्त्वं विभावयेत्‌
विमानेनोपतिष्ठन्ति सत्त्ववृत्तीः सुरस्त्रियः
यथा सङ्कल्पयेद्बुद्ध्या यदा वा मत्परः पुमान्‌
मयि सत्ये मनो युञ्जंस्तथा तत्समुपाश्नुते
यो वै मद्भावमापन्न ईशितुर्वशितुः पुमान्‌
कुतश्चिन्न विहन्येत तस्य चाज्ञा यथा मम
मद्भक्त्या शुद्धसत्त्वस्य योगिनो धारणाविदः
तस्य त्रैकालिकी बुद्धिर्जन्ममृत्यूपबृंहिता
अग्न्यादिभिर्न हन्येत मुनेर्योगमयं वपुः
मद्योगशान्तचित्तस्य यादसामुदकं यथा
मद्विभूतीरभिध्यायन्श्रीवत्सास्त्रविभूषिताः
ध्वजातपत्रव्यजनैः स भवेदपराजितः
उपासकस्य मामेवं योगधारणया मुनेः
सिद्धयः पूर्वकथिता उपतिष्ठन्त्यशेषतः
जितेन्द्रियस्य दान्तस्य जितश्वासात्मनो मुनेः
मद्धारणां धारयतः का सा सिद्धिः सुदुर्लभा
अन्तरायान्वदन्त्येता युञ्जतो योगमुत्तमम्‌
मया सम्पद्यमानस्य कालक्षपणहेतवः
जन्मौषधितपोमन्त्रैर्यावतीरिह सिद्धयः
योगेनाप्नोति ताः सर्वा नान्यैर्योगगतिं व्रजेत्‌
सर्वासामपि सिद्धीनां हेतुः पतिरहं प्रभुः
अहं योगस्य साङ्ख्यस्य धर्मस्य ब्रह्मवादिनाम्‌
अहमात्मान्तरो बाह्योऽनावृतः सर्वदेहिनाम्‌
यथा भूतानि भूतेषु बहिरन्तः स्वयं तथा
श्रीउद्धव उवाच
त्वं ब्रह्म परमं साक्षादनाद्यन्तमपावृतम्‌
सर्वेषामपि भावानां त्राणस्थित्यप्ययोद्भवः
उच्चावचेषु भूतेषु दुर्ज्ञेयमकृतात्मभिः
उपासते त्वां भगवन्याथातथ्येन ब्राह्मणाः
येषु येषु च भूतेषु भक्त्या त्वां परमर्षयः
उपासीनाः प्रपद्यन्ते संसिद्धिं तद्वदस्व मे
गूढश्चरसि भूतात्मा भूतानां भूतभावन
न त्वां पश्यन्ति भूतानि पश्यन्तं मोहितानि ते
याः काश्च भूमौ दिवि वै रसायां विभूतयो दिक्षु महाविभूते
ता मह्यमाख्याह्यनुभावितास्ते नमामि ते तीर्थपदाङ्घ्रिपद्मम्‌
श्रीभगवानुवाच
एवमेतदहं पृष्टः प्रश्नं प्रश्नविदां वर
युयुत्सुना विनशने सपत्नैरर्जुनेन वै
ज्ञात्वा ज्ञातिवधं गर्ह्यमधर्मं राज्यहेतुकम्‌
ततो निवृत्तो हन्ताहं हतोऽयमिति लौकिकः
स तदा पुरुषव्याघ्रो युक्त्या मे प्रतिबोधितः
अभ्यभाषत मामेवं यथा त्वं रणमूर्धनि
अहमात्मोद्धवामीषां भूतानां सुहृदीश्वरः
अहं सर्वाणि भूतानि तेषां स्थित्युद्भवाप्ययः
अहं गतिर्गतिमतां कालः कलयतामहम्‌
गुनाणां चाप्यहं साम्यं गुणिन्यौत्पत्तिको गुणः
गुणिनामप्यहं सूत्रं महतां च महानहम्‌
सूक्ष्माणामप्यहं जीवो दुर्जयानामहं मनः
हिरण्यगर्भो वेदानां मन्त्राणां प्रणवस्त्रिवृत्‌
अक्षराणामकारोऽस्मि पदानि च्छन्दुसामहम्‌
इन्द्रोऽहं सर्वदेवानां वसूनामस्मि हव्यवाट्‌
आदित्यानामहं विष्णू रुद्राणां नीललोहितः
ब्रह्मर्षीणां भृगुरहं राजर्षीणामहं मनुः
देवर्षीणां नारदोऽहं हविर्धान्यस्मि धेनुषु
सिद्धेश्वराणां कपिलः सुपर्णोऽहं पतत्रिणाम्‌
प्रजापतीनां दक्षोऽहं पित्ऱ्‌^ईणामहमर्यमा
मां विद्ध्युद्धव दैत्यानां प्रह्लादमसुरेश्वरम्‌
सोमं नक्षत्रौषधीनां धनेशं यक्षरक्षसाम्‌
ऐरावतं गजेन्द्राणां यादसां वरुणं प्रभुम्‌
तपतां द्युमतां सूर्यं मनुष्याणां च भूपतिम्‌
उच्चैःश्रवास्तुरङ्गाणां धातूनामस्मि काञ्चनम्‌
यमः संयमतां चाहम्सर्पाणामस्मि वासुकिः
नागेन्द्राणामनन्तोऽहं मृगेन्द्रः शृङ्गिदंष्ट्रिणाम्‌
आश्रमाणामहं तुर्यो वर्णानां प्रथमोऽनघ
तीर्थानां स्रोतसां गङ्गा समुद्रः सरसामहम्‌
आयुधानां धनुरहं त्रिपुरघ्नो धनुष्मताम्‌
धिष्ण्यानामस्म्यहं मेरुर्गहनानां हिमालयः
वनस्पतीनामश्वत्थ ओषधीनामहं यवः
पुरोधसां वसिष्ठोऽहं ब्रह्मिष्ठानां बृहस्पतिः
स्कन्दोऽहं सर्वसेनान्यामग्रण्यां भगवानजः
यज्ञानां ब्रह्मयज्ञोऽहं व्रतानामविहिंसनम्‌
वाय्वग्न्यर्काम्बुवागात्मा शुचीनामप्यहं शुचिः
योगानामात्मसंरोधो मन्त्रोऽस्मि विजिगीषताम्‌
आन्वीक्षिकी कौशलानां विकल्पः ख्यातिवादिनाम्‌
स्त्रीणां तु शतरूपाहं पुंसां स्वायम्भुवो मनुः
नारायणो मुनीनां च कुमारो ब्रह्मचारिणाम्‌
धर्माणामस्मि सन्न्यासः क्षेमाणामबहिर्मतिः
गुह्यानां सुनृतं मौनं मिथुनानामजस्त्वहम्‌
संवत्सरोऽस्म्यनिमिषामृतूनां मधुमाधवौ
मासानां मार्गशीर्षोऽहं नक्षत्राणां तथाभिजित्‌
अहं युगानां च कृतं धीराणां देवलोऽसितः
द्वैपायनोऽस्मि व्यासानां कवीनां काव्य आत्मवान्‌
वासुदेवो भगवतां त्वं तु भागवतेष्वहम्‌
किम्पुरुषानां हनुमान्विद्याध्राणां सुदर्शनः
रत्नानां पद्मरागोऽस्मि पद्मकोशः सुपेशसाम्‌
कुशोऽस्मि दर्भजातीनां गव्यमाज्यं हविःष्वहम्‌
व्यवसायिनामहं लक्ष्मीः कितवानां छलग्रहः
तितिक्षास्मि तितिक्षूणां सत्त्वं सत्त्ववतामहम्‌
ओजः सहो बलवतां कर्माहं विद्धि सात्वताम्‌
सात्वतां नवमूर्तीनामादिमूर्तिरहं परा
विश्वावसुः पूर्वचित्तिर्गन्धर्वाप्सरसामहम्‌
भूधराणामहं स्थैर्यं गन्धमात्रमहं भुवः
अपां रसश्च परमस्तेजिष्ठानां विभावसुः
प्रभा सूर्येन्दुताराणां शब्दोऽहं नभसः परः
ब्रह्मण्यानां बलिरहं वीराणामहमर्जुनः
भूतानां स्थितिरुत्पत्तिरहं वै प्रतिसङ्क्रमः
गत्युक्त्युत्सर्गोपादानमानन्दस्पर्शलक्षनम्‌
आस्वादश्रुत्यवघ्राणमहं सर्वेन्द्रियेन्द्रियम्‌
पृथिवी वायुराकाश आपो ज्योतिरहं महान्‌
विकारः पुरुषोऽव्यक्तं रजः सत्त्वं तमः परम्‌
अहमेतत्प्रसङ्ख्यानं ज्ञानं तत्त्वविनिश्चयः
मयेश्वरेण जीवेन गुणेन गुणिना विना
सर्वात्मनापि सर्वेण न भावो विद्यते क्वचित्‌
सङ्ख्यानं परमाणूनां कालेन क्रियते मया
न तथा मे विभूतीनां सृजतोऽण्डानि कोटिशः
तेजः श्रीः कीर्तिरैश्वर्यं ह्रीस्त्यागः सौभगं भगः
वीर्यं तितिक्षा विज्ञानं यत्र यत्र स मेऽंशकः
एतास्ते कीर्तिताः सर्वाः सङ्क्षेपेण विभूतयः
मनोविकारा एवैते यथा वाचाभिधीयते
वाचं यच्छ मनो यच्छ प्राणान्यच्छेद्रियाणि च
आत्मानमात्मना यच्छ न भूयः कल्पसेऽध्वने
यो वै वाङ्मनसी संयगसंयच्छन्धिया यतिः
तस्य व्रतं तपो दानं स्रवत्यामघटाम्बुवत्‌
तस्माद्वचो मनः प्राणान्नियच्छेन्मत्परायणः
मद्भक्तियुक्तया बुद्ध्या ततः परिसमाप्यते
श्रीउद्धव उवाच
यस्त्वयाभिहितः पूर्वं धर्मस्त्वद्भक्तिलक्षणः
वर्णाशमाचारवतां सर्वेषां द्विपदामपि
यथानुष्ठीयमानेन त्वयि भक्तिर्नृणां भवेत्‌
स्वधर्मेणारविन्दाक्ष तन्ममाख्यातुमर्हसि
पुरा किल महाबाहो धर्मं परमकं प्रभो
यत्तेन हंसरूपेण ब्रह्मणेऽभ्यात्थ माधव
स इदानीं सुमहता कालेनामित्रकर्शन
न प्रायो भविता मर्त्य लोके प्रागनुशासितः
वक्ता कर्ताविता नान्यो धर्मस्याच्युत ते भुवि
सभायामपि वैरिञ्च्यां यत्र मूर्तिधराः कलाः
कर्त्रावित्रा प्रवक्त्रा च भवता मधुसूदन
त्यक्ते महीतले देव विनष्टं कः प्रवक्ष्यति
तत्त्वं नः सर्वधर्मज्ञ धर्मस्त्वद्भक्तिलक्षणः
यथा यस्य विधीयेत तथा वर्णय मे प्रभो
श्रीशुक उवाच
इत्थं स्वभृत्यमुख्येन पृष्टः स भगवान्हरिः
प्रीतः क्षेमाय मर्त्यानां धर्मानाह सनातनान्‌
श्रीभगवानुवाच
धर्म्य एष तव प्रश्नो नैःश्रेयसकरो नृणाम्‌
वर्णाश्रमाचारवतां तमुद्धव निबोध मे
आदौ कृतयुगे वर्णो नृणां हंस इति स्मृतः
कृतकृत्याः प्रजा जात्या तस्मात्कृतयुगं विदुः
वेदः प्रणव एवाग्रे धर्मोऽहं वृषरूपधृक्‌
उपासते तपोनिष्ठा हंसं मां मुक्तकिल्बिषाः
त्रेतामुखे महाभाग प्राणान्मे हृदयात्त्रयी
विद्या प्रादुरभूत्तस्या अहमासं त्रिवृन्मखः
विप्रक्षत्रियविट्शूद्रा मुखबाहूरुपादजाः
वैराजात्पुरुषाज्जाता य आत्माचारलक्षणाः
गृहाश्रमो जघनतो ब्रह्मचर्यं हृदो मम
वक्षःस्थलाद्वनेवासः सन्न्यासः शिरसि स्थितः
वर्णानामाश्रमाणां च जन्मभूम्यनुसारिणीः
आसन्प्रकृतयो न्ऱ्‌^ईनां नीचैर्नीचोत्तमोत्तमाः
शमो दमस्तपः शौचं सन्तोषः क्षान्तिरार्जवम्‌
मद्भक्तिश्च दया सत्यं ब्रह्मप्रकृतयस्त्विमाः
तेजो बलं धृतिः शौर्यं तितिक्षौदार्यमुद्यमः
स्थैर्यं ब्रह्मन्यमैश्वर्यं क्षत्रप्रकृतयस्त्विमाः
आस्तिक्यं दाननिष्ठा च अदम्भो ब्रह्मसेवनम्‌
अतुष्टिरर्थोपचयैर्वैश्यप्रकृतयस्त्विमाः
शुश्रूषणं द्विजगवां देवानां चाप्यमायया
तत्र लब्धेन सन्तोषः शूद्रप्रकृतयस्त्विमाः
अशौचमनृतं स्तेयं नास्तिक्यं शुष्कविग्रहः
कामः क्रोधश्च तर्षश्च स भावोऽन्त्यावसायिनाम्‌
अहिंसा सत्यमस्तेयमकामक्रोधलोभता
भूतप्रियहितेहा च धर्मोऽयं सार्ववर्णिकः
द्वितीयं प्राप्यानुपूर्व्याज्जन्मोपनयनं द्विजः
वसन्गुरुकुले दान्तो ब्रह्माधीयीत चाहूतः
मेखलाजिनदण्डाक्ष ब्रह्मसूत्रकमण्डलून्‌
जटिलोऽधौतदद्वासोऽरक्तपीठः कुशान्दधत्‌
स्नानभोजनहोमेषु जपोच्चारे च वाग्यतः
न च्छिन्द्यान्नखरोमाणि कक्षोपस्थगतान्यपि
रेतो नावकिरेज्जातु ब्रह्मव्रतधरः स्वयम्‌
अवकीर्णेऽवगाह्याप्सु यतासुस्त्रिपदां जपेत्‌
अग्न्यर्काचार्यगोविप्र गुरुवृद्धसुराञ्शुचिः
समाहित उपासीत सन्ध्ये द्वे यतवाग्जपन्‌
आचार्यं मां विजानीयान्नावन्मन्येत कर्हिचित्‌
न मर्त्यबुद्ध्यासूयेत सर्वदेवमयो गुरुः
सायं प्रातरुपानीय भैक्ष्यं तस्मै निवेदयेत्‌
यच्चान्यदप्यनुज्ञातमुपयुञ्जीत संयतः
शुश्रूषमाण आचार्यं सदोपासीत नीचवत्‌
यानशय्यासनस्थानैर्नातिदूरे कृताञ्जलिः
एवंवृत्तो गुरुकुले वसेद्भोगविवर्जितः
विद्या समाप्यते यावद्बिभ्रद्व्रतमखण्डितम्‌
यद्यसौ छन्दसां लोकमारोक्ष्यन्ब्रह्मविष्टपम्‌
गुरवे विन्यसेद्देहं स्वाध्यायार्थं बृहद्व्रतः
अग्नौ गुरावात्मनि च सर्वभूतेषु मां परम्‌
अपृथग्धीरुपसीत ब्रह्मवर्चस्व्यकल्मषः
स्त्रीणां निरीक्षणस्पर्श संलापक्ष्वेलनादिकम्‌
प्राणिनो मिथुनीभूतानगृहस्थोऽग्रतस्त्यजेत्‌
शौचमाचमनं स्नानं सन्ध्योपास्तिर्ममार्चनम्‌
तीर्थसेवा जपोऽस्पृश्या भक्ष्यासम्भाष्यवर्जनम्‌
सर्वाश्रमप्रयुक्तोऽयं नियमः कुलनन्दन
मद्भावः सर्वभूतेषु मनोवाक्कायसंयमः
एवं बृहद्व्रतधरो ब्राह्मणोऽग्निरिव ज्वलन्‌
मद्भक्तस्तीव्रतपसा दग्धकर्माशयोऽमलः
अथानन्तरमावेक्ष्यन्यथाजिज्ञासितागमः
गुरवे दक्षिणां दत्त्वा स्नायाद्गुर्वनुमोदितः
गृहं वनं वोपविशेत्प्रव्रजेद्वा द्विजोत्तमः
आश्रमादाश्रमं गच्छेन्नान्यथामत्परश्चरेत्‌
गृहार्थी सदृशीं भार्यामुद्वहेदजुगुप्सिताम्‌
यवीयसीं तु वयसा यं सवर्णामनु क्रमात्‌
इज्याध्ययनदानानि सर्वेषां च द्विजन्मनाम्‌
प्रतिग्रहोऽध्यापनं च ब्राह्मणस्यैव याजनम्‌
प्रतिग्रहं मन्यमानस्तपस्तेजोयशोनुदम्‌
अन्याभ्यामेव जीवेत शिलैर्वा दोषदृक्तयोः
ब्राह्मणस्य हि देहोऽयं क्षुद्रकामाय नेष्यते
कृच्छ्राय तपसे चेह प्रेत्यानन्तसुखाय च
शिलोञ्छवृत्त्या परितुष्टचित्तो धर्मं महान्तं विरजं जुषाणः
मय्यर्पितात्मा गृह एव तिष्ठन्नातिप्रसक्तः समुपैति शान्तिम्‌
समुद्धरन्ति ये विप्रं सीदन्तं मत्परायणम्‌
तानुद्धरिष्ये न चिरादापद्भ्यो नौरिवार्णवात्‌
सर्वाः समुद्धरेद्राजा पितेव व्यसनात्प्रजाः
आत्मानमात्मना धीरो यथा गजपतिर्गजान्‌
एवंविधो नरपतिर्विमानेनार्कवर्चसा
विधूयेहाशुभं कृत्स्नमिन्द्रेण सह मोदते
सीदन्विप्रो वणिग्वृत्त्या पण्यैरेवापदं तरेत्‌
खड्गेन वापदाक्रान्तो न श्ववृत्त्या कथञ्चन
वैश्यवृत्त्या तु राजन्यो जीवेन्मृगययापदि
चरेद्वा विप्ररूपेण न श्ववृत्त्या कथञ्चन
शूद्रवृत्तिं भजेद्वैश्यः शूद्रः कारुकटक्रियाम्‌
कृच्छ्रान्मुक्तो न गर्ह्येण वृत्तिं लिप्सेत कर्मणा
वेदाध्यायस्वधास्वाहा बल्यन्नाद्यैर्यथोदयम्‌
देवर्षिपितृभूतानि मद्रूपाण्यन्वहं यजेत्‌
यदृच्छयोपपन्नेन शुक्लेनोपार्जितेन वा
धनेनापीडयन्भृत्यान्न्यायेनैवाहरेत्क्रतून्‌
कुटुम्बेषु न सज्जेत न प्रमाद्येत्कुटुम्ब्यपि
विपश्चिन्नश्वरं पश्येददृष्टमपि दृष्टवत्‌
पुत्रदाराप्तबन्धूनां सङ्गमः पान्थसङ्गमः
अनुदेहं वियन्त्येते स्वप्नो निद्रानुगो यथा
इत्थं परिमृशन्मुक्तो गृहेष्वतिथिवद्वसन्‌
न गृहैरनुबध्येत निर्ममो निरहङ्कृतः
कर्मभिर्गृहमेधीयैरिष्ट्वा मामेव भक्तिमान्‌
तिष्ठेद्वनं वोपविशेत्प्रजावान्वा परिव्रजेत्‌
यस्त्वासक्तमतिर्गेहे पुत्रवित्तैषणातुरः
स्त्रैणः कृपणधीर्मूढो ममाहमिति बध्यते
अहो मे पितरौ वृद्धौ भार्या बालात्मजात्मजाः
अनाथा मामृते दीनाः कथं जीवन्ति दुःखिताः
एवं गृहाशयाक्षिप्त हृदयो मूढधीरयम्‌
अतृप्तस्ताननुध्यायन्मृतोऽन्धं विशते तमः
श्रीभगवानुवाच
वनं विविक्षुः पुत्रेषु भार्यां न्यस्य सहैव वा
वन एव वसेच्छान्तस्तृतीयं भागमायुषः
कन्दमूलफलैर्वन्यैर्मेध्यैर्वृत्तिं प्रकल्पयेत्‌
वसीत वल्कलं वासस्तृणपर्णाजिनानि वा
केशरोमनखश्मश्रु मलानि बिभृयाद्दतः
न धावेदप्सु मज्जेत त्रि कालं स्थण्डिलेशयः
ग्रीष्मे तप्येत पञ्चाग्नीन्वर्षास्वासारषाड्जले
आकण्थमग्नः शिशिर एवं वृत्तस्तपश्चरेत्‌
अग्निपक्वं समश्नीयात्कालपक्वमथापि वा
उलूखलाश्मकुट्टो वा दन्तोलूखल एव वा
स्वयं सञ्चिनुयात्सर्वमात्मनो वृत्तिकारणम्‌
देशकालबलाभिज्ञो नाददीतान्यदाहृतम्‌
वन्यैश्चरुपुरोडाशैर्निर्वपेत्कालचोदितान्‌
न तु श्रौतेन पशुना मां यजेत वनाश्रमी
अग्निहोत्रं च दर्शश्च पौर्णमासश्च पूर्ववत्‌
चातुर्मास्यानि च मुनेराम्नातानि च नैगमैः
एवं चीर्णेन तपसा मुनिर्धमनिसन्ततः
मां तपोमयमाराध्य ऋषिलोकादुपैति माम्‌
यस्त्वेतत्कृच्छ्रतश्चीर्णं तपो निःश्रेयसं महत्‌
कामायाल्पीयसे युञ्ज्याद्बालिशः कोऽपरस्ततः
यदासौ नियमेऽकल्पो जरया जातवेपथुः
आत्मन्यग्नीन्समारोप्य मच्चित्तोऽग्निं समाविशेत्‌
यदा कर्मविपाकेषु लोकेषु निरयात्मसु
विरागो जायते सम्यङ्न्यस्ताग्निः प्रव्रजेत्ततः
इष्ट्वा यथोपदेशं मां दत्त्वा सर्वस्वमृत्विजे
अग्नीन्स्वप्राण आवेश्य निरपेक्षः परिव्रजेत्‌
विप्रस्य वै सन्न्यसतो देवा दारादिरूपिणः
विघ्नान्कुर्वन्त्ययं ह्यस्मानाक्रम्य समियात्परम्‌
बिभृयाच्चेन्मुनिर्वासः कौपीनाच्छादनं परम्‌
त्यक्तं न दण्डपात्राभ्यामन्यत्किञ्चिदनापदि
दृष्टिपूतं न्यसेत्पादं वस्त्रपूतं पिबेज्जलम्‌
सत्यपूतां वदेद्वाचं मनःपूतं समाचरेत्‌
मौनानीहानिलायामा दण्डा वाग्देहचेतसाम्‌
न ह्येते यस्य सन्त्यङ्ग वेणुभिर्न भवेद्यतिः
भिक्षां चतुर्षु वर्णेषु विगर्ह्यान्वर्जयंश्चरेत्‌
सप्तागारानसङ्कॢप्तांस्तुष्येल्लब्धेन तावता
बहिर्जलाशयं गत्वा तत्रोपस्पृश्य वाग्यतः
विभज्य पावितं शेषं भुञ्जीताशेषमाहृतम्‌
एकश्चरेन्महीमेतां निःसङ्गः संयतेन्द्रियः
आत्मक्रीड आत्मरत आत्मवान्समदर्शनः
विविक्तक्षेमशरणो मद्भावविमलाशयः
आत्मानं चिन्तयेदेकमभेदेन मया मुनिः
अन्वीक्षेतात्मनो बन्धं मोक्षं च ज्ञाननिष्ठया
बन्ध इन्द्रियविक्षेपो मोक्ष एषां च संयमः
तस्मान्नियम्य षड्वर्गं मद्भावेन चरेन्मुनिः
विरक्तः क्षुद्रकामेभ्यो लब्ध्वात्मनि सुखं महत्‌
पुरग्रामव्रजान्सार्थान्भिक्षार्थं प्रविशंश्चरेत्‌
पुण्यदेशसरिच्छैल वनाश्रमवतीं महीम्‌
वानप्रस्थाश्रमपदेष्वभीक्ष्णं भैक्ष्यमाचरेत्‌
संसिध्यत्याश्वसम्मोहः शुद्धसत्त्वः शिलान्धसा
नैतद्वस्तुतया पश्येद्दृश्यमानं विनश्यति
असक्तचित्तो विरमेदिहामुत्रचिकीर्षितात्‌
यदेतदात्मनि जगन्मनोवाक्प्राणसंहतम्‌
सर्वं मायेति तर्केण स्वस्थस्त्यक्त्वा न तत्स्मरेत्‌
ज्ञाननिष्ठो विरक्तो वा मद्भक्तो वानपेक्षकः
सलिङ्गानाश्रमांस्त्यक्त्वा चरेदविधिगोचरः
बुधो बालकवत्क्रीडेत्कुशलो जडवच्चरेत्‌
वदेदुन्मत्तवद्विद्वान्गोचर्यां नैगमश्चरेत्‌
वेदवादरतो न स्यान्न पाषण्डी न हैतुकः
शुष्कवादविवादे न कञ्चित्पक्षं समाश्रयेत्‌
नोद्विजेत जनाद्धीरो जनं चोद्वेजयेन्न तु
अतिवादांस्तितिक्षेत नावमन्येत कञ्चन
देहमुद्दिश्य पशुवद्वैरं कुर्यान्न केनचित्‌
एक एव परो ह्यात्मा भूतेष्वात्मन्यवस्थितः
यथेन्दुरुदपात्रेषु भूतान्येकात्मकानि च
अलब्ध्वा न विषीदेत काले कालेऽशनं क्वचित्‌
लब्ध्वा न हृष्येद्धृतिमानुभयं दैवतन्त्रितम्‌
आहारार्थं समीहेत युक्तं तत्प्राणधारणम्‌
तत्त्वं विमृश्यते तेन तद्विज्ञाय विमुच्यते
यदृच्छयोपपन्नान्नमद्याच्छ्रेष्ठमुतापरम्‌
तथा वासस्तथा शय्यां प्राप्तं प्राप्तं भजेन्मुनिः
शौचमाचमनं स्नानं न तु चोदनया चरेत्‌
अन्यांश्च नियमाञ्ज्ञानी यथाहं लीलयेश्वरः
न हि तस्य विकल्पाख्या या च मद्वीक्षया हता
आदेहान्तात्क्वचित्ख्यातिस्ततः सम्पद्यते मया
दुःखोदर्केषु कामेषु जातनिर्वेद आत्मवान्‌
अज्ज्ञासितमद्धर्मो मुनिं गुरुमुपव्रजेत्‌
तावत्परिचरेद्भक्तः श्रद्धावाननसूयकः
यावद्ब्रह्म विजानीयान्मामेव गुरुमादृतः
यस्त्वसंयतषड्वर्गः प्रचण्डेन्द्रियसारथिः
ज्ञानवैराग्यरहितस्त्रिदण्डमुपजीवति
सुरानात्मानमात्मस्थं निह्नुते मां च धर्महा
अविपक्वकषायोऽस्मादमुष्माच्च विहीयते
भिक्षोर्धर्मः शमोऽहिंसा तप ईक्षा वनौकसः
गृहिणो भूतरक्षेज्या द्विजस्याचार्यसेवनम्‌
ब्रह्मचर्यं तपः शौचं सन्तोषो भूतसौहृदम्‌
गृहस्थस्याप्यृतौ गन्तुः सर्वेषां मदुपासनम्‌
इति मां यः स्वधर्मेण भजेन्नित्यमनन्यभाक्‌
सर्वभूतेषु मद्भावो मद्भक्तिं विन्दते दृढाम्‌
भक्त्योद्धवानपायिन्या सर्वलोकमहेश्वरम्‌
सर्वोत्पत्त्यप्ययं ब्रह्म कारणं मोपयाति सः
इति स्वधर्मनिर्णिक्त सत्त्वो निर्ज्ञातमद्गतिः
ज्ञानविज्ञानसम्पन्नो न चिरात्समुपैति माम्‌
वर्णाश्रमवतां धर्म एष आचारलक्षणः
स एव मद्भक्तियुतो निःश्रेयसकरः परः
एतत्तेऽभिहितं साधो भवान्पृच्छति यच्च माम्‌
यथा स्वधर्मसंयुक्तो भक्तो मां समियात्परम्‌
श्रीभगवानुवाच
यो विद्याश्रुतसम्पन्नः आत्मवान्नानुमानिकः
मयामात्रमिदं ज्ञात्वा ज्ञानं च मयि सन्न्यसेत्‌
ज्ञानिनस्त्वहमेवेष्टः स्वार्थो हेतुश्च सम्मतः
स्वर्गश्चैवापवर्गश्च नान्योऽर्थो मदृते प्रियः
ज्ञानविज्ञानसंसिद्धाः पदं श्रेष्ठं विदुर्मम
ज्ञानी प्रियतमोऽतो मे ज्ञानेनासौ बिभर्ति माम्‌
तपस्तीर्थं जपो दानं पवित्राणीतराणि च
नालं कुर्वन्ति तां सिद्धिं या ज्ञानकलया कृता
तस्माज्ज्ञानेन सहितं ज्ञात्वा स्वात्मानमुद्धव
ज्ञानविज्ञानसम्पन्नो भज मां भक्तिभावतः
ज्ञानविज्ञानयज्ञेन मामिष्ट्वात्मानमात्मनि
सर्वयज्ञपतिं मां वै संसिद्धिं मुनयोऽगमन्‌
त्वय्युद्धवाश्रयति यस्त्रिविधो विकारो
मायान्तरापतति नाद्यपवर्गयोर्यत्‌
जन्मादयोऽस्य यदमी तव तस्य किं स्युर्‌
आद्यन्तयोर्यदसतोऽस्ति तदेव मध्ये
श्रीउद्धव उवाच
ज्ञानं विशुद्धं विपुलं यथैतद्वैराग्यविज्ञानयुतं पुराणम्‌
आख्याहि विश्वेश्वर विश्वमूर्ते त्वद्भक्तियोगं च महद्विमृग्यम्‌
तापत्रयेणाभिहतस्य घोरे सन्तप्यमानस्य भवाध्वनीश
पश्यामि नान्यच्छरणं तवाङ्घ्रि द्वन्द्वातपत्रादमृताभिवर्षात्‌
दष्टं जनं सम्पतितं बिलेऽस्मिन्कालाहिना क्षुद्रसुखोरुतर्षम्‌
समुद्धरैनं कृपयापवर्ग्यैर्वचोभिरासिञ्च महानुभाव
श्रीभगवानुवाच
इत्थमेतत्पुरा राजा भीष्मं धर्मभृतां वरम्‌
अजातशत्रुः पप्रच्छ सर्वेषां नोऽनुशृण्वताम्‌
निवृत्ते भारते युद्धे सुहृन्निधनविह्वलः
श्रुत्वा धर्मान्बहून्पश्चान्मोक्षधर्मानपृच्छत
तानहं तेऽभिधास्यामि देवव्रतमखाच्छ्रुतान्‌
ज्ञानवैराग्यविज्ञान श्रद्धाभक्त्युपबृंहितान्‌
नवैकादश पञ्च त्रीन्भावान्भूतेषु येन वै
ईक्षेताथाइकमप्येषु तज्ज्ञानं मम निश्चितम्‌
एतदेव हि विज्ञानं न तथैकेन येन यत्‌
स्थित्युत्पत्त्यप्ययान्पश्येद्भावानां त्रिगुणात्मनाम्‌
आदावन्ते च मध्ये च सृज्यात्सृज्यं यदन्वियात्‌
पुनस्तत्प्रतिसङ्क्रामे यच्छिष्येत तदेव सत्‌
श्रुतिः प्रत्यक्षमैतिह्यमनुमानं चतुष्टयम्‌
प्रमाणेष्वनवस्थानाद्विकल्पात्स विरज्यते
कर्मणां परिणामित्वादाविरिञ्च्यादमङ्गलम्‌
विपश्चिन्नश्वरं पश्येददृष्टमपि दृष्टवत्‌
भक्तियोगः पुरैवोक्तः प्रीयमाणाय तेऽनघ
पुनश्च कथयिष्यामि मद्भक्तेः कारणं परं
श्रद्धामृतकथायां मे शश्वन्मदनुकीर्तनम्‌
परिनिष्ठा च पूजायां स्तुतिभिः स्तवनं मम
आदरः परिचर्यायां सर्वाङ्गैरभिवन्दनम्‌
मद्भक्तपूजाभ्यधिका सर्वभूतेषु मन्मतिः
मदर्थेष्वङ्गचेष्टा च वचसा मद्गुणेरणम्‌
मय्यर्पणं च मनसः सर्वकामविवर्जनम्‌
मदर्थेऽर्थपरित्यागो भोगस्य च सुखस्य च
इष्टं दत्तं हुतं जप्तं मदर्थं यद्व्रतं तपः
एवं धर्मैर्मनुष्याणामुद्धवात्मनिवेदिनाम्‌
मयि सञ्जायते भक्तिः कोऽन्योऽर्थोऽस्यावशिष्यते
यदात्मन्यर्पितं चित्तं शान्तं सत्त्वोपबृंहितम्‌
धर्मं ज्ञानं स वैराग्यमैश्वर्यं चाभिपद्यते
यदर्पितं तद्विकल्पे इन्द्रियैः परिधावति
रजस्वलं चासन्निष्ठं चित्तं विद्धि विपर्ययम्‌
धर्मो मद्भक्तिकृत्प्रोक्तो ज्ञानं चैकात्म्यदर्शनम्‌
गुणेस्वसङ्गो वैराग्यमैश्वर्यं चाणिमादयः
श्रीउद्धव उवाच यमः कतिविधः प्रोक्तो
नियमो वारिकर्षण कः शमः को दमः कृष्ण
का तितिक्षा धृतिः प्रभो किं दानं किं तपः शौर्यं
किम्सत्यमृतमुच्यते कस्त्यागः किं धनं चेष्टं
को यज्ञः का च दक्षिणा पुंसः किं स्विद्बलं श्रीमन्‌
भगो लाभश्च केशव का विद्या ह्रीः परा का श्रीः
किं सुखं दुःखमेव च कः पण्डितः कश्च मूर्खः
कः पन्था उत्पथश्च कः कः स्वर्गो नरकः कः स्वित्‌
को बन्धुरुत किं गृहम्क आढ्यः को दरिद्रो वा
कृपणः कः क ईश्वरः एतान्प्रश्नान्मम ब्रूहि
विपरीतांश्च सत्पते श्रीभगवानुवाच
अहिंसा सत्यमस्तेयमसङ्गो ह्रीरसञ्चयः
आस्तिक्यं ब्रह्मचर्यं च मौनं स्थैर्यं क्षमाभयम्‌
शौचं जपस्तपो होमः श्रद्धातिथ्यं मदर्चनम्‌
तीर्थाटनं परार्थेहा तुष्टिराचार्यसेवनम्‌
एते यमाः सनियमा उभयोर्द्वादश स्मृताः
पुंसामुपासितास्तात यथाकामं दुहन्ति हि
शमो मन्निष्ठता बुद्धेर्दम इन्द्रियसंयमः
तितिक्षा दुःखसम्मर्षो जिह्वोपस्थजयो धृतिः
दण्डन्यासः परं दानं कामत्यागस्तपः स्मृतम्‌
स्वभावविजयः शौर्यं सत्यं च समदर्शनम्‌
अन्यच्च सुनृता वाणी कविभिः परिकीर्तिता
कर्मस्वसङ्गमः शौचं त्यागः सन्न्यास उच्यते
धर्म इष्टं धनं न्ऱ्‌^ईणां यज्ञोऽहं भगवत्तमः
दक्षिणा ज्ञानसन्देशः प्राणायामः परं बलम्‌
भगो म ऐश्वरो भावो लाभो मद्भक्तिरुत्तमः
विद्यात्मनि भिदाबाधो जुगुप्सा ह्रीरकर्मसु
श्रीर्गुणा नैरपेक्ष्याद्याः सुखं दुःखसुखात्ययः
दुःखं कामसुखापेक्षा पण्डितो बन्धमोक्षवित्‌
मूर्खो देहाद्यहंबुद्धिः पन्था मन्निगमः स्मृतः
उत्पथश्चित्तविक्षेपः स्वर्गः सत्त्वगुणोदयः
नरकस्तमौन्नाहो बन्धुर्गुरुरहं सखे
गृहं शरीरं मानुष्यं गुणाढ्यो ह्याढ्य उच्यते
दरिद्रो यस्त्वसन्तुष्टः कृपणो योऽजितेन्द्रियः
गुणेष्वसक्तधीरीशो गुणसङ्गो विपर्ययः
एत उद्धव ते प्रश्नाः सर्वे साधु निरूपिताः
किं वर्णितेन बहुना लक्षणं गुणदोषयोः
गुणदोषदृशिर्दोषो गुणस्तूभयवर्जितः
श्रीउद्धव उवाच
विधिश्च प्रतिषेधश्च निगमो हीश्वरस्य ते
अवेक्षतेऽरविण्डाक्ष गुणं दोषं च कर्मणाम्‌
वर्णाश्रमविकल्पं च प्रतिलोमानुलोमजम्‌
द्रव्यदेशवयःकालान्स्वर्गं नरकमेव च
गुणदोषभिदादृष्टिमन्तरेण वचस्तव
निःश्रेयसं कथं न्ऱ्‌^ईणां निषेधविधिलक्षणम्‌
पितृदेवमनुष्यानां वेदश्चक्षुस्तवेश्वर
श्रेयस्त्वनुपलब्धेऽर्थे साध्यसाधनयोरपि
गुणदोषभिदादृष्टिर्निगमात्ते न हि स्वतः
निगमेनापवादश्च भिदाया इति ह भ्रमः
श्रीभगवानुवाच
योगास्त्रयो मया प्रोक्ता न्ऱ्‌^ईणां श्रेयोविधित्सया
ज्ञानं कर्म च भक्तिश्च नोपायोऽन्योऽस्ति कुत्रचित्‌
निर्विण्णानां ज्ञानयोगो न्यासिनामिह कर्मसु
तेष्वनिर्विण्णचित्तानां कर्मयोगस्तु कामिनाम्‌
यदृच्छया मत्कथादौ जातश्रद्धस्तु यः पुमान्‌
न निर्विण्णो नातिसक्तो भक्तियोगोऽस्य सिद्धिदः
तावत्कर्माणि कुर्वीत न निर्विद्येत यावता
मत्कथाश्रवणादौ वा श्रद्धा यावन्न जायते
स्वधर्मस्थो यजन्यज्ञैरनाशीःकाम उद्धव
न याति स्वर्गनरकौ यद्यन्यन्न समाचरेत्‌
अस्मिंल्लोके वर्तमानः स्वधर्मस्थोऽनघः शुचिः
ज्ञानं विशुद्धमाप्नोति मद्भक्तिं वा यदृच्छया
स्वर्गिणोऽप्येतमिच्छन्ति लोकं निरयिणस्तथा
साधकं ज्ञानभक्तिभ्यामुभयं तदसाधकम्‌
न नरः स्वर्गतिं काङ्क्षेन्नारकीं वा विचक्षणः
नेमं लोकं च काङ्क्षेत देहावेशात्प्रमाद्यति
एतद्विद्वान्पुरा मृत्योरभवाय घटेत सः
अप्रमत्त इदं ज्ञात्वा मर्त्यमप्यर्थसिद्धिदम्‌
छिद्यमानं यमैरेतैः कृतनीडं वनस्पतिम्‌
खगः स्वकेतमुत्सृज्य क्षेमं याति ह्यलम्पटः
अहोरात्रैश्छिद्यमानं बुद्ध्वायुर्भयवेपथुः
मुक्तसङ्गः परं बुद्ध्वा निरीह उपशाम्यति
नृदेहमाद्यं सुलभं सुदुर्लभं
प्लवं सुकल्पं गुरुकर्णधारम्‌
मयानुकूलेन नभस्वतेरितं
पुमान्भवाब्धिं न तरेत्स आत्महा
यदारम्भेषु निर्विण्णो विरक्तः संयतेन्द्रियः
अभ्यासेनात्मनो योगी धारयेदचलं मनः
धार्यमाणं मनो यर्हि भ्राम्यदश्वनवस्थितम्‌
अतन्द्रितोऽनुरोधेन मार्गेणात्मवशं नयेत्‌
मनोगतिं न विसृजेज्जितप्राणो जितेन्द्रियः
सत्त्वसम्पन्नया बुद्ध्या मन आत्मवशं नयेत्‌
एष वै परमो योगो मनसः सङ्ग्रहः स्मृतः
हृदयज्ञत्वमन्विच्छन्दम्यस्येवार्वतो मुहुः
साङ्ख्येन सर्वभावानां प्रतिलोमानुलोमतः
भवाप्ययावनुध्यायेन्मनो यावत्प्रसीदति
निर्विण्णस्य विरक्तस्य पुरुषस्योक्तवेदिनः
मनस्त्यजति दौरात्म्यं चिन्तितस्यानुचिन्तया
यमादिभिर्योगपथैरान्वीक्षिक्या च विद्यया
ममार्चोपासनाभिर्वा नान्यैर्योग्यं स्मरेन्मनः
यदि कुर्यात्प्रमादेन योगी कर्म विगर्हितम्‌
योगेनैव दहेदंहो नान्यत्तत्र कदाचन
स्वे स्वेऽधिकारे या निष्ठा स गुणः परिकीर्तितः
कर्मणां जात्यशुद्धानामनेन नियमः कृतः
गुणदोषविधानेन सङ्गानां त्याजनेच्छया
जातश्रद्धो मत्कथासु निर्विण्णः सर्वकर्मसु
वेद दुःखात्मकान्कामान्परित्यागेऽप्यनीश्वरः
ततो भजेत मां प्रीतः श्रद्धालुर्दृढनिश्चयः
जुषमाणश्च तान्कामान्दुःखोदर्कांश्च गर्हयन्‌
प्रोक्तेन भक्तियोगेन भजतो मासकृन्मुनेः
कामा हृदय्या नश्यन्ति सर्वे मयि हृदि स्थिते
भिद्यते हृदयग्रन्थिश्छिद्यन्ते सर्वसंशयाः
क्षीयन्ते चास्य कर्माणि मयि दृष्टेऽखिलात्मनि
तस्मान्मद्भक्तियुक्तस्य योगिनो वै मदात्मनः
न ज्ञानं न च वैराग्यं प्रायः श्रेयो भवेदिह
यत्कर्मभिर्यत्तपसा ज्ञानवैराग्यतश्च यत्‌
योगेन दानधर्मेण श्रेयोभिरितरैरपि
सर्वं मद्भक्तियोगेन मद्भक्तो लभतेऽञ्जसा
स्वर्गापवर्गं मद्धाम कथञ्चिद्यदि वाञ्छति
न किञ्चित्साधवो धीरा भक्ता ह्येकान्तिनो मम
वाञ्छन्त्यपि मया दत्तं कैवल्यमपुनर्भवम्‌
नैरपेक्ष्यं परं प्राहुर्निःश्रेयसमनल्पकम्‌
तस्मान्निराशिषो भक्तिर्निरपेक्षस्य मे भवेत्‌
न मय्येकान्तभक्तानां गुणदोषोद्भवा गुणाः
साधूनां समचित्तानां बुद्धेः परमुपेयुषाम्‌
एवमेतान्मया दिष्टाननुतिष्ठन्ति मे पथः
क्षेमं विन्दन्ति मत्स्थानं यद्ब्रह्म परमं विदुः
श्रीभगवानुवाच
य एतान्मत्पथो हित्वा भक्तिज्ञानक्रियात्मकान्‌
क्षुद्रान्कामांश्चलैः प्राणैर्जुषन्तः संसरन्ति ते
स्वे स्वेऽधिकारे या निष्ठा स गुणः परिकीर्तितः
विपर्ययस्तु दोषः स्यादुभयोरेष निश्चयः
शुद्ध्यशुद्धी विधीयेते समानेष्वपि वस्तुषु
द्रव्यस्य विचिकित्सार्थं गुणदोषौ शुभाशुभौ
धर्मार्थं व्यवहारार्थं यात्रार्थमिति चानघ
दर्शितोऽयं मयाचारो
धर्ममुद्वहतां धुरम्‌
भूम्यम्ब्वग्न्यनिलाकाशा भूतानां पञ्चधातवः
आब्रह्मस्थावरादीनां शारीरा आत्मसंयुताः
वेदेन नामरूपाणि विषमाणि समेष्वपि
धातुषूद्धव कल्प्यन्त एतेषां स्वार्थसिद्धये
देशकालादिभावानां वस्तूनां मम सत्तम
गुणदोषौ विधीयेते नियमार्थं हि कर्मणाम्‌
अकृष्णसारो देशानामब्रह्मण्योऽसुचिर्भवेत्‌
कृष्णसारोऽप्यसौवीर कीकटासंस्कृतेरिणम्‌
कर्मण्यो गुणवान्कालो द्रव्यतः स्वत एव वा
यतो निवर्तते कर्म स दोषोऽकर्मकः स्मृतः
द्रव्यस्य शुद्ध्यशुद्धी च द्रव्येण वचनेन च
संस्कारेणाथ कालेन महत्वाल्पतयाथ वा
शक्त्याशक्त्याथ वा बुद्ध्या समृद्ध्या च यदात्मने
अघं कुर्वन्ति हि यथा देशावस्थानुसारतः
धान्यदार्वस्थितन्तूनां रसतैजसचर्मणाम्‌
कालवाय्वग्निमृत्तोयैः पार्थिवानां युतायुतैः
अमेध्यलिप्तं यद्येन गन्धलेपं व्यपोहति
भजते प्रकृतिं तस्य तच्छौचं तावदिष्यते
स्नानदानतपोऽवस्था वीर्यसंस्कारकर्मभिः
मत्स्मृत्या चात्मनः शौचं शुद्धः कर्माचरेद्द्विजः
मन्त्रस्य च परिज्ञानं कर्मशुद्धिर्मदर्पणम्‌
धर्मः सम्पद्यते षड्भिरधर्मस्तु विपर्ययः
क्वचिद्गुणोऽपि दोषः स्याद्दोषोऽपि विधिना गुणः
गुणदोषार्थनियमस्तद्भिदामेव बाधते
समानकर्माचरणं पतितानां न पातकम्‌
औत्पत्तिको गुणः सङ्गो न शयानः पतत्यधः
यतो यतो निवर्तेत विमुच्येत ततस्ततः
एष धर्मो नृणां क्षेमः शोकमोहभयापहः
विषयेषु गुणाध्यासात्पुंसः सङ्गस्ततो भवेत्‌
सङ्गात्तत्र भवेत्कामः कामादेव कलिर्नृणाम्‌
कलेर्दुर्विषहः क्रोधस्तमस्तमनुवर्तते
तमसा ग्रस्यते पुंसश्चेतना व्यापिनी द्रुतम्‌
तया विरहितः साधो जन्तुः शून्याय कल्पते
ततोऽस्य स्वार्थविभ्रंशो मूर्च्छितस्य मृतस्य च
विषयाभिनिवेशेन नात्मानं वेद नापरम्‌
वृक्ष जीविकया जीवन्व्यर्थं भस्त्रेव यः श्वसन्‌
फलश्रुतिरियं न्ऱ्‌^ईणां न श्रेयो रोचनं परम्‌
श्रेयोविवक्षया प्रोक्तं यथा भैषज्यरोचनम्‌
उत्पत्त्यैव हि कामेषु प्राणेषु स्वजनेषु च
आसक्तमनसो मर्त्या आत्मनोऽनर्थहेतुषु
नतानविदुषः स्वार्थं भ्राम्यतो वृजिनाध्वनि
कथं युञ्ज्यात्पुनस्तेषु तांस्तमो विशतो बुधः
एवं व्यवसितं केचिदविज्ञाय कुबुद्धयः
फलश्रुतिं कुसुमितां न वेदज्ञा वदन्ति हि
कामिनः कृपणा लुब्धाः पुष्पेषु फलबुद्धयः
अग्निमुग्धा धूमतान्ताः स्वं लोकं न विदन्ति ते
न ते मामङ्ग जानन्ति हृदिस्थं य इदं यतः
उक्थशस्त्रा ह्यसुतृपो यथा नीहारचक्षुषः
ते मे मतमविज्ञाय परोक्षं विषयात्मकाः
हिंसायां यदि रागः स्याद्यज्ञ एव न चोदना
हिंसाविहारा ह्यालब्धैः पशुभिः स्वसुखेच्छया
यजन्ते देवता यज्ञैः पितृभूतपतीन्खलाः
स्वप्नोपमममुं लोकमसन्तं श्रवणप्रियम्‌
आशिषो हृदि सङ्कल्प्य त्यजन्त्यर्थान्यथा वणिक्‌
रजःसत्त्वतमोनिष्ठा रजःसत्त्वतमोजुषः
उपासत इन्द्रमुख्यान्देवादीन्न यथैव माम्‌
इष्ट्वेह देवता यज्ञैर्गत्वा रंस्यामहे दिवि
तस्यान्त इह भूयास्म महाशाला महाकुलाः
एवं पुष्पितया वाचा व्याक्षिप्तमनसां नृणाम्‌
मानिनां चातिलुब्धानां मद्वार्तापि न रोचते
वेदा ब्रह्मात्मविषयास्त्रिकाण्डविषया इमे
परोक्षवादा ऋषयः परोक्षं मम च प्रियम्‌
शब्दब्रह्म सुदुर्बोधं प्राणेन्द्रियमनोमयम्‌
अनन्तपारं गम्भीरं दुर्विगाह्यं समुद्रवत्‌
मयोपबृंहितं भूम्ना ब्रह्मणानन्तशक्तिना
भूतेषु घोषरूपेण विसेषूर्णेव लक्ष्यते
यथोर्णनाभिर्हृदयादूर्णामुद्वमते मुखात्‌
आकाशाद्घोषवान्प्राणो मनसा स्पर्शरूपिणा
छन्दोमयोऽमृतमयः सहस्रपदवीं प्रभुः
ओंकाराद्व्यञ्जितस्पर्श स्वरोष्मान्तस्थभूषिताम्‌
विचित्रभाषाविततां छन्दोभिश्चतुरुत्तरैः
अनन्तपारां बृहतीं सृजत्याक्षिपते स्वयम्‌
गायत्र्युष्णिगनुष्टुप्च बृहती पङ्क्तिरेव च
त्रिष्टुब्जगत्यतिच्छन्दो ह्यत्यष्ट्यतिजगद्विराट्‌
किं विधत्ते किमाचष्टे किमनूद्य विकल्पयेत्‌
इत्यस्या हृदयं लोके नान्यो मद्वेद कश्चन
मां विधत्तेऽभिधत्ते मां विकल्प्यापोह्यते त्वहम्‌
एतावान्सर्ववेदार्थः शब्द आस्थाय मां भिदाम्‌
मायामात्रमनूद्यान्ते प्रतिषिध्य प्रसीदति
श्रीउद्धव उवाच
कति तत्त्वानि विश्वेश सङ्ख्यातान्यृषिभिः प्रभो
नवैकादश पञ्च त्रीण्यात्थ त्वमिह शुश्रुम
केचित्षड्विंशतिं प्राहुरपरे पञ्चविंशतिं
सप्तैके नव षट्केचिच्चत्वार्येकादशापरे
केचित्सप्तदश प्राहुः षोडशैके त्रयोदश
एतावत्त्वं हि सङ्ख्यानामृषयो यद्विवक्षया
गायन्ति पृथगायुष्मन्निदं नो वक्तुमर्हसि
श्रीभगवानुवाच
युक्तं च सन्ति सर्वत्र भाषन्ते ब्राह्मणा यथा
मायां मदीयामुद्गृह्य वदतां किं नु दुर्घटम्‌
नैतदेवं यथात्थ त्वं यदहं वच्मि तत्तथा
एवं विवदतां हेतुं शक्तयो मे दुरत्ययाः
यासां व्यतिकरादासीद्विकल्पो वदतां पदम्‌
प्राप्ते शमदमेऽप्येति वादस्तमनु शाम्यति
परस्परानुप्रवेशात्तत्त्वानां पुरुषर्षभ
पौर्वापर्यप्रसङ्ख्यानं यथा वक्तुर्विवक्षितम्‌
एकस्मिन्नपि दृश्यन्ते प्रविष्टानीतराणि च
पूर्वस्मिन्वा परस्मिन्वा तत्त्वे तत्त्वानि सर्वशः
पौर्वापर्यमतोऽमीषां प्रसङ्ख्यानमभीप्सताम्‌
यथा विविक्तं यद्वक्त्रं गृह्णीमो युक्तिसम्भवात्‌
अनाद्यविद्यायुक्तस्य पुरुषस्यात्मवेदनम्‌
स्वतो न सम्भवादन्यस्तत्त्वज्ञो ज्ञानदो भवेत्‌
पुरुषेश्वरयोरत्र न वैलक्षण्यमण्वपि
तदन्यकल्पनापार्था ज्ञानं च प्रकृतेर्गुणः
प्रकृतिर्गुणसाम्यं वै प्रकृतेर्नात्मनो गुणाः
सत्त्वं रजस्तम इति स्थित्युत्पत्त्यन्तहेतवः
सत्त्वं ज्ञानं रजः कर्म तमोऽज्ञानमिहोच्यते
गुणव्यतिकरः कालः स्वभावः सूत्रमेव च
पुरुषः प्रकृतिर्व्यक्तमहङ्कारो नभोऽनिलः
ज्योतिरापः क्षितिरिति तत्त्वान्युक्तानि मे नव
श्रोत्रं त्वग्दर्शनं घ्राणो जिह्वेति ज्ञानशक्तयः
वाक्पाण्युपस्थपाय्वङ्घ्रिः कर्माण्यङ्गोभयं मनः
शब्दः स्पर्शो रसो गन्धो रूपं चेत्यर्थजातयः
गत्युक्त्युत्सर्गशिल्पानि कर्मायतनसिद्धयः
सर्गादौ प्रकृतिर्ह्यस्य कार्यकारणरूपिणी
सत्त्वादिभिर्गुणैर्धत्ते पुरुषोऽव्यक्त ईक्षते
व्यक्तादायो विकुर्वाणा धातवः पुरुषेक्षया
लब्धवीर्याः सृजन्त्यण्डं संहताः प्रकृतेर्बलात्‌
सप्तैव धातव इति तत्रार्थाः पञ्च खादयः
ज्ञानमात्मोभयाधारस्ततो देहेन्द्रियासवः
षडित्यत्रापि भूतानि पञ्च षष्ठः परः पुमान्‌
तैर्युइत आत्मसम्भूतैः सृष्ट्वेदं समपाविशत्‌
चत्वार्येवेति तत्रापि तेज आपोऽन्नमात्मनः
जातानि तैरिदं जातं जन्मावयविनः खलु
सङ्ख्याने सप्तदशके भूतमात्रेन्द्रियाणि च
पञ्च पञ्चैकमनसा आत्मा सप्तदशः स्मृतः
तद्वत्षोडशसङ्ख्याने आत्मैव मन उच्यते
भूतेन्द्रियाणि पञ्चैव मन आत्मा त्रयोदश
एकादशत्व आत्मासौ महाभूतेन्द्रियाणि च
अष्टौ प्रकृतयश्चैव पुरुषश्च नवेत्यथ
इति नानाप्रसङ्ख्यानं तत्त्वानामृषिभिः कृतम्‌
सर्वं न्याय्यं युक्तिमत्त्वाद्विदुषां किमशोभनम्‌
श्रीउद्धव उवाच
प्रकृतिः पुरुषश्चोभौ यद्यप्यात्मविलक्षणौ
अन्योन्यापाश्रयात्कृष्ण दृश्यते न भिदा तयोः
प्रकृतौ लक्ष्यते ह्यात्मा प्रकृतिश्च तथात्मनि
एवं मे पुण्डरीकाक्ष महान्तं संशयं हृदि
छेत्तुमर्हसि सर्वज्ञ वचोभिर्नयनैपुणैः
त्वत्तो ज्ञानं हि जीवानां प्रमोषस्तेऽत्र शक्तितः
त्वमेव ह्यात्ममायाया गतिं वेत्थ न चापरः
श्रीभगवानुवाच
प्रकृतिः पुरुषश्चेति विकल्पः पुरुषर्षभ
एष वैकारिकः सर्गो गुणव्यतिकरात्मकः
ममाङ्ग माया गुणमय्यनेकधा विकल्पबुद्धीश्च गुणैर्विधत्ते
वैकारिकस्त्रिविधोऽध्यात्ममेकमथाधिदैवमधिभूतमन्यत्‌
दृग्रूपमार्कं वपुरत्र रन्ध्रे परस्परं सिध्यति यः स्वतः खे
आत्मा यदेषामपरो य आद्यः स्वयानुभूत्याखिलसिद्धसिद्धिः
एवं त्वगादि श्रवणादि चक्षुर्‌
जिह्वादि नासादि च चित्तयुक्तम्‌
योऽसौ गुणक्षोभकृतो विकारः प्रधानमूलान्महतः प्रसूतः
अहं त्रिवृन्मोहविकल्पहेतुर्वैकारिकस्तामस ऐन्द्रियश्च
आत्मापरिज्ञानमयो विवादो ह्यस्तीति नास्तीति भिदार्थनिष्ठः
व्यर्थोऽपि नैवोपरमेत पुंसां मत्तः परावृत्तधियां स्वलोकात्‌
श्रीउद्धव उवाच
त्वत्तः परावृत्तधियः स्वकृतैः कर्मभिः प्रभो
उच्चावचान्यथा देहान्गृह्णन्ति विसृजन्ति च
तन्ममाख्याहि गोविन्द दुर्विभाव्यमनात्मभिः
न ह्येतत्प्रायशो लोके विद्वांसः सन्ति वञ्चिताः
श्रीभगवानुवाच
मनः कर्ममयं ण्ऱ्‌^ईणामिन्द्रियैः पञ्चभिर्युतम्‌
लोकाल्लोकं प्रयात्यन्य आत्मा तदनुवर्तते
ध्यायन्मनोऽनु विषयान्दृष्टान्वानुश्रुतानथ
उद्यत्सीदत्कर्मतन्त्रं स्मृतिस्तदनु शाम्यति
विषयाभिनिवेशेन नात्मानं यत्स्मरेत्पुनः
जन्तोर्वै कस्यचिद्धेतोर्मृत्युरत्यन्तविस्मृतिः
जन्म त्वात्मतया पुंसः सर्वभावेन भूरिद
विषयस्वीकृतिं प्राहुर्यथा स्वप्नमनोरथः
स्वप्नं मनोरथं चेत्थं प्राक्तनं न स्मरत्यसौ
तत्र पूर्वमिवात्मानमपूर्वम्चानुपश्यति
इन्द्रियायनसृष्ट्येदं त्रैविध्यं भाति वस्तुनि
बहिरन्तर्भिदाहेतुर्जनोऽसज्जनकृद्यथा
नित्यदा ह्यङ्ग भूतानि भवन्ति न भवन्ति च
कालेनालक्ष्यवेगेन सूक्ष्मत्वात्तन्न दृश्यते
यथार्चिषां स्रोतसां च फलानां वा वनस्पतेः
तथैव सर्वभूतानां वयोऽवस्थादयः कृताः
सोऽयं दीपोऽर्चिषां यद्वत्स्रोतसां तदिदं जलम्‌
सोऽयं पुमानिति नृणां मृषा गीर्धीर्मृषायुषाम्‌
मा स्वस्य कर्मबीजेन जायते सोऽप्ययं पुमान्‌
म्रियते वामरो भ्रान्त्या यथाग्निर्दारुसंयुतः
निषेकगर्भजन्मानि बाल्यकौमारयौवनम्‌
वयोमध्यं जरा मृत्युरित्यवस्थास्तनोर्नव
एता मनोरथमयीर्हान्यस्योच्चावचास्तनूः
गुणसङ्गादुपादत्ते क्वचित्कश्चिज्जहाति च
आत्मनः पितृपुत्राभ्यामनुमेयौ भवाप्ययौ
न भवाप्ययवस्तूनामभिज्ञो द्वयलक्षणः
तरोर्बीजविपाकाभ्यां यो विद्वाञ्जन्मसंयमौ
तरोर्विलक्षणो द्रष्टा एवं द्रष्टा तनोः पृथक्‌
प्रकृतेरेवमात्मानमविविच्याबुधः पुमान्‌
तत्त्वेन स्पर्शसम्मूढः संसारं प्रतिपद्यते
सत्त्वसङ्गादृषीन्देवान्रजसासुरमानुषान्‌
तमसा भूततिर्यक्त्वं भ्रामितो याति कर्मभिः
नृत्यतो गायतः पश्यन्यथैवानुकरोति तान्‌
एवं बुद्धिगुणान्पश्यन्ननीहोऽप्यनुकार्यते
यथाम्भसा प्रचलता तरवोऽपि चला इव
चक्षुसा भ्राम्यमाणेन दृश्यते भ्रमतीव भूः
यथा मनोरथधियो विषय्षानुभवो मृषा
स्वप्नदृष्टाश्च दाशार्ह तथा संसार आत्मनः
अर्थे ह्यविद्यमानेऽपि संसृतिर्न निवर्तते
ध्यायतो विषयानस्य स्वप्नेऽनर्थागमो यथा
तस्मादुद्धव मा भुङ्क्ष्व विषयानसदिन्द्रियैः
आत्माग्रहणनिर्भातं पश्य वैकल्पिकं भ्रमम्‌
क्षिप्तोऽवमानितोऽसद्भिः प्रलब्धोऽसूयितोऽथ वा
ताडितः सन्निरुद्धो वा वृत्त्या वा परिहापितः
निष्ठ्युतो मूत्रितो वाज्ञैर्बहुधैवं प्रकम्पितः
श्रेयस्कामः कृच्छ्रगत आत्मनात्मानमुद्धरेत्‌
श्रीउद्धव उवाच
यथैवमनुबुध्येयं
वद नो वदतां वर
सुदुःषहमिमं मन्य आत्मन्यसदतिक्रमम्‌
विदुषामपि विश्वात्मन्प्रकृतिर्हि बलीयसी
ऋते त्वद्धर्मनिरतान्शान्तांस्ते चरणालयान्‌
श्रीबादरायणिरुवाच
स एवमाशंसित उद्धवेन भागवतमुख्येन दाशार्हमुख्यः
सभाजयन्भृत्यवचो मुकुन्दस्तमाबभाषे श्रवणीयवीर्यः
श्रीभगवानुवाच
बार्हस्पत्य स नास्त्यत्र साधुर्वै दुर्जनेरितैः
दुरक्तैर्भिन्नमात्मानं यः समाधातुमीश्वरः
न तथा तप्यते विद्धः पुमान्बाणैस्तु मर्मगैः
यथा तुदन्ति मर्मस्था ह्यसतां परुषेषवः
कथयन्ति महत्पुण्यमितिहासमिहोद्धव
तमहं वर्णयिष्यामि निबोध सुसमाहितः
केनचिद्भिक्षुणा गीतं परिभूतेन दुर्जनैः
स्मरता धृतियुक्तेन विपाकं निजकर्मणाम्‌
अवन्तिषु द्विजः कश्चिदासीदाढ्यतमः श्रिया
वार्तावृत्तिः कदर्यस्तु कामी लुब्धोऽतिकोपनः
ज्ञातयोऽतिथयस्तस्य वाङ्मात्रेणापि नार्चिताः
शून्यावसथ आत्मापि काले कामैरनर्चितः
दुह्शीलस्य कदर्यस्य द्रुह्यन्ते पुत्रबान्धवाः
दारा दुहितरो भृत्या विषण्णा नाचरन्प्रियम्‌
तस्यैवं यक्षवित्तस्य च्युतस्योभयलोकतः
धर्मकामविहीनस्य चुक्रुधुः पञ्चभागिनः
तदवध्यानविस्रस्त पुण्यस्कन्धस्य भूरिद
अर्थोऽप्यगच्छन्निधनं बह्वायासपरिश्रमः
ज्ञात्यो जगृहुः किञ्चित्किञ्चिद्दस्यव उद्धव
दैवतः कालतः किञ्चिद्ब्रह्मबन्धोर्नृपार्थिवात्‌
स एवं द्रविणे नष्टे धर्मकामविवर्जितः
उपेक्षितश्च स्वजनैश्चिन्तामाप दुरत्ययाम्‌
तस्यैवं ध्यायतो दीर्घं नष्टरायस्तपस्विनः
खिद्यतो बाष्पकण्ठस्य निर्वेदः सुमहानभूत्‌
स चाहेदमहो कष्टं वृथात्मा मेऽनुतापितः
न धर्माय न कामाय यस्यार्थायास ईदृशः
प्रायेणाथाः कदर्याणां न सुखाय कदाचन
इह चात्मोपतापाय मृतस्य नरकाय च
यशो यशस्विनां शुद्धं श्लाघ्या ये गुणिनां गुणाः
लोभः स्वल्पोऽपि तान्हन्ति श्वित्रो रूपमिवेप्सितम्‌
अर्थस्य साधने सिद्धे उत्कर्षे रक्षणे व्यये
नाशोपभोग आयासस्त्रासश्चिन्ता भ्रमो नृणाम्‌
स्तेयं हिंसानृतं दम्भः कामः क्रोधः स्मयो मदः
भेदो वैरमविश्वासः संस्पर्धा व्यसनानि च
एते पञ्चदशानर्था ह्यर्थमूला मता नृणाम्‌
तस्मादनर्थमर्थाख्यं श्रेयोऽर्थी दूरतस्त्यजेत्‌
भिद्यन्ते भ्रातरो दाराः पितरः सुहृदस्तथा
एकास्निग्धाः काकिणिना सद्यः सर्वेऽरयः कृताः
अर्थेनाल्पीयसा ह्येते संरब्धा दीप्तमन्यवः
त्यजन्त्याशु स्पृधो घ्नन्ति सहसोत्सृज्य सौहृदम्‌
लब्ध्वा जन्मामरप्रार्थ्यं मानुष्यं तद्द्विजाग्र्यताम्‌
तदनादृत्य ये स्वार्थं घ्नन्ति यान्त्यशुभां गतिम्‌
स्वर्गापवर्गयोर्द्वारं प्राप्य लोकमिमं पुमान्‌
द्रविणे कोऽनुषज्जेत मर्त्योऽनर्थस्य धामनि
देवर्षिपितृभूतानि ज्ञातीन्बन्धूंश्च भागिनः
असंविभज्य चात्मानं यक्षवित्तः पतत्यधः
व्यर्थयार्थेहया वित्तं प्रमत्तस्य वयो बलम्‌
कुशला येन सिध्यन्ति जरठः किं नु साधये
कस्मात्सङ्क्लिश्यते विद्वान्व्यर्थयार्थेहयासकृत्‌
कस्यचिन्मायया नूनं लोकोऽयं सुविमोहितः
किं धनैर्धनदैर्वा किं कामैर्वा कामदैरुत
मृत्युना ग्रस्यमानस्य कर्मभिर्वोत जन्मदैः
नूनं मे भगवांस्तुष्टः सर्वदेवमयो हरिः
येन नीतो दशामेतां निर्वेदश्चात्मनः प्लवः
सोऽहं कालावशेषेण शोषयिष्येऽङ्गमात्मनः
अप्रमत्तोऽखिलस्वार्थे यदि स्यात्सिद्ध आत्मनि
तत्र मामनुमोदेरन्देवास्त्रिभुवनेश्वराः
मुहूर्तेन ब्रह्मलोकं खट्वाङ्गः समसाधयत्‌
श्रीभगवानुवाच
इत्यभिप्रेत्य मनसा ह्यावन्त्यो द्विजसत्तमः
उन्मुच्य हृदयग्रन्थीन्शान्तो भिक्षुरभून्मुनिः
स चचार महीमेतां संयतात्मेन्द्रियानिलः
भिक्षार्थं नगरग्रामानसङ्गोऽलक्षितोऽविशत्‌
तं वै प्रवयसं भिक्षुमवधूतमसज्जनाः
दृष्ट्वा पर्यभवन्भद्र बह्वीभिः परिभूतिभिः
केचित्त्रिवेणुं जगृहुरेके पात्रं कमण्डलुम्‌
पीठं चैकेऽक्षसूत्रं च कन्थां चीराणि केचन
प्रदाय च पुनस्तानि दर्शितान्याददुर्मुनेः
अन्नं च भैक्ष्यसम्पन्नं भुञ्जानस्य सरित्तटे
मूत्रयन्ति च पापिष्ठाः ष्ठीवन्त्यस्य च मूर्धनि
यतवाचं वाचयन्ति ताडयन्ति न वक्ति चेत्‌
तर्जयन्त्यपरे वाग्भिः स्तेनोऽयमिति वादिनः
बध्नन्ति रज्ज्वा तं केचिद्बध्यतां बध्यतामिति
क्षिपन्त्येकेऽवजानन्त एष धर्मध्वजः शठः
क्षीणवित्त इमां वृत्तिमग्रहीत्स्वजनोज्झितः
अहो एष महासारो धृतिमान्गिरिराडिव
मौनेन साधयत्यर्थं बकवद्दृढनिश्चयः
इत्येके विहसन्त्येनमेके दुर्वातयन्ति च
तं बबन्धुर्निरुरुधुर्यथा क्रीडनकं द्विजम्‌
एवं स भौतिकं दुःखं दैविकं दैहिकं च यत्‌
भोक्तव्यमात्मनो दिष्टं प्राप्तं प्राप्तमबुध्यत
परिभूत इमां गाथामगायत नराधमैः
पातयद्भिः स्व धर्मस्थो धृतिमास्थाय सात्त्विकीम्‌
द्विज उवाच
नायं जनो मे सुखदुःखहेतुर्न देवतात्मा ग्रहकर्मकालाः
मनः परं कारणमामनन्ति संसारचक्रं परिवर्तयेद्यत्‌
मनो गुणान्वै सृजते बलीयस्ततश्च कर्माणि विलक्षणानि
शुक्लानि कृष्णान्यथ लोहितानि तेभ्यः सवर्णाः सृतयो भवन्ति
अनीह आत्मा मनसा समीहता हिरण्मयो मत्सख उद्विचष्टे
मनः स्वलिङ्गं परिगृह्य कामान्जुषन्निबद्धो गुणसङ्गतोऽसौ
दानं स्वधर्मो नियमो यमश्च श्रुतं च कर्माणि च सद्व्रतानि
सर्वे मनोनिग्रहलक्षणान्ताः परो हि योगो मनसः समाधिः
समाहितं यस्य मनः प्रशान्तं दानादिभिः किं वद तस्य कृत्यम्‌
असंयतं यस्य मनो विनश्यद्दानादिभिश्चेदपरं किमेभिः
मनोवशेऽन्ये ह्यभवन्स्म देवा मनश्च नान्यस्य वशं समेति
भीष्मो हि देवः सहसः सहीयान्युञ्ज्याद्वशे तं स हि देवदेवः
तम्दुर्जयं शत्रुमसह्यवेगमरुन्तुदं तन्न विजित्य केचित्‌
कुर्वन्त्यसद्विग्रहमत्र मर्त्यैर्मित्राण्युदासीनरिपून्विमूढाः
देहं मनोमात्रमिमं गृहीत्वा ममाहमित्यन्धधियो मनुष्याः
एषोऽहमन्योऽयमिति भ्रमेण दुरन्तपारे तमसि भ्रमन्ति
जनस्तु हेतुः सुखदुःखयोश्चेत्किमात्मनश्चात्र हि भौमयोस्तत्‌
जिह्वां क्वचित्सन्दशति स्वदद्भिस्तद्वेदनायां कतमाय कुप्येत्‌
दुःखस्य हेतुर्यदि देवतास्तु किमात्मनस्तत्र विकारयोस्तत्‌
यदङ्गमङ्गेन निहन्यते क्वचित्क्रुध्येत कस्मै पुरुषः स्वदेहे
आत्मा यदि स्यात्सुखदुःखहेतुः किमन्यतस्तत्र निजस्वभावः
न ह्यात्मनोऽन्यद्यदि तन्मृषा स्यात्क्रुध्येत कस्मान्न सुखं न दुःखम्‌
ग्रहा निमित्तं सुखदुःखयोश्चेत्किमात्मनोऽजस्य जनस्य ते वै
ग्रहैर्ग्रहस्यैव वदन्ति पीडां क्रुध्येत कस्मै पुरुषस्ततोऽन्यः
कर्मास्तु हेतुः सुखदुःखयोश्चेत्किमात्मनस्तद्धि जडाजडत्वे
देहस्त्वचित्पुरुषोऽयं सुपर्णः क्रुध्येत कस्मै न हि कर्म मूलम्‌
कालस्तु हेतुः सुखदुःखयोश्चेत्किमात्मनस्तत्र तदात्मकोऽसौ
नाग्नेर्हि तापो न हिमस्य तत्स्यात्क्रुध्येत कस्मै न परस्य द्वन्द्वम्‌
न केनचित्क्वापि कथञ्चनास्य द्वन्द्वोपरागः परतः परस्य
यथाहमः संसृतिरूपिणः स्यादेवं प्रबुद्धो न बिभेति भूतैः
एतां स आस्थाय परात्मनिष्ठामध्यासितां पूर्वतमैर्महर्षिभिः
अहं तरिष्यामि दुरन्तपारं तमो मुकुन्दाङ्घ्रिनिषेवयैव
श्रीभगवानुवाच
निर्विद्य नष्टद्रविणे गतक्लमः प्रव्रज्य गां पर्यटमान इत्थम्‌
निराकृतोऽसद्भिरपि स्वधर्मादकम्पितोऽमूं मुनिराह गाथाम्‌
सुखदुःखप्रदो नान्यः पुरुषस्यात्मविभ्रमः
मित्रोदासीनरिपवः संसारस्तमसः कृतः
तस्मात्सर्वात्मना तात निगृहाण मनो धिया
मय्यावेशितया युक्त एतावान्योगसङ्ग्रहः
य एतां भिक्षुणा गीतां ब्रह्मनिष्ठां समाहितः
धारयञ्छ्रावयञ्छृण्वन्द्वन्द्वैर्नैवाभिभूयते
श्रीभगवानुवाच
अथ ते सम्प्रवक्ष्यामि साङ्ख्यं पूर्वैर्विनिश्चितम्‌
यद्विज्ञाय पुमान्सद्यो जह्याद्वैकल्पिकं भ्रमम्‌
आसीज्ज्ञानमथो अर्थ एकमेवाविकल्पितम्‌
यदा विवेकनिपुणा आदौ कृतयुगेऽयुगे
तन्मायाफलरूपेण केवलं निर्विकल्पितम्‌
वाङ्मनोऽगोचरं सत्यं द्विधा समभवद्बृहत्‌
तयोरेकतरो ह्यर्थः प्रकृतिः सोभयात्मिका
ज्ञानं त्वन्यतमो भावः पुरुषः सोऽभिधीयते
तमो रजः सत्त्वमिति प्रकृतेरभवन्गुणाः
मया प्रक्षोभ्यमाणायाः पुरुषानुमतेन च
तेभ्यः समभवत्सूत्रं महान्सूत्रेण संयुतः
ततो विकुर्वतो जातो योऽहङ्कारो विमोहनः
वैकारिकस्तैजसश्च तामसश्चेत्यहं त्रिवृत्‌
तन्मात्रेन्द्रियमनसां कारणं चिदचिन्मयः
अर्थस्तन्मात्रिकाज्जज्ञे तामसादिन्द्रियाणि च
तैजसाद्देवता आसन्नेकादश च वैकृतात्‌
मया सञ्चोदिता भावाः सर्वे संहत्यकारिणः
अण्डमुत्पादयामासुर्ममायतनमुत्तमम्‌
तस्मिन्नहं समभवमण्डे सलिलसंस्थितौ
मम नाभ्यामभूत्पद्मं विश्वाख्यं तत्र चात्मभूः
सोऽसृजत्तपसा युक्तो रजसा मदनुग्रहात्‌
लोकान्सपालान्विश्वात्मा भूर्भुवः स्वरिति त्रिधा
देवानामोक आसीत्स्वर्भूतानां च भुवः पदम्‌
मर्त्यादीनां च भूर्लोकः सिद्धानां त्रितयात्परम्‌
अधोऽसुराणां नागानां भूमेरोकोऽसृजत्प्रभुः
त्रिलोक्यां गतयः सर्वाः कर्मणां त्रिगुणात्मनाम्‌
योगस्य तपसश्चैव न्यासस्य गतयोऽमलाः
महर्जनस्तपः सत्यं भक्तियोगस्य मद्गतिः
मया कालात्मना धात्रा कर्मयुक्तमिदं जगत्‌
गुणप्रवाह एतस्मिन्नुन्मज्जति निमज्जति
अणुर्बृहत्कृशः स्थूलो यो यो भावः प्रसिध्यति
सर्वोऽप्युभयसंयुक्तः प्रकृत्या पुरुषेण च
यस्तु यस्यादिरन्तश्च स वै मध्यं च तस्य सन्‌
विकारो व्यवहारार्थो यथा तैजसपार्थिवाः
यदुपादाय पूर्वस्तु भावो विकुरुतेऽपरम्‌
आदिरन्तो यदा यस्य तत्सत्यमभिधीयते
प्रकृतिर्यस्योपादानमाधारः पुरुषः परः
सतोऽभिव्यञ्जकः कालो ब्रह्म तत्त्रितयं त्वहम्‌
सर्गः प्रवर्तते तावत्पौर्वापर्येण नित्यशः
महान्गुणविसर्गार्थः स्थित्यन्तो यावदीक्षणम्‌
विराण्मयासाद्यमानो लोककल्पविकल्पकः
पञ्चत्वाय विशेषाय कल्पते भुवनैः सह
अन्ने प्रलीयते मर्त्यमन्नं धानासु लीयते
धाना भूमौ प्रलीयन्ते भूमिर्गन्धे प्रलीयते
अप्सु प्रलीयते गन्ध आपश्च स्वगुणे रसे
लीयते ज्योतिषि रसो ज्योती रूपे प्रलीयते
रूपं वायौ स च स्पर्शे लीयते सोऽपि चाम्बरे
अम्बरं शब्दतन्मात्र इन्द्रियाणि स्वयोनिषु
योनिर्वैकारिके सौम्य लीयते मनसीश्वरे
शब्दो भूतादिमप्येति भूतादिर्महति प्रभुः
स लीयते महान्स्वेषु गुणेसु गुणवत्तमः
तेऽव्यक्ते सम्प्रलीयन्ते तत्काले लीयतेऽव्यये
कालो मायामये जीवे जीव आत्मनि मय्यजे
आत्मा केवल आत्मस्थो विकल्पापायलक्षणः
एवमन्वीक्षमाणस्य कथं वैकल्पिको भ्रमः
मनसो हृदि तिष्ठेत व्योम्नीवार्कोदये तमः
एष साङ्ख्यविधिः प्रोक्तः संशयग्रन्थिभेदनः
प्रतिलोमानुलोमाभ्यां परावरदृश मया
श्रीभगवानुवाच
गुणानामसम्मिश्राणां पुमान्येन यथा भवेत्‌
तन्मे पुरुषवर्येदमुपधारय शंसतः
शमो दमस्तितिक्षेक्षा तपः सत्यं दया स्मृतिः
तुष्टिस्त्यागोऽस्पृहा श्रद्धा ह्रीर्दयादिः स्वनिर्वृतिः
काम ईहा मदस्तृष्णा स्तम्भ आशीर्भिदा सुखम्‌
मदोत्साहो यशःप्रीतिर्हास्यं वीर्यं बलोद्यमः
क्रोधो लोभोऽनृतं हिंसा याच्ञा दम्भः क्लमः कलिः
शोकमोहौ विषादार्ती निद्राशा भीरनुद्यमः
सत्त्वस्य रजसश्चैतास्तमसश्चानुपूर्वशः
वृत्तयो वर्णितप्रायाः सन्निपातमथो शृणु
सन्निपातस्त्वहमिति ममेत्युद्धव या मतिः
व्यवहारः सन्निपातो मनोमात्रेन्द्रियासुभिः
धर्मे चार्थे च कामे च यदासौ परिनिष्ठितः
गुणानां सन्निकर्षोऽयं श्रद्धारतिधनावहः
प्रवृत्तिलक्षणे निष्ठा पुमान्यर्हि गृहाश्रमे
स्वधर्मे चानु तिष्ठेत गुणानां समितिर्हि सा
पुरुषं सत्त्वसंयुक्तमनुमीयाच्छमादिभिः
कामादिभी रजोयुक्तं क्रोधाद्यैस्तमसा युतम्‌
यदा भजति मां भक्त्या निरपेक्षः स्वकर्मभिः
तं सत्त्वप्रकृतिं विद्यात्पुरुषं स्त्रियमेव वा
यदा आशिष आशास्य मां भजेत स्वकर्मभिः
तं रजःप्रकृतिं विद्याथिंसामाशास्य तामसम्‌
सत्त्वं रजस्तम इति गुणा जीवस्य नैव मे
चित्तजा यैस्तु भूतानां सज्जमानो निबध्यते
यदेतरौ जयेत्सत्त्वं भास्वरं विशदं शिवम्‌
तदा सुखेन युज्येत धर्मज्ञानादिभिः पुमान्‌
यदा जयेत्तमः सत्त्वं रजः सङ्गं भिदा चलम्‌
तदा दुःखेन युज्येत कर्मणा यशसा श्रिया
यदा जयेद्रजः सत्त्वं तमो मूढं लयं जडम्‌
युज्येत शोकमोहाभ्यां निद्रया हिंसयाशया
यदा चित्तं प्रसीदेत इन्द्रियाणां च निर्वृतिः
देहेऽभयं मनोऽसङ्गं तत्सत्त्वं विद्धि मत्पदम्‌
विकुर्वन्क्रियया चाधीरनिवृत्तिश्च चेतसाम्‌
गात्रास्वास्थ्यं मनो भ्रान्तं रज एतैर्निशामय
सीदच्चित्तं विलीयेत चेतसो ग्रहणेऽक्षमम्‌
मनो नष्टं तमो ग्लानिस्तमस्तदुपधारय
एधमाने गुणे सत्त्वे देवानां बलमेधते
असुराणां च रजसि तमस्युद्धव रक्षसाम्‌
सत्त्वाज्जागरणं विद्याद्रजसा स्वप्नमादिशेत्‌
प्रस्वापं तमसा जन्तोस्तुरीयं त्रिषु सन्ततम्‌
उपर्युपरि गच्छन्ति सत्त्वेन ब्राह्मणा जनाः
तमसाधोऽध आमुख्याद्रजसान्तरचारिणः
सत्त्वे प्रलीनाः स्वर्यान्ति नरलोकं रजोलयाः
तमोलयास्तु निरयं यान्ति मामेव निर्गुणाः
मदर्पणं निष्फलं वा सात्त्विकं निजकर्म तत्‌
राजसं फलसङ्कल्पं हिंसाप्रायादि तामसम्‌
कैवल्यं सात्त्विकं ज्ञानं रजो वैकल्पिकं च यत्‌
प्राकृतं तामसं ज्ञानं मन्निष्ठं निर्गुणं स्मृतम्‌
वनं तु सात्त्विको वासो ग्रामो राजस उच्यते
तामसं द्यूतसदनं मन्निकेतं तु निर्गुणम्‌
सात्त्विकः कारकोऽसङ्गी रागान्धो राजसः स्मृतः
तामसः स्मृतिविभ्रष्टो निर्गुणो मदपाश्रयः
सात्त्विक्याध्यात्मिकी श्रद्धा कर्मश्रद्धा तु राजसी
तामस्यधर्मे या श्रद्धा मत्सेवायां तु निर्गुणा
पथ्यं पूतमनायस्तमाहार्यं सात्त्विकं स्मृतम्‌
राजसं चेन्द्रियप्रेष्ठं तामसं चार्तिदाशुचि
सात्त्विकं सुखमात्मोत्थं विषयोत्थं तु राजसम्‌
तामसं मोहदैन्योत्थं निर्गुणं मदपाश्रयम्‌
द्रव्यं देशः फलं कालो ज्ञानं कर्म च कारकः
श्रद्धावस्थाकृतिर्निष्ठा त्रैगुण्यः सर्व एव हि
सर्वे गुणमया भावाः पुरुषाव्यक्तधिष्ठिताः
दृष्टं श्रुतं अनुध्यातं बुद्ध्या वा पुरुषर्षभ
एताः संसृतयः पुंसो गुणकर्मनिबन्धनाः
येनेमे निर्जिताः सौम्य गुणा जीवेन चित्तजाः
भक्तियोगेन मन्निष्ठो मद्भावाय प्रपद्यते
तस्माद्देहमिमं लब्ध्वा ज्ञानविज्ञानसम्भवम्‌
गुणसङ्गं विनिर्धूय मां भजन्तु विचक्षणाः
निःसङ्गो मां भजेद्विद्वानप्रमत्तो जितेन्द्रियः
रजस्तमश्चाभिजयेत्सत्त्वसंसेवया मुनिः
सत्त्वं चाभिजयेद्युक्तो नैरपेक्ष्येण शान्तधीः
सम्पद्यते गुणैर्मुक्तो जीवो जीवं विहाय माम्‌
जीवो जीवविनिर्मुक्तो गुणैश्चाशयसम्भवैः
मयैव ब्रह्मणा पूर्णो न बहिर्नान्तरश्चरेत्‌
श्रीभगवानुवाच
मल्लक्षणमिमं कायं लब्ध्वा मद्धर्म आस्थितः
आनन्दं परमात्मानमात्मस्थं समुपैति माम्‌
गुणमय्या जीवयोन्या विमुक्तो ज्ञाननिष्ठया
गुणेषु मायामात्रेषु दृश्यमानेष्ववस्तुतः
वर्तमानोऽपि न पुमान्युज्यतेऽवस्तुभिर्गुणैः
सङ्गं न कुर्यादसतां शिश्नोदरतृपां क्वचित्‌
तस्यानुगस्तमस्यन्धे पतत्यन्धानुगान्धवत्‌
ऐलः सम्राडिमां गाथामगायत बृहच्छ्रवाः
उर्वशीविरहान्मुह्यन्निर्विण्णः शोकसंयमे
त्यक्त्वात्मानं व्रयन्तीं तां नग्न उन्मत्तवन्नृपः
विलपन्नन्वगाज्जाये घोरे तिष्ठेति विक्लवः
कामानतृप्तोऽनुजुषन्क्षुल्लकान्वर्षयामिनीः
न वेद यान्तीर्नायान्तीरुर्वश्याकृष्टचेतनः
ऐल उवाच
अहो मे मोहविस्तारः कामकश्मलचेतसः
देव्या गृहीतकण्ठस्य नायुःखण्डा इमे स्मृताः
नाहं वेदाभिनिर्मुक्तः सूर्यो वाभ्युदितोऽमुया
मूषितो वर्षपूगानां बताहानि गतान्युत
अहो मे आत्मसम्मोहो येनात्मा योषितां कृतः
क्रीडामृगश्चक्रवर्ती नरदेवशिखामणिः
सपरिच्छदमात्मानं हित्वा तृणमिवेश्वरम्‌
यान्तीं स्त्रियं चान्वगमं नग्न उन्मत्तवद्रुदन्‌
कुतस्तस्यानुभावः स्यात्तेज ईशत्वमेव वा
योऽन्वगच्छं स्त्रियं यान्तीं खरवत्पादताडितः
किं विद्यया किं तपसा किं त्यागेन श्रुतेन वा
किं विविक्तेन मौनेन स्त्रीभिर्यस्य मनो हृतम्‌
स्वार्थस्याकोविदं धिङ्मां मूर्खं पण्डितमानिनम्‌
योऽहमीश्वरतां प्राप्य स्त्रीभिर्गोखरवज्जितः
सेवतो वर्षपूगान्मे उर्वश्या अधरासवम्‌
न तृप्यत्यात्मभूः कामो वह्निराहुतिभिर्यथा
पुंश्चल्यापहृतं चित्तं को न्वन्यो मोचितुं प्रभुः
आत्मारामेश्वरमृते भगवन्तमधोक्षजम्‌
बोधितस्यापि देव्या मे सूक्तवाक्येन दुर्मतेः
मनोगतो महामोहो नापयात्यजितात्मनः
किमेतया नोऽपकृतं रज्ज्वा वा सर्पचेतसः
द्रष्टुः स्वरूपाविदुषो योऽहं यदजितेन्द्रियः
क्वायं मलीमसः कायो दौर्गन्ध्याद्यात्मकोऽशुचिः
क्व गुणाः सौमनस्याद्या ह्यध्यासोऽविद्यया कृतः
पित्रोः किं स्वं नु भार्यायाः स्वामिनोऽग्नेः श्वगृध्रयोः
किमात्मनः किं सुहृदामिति यो नावसीयते
तस्मिन्कलेवरेऽमेध्ये तुच्छनिष्ठे विषज्जते
अहो सुभद्रं सुनसं सुस्मितं च मुखं स्त्रियः
त्वङ्मांसरुधिरस्नायु मेदोमज्जास्थिसंहतौ
विण्मूत्रपूये रमतां कृमीणां कियदन्तरम्‌
अथापि नोपसज्जेत स्त्रीषु स्त्रैणेषु चार्थवित्‌
विषयेन्द्रियसंयोगान्मनः क्षुभ्यति नान्यथा
अदृष्टादश्रुताद्भावान्न भाव उपजायते
असम्प्रयुञ्जतः प्राणान्शाम्यति स्तिमितं मनः
तस्मात्सङ्गो न कर्तव्यः स्त्रीषु स्त्रैणेषु चेन्द्रियैः
विदुषां चाप्यविस्रब्धः षड्वर्गः किमु मादृशाम्‌
श्रीभगवानुवाच
एवं प्रगायन्नृपदेवदेवः स उर्वशीलोकमथो विहाय
आत्मानमात्मन्यवगम्य मां वै उपारमज्ज्ञाअनविधूतमोहः
ततो दुःसङ्गमुत्सृज्य सत्सु सज्जेत बुद्धिमान्‌
सन्त एवास्य छिन्दन्ति मनोव्यासङ्गमुक्तिभिः
सन्तोऽनपेक्षा मच्चित्ताः प्रशान्ताः समदर्शिनः
निर्ममा निरहङ्कारा निर्द्वन्द्वा निष्परिग्रहाः
तेषु नित्यं महाभाग महाभागेषु मत्कथाः
सम्भवन्ति हि ता न्ऱ्‌^ईणां जुषतां प्रपुनन्त्यघम्‌
ता ये शृण्वन्ति गायन्ति ह्यनुमोदन्ति चादृताः
मत्पराः श्रद्दधानाश्च भक्तिं विन्दन्ति ते मयि
भक्तिं लब्धवतः साधोः किमन्यदवशिष्यते
मय्यनन्तगुणे ब्रह्मण्यानन्दानुभवात्मनि
यथोपश्रयमाणस्य भगवन्तं विभावसुम्‌
शीतं भयं तमोऽप्येति साधून्संसेवतस्तथा
निमज्ज्योन्मज्जतां घोरे भवाब्धौ परमायणम्‌
सन्तो ब्रह्मविदः शान्ता नौर्दृढेवाप्सु मज्जताम्‌
अन्नं हि प्राणिनां प्राण आर्तानां शरणं त्वहम्‌
धर्मो वित्तं नृणां प्रेत्य सन्तोऽर्वाग्बिभ्यतोऽरणम्‌
सन्तो दिशन्ति चक्षूंसि बहिरर्कः समुत्थितः
देवता बान्धवाः सन्तः सन्त आत्माहमेव च
वैतसेनस्ततोऽप्येवमुर्वश्या लोकनिष्पृहः
मुक्तसङ्गो महीमेतामात्मारामश्चचार ह
श्रीउद्धव उवाच
क्रियायोगं समाचक्ष्व भवदाराधनं प्रभो
यस्मात्त्वां ये यथार्चन्ति सात्वताः सात्वतर्षभ
एतद्वदन्ति मुनयो मुहुर्निःश्रेयसं नृणाम्‌
नारदो भगवान्व्यास आचार्योऽङ्गिरसः सुतः
निःसृतं ते मुखाम्भोजाद्यदाह भगवानजः
पुत्रेभ्यो भृगुमुख्येभ्यो देव्यै च भगवान्भवः
एतद्वै सर्ववर्णानामाश्रमाणां च सम्मतम्‌
श्रेयसामुत्तमं मन्ये स्त्रीशूद्राणां च मानद
एतत्कमलपत्राक्ष कर्मबन्धविमोचनम्‌
भक्ताय चानुरक्ताय ब्रूहि विश्वेश्वरेश्वर
श्रीभगवानुवाच
न ह्यन्तोऽनन्तपारस्य कर्मकाण्डस्य चोद्धव
सङ्क्षिप्तं वर्णयिष्यामि यथावदनुपूर्वशः
वैदिकस्तान्त्रिको मिश्र इति मे त्रिविधो मखः
त्रयाणामीप्सितेनैव विधिना मां समर्चरेत्‌
यदा स्वनिगमेनोक्तं द्विजत्वं प्राप्य पूरुषः
यथा यजेत मां भक्त्या श्रद्धया तन्निबोध मे
अर्चायां स्थण्डिलेऽग्नौ वा सूर्ये वाप्सु हृदि द्विजः
द्रव्येण भक्तियुक्तोऽर्चेत्स्वगुरुं माममायया
पूर्वं स्नानं प्रकुर्वीत धौतदन्तोऽङ्गशुद्धये
उभयैरपि च स्नानं मन्त्रैर्मृद्ग्रहणादिना
सन्ध्योपास्त्यादिकर्माणि वेदेनाचोदितानि मे
पूजां तैः कल्पयेत्सम्यक्‌ सङ्कल्पः कर्मपावनीम्‌
शैली दारुमयी लौही लेप्या लेख्या च सैकती
मनोमयी मणिमयी प्रतिमाष्टविधा स्मृता
चलाचलेति द्विविधा प्रतिष्ठा जीवमन्दिरम्‌
उद्वासावाहने न स्तः स्थिरायामुद्धवार्चने
अस्थिरायां विकल्पः स्यात्स्थण्डिले तु भवेद्द्वयम्‌
स्नपनं त्वविलेप्यायामन्यत्र परिमार्जनम्‌
द्रव्यैः प्रसिद्धैर्मद्यागः प्रतिमादिष्वमायिनः
भक्तस्य च यथालब्धैर्हृदि भावेन चैव हि
स्नानालङ्करणं प्रेष्ठमर्चायामेव तूद्धव
स्थण्डिले तत्त्वविन्यासो वह्नावाज्यप्लुतं हविः
सूर्ये चाभ्यर्हणं प्रेष्ठं सलिले सलिलादिभिः
श्रद्धयोपाहृतं प्रेष्ठं भक्तेन मम वार्यपि
भूर्यप्यभक्तोपाहृतं न मे तोषाय कल्पते
गन्धो धूपः सुमनसो दीपोऽन्नाद्यं च किं पुनः
शुचिः सम्भृतसम्भारः प्राग्दर्भैः कल्पितासनः
आसीनः प्रागुदग्वार्चेदर्चायां त्वथ सम्मुखः
कृतन्यासः कृतन्यासां मदर्चां पाणिनामृजेत्‌
कलशं प्रोक्षणीयं च यथावदुपसाधयेत्‌
तदद्भिर्देवयजनं द्रव्याण्यात्मानमेव च
प्रोक्ष्य पात्राणि त्रीण्यद्भिस्तैस्तैर्द्रव्यैश्च साधयेत्‌
पाद्यार्घ्याचमनीयार्थं त्रीणि पात्राणि देशिकः
हृदा शीर्ष्णाथ शिखया गायत्र्या चाभिमन्त्रयेत्‌
पिण्डे वाय्वग्निसंशुद्धे हृत्पद्मस्थां परां मम
अण्वीं जीवकलां ध्यायेन्नादान्ते सिद्धभाविताम्‌
तयात्मभूतया पिण्डे व्याप्ते सम्पूज्य तन्मयः
आवाह्यार्चादिषु स्थाप्य न्यस्ताङ्गं मां प्रपूजयेत्‌
पाद्योपस्पर्शार्हणादीनुपचारान्प्रकल्पयेत्‌
धर्मादिभिश्च नवभिः कल्पयित्वासनं मम
पद्ममष्टदलं तत्र कर्णिकाकेसरोज्ज्वलम्‌
उभाभ्यां वेदतन्त्राभ्यां मह्यं तूभयसिद्धये
सुदर्शनं पाञ्चजन्यं गदासीषुधनुर्हलान्‌
मुषलं कौस्तुभं मालां श्रीवत्सं चानुपूजयेत्‌
नन्दं सुनन्दं गरुडं प्रचण्डं चण्डं एव च
महाबलं बलं चैव कुमुदं कमुदेक्षणम्‌
दुर्गां विनायकं व्यासं विष्वक्षेनं गुरून्सुरान्‌
स्वे स्वे स्थाने त्वभिमुखान्पूजयेत्प्रोक्षणादिभिः
चन्दनोशीरकर्पूर कुङ्कुमागुरुवासितैः
सलिलैः स्नापयेन्मन्त्रैर्नित्यदा विभवे सति
स्वर्णघर्मानुवाकेन महापुरुषविद्यया
पौरुषेणापि सूक्तेन सामभी राजनादिभिः
वस्त्रोपवीताभरण पत्रस्रग्गन्धलेपनैः
अलङ्कुर्वीत सप्रेम मद्भक्तो मां यथोचितम्‌
पाद्यमाचमनीयं च गन्धं सुमनसोऽक्षतान्‌
धूपदीपोपहार्याणि दद्यान्मे श्रद्धयार्चकः
गुडपायससर्पींषि शष्कुल्यापूपमोदकान्‌
संयावदधिसूपांश्च नैवेद्यं सति कल्पयेत्‌
अभ्यङ्गोन्मर्दनादर्श दन्तधावाभिषेचनम्‌
अन्नाद्यगीतनृत्यानि पर्वणि स्युरुतान्वहम्‌
विधिना विहिते कुण्डे मेखलागर्तवेदिभिः
अग्निमाधाय परितः समूहेत्पाणिनोदितम्‌
परिस्तीर्याथ पर्युक्षेदन्वाधाय यथाविधि
प्रोक्षण्यासाद्य द्रव्याणि प्रोक्ष्याग्नौ भावयेत माम्‌
तप्तजाम्बूनदप्रख्यं शङ्खचक्रगदाम्बुजैः
लसच्चतुर्भुजं शान्तं पद्मकिञ्जल्कवाससम्‌
स्फुरत्किरीटकटक कटिसूत्रवराङ्गदम्‌
श्रीवत्सवक्षसं भ्राजत्‌ कौस्तुभं वनमालिनम्‌
ध्यायन्नभ्यर्च्य दारूणि हविषाभिघृतानि च
प्रास्याज्यभागावाघारौ दत्त्वा चाज्यप्लुतं हविः
जुहुयान्मूलमन्त्रेण षोडशर्चावदानतः
धर्मादिभ्यो यथान्यायं मन्त्रैः स्विष्टिकृतं बुधः
अभ्यर्च्याथ नमस्कृत्य पार्षदेभ्यो बलिं हरेत्‌
मूलमन्त्रं जपेद्ब्रह्म स्मरन्नारायणात्मकम्‌
दत्त्वाचमनमुच्छेषं विष्वक्षेनाय कल्पयेत्‌
मुखवासं सुरभिमत्ताम्बूलाद्यमथार्हयेत्‌
उपगायन्गृणन्नृत्यन्कर्माण्यभिनयन्मम
मत्कथाः श्रावयन्शृण्वन्मुहूर्तं क्षणिको भवेत्‌
स्तवैरुच्चावचैः स्तोत्रैः पौराणैः प्राकृतैरपि
स्तुत्वा प्रसीद भगवन्निति वन्देत दण्डवत्‌
शिरो मत्पादयोः कृत्वा बाहुभ्यां च परस्परम्‌
प्रपन्नं पाहि मामीश भीतं मृत्युग्रहार्णवात्‌
इति शेषां मया दत्तां शिरस्याधाय सादरम्‌
उद्वासयेच्चेदुद्वास्यं ज्योतिर्ज्योतिषि तत्पुनः
अर्चादिषु यदा यत्र श्रद्धा मां तत्र चार्चयेत्‌
सर्वभूतेष्वात्मनि च सर्वात्माहमवस्थितः
एवं क्रियायोगपथैः पुमान्वैदिकतान्त्रिकैः
अर्चन्नुभयतः सिद्धिं मत्तो विन्दत्यभीप्सिताम्‌
मदर्चां सम्प्रतिष्ठाप्य मन्दिरं कारयेद्दृढम्‌
पुष्पोद्यानानि रम्याणि पूजायात्रोत्सवाश्रितान्‌
पूजादीनां प्रवाहार्थं महापर्वस्वथान्वहम्‌
क्षेत्रापणपुरग्रामान्दत्त्वा मत्सार्ष्टितामियात्‌
प्रतिष्ठया सार्वभौमं सद्मना भुवनत्रयम्‌
पूजादिना ब्रह्मलोकं त्रिभिर्मत्साम्यतामियात्‌
मामेव नैरपेक्ष्येण भक्तियोगेन विन्दति
भक्तियोगं स लभत एवं यः पूजयेत माम्‌
यः स्वदत्तां परैर्दत्तां हरेत सुरविप्रयोः
वृत्तिं स जायते विड्भुग्वर्षाणामयुतायुतम्‌
कर्तुश्च सारथेर्हेतोरनुमोदितुरेव च
कर्मणां भागिनः प्रेत्य भूयो भूयसि तत्फलम्‌
श्रीभगवानुवाच
परस्वभावकर्माणि न प्रशंसेन्न गर्हयेत्‌
विश्वमेकामकं पश्यन्प्रकृत्या पुरुषेण च
परस्वभावकर्माणि यः प्रशंसति निन्दति
स आशु भ्रश्यते स्वार्थादसत्यभिनिवेशतः
तैजसे निद्रयापन्ने पिण्डस्थो नष्टचेतनः
मायां प्राप्नोति मृत्युं वा तद्वन्नानार्थदृक्पुमान्‌
किं भद्रं किमभद्रं वा द्वैतस्यावस्तुनः कियत्‌
वाचोदितं तदनृतं मनसा ध्यातमेव च
छायाप्रत्याह्वयाभासा ह्यसन्तोऽप्यर्थकारिणः
एवं देहादयो भावा यच्छन्त्यामृत्युतो भयम्‌
आत्मैव तदिदं विश्वं सृज्यते सृजति प्रभुः
त्रायते त्राति विश्वात्मा ह्रियते हरतीश्वरः
तस्मान्न ह्यात्मनोऽन्यस्मादन्यो भावो निरूपितः
निरूपितेऽयं त्रिविधा निर्मूल भातिरात्मनि
इदं गुणमयं विद्धि त्रिविधं मायया कृतम्‌
एतद्विद्वान्मदुदितं ज्ञानविज्ञाननैपुणम्‌
न निन्दति न च स्तौति लोके चरति सूर्यवत्‌
प्रत्यक्षेणानुमानेन निगमेनात्मसंविदा
आद्यन्तवदसज्ज्ञात्वा निःसङ्गो विचरेदिह
श्रीउद्धव उवाच
नैवात्मनो न देहस्य संसृतिर्द्रष्टृदृश्ययोः
अनात्मस्वदृशोरीश कस्य स्यादुपलभ्यते
आत्माव्ययोऽगुणः शुद्धः स्वयंज्योतिरनावृतः
अग्निवद्दारुवदचिद्देहः कस्येह संसृतिः
श्रीभगवानुवाच
यावद्देहेन्द्रियप्राणैरात्मनः सन्निकर्षणम्‌
संसारः फलवांस्तावदपार्थोऽप्यविवेकिनः
अर्थे ह्यविद्यमानेऽपि संसृतिर्न निवर्तते
ध्यायतो विषयानस्य स्वप्नेऽनर्थागमो यथा
यथा ह्यप्रतिबुद्धस्य प्रस्वापो बह्वनर्थभृत्‌
स एव प्रतिबुद्धस्य न वै मोहाय कल्पते
शोकहर्षभयक्रोध लोभमोहस्पृहादयः
अहङ्कारस्य दृश्यन्ते जन्ममृत्युश्च नात्मनः
देहेन्द्रियप्राणमनोऽभिमानो जीवोऽन्तरात्मा गुणकर्ममूर्तिः
सूत्रं महानित्युरुधेव गीतः संसार आधावति कालतन्त्रः
अमूलमेतद्बहुरूपरूपितं मनोवचःप्राणशरीरकर्म
ज्ञानासिनोपासनया शितेन च्छित्त्वा मुनिर्गां विचरत्यतृष्णः
ज्ञानं विवेको निगमस्तपश्च प्रत्यक्षमैतिह्यमथानुमानम्‌
आद्यन्तयोरस्य यदेव केवलं कालश्च हेतुश्च तदेव मध्ये
यथा हिरण्यं स्वकृतं पुरस्तात्पश्चाच्च सर्वस्य हिरण्मयस्य
तदेव मध्ये व्यवहार्यमाणं नानापदेशैरहमस्य तद्वत्‌
विज्ञानमेतत्त्रियवस्थमङ्ग गुणत्रयं कारणकर्यकर्तृ
समन्वयेन व्यतिरेकतश्च येनैव तुर्येण तदेव सत्यम्‌
न यत्पुरस्तादुत यन्न पश्चान्मध्ये च तन्न व्यपदेशमात्रम्‌
भूतं प्रसिद्धं च परेण यद्यत्तदेव तत्स्यादिति मे मनीषा
अविद्यमानोऽप्यवभासते यो वैकारिको राजससर्ग एसः
ब्रह्म स्वयं ज्योतिरतो विभाति ब्रह्मेन्द्रियार्थात्मविकारचित्रम्‌
एवं स्फुतं ब्रह्मविवेकहेतुभिः
परापवादेन विशारदेन
छित्त्वात्मसन्देहमुपारमेत
स्वानन्दतुष्टोऽखिलकामुकेभ्यः
नात्मा वपुः पार्थिवमिन्द्रियाणि देवा ह्यसुर्वायुर्जलम्हुताशः
मनोऽन्नमात्रं धिषणा च सत्त्वमहङ्कृतिः खं क्षितिरर्थसाम्यम्‌
समाहितैः कः करणैर्गुणात्मभिर्‌
गुणो भवेन्मत्सुविविक्तधाम्नः
विक्षिप्यमाणैरुत किं नु दूषणं
घनैरुपेतैर्विगतै रवेः किम्‌
यथा नभो वाय्वनलाम्बुभूगुणैर्‌
गतागतैर्वर्तुगुणैर्न सज्जते
तथाक्षरं सत्त्वरजस्तमोमलैर्‌
अहंमतेः संसृतिहेतुभिः परम्‌
तथापि सङ्गः परिवर्जनीयो गुणेषु मायारचितेषु तावत्‌
मद्भक्तियोगेन दृढेन यावद्रजो निरस्येत मनःकषायः
यथामयोऽसाधु चिकित्सितो नृणां पुनः पुनः सन्तुदति प्ररोहन्‌
एवं मनोऽपक्वकषायकर्म कुयोगिनं विध्यति सर्वसङ्गम्‌
कुयोगिनो ये विहितान्तरायैर्मनुष्यभूतैस्त्रिदशोपसृष्टैः
ते प्राक्तनाभ्यासबलेन भूयो युञ्जन्ति योगं न तु कर्मतन्त्रम्‌
करोति कर्म क्रियते च जन्तुः केनाप्यसौ चोदित आनिपतात्‌
न तत्र विद्वान्प्रकृतौ स्थितोऽपि निवृत्ततृष्णः स्वसुखानुभूत्या
तिष्ठन्तमासीनमुत व्रजन्तं शयानमुक्षन्तमदन्तमन्नम्‌
स्वभावमन्यत्किमपीहमानमात्मानमात्मस्थमतिर्न वेद
यदि स्म पश्यत्यसदिन्द्रियार्थं नानानुमानेन विरुद्धमन्यत्‌
न मन्यते वस्तुतया मनीषी स्वाप्नं यथोत्थाय तिरोदधानम्‌
पूर्वं गृहीतं गुणकर्मचित्रमज्ञानमात्मन्यविविक्तमङ्ग
निवर्तते तत्पुनरीक्षयैव न गृह्यते नापि विसृय्य आत्मा
यथा हि भानोरुदयो नृचक्षुषां तमो निहन्यान्न तु सद्विधत्ते
एवं समीक्षा निपुणा सती मे हन्यात्तमिस्रं पुरुषस्य बुद्धेः
एष स्वयंज्योतिरजोऽप्रमेयो महानुभूतिः सकलानुभूतिः
एकोऽद्वितीयो वचसां विरामे येनेषिता वागसवश्चरन्ति
एतावानात्मसम्मोहो यद्विकल्पस्तु केवले
आत्मनृते स्वमात्मानमवलम्बो न यस्य हि
यन्नामाकृतिभिर्ग्राह्यं पञ्चवर्णमबाधितम्‌
व्यर्थेनाप्यर्थवादोऽयं द्वयं पण्डितमानिनाम्‌
योगिनोऽपक्वयोगस्य युञ्जतः काय उत्थितैः
उपसर्गैर्विहन्येत तत्रायं विहितो विधिः
योगधारणया कांश्चिदासनैर्धारणान्वितैः
तपोमन्त्रौषधैः कांश्चिदुपसर्गान्विनिर्दहेत्‌
कांश्चिन्ममानुध्यानेन नामसङ्कीर्तनादिभिः
योगेश्वरानुवृत्त्या वा हन्यादशुभदान्शनैः
केचिद्देहमिमं धीराः सुकल्पं वयसि स्थिरम्‌
विधाय विविधोपायैरथ युञ्जन्ति सिद्धये
न हि तत्कुशलादृत्यं तदायासो ह्यपार्थकः
अन्तवत्त्वाच्छरीरस्य फलस्येव वनस्पतेः
योगं निषेवतो नित्यं कायश्चेत्कल्पतामियात्‌
तच्छ्रद्दध्यान्न मतिमान्योगमुत्सृज्य मत्परः
योगचर्यामिमां योगी विचरन्मदपाश्रयः
नान्तरायैर्विहन्येत निःस्पृहः स्वसुखानुभूः
श्रीउद्धव उवाच
सुदुस्तरामिमां मन्ये योगचर्यामनात्मनः
यथाञ्जसा पुमान्सिद्ध्येत्तन्मे ब्रूह्यञ्जसाच्युत
प्रायशः पुण्दरीकाक्ष युञ्यन्तो योगिनो मनः
विषीदन्त्यसमाधानान्मनोनिग्रहकर्शिताः
अथात आनन्ददुघं पदाम्बुजं हंसाः श्रयेरन्नरविन्दलोचन
सुखं नु विश्वेश्वर योगकर्मभिस्त्वन्माययामी विहता न मानिनः
किं चित्रमच्युत तवैतदशेषबन्धो दासेष्वनन्यशरणेसु यदात्मसात्त्वम्‌
योऽरोचयत्सह मृगैः स्वयमीश्वराणां श्रीमत्किरीटतटपीडितपादपीठः
तं त्वाखिलात्मदयितेश्वरमाश्रितानां
सर्वार्थदं स्वकृतविद्विसृजेत को नु
को वा भजेत्किमपि विस्मृतयेऽनु भूत्यै
किं वा भवेन्न तव पादरजोजुषां नः
नैवोपयन्त्यपचितिं कवयस्तवेश
ब्रह्मायुषापि कृतमृद्धमुदः स्मरन्तः
योऽन्तर्बहिस्तनुभृतामशुभं विधुन्वन्न्‌
आचार्यचैत्त्यवपुषा स्वगतिं व्यनक्ति
श्रीशुक उवाच
इत्युद्धवेनात्यनुरक्तचेतसा पृष्टो जगत्क्रीडनकः स्वशक्तिभिः
गृहीतमूर्तित्रय ईश्वरेश्वरो जगाद सप्रेममनोहरस्मितः
श्रीभगवानुवाच
हन्त ते कथयिष्यामि मम धर्मान्सुमङ्गलान्‌
यान्श्रद्धयाचरन्मर्त्यो मृत्युं जयति दुर्जयम्‌
कुर्यात्सर्वाणि कर्माणि मदर्थं शनकैः स्मरन्‌
मय्यर्पितमनश्चित्तो मद्धर्मात्ममनोरतिः
देशान्पुण्यानाश्रयेत मद्भक्तैः साधुभिः श्रितान्‌
देवासुरमनुष्येषु मद्भक्ताचरितानि च
पृथक्सत्रेण वा मह्यं पर्वयात्रामहोत्सवान्‌
कारयेद्गीतनृत्याद्यैर्महाराजविभूतिभिः
मामेव सर्वभूतेषु बहिरन्तरपावृतम्‌
ईक्षेतात्मनि चात्मानं यथा खममलाशयः
इति सर्वाणि भूतानि मद्भावेन महाद्युते
सभाजयन्मन्यमानो ज्ञानं केवलमाश्रितः
ब्राह्मणे पुक्कसे स्तेने ब्रह्मण्येऽर्के स्फुलिङ्गके
अक्रूरे क्रूरके चैव समदृक्पण्डितो मतः
नरेष्वभीक्ष्णं मद्भावं पुंसो भावयतोऽचिरात्‌
स्पर्धासूयातिरस्काराः साहङ्कारा वियन्ति हि
विसृज्य स्मयमानान्स्वान्दृशं व्रीडां च दैहिकीम्‌
प्रणमेद्दण्डवद्भूमावाश्वचाण्डालगोखरम्‌
यावत्सर्वेषु भूतेषु मद्भावो नोपजायते
तावदेवमुपासीत वाङ्मनःकायवृत्तिभिः
सर्वं ब्रह्मात्मकं तस्य विद्ययात्ममनीषया
परिपश्यन्नुपरमेत्सर्वतो मुइतसंशयः
अयं हि सर्वकल्पानां सध्रीचीनो मतो मम
मद्भावः सर्वभूतेषु मनोवाक्कायवृत्तिभिः
न ह्यङ्गोपक्रमे ध्वंसो मद्धर्मस्योद्धवाण्वपि
मया व्यवसितः सम्यङ्निर्गुणत्वादनाशिषः
यो यो मयि परे धर्मः कल्प्यते निष्फलाय चेत्‌
तदायासो निरर्थः स्याद्भयादेरिव सत्तम
एषा बुद्धिमतां बुद्धिर्मनीषा च मनीषिणाम्‌
यत्सत्यमनृतेनेह मर्त्येनाप्नोति मामृतम्‌
एष तेऽभिहितः कृत्स्नो ब्रह्मवादस्य सङ्ग्रहः
समासव्यासविधिना देवानामपि दुर्गमः
अभीक्ष्णशस्ते गदितं ज्ञानं विस्पष्टयुक्तिमत्‌
एतद्विज्ञाय मुच्येत पुरुषो नष्टसंशयः
सुविविक्तं तव प्रश्नं मयैतदपि धारयेत्‌
सनातनं ब्रह्मगुह्यं परं ब्रह्माधिगच्छति
य एतन्मम भक्तेषु सम्प्रदद्यात्सुपुष्कलम्‌
तस्याहं ब्रह्मदायस्य ददाम्यात्मानमात्मना
य एतत्समधीयीत पवित्रं परमं शुचि
स पूयेताहरहर्मां ज्ञानदीपेन दर्शयन्‌
य एतच्छ्रद्धया नित्यमव्यग्रः शृणुयान्नरः
मयि भक्तिं परां कुर्वन्कर्मभिर्न स बध्यते
अप्युद्धव त्वया ब्रह्म सखे समवधारितम्‌
अपि ते विगतो मोहः शोकश्चासौ मनोभवः
नैतत्त्वया दाम्भिकाय नास्तिकाय शठाय च
अशुश्रूषोरभक्ताय दुर्विनीताय दीयताम्‌
एतैर्दोषैर्विहीनाय ब्रह्मण्याय प्रियाय च
साधवे शुचये ब्रूयाद्भक्तिः स्याच्छूद्रयोषिताम्‌
नैतद्विज्ञाय जिज्ञासोर्ज्ञातव्यमवशिष्यते
पीत्वा पीयूषममृतं पातव्यं नावशिष्यते
ज्ञाने कर्मणि योगे च वार्तायां दण्डधारणे
यावानर्थो नृणां तात तावांस्तेऽहं चतुर्विधः
मर्त्यो यदा त्यक्तसमस्तकर्मा निवेदितात्मा विचिकीर्षितो मे
तदामृतत्वं प्रतिपद्यमानो मयात्मभूयाय च कल्पते वै
श्रीशुक उवाच
स एवमादर्शितयोगमार्गस्तदोत्तमःश्लोकवचो निशम्य
बद्धाञ्जलिः प्रीत्युपरुद्धकण्ठो न किञ्चिदूचेऽश्रुपरिप्लुताक्षः
विष्टभ्य चित्तं प्रणयावघूर्णं धैर्येण राजन्बहुमन्यमानः
कृताञ्जलिः प्राह यदुप्रवीरं शीर्ष्णा स्पृशंस्तच्चरणारविन्दम्‌
श्रीउद्धव उवाच
विद्रावितो मोहमहान्धकारो य आश्रितो मे तव सन्निधानात्‌
विभावसोः किं नु समीपगस्य शीतं तमो भीः प्रभवन्त्यजाद्य
प्रत्यर्पितो मे भवतानुकम्पिना भृत्याय विज्ञानमयः प्रदीपः
हित्वा कृतज्ञस्तव पादमूलं कोऽन्यं समीयाच्छरणं त्वदीयम्‌
वृक्णश्च मे सुदृढः स्नेहपाशो दाशार्हवृष्ण्यन्धकसात्वतेषु
प्रसारितः सृष्टिविवृद्धये त्वया स्वमायया ह्यात्मसुबोधहेतिना
नमोऽस्तु ते महायोगिन्प्रपन्नमनुशाधि माम्‌
यथा त्वच्चरणाम्भोजे रतिः स्यादनपायिनी
श्रीभगवानुवाच
गच्छोद्धव मयादिष्टो बदर्याख्यं ममाश्रमम्‌
तत्र मत्पादतीर्थोदे स्नानोपस्पर्शनैः शुचिः
ईक्षयालकनन्दाया विधूताशेषकल्मषः
वसानो वल्कलान्यङ्ग वन्यभुक्सुखनिःस्पृहः
तितिक्षुर्द्वन्द्वमात्राणां सुशीलः संयतेन्द्रियः
शान्तः समाहितधिया ज्ञानविज्ञानसंयुतः
मत्तोऽनुशिक्षितं यत्ते विविक्तमनुभावयन्‌
मय्यावेशितवाक्चित्तो मद्धर्मनिरतो भव
अतिव्रज्य गतीस्तिस्रो मामेष्यसि ततः परम्‌
श्रीशुक उवाच
स एवमुक्तो हरिमेधसोद्धवः प्रदक्षिणं तं परिसृत्य पादयोः
शिरो निधायाश्रुकलाभिरार्द्रधीर्न्यषिञ्चदद्वन्द्वपरोऽप्यपक्रमे
सुदुस्त्यजस्नेहवियोगकातरो न शक्नुवंस्तं परिहातुमातुरः
कृच्छ्रं ययौ मूर्धनि भर्तृपादुके बिभ्रन्नमस्कृत्य ययौ पुनः पुनः
ततस्तमन्तर्हृदि सन्निवेश्य गतो महाभागवतो विशालाम्‌
यथोपदिष्टां जगदेकबन्धुना तपः समास्थाय हरेरगाद्गतिम्‌
य एतदानन्दसमुद्रसम्भृतं ज्ञानामृतं भागवताय भाषितम्‌
कृष्णेन योगेश्वरसेविताङ्घ्रिणा सच्छ्रद्धयासेव्य जगद्विमुच्यते
भवभयमपहन्तुं ज्ञानविज्ञानसारं
निगमकृदुपजह्रे भृङ्गवद्वेदसारम्‌
अमृतमुदधितश्चापाययद्भृत्यवर्गान्‌
पुरुषमृषभमाद्यं कृष्णसंज्ञं नतोऽस्मि
श्रीराजोवाच
ततो महाभागवत उद्धवे निर्गते वनम्‌
द्वारवत्यां किमकरोद्भगवान्भूतभावनः
ब्रह्मशापोपसंसृष्टे स्वकुले यादवर्षभः
प्रेयसीं सर्वनेत्राणां तनुं स कथमत्यजत्‌
प्रत्याक्रष्टुं नयनमबला यत्र लग्नं न शेकुः
कर्णाविष्टं न सरति ततो यत्सतामात्मलग्नम्‌
यच्छ्रीर्वाचां जनयति रतिं किं नु मानं कवीनां
दृष्ट्वा जिष्णोर्युधि रथगतं यच्च तत्साम्यमीयुः
श्री ऋषिरुवाच
दिवि भुव्यन्तरिक्षे च महोत्पातान्समुत्थितान्‌
दृष्ट्वासीनान्सुधर्मायां कृष्णः प्राह यदूनिदम्‌
श्रीभगवानुवाच
एते घोरा महोत्पाता द्वार्वत्यां यमकेतवः
मुहूर्तमपि न स्थेयमत्र नो यदुपुङ्गवाः
स्त्रियो बालाश्च वृद्धाश्च शङ्खोद्धारं व्रजन्त्वितः
वयं प्रभासं यास्यामो यत्र प्रत्यक्सरस्वती
तत्राभिषिच्य शुचय उपोष्य सुसमाहिताः
देवताः पूजयिष्यामः स्नपनालेपनार्हणैः
ब्राह्मणांस्तु महाभागान्कृतस्वस्त्ययना वयम्‌
गोभूहिरण्यवासोभिर्गजाश्वरथवेश्मभिः
विधिरेष ह्यरिष्टघ्नो मङ्गलायनमुत्तमम्‌
देवद्विजगवां पूजा भूतेषु परमो भवः
इति सर्वे समाकर्ण्य यदुवृद्धा मधुद्विषः
तथेति नौभिरुत्तीर्य प्रभासं प्रययू रथैः
तस्मिन्भगवतादिष्टं यदुदेवेन यादवाः
चक्रुः परमया भक्त्या सर्वश्रेयोपबृंहितम्‌
ततस्तस्मिन्महापानं पपुर्मैरेयकं मधु
दिष्टविभ्रंशितधियो यद्द्रवैर्भ्रश्यते मतिः
महापानाभिमत्तानां वीराणां दृप्तचेतसाम्‌
कृष्णमायाविमूढानां सङ्घर्षः सुमहानभूत्‌
युयुधुः क्रोधसंरब्धा वेलायामाततायिनः
धनुर्भिरसिभिर्भल्लैर्गदाभिस्तोमरर्ष्टिभिः
पतत्पताकै रथकुञ्जरादिभिः खरोष्ट्रगोभिर्महिषैर्नरैरपि
मिथः समेत्याश्वतरैः सुदुर्मदा न्यहन्शरैर्दद्भिरिव द्विपा वने
प्रद्युम्नसाम्बौ युधि रूढमत्सराव्‌
अक्रूरभोजावनिरुद्धसात्यकी
सुभद्रसङ्ग्रामजितौ सुदारुणौ
गदौ सुमित्रासुरथौ समीयतुः
अन्ये च ये वै निशठोल्मुकादयः सहस्रजिच्छतजिद्भानुमुख्याः
अन्योन्यमासाद्य मदान्धकारिता जघ्नुर्मुकुन्देन विमोहिता भृशम्‌
दाशार्हवृष्ण्यन्धकभोजसात्वता
मध्वर्बुदा माथुरशूरसेनाः
विसर्जनाः कुकुराः कुन्तयश्च
मिथस्तु जघ्नुः सुविसृज्य सौहृदम्‌
पुत्रा अयुध्यन्पितृभिर्भ्रातृभिश्च
स्वस्रीयदौहित्रपितृव्यमातुलैः
मित्राणि मित्रैः सुहृदः सुहृद्भिर्‌
ज्ञातींस्त्वहन्ज्ञातय एव मूढाः
शरेषु हीयमाएषु भज्यमानेसु धन्वसु
शस्त्रेषु क्षीयमानेषु मुष्टिभिर्जह्रुरेरकाः
ता वज्रकल्पा ह्यभवन्परिघा मुष्टिना भृताः
जघ्नुर्द्विषस्तैः कृष्णेन वार्यमाणास्तु तं च ते
प्रत्यनीकं मन्यमाना बलभद्रं च मोहिताः
हन्तुं कृतधियो राजन्नापन्ना आततायिनः
अथ तावपि सङ्क्रुद्धावुद्यम्य कुरुनन्दन
एरकामुष्टिपरिघौ चरन्तौ जघ्नतुर्युधि
ब्रह्मशापोपसृष्टानां कृष्णमायावृतात्मनाम्‌
स्पर्धाक्रोधः क्षयं निन्ये वैणवोऽग्निर्यथा वनम्‌
एवं नष्टेषु सर्वेषु कुलेषु स्वेषु केशवः
अवतारितो भुवो भार इति मेनेऽवशेषितः
रामः समुद्रवेलायां योगमास्थाय पौरुषम्‌
तत्याज लोकं मानुष्यं संयोज्यात्मानमात्मनि
रामनिर्याणमालोक्य भगवान्देवकीसुतः
निषसाद धरोपस्थे तुष्णीमासाद्य पिप्पलम्‌
बिभ्रच्चतुर्भुजं रूपं भ्रायिष्णु प्रभया स्वया
दिशो वितिमिराः कुर्वन्विधूम इव पावकः
श्रीवत्साङ्कं घनश्यामं तप्तहाटकवर्चसम्‌
कौशेयाम्बरयुग्मेन परिवीतं सुमङ्गलम्‌
सुन्दरस्मितवक्त्राब्जं नीलकुन्तलमण्डितम्‌
पुण्डरीकाभिरामाक्षं स्फुरन्मकरकुण्डलम्‌
कटिसूत्रब्रह्मसूत्र किरीटकटकाङ्गदैः
हारनूपुरमुद्राभिः कौस्तुभेन विराजितम्‌
वनमालापरीताङ्गं मूर्तिमद्भिर्निजायुधैः
कृत्वोरौ दक्षिणे पादमासीनं पङ्कजारुणम्‌
मुषलावशेषायःखण्ड कृतेषुर्लुब्धको जरा
मृगास्याकारं तच्चरणं विव्याध मृगशङ्कया
चतुर्भुजं तं पुरुषं दृष्ट्वा स कृतकिल्बिषः
भीतः पपात शिरसा पादयोरसुरद्विषः
अजानता कृतमिदं पापेन मधुसूदन
क्षन्तुमर्हसि पापस्य उत्तमःश्लोक मेऽनघ
यस्यानुस्मरणं नृणामज्ञानध्वान्तनाशनम्‌
वदन्ति तस्य ते विष्णो मयासाधु कृतं प्रभो
तन्माशु जहि वैकुण्ठ पाप्मानं मृगलुब्धकम्‌
यथा पुनरहं त्वेवं न कुर्यां सदतिक्रमम्‌
यस्यात्मयोगरचितं न विदुर्विरिञ्चो
रुद्रादयोऽस्य तनयाः पतयो गिरां ये
त्वन्मायया पिहितदृष्टय एतदञ्जः
किं तस्य ते वयमसद्गतयो गृणीमः
श्रीभगवानुवाच
मा भैर्जरे त्वमुत्तिष्ठ काम एष कृतो हि मे
याहि त्वं मदनुज्ञातः स्वर्गं सुकृतिनां पदम्‌
इत्यादिष्टो भगवता कृष्णेनेच्छाशरीरिणा
त्रिः परिक्रम्य तं नत्वा विमानेन दिवं ययौ
दारुकः कृष्णपदवीमन्विच्छन्नधिगम्य ताम्‌
वायुं तुलसिकामोदमाघ्रायाभिमुखं ययौ
तं तत्र तिग्मद्युभिरायुधैर्वृतं
ह्यश्वत्थमूले कृतकेतनं पतिम्‌
स्नेहप्लुतात्मा निपपात पादयो
रथादवप्लुत्य सबाष्पलोचनः
अपश्यतस्त्वच्चरणाम्बुजं प्रभो दृष्टिः प्रणष्टा तमसि प्रविष्टा
दिशो न जाने न लभे च शान्तिं यथा निशायामुडुपे प्रणष्टे
इति ब्रुवति सूते वै रथो गरुडलाञ्छनः
खमुत्पपात राजेन्द्र साश्वध्वज उदीक्षतः
तमन्वगच्छन्दिव्यानि विष्णुप्रहरणानि च
तेनातिविस्मितात्मानं सूतमाह जनार्दनः
गच्छ द्वारवतीं सूत ज्ञातीनां निधनं मिथः
सङ्कर्षणस्य निर्याणं बन्धुभ्यो ब्रूहि मद्दशाम्‌
द्वारकायां च न स्थेयं भवद्भिश्च स्वबन्धुभिः
मया त्यक्तां यदुपुरीं समुद्रः प्लावयिष्यति
स्वं स्वं परिग्रहं सर्वे आदाय पितरौ च नः
अर्जुनेनाविताः सर्व इन्द्रप्रस्थं गमिष्यथ
त्वं तु मद्धर्ममास्थाय ज्ञाननिष्ठ उपेक्षकः
मन्मायारचितामेतां विज्ञयोपशमं व्रज
इत्युक्तस्तं परिक्रम्य नमस्कृत्य पुनः पुनः
तत्पादौ शीर्ष्ण्युपाधाय दुर्मनाः प्रययौ पुरीम्‌
श्रीशुक उवाच
अथ तत्रागमद्ब्रह्मा भवान्या च समं भवः
महेन्द्रप्रमुखा देवा मुनयः सप्रजेश्वराः
पितरः सिद्धगन्धर्वा विद्याधरमहोरगाः
चारणा यक्षरक्षांसि किन्नराप्सरसो द्विजाः
द्रष्टुकामा भगवतो निर्याणं परमोत्सुकाः
गायन्तश्च गृणन्तश्च शौरेः कर्माणि जन्म च
ववृषुः पुष्पवर्षाणि विमानावलिभिर्नभः
कुर्वन्तः सङ्कुलं राजन्भक्त्या परमया युताः
भगवान्पितामहं वीक्ष्य विभूतीरात्मनो विभुः
संयोज्यात्मनि चात्मानं पद्मनेत्रे न्यमीलयत्‌
लोकाभिरामां स्वतनुं धारणाध्यानमङ्गलम्‌
योगधारणयाग्नेय्या दग्ध्वा धामाविशत्स्वकम्‌
दिवि दुन्दुभयो नेदुः पेतुः सुमनसश्च खात्‌
सत्यं धर्मो धृतिर्भूमेः कीर्तिः श्रीश्चानु तं ययुः
देवादयो ब्रह्ममुख्या न विशन्तं स्वधामनि
अविज्ञातगतिं कृष्णं ददृशुश्चातिविस्मिताः
सौदामन्या यथाक्लाशे यान्त्या हित्वाभ्रमण्डलम्‌
गतिर्न लक्ष्यते मर्त्यैस्तथा कृष्णस्य दैवतैः
ब्रह्मरुद्रादयस्ते तु दृष्ट्वा योगगतिं हरेः
विस्मितास्तां प्रशंसन्तः स्वं स्वं लोकं ययुस्तदा
राजन्परस्य तनुभृज्जननाप्ययेहा
मायाविडम्बनमवेहि यथा नटस्य
सृष्ट्वात्मनेदमनुविश्य विहृत्य चान्ते
संहृत्य चात्ममहिनोपरतः स आस्ते
मर्त्येन यो गुरुसुतं यमलोकनीतं
त्वां चानयच्छरणदः परमास्त्रदग्धम्‌
जिग्येऽन्तकान्तकमपीशमसावनीशः
किं स्वावने स्वरनयन्मृगयुं सदेहम्‌
तथाप्यशेषस्थितिसम्भवाप्ययेष्व्‌
अनन्यहेतुर्यदशेषशक्तिधृक्‌
नैच्छत्प्रणेतुं वपुरत्र शेषितं
मर्त्येन किं स्वस्थगतिं प्रदर्शयन्‌
य एतां प्रातरुत्थाय कृष्णस्य पदवीं पराम्‌
प्रयतः कीर्तयेद्भक्त्या तामेवाप्नोत्यनुत्तमाम्‌
दारुको द्वारकामेत्य वसुदेवोग्रसेनयोः
पतित्वा चरणावस्रैर्न्यषिञ्चत्कृष्णविच्युतः
कथयामास निधनं वृष्णीनां कृत्स्नशो नृप
तच्छ्रुत्वोद्विग्नहृदया जनाः शोकविर्मूर्च्छिताः
तत्र स्म त्वरिता जग्मुः कृष्णविश्लेषविह्वलाः
व्यसवः शेरते यत्र ज्ञातयो घ्नन्त आननम्‌
देवकी रोहिणी चैव वसुदेवस्तथा सुतौ
कृष्णरामावपश्यन्तः शोकार्ता विजहुः स्मृतिम्‌
प्राणांश्च विजहुस्तत्र भगवद्विरहातुराः
उपगुह्य पतींस्तात चितामारुरुहुः स्त्रियः
रामपत्न्यश्च तद्देहमुपगुह्याग्निमाविशन्‌
वसुदेवपत्न्यस्तद्गात्रं प्रद्युम्नादीन्हरेः स्नुषाः
कृष्णपत्न्योऽविशन्नग्निं रुक्मिण्याद्यास्तदात्मिकाः
अर्जुनः प्रेयसः सख्युः कृष्णस्य विरहातुरः
आत्मानं सान्त्वयामास कृष्णगीतैः सदुक्तिभिः
बन्धूनां नष्टगोत्राणामर्जुनः साम्परायिकम्‌
हतानां कारयामास यथावदनुपूर्वशः
द्वारकां हरिणा त्यक्तां समुद्रोऽप्लावयत्क्षणात्‌
वर्जयित्वा महाराज श्रीमद्भगवदालयम्‌
नित्यं सन्निहितस्तत्र भगवान्मधुसूदनः
स्मृत्याशेषाशुभहरं सर्वमङ्गलमङ्गलम्‌
स्त्रीबालवृद्धानादाय हतशेषान्धनञ्जयः
इन्द्रप्रस्थं समावेश्य वज्रं तत्राभ्यषेचयत्‌
श्रुत्वा सुहृद्वधं राजन्नर्जुनात्ते पितामहाः
त्वां तु वंशधरं कृत्वा जग्मुः सर्वे महापथम्‌
य एतद्देवदेवस्य विष्णोः कर्माणि जन्म च
कीर्तयेच्छ्रद्धया मर्त्यः सर्वपापैः प्रमुच्यते
इत्थं हरेर्भगवतो रुचिरावतार
वीर्याणि बालचरितानि च शन्तमानि
अन्यत्र चेह च श्रुतानि गृणन्मनुष्यो
भक्तिं परां परमहंसगतौ लभेत
श्रीशुक उवाच
योऽन्त्यः पुरञ्जयो नाम भविष्यो बारहद्रथः
तस्यामात्यस्तु शुनको हत्वा स्वामिनमात्मजम्‌
प्रद्योतसंज्ञं राजानं कर्ता यत्पालकः सुतः
विशाखयूपस्तत्पुत्रो भविता राजकस्ततः
नन्दिवर्धनस्तत्पुत्रः पञ्च प्रद्योतना इमे
अष्टत्रिंशोत्तरशतं भोक्ष्यन्ति पृथिवीं नृपाः
शिशुनागस्ततो भाव्यः काकवर्णस्तु तत्सुतः
क्षेमधर्मा तस्य सुतः क्षेत्रज्ञः क्षेमधर्मजः
विधिसारः सुतस्तस्या जातशत्रुर्भविष्यति
दर्भकस्तत्सुतो भावी दर्भकस्याजयः स्मृतः
नन्दिवर्धन आजेयो महानन्दिः सुतस्ततः
शिशुनागा दशैवैते सष्ट्युत्तरशतत्रयम्‌
समा भोक्ष्यन्ति पृथिवीं कुरुश्रेष्ठ कलौ नृपाः
महानन्दिसुतो राजन्शूद्रागर्भोद्भवो बली
महापद्मपतिः कश्चिन्नन्दः क्षत्रविनाशकृत्‌
ततो नृपा भविष्यन्ति शूद्रप्रायास्त्वधार्मिकाः
स एकच्छत्रां पृथिवीमनुल्लङ्घितशासनः
शासिष्यति महापद्मो द्वितीय इव भार्गवः
तस्य चाष्टौ भविष्यन्ति सुमाल्यप्रमुखाः सुताः
य इमां भोक्ष्यन्ति महीं राजानश्च शतं समाः
नव नन्दान्द्विजः कश्चित्प्रपन्नानुद्धरिष्यति
तेषां अभावे जगतीं मौर्या भोक्ष्यन्ति वै कलौ
स एव चन्द्रगुप्तं वै द्विजो राज्येऽभिषेक्ष्यति
तत्सुतो वारिसारस्तु ततश्चाशोकवर्धनः
सुयशा भविता तस्य सङ्गतः सुयशःसुतः
शालिशूकस्ततस्तस्य सोमशर्मा भविष्यति
शतधन्वा ततस्तस्य भविता तद्बृहद्रथः
मौर्या ह्येते दश नृपाः सप्तत्रिंशच्छतोत्तरम्‌
समा भोक्ष्यन्ति पृथिवीं कलौ कुरुकुलोद्वह
अग्निमित्रस्ततस्तस्मात्सुज्येष्ठो भविता ततः
वसुमित्रो भद्रकश्च पुलिन्दो भविता सुतः
ततो घोषः सुतस्तस्माद्वज्रमित्रो भविष्यति
ततो भागवतस्तस्माद्देवभूतिः कुरूद्वह
शुङ्गा दशैते भोक्ष्यन्ति भूमिं वर्षशताधिकम्‌
ततः काण्वानियं भूमिर्यास्यत्यल्पगुणान्नृप
शुङ्गं हत्वा देवभूतिं काण्वोऽमात्यस्तु कामिनम्‌
स्वयं करिष्यते राज्यं वसुदेवो महामतिः
तस्य पुत्रस्तु भूमित्रस्तस्य नारायणः सुतः
काण्वायना इमे भूमिं चत्वारिंशच्च पञ्च च
शतानि त्रीणि भोक्ष्यन्ति वर्षाणां च कलौ युगे
हत्वा काण्वं सुशर्माणं तद्भृत्यो वृषलो बली
गां भोक्ष्यत्यन्ध्रजातीयः कञ्चित्कालमसत्तमः
कृष्णनामाथ तद्भ्राता भविता पृथिवीपतिः
श्रीशान्तकर्णस्तत्पुत्रः पौर्णमासस्तु तत्सुतः
लम्बोदरस्तु तत्पुत्रस्तस्माच्चिबिलको नृपः
मेघस्वातिश्चिबिलकादटमानस्तु तस्य च
अनिष्टकर्मा हालेयस्तलकस्तस्य चात्मजः
पुरीषभीरुस्तत्पुत्रस्ततो राजा सुनन्दनः
चकोरो बहवो यत्र शिवस्वातिररिन्दमः
तस्यापि गोमती पुत्रः पुरीमान्भविता ततः
मेदशिराः शिवस्कन्दो यज्ञश्रीस्तत्सुतस्ततः
विजयस्तत्सुतो भाव्यश्चन्द्रविज्ञः सलोमधिः
एते त्रिंशन्नृपतयश्चत्वार्यब्दशतानि च
षट्पञ्चाशच्च पृथिवीं भोक्ष्यन्ति कुरुनन्दन
सप्ताभीरा आवभृत्या दश गर्दभिनो नृपाः
कङ्काः षोडश भूपाला भविष्यन्त्यतिलोलुपाः
ततोऽष्टौ यवना भाव्याश्चतुर्दश तुरुष्ककाः
भूयो दश गुरुण्डाश्च मौला एकादशैव तु
एते भोक्ष्यन्ति पृथिवीं दश वर्षशतानि च
नवाधिकां च नवतिं मौला एकादश क्षितिम्‌
भोक्ष्यन्त्यब्दशतान्यङ्ग त्रीणि तैः संस्थिते ततः
किलकिलायां नृपतयो भूतनन्दोऽथ वङ्गिरिः
शिशुनन्दिश्च तद्भ्राता यशोनन्दिः प्रवीरकः
इत्येते वै वर्षशतं भविष्यन्त्यधिकानि षट्‌
तेषां त्रयोदश सुता भवितारश्च बाह्लिकाः
पुष्पमित्रोऽथ राजन्यो दुर्मित्रोऽस्य तथैव च
एककाला इमे भूपाः सप्तान्ध्राः सप्त कौशलाः
विदूरपतयो भाव्या निषधास्तत एव हि
मागधानां तु भविता विश्वस्फूर्जिः पुरञ्जयः
करिष्यत्यपरो वर्णान्पुलिन्दयदुमद्रकान्‌
प्रजाश्चाब्रह्मभूयिष्ठाः स्थापयिष्यति दुर्मतिः
वीर्यवान्क्षत्रमुत्साद्य पद्मवत्यां स वै पुरि
अनुगङ्गमाप्रयागं गुप्तां भोक्ष्यति मेदिनीम्‌
सौराष्ट्रावन्त्याभीराश्च शूरा अर्बुदमालवाः
व्रात्या द्विजा भविष्यन्ति शूद्रप्राया जनाधिपाः
सिन्धोस्तटं चन्द्रभागां कौन्तीं काश्मीरमण्डलम्‌
भोक्ष्यन्ति शूद्रा व्रात्याद्या म्लेच्छाश्चाब्रह्मवर्चसः
तुल्यकाला इमे राजन्म्लेच्छप्रायाश्च भूभृतः
एतेऽधर्मानृतपराः फल्गुदास्तीव्रमन्यवः
स्त्रीबालगोद्विजघ्नाश्च परदारधनादृताः
उदितास्तमितप्राया अल्पसत्त्वाल्पकायुषः
असंस्कृताः क्रियाहीना रजसा तमसावृताः
प्रजास्ते भक्षयिष्यन्ति म्लेच्छा राजन्यरूपिणः
तन्नाथास्ते जनपदास्तच्छीलाचारवादिनः
अन्योन्यतो राजभिश्च क्षयं यास्यन्ति पीडिताः
श्रीशुक उवाच
ततश्चानुदिनं धर्मः सत्यं शौचं क्षमा दया
कालेन बलिना राजन्नङ्क्ष्यत्यायुर्बलं स्मृतिः
वित्तमेव कलौ न्ऱ्‌^ईणां जन्माचारगुणोदयः
धर्मन्यायव्यवस्थायां कारणं बलमेव हि
दाम्पत्येऽभिरुचिर्हेतुर्मायैव व्यावहारिके
स्त्रीत्वे पुंस्त्वे च हि रतिर्विप्रत्वे सूत्रमेव हि
लिङ्गं एवाश्रमख्यातावन्योन्यापत्तिकारणम्‌
अवृत्त्या न्यायदौर्बल्यं पाण्डित्ये चापलं वचः
अनाढ्यतैवासाधुत्वे साधुत्वे दम्भ एव तु
स्वीकार एव चोद्वाहे स्नानमेव प्रसाधनम्‌
दूरे वार्ययनं तीर्थं लावण्यं केशधारणम्‌
उदरंभरता स्वार्थः सत्यत्वे धार्ष्ट्यमेव हि
दाक्ष्यं कुटुम्बभरणं यशोऽर्थे धर्मसेवनम्‌
एवं प्रजाभिर्दुष्टाभिराकीर्णे क्षितिमण्डले
ब्रह्मविट्क्षत्रशूद्राणां यो बली भविता नृपः
प्रजा हि लुब्धै राजन्यैर्निर्घृणैर्दस्युधर्मभिः
आच्छिन्नदारद्रविणा यास्यन्ति गिरिकाननम्‌
शाकमूलामिषक्षौद्र फलपुष्पाष्टिभोजनाः
अनावृष्ट्या विनङ्क्ष्यन्ति दुर्भिक्षकरपीडिताः
शीतवातातपप्रावृड्‌ हिमैरन्योन्यतः प्रजाः
क्षुत्तृड्भ्यां व्याधिभिश्चैव सन्तप्स्यन्ते च चिन्तया
त्रिंशद्विंशति वर्षाणि
परमायुः कलौ नृणाम्‌
क्षीयमाणेषु देहेषु देहिनां कलिदोषतः
वर्णाश्रमवतां धर्मे नष्टे वेदपथे नृणाम्‌
पाषण्डप्रचुरे धर्मे दस्युप्रायेषु राजसु
चौर्यानृतवृथाहिंसा नानावृत्तिषु वै नृषु
शूद्रप्रायेषु वर्णेषु च्छागप्रायासु धेनुषु
गृहप्रायेष्वाश्रमेषु यौनप्रायेषु बन्धुषु
अणुप्रायास्वोषधीषु शमीप्रायेषु स्थास्नुषु
विद्युत्प्रायेषु मेघेषु शून्यप्रायेषु सद्मसु
इत्थं कलौ गतप्राये जनेषु खरधर्मिषु
धर्मत्राणाय सत्त्वेन भगवानवतरिष्यति
चराचरगुरोर्विष्णोरीश्वरस्याखिलात्मनः
धर्मत्राणाय साधूनां जन्म कर्मापनुत्तये
शम्भलग्राममुख्यस्य ब्राह्मणस्य महात्मनः
भवने विष्णुयशसः कल्किः प्रादुर्भविष्यति
अश्वमाशुगमारुह्य देवदत्तं जगत्पतिः
असिनासाधुदमनमष्टैश्वर्यगुणान्वितः
विचरन्नाशुना क्षौण्यां हयेनाप्रतिमद्युतिः
नृपलिङ्गच्छदो दस्यून्कोटिशो निहनिष्यति
अथ तेषां भविष्यन्ति मनांसि विशदानि वै
वासुदेवाङ्गरागाति पुण्यगन्धानिलस्पृशाम्‌
पौरजानपदानां वै हतेष्वखिलदस्युषु
तेषां प्रजाविसर्गश्च स्थविष्ठः सम्भविष्यति
वासुदेवे भगवति सत्त्वमूर्तौ हृदि स्थिते
यदावतीर्णो भगवान्कल्किर्धर्मपतिर्हरिः
कृतं भविष्यति तदा प्रजासूतिश्च सात्त्विकी
यदा चन्द्रश्च सूर्यश्च तथा तिष्यबृहस्पती
एकराशौ समेष्यन्ति भविष्यति तदा कृतम्‌
येऽतीता वर्तमाना ये भविष्यन्ति च पार्थिवाः
ते त उद्देशतः प्रोक्ता वंशीयाः सोमसूर्ययोः
आरभ्य भवतो जन्म यावन्नन्दाभिषेचनम्‌
एतद्वर्षसहस्रं तु शतं पञ्चदशोत्तरम्‌
सप्तर्षीणां तु यौ पूर्वौ दृश्येते उदितौ दिवि
तयोस्तु मध्ये नक्षत्रं दृश्यते यत्समं निशि
तेनैव ऋषयो युक्तास्तिष्ठन्त्यब्दशतं नृणाम्‌
ते त्वदीये द्विजाः काल अधुना चाश्रिता मघाः
विष्णोर्भगवतो भानुः कृष्णाख्योऽसौ दिवं गतः
तदाविशत्कलिर्लोकं पापे यद्रमते जनः
यावत्स पादपद्माभ्यां स्पृशनास्ते रमापतिः
तावत्कलिर्वै पृथिवीं पराक्रन्तुं न चाशकत्‌
यदा देवर्षयः सप्त मघासु विचरन्ति हि
तदा प्रवृत्तस्तु कलिर्द्वादशाब्दशतात्मकः
यदा मघाभ्यो यास्यन्ति पूर्वाषाढां महर्षयः
तदा नन्दात्प्रभृत्येष कलिर्वृद्धिं गमिष्यति
यस्मिन्कृष्णो दिवं यातस्तस्मिन्नेव तदाहनि
प्रतिपन्नं कलियुगमिति प्राहुः पुराविदः
दिव्याब्दानां सहस्रान्ते चतुर्थे तु पुनः कृतम्‌
भविष्यति तदा न्ऱ्‌^ईणां मन आत्मप्रकाशकम्‌
इत्येष मानवो वंशो यथा सङ्ख्यायते भुवि
तथा विट्शूद्रविप्राणां तास्ता ज्ञेया युगे युगे
एतेषां नामलिङ्गानां पुरुषाणां महात्मनाम्‌
कथामात्रावशिष्टानां कीर्तिरेव स्थिता भुवि
देवापिः शान्तनोर्भ्राता मरुश्चेक्ष्वाकुवंशजः
कलापग्राम आसाते महायोगबलान्वितौ
ताविहैत्य कलेरन्ते वासुदेवानुशिक्षितौ
वर्णाश्रमयुतं धर्मं पूर्ववत्प्रथयिष्यतः
कृतं त्रेता द्वापरं च कलिश्चेति चतुर्युगम्‌
अनेन क्रमयोगेन भुवि प्राणिषु वर्तते
राजन्नेते मया प्रोक्ता नरदेवास्तथापरे
भूमौ ममत्वं कृत्वान्ते हित्वेमां निधनं गताः
कृमिविड्भस्मसंज्ञान्ते राजनाम्नोऽपि यस्य च
भूतध्रुक्तत्कृते स्वार्थं किं वेद निरयो यतः
कथं सेयमखण्डा भूः पूर्वैर्मे पुरुषैर्धृता
मत्पुत्रस्य च पौत्रस्य मत्पूर्वा वंशजस्य वा
तेजोऽबन्नमयं कायं गृहीत्वात्मतयाबुधाः
महीं ममतया चोभौ हित्वान्तेऽदर्शनं गताः
ये ये भूपतयो राजन्भुञ्जते भुवमोजसा
कालेन ते कृताः सर्वे कथामात्राः कथासु च
श्रीशुक उवाच
दृष्ट्वात्मनि जये व्यग्रान्नृपान्हसति भूरियम्‌
अहो मा विजिगीषन्ति मृत्योः क्रीडनका नृपाः
काम एष नरेन्द्राणां मोघः स्याद्विदुषामपि
येन फेनोपमे पिण्डे येऽतिविश्रम्भिता नृपाः
पूर्वं निर्जित्य षड्वर्गं जेष्यामो राजमन्त्रिणः
ततः सचिवपौराप्त करीन्द्रानस्य कण्टकान्‌
एवं क्रमेण जेष्यामः पृथ्वीं सागरमेखलाम्‌
इत्याशाबद्धहृदया न पश्यन्त्यन्तिकेऽन्तकम्‌
समुद्रावरणां जित्वा मां विशन्त्यब्धिमोजसा
कियदात्मजयस्यैतन्मुक्तिरात्मजये फलम्‌
यां विसृज्यैव मनवस्तत्सुताश्च कुरूद्वह
गता यथागतं युद्धे तां मां जेष्यन्त्यबुद्धयः
मत्कृते पितृपुत्राणां भ्रातृणां चापि विग्रहः
जायते ह्यसतां राज्ये ममताबद्धचेतसाम्‌
ममैवेयं मही कृत्स्ना न ते मूढेति वादिनः
स्पर्धमाना मिथो घ्नन्ति म्रियन्ते मत्कृते नृपाः
पृथुः पुरूरवा गाधिर्नहुषो भरतोऽर्जुनः
मान्धाता सगरो रामः खट्वाङ्गो धुन्धुहा रघुः
तृणबिन्दुर्ययातिश्च शर्यातिः शन्तनुर्गयः
भगीरथः कुवलयाश्वः ककुत्स्थो नैषधो नृगः
हिरण्यकशिपुर्वृत्रो रावणो लोकरावणः
नमुचिः शम्बरो भौमो हिरण्याक्षोऽथ तारकः
अन्ये च बहवो दैत्या राजानो ये महेश्वराः
सर्वे सर्वविदः शूराः सर्वे सर्वजितोऽजिताः
ममतां मय्यवर्तन्त कृत्वोच्चैर्मर्त्यधर्मिणः
कथावशेषाः कालेन ह्यकृतार्थाः कृता विभो
कथा इमास्ते कथिता महीयसां विताय लोकेषु यशः परेयुषाम्‌
विज्ञानवैराग्यविवक्षया विभो वचोविभूतीर्न तु पारमार्थ्यम्‌
यस्तूत्तमःश्लोकगुणानुवादः सङ्गीयतेऽभीक्ष्णममङ्गलघ्नः
तमेव नित्यं शृणुयादभीक्ष्णं कृष्णेऽमलां भक्तिमभीप्समानः
श्रीराजोवाच
केनोपायेन भगवन्कलेर्दोषान्कलौ जनाः
विधमिष्यन्त्युपचितांस्तन्मे ब्रूहि यथा मुने
युगानि युगधर्मांश्च मानं प्रलयकल्पयोः
कालस्येश्वररूपस्य गतिं विष्णोर्महात्मनः
श्रीशुक उवाच
कृते प्रवर्तते धर्मश्चतुष्पात्तज्जनैर्धृतः
सत्यं दया तपो दानमिति पादा विभोर्नृप
सन्तुष्टाः करुणा मैत्राः शान्ता दान्तास्तितिक्षवः
आत्मारामाः समदृशः प्रायशः श्रमणा जनाः
त्रेतायां धर्मपादानां तुर्यांशो हीयते शनैः
अधर्मपादैरनृत हिंषासन्तोषविग्रहैः
तदा क्रियातपोनिष्ठा नातिहिंस्रा न लम्पटाः
त्रैवर्गिकास्त्रयीवृद्धा वर्णा ब्रह्मोत्तरा नृप
तपःसत्यदयादानेष्वर्धं ह्रस्वति द्वापरे
हिंसातुष्ट्यनृतद्वेषैर्धर्मस्याधर्मलक्षणैः
यशस्विनो महाशीलाः स्वाध्यायाध्ययने रताः
आध्याः कुटुम्बिनो हृष्टा वर्णाः क्षत्रद्विजोत्तराः
कलौ तु धर्मपादानां तुर्यांशोऽधर्महेतुभिः
एधमानैः क्षीयमाणो ह्यन्ते सोऽपि विनङ्क्ष्यति
तस्मिन्लुब्धा दुराचारा निर्दयाः शुष्कवैरिणः
दुर्भगा भूरितर्षाश्च शूद्रदासोत्तराः प्रजाः
सत्त्वं रजस्तम इति दृश्यन्ते पुरुषे गुणाः
कालसञ्चोदितास्ते वै परिवर्तन्त आत्मनि
प्रभवन्ति यदा सत्त्वे मनोबुद्धीन्द्रियाणि च
तदा कृतयुगं विद्याज्ज्ञाने तपसि यद्रुचिः
यदा कर्मसु काम्येषु भक्तिर्यशसि देहिनाम्‌
तदा त्रेता रजोवृत्तिरिति जानीहि बुद्धिमन्‌
यदा लोभस्त्वसन्तोषो मानो दम्भोऽथ मत्सरः
कर्मणां चापि काम्यानां द्वापरं तद्रजस्तमः
यदा मायानृतं तन्द्रा निद्रा हिंसा विषादनम्‌
शोकमोहौ भयं दैन्यं स कलिस्तामसः स्मृतः
तस्मात्क्षुद्रदृशो मर्त्याः क्षुद्रभाग्या महाशनाः
कामिनो वित्तहीनाश्च स्वैरिण्यश्च स्त्रियोऽसतीः
दस्यूत्कृष्टा जनपदा वेदाः पाषण्डदूषिताः
राजानश्च प्रजाभक्षाः शिश्नोदरपरा द्विजाः
अव्रता बटवोऽशौचा भिक्षवश्च कुटुम्बिनः
तपस्विनो ग्रामवासा न्यासिनोऽत्यर्थलोलुपाः
ह्रस्वकाया महाहारा भूर्यपत्या गतह्रियः
शश्वत्कटुकभाषिण्यश्चौर्यमायोरुसाहसाः
पणयिष्यन्ति वै क्षुद्राः किराटाः कूटकारिणः
अनापद्यपि मंस्यन्ते वार्तां साधु जुगुप्सिताम्‌
पतिं त्यक्ष्यन्ति निर्द्रव्यं भृत्या अप्यखिलोत्तमम्‌
भृत्यं विपन्नं पतयः कौलं गाश्चापयस्विनीः
पितृभ्रातृसुहृज्ज्ञातीन्हित्वा सौरतसौहृदाः
ननान्दृश्यालसंवादा दीनाः स्त्रैणाः कलौ नराः
शूद्राः प्रतिग्रहीष्यन्ति तपोवेषोपजीविनः
धर्मं वक्ष्यन्त्यधर्मज्ञा अधिरुह्योत्तमासनम्‌
नित्यं उद्विग्नमनसो दुर्भिक्षकरकर्शिताः
निरन्ने भूतले राजननावृष्टिभयातुराः
वासोऽन्नपानशयन व्यवायस्नानभूषणैः
हीनाः पिशाचसन्दर्शा भविष्यन्ति कलौ प्रजाः
कलौ काकिणिकेऽप्यर्थे विगृह्य त्यक्तसौहृदाः
त्यक्ष्यन्ति च प्रियान्प्राणान्हनिष्यन्ति स्वकानपि
न रक्षिष्यन्ति मनुजाः स्थविरौ पितरावपि
पुत्रान्भार्यां च कुलजां क्षुद्राः शिश्नोदरंभराः
कलौ न राजन्जगतां परं गुरुं त्रिलोकनाथानतपादपङ्कजम्‌
प्रायेण मर्त्या भगवन्तमच्युतं यक्ष्यन्ति पाषण्डविभिन्नचेतसः
यन्नामधेयं म्रियमाण आतुरः पतन्स्खलन्वा विवशो गृणन्पुमान्‌
विमुक्तकर्मार्गल उत्तमां गतिं प्राप्नोति यक्ष्यन्ति न तं कलौ जनाः
पुंसां कलिकृतान्दोषान्द्रव्यदेशात्मसम्भवान्‌
सर्वान्हरति चित्तस्थो भगवान्पुरुषोत्तमः
श्रुतः सङ्कीर्तितो ध्यातः पूजितश्चादृतोऽपि वा
नृणां धुनोति भगवान्हृत्स्थो जन्मायुताशुभम्‌
यथा हेम्नि स्थितो वह्निर्दुर्वर्णं हन्ति धातुजम्‌
एवमात्मगतो विष्णुर्योगिनामशुभाशयम्‌
विद्यातपःप्राणनिरोधमैत्री तीर्थाभिषेकव्रतदानजप्यैः
नात्यन्तशुद्धिं लभतेऽन्तरात्मा यथा हृदिस्थे भगवत्यनन्ते
तस्मात्सर्वात्मना राजन्हृदिस्थं कुरु केशवम्‌
म्रियमाणो ह्यवहितस्ततो यासि परां गतिम्‌
म्रियमाणैरभिध्येयो भगवान्परमेश्वरः
आत्मभावं नयत्यङ्ग सर्वात्मा सर्वसंश्रयः
कलेर्दोषनिधे राजन्नस्ति ह्येको महान्गुणः
कीर्तनादेव कृष्णस्य मुक्तसङ्गः परं व्रजेत्‌
कृते यद्ध्यायतो विष्णुं त्रेतायां यजतो मखैः
द्वापरे परिचर्यायां कलौ तद्धरिकीर्तनात्‌
श्रीशुक उवाच
कालस्ते परमाण्वादिर्द्विपरार्धावधिर्नृप
कथितो युगमानं च शृणु कल्पलयावपि
चतुर्युगसहस्रं तु ब्रह्मणो दिनमुच्यते
स कल्पो यत्र मनवश्चतुर्दश विशाम्पते
तदन्ते प्रलयस्तावान्ब्राह्मी रात्रिरुदाहृता
त्रयो लोका इमे तत्र कल्पन्ते प्रलयाय हि
एष नैमित्तिकः प्रोक्तः प्रलयो यत्र विश्वसृक्‌
शेतेऽनन्तासनो विश्वमात्मसात्कृत्य चात्मभूः
द्विपरार्धे त्वतिक्रान्ते ब्रह्मणः परमेष्ठिनः
तदा प्रकृतयः सप्त कल्पन्ते प्रलयाय वै
एष प्राकृतिको राजन्प्रलयो यत्र लीयते
अण्डकोषस्तु सङ्घातो विघाट उपसादिते
पर्जन्यः शतवर्षाणि भूमौ राजन्न वर्षति
तदा निरन्ने ह्यन्योन्यं भक्ष्यमाणाः क्षुधार्दिताः
क्षयं यास्यन्ति शनकैः कालेनोपद्रुताः प्रजाः
सामुद्रं दैहिकं भौमं रसं सांवर्तको रविः
रश्मिभिः पिबते घोरैः सर्वं नैव विमुञ्चति
ततः संवर्तको वह्निः सङ्कर्षणमुखोत्थितः
दहत्यनिलवेगोत्थः शून्यान्भूविवरानथ
उपर्यधः समन्ताच्च शिखाभिर्वह्निसूर्ययोः
दह्यमानं विभात्यण्डं दग्धगोमयपिण्डवत्‌
ततः प्रचण्डपवनो वर्षाणामधिकं शतम्‌
परः सांवर्तको वाति धूम्रं खं रजसावृतम्‌
ततो मेघकुलान्यङ्ग चित्र वर्णान्यनेकशः
शतं वर्षाणि वर्षन्ति नदन्ति रभसस्वनैः
तत एकोदकं विश्वं
ब्रह्माण्डविवरान्तरम्‌
तदा भूमेर्गन्धगुणं ग्रसन्त्याप उदप्लवे
ग्रस्तगन्धा तु पृथिवी प्रलयत्वाय कल्पते
अपां रसमथो तेजस्ता लीयन्तेऽथ नीरसाः
ग्रसते तेजसो रूपं वायुस्तद्रहितं तदा
लीयते चानिले तेजो वायोः खं ग्रसते गुणम्‌
स वै विशति खं राजंस्ततश्च नभसो गुणम्‌
शब्दं ग्रसति भूतादिर्नभस्तमनु लीयते
तैजसश्चेन्द्रियाण्यङ्ग देवान्वैकारिको गुणैः
महान्ग्रसत्यहङ्कारं गुणाः सत्त्वादयश्च तम्‌
ग्रसतेऽव्याकृतं राजन्गुणान्कालेन चोदितम्‌
न तस्य कालावयवैः परिणामादयो गुणाः
अनाद्यनन्तमव्यक्तं नित्यं कारणमव्ययम्‌
न यत्र वाचो न मनो न सत्त्वं तमो रजो वा महदादयोऽमी
न प्राणबुद्धीन्द्रियदेवता वा न सन्निवेशः खलु लोककल्पः
न स्वप्नजाग्रन्न च तत्सुषुप्तं न खं जलं भूरनिलोऽग्निरर्कः
संसुप्तवच्छून्यवदप्रतर्क्यं तन्मूलभूतं पदमामनन्ति
लयः प्राकृतिको ह्येष पुरुषाव्यक्तयोर्यदा
शक्तयः सम्प्रलीयन्ते विवशाः कालविद्रुताः
बुद्धीन्द्रियार्थरूपेण ज्ञानं भाति तदाश्रयम्‌
दृश्यत्वाव्यतिरेकाभ्यामाद्यन्तवदवस्तु यत्‌
दीपश्चक्षुश्च रूपं च ज्योतिषो न पृथग्भवेत्‌
एवं धीः खानि मात्राश्च न स्युरन्यतमादृतात्‌
बुद्धेर्जागरणं स्वप्नः सुषुप्तिरिति चोच्यते
मायामात्रमिदं राजन्नानात्वं प्रत्यगात्मनि
यथा जलधरा व्योम्नि भवन्ति न भवन्ति च
ब्रह्मणीदं तथा विश्वमवयव्युदयाप्ययात्‌
सत्यं ह्यवयवः प्रोक्तः सर्वावयविनामिह
विनार्थेन प्रतीयेरन्पटस्येवाङ्ग तन्तवः
यत्सामान्यविशेषाभ्यामुपलभ्येत स भ्रमः
अन्योन्यापाश्रयात्सर्वमाद्यन्तवदवस्तु यत्‌
विकारः ख्यायमानोऽपि प्रत्यगात्मानमन्तरा
न निरूप्योऽस्त्यणुरपि स्याच्चेच्चित्सम आत्मवत्‌
न हि सत्यस्य नानात्वमविद्वान्यदि मन्यते
नानात्वं छिद्रयोर्यद्वज्ज्योतिषोर्वातयोरिव
यथा हिरण्यं बहुधा समीयते नृभिः क्रियाभिर्व्यवहारवर्त्मसु
एवं वचोभिर्भगवानधोक्षजो व्याख्यायते लौकिकवैदिकैर्जनैः
यथा घनोऽर्कप्रभवोऽर्कदर्शितो
ह्यर्कांशभूतस्य च चक्षुषस्तमः
एवं त्वहं ब्रह्मगुणस्तदीक्षितो
ब्रह्मांशकस्यात्मन आत्मबन्धनः
घनो यदार्कप्रभवो विदीर्यते चक्षुः स्वरूपं रविमीक्षते तदा
यदा ह्यहङ्कार उपाधिरात्मनो जिज्ञासया नश्यति तर्ह्यनुस्मरेत्‌
यदैवमेतेन विवेकहेतिना मायामयाहङ्करणात्मबन्धनम्‌
छित्त्वाच्युतात्मानुभवोऽवतिष्ठते तमाहुरात्यन्तिकमङ्ग सम्प्लवम्‌
नित्यदा सर्वभूतानां ब्रह्मादीनां परन्तप
उत्पत्तिप्रलयावेके सूक्ष्मज्ञाः सम्प्रचक्षते
कालस्रोतोजवेनाशु ह्रियमाणस्य नित्यदा
परिणामिनां अवस्थास्ता जन्मप्रलयहेतवः
अनाद्यन्तवतानेन कालेनेश्वरमूर्तिना
अवस्था नैव दृश्यन्ते वियति ज्योतिषां इव
नित्यो नैमित्तिकश्चैव तथा प्राकृतिको लयः
आत्यन्तिकश्च कथितः कालस्य गतिरीदृशी
एताः कुरुश्रेष्ठ जगद्विधातुर्नारायणस्याखिलसत्त्वधाम्नः
लीलाकथास्ते कथिताः समासतः कार्त्स्न्येन नाजोऽप्यभिधातुमीशः
संसारसिन्धुमतिदुस्तरमुत्तितीर्षोर्‌
नान्यः प्लवो भगवतः पुरुषोत्तमस्य
लीलाकथारसनिषेवणमन्तरेण
पुंसो भवेद्विविधदुःखदवार्दितस्य
पुराणसंहितामेतामृषिर्नारायणोऽव्ययः
नारदाय पुरा प्राह कृष्णद्वैपायनाय सः
स वै मह्यं महाराज भगवान्बादरायणः
इमां भागवतीं प्रीतः संहितां वेदसम्मिताम्‌
इमां वक्ष्यत्यसौ सूत ऋषिभ्यो नैमिषालये
दीर्घसत्रे कुरुश्रेष्ठ सम्पृष्टः शौनकादिभिः
श्रीशुक उवाच
अत्रानुवर्ण्यतेऽभीक्ष्णं विश्वात्मा भगवान्हरिः
यस्य प्रसादजो ब्रह्मा रुद्रः क्रोधसमुद्भवः
त्वं तु राजन्मरिष्येति पशुबुद्धिमिमां जहि
न जातः प्रागभूतोऽद्य देहवत्त्वं न नङ्क्ष्यसि
न भविष्यसि भूत्वा त्वं पुत्रपौत्रादिरूपवान्‌
बीजाङ्कुरवद्देहादेर्व्यतिरिक्तो यथानलः
स्वप्ने यथा शिरश्छेदं पञ्चत्वाद्यात्मनः स्वयम्‌
यस्मात्पश्यति देहस्य तत आत्मा ह्यजोऽमरः
घटे भिन्ने घटाकाश आकाशः स्याद्यथा पुरा
एवं देहे मृते जीवो ब्रह्म सम्पद्यते पुनः
मनः सृजति वै देहान्गुणान्कर्माणि चात्मनः
तन्मनः सृजते माया ततो जीवस्य संसृतिः
स्नेहाधिष्ठानवर्त्यग्नि संयोगो यावदीयते
तावद्दीपस्य दीपत्वमेवं देहकृतो भवः
रजःसत्त्वतमोवृत्त्या जायतेऽथ विनश्यति
न तत्रात्मा स्वयंज्योतिर्यो व्यक्ताव्यक्तयोः परः
आकाश इव चाधारो ध्रुवोऽनन्तोपमस्ततः
एवमात्मानमात्मस्थमात्मनैवामृश प्रभो
बुद्ध्यानुमानगर्भिण्या वासुदेवानुचिन्तया
चोदितो विप्रवाक्येन न त्वां धक्ष्यति तक्षकः
मृत्यवो नोपधक्ष्यन्ति मृत्यूनां मृत्युमीश्वरम्‌
अहं ब्रह्म परं धाम ब्रह्माहं परमं पदम्‌
एवं समीक्ष्य चात्मानमात्मन्याधाय निष्कले
दशन्तं तक्षकं पादे लेलिहानं विषाननैः
न द्रक्ष्यसि शरीरं च विश्वं च पृथगात्मनः
एतत्ते कथितं तात यदात्मा पृष्टवान्नृप
हरेर्विश्वात्मनश्चेष्टां किं भूयः श्रोतुमिच्छसि
सूत उवाच
एतन्निशम्य मुनिनाभिहितं परीक्षिद्‌
व्यासात्मजेन निखिलात्मदृशा समेन
तत्पादमूलमुपसृत्य नतेन मूर्ध्ना
बद्धाञ्जलिस्तमिदमाह स विष्णुरातः
राजोवाच
सिद्धोऽस्म्यनुगृहीतोऽस्मि भवता करुणात्मना
श्रावितो यच्च मे साक्षादनादिनिधनो हरिः
नात्यद्भुतमहं मन्ये महतामच्युतात्मनाम्‌
अज्ञेषु तापतप्तेषु भूतेषु यदनुग्रहः
पुराणसंहितामेतामश्रौष्म भवतो वयम्‌
यस्यां खलूत्तमःश्लोको भगवाननवर्ण्यते
भगवंस्तक्षकादिभ्यो मृत्युभ्यो न बिभेम्यहम्‌
प्रविष्टो ब्रह्म निर्वाणमभयं दर्शितं त्वया
अनुजानीहि मां ब्रह्मन्वाचं यच्छाम्यधोक्षजे
मुक्तकामाशयं चेतः प्रवेश्य विसृजाम्यसून्‌
अज्ञानं च निरस्तं मे ज्ञानविज्ञाननिष्ठया
भवता दर्शितं क्षेमं परं भगवतः पदम्‌
सूत उवाच
इत्युक्तस्तमनुज्ञाप्य भगवान्बादरायणिः
जगाम भिक्षुभिः साकं नरदेवेन पूजितः
परीक्षिदपि राजर्षिरात्मन्यात्मानमात्मना
समाधाय परं दध्यावस्पन्दासुर्यथा तरुः
प्राक्कूले बर्हिष्यासीनो गङ्गाकूल उदङ्मुखः
ब्रह्मभूतो महायोगी निःसङ्गश्छिन्नसंशयः
तक्षकः प्रहितो विप्राः क्रुद्धेन द्विजसूनुना
हन्तुकामो नृपं गच्छन्ददर्श पथि कश्यपम्‌
तं तर्पयित्वा द्रविणैर्निवर्त्य विषहारिणम्‌
द्विजरूपप्रतिच्छन्नः कामरूपोऽदशन्नृपम्‌
ब्रह्मभूतस्य राजर्षेर्देहोऽहिगरलाग्निना
बभूव भस्मसात्सद्यः पश्यतां सर्वदेहिनाम्‌
हाहाकारो महानासीद्भुवि खे दिक्षु सर्वतः
विस्मिता ह्यभवन्सर्वे देवासुरनरादयः
देवदुन्दुभयो नेदुर्गन्धर्वाप्सरसो जगुः
ववृषुः पुष्पवर्षाणि विबुधाः साधुवादिनः
जन्मेजयः स्वपितरं श्रुत्वा तक्षकभक्षितम्‌
यथाजुहाव सन्क्रुद्धो नागान्सत्रे सह द्विजैः
सर्पसत्रे समिद्धाग्नौ दह्यमानान्महोरगान्‌
दृष्ट्वेन्द्रं भयसंविग्नस्तक्षकः शरणं ययौ
अपश्यंस्तक्षकं तत्र राजा पारीक्षितो द्विजान्‌
उवाच तक्षकः कस्मान्न दह्येतोरगाधमः
तं गोपायति राजेन्द्र शक्रः शरणमागतम्‌
तेन संस्तम्भितः सर्पस्तस्मान्नाग्नौ पतत्यसौ
पारीक्षित इति श्रुत्वा प्राहर्त्विज उदारधीः
सहेन्द्रस्तक्षको विप्रा नाग्नौ किमिति पात्यते
तच्छ्रुत्वाजुहुवुर्विप्राः सहेन्द्रं तक्षकं मखे
तक्षकाशु पतस्वेह सहेन्द्रेण मरुत्वता
इति ब्रह्मोदिताक्षेपैः स्थानादिन्द्रः प्रचालितः
बभूव सम्भ्रान्तमतिः सविमानः सतक्षकः
तं पतन्तं विमानेन सहतक्षकमम्बरात्‌
विलोक्याङ्गिरसः प्राह राजानं तं बृहस्पतिः
नैष त्वया मनुष्येन्द्र वधमर्हति सर्पराट्‌
अनेन पीतममृतमथ वा अजरामरः
जीवितं मरणं जन्तोर्गतिः स्वेनैव कर्मणा
राजंस्ततोऽन्यो नास्त्यस्य प्रदाता सुखदुःखयोः
सर्पचौराग्निविद्युद्भ्यः क्षुत्तृद्व्याध्यादिभिर्नृप
पञ्चत्वमृच्छते जन्तुर्भुङ्क्त आरब्धकर्म तत्‌
तस्मात्सत्रमिदं राजन्संस्थीयेताभिचारिकम्‌
सर्पा अनागसो दग्धा जनैर्दिष्टं हि भुज्यते
सूत उवाच
इत्युक्तः स तथेत्याह महर्षेर्मानयन्वचः
सर्पसत्रादुपरतः पूजयामास वाक्पतिम्‌
सैषा विष्णोर्महामाया बाध्ययालक्षणा यया
मुह्यन्त्यस्यैवात्मभूता भूतेषु गुणवृत्तिभिः
न यत्र दम्भीत्यभया विराजिता मायात्मवादेऽसकृदात्मवादिभिः
न यद्विवादो विविधस्तदाश्रयो मनश्च सङ्कल्पविकल्पवृत्ति यत्‌
न यत्र सृज्यं सृजतोभयोः परं श्रेयश्च जीवस्त्रिभिरन्वितस्त्वहम्‌
तदेतदुत्सादितबाध्यबाधकं निषिध्य चोर्मीन्विरमेत तन्मुनिः
परं पदं वैष्णवमामनन्ति तद्यन्नेति नेतीत्यतदुत्सिसृक्षवः
विसृज्य दौरात्म्यमनन्यसौहृदा हृदोपगुह्यावसितं समाहितैः
त एतदधिगच्छन्ति विष्णोर्यत्परमं पदम्‌
अहं ममेति दौर्जन्यं न येषां देहगेहजम्‌
अतिवादांस्तितिक्षेत नावमन्येत कञ्चन
न चेमं देहमाश्रित्य वैरं कुर्वीत केनचित्‌
नमो भगवते तस्मै कृष्णायाकुण्ठमेधसे
यत्पादाम्बुरुहध्यानात्संहितामध्यगामिमाम्‌
श्रीशौनक उवाच
पैलादिभिर्व्यासशिष्यैर्वेदाचार्यैर्महात्मभिः
वेदाश्च कथिता व्यस्ता एतत्सौम्याभिधेहि नः
सूत उवाच
समाहितात्मनो ब्रह्मन्ब्रह्मणः परमेष्ठिनः
हृद्याकाशादभून्नादो वृत्तिरोधाद्विभाव्यते
यदुपासनया ब्रह्मन्योगिनो मलमात्मनः
द्रव्यक्रियाकारकाख्यं धूत्वा यान्त्यपुनर्भवम्‌
ततोऽभूत्त्रिवृदोंकारो योऽव्यक्तप्रभवः स्वराट्‌
यत्तल्लिङ्गं भगवतो ब्रह्मणः परमात्मनः
शृणोति य इमं स्फोटं सुप्तश्रोत्रे च शून्यदृक्‌
येन वाग्व्यज्यते यस्य व्यक्तिराकाश आत्मनः
स्वधाम्नो ब्राह्मणः साक्षाद्वाचकः परमात्मनः
स सर्वमन्त्रोपनिषद्वेदबीजं सनातनम्‌
तस्य ह्यासंस्त्रयो वर्णा अकाराद्या भृगूद्वह
धार्यन्ते यैस्त्रयो भावा गुणनामार्थवृत्तयः
ततोऽक्षरसमाम्नायमसृजद्भगवानजः
अन्तस्थोष्मस्वरस्पर्श ह्रस्वदीर्घादिलक्षणम्‌
तेनासौ चतुरो वेदांश्चतुर्भिर्वदनैर्विभुः
सव्याहृतिकान्सोंकारांश्चातुर्होत्रविवक्षया
पुत्रानध्यापयत्तांस्तु ब्रह्मर्षीन्ब्रह्मकोविदान्‌
ते तु धर्मोपदेष्टारः स्वपुत्रेभ्यः समादिशन्‌
ते परम्परया प्राप्तास्तत्तच्छिष्यैर्धृतव्रतैः
चतुर्युगेष्वथ व्यस्ता द्वापरादौ महर्षिभिः
क्षीणायुषः क्षीणसत्त्वान्दुर्मेधान्वीक्ष्य कालतः
वेदान्ब्रह्मर्षयो व्यस्यन्हृदिस्थाच्युतचोदिताः
अस्मिन्नप्यन्तरे ब्रह्मन्भगवान्लोकभावनः
ब्रह्मेशाद्यैर्लोकपालैर्याचितो धर्मगुप्तये
पराशरात्सत्यवत्यामंशांशकलया विभुः
अवतीर्णो महाभाग वेदं चक्रे चतुर्विधम्‌
ऋगथर्वयजुःसाम्नां राशीरुद्धृत्य वर्गशः
चतस्रः संहिताश्चक्रे मन्त्रैर्मणिगणा इव
तासां स चतुरः शिष्यानुपाहूय महामतिः
एकैकां संहितां ब्रह्मन्नेकैकस्मै ददौ विभुः
पैलाय संहितामाद्यां बह्वृचाख्यां उवाच ह
वैशम्पायनसंज्ञाय निगदाख्यं यजुर्गणम्‌
साम्नां जैमिनये प्राह तथा छन्दोगसंहिताम्‌
अथर्वाङ्गिरसीं नाम स्वशिष्याय सुमन्तवे
पैलः स्वसंहितामूचे इन्द्रप्रमितये मुनिः
बाष्कलाय च सोऽप्याह शिष्येभ्यः संहितां स्वकाम्‌
चतुर्धा व्यस्य बोध्याय याज्ञवल्क्याय भार्गव
पराशरायाग्निमित्र इन्द्रप्रमितिरात्मवान्‌
अध्यापयत्संहितां स्वां माण्डूकेयमृषिं कविम्‌
तस्य शिष्यो देवमित्रः सौभर्यादिभ्य ऊचिवान्‌
शाकल्यस्तत्सुतः स्वां तु पञ्चधा व्यस्य संहिताम्‌
वात्स्यमुद्गलशालीय गोखल्यशिशिरेष्वधात्‌
जातूकर्ण्यश्च तच्छिष्यः सनिरुक्तां स्वसंहिताम्‌
बलाकपैलजाबाल विरजेभ्यो ददौ मुनिः
बाष्कलिः प्रतिशाखाभ्यो वालखिल्याख्यसंहिताम्‌
चक्रे वालायनिर्भज्यः काशारश्चैव तां दधुः
बह्वृचाः संहिता ह्येता एभिर्ब्रह्मर्षिभिर्धृताः
श्रुत्वैतच्छन्दसां व्यासं सर्वपापैः प्रमुच्यते
वैशम्पायनशिष्या वै चरकाध्वर्यवोऽभवन्‌
यच्चेरुर्ब्रह्महत्यांहः क्षपणं स्वगुरोर्व्रतम्‌
याज्ञवल्क्यश्च तच्छिष्य आहाहो भगवन्कियत्‌
चरितेनाल्पसाराणां चरिष्येऽहं सुदुश्चरम्‌
इत्युक्तो गुरुरप्याह कुपितो याह्यलं त्वया
विप्रावमन्त्रा शिष्येण मदधीतं त्यजाश्विति
देवरातसुतः सोऽपि छर्दित्वा यजुषां गणम्‌
ततो गतोऽथ मुनयो ददृशुस्तान्यजुर्गणान्‌
यजूंषि तित्तिरा भूत्वा तल्लोलुपतयाददुः
तैत्तिरीया इति यजुः शाखा आसन्सुपेशलाः
याज्ञवल्क्यस्ततो ब्रह्मंश्छन्दांस्यधि गवेषयन्‌
गुरोरविद्यमानानि सूपतस्थेऽर्कमीश्वरम्‌
श्रीयाज्ञवल्क्य उवाच
ओं नमो भगवते आदित्यायाखिलजगतामात्मस्वरूपेण काल
स्वरूपेण चतुर्विधभूतनिकायानां ब्रह्मादिस्तम्बपर्यन्तानामन्तर्हृदयेषु
बहिरपि चाकाश इवोपाधिनाव्यवधीयमानो भवानेक
एव क्षणलवनिमेषावयवोपचितसंवत्सरगणेनापामादान
विसर्गाभ्यामिमां लोकयात्रामनुवहति
यदु ह वाव विबुधर्षभ सवितरदस्तपत्यनुसवनमहर्‌
अहराम्नायविधिनोपतिष्ठमानानामखिलदुरितवृजिन
बीजावभर्जन भगवतः समभिधीमहि तपन मण्डलम्‌
य इह वाव स्थिरचरनिकराणां निजनिकेतनानां मनैन्द्रियासु
गणाननात्मनः स्वयमात्मान्तर्यामी प्रचोदयति
य एवेमं लोकमतिकरालवदनान्धकारसंज्ञाजगरग्रह
गिलितं मृतकमिव विचेतनमवलोक्यानुकम्पया परमकारुणिक
ईक्षयैवोत्थाप्याहरहरनुसवनं श्रेयसि स्वधर्माख्यात्माव
स्थने प्रवर्तयति
अवनिपतिरिवासाधूनां भयमुदीरयन्नटति परित आशापालैस्‌
तत्र तत्र कमलकोशाञ्जलिभिरुपहृतार्हणः
अथ ह भगवंस्तव चरणनलिनयुगलं त्रिभुवनगुरुभिरभिवन्दितम्‌
अहमयातयामयजुष्काम उपसरामीति
सूत उवाच
एवं स्तुतः स भगवान्वाजिरूपधरो रविः
यजूंष्ययातयामानि मुनयेऽदात्प्रसादितः
यजुर्भिरकरोच्छाखा दश पञ्च शतैर्विभुः
जगृहुर्वाजसन्यस्ताः काण्वमाध्यन्दिनादयः
जैमिनेः समगस्यासीत्सुमन्तुस्तनयो मुनिः
सुत्वांस्तु तत्सुतस्ताभ्यामेकैकां प्राह संहिताम्‌
सुकर्मा चापि तच्छिष्यः सामवेदतरोर्महान्‌
सहस्रसंहिताभेदं चक्रे साम्नां ततो द्विज
हिरण्यनाभः कौशल्यः पौष्यञ्जिश्च सुकर्मणः
शिष्यौ जगृहतुश्चान्य आवन्त्यो ब्रह्मवित्तमः
उदीच्याः सामगाः शिष्या आसन्पञ्चशतानि वै
पौष्यञ्ज्यावन्त्ययोश्चापि तांश्च प्राच्यान्प्रचक्षते
लौगाक्षिर्माङ्गलिः कुल्यः कुशीदः कुक्षिरेव च
पौष्यञ्जिसिष्या जगृहुः संहितास्ते शतं शतम्‌
कृतो हिरण्यनाभस्य चतुर्विंशति संहिताः
शिष्य ऊचे स्वशिष्येभ्यः शेषा आवन्त्य आत्मवान्‌
सूत उवाच
अथर्ववित्सुमन्तुश्च शिष्यमध्यापयत्स्वकाम्‌
संहितां सोऽपि पथ्याय वेददर्शाय चोक्तवान्‌
शौक्लायनिर्ब्रह्मबलिर्मोदोषः पिप्पलायनिः
वेददर्शस्य शिष्यास्ते पथ्यशिष्यानथो शृणु
कुमुदः शुनको ब्रह्मन्जाजलिश्चाप्यथर्ववित्‌
बभ्रुः शिष्योऽथान्गिरसः सैन्धवायन एव च
अधीयेतां संहिते द्वे सावर्णाद्यास्तथापरे
नक्षत्रकल्पः शान्तिश्च कश्यपाङ्गिरसादयः
एते आथर्वणाचार्याः शृणु पौराणिकान्मुने
त्रय्यारुणिः कश्यपश्च सावर्णिरकृतव्रनः
वैशम्पायनहारीतौ षड्वै पौराणिका इमे
अधीयन्त व्यासशिष्यात्संहितां मत्पितुर्मुखात्‌
एकैकामहमेतेषां शिष्यः सर्वाः समध्यगाम्‌
कश्यपोऽहं च सावर्णी रामशिष्योऽकृतव्रनः
अधीमहि व्यासशिष्याच्चत्वारो मूलसंहिताः
पुराणलक्षणं ब्रह्मन्ब्रह्मर्षिभिर्निरूपितम्‌
शृणुष्व बुद्धिमाश्रित्य वेदशास्त्रानुसारतः
सर्गोऽस्याथ विसर्गश्च वृत्तिरक्षान्तराणि च
वंशो वंशानुचरीतं संस्था हेतुरपाश्रयः
दशभिर्लक्षणैर्युक्तं पुराणं तद्विदो विदुः
केचित्पञ्चविधं ब्रह्मन्महदल्पव्यवस्थया
अव्याकृतगुणक्षोभान्महतस्त्रिवृतोऽहमः
भूतसूक्ष्मेन्द्रियार्थानां सम्भवः सर्ग उच्यते
पुरुषानुगृहीतानामेतेषां वासनामयः
विसर्गोऽयं समाहारो बीजाद्बीजं चराचरम्‌
वृत्तिर्भूतानि भूतानां चराणामचराणि च
कृता स्वेन नृणां तत्र कामाच्चोदनयापि वा
रक्षाच्युतावतारेहा विश्वस्यानु युगे युगे
तिर्यङ्मर्त्यर्षिदेवेषु हन्यन्ते यैस्त्रयीद्विषः
मन्वन्तरं मनुर्देवा मनुपुत्राः सुरेश्वराः
र्षयोऽंशावताराश्च हरेः षड्विधमुच्यते
राज्ञां ब्रह्मप्रसूतानां वंशस्त्रैकालिकोऽन्वयः
वंशानुचरितं तेषाम्वृत्तं वंशधरास्च ये
नैमित्तिकः प्राकृतिको नित्य आत्यन्तिको लयः
संस्थेति कविभिः प्रोक्तश्चतुर्धास्य स्वभावतः
हेतुर्जीवोऽस्य सर्गादेरविद्याकर्मकारकः
यं चानुशायिनं प्राहुरव्याकृतमुतापरे
व्यतिरेकान्वयो यस्य जाग्रत्स्वप्नसुषुप्तिषु
मायामयेषु तद्ब्रह्म जीववृत्तिष्वपाश्रयः
पदार्थेषु यथा द्रव्यं सन्मात्रं रूपनामसु
बीजादिपञ्चतान्तासु ह्यवस्थासु युतायुतम्‌
विरमेत यदा चित्तं हित्वा वृत्तित्रयं स्वयम्‌
योगेर्ल वा तदात्मानं वेदेहाया निवर्तते
एवं लक्षणलक्ष्याणि पुराणानि पुराविदः
मुनयोऽष्टादश प्राहुः क्षुल्लकानि महान्ति च
ब्राह्मं पाद्मं वैष्णवं च शैवं लैङ्गं सगारुडं
नारदीयं भागवतमाग्नेयं स्कान्दसंज्ञितम्‌
भविष्यं ब्रह्मवैवर्तं मार्कण्डेयं सवामनम्‌
वाराहं मात्स्यं कौर्मं च ब्रह्माण्डाख्यमिति त्रिषट्‌
ब्रह्मन्निदं समाख्यातं शाखाप्रणयनं मुनेः
शिष्यशिष्यप्रशिष्याणां ब्रह्मतेजोविवर्धनम्‌
श्रीशौनक उवाच
सूत जीव चिरं साधो वद नो वदतां वर
तमस्यपारे भ्रमतां न्ऱ्‌^ईणां त्वं पारदर्शनः
आहुश्चिरायुषमृषिं मृकण्डुतनयं जनाः
यः कल्पान्ते ह्युर्वरितो येन ग्रस्तमिदं जगत्‌
स वा अस्मत्कुलोत्पन्नः कल्पेऽस्मिन्भार्गवर्षभः
नैवाधुनापि भूतानां सम्प्लवः कोऽपि जायते
एक एवार्णवे भ्राम्यन्ददर्श पुरुषं किल
वटपत्रपुटे तोकं शयानं त्वेकमद्भुतम्‌
एष नः संशयो भूयान्सूत कौतूहलं यतः
तं नश्छिन्धि महायोगिन्पुराणेष्वपि सम्मतः
सूत उवाच
प्रश्नस्त्वया महर्षेऽयं कृतो लोकभ्रमापहः
नारायणकथा यत्र गीता कलिमलापहा
प्राप्तद्विजातिसंस्कारो मार्कण्डेयः पितुः क्रमात्‌
छन्दांस्यधीत्य धर्मेण तपःस्वाध्यायसंयुतः
बृहद्व्रतधरः शान्तो जटिलो वल्कलाम्बरः
बिभ्रत्कमण्डलुं दण्डमुपवीतं समेखलम्‌
कृष्णाजिनं साक्षसूत्रं कुशांश्च नियमर्द्धये
अग्न्यर्कगुरुविप्रात्मस्वर्चयन्सन्ध्ययोर्हरिम्‌
सायं प्रातः स गुरवे भैक्ष्यमाहृत्य वाग्यतः
बुभुजे गुर्वनुज्ञातः सकृन्नो चेदुपोषितः
एवं तपःस्वाध्यायपरो वर्षाणामयुतायुतम्‌
आराधयन्हृषीकेशं जिग्ये मृत्युं सुदुर्जयम्‌
ब्रह्मा भृगुर्भवो दक्षो ब्रह्मपुत्राश्च येऽपरे
नृदेवपितृभूतानि तेनासन्नतिविस्मिताः
इत्थं बृहद्व्रतधरस्तपःस्वाध्यायसंयमैः
दध्यावधोक्षजं योगी ध्वस्तक्लेशान्तरात्मना
तस्यैवं युञ्जतश्चित्तं महायोगेन योगिनः
व्यतीयाय महान्कालो मन्वन्तरषडात्मकः
एतत्पुरन्दरो ज्ञात्वा सप्तमेऽस्मिन्किलान्तरे
तपोविशङ्कितो ब्रह्मन्नारेभे तद्विघातनम्‌
गन्धर्वाप्सरसः कामं वसन्तमलयानिलौ
मुनये प्रेषयामास रजस्तोकमदौ तथा
ते वै तदाश्रमं जग्मुर्हिमाद्रेः पार्श्व उत्तरे
पुष्पभद्रा नदी यत्र चित्राख्या च शिला विभो
तदाश्रमपदं पुण्यं पुण्यद्रुमलताञ्चितम्‌
पुण्यद्विजकुलाकीऋनं पुण्यामलजलाशयम्‌
मत्तभ्रमरसङ्गीतं मत्तकोकिलकूजितम्‌
मत्तबर्हिनटाटोपं मत्तद्विजकुलाकुलम्‌
वायुः प्रविष्ट आदाय हिमनिर्झरशीकरान्‌
सुमनोभिः परिष्वक्तो ववावुत्तम्भयन्स्मरम्‌
उद्यच्चन्द्रनिशावक्त्रः प्रवालस्तबकालिभिः
गोपद्रुमलताजालैस्तत्रासीत्कुसुमाकरः
अन्वीयमानो गन्धर्वैर्गीतवादित्रयूथकैः
अदृश्यतात्तचापेषुः स्वःस्त्रीयूथपतिः स्मरः
हुत्वाग्निं समुपासीनं ददृशुः शक्रकिङ्कराः
मीलिताक्षं दुराधर्षं मूर्तिमन्तमिवानलम्‌
ननृतुस्तस्य पुरतः स्त्रियोऽथो गायका जगुः
मृदङ्गवीणापणवैर्वाद्यं चक्रुर्मनोरमम्‌
सन्दधेऽस्त्रं स्वधनुषि कामः पञ्चमुखं तदा
मधुर्मनो रजस्तोक इन्द्रभृत्या व्यकम्पयन्‌
क्रीडन्त्याः पुञ्जिकस्थल्याः कन्दुकैः स्तनगौरवात्‌
भृशमुद्विग्नमध्यायाः केशविस्रंसितस्रजः
इतस्ततो भ्रमद्दृष्टेश्चलन्त्या अनु कन्दुकम्‌
वायुर्जहार तद्वासः सूक्ष्मं त्रुटितमेखलम्‌
विससर्ज तदा बाणं मत्वा तं स्वजितं स्मरः
सर्वं तत्राभवन्मोघमनीशस्य यथोद्यमः
त इत्थमपकुर्वन्तो मुनेस्तत्तेजसा मुने
दह्यमाना निववृतुः प्रबोध्याहिमिवार्भकाः
इतीन्द्रानुचरैर्ब्रह्मन्धर्षितोऽपि महामुनिः
यन्नागादहमो भावं न तच्चित्रं महत्सु हि
दृष्ट्वा निस्तेजसं कामं सगणं भगवान्स्वराट्‌
श्रुत्वानुभावं ब्रह्मर्षेर्विस्मयं समगात्परम्‌
तस्यैवं युञ्जतश्चित्तं तपःस्वाध्यायसंयमैः
अनुग्रहायाविरासीन्नरनारायणो हरिः
तौ शुक्लकृष्णौ नवकञ्जलोचनौ
चतुर्भुजौ रौरववल्कलाम्बरौ
पवित्रपाणी उपवीतकं त्रिवृत्‌
कमण्डलुं दण्डमृजुं च वैणवम्‌
पद्माक्षमालामुत जन्तुमार्जनं
वेदं च साक्षात्तप एव रूपिणौ
तपत्तडिद्वर्णपिशङ्गरोचिषा
प्रांशू दधानौ विबुधर्षभार्चितौ
ते वै भगवतो रूपे नरनारायणावृषी
दृष्ट्वोत्थायादरेणोच्चैर्ननामाङ्गेन दण्डवत्‌
स तत्सन्दर्शनानन्द निर्वृतात्मेन्द्रियाशयः
हृष्टरोमाश्रुपूर्णाक्षो न सेहे तावुदीक्षितुम्‌
उत्थाय प्राञ्जलिः प्रह्व औत्सुक्यादाश्लिषन्निव
नमो नम इतीशानौ बभाशे गद्गदाक्षरम्‌
तयोरासनमादाय पादयोरवनिज्य च
अर्हणेनानुलेपेन धूपमाल्यैरपूजयत्‌
सुखमासनमासीनौ प्रसादाभिमुखौ मुनी
पुनरानम्य पादाभ्यां गरिष्ठाविदमब्रवीत्‌
श्रीमार्कण्डेय उवाच
किं वर्णये तव विभो यदुदीरितोऽसुः
संस्पन्दते तमनु वाङ्मनैन्द्रियाणि
स्पन्दन्ति वै तनुभृतामजशर्वयोश्च
स्वस्याप्यथापि भजतामसि भावबन्धुः
मूर्ती इमे भगवतो भगवंस्त्रिलोक्याः
क्षेमाय तापविरमाय च मृत्युजित्यै
नाना बिभर्ष्यवितुमन्यतनूर्यथेदं
सृष्ट्वा पुनर्ग्रससि सर्वमिवोर्णनाभिः
तस्यावितुः स्थिरचरेशितुरङ्घ्रिमूलं
यत्स्थं न कर्मगुणकालरजः स्पृशन्ति
यद्वै स्तुवन्ति निनमन्ति यजन्त्यभीक्ष्णं
ध्यायन्ति वेदहृदया मुनयस्तदाप्त्यै
नान्यं तवाङ्घ्र्युपनयादपवर्गमूर्तेः
क्षेमं जनस्य परितोभिय ईश विद्मः
ब्रह्मा बिभेत्यलमतो द्विपरार्धधिष्ण्यः
कालस्य ते किमुत तत्कृतभौतिकानाम्‌
तद्वै भजाम्यृतधियस्तव पादमूलं
हित्वेदमात्मच्छदि चात्मगुरोः परस्य
देहाद्यपार्थमसदन्त्यमभिज्ञमात्रं
विन्देत ते तर्हि सर्वमनीषितार्थम्‌
सत्त्वं रजस्तम इतीश तवात्मबन्धो
मायामयाः स्थितिलयोदयहेतवोऽस्य
लीला धृता यदपि सत्त्वमयी प्रशान्त्यै
नान्ये नृणां व्यसनमोहभियश्च याभ्याम्‌
तस्मात्तवेह भगवन्नथ तावकानां
शुक्लां तनुं स्वदयितां कुशला भजन्ति
यत्सात्वताः पुरुषरूपमुशन्ति सत्त्वं
लोको यतोऽभयमुतात्मसुखं न चान्यत्‌
तस्मै नमो भगवते पुरुषाय भूम्ने
विश्वाय विश्वगुरवे परदैवताय
नारायणाय ऋषये च नरोत्तमाय
हंसाय संयतगिरे निगमेश्वराय
यं वै न वेद वितथाक्षपथैर्भ्रमद्धीः
सन्तं स्वकेष्वसुषु हृद्यपि दृक्पथेषु
तन्माययावृतमतिः स उ एव साक्षाद्‌
आद्यस्तवाखिलगुरोरुपसाद्य वेदम्‌
यद्दर्शनं निगम आत्मरहःप्रकाशं
मुह्यन्ति यत्र कवयोऽजपरा यतन्तः
तं सर्ववादविषयप्रतिरूपशीलं
वन्दे महापुरुषमात्मनिगूढबोधम्‌
सूत उवाच
संस्तुतो भगवानित्थं मार्कण्डेयेन धीमता
नारायणो नरसखः प्रीत आह भृगूद्वहम्‌
श्रीभगवानुवाच
भो भो ब्रह्मर्षिवर्योऽसि सिद्ध आत्मसमाधिना
मयि भक्त्यानपायिन्या तपःस्वाध्यायसंयमैः
वयं ते परितुष्टाः स्म त्वद्बृहद्व्रतचर्यया
वरं प्रतीच्छ भद्रं ते वरदोऽस्मि त्वदीप्सितम्‌
श्रीऋषिरुवाच
जितं ते देवदेवेश प्रपन्नार्तिहराच्युत
वरेणैतावतालं नो यद्भवान्समदृश्यत
गृहीत्वाजादयो यस्य श्रीमत्पादाब्जदर्शनम्‌
मनसा योगपक्वेन स भवान्मेऽक्षिगोचरः
अथाप्यम्बुजपत्राक्ष पुण्यश्लोकशिखामणे
द्रक्ष्ये मायां यया लोकः सपालो वेद सद्भिदाम्‌
सूत उवाच
इतीडितोऽर्चितः काममृषिणा भगवान्मुने
तथेति स स्मयन्प्रागाद्बदर्याश्रममीश्वरः
तमेव चिन्तयन्नर्थमृषिः स्वाश्रम एव सः
वसन्नग्न्यर्कसोमाम्बु भूवायुवियदात्मसु
ध्यायन्सर्वत्र च हरिं भावद्रव्यैरपूजयत्‌
क्वचित्पूजां विसस्मार प्रेमप्रसरसम्प्लुतः
तस्यैकदा भृगुश्रेष्ठ पुष्पभद्रातटे मुनेः
उपासीनस्य सन्ध्यायां ब्रह्मन्वायुरभून्महान्‌
तं चण्डशब्दं समुदीरयन्तं बलाहका अन्वभवन्करालाः
अक्षस्थविष्ठा मुमुचुस्तडिद्भिः स्वनन्त उच्चैरभि वर्षधाराः
ततो व्यदृश्यन्त चतुः समुद्राः समन्ततः क्ष्मातलमाग्रसन्तः
समीरवेगोर्मिभिरुग्रनक्र महाभयावर्तगभीरघोषाः
अन्तर्बहिश्चाद्भिरतिद्युभिः खरैः
शतह्रदाभिरुपतापितं जगत्‌
चतुर्विधं वीक्ष्य सहात्मना मुनिर्‌
जलाप्लुतां क्ष्मां विमनाः समत्रसत्‌
तस्यैवमुद्वीक्षत ऊर्मिभीषणः प्रभञ्जनाघूर्णितवार्महार्णवः
आपूर्यमाणो वरषद्भिरम्बुदैः क्ष्मामप्यधाद्द्वीपवर्षाद्रिभिः समम्‌
सक्ष्मान्तरिक्षं सदिवं सभागणं
त्रैलोक्यमासीत्सह दिग्भिराप्लुतम्‌
स एक एवोर्वरितो महामुनिर्‌
बभ्राम विक्षिप्य जटा जडान्धवत्‌
क्षुत्तृट्परीतो मकरैस्तिमिङ्गिलैर्‌
उपद्रुतो वीचिनभस्वताहतः
तमस्यपारे पतितो भ्रमन्दिशो
न वेद खं गां च परिश्रमेषितः
क्रचिन्मग्नो महावर्ते तरलैस्ताडितः क्वचित्‌
यादोभिर्भक्ष्यते क्वापि स्वयमन्योन्यघातिभिः
क्वचिच्छोकं क्वचिन्मोहं क्वचिद्दुःखं सुखं भयम्‌
क्वचिन्मृत्युमवाप्नोति व्याध्यादिभिरुतार्दितः
अयुतायतवर्षाणां सहस्राणि शतानि च
व्यतीयुर्भ्रमतस्तस्मिन्विष्णुमायावृतात्मनः
स कदाचिद्भ्रमंस्तस्मिन्पृथिव्याः ककुदि द्विजः
न्याग्रोधपोतं ददृशे फलपल्लवशोभितम्‌
प्रागुत्तरस्यां शाखायां तस्यापि ददृशे शिशुम्‌
शयानं पर्णपुटके ग्रसन्तं प्रभया तमः
महामरकतश्यामं श्रीमद्वदनपङ्कजम्‌
कम्बुग्रीवं महोरस्कं सुनसं सुन्दरभ्रुवम्‌
श्वासैजदलकाभातं कम्बुश्रीकर्णदाडिमम्‌
विद्रुमाधरभासेषच्‌ छोणायितसुधास्मितम्‌
पद्मगर्भारुणापाङ्गं हृद्यहासावलोकनम्‌
श्वासैजद्वलिसंविग्न निम्ननाभिदलोदरम्‌
चार्वङ्गुलिभ्यां पाणिभ्यामुन्नीय चरणाम्बुजम्‌
मुखे निधाय विप्रेन्द्रो धयन्तं वीक्ष्य विस्मितः
तद्दर्शनाद्वीतपरिश्रमो मुदा प्रोत्फुल्लहृत्पौल्मविलोचनाम्बुजः
प्रहृष्टरोमाद्भुतभावशङ्कितः प्रष्टुं पुरस्तं प्रससार बालकम्‌
तावच्छिशोर्वै श्वसितेन भार्गवः
सोऽन्तः शरीरं मशको यथाविशत्‌
तत्राप्यदो न्यस्तमचष्ट कृत्स्नशो
यथा पुरामुह्यदतीव विस्मितः
खं रोदसी भागणानद्रिसागरान्द्वीपान्सवर्षान्ककुभः सुरासुरान्‌
वनानि देशान्सरितः पुराकरान्खेटान्व्रजानाश्रमवर्णवृत्तयः
महान्ति भूतान्यथ भौतिकान्यसौ कालं च नानायुगकल्पकल्पनम्‌
यत्किञ्चिदन्यद्व्यवहारकारणं ददर्श विश्वं सदिवावभासितम्‌
हिमालयं पुष्पवहां च तां नदीं निजाश्रमं यत्र ऋषी अपश्यत
विश्वं विपश्यञ्छ्वसिताच्छिशोर्वै बहिर्निरस्तो न्यपतल्लयाब्धौ
तस्मिन्पृथिव्याः ककुदि प्ररूढं वटं च तत्पर्णपुटे शयानम्‌
तोकं च तत्प्रेमसुधास्मितेन निरीक्षितोऽपाङ्गनिरीक्षणेन
अथ तं बालकं वीक्ष्य नेत्राभ्यां धिष्ठितं हृदि
अभ्ययादतिसङ्क्लिष्टः परिष्वक्तुमधोक्षजम्‌
तावत्स भगवान्साक्षाद्योगाधीशो गुहाशयः
अन्तर्दध ऋषेः सद्यो यथेहानीशनिर्मिता
तमन्वथ वटो ब्रह्मन्सलिलं लोकसम्प्लवः
तिरोधायि क्षणादस्य स्वाश्रमे पूर्ववत्स्थितः
सूत उवाच
स एवमनुभूयेदं नारायणविनिर्मितम्‌
वैभवं योगमायायास्तमेव शरणं ययौ
श्रीमार्कण्डेय उवाच
प्रपन्नोऽस्म्यङ्घ्रिमूलं ते प्रपन्नाभयदं हरे
यन्माययापि विबुधा मुह्यन्ति ज्ञानकाशया
सूत उवाच
तमेवं निभृतात्मानं वृषेण दिवि पर्यटन्‌
रुद्राण्या भगवान्रुद्रो ददर्श स्वगणैर्वृतः
अथोमा तमृषिं वीक्ष्य गिरिशं समभाषत
पश्येमं भगवन्विप्रं निभृतात्मेन्द्रियाशयम्‌
निभृतोदझषव्रातो वातापाये यथार्णवः
कुर्वस्य तपसः साक्षात्संसिद्धिं सिद्धिदो भवान्‌
श्रीभगवानुवाच
नैवेच्छत्याशिषः क्वापि ब्रह्मर्षिर्मोक्षमप्युत
भक्तिं परां भगवति लब्धवान्पुरुषेऽव्यये
अथापि संवदिष्यामो भवान्येतेन साधुना
अयं हि परमो लाभो नृणां साधुसमागमः
सूत उवाच
इत्युक्त्वा तमुपेयाय भगवान्स सतां गतिः
ईशानः सर्वविद्यानामीश्वरः सर्वदेहिनाम्‌
तयोरागमनं साक्षादीशयोर्जगदात्मनोः
न वेद रुद्धधीवृत्तिरात्मानं विश्वमेव च
भगवांस्तदभिज्ञाय गिरिशो योगमायया
आविशत्तद्गुहाकाशं वायुश्छिद्रमिवेश्वरः
आत्मन्यपि शिवं प्राप्तं तडित्पिङ्गजटाधरम्‌
त्र्यक्षं दशभुजं प्रांशुमुद्यन्तमिव भास्करम्‌
व्याघ्रचर्माम्बरं शूल धनुरिष्वसिचर्मभिः
अक्षमालाडमरुक कपालं परशुं सह
बिभ्राणं सहसा भातं विचक्ष्य हृदि विस्मितः
किमिदं कुत एवेति समाधेर्विरतो मुनिः
नेत्रे उन्मील्य ददृशे सगणं सोमयागतम्‌
रुद्रं त्रिलोकैकगुरुं ननाम शिरसा मुनिः
तस्मै सपर्यां व्यदधात्सगणाय सहोमया
स्वागतासनपाद्यार्घ्य गन्धस्रग्धूपदीपकैः
आह त्वात्मानुभावेन पूर्णकामस्य ते विभो
करवाम किमीशान येनेदं निर्वृतं जगत्‌
नमः शिवाय शान्ताय सत्त्वाय प्रमृडाय च
रजोजुषेऽथ घोराय नमस्तुभ्यं तमोजुषे
सूत उवाच
एवं स्तुतः स भगवानादिदेवः सतां गतिः
परितुष्टः प्रसन्नात्मा प्रहसंस्तमभाषत
श्रीभगवानुवाच
वरं वृणीष्व नः कामं वरदेशा वयं त्रयः
अमोघं दर्शनं येषां मर्त्यो यद्विन्दतेऽमृतम्‌
ब्राह्मणाः साधवः शान्ता निःसङ्गा भूतवत्सलाः
एकान्तभक्ता अस्मासु निर्वैराः समदर्शिनः
सलोका लोकपालास्तान्वन्दन्त्यर्चन्त्युपासते
अहं च भगवान्ब्रह्मा स्वयं च हरिरीश्वरः
न ते मय्यच्युतेऽजे च भिदामण्वपि चक्षते
नात्मनश्च जनस्यापि तद्युष्मान्वयमीमहि
न ह्यम्मयानि तीर्थानि न देवाश्चेतनोज्झिताः
ते पुनन्त्युरुकालेन यूयं दर्शनमात्रतः
ब्राह्मणेभ्यो नमस्यामो येऽस्मद्रूपं त्रयीमयम्‌
बिभ्रत्यात्मसमाधान तपःस्वाध्यायसंयमैः
श्रवणाद्दर्शनाद्वापि महापातकिनोऽपि वः
शुध्येरन्नन्त्यजाश्चापि किमु सम्भाषणादिभिः
सूत उवाच
इति चन्द्रललामस्य धर्मगह्योपबृंहितम्‌
वचोऽमृतायनमृषिर्नातृप्यत्कर्णयोः पिबन्‌
स चिरं मायया विष्णोर्भ्रामितः कर्शितो भृशम्‌
शिववागमृतध्वस्त क्लेशपुञ्जस्तमब्रवीत्‌
श्रीमार्कण्डेय उवाच
अहो ईश्वरलीलेयं दुर्विभाव्या शरीरिणाम्‌
यन्नमन्तीशितव्यानि स्तुवन्ति जगदीश्वराः
धर्मं ग्राहयितुं प्रायः प्रवक्तारश्च देहिनाम्‌
आचरन्त्यनुमोदन्ते क्रियमाणं स्तुवन्ति च
नैतावता भगवतः स्वमायामयवृत्तिभिः
न दुष्येतानुभावस्तैर्मायिनः कुहकं यथा
सृष्ट्वेदं मनसा विश्वमात्मनानुप्रविश्य यः
गुणैः कुर्वद्भिराभाति कर्तेव स्वप्नदृग्यथा
तस्मै नमो भगवते त्रिगुणाय गुणात्मने
केवलायाद्वितीयाय गुरवे ब्रह्ममूर्तये
कं वृणे नु परं भूमन्वरं त्वद्वरदर्शनात्‌
यद्दर्शनात्पूर्णकामः सत्यकामः पुमान्भवेत्‌
वरमेकं वृणेऽथापि पूर्णात्कामाभिवर्षणात्‌
भगवत्यच्युतां भक्तिं तत्परेषु तथा त्वयि
सूत उवाच
इत्यर्चितोऽभिष्टुतश्च मुनिना सूक्तया गिरा
तमाह भगवाञ्छर्वः शर्वया चाभिनन्दितः
कामो महर्षे सर्वोऽयं भक्तिमांस्त्वमधोक्षजे
आकल्पान्ताद्यशः पुण्यमजरामरता तथा
ज्ञानं त्रैकालिकं ब्रह्मन्विज्ञानं च विरक्तिमत्‌
ब्रह्मवर्चस्विनो भूयात्पुराणाचार्यतास्तु ते
सूत उवाच
एवं वरान्स मुनये दत्त्वागात्त्र्यक्ष ईश्वरः
देव्यै तत्कर्म कथयन्ननुभूतं पुरामुना
सोऽप्यवाप्तमहायोग महिमा भार्गवोत्तमः
विचरत्यधुनाप्यद्धा हरावेकान्ततां गतः
अनुवर्णितमेतत्ते मार्कण्डेयस्य धीमतः
अनुभूतं भगवतो मायावैभवमद्भुतम्‌
एतत्केचिदविद्वांसो मायासंसृतिरात्मनः
अनाद्यावर्तितं न्ऱ्‌^ईणां कादाचित्कं प्रचक्षते
य एवमेतद्भृगुवर्य वर्णितं रथाङ्गपाणेरनुभावभावितम्‌
संश्रावयेत्संशृणुयादु तावुभौ तयोर्न कर्माशयसंसृतिर्भवेत्‌
श्रीशौनक उवाच
अथेममर्थं पृच्छामो भवन्तं बहुवित्तमम्‌
समस्ततन्त्रराद्धान्ते भवान्भागवत तत्त्ववित्‌
तान्त्रिकाः परिचर्यायां केवलस्य श्रियः पतेः
अङ्गोपाङ्गायुधाकल्पं कल्पयन्ति यथा च यैः
तन्नो वर्णय भद्रं ते क्रियायोगं बुभुत्सताम्‌
येन क्रियानैपुणेन मर्त्यो यायादमर्त्यताम्‌
सूत उवाच
नमस्कृत्य गुरून्वक्ष्ये विभूतीर्वैष्णवीरपि
याः प्रोक्ता वेदतन्त्राभ्यामाचार्यैः पद्मजादिभिः
मायाद्यैर्नवभिस्तत्त्वैः स विकारमयो विराट्‌
निर्मितो दृश्यते यत्र सचित्के भुवनत्रयम्‌
एतद्वै पौरुषं रूपं भूः पादौ द्यौः शिरो नभः
नाभिः सूर्योऽक्षिणी नासे वायुः कर्णौ दिशः प्रभोः
प्रजापतिः प्रजननमपानो मृत्युरीशितुः
तद्बाहवो लोकपाला मनश्चन्द्रो भ्रुवौ यमः
लज्जोत्तरोऽधरो लोभो दन्ता ज्योत्स्ना स्मयो भ्रमः
रोमाणि भूरुहा भूम्नो मेघाः पुरुषमूर्धजाः
यावानयं वै पुरुषो यावत्या संस्थया मितः
तावानसावपि महा पुरुषो लोकसंस्थया
कौस्तुभव्यपदेशेन स्वात्मज्योतिर्बिभर्त्यजः
तत्प्रभा व्यापिनी साक्षात्श्रीवत्समुरसा विभुः
स्वमायां वनमालाख्यां नानागुणमयीं दधत्‌
वासश्छन्दोमयं पीतं ब्रह्मसूत्रं त्रिवृत्स्वरम्‌
बिभर्ति साङ्ख्यं योगं च देवो मकरकुण्डले
मौलिं पदं पारमेष्ठ्यं सर्वलोकाभयङ्करम्‌
अव्याकृतमनन्ताख्यमासनं यदधिष्ठितः
धर्मज्ञानादिभिर्युक्तं सत्त्वं पद्ममिहोच्यते
ओजःसहोबलयुतं मुख्यतत्त्वं गदां दधत्‌
अपां तत्त्वं दरवरं तेजस्तत्त्वं सुदर्शनम्‌
नभोनिभं नभस्तत्त्वमसिं चर्म तमोमयम्‌
कालरूपं धनुः शार्ङ्गं तथा कर्ममयेषुधिम्‌
इन्द्रियाणि शरानाहुराकूतीरस्य स्यन्दनम्‌
तन्मात्राण्यस्याभिव्यक्तिं मुद्रयार्थक्रियात्मताम्‌
मण्डलं देवयजनं दीक्षा संस्कार आत्मनः
परिचर्या भगवत आत्मनो दुरितक्षयः
भगवान्भगशब्दार्थं लीलाकमलमुद्वहन्‌
धर्मं यशश्च भगवांश्चामरव्यजनेऽभजत्‌
आतपत्रं तु वैकुण्ठं द्विजा धामाकुतोभयम्‌
त्रिवृद्वेदः सुपर्णाख्यो यज्ञं वहति पूरुषम्‌
अनपायिनी भगवती शृईः साक्षादात्मनो हरेः
विष्वक्षेनस्तन्त्रमूर्तिर्विदितः पार्षदाधिपः
नन्दादयोऽष्टौ द्वाःस्थाश्च तेऽणिमाद्या हरेर्गुणाः
वासुदेवः सङ्कर्षणः प्रद्युम्नः पुरुषः स्वयम्‌
अनिरुद्ध इति ब्रह्मन्मूर्तिव्यूहोऽभिधीयते
स विश्वस्तैजसः प्राज्ञस्तुरीय इति वृत्तिभिः
अर्थेन्द्रियाशयज्ञानैर्भगवान्परिभाव्यते
अङ्गोपाङ्गायुधाकल्पैर्भगवांस्तच्चतुष्टयम्‌
बिभर्ति स्म चतुर्मूर्तिर्भगवान्हरिरीश्वरः
द्विजऋषभ स एष ब्रह्मयोनिः स्वयंदृक्‌
स्वमहिमपरिपूर्णो मायया च स्वयैतत्‌
सृजति हरति पातीत्याख्ययानावृताक्षो
विवृत इव निरुक्तस्तत्परैरात्मलभ्यः
श्रीकृष्ण कृष्णसख वृष्ण्यृषभावनिध्रुग्‌
राजन्यवंशदहनानपवर्गवीर्य
गोविन्द गोपवनिताव्रजभृत्यगीत
तीर्थश्रवः श्रवणमङ्गल पाहि भृत्यान्‌
य इदं कल्य उत्थाय महापुरुषलक्षणम्‌
तच्चित्तः प्रयतो जप्त्वा ब्रह्म वेद गुहाशयम्‌
श्रीशौनक उवाच
शुको यदाह भगवान्विष्णुराताय शृण्वते
सौरो गणो मासि मासि नाना वसति सप्तकः
तेषां नामानि कर्माणि नियुक्तानामधीश्वरैः
ब्रूहि नः श्रद्दधानानां व्यूहं सूर्यात्मनो हरेः
सूत उवाच
अनाद्यविद्यया विष्णोरात्मनः सर्वदेहिनाम्‌
निर्मितो लोकतन्त्रोऽयं लोकेषु परिवर्तते
एक एव हि लोकानां सूर्य आत्मादिकृद्धरिः
सर्ववेदक्रियामूलमृषिभिर्बहुधोदितः
कालो देशः क्रिया कर्ता करणं कार्यमागमः
द्रव्यं फलमिति ब्रह्मन्नवधोक्तोऽजया हरिः
मध्वादिषु द्वादशसु भगवान्कालरूपधृक्‌
लोकतन्त्राय चरति पृथग्द्वादशभिर्गणैः
धाता कृतस्थली हेतिर्वासुकी रथकृन्मुने
पुलस्त्यस्तुम्बुरुरिति मधुमासं नयन्त्यमी
अर्यमा पुलहोऽथौजाः प्रहेतिः पुञ्जिकस्थली
नारदः कच्छनीरश्च नयन्त्येते स्म माधवम्‌
मित्रोऽत्रिः पौरुषेयोऽथ तक्षको मेनका हहाः
रथस्वन इति ह्येते शुक्रमासं नयन्त्यमी
वसिष्ठो वरुणो रम्भा सहजन्यस्तथा हुहूः
शुक्रश्चित्रस्वनश्चैव शुचिमासं नयन्त्यमी
इन्द्रो विश्वावसुः श्रोता एलापत्रस्तथाङ्गिराः
प्रम्लोचा राक्षसो वर्यो नभोमासं नयन्त्यमी
विवस्वानुग्रसेनश्च व्याघ्र आसारणो भृगुः
अनुम्लोचा शङ्खपालो नभस्याख्यं नयन्त्यमी
पूषा धनञ्जयो वातः सुषेणः सुरुचिस्तथा
घृताची गौतमश्चेति तपोमासं नयन्त्यमी
ऋतुर्वर्चा भरद्वाजः पर्जन्यः सेनजित्तथा
विश्व ऐरावतश्चैव तपस्याख्यं नयन्त्यमी
अथांशुः कश्यपस्तार्क्ष्य ऋतसेनस्तथोर्वशी
विद्युच्छत्रुर्महाशङ्खः सहोमासं नयन्त्यमी
भगः स्फूर्जोऽरिष्टनेमिरूर्ण आयुश्च पञ्चमः
कर्कोटकः पूर्वचित्तिः पुष्यमासं नयन्त्यमी
त्वष्टा ऋचीकतनयः कम्बलश्च तिलोत्तमा
ब्रह्मापेतोऽथ सतजिद्धृतराष्ट्र इषम्भराः
विष्णुरश्वतरो रम्भा सूर्यवर्चाश्च सत्यजित्‌
विश्वामित्रो मखापेत ऊर्जमासं नयन्त्यमी
एता भगवतो विष्णोरादित्यस्य विभूतयः
स्मरतां सन्ध्ययोर्न्ऱ्‌^ईणां हरन्त्यंहो दिने दिने
द्वादशस्वपि मासेषु देवोऽसौ षड्भिरस्य वै
चरन्समन्तात्तनुते परत्रेह च सन्मतिम्‌
सामर्ग्यजुर्भिस्तल्लिङ्गैरृषयः संस्तुवन्त्यमुम्‌
गन्धर्वास्तं प्रगायन्ति नृत्यन्त्यप्सरसोऽग्रतः
उन्नह्यन्ति रथं नागा ग्रामण्यो रथयोजकाः
चोदयन्ति रथं पृष्ठे नैरृता बलशालिनः
वालखिल्याः सहस्राणि षष्टिर्ब्रह्मर्षयोऽमलाः
पुरतोऽभिमुखं यान्ति स्तुवन्ति स्तुतिभिर्विभुम्‌
एवं ह्यनादिनिधनो भगवान्हरिरीश्वरः
कल्पे कल्पे स्वमात्मानं व्यूह्य लोकानवत्यजः
सूत उवाच
नमो धर्माय महते नमः कृष्णाय वेधसे
ब्रह्मणेभ्यो नमस्कृत्य धर्मान्वक्ष्ये सनातनान्‌
एतद्वः कथितं विप्रा विष्णोश्चरितमद्भुतम्‌
भवद्भिर्यदहं पृष्टो नराणां पुरुषोचितम्‌
अत्र सङ्कीर्तितः साक्षात्सर्वपापहरो हरिः
नारायणो हृषीकेशो भगवान्सात्वताम्पतिः
अत्र ब्रह्म परं गुह्यं जगतः प्रभवाप्ययम्‌
ज्ञानं च तदुपाख्यानं प्रोक्तं विज्ञानसंयुतम्‌
भक्तियोगः समाख्यातो वैराग्यं च तदाश्रयम्‌
पारीक्षितमुपाख्यानं नारदाख्यानमेव च
प्रायोपवेशो राजर्षेर्विप्रशापात्परीक्षितः
शुकस्य ब्रह्मर्षभस्य संवादश्च परीक्षितः
योगधारणयोत्क्रान्तिः संवादो नारदाजयोः
अवतारानुगीतं च सर्गः प्राधानिकोऽग्रतः
विदुरोद्धवसंवादः क्षत्तृमैत्रेययोस्ततः
पुराणसंहिताप्रश्नो महापुरुषसंस्थितिः
ततः प्राकृतिकः सर्गः सप्त वैकृतिकाश्च ये
ततो ब्रह्माण्डसम्भूतिर्वैराजः पुरुषो यतः
कालस्य स्थूलसूक्ष्मस्य गतिः पद्मसमुद्भवः
भुव उद्धरणेऽम्भोधेर्हिरण्याक्षवधो यथा
ऊर्ध्वतिर्यगवाक्सर्गो रुद्रसर्गस्तथैव च
अर्धनारीश्वरस्याथ यतः स्वायम्भुवो मनुः
शतरूपा च या स्त्रीणामाद्या प्रकृतिरुत्तमा
सन्तानो धर्मपत्नीनां कर्दमस्य प्रजापतेः
अवतारो भगवतः कपिलस्य महात्मनः
देवहूत्याश्च संवादः कपिलेन च धीमता
नवब्रह्मसमुत्पत्तिर्दक्षयज्ञविनाशनम्‌
ध्रुवस्य चरितं पश्चात्पृथोः प्राचीनबर्हिषः
नारदस्य च संवादस्ततः प्रैयव्रतं द्विजाः
नाभेस्ततोऽनुचरितमृषभस्य भरतस्य च
द्वीपवर्षसमुद्राणां गिरिनद्युपवर्णनम्‌
ज्योतिश्चक्रस्य संस्थानं पातालनरकस्थितिः
दक्षजन्म प्रचेतोभ्यस्तत्पुत्रीणां च सन्ततिः
यतो देवासुरनरास्तिर्यङ्नगखगादयः
त्वाष्ट्रस्य जन्मनिधनं पुत्रयोश्च दितेर्द्विजाः
दैत्येश्वरस्य चरितं प्रह्रादस्य महात्मनः
मन्वन्तरानुकथनं गजेन्द्रस्य विमोक्षणम्‌
मन्वन्तरावताराश्च विष्णोर्हयशिरादयः
कौर्मं मात्स्यं नारसिंहं वामनं च जगत्पतेः
क्षीरोदमथनं तद्वदमृतार्थे दिवौकसाम्‌
देवासुरमहायुद्धं राजवंशानुकीर्तनम्‌
इक्ष्वाकुजन्म तद्वंशः सुद्युम्नस्य महात्मनः
इलोपाख्यानमत्रोक्तं तारोपाख्यानमेव च
सूर्यवंशानुकथनं शशादाद्या नृगादयः
सौकन्यं चाथ शर्यातेः ककुत्स्थस्य च धीमतः
खट्वाङ्गस्य च मान्धातुः सौभरेः सगरस्य च
रामस्य कोशलेन्द्रस्य चरितं किल्बिषापहम्‌
निमेरङ्गपरित्यागो जनकानां च सम्भवः
रामस्य भार्गवेन्द्रस्य निःक्षतृईकरणं भुवः
ऐलस्य सोमवंशस्य ययातेर्नहुषस्य च
दौष्मन्तेर्भरतस्यापि शान्तनोस्तत्सुतस्य च
ययातेर्ज्येष्ठपुत्रस्य यदोर्वंशोऽनुकीर्तितः
यत्रावतीऋणो भगवान्कृष्णाख्यो जगदीश्वरः
वसुदेवगृहे जन्म ततो वृद्धिश्च गोकुले
तस्य कर्माण्यपाराणि कीर्तितान्यसुरद्विषः
पूतनासुपयःपानं शकटोच्चाटनं शिशोः
तृणावर्तस्य निष्पेषस्तथैव बकवत्सयोः
अघासुरवधो धात्रा वत्सपालावगूहनम्‌
धेनुकस्य सहभ्रातुः प्रलम्बस्य च सङ्क्षयः
गोपानां च परित्राणं दावाग्नेः परिसर्पतः
दमनं कालियस्याहेर्महाहेर्नन्दमोक्षणम्‌
व्रतचर्या तु कन्यानां यत्र तुष्टोऽच्युतो व्रतैः
प्रसादो यज्ञपत्नीभ्यो विप्राणां चानुतापनम्‌
गोवर्धनोद्धारणं च शक्रस्य सुरभेरथ
यज्ञभिषेकः कृष्णस्य स्त्रीभिः क्रीडा च रात्रिषु
शङ्खचूडस्य दुर्बुद्धेर्वधोऽरिष्टस्य केशिनः
अक्रूरागमनं पश्चात्प्रस्थानं रामकृष्णयोः
व्रजस्त्रीणां विलापश्च मथुरालोकनं ततः
गजमुष्टिकचाणूर कंसादीनां तथा वधः
मृतस्यानयनं सूनोः पुनः सान्दीपनेर्गुरोः
मथुरायां निवसता यदुचक्रस्य यत्प्रियम्‌
कृतमुद्धवरामाभ्यां युतेन हरिणा द्विजाः
जरासन्धसमानीत सैन्यस्य बहुशो वधः
घातनं यवनेन्द्रस्य कुशस्थल्या निवेशनम्‌
आदानं पारिजातस्य सुधर्मायाः सुरालयात्‌
रुक्मिण्या हरणं युद्धे प्रमथ्य द्विषतो हरेः
हरस्य जृम्भणं युद्धे बाणस्य भुजकृन्तनम्‌
प्राग्ज्योतिषपतिं हत्वा कन्यानां हरणं च यत्‌
चैद्यपौण्ड्रकशाल्वानां दन्तवक्रस्य दुर्मतेः
शम्बरो द्विविदः पीठो मुरः पञ्चजनादयः
माहात्म्यं च वधस्तेषां वाराणस्याश्च दाहनम्‌
भारावतरणं भूमेर्निमित्तीकृत्य पाण्डवान्‌
विप्रशापापदेशेन संहारः स्वकुलस्य च
उद्धवस्य च संवादो वसुदेवस्य चाद्भुतः
यत्रात्मविद्या ह्यखिला प्रोक्ता धर्मविनिर्णयः
ततो मर्त्यपरित्याग आत्मयोगानुभावतः
युगलक्षणवृत्तिश्च कलौ न्ऱ्‌^ईणामुपप्लवः
चतुर्विधश्च प्रलय उत्पत्तिस्त्रिविधा तथा
देहत्यागश्च राजर्षेर्विष्णुरातस्य धीमतः
शाखाप्रणयनमृषेर्मार्कण्डेयस्य सत्कथा
महापुरुषविन्यासः सूर्यस्य जगदात्मनः
इति चोक्तं द्विजश्रेष्ठा यत्पृष्टोऽहमिहास्मि वः
लीलावतारकर्माणि कीर्तितानीह सर्वशः
पतितः स्खलितश्चार्तः क्षुत्त्वा वा विवशो गृणन्‌
हरये नम इत्युच्चैर्मुच्यते सर्वपातकात्‌
सङ्कीर्त्यमानो भगवाननन्तः श्रुतानुभावो व्यसनं हि पुंसाम्‌
प्रविश्य चित्तं विधुनोत्यशेषं यथा तमोऽर्कोऽभ्रमिवातिवातः
मृषा गिरस्ता ह्यसतीरसत्कथा न कथ्यते यद्भगवानधोक्षजः
तदेव सत्यं तदु हैव मङ्गलं तदेव पुण्यं भगवद्गुणोदयम्‌
तदेव रम्यं रुचिरं नवं नवं तदेव शश्वन्मनसो महोत्सवम्‌
तदेव शोकार्णवशोषणं नृणां यदुत्तमःश्लोकयशोऽनुगीयते
न यद्वचश्चित्रपदं हरेर्यशो
जगत्पवित्रं प्रगृणीत कर्हिचित्‌
तद्ध्वाङ्क्षतीऋथं न तु हंससेवितं
यत्राच्युतस्तत्र हि साधवोऽमलाः
तद्वाग्विसर्गो जनताघसम्प्लवो यस्मिन्प्रतिश्लोकमबद्धवत्यपि
नामान्यनन्तस्य यशोऽङ्कितानि यत्शृण्वन्ति गायन्ति गृणन्ति साधवः
नैष्कर्म्यमप्यच्युतभाववर्जितं
न शोभते ज्ञानमलं निरञ्जनम्‌
कुतः पुनः शश्वदभद्रमीश्वरे
न ह्यर्पितं कर्म यदप्यनुत्तमम्‌
यशःश्रियामेव परिश्रमः परो वर्णाश्रमाचारतपःश्रुतादिषु
अविस्मृतिः श्रीधरपादपद्मयोर्गुणानुवादश्रवणादरादिभिः
अविस्मृतिः कृष्णपदारविन्दयोः क्षिणोत्यभद्राणि च शं तनोति
सत्त्वस्य शुद्धिं परमात्मभक्तिं ज्ञानं च विज्ञानविरागयुक्तम्‌
यूयं द्विजाग्र्या बत भूरिभागा यच्छश्वदात्मन्यखिलात्मभूतम्‌
नारायणं देवमदेवमीशमजस्रभावा भजताविवेश्य
अहं च संस्मारित आत्मतत्त्वं श्रुतं पुरा मे परमर्षिवक्त्रात्‌
प्रायोपवेशे नृपतेः परीक्षितः सदस्यृषीणां महतां च शृण्वताम्‌
एतद्वः कथितं विप्राः कथनीयोरुकर्मणः
माहात्म्यं वासुदेवस्य सर्वाशुभविनाशनम्‌
य एतत्श्रावयेन्नित्यं यामक्षणमनन्यधीः
श्लोकमेकं तदर्धं वा पादं पादार्धमेव वा
श्रद्धावान्योऽनुशृणुयात्पुनात्यात्मानमेव सः
द्वादश्यामेकादश्यां वा शृण्वन्नायुष्यवान्भवेत्‌
पठत्यनश्नन्प्रयतः पूतो भवति पातकात्‌
पुष्करे मथुरयां च द्वारवत्यां यतात्मवान्‌
उपोष्य संहितामेतां पठित्वा मुच्यते भयात्‌
देवता मुनयः सिद्धाः पितरो मनवो नृपाः
यच्छन्ति कामान्गृणतः शृण्वतो यस्य कीर्तनात्‌
ऋचो यजूंषि सामानि द्विजोऽधीत्यानुविन्दते
मधुकुल्या घृतकुल्याः पयःकुल्याश्च तत्फलम्‌
पुराणसंहितामेतामधीत्य प्रयतो द्विजः
प्रोक्तं भगवता यत्तु तत्पदं परमं व्रजेत्‌
विप्रोऽधीत्याप्नुयात्प्रज्ञां राजन्योदधिमेखलाम्‌
वैश्यो निधिपतित्वं च शूद्रः शुध्येत पातकात्‌
कलिमलसंहतिकालनोऽखिलेशो हरिरितरत्र न गीयते ह्यभीक्ष्णम्‌
इह तु पुनर्भगवानशेषमूर्तिः परिपठितोऽनुपदं कथाप्रसङ्गैः
तमहमजमनन्तमात्मतत्त्वं जगदुदयस्थितिसंयमात्मशक्तिम्‌
द्युपतिभिरजशक्रशङ्कराद्यैर्दुरवसितस्तवमच्युतं नतोऽस्मि
उपचितनवशक्तिभिः स्व आत्मन्युपरचितस्थिरजङ्गमालयाय
भगवत उपलब्धिमात्रधम्ने सुरऋषभाय नमः सनातनाय
स्वसुखनिभृतचेतास्तद्व्युदस्तान्यभावो
१२१२०६९२ऽप्यजितरुचिरलीलाकृष्टसारस्तदीयम्‌
व्यतनुत कृपया यस्तत्त्वदीपं पुराणं
तमखिलवृजिनघ्नं व्याससूनुं नतोऽस्मि
सूत उवाच
यं ब्रह्मा वरुणेन्द्ररुद्रमरुतः स्तुन्वन्ति दिव्यैः स्तवैर्‌
वेदैः साङ्गपदक्रमोपनिषदैर्गायन्ति यं सामगाः
ध्यानावस्थिततद्गतेन मनसा पश्यन्ति यं योगिनो
यस्यान्तं न विदुः सुरासुरगणा देवाय तस्मै नमः
पृष्ठे भ्राम्यदमन्दमन्दरगिरिग्रावाग्रकण्डूयनान्‌
निद्रालोः कमठाकृतेर्भगवतः श्वासानिलाः पान्तु वः
यत्संस्कारकलानुवर्तनवशाद्वेलानिभेनाम्भसां
यातायातमतन्द्रितं जलनिधेर्नाद्यापि विश्राम्यति
पुराणसङ्ख्यासम्भूतिमस्य वाच्यप्रयोजने
दानं दानस्य माहात्म्यं पाठादेश्च निबोधत
ब्राह्मं दश सहस्राणि पाद्मं पञ्चोनषष्टि च
श्रीवैष्णवं त्रयोविंशच्चतुर्विंशति शैवकम्‌
दशाष्टौ श्रीभागवतं नारदं पञ्चविंशति
मार्कण्डं नव वाह्नं च दशपञ्च चतुःशतम्‌
चतुर्दश भविष्यं स्यात्तथा पञ्चशतानि च
दशाष्टौ ब्रह्मवैवर्तं लैङ्गमेकादशैव तु
चतुर्विंशति वाराहमेकाशीतिसहस्रकम्‌
स्कान्दं शतं तथा चैकं वामनं दश कीर्तितम्‌
कौर्मं सप्तदशाख्यातं मात्स्यं तत्तु चतुर्दश
एकोनविंशत्सौपर्णं ब्रह्माण्डं द्वादशैव तु
एवं पुराणसन्दोहश्चतुर्लक्ष उदाहृतः
तत्राष्टदशसाहस्रं श्रीभागवतं इष्यते
इदं भगवता पूर्वं ब्रह्मणे नाभिपङ्कजे
स्थिताय भवभीताय कारुण्यात्सम्प्रकाशितम्‌
आदिमध्यावसानेषु वैराग्याख्यानसंयुतम्‌
हरिलीलाकथाव्राता मृतानन्दितसत्सुरम्‌
सर्ववेदान्तसारं यद्ब्रह्मात्मैकत्वलक्षणम्‌
वस्त्वद्वितीयं तन्निष्ठं कैवल्यैकप्रयोजनम्‌
प्रौष्ठपद्यां पौर्णमास्यां हेमसिंहसमन्वितम्‌
ददाति यो भागवतं स याति परमां गतिम्‌
राजन्ते तावदन्यानि पुराणानि सतां गणे
यावद्भागवतं नैव श्रूयतेऽमृतसागरम्‌
सर्ववेदान्तसारं हि श्रीभागवतमिष्यते
तद्रसामृततृप्तस्य नान्यत्र स्याद्रतिः क्वचित्‌
निम्नगानां यथा गङ्गा देवानामच्युतो यथा
वैष्णवानां यथा शम्भुः पुराणानामिदम्तथा
क्षेत्राणां चैव सर्वेषां यथा काशी ह्यनुत्तमा
तथा पुराणव्रातानां श्रीमद्भागवतं द्विजाः
श्रीमद्भागवतं पुराणममलं यद्वैष्णवानां प्रियं
यस्मिन्पारमहंस्यमेकममलं ज्ञानं परं गीयते
तत्र ज्ञानविरागभक्तिसहितं नैष्कर्म्यमाविस्कृतं
तच्छृण्वन्सुपठन्विचारणपरो भक्त्या विमुच्येन्नरः
कस्मै येन विभासितोऽयमतुलो ज्ञानप्रदीपः पुरा
तद्रूपेण च नारदाय मुनये कृष्णाय तद्रूपिणा
योगीन्द्राय तदात्मनाथ भगवद्राताय कारुण्यतस्‌
तच्छुद्धं विमलं विशोकममृतं सत्यं परं धीमहि
नमस्तस्मै भगवते वासुदेवाय साक्षिणे
य इदम्कृपया कस्मै व्याचचक्षे मुमुक्षवे
योगीन्द्राय नमस्तस्मै शुकाय ब्रह्मरूपिणे
संसारसर्पदष्टं यो विष्णुरातममूमुचत्‌
भवे भवे यथा भक्तिः पादयोस्तव जायते
तथा कुरुष्व देवेश नाथस्त्वं नो यतः प्रभो
नामसङ्कीर्तनं यस्य सर्वपाप प्रणाशनम्‌
प्रणामो दुःखशमनस्तं नमामि हरिं परम्‌
         ॥ ॐ तत्सत्‌ ॥
         
         
         
         kchvlkkjsh jhgfladfadf
         akdsgjha;sdkfa;ldfjkoe8rykje
